%% THIS IS THE INTRODUCTORY MATERIAL

\nocite{*}

\title{The CRing Project}
\author{GNU Free Documentation License, v. 1.2}
\date{2010}

\maketitle
\begin{verbatim}
Copyright (C) 2010 Akhil Mathew 
Permission is granted to copy, distribute and/or modify this
document under the terms of the GNU Free Documentation License,
Version 1.2 or any later version published by the Free Software
Foundation; with no Invariant Sections, no Front-Cover Texts,
and no Back-Cover Texts. A copy of the license is included in
the section entitled "GNU Free Documentation License". \end{verbatim}


\fancyhead{}
\fancyfoot{}
\tableofcontents

\newpage 


\pagestyle{fancy}



\chapter*{Introduction}

The following is a massively collaborative, open source textbook on
commutative algebra. The project is currently in its infancy, and needs
contributions! See the next chapter for how to contribute.

\section*{Prerequisites}
The prerequisite  is a basic acquaintance with modern
algebra. While even the notion of a ring is introduced from scratch, it is
done so rather rapidly, and the reader is advised to consult another source.
In addition, the notes do not hesitate to use the language of categories.
Besides that, the notes are mostly self-contained. (Material explaining the
category theory should be added sometime.) 

\section*{Genesis of the project}
In the fall of 2010, Jacob Lurie taught a course (Math 221) on commutative
algebra at Harvard. 
The course started with foundational material, but swiftly progressed to touch on a wide range of
material, and culminated in the study of regular local rings. 
Akhil Mathew sat in on this course and took detailed
(``live-\TeX ed'') notes, which are still available on his website in unedited
form at 
\url{http://people.fas.harvard.edu/~amathew}. 

As the course was very well-taught, Akhil decided that it would be an interesting project to 
edit the notes I had taken into a mini-textbook of sort. This book started out as an
attempt 
at such a project, and the initial contribution was his notes. However, the
book is now intended to be a massively collaborative
project.

N.B. The following project is not endorsed by Jacob Lurie.


\section*{Corrections}
Please email corrections to
\verb=cring.project@gmail.com=.

\section*{Version}
This file was last updated August 30, 2011.



