%% THIS IS THE INTRODUCTORY MATERIAL

\nocite{*}

\title{The CRing Project}
\author{GNU Free Documentation License, v. 1.2}
\date{2010}

\maketitle
\begin{verbatim}
Copyright (C) 2010 CRing Project 
Permission is granted to copy, distribute and/or modify this
document under the terms of the GNU Free Documentation License,
Version 1.2 or any later version published by the Free Software
Foundation; with no Invariant Sections, no Front-Cover Texts,
and no Back-Cover Texts. A copy of the license is included in
the section entitled "GNU Free Documentation License". \end{verbatim}


\fancyhead{}
\fancyfoot{}
\tableofcontents

\newpage 


\pagestyle{fancy}



\chapter*{Introduction}

This  is a massively collaborative, open source textbook on
commutative algebra. The project is currently in its infancy, and needs
contributions! (See the next chapter for how to contribute.)

This document, and
its \LaTeX source code, may be downloaded from
\url{http://people.fas.harvard.edu/~amathew/cr.html}.

\section*{Prerequisites}
This book is intended to be accessible to undergraduates. The prerequisite  is a basic acquaintance with modern
algebra. While even the notion of a ring is introduced from scratch, it is
done so rather rapidly, and the reader is advised to consult another source.
In addition, we do not hesitate to use the language of categories,
which is developed in Chapter~\ref{categorychapter}. The book is intended to
provide preparation to study textbooks on algebraic geometry such as
\cite{Ha77}.

\section*{The CRing project}

The CRing project is an attempt, started by several undergraduates to create
a collaborative, open-source textbook on commutative algebra. The main website for this project is 
\url{http://people.fas.harvard.edu/~amathew/cr.html}.

In addition, there is a
git repository at \url{http://cring.adeel.ru}. The git repository provides 
slightly newer versions of the source code.

Discussion of the project can happen at the CRing project blog,
\url{http://cringproject.wordpress.com}.

\section*{Genesis of the project}
The project grew out of notes taken by Akhil Mathew in a class on commutative
algebra (Math 221) taught by Jacob Lurie in the fall of 2010 at Harvard.  


\section*{Corrections}
Please email corrections to
\verb=cring.project@gmail.com=.

\section*{Acknowledgments}

We thank Johan deJong for helpful advice and a blog link.


\section*{Version}
This file was last updated August 30, 2011.



