

\title{The CRing Project}
\author{Creative Commons Attribution 3.0 Unported License, \\
\url{http://creativecommons.org/licenses/by/3.0/} \\ 
}

\date{2010}

\maketitle

\fancyhead{}
\fancyfoot{}
\fancyfoot[C]{\small \textbf{\thepage}}
\fancyhead[LE, RO]{\thechapter, \S \arabic{section}}

\tableofcontents

\newpage 


\pagestyle{fancy}



\chapter*{Introduction}

The following is a massively collaborative, open source textbook on
commutative algebra. The project is currently in its infancy, and needs
contributions! See the next chapter for how to contribute.

\section{Prerequisites}
The prerequisite  is a basic acquaintance with modern
algebra. While even the notion of a ring is introduced from scratch, it is
done so rather rapidly, and the reader is advised to consult another source.
In addition, the notes do not hesitate to use the language of categories.
Besides that, the notes are mostly self-contained. (Material explaining the
category theory should be added sometime.) 

\section{Genesis of the project}
In the fall of 2010, Jacob Lurie taught a course (Math 221) on commutative
algebra at Harvard. 
The course started with foundational material, but swiftly progressed to touch on a wide range of
material, and culminated in the study of regular local rings. 
Akhil Mathew sat in on this course and took detailed
(``live-\TeX ed'') notes, which are still available on his website in unedited
form at 
\url{http://people.fas.harvard.edu/~amathew}. 

As the course was very well-taught, Akhil decided that it would be an interesting project to 
edit the notes I had taken into a mini-textbook of sort. This book started out as an
attempt 
at such a project, and the initial contribution was his notes. However, the
book is now intended to be a massively collaborative
project.

N.B. The following project is not endorsed by Jacob Lurie.

\section{Notes}

The present notes will remain \textbf{open source}; the source code can be
downloaded from \url{http://people.fas.harvard.edu/~amathew/cr.html}.

Please note that the Creative Commons Attribution
3.0 Unported License applies. For instance, this means that you are welcome to
use (e.g. copy and paste) pieces of this work if attribution is given.

\section{Corrections}
Please email corrections to
\verb=amathew@college.harvard.edu=.

\newpage

\chapter*{Contributions}

The current main source for the book comes from notes live-TeXed by Akhil
Mathew. It is hoped that eventually there will be a longer list of contributors!

A list of contributions will be maintained at 
\section{How to contribute}


{To contribute,} email submissions to \verb=amathew@college.harvard.edu=. 
Contributions do not have to be polished; they can be rough sketches written
for any purpose at all---half-finished homework writeups, term papers, blog
posts, and others are all welcome.

Contributions in editing the chapters are also welcome. To do this, simply
download the source, edit the files, and email the modifications to the same
address.
%\fancyfoot[C]{\thepage}
%\fancyhead[R]{{Notes on \thecoursename}}
%\fancyhead[L]{\textit{Lecture} \thesection}



\renewcommand{\whattosay}{Lecture \thesection \\ }
%\pagestyle{fancy}
