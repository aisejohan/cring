\chapter{Noetherian rings and modules}
\newcommand{\supp}{\mathrm{supp}}

The finiteness condition of a noetherian ring makes commutative algebra much
nicer.

\section{Basics}

\subsection{The noetherian condition}
\begin{definition} 
Let $R$ be a commutative ring and $M$ an $R$-module. We say that $M$ is
\textbf{noetherian} if every submodule of $M$ is finitely generated.
\end{definition} 

\begin{definition} 
$R$ is \textbf{noetherian} if $R$ is noetherian as an $R$-module. In
particular, this says that all of its ideals are finitely generated.
\end{definition} 

\begin{example} 
\begin{enumerate}
\item Any field is noetherian. There are two ideals: $(1)$ and $(0)$. 
\item Any PID is noetherian: any ideal is generated by one element. So
$\mathbb{Z}$ is noetherian.
\end{enumerate}
\end{example} 

First, let's just think about the condition of modules. Here is a convenient
reformulation of it.

\begin{proposition} $M$ is a module over $R$.
The following are equivalent:
\begin{enumerate}
\item $M$ is noetherian. 
\item Every chain of submodules of $M$, $M_0 \subset M_1 \subset \dots$,
eventually stabilizes at some $M_N$. (Ascending chain condition.)
\end{enumerate}
\end{proposition} 
\begin{proof} 
Say $M$ is noetherian and we have such a chain
\[ M_0 \subset M_1 \subset \dots.  \]
Write
\[ M' = \bigcup M_i \subset M,  \]
which is finitely generated since $M$ is noetherian. Let it be generated by
$x_1, \dots,x_n$. Each of these finitely many elements is in the union, so
they are all contained in some $M_N$. This means that
\[ M' \subset M_N, \quad \mathrm{so} \quad M_N = M'  \]
and the chain stabilizes.

For the converse, assume the ACC.  Let $M' \subset M$ be any submodule.  Define
a chain of submodules $M_0 \subset M_1 \subset  \dots \subset M'$ as follows. First, just take
$M_0 = \left\{0\right\}$. Take $M_{n+1}$ to be $M_n$ plus the submodule
generated by some $x \in M' - M_n$, if this is possible.  So $M_0$ is zero,
$M_1$ is generated by some nonzero element of $M'$, $M_2$ is $M_1$ together
with some element of $M'$ not in $M_1$. By construction, we have an ascending
chain, so it stabilizes at some finite place.  This means at some point, it is
impossible to choose something in $M'$ that does not belong to some $M_N$. In
particular, $M'$ is generated by $N$ elements, since $M_N$ is. 
\end{proof} 

\subsection{Stability properties}
\begin{proposition} 
If 
\[ M' \rightarrowtail  M \twoheadrightarrow M''  \]
is an exact sequence of modules, then $M$ is noetherian if and only if $M',
M''$ are.
\end{proposition} 

One direction says that noetherianness is preserved under subobjects and
quotients.
\begin{proof} 
If $M$ is noetherian, then every submodule of $M'$ is a submodule of $M$, so is
finitely generated. So $M'$ is noetherian too. Now we show that $M''$ is
noetherian. Let $N \subset M''$ and let
$\widetilde{N} \subset M$ the inverse image. Then $\widetilde{N}$ is finitely generated, so
$N$---as the homomorphic image of $\widetilde{N}$---is finitely generated 
So $M''$ is noetherian.

Suppose $M', M''$ noetherian. We prove $M$ noetherian.
Let's verify the ascending chain condition. Consider
\[ M_1 \subset M_2 \subset \dots \subset M.  \]
Let $M_i''$ denote the image of $M_i$ in $M''$ and let $M'_i$ be the
intersection of $M_i$ with $M'$. Here we think of $M'$ as a submodule of $M$.
These are ascending chains of submodules of $M', M''$, respectively, so they
stabilize by noetherianness.
So for some $N$, we have
that $n \geq N$ implies 
\[ M'_n = M'_{n+1}, \quad M''_n = M''_{n+1}.  \]

We claim that this implies, for such $n$, 
\[ M_n = M_{n+1}.  \]
Why? Say $x \in M_{n+1} \subset M$. Then $x$ maps into something in $M''_{n+1} = M''_n$.  
So there is something in $M_n$, call it $y$, such that $x,y$ go to the same
thing in $M''$. In particular, 
\[ x - y \in M_{n+1} \]
goes to zero in $M''$, so $x-y \in M'$. Thus $x-y \in M'_{n+1} = M'_n$. In
particular, 
\[ x = (x-y) + y \in M'_n + M_n = M_n.  \]
So $x \in M_n$, and 
\[ M_n = M_{n+1} . \]
This proves the result.
\end{proof} 

The class of noetherian modules is thus ``robust.'' We can get from that the
following.

\begin{proposition} 
If $\phi: A \to B$ is a surjection of commutative rings and $A$ is noetherian, then $B$ is
noetherian too.
\end{proposition} 
\begin{proof} 
Indeed, $B$ is noetherian as an $A$-module; indeed, it is the quotient of a
noetherian $A$-module (namely, $A$). However, it is easy to see that the
$A$-submodules of $B$ are just the $B$-modules in $B$, so $B$ is noetherian as a
$B$-module too. So $B$ is noetherian.  
\end{proof} 

Another easy stability property:

\begin{proposition} 
Let $R$ be a commutative ring, $S \subset R$ a multiplicatively closed subset.   If
$R$ is noetherian, then $S^{-1}R$ is noetherian.
\end{proposition} 
I.e., the class of noetherian rings is closed under localization.
\begin{proof} 
Say $\phi: R \to S^{-1}R$ is the canonical map. Let $I \subset S^{-1}R$ be an
ideal. Then $\phi^{-1}(I) \subset R$ is an ideal, so finitely generated. It
follows that
\[ \phi^{-1}(I)( S^{-1}R )\subset S^{-1}R  \]
is finitely generated as an ideal in $S^{-1}R$; the generators are the images
of the generators of $\phi^{-1}(I)$.

Now we claim that
\[  \phi^{-1}(I)( S^{-1}R ) = I . \]
The inclusion $\subset$ is trivial. For the latter inclusion, if $x/s \in I$,
then $x \in \phi^{-1}(I)$, so 
\[ x = (1/s) x \in (S^{-1}R) \phi^{-1}(I).  \] This proves the claim and
implies that $I$ is finitely generated.
\end{proof} 

\subsection{The basis theorem}
Let us now prove something a little less formal.

\begin{theorem}[Hilbert basis theorem]
If $R$ is a noetherian ring, then the polynomial ring $R[X]$ is noetherian.
\end{theorem} 
\begin{proof} 
Let $I \subset R[X]$ be an ideal. We prove that it is finitely generated. For
each $m \in \mathbb{Z}_{\geq 0}$, let $I(n)$ be the collection of elements 
$ a\in R$ consisting of the coefficients of $x^n$ of elements of $I$ of degree
$\leq n$.
This is an ideal, as is easily seen.

In fact, we claim that
\[ I(1) \subset I(2) \subset \dots  \]
which follows because if $ a\in I(1)$, there is an element $aX + \dots$ in $I$.
Thus $X(aX + \dots) = aX^2 + \dots \in I$, so $a \in I(2)$. And so on.

Since $R$ is noetherian, this chain stabilizes at some $I(N). $
Also, because $R$ is noetherian, each $I(n)$ is generated by finitely many
elements $a_{n,1}, \dots, a_{n, m_n} \in I(n)$. All of these come from polynomials
$P_{n,i} \in I$ such that $P_{n,i} = a_{n,i} X^n + \dots$.

The claim is that the $P_{n,i}$ for $n \leq N$ and $i \leq m_n$ generate $I$. 
This is a finite set of polynomials, so if we prove the claim, we will have
proved the basis theorem. Let $J$ be the ideal generated by
$\left\{P_{n,i}, n \leq N, i \leq m_n \right\}$. We know $J \subset I$. We must
prove $I \subset J$.

We will show that any element $P(X) \in I$ of degree $n$ belongs to $J$ by
induction on $n$. The degree is the largest nonzero coefficient. In particular,
the zero polynomial does not have a degree, but the zero polynomial is
obviously in $J$.

There are two cases. In the first case, $n \geq N$. Then we write
\[ P(X) = a X^n + \dots .  \] By definition $a \in I(n) = I(N)$ since the
chain of ideals $I(n)$ stabilized. Thus we can write $a$ in terms of the
generators:  $a = \sum a_{N, i} \lambda_i$ for some
$\lambda_i \in R$. Define the polynomial
\[ Q = \sum \lambda_i P_{N, i} x^{n-N} \in J.  \] Then $Q$ has degree $n$ and
the leading term is just $a$.  In particular, 
\[ P - Q  \]
is in $I$ and has degree less than $n$. By the inductive hypothesis, this
belongs to $J$, and since $Q \in J$, it follows that $P \in J$. 

Now consider the case of $n < N$. 
Again, we write $P(X) = a X^n + \dots$. Then $a \in I(n)$.  We can write 
\[ a = \sum a_{n,i} \lambda_i, \quad \lambda_i \in R.  \]
But the $a_{n,i} \in I(n)$. The polynomial
\[ Q = \sum \lambda_i P_{n,i}   \]
belongs to $J$ since $n  < N$. In the same way, $P-Q \in I$ has a lower degree.
Induction as before implies that $P \in J$. 
\end{proof} 


\begin{example} 
Let $k$ be a field. Then $k[x_1, \dots, x_n]$ is noetherian for any $n$, by the
Hilbert basis theorem and induction on $n$. 
\end{example} 


\begin{example} 
Any finitely generated commutative ring $R$ is noetherian. Indeed, then there
is a surjection
\[ \mathbb{Z}[x_1, \dots, x_n] \twoheadrightarrow R  \]
where the $x_i$ get mapped onto generators in $R$. The former is noetherian by
the basis theorem, and $R$ is as a quotient noetherian. 
\end{example} 


\begin{corollary} 
Any ring $R$ can be obtained as a filtered direct limit of noetherian rings.
\end{corollary} 
\begin{proof} 
Indeed, $R$ is the filtered direct limit of its finitely generated subrings. 
\end{proof} 
This observation is sometimes useful in commutative algebra and algebraic
geometry, in order to reduce questions about arbitrary commutative rings to
noetherian rings. Noetherian rings have strong finiteness hypotheses that let
you get numerical invariants that may be useful. For instance, we can do things
like inducting on the dimension for noetherian local rings.

\begin{example} 
Take $R = \mathbb{C}[x_1, \dots, x_n]$. For any algebraic variety $V$ defined
by polynomial equations, we know that $V$ is  the vanishing locus of some ideal
$I \subset R$. Using the Hilbert basis theorem, we have shown that $I$ is
finitely generated. This implies that $V$ can be described by \emph{finitely}
many polynomial equations. 
\end{example} 

\subsection{More on noetherian rings}
Let $R$ be a noetherian ring.  
\begin{proposition} 
An $R$-module $M$ is noetherian if and only if $M$ is finitely generated.
\end{proposition} 
\begin{proof} 
The only if direction is obvious. A module is noetherian if and only if every
submodule is finitely generated. 

For the if direction, if $M$ is finitely generated, then there is  a surjection
of $R$-modules
\[ R^n \to M  \]
where $R$ is noetherian. So $R^n$ is noetherian because it is a successive
extension of copies of $R$ and an extension of two noetherian modules is also
noetherian. So $M$ is a quotient of a noetherian module and is noetherian.

\end{proof} 

\section{Associated primes}

\renewcommand{\k}{\kappa}
Today, we will continue with the structure theory for noetherian modules.

\subsection{The support}
The first piece of intuition to have is the following. Let $R$ be noetherian;
consider $\spec R$. An $R$-module $M$ is supposed to be thought of as somehow
spread out over $\spec R$. If $\mathfrak{p} \in \spec R$, then 
\[ \k(\mathfrak{p}) = \mathrm{fr.  \ field \ } R/\mathfrak{p}  \]
which is the residue field of $R_{\mathfrak{p}}$. If $M$ is any $R$-module, we
can consider $M \otimes_R \k(\mathfrak{p})$ for each $\mathfrak{p}$; it is a
vector space over $\k(\mathfrak{p})$. If $M$ is finitely generated, then $M \otimes_R
\k(\mathfrak{p})$ is a finite-dimensional vector space.

\begin{definition} 
Let $M$ be a finitely generated $R$-module. Then $\supp M$ is defined to be the set of primes
$\mathfrak{p} \in \spec R$ such that
\[ M \otimes_R \k(\mathfrak{p}) \neq 0.  \]
\end{definition} 

You're supposed to think of a module $M$ as something like a vector bundle over
$\spec R$. At each $\mathfrak{p} \in \spec R$, we associate the vector space $M
\otimes_R \k(\mathfrak{p})$. It's not really a vector bundle, since the fibers
don't have to have the same dimension. For instance, the support of the
$\mathbb{Z}$-module $\mathbb{Z}/p$ just consists of the prime $(p)$. The fibers
don't have the same dimension.

Nonetheless, we can talk about the support, i.e. the set of spaces where the
vector space is not zero.

\newcommand{\ann}{\mathrm{Ann}}
\begin{remark} 
$\mathfrak{p} \in \supp M$ if and only if $M_{\mathfrak{p}} \neq 0$. This is
because
\[ (M \otimes_R R_{\mathfrak{p}})/ \mathfrak{p} R_{\mathfrak{p}} (M \otimes_R
R_{\mathfrak{p}})  = M_{\mathfrak{p}}
\otimes_{R_{\mathfrak{p}}} \k(\mathfrak{p})  \]
and we can use Nakayama's lemma over the local ring $R_{\mathfrak{p}}$.  (We
are using the fact that $M$ is finitely generated.)
\end{remark} 

\begin{remark} 
$M = 0$ if and only if $\supp M = \emptyset$. This is because $M= 0$ if and
only if $M_{\mathfrak{p}} = 0$ for all localizations.  We saw this earlier.
\end{remark} 

We will see soon that that $\supp M$ is closed in $\spec R$. You imagine that
$M$ lives on this closed subset $\supp M$, in some sense.


\newcommand{\ass}{\mathrm{Ass}}

\subsection{Associated primes}
Throughout, $R$ is noetherian.

\begin{definition} 
Let $M$ be a finitely generated $R$-module.  The prime ideal $\mathfrak{p}$ is said to be
\textbf{associated} to $M$ if there exists an element $x \in M$ such that
$\mathfrak{p}$ is the annihilator of $x$.  The set of associated primes is
$\ass(M)$.
\end{definition} 

Note that the annihilator of an element $x \in M$ is not necessarily prime, but
it is possible that the annihilator might be prime, in which case it is
associated.

The first claim is that there are some.
\begin{proposition} 
If $M \neq 0$, then there is an associated prime.
\end{proposition} 
\begin{proof} 
Let $I$ be a maximal element among the annihilators of nonzero elements $x \in M$. 
Then $1 \notin I$ because the annihilator of a nonzero element is not the full
ring. The existence of $I$ is guaranteed thanks to the noetherianness of
$R$.\footnote{It is a well-known argument that in a noetherian ring, any subset
of ideals contains a maximal element.}

So $I$ is the annihilator $\ann(x)$ of some $x \in M - \left\{0\right\}$. I claim that
$I$ is prime, hence an associated prime.  
Indeed, suppose $ab \in I$ where $a,b \in R$. This means that
\[ (ab)x \neq 0.  \]
Consider the annihilator of $bx$. This contains the annihilator of $x$, so $I$;
it also contains $a$. Maximality tells us that either $bx = 0$ (in which case
$b \in I$) or $\ann(bx) = I$ and  then $a \in \ann(bx) = I$. So either $a,b \in I$.
And $I$ is prime. 
\end{proof} 


\begin{proposition} \label{finiteassm}
Any finitely generated $R$-module has only finitely many associated primes.
\end{proposition}

The idea is going to be to use the fact that $M$ is finitely generated to build
$M$ out of finitely many pieces, and use that to bound the number of associated
primes to each piece.

\begin{lemma} 
Suppose we have an exact sequence of finitely generated $R$-modules
\[ 0 \to M' \to M \to M'' \to 0.  \]
Then 
\[\ass(M') \subset \ass(M) \subset \ass(M') \cup \ass(M'')  \]
\end{lemma} 
\begin{proof} 
The first claim is obvious. If $\mathfrak{p}$ is the annihilator of something
in $M'$, it is an annihilator of something in $M$ (namely its image), because
$M' \to M$ is injective. 

The hard direction is the other direction. Suppose $\mathfrak{p} \in \ass(M)$.
Then there is $x \in M$ such that
\[ \mathfrak{p} = \ann(x).  \]
Consider the submodule $Rx \subset M$.  If $Rx \cap M' \neq 0$, then we can
choose $y \in Rx \cap M' - \left\{0\right\}$. I claim that $\ann(y) =
\mathfrak{p}$ and so $\mathfrak{p} \in \ass(M')$.

Now $ y = ax$ for some $a \in R$. The annihilator of $y$ is the set of elements
$b \in R$ such that
\[ abx = 0  \]
or $ab \in \mathfrak{p}$. But $y = ax \neq 0$, so $a \notin \mathfrak{p}$. As a
result, the condition $b \in \ann(y)$ is the same as $b \in \mathfrak{p}$. In
other words, 
\[ \ann(y) = \mathfrak{p}  \]
which proves the claim.

What if the intersection $Rx \cap M' = 0$. Let $\phi: M \twoheadrightarrow M''$
be the surjection. I claim that $\mathfrak{p} = \ann(\phi(x))$ and
$\mathfrak{p} \in \ass(M'')$.  The proof is as follows. Clearly $\mathfrak{p}$
annihilates $\phi(x)$ as it annihilates $x$. Suppose $a \in \ann(\phi(x))$.
This means that $\phi(ax) = 0$, so $ax \in \ker \phi$; but $\ker \phi \cap Rx =
0$. So $ax = 0$ and $a \in \mathfrak{p}$. So $\ann(\phi(x)) = \mathfrak{p}$. 
\end{proof} 


\begin{lemma} 
For any finitely generated $R$-module $M$, there exists a finite filtration
\[ 0 = M_0 \subset M_1 \subset \dots \subset M_k = M  \]
such that the quotients are isomorphic to various $R/\mathfrak{p}_i$.
\end{lemma} 
\begin{proof} 
Let $M' \subset M$ be maximal among submodules for which such a filtration
exists. What we'd like to show is that $M' = M$, but a priori we don't know
this.  Now $M'$ is well-defined since $0$ has a filtration and $M$ is
noetherian.  There is a filtration
\[ 0 = M_0 \subset M_1 \subset \dots \subset M_l = M' \subset M.  \]
Now what can we say? If $M' = M$, we're done, as we said. Otherwise, look at
the quotient $M/M' \neq 0$. There is an associated prime of $M/M'$. So there is
a prime $\mathfrak{p}$ which is the annihilator of $x \in M/M'$. This means
that there is an injection 
\[ R/\mathfrak{p} \to M/M'.  \]
Now, we just make $M'$ bigger by taking $M_{l+1}$ as the inverse image in $M$
of $R/\mathfrak{p} \subset M/M'$. We have thus extended this filtration one
step further since $M_{l+1}/M_l \simeq R/\mathfrak{p}$, a contradiction since
$M'$ was maximal.
\end{proof} 

Now we are in a position to meet the goal. 
\begin{proof}[Pf of Proposition~\ref{finiteassm}]
Suppose $M$ is finitely generated Take our filtration
\[ 0 = M_0 \subset M_1 \subset \dots \subset M_k = M.  \]
By induction, we show that $\ass(M_i)$ is finite for each $i$. It is obviously
true for $i=0$. In general, we have an exact sequence
\[ 0 \to M_i \to M_{i+1} \to R/\mathfrak{p}_i \to 0  \]
which implies that
\[ \ass(M_{i+1}) \subset \ass(M_i) \cup \ass(R/\mathfrak{p}_i) = \ass(M_i)
\cup \left\{\mathfrak{p}_i\right\} . \]
This proves the claim and the proposition; it also shows that the number of
associated primes is at most the length of the filtration. 

\end{proof} 



Let us first describe how associated primes localize.
\begin{proposition} 
Let $R$ noetherian, $M$ finitely generated and $S \subset R$ multiplicatively closed. 
Then 
\[ \ass(S^{-1}M)  = \left\{S^{-1}\mathfrak{p}: \mathfrak{p} \in \ass(M),
\mathfrak{p}\cap S  = \emptyset \right\} . \]
\end{proposition} 
Here $S^{-1}M$ is considered as an $S^{-1}R$-module.

We've seen that prime ideals in $S^{-1}R$ can be identified as a subset of
$\spec R$. This shows that this notion is compatible with localization.

\begin{proof} 
We prove the easy direction. Suppose $\mathfrak{p} \in \ass(M)$ and
$\mathfrak{p} \cap S = \emptyset$. Then $\mathfrak{p} = \ann(x)$ for some $x
\in M$. Then the annihilator of $x/1$ is just $S^{-1}\mathfrak{p}$, as one
easily sees. Thus $S^{-1}\mathfrak{p} \in \ass(S^{-1}M)$.

The harder direction is left as an exercise.
\end{proof} 

The next claim is that the support and the associated primes are related.

\begin{proposition} The support is the closure of the associated primes:
\[ \supp M  = \bigcup_{\mathfrak{q} \in \ass(M)}
\overline{\left\{\mathfrak{q}\right\}} \]
\end{proposition} 

\begin{corollary} 
$\supp(M)$ is closed.
\end{corollary} 
\begin{proof} 
Indeed, the above result says that
\[ \supp M  = \bigcup_{\mathfrak{q} \in \ass(M)}
\overline{\left\{\mathfrak{q}\right\}}. \]
\end{proof} 
\begin{corollary} 
The ring $R$ has finitely many minimal prime ideals. 
\end{corollary} 
\begin{proof} 
Indeed, every minimal prime ideal is an associated prime for the $R$-module $R$
itself. Why is this? Well, $\supp R = \spec R$. Thus every prime ideal of $R$
contains an associated prime. And $R$ has finitely many associated primes.
\end{proof} 

\begin{remark} 
So $\spec R$ is the finite union of irreducible pieces
$\overline{\mathfrak{q}}$ if $R$ is noetherian.
\end{remark}

Let us prove the proposition.
\begin{proof} 
First, the easy direction. We show that $\supp(M)$ contains the set of primes
$\mathfrak{p}$ containing an associated prime. So let $\mathfrak{q}$ be an
associated prime and $\mathfrak{p} \supset \mathfrak{q}$. We show that
\[ \mathfrak{p} \in \supp M, \ i.e. \ M_{\mathfrak{p}} \neq 0.  \]
But there is an injective map
\[ R/\mathfrak{q} \to M  \]
so an injective map
\[ (R/\mathfrak{q})_{\mathfrak{p}} \to M_{\mathfrak{p}}  \]
where the first thing is nonzero since nothing nonzero in $R/\mathfrak{q}$ can be
annihilated by something not in $\mathfrak{p}$. So $M_{\mathfrak{p}} \neq 0$. 

The hard direction is the converse. Say that $\mathfrak{p} \in \supp M$. We
have to show that $\mathfrak{p}$ contains an associated prime.  
Now $M_{\mathfrak{p}} \neq 0$ and it is a finitely generated $R_{\mathfrak{p}}$-module, where
$R_{\mathfrak{p}}$ is noetherian. So this has an associated prime. 
\[ \ass(M_{\mathfrak{p}}) \neq \emptyset  \]
and we can find an element $\mathfrak{q}_{\mathfrak{p}} \subset
R_{\mathfrak{p}}$ in there, where
$\mathfrak{q}$ is a prime of $R$ contained in $\mathfrak{p}$. But by the above
fact about localization and associated primes, we have that 
\[ \mathfrak{q} \in \ass(M)  \]
and we have already seen that $\mathfrak{q} \subset \mathfrak{p}$. This proves
the other inclusion and establishes the result. 
\end{proof} 

We have just seen that $\supp M$ is a closed subset of $\spec R$ and is a union
of finitely many irreducible subsets.  More precisely, 
\[ \supp M = \bigcup_{\mathfrak{q} \in \ass(M)}
\overline{\left\{\mathfrak{q}\right\}}  \]
though there might be some redundancy in this expression. Some associated prime might be contained
in others.  

\begin{definition} 
A prime $\mathfrak{p} \in \ass(M)$ is an \textbf{isolated} associated prime of
$M$ if it is minimal (with respect to the ordering on $\ass(M)$); it is
\textbf{embedded} otherwise. 
\end{definition} 

So the embedded primes are not needed to describe the support of $M$. 

\subsection{The case of one associated prime}
\begin{proposition} 
Let $M$ be a finitely generated $R$-module. Then 
\[ \supp M = \left\{\mathfrak{p} \in \spec R: \mathfrak{p} \
\mathrm{contains \ an \ associated \ prime}\right\} . \]
\end{proposition} 



\begin{definition} 
A finitely generated $R$-module $M$ is \textbf{$\mathfrak{p}$-primary} if 
\[ \ass(M) = \left\{\mathfrak{p}\right\} . \]
If $\ass(M)$ consists of a point, we call $M$ \textbf{primary}.
\end{definition} 


\lecture{10/4}

\textbf{For the remainder of this lecture, $R$ is a noetherian ring, and $M$ a
finitely generated $R$-module. $S \subset R$ is a multiplicatively closed
subset.}


\subsection{A loose end}
Let us start with an assertion we made last time, but we didn't prove. Namely,
that
\[ \ass(S^{-1}M) = \left\{S^{-1}\mathfrak{p}, \mathfrak{p} \in \ass(M),
\mathfrak{p} \cap S = \emptyset\right\}.  \]
We proved the easy direction, that if $\mathfrak{p} \in \ass(M)$ and does not
intersect $S$, then $S^{-1}\mathfrak{p}$ is an associated prime of $S^{-1}M$.

\begin{proposition} 
The reverse inclusion also holds.
\end{proposition} 
\begin{proof} 
Let $\mathfrak{q} \in \ass(S^{-1}M)$. This means that $\mathfrak{q} =
\ann(x/s)$ for some $x \in M$, $s \in S$. 

Call the map $R \to S^{-1}R  $ to be $\phi$.
Then $\phi^{-1}(\mathfrak{q})$ is the set of elements $a \in R$ such that
\[ \frac{ax}{s} = 0 \in S^{-1}M . \]
In other words, by definition of the localization, this is 
\[ \bigcup_{t \in S} \left\{a \in R: atx = 0 \in M\right\} = \bigcup \ann(tx)
\subset R.\]
We know, however, that among elements of the form $\ann(tx)$, there is a
\emph{maximal} element $I=\ann(t_0 x)$ for some $t_0 \in S$. Indeed, $R$ is
noetherian. 
Then if you think about any other annihilator $I' = \ann(tx)$, then $I', I$ are
both contained in $\ann(t_0 t x)$. However, 
\[ I \subset \ann(t_0 x)  \]
and $I$ is maximal, so $I = \ann(t_0 t x)$ and
\[ I' \subset I.  \] That is $I$ contains all these other annihilators.
In particular, the big union above, i.e. $\phi^{-1}(\mathfrak{q})$, is just
\[ I = \ann(t_0 x).  \]
It follows that $\phi^{-1}(\mathfrak{q})$ is the annihilator of $\ann(t_0 x)$,
so this is an associated prime of $M$. This means that every associated prime
of $S^{-1}M$ comes from an associated prime of $M$. That completes the proof.
\end{proof} 

\subsection{Primary modules}

\begin{definition} 
Let $\mathfrak{p} \subset R$ be prime, $M$ a finitely generated $R$-module. Then $M$ is
\textbf{$\mathfrak{p}$-primary} if 
\[ \ass(M) = \left\{\mathfrak{p}\right\}.  \]
Let's say that the zero module is not primary.

A module is \textbf{primary} if it is $\mathfrak{p}$-primary for some
$\mathfrak{p}$, i.e. has precisely one associated prime. 
\end{definition} 

\begin{proposition} 
Let $M$ be a finitely generated $R$-module. Then $M$ is \textbf{$\mathfrak{p}$}-primary if
and only if, for every $m \in M - \left\{0\right\}$, 
the annihilator $\ann(m)$ has radical $\mathfrak{p}$.
\end{proposition} 
\begin{proof} 
We first need a small observation.

\begin{lemma} 
If $M$ is $\mathfrak{p}$-primary, so is any nonzero submodule of $M$ is
$\mathfrak{p}$-primary.
\end{lemma} 
\begin{proof} 
Indeed, any associated prime of the submodule is an associated prime of $M$.
Note that the submodule, if it is nonzero, it has an associated prime. That has
to be $\mathfrak{p}$.
\end{proof} 

Assume first $M$ to be $\mathfrak{p}$-primary. Let $x \in M$, $x \neq 0$. Let
$I = \ann(x)$. So by definition there is an injection
\[ R/I \to M  \]
sending $1 \to x$. As a result, $R/I$ is $\mathfrak{p}$-primary by the above
lemma. We want to know that $\mathfrak{p}  = \rad(I)$. 
We saw that the support $\supp R/I = \left\{\mathfrak{q}: \mathfrak{q}
\supset I\right\}$ is the union of the closures of the associated primes. In
this case, 
\[ \supp(R/I) = \left\{\mathfrak{q}: \mathfrak{q} \supset \mathfrak{p}\right\}
.\]
But we know that $\rad(I) = \bigcap_{\mathfrak{q} \supset I} \mathfrak{q}$,
which by the above is just $\mathfrak{p}$. This proves that $\rad(I) =
\mathfrak{p}$.
We have shown that if $R/I$ is primary, then $I$ has radical $\mathfrak{p}$.

The converse is easy. 
Suppose the condition holds and $\mathfrak{q} \in \ass(M)$, so $\mathfrak{q} =
\ann(x)$ for $x \neq 0$. But then $\rad(\mathfrak{q}) = \mathfrak{p}$, so 
\[ \mathfrak{q} = \mathfrak{p}  \] and $\ass(M) = \left\{\mathfrak{p}\right\}$.
\end{proof} 

We have another characterization.

\begin{proposition}
Let $M \neq 0$ be a finitely generated $R$-module. Then $M$ is primary iff for each $a \in
R$, either multiplication $a: M \to M$ is injective or nilpotent.
\end{proposition}
\begin{proof} 
Suppose $M$ to be $\mathfrak{p}$-primary. Then multiplication by anything in
$\mathfrak{p}$ is nilpotent because the annihilator of everything nonzero has
radical $\mathfrak{p}$. But if $a \notin \mathfrak{p}$, then $\ann(x)$ for
$x \in M - \left\{0\right\}$ has radical $\mathfrak{p}$ and cannot contain $a$. 

Other direction, now. Assume that every element of $a$ acts either injectively or nilpotently on $M$.
Let $I \subset R$ be the collection of elements $a \in R$ such that $a^n M = 0$
for $n$ large. Then $I$ is an ideal; it is closed under addition by the
binomial formula. If $a, b \in I$ and $a^n, b^n$ act by zero, then $(a+b)^{2n}$
acts by zero as well.


I claim that $I$ is actually prime. If $a,b \notin I$, then $a,b$ act by
multiplication injectively on $I$. So $a: M \to M, b: M \to M$ are injective.
However, a composition of injections is injective, so $ab$ acts injectively and
$ab \notin I$. So $I$ is prime.

We need now to check that if $x \in M$ is nonzero, then $\ann(x)$ has radical
$I$. This is because something $a \in R$ has a power that kills $x$,
multiplication $M \stackrel{a}{\to} M$ can't be injective, so it must be
nilpotent. Conversely, if $a \in I$, then a power of $a$ is nilpotent, so it
must kill $x$. 
\end{proof} 

So we have this notion of a primary module. The idea is that all the torsion is
somehow concentrated in some prime.

\section{Primary decomposition} This is the structure theorem for modules
over a noetherian ring, in some sense.

\begin{definition} 
Let $M$ be a finitely generated $R$-module. A submodule $N \subset M$ is
\textbf{$\mathfrak{p}$-coprimary} if $M/N$ is $\mathfrak{p}$-primary.

Similarly, we can say that $N \subset M$ is \textbf{coprimary}.
\end{definition} 

\begin{definition} 
$N \subsetneq M$ is \textbf{irreducible} if whenever $N = N_1 \cap N_2$ for $N_1,
N_2 \subset M$, then either one of $N_1, N_2$ equals $N$. It is not
nontrivially the intersection of larger submodules. 
\end{definition} 

\begin{proposition} 
An irreducible submodule $N \subset M$ is coprimary.
\end{proposition} 
\begin{proof} 
Say $a \in R$. We'd like to show that 
\[ M/N \stackrel{a}{\to} M/N  \]
is either injective or nilpotent.
Consider  the following submodule of $M/N$:
\[ K(n) =  \left\{x \in M/N: a^n x = 0\right\} . \]
Then $K(0) \subset K(1) \subset \dots$; this chain stops by noetherianness as
the quotient module is noetherian.
In particular, $K(n) = K(2n)$ for large $n$. 

In particular, if $x \in M/N$ is divisible by $a^n$ ($n$ large) and nonzero, then $a^n x$
is also nonzero. Indeed, say $x = a^n y$; then $y \notin K(n)$, so $a^{n}x =
a^{2n}y \neq 0$ or we would have $y \in K(2n) = K(n)$. In $M/N$, the submodules
\[ a^n(M/N) \cap \ker(a^n)  \]
are equal to zero for large $n$. But our assumption was that $N$ is
irreducible.  So one of these submodules of $M/N$ is zero. I.e., either
$a^n(M/N) = 0$ or $\ker a^n = 0$. We get either injectivity or nilpotence on
$M/N$. This proves the result.
\end{proof} 

\begin{proposition} 
$M$ has an irreducible decomposition. There exist finitely many irreducible
submodules $N_1, \dots, N_k$ with
\[  N_1 \cap \dots \cap N_k = 0. \]
\end{proposition} 
In other words,
\[  M \to \bigoplus M/N_i  \]
is injective.
So a finitely generated module over a noetherian ring can be imbedded in a direct sum of
primary modules. 

\begin{proof} 
Let $M' \subset M$ be a maximal submodule of $M$ such that $M'$ cannot be
written as an intersection of finitely many irreducible submodules. If no such
$M'$ exists, then we're done, because then $0$ can be written as an
intersection of finitely many irreducible submodules.

Now $M'$ is not irreducible, or it would be the intersection of one irreducible
submodule.
Then $M'$ can be written as $M_1' \cap M_2'$ for two strictly
larger submodules of $M$.  But $M_1', M_2'$ admit decompositions as
intersections of irreducibles. So $M'$ does as well, contradiction. 
\end{proof} 

For any $M$, we have an \textbf{irreducible decomposition}
\[ 0 = \bigcap N_i  \]
for the $N_i$ a finite set of irreducible (and thus coprimary) submodules. 
This decomposition here is highly non-unique and non-canonical. Let's try to
pare it down to something which is a lot more canonical.

The first claim is that we can collect together modules which are coprimary for
some prime. 
\begin{lemma} 
Let $N_1, N_2 \subset M$ be $\mathfrak{p}$-coprimary submodules. Then $N_1 \cap
N_2$ is also $\mathfrak{p}$-coprimary.
\end{lemma} 
\begin{proof} 
We have to show that $M/N_1 \cap N_2$ is $\mathfrak{p}$-primary. Indeed, we have an injection
\[ M/N_1 \cap N_2 \rightarrowtail  M/N_1 \oplus M/N_2  \]
which implies that $\ass(M/N_1 \cap N_2) \subset \ass(M/N_1) \cup \ass(M/N_2) =
\left\{\mathfrak{p}\right\}$. So we're done. 
\end{proof} 

In particular, if we don't want irreducibility but only primariness in the
decomposition
\[ 0 = \bigcap N_i,  \]
we can assume that each $N_i$ is $\mathfrak{p}_i$ coprimary for some prime
$\mathfrak{p}_i$ with the $\mathfrak{p}_i$ distinct.

\begin{definition} 
Such a decomposition of zero is called a \textbf{primary decomposition}.
\end{definition} 

We can further assume that 
\[ N_i \not\supset \bigcap_{j \neq i} N_j  \]
or we could omit one of the $N_i$. Let's assume that the decomposition is
minimal.
Then the decomposition is called a \textbf{reduced primary decomposition}.

Again, what this tells us is that $M \rightarrowtail  \bigoplus M/N_i$. What we
have shown is that $M$ can be imbedded in a sum of pieces, each of which is
$\mathfrak{p}$-primary for some prime, and the different primes are distinct.

This is \textbf{not} unique. However, 

\begin{proposition} 
The primes $\mathfrak{p}_i$ that appear in a reduced primary decomposition of zero are
uniquely determined. They are the associated primes of $M$.
\end{proposition} 
\begin{proof} 
All the associated primes of $M$ have to be there, because we have the injection
\[ M \rightarrowtail  \bigoplus M/N_i  \]
so the associated primes of $M$ are among those of $M/N_i$ (i.e. the
$\mathfrak{p}_i$).

The hard direction is to see that each $\mathfrak{p}_i$ is an associated prime.
I.e. if $M/N_i$ is $\mathfrak{p}_i$-primary, then $\mathfrak{p}_i \in \ass(M)$;
we don't need to use primary modules except for primes in the associated primes. 
Here we need to use the fact that our decomposition has no redundancy.  Without
loss of generality, it suffices to show that $\mathfrak{p}_1$, for instance,
belongs to $\ass(M)$. We will use the fact that
\[ N_1 \not\supset N_2 \cap \dots .  \]
So this tells us that $N_2 \cap N_3 \cap \dots$ is not equal to zero, or we
would have a containment. We have a map
\[ N_2 \cap \dots \cap N_k \to M/N_1;  \]
it is injective, since the kernel is $N_1 \cap N_2 \cap \dots \cap N_k = 0$ as
this is a decomposition.
However, $M/N_1$ is $\mathfrak{p}_1$-primary, so $N_2 \cap \dots \cap N_k$ is
$\mathfrak{p}_1$-primary. In particular, $\mathfrak{p}_1$ is an associated
prime of $N_2 \cap \dots \cap N_k$, hence of  $M$.
\end{proof} 

The primes are determined. The factors are not. However, in some cases they are.

\begin{proposition} 
Let $\mathfrak{p}_i$ be a minimal associated prime of $M$, i.e. not containing
any smaller associated prime. Then the submodule $N_i$  corresponding to
$\mathfrak{p}_i$ in the reduced primary decomposition is uniquely determined:
it is the kernel of 
\[ M \to M_{\mathfrak{p}_i}.  \]
\end{proposition} 

\begin{proof} 
We have that $\bigcap N_j = \left\{0\right\} \subset M$. When we localize at
$\mathfrak{p}_i$, we find that
\[ (\bigcap N_j)_{\mathfrak{p}_i} = \bigcap (N_j)_{\mathfrak{p}_i} =0 \]
as localization is an exact functor. If $j \neq i$, then $M/N_j$ is
$\mathfrak{p}_j$ primary, and has only $\mathfrak{p}_j$ as an associated prime.
It follows that $(M/N_j)_{\mathfrak{p}_i}$ has no associated primes, since the
only associated prime could be $\mathfrak{p}_j$, and that's not contained in
$\mathfrak{p}_j$.
In particular, $(N_j)_{\mathfrak{p}_i} = M_{\mathfrak{p}_i}$.

Thus, when we localize the primary decomposition at $\mathfrak{p}_i$, we get
a trivial primary decomposition: most of the factors are the full
$M_{\mathfrak{p}_i}$.  It follows that $(N_i)_{\mathfrak{p}_i}=0$. When we draw
a commutative diagram
\[ 
\xymatrix{
N_i \ar[r] \ar[d]  &  (N_i)_{\mathfrak{p}_i} = 0 \ar[d]  \\
M \ar[r] &  M_{\mathfrak{p}_i}.
}
\]
we find that $N_i$ goes to zero in the localization.

Now if $x \in \ker(M \to M_{\mathfrak{p}_i}$, then $sx = 0$ for some $s \notin
\mathfrak{p}_i$. When we take the map $M \to M/N_i$, $sx$ maps to zero; but $s$
acts injectively on $M/N_i$, so $x$ maps to zero in $M/N_i$, i.e. is zero in
$N_i$.
\end{proof} 

This has been abstract, so:
\begin{example} Let $ R = \mathbb{Z}$.
Let $M = \mathbb{Z} \oplus \mathbb{Z}/p$. Then zero can be written as 
\[ \mathbb{Z} \cap \mathbb{Z}/p  \]
as submodules of $M$. But $\mathbb{Z}$ is $\mathfrak{p}$-coprimary, while
$\mathbb{Z}/p$ is $(0)$-coprimary. 

This is not unique. We could have considered 
\[ \{(n,n), n \in \mathbb{Z}\} \subset M.  \]
However, the zero-coprimary part has to be the $p$-torsion. This is because
$(0)$ is the minimal ideal. 

The decomposition is always unique, in general, if
we have no inclusions among the prime ideals. For $\mathbb{Z}$-modules, this
means that primary decomposition is unique for torsion modules. 
Any torsion group is a direct sum of the $p$-power torsion over all primes $p$. 
\end{example} 

