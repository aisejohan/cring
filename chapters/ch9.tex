\chapter{\'Etale, unramified, and smooth morphisms}

(The following code was contributed by the Stacks Project authors.)
\section{Formally unramified maps}
\label{section-formally-unramified}

\noindent
It turns out to be logically more efficient to define
the notion of a formally unramified map before introducing
the notion of a formally \'etale one.

\begin{definition}
\label{definition-formally-unramified}
Let $R \to S$ be a ring map.
We say $S$ is {\it formally unramified over $R$} if for every
commutative solid diagram
$$
\xymatrix{
S \ar[r] \ar@{-->}[rd] & A/I \\
R \ar[r] \ar[u] & A \ar[u]
}
$$
where $I \subset A$ is an ideal of square zero, there exists
at most one dotted arrow making the diagram commute.
\end{definition}

\begin{lemma}
\label{lemma-characterize-formally-unramified}
Let $R \to S$ be a ring map.
The following are equivalent:
\begin{enumerate}
\item $R \to S$ is formally unramified,
\item the module of differentials $\Omega_{S/R}$ is zero.
\end{enumerate}
\end{lemma}

\begin{proof}
Let $J = \text{Ker}(S \otimes_R S \to S)$ be the kernel of
the multiplication map. Let $A_{univ} = S \otimes_R S/J^2$. Recall
that $I_{univ} = J/J^2$ is isomorphic to $\Omega_{S/R}$, see
Lemma \ref{lemma-differentials-diagonal}. Moreover, the two $R$-algebra maps
$\sigma_1, \sigma_2 : S \to A_{univ}$, $\sigma_1(s) = s \otimes 1 \bmod J^2$,
and $\sigma_2(s) = 1 \otimes s \bmod J^2$ differ by the
universal derivation $\text{d} : S \to \Omega_{S/R} = I_{univ}$.

\medskip\noindent
Assume $R \to S$ formally unramified.
Then we see that $\sigma_1 = \sigma_2$.
Hence $\text{d}(s) = 0$ for all $s \in S$.
Hence $\Omega_{S/R} = 0$.

\medskip\noindent
Assume that $\Omega_{S/R} = 0$. Let $A, I, R \to A, S \to A/I$
be a solid diagram as in Definition \ref{definition-formally-unramified}.
Let $\tau_1, \tau_2 : S \to A$ be two dotted arrows making the
diagram commute. Consider the $R$-algebra map $A_{univ} \to A$
defined by the rule $s_1 \otimes s_2 \mapsto \tau_1(s_1)\tau_2(s_2)$.
We omit the verification that this is well defined. Since $A_{univ} \cong S$
as $I_{univ} = \Omega_{S/R} = 0$ we conclude that $\tau_1 = \tau_2$.
\end{proof}

\begin{lemma}
\label{lemma-formally-unramified-local}
Let $R \to S$ be a ring map.
The following are equivalent:
\begin{enumerate}
\item $R \to S$ is formally unramified,
\item $R \to S_{\mathfrak q}$ is formally unramified for all
primes $\mathfrak q$ of $S$, and
\item $R_{\mathfrak p} \to S_{\mathfrak q}$ is formally unramified
for all primes $\mathfrak q$ of $S$ with $\mathfrak p = R \cap \mathfrak q$.
\end{enumerate}
\end{lemma}

\begin{proof}
We have seen in
Lemma \ref{lemma-characterize-formally-unramified}
that (1) is equivalent to
$\Omega_{S/R} = 0$. Similarly, by
Lemma \ref{lemma-differentials-localize}
we see that (2) and (3)
are equivalent to $(\Omega_{S/R})_{\mathfrak q} = 0$ for all
$\mathfrak q$. Hence the equivalence follows from
Lemma \ref{lemma-characterize-zero-local}.
\end{proof}

\begin{lemma}
\label{lemma-formally-unramified-localize}
Let $A \to B$ be a formally unramified ring map.
\begin{enumerate}
\item For $S \subset A$ a multiplicative subset,
$S^{-1}A \to S^{-1}B$ is formally unramified.
\item For $S \subset B$ a multiplicative subset,
$A \to S^{-1}B$ is formally unramified.
\end{enumerate}
\end{lemma}

\begin{proof}
Follows from
Lemma \ref{lemma-formally-unramified-local}.
(You can also deduce it from
Lemma \ref{lemma-characterize-formally-unramified}
combined with
Lemma \ref{lemma-differentials-localize}.)
\end{proof}



\section{Conormal modules and universal thickenings}
\label{section-conormal}

\noindent
It turns out that one can define the first infinitesimal neighbourhood
not just for a closed immersion of schemes, but already for any formally
unramified morphism. This is based on the following algebraic fact.

\begin{lemma}
\label{lemma-universal-thickening}
Let $R \to S$ be a formally unramified ring map. There exists a surjection of
$R$-algebras $S' \to S$ whose kernel is an ideal of square zero with the
following universal property: Given any commutative diagram
$$
\xymatrix{
S \ar[r]_{a} & A/I \\
R \ar[r]^b \ar[u] & A \ar[u]
}
$$
where $I \subset A$ is an ideal of square zero, there is a unique $R$-algebra
map $a' : S' \to A$ such that $S' \to A \to A/I$ is equal to $S' \to S \to A$.
\end{lemma}

\begin{proof}
Choose a set of generators $z_i \in S$, $i \in I$ for $S$ as an $R$-algebra.
Let $P = R[\{x_i\_{i \in I}]$ denote the polynomial ring on generators
$x_i$, $i \in I$. Consider the $R$-algebra map $P \to S$ which maps
$x_i$ to $z_i$. Let $J = \text{Ker}(P \to S)$. Consider the map
$$
\text{d} : J/J^2 \longrightarrow \Omega_{P/R} \otimes_P S
$$
see
Lemma \ref{lemma-differential-seq}.
This is surjective since $\Omega_{S/R} = 0$ by assumption, see
Lemma \ref{lemma-characterize-formally-unramified}.
Note that $\Omega_{P/R}$ is free on $\text{d}x_i$, and hence the module
$\Omega_{P/R} \otimes_P S$ is free over $S$. Thus we may choose a splitting
of the surjection above and write
$$
J/J^2 = K \oplus \Omega_{P/R} \otimes_P S
$$
Let $J^2 \subset J' \subset J$ be the ideal of $P$ such that
$J'/J^2$ is the second summand in the decomposition above.
Set $S' = P/J'$. We obtain a short exact sequence
$$
0 \to J/J' \to S' \to S \to 0
$$
and we see that $J/J' \cong K$ is a square zero ideal in $S'$. Hence
$$
\xymatrix{
S \ar[r]_1 & S \\
R \ar[r] \ar[u] & S' \ar[u]
}
$$
is a diagram as above. In fact we claim that this is an initial object in
the category of diagrams. Namely, let $(I \subset A, a, b)$ be an arbitrary
diagram. We may choose an $R$-algebra map $\beta : P \to A$ such that
$$
\xymatrix{
S \ar[r]_1 & S \ar[r]_a & A/I \\
R \ar[r] \ar@/_/[rr]_b \ar[u] & P \ar[u] \ar[r]^\beta & A \ar[u]
}
$$
is commutative. Now it may not be the case that $\beta(J') = 0$, in other
words it may not be true that $\beta$ factors through $S' = P/J'$.
But what is clear is that $\beta(J') \subset I$ and
since $\beta(J) \subset I$ and $I^2 = 0$ we have $\beta(J^2) = 0$.
Thus the ``obstruction'' to finding a morphism from
$(J/J' \subset S', 1, R \to S')$ to $(I \subset A, a, b)$ is
the corresponding $S$-linear map $\overline{\beta} : J'/J^2 \to I$.
The choice in picking $\beta$ lies in the choice of $\beta(x_i)$.
A different choice of $\beta$, say $\beta'$, is gotten by taking
$\beta'(x_i) = \beta(x_i) + \delta_i$ with $\delta_i \in I$.
In this case, for $g \in J'$, we obtain
$$
\beta'(g) =
\beta(g) + \sum\nolimits_i \delta_i \frac{\partial g}{\partial x_i}.
$$
Since the map $\text{d}|_{J'/J^2} : J'/J^2 \to \Omega_{P/R} \otimes_P S$
given by $g \mapsto \frac{\partial g}{\partial x_i}\text{d}x_i$
is an isomorphism by construction, we see that there is a unique choice
of $\delta_i \in I$ such that $\beta'(g) = 0$ for all $g \in J'$.
(Namely, $\delta_i$ is $-\overline{\beta}(g)$ where $g \in J'/J^2$
is the unique element with $\frac{\partial g}{\partial x_j} = 1$ if
$i = j$ and $0$ else.) The uniqueness of the solution implies the
uniqueness required in the lemma.
\end{proof}

\noindent
In the situation of
Lemma \ref{lemma-universal-thickening}
the $R$-algebra map $S' \to S$ is unique up to unique isomorphism.

\begin{definition}
\label{definition-universal-thickening}
Let $R \to S$ be a formally unramified ring map.
\begin{enumerate}
\item The {\it universal first order thickening} of $S$ over $R$ is
the surjection of $R$-algebras $S' \to S$ of
Lemma \ref{lemma-universal-thickening}.
\item The {\it conormal module} of $R \to S$ is the kernel $I$ of the
universal first order thickening $S' \to S$, seen as a $S$-module.
\end{enumerate}
We often denote the conormal module {\it $C_{S/R}$} in this situation.
\end{definition}

\begin{lemma}
\label{lemma-universal-thickening-quotient}
Let $I \subset R$ be an ideal of a ring.
The universal first order thickening of $R/I$ over $R$
is the surjection $R/I^2 \to R/I$. The conormal module
of $R/I$ over $R$ is $C_{(R/I)/R} = I/I^2$.
\end{lemma}

\begin{proof}
Omitted.
\end{proof}

\begin{lemma}
\label{lemma-universal-thickening-localize}
Let $A \to B$ be a formally unramified ring map.
Let $\varphi : B' \to B$ be the universal first order thickening of
$B$ over $A$.
\begin{enumerate}
\item Let $S \subset A$ be a multiplicative subset.
Then $S^{-1}B' \to S^{-1}B$ is the universal first order thickening of
$S^{-1}B$ over $S^{-1}A$. In particular $S^{-1}C_{B/A} = C_{S^{-1}B/S^{-1}A}$.
\item Let $S \subset B$ be a multiplicative subset.
Then $S' = \varphi^{-1}(S)$ is a multiplicative subset in $B'$
and $(S')^{-1}B' \to S^{-1}B$ is the universal first order thickening
of $S^{-1}B$ over $A$. In particular $S^{-1}C_{B/A} = C_{S^{-1}B/A}$.
\end{enumerate}
Note that the lemma makes sense by
Lemma \ref{lemma-formally-unramified-localize}.
\end{lemma}

\begin{proof}
With notation and assumptions as in (1). Let $(S^{-1}B)' \to S^{-1}B$
be the universal first order thickening of $S^{-1}B$ over $S^{-1}A$.
Note that $S^{-1}B' \to S^{-1}B$ is a surjection of $S^{-1}A$-algebras
whose kernel has square zero. Hence by definition we obtain a map
$(S^{-1}B)' \to S^{-1}B'$ compatible with the maps towards $S^{-1}B$.
Consider any commutative diagram
$$
\xymatrix{
B \ar[r] & S^{-1}B \ar[r] & D/I \\
A \ar[r] \ar[u] & S^{-1}A \ar[r] \ar[u] & D \ar[u]
}
$$
where $I \subset D$ is an ideal of square zero. Since $B'$ is the universal
first order thickening of $B$ over $A$ we obtain an $A$-algebra map
$B' \to D$. But it is clear that the image of $S$ in $D$ is mapped to
invertible elements of $D$, and hence we obtain a compatible map
$S^{-1}B' \to D$. Applying this to $D = (S^{-1}B)'$ we see that we get
a map $S^{-1}B' \to (S^{-1}B)'$. We omit the verification that this map
is inverse to the map described above.

\medskip\noindent
With notation and assumptions as in (2). Let $(S^{-1}B)' \to S^{-1}B$
be the universal first order thickening of $S^{-1}B$ over $A$.
Note that $(S')^{-1}B' \to S^{-1}B$ is a surjection of $A$-algebras
whose kernel has square zero. Hence by definition we obtain a map
$(S^{-1}B)' \to (S')^{-1}B'$ compatible with the maps towards $S^{-1}B$.
Consider any commutative diagram
$$
\xymatrix{
B \ar[r] & S^{-1}B \ar[r] & D/I \\
A \ar[r] \ar[u] & A \ar[r] \ar[u] & D \ar[u]
}
$$
where $I \subset D$ is an ideal of square zero. Since $B'$ is the universal
first order thickening of $B$ over $A$ we obtain an $A$-algebra map
$B' \to D$. But it is clear that the image of $S'$ in $D$ is mapped to
invertible elements of $D$, and hence we obtain a compatible map
$(S')^{-1}B' \to D$. Applying this to $D = (S^{-1}B)'$ we see that we get
a map $(S')^{-1}B' \to (S^{-1}B)'$. We omit the verification that this map
is inverse to the map described above.
\end{proof}

\begin{lemma}
\label{lemma-differentials-universal-thickening}
Let $R \to A  \to B$ be ring maps. Assume $A \to B$ formally unramified.
Let $B' \to B$ be the universal first order thickening of $B$ over $A$.
Then $B'$ is formally unramified over $A$, and the canonical map
$\Omega_{A/R} \otimes_A B \to \Omega_{B'/R} \otimes_{B'} B$ is an
isomorphism.
\end{lemma}

\begin{proof}
We are going to use the construction of $B'$ from the proof of
Lemma \ref{lemma-universal-thickening}
allthough in principle it should be possible to deduce these results
formally from the definition. Namely, we choose a presentation
$B = P/J$, where $P = A[x_i]$ is a polynomial ring over $A$.
Next, we choose elements $f_i \in J$ such that
$\text{d}f_i = \text{d}x_i \otimes 1$ in $\Omega_{P/A} \otimes_P B$.
Having made these choices we have
$B' = P/J'$ with $J' = (f_i) + J^2$, see proof of
Lemma \ref{lemma-universal-thickening}.

\medskip\noindent
Consider the canonical exact sequence
$$
J'/(J')^2 \to \Omega_{P/A} \otimes_P B' \to \Omega_{B'/A} \to 0
$$
see
Lemma \ref{lemma-differential-seq}.
By construction the classes of the $f_i \in J'$ map to elements of
the module $\Omega_{P/A} \otimes_P B'$ which generate it modulo
$J'/J^2$ by construction. Since $J'/J^2$ is a nilpotent ideal, we see
that these elements generate the module alltogether (by
Nakayama's Lemma \ref{lemma-NAK}). This proves that $\Omega_{B'/A} = 0$
and hence that $B'$ is formally unramified over $A$, see
Lemma \ref{lemma-characterize-formally-unramified}.

\medskip\noindent
Since $P$ is a polynomial ring over $A$ we have
$\Omega_{P/R} = \Omega_{A/R} \otimes_A P \oplus \bigoplus P\text{d}x_i$.
We are going to use this decomposition.
Consider the following exact sequence
$$
J'/(J')^2 \to
\Omega_{P/R} \otimes_P B' \to
\Omega_{B'/R} \to 0
$$
see
Lemma \ref{lemma-differential-seq}.
We may tensor this with $B$ and obtain the exact sequence
$$
J'/(J')^2 \otimes_{B'} B \to
\Omega_{P/R} \otimes_P B \to
\Omega_{B'/R} \otimes_{B'} B \to 0
$$
If we remember that $J' = (f_i) + J^2$
then we see that the first arrow annihilates the submodule $J^2/(J')^2$.
In terms of the direct sum decomposition
$\Omega_{P/R} \otimes_P B =
\Omega_{A/R} \otimes_A B \oplus \bigoplus B\text{d}x_i $ given
we see that the submodule $(f_i)/(J')^2 \otimes_{B'} B$ maps
isomorphically onto the summand $\bigoplus B\text{d}x_i$. Hence what is
left of this exact sequence is an isomorphism
$\Omega_{A/R} \otimes_A B \to \Omega_{B'/R} \otimes_{B'} B$
as desired.
\end{proof}









\section{Formally \'etale maps}
\label{section-formally-etale}

\begin{definition}
\label{definition-formally-etale}
Let $R \to S$ be a ring map.
We say $S$ is {\it formally \'etale over $R$} if for every
commutative solid diagram
$$
\xymatrix{
S \ar[r] \ar@{-->}[rd] & A/I \\
R \ar[r] \ar[u] & A \ar[u]
}
$$
where $I \subset A$ is an ideal of square zero, there exists
a unique dotted arrow making the diagram commute.
\end{definition}

\noindent
Clearly a ring map is formally \'etale if and only if
it is booth formally smooth and formally unramified.

\begin{lemma}
\label{lemma-formally-etale-etale}
Let $R \to S$ be a ring map of finite presentation.
The following are equivalent:
\begin{enumerate}
\item $R \to S$ is formally \'etale,
\item $R \to S$ is \'etale.
\end{enumerate}
\end{lemma}

\begin{proof}
Assume that $R \to S$ is formally \'etale.
Then $R \to S$ is smooth by Lemma \ref{lemma-formally-smooth-smooth}.
By Lemma \ref{lemma-characterize-formally-unramified}
we have $\Omega_{S/R} = 0$.
Hence $R \to S$ is \'etale by definition.

\medskip\noindent
Assume that $R \to S$ is \'etale.
Then $R \to S$ is formally smooth by
Lemma \ref{lemma-smooth-formally-smooth}.
By Lemma \ref{lemma-characterize-formally-unramified}
it is formally unramified. Hence $R \to S$ is formally \'etale.
\end{proof}

\begin{lemma}
\label{lemma-colimit-formally-etale}
Let $R$ be a ring. Let $I$ be a directed partially ordered set.
Let $(S_i, \varphi_{ii'})$ be a system of $R$-algebras
over $I$. If each $R \to S_i$ is formally \'etale, then
$S = \text{colim}_{i \in I}\ S_i$ is formally \'etale over $R$
\end{lemma}

\begin{proof}
Consider a diagram as in Definition \ref{definition-formally-etale}.
By assumption we get unique $R$-algebra maps $S_i \to A$ lifting
the compositions $S_i \to S \to A/I$. Hence these are compatible
with the transition maps $\varphi_{ii'}$ and define a lift
$S \to A$. This proves existence.
The uniqueness is clear by restricting to each $S_i$.
\end{proof}

\begin{lemma}
\label{lemma-localization-formally-etale}
Let $R$ be a ring. Let $S \subset R$ be any multiplicative subset.
Then the ring map $R \to S^{-1}R$ is formally \'etale.
\end{lemma}

\begin{proof}
Let $I \subset A$ be an ideal of square zero. What we are saying
here is that given a ring map $\varphi : R \to A$ such that
$\varphi(f) \mod I$ is invertible for all $f \in S$ we have also that
$\varphi(f)$ is invertible in $A$ for all $f \in S$. This is true because
$A^*$ is the inverse image of $(A/I)^*$ under the canonical map
$A \to A/I$.
\end{proof}





\section{Unramified ring maps}
\label{section-unramified}

\noindent
The definition of a G-unramified ring map is the one from EGA.
The definition of an unramified ring map is the one from \cite{Henselian}.

\begin{definition}
\label{definition-unramified}
Let $R \to S$ be a ring map.
\begin{enumerate}
\item We say $R \to S$ is {\it unramified} if $R \to S$ is of
finite type and $\Omega_{S/R} = 0$.
\item We say $R \to S$ is {\it G-unramified} if $R \to S$ is of finite
presentation and $\Omega_{S/R} = 0$.
\item Given a prime $\mathfrak q$ of $S$ we say that $S$ is
{\it unramified at $\mathfrak q$} if there exists a
$g \in S$, $g \not \in \mathfrak q$ such that $R \to S_g$ is unramified.
\item Given a prime $\mathfrak q$ of $S$ we say that $S$ is
{\it G-unramified at $\mathfrak q$} if there exists a
$g \in S$, $g \not \in \mathfrak q$ such that $R \to S_g$ is G-unramified.
\end{enumerate}
\end{definition}

\noindent
Of course a G-unramified map is unramified.

\begin{lemma}
\label{lemma-formally-unramified-unramified}
Let $R \to S$ be a ring map. The following are equivalent
\begin{enumerate}
\item $R \to S$ is formally unramified and of finite type, and
\item $R \to S$ is unramified.
\end{enumerate}
Moreover, also the following are equivalent
\begin{enumerate}
\item $R \to S$ is formally unramified and of finite presentation, and
\item $R \to S$ is G-unramified.
\end{enumerate}
\end{lemma}

\begin{proof}
Follows from Lemma \ref{lemma-characterize-formally-unramified}
and the definitions.
\end{proof}

\begin{lemma}
\label{lemma-unramified}
Properties of unramified and G-unramified ring maps.
\begin{enumerate}
\item The base change of an unramified ring map is unramified.
The base change of a G-unramified ring map is G-unramified.
\item The composition of unramified ring maps is unramified.
The composition of G-unramified ring maps is G-unramified.
\item Any principal localization $R \to R_f$ is G-unramified and
unramified.
\item If $I \subset R$ is an ideal, then $R \to R/I$ is unramified.
If $I \subset R$ is a finitely generated ideal, then $R \to R/I$ is
G-unramified.
\item An \'etale ring map is G-unramified and unramified.
\item If $R \to S$ is of finite type (resp.\ finite presentation),
$\mathfrak q \subset S$ is a prime and $(\Omega_{S/R})_{\mathfrak q} = 0$,
then $R \to S$ is unramified (resp.\ G-unramified) at $\mathfrak q$.
\item If $R \to S$ is of finite type (resp.\ finite presentation),
$\mathfrak q \subset S$ is a prime and
$\Omega_{S/R} \otimes_S \kappa(\mathfrak q) = 0$, then
$R \to S$ is unramified (resp.\ G-unramified) at $\mathfrak q$.
\item If $R \to S$ is of finite type (resp.\ finite presentation),
$\mathfrak q \subset S$ is a prime lying over $\mathfrak p \subset R$ and
$(\Omega_{S \otimes_R \kappa(\mathfrak p)/\kappa(\mathfrak p)})_{\mathfrak q}
= 0$, then $R \to S$ is unramified (resp.\ G-unramified) at $\mathfrak q$.
\item If $R \to S$ is of finite type (resp.\ presentation),
$\mathfrak q \subset S$ is a prime lying over $\mathfrak p \subset R$ and
$(\Omega_{S \otimes_R \kappa(\mathfrak p)/\kappa(\mathfrak p)})
\otimes_{S \otimes_R \kappa(\mathfrak p)} \kappa(\mathfrak q) = 0$,
then $R \to S$ is unramified (resp.\ G-unramified) at $\mathfrak q$.
\item If $R \to S$ is a ring map, $g_1, \ldots, g_m \in S$ generate
the unit ideal and $R \to S_{g_j}$ is unramified (resp.\ G-unramified) for
$j = 1, \ldots, m$, then $R \to S$ is unramified (resp.\ G-unramified).
\item If $R \to S$ is a ring map which is unramified (resp.\ G-unramified)
at every prime of $S$, then $R \to S$ is unramified (resp.\ G-unramified).
\item If $R \to S$ is G-unramified, then there exists a finite type
$\mathbf{Z}$-algebra $R_0$ and a G-unramified ring map $R_0 \to S_0$
and a ring map $R_0 \to R$ such that $S = R \otimes_{R_0} S_0$.
\item If $R \to S$ is unramified, then there exists a finite type
$\mathbf{Z}$-algebra $R_0$ and an unramified ring map $R_0 \to S_0$
and a ring map $R_0 \to R$ such that $S$ is a quotient of
$R \otimes_{R_0} S_0$.
\end{enumerate}
\end{lemma}

\begin{proof}
We prove each point, in order.

\medskip\noindent
Ad (1). Follows from Lemmas \ref{lemma-differentials-base-change}
and \ref{lemma-base-change-finiteness}.

\medskip\noindent
Ad (2). Follows from Lemmas \ref{lemma-exact-sequence-differentials}
and \ref{lemma-base-change-finiteness}.

\medskip\noindent
Ad (3). Follows by direct computation of $\Omega_{R_f/R}$ which we omit.

\medskip\noindent
Ad (4). We have $\Omega_{(R/I)/R} = 0$, see
Lemma \ref{lemma-trivial-differential-surjective},
and the ring map $R \to R/I$
is of finite type. If $I$ is a finitely generated ideal then $R \to R/I$
is of finite presentation.

\medskip\noindent
Ad (5). See discussion following Definition \ref{definition-etale}.

\medskip\noindent
Ad (6). In this case $\Omega_{S/R}$ is a finite $S$-module (see
Lemma \ref{lemma-differentials-finitely-generated}) and hence there
exists a $g \in S$, $g \not \in \mathfrak q$ such that
$(\Omega_{S/R})_g = 0$. By Lemma \ref{lemma-differentials-localize}
this means that $\Omega_{S_g/R} = 0$ and hence $R \to S_g$ is
unramified as desired.

\medskip\noindent
Ad (7). Use Nakayama's lemma (Lemma \ref{lemma-NAK}) to see that
the condition is equivalent to the condition of (6).

\medskip\noindent
Ad (8) \& (9). These are equivalent in the same manner that (6) and (7)
are equivalent. Moreover
$\Omega_{S \otimes_R \kappa(\mathfrak p)/\kappa(\mathfrak p)} =
\Omega_{S/R} \otimes_S (S \otimes_R \kappa(\mathfrak p))$ by
Lemma \ref{lemma-differentials-base-change}.
Hence we see that (9) is equivalent to (7) since
the $\kappa(\mathfrak q)$ vector spaces in both are canonically
isomorphic.

\medskip\noindent
Ad (10). Follows from from Lemmas \ref{lemma-cover}
and \ref{lemma-differentials-localize}.

\medskip\noindent
Ad (11). Follows from (6) and (7) and the fact that the spectrum of $S$
is quasi-compact.

\medskip\noindent
Ad (12). Write $S = R[x_1, \ldots, x_n]/(g_1, \ldots, g_m)$.
As $\Omega_{S/R} = 0$ we can write
$$
\text{d}x_i = \sum h_{ij}\text{d}g_j + \sum a_{ijk}g_j\text{d}x_k
$$
in $\Omega_{R[x_1, \ldots, x_n]/R}$
for some $h_{ij}, a_{ijk} \in R[x_1, \ldots, x_n]$.
Choose a finitely generated
$\mathbf{Z}$-subalgebra $R_0 \subset R$ containing all the coefficients of the
polynomials $g_i, h_{ij}, a_{ijk}$. Set
$S_0 = R_0[x_1, \ldots, x_n]/(g_1, \ldots, g_m)$. This works.

\medskip\noindent
Ad (13). Write $S = R[x_1, \ldots, x_n]/I$.
As $\Omega_{S/R} = 0$ we can write
$$
\text{d}x_i = \sum h_{ij}\text{d}g_{ij} + \sum g'_{ik}\text{d}x_k
$$
in $\Omega_{R[x_1, \ldots, x_n]/R}$
for some $h_{ij} \in R[x_1, \ldots, x_n]$ and $g_{ij}, g'_{ik} \in I$.
Choose a finitely generated $\mathbf{Z}$-subalgebra $R_0 \subset R$
containing all the coefficients of the
polynomials $g_{ij}, h_{ij}, g'_{ik}$. Set
$S_0 = R_0[x_1, \ldots, x_n]/(g_{ij}, g'_{ik})$. This works.
\end{proof}

\begin{lemma}
\label{lemma-diagonal-unramified}
Let $R \to S$ be a ring map.
If $R \to S$ is unramified, then there exists an idempotent
$e \in S \otimes_R S$ such that $S \otimes_R S \to S$ is isomorphic
to $S \otimes_R S \to (S \otimes_R S)_e$.
\end{lemma}

\begin{proof}
Let $J = \text{Ker}(S \otimes_R S \to S)$. By assumption
$J/J^2 = 0$, see
Lemma \ref{lemma-differentials-diagonal}.
Since $S$ is of finite type over $R$ we
see that $J$ is finitely generated, namely by
$x_i \otimes 1 - 1 \otimes x_i$, where $x_i$ generate $S$ over $R$.
We win by Lemma \ref{lemma-ideal-is-squared-union-connected}.
\end{proof}

\begin{lemma}
\label{lemma-unramified-at-prime}
Let $R \to S$ be a ring map.
Let $\mathfrak q \subset S$ be
a prime lying over $\mathfrak p$ in $R$.
If $S/R$ is unramified at $\mathfrak q$ then
\begin{enumerate}
\item we have $\mathfrak p S_{\mathfrak q} = \mathfrak qS_{\mathfrak q}$
is the maximal ideal of the local ring $S_{\mathfrak q}$, and
\item the field extension $\kappa(\mathfrak p) \subset \kappa(\mathfrak q)$
is finite separable.
\end{enumerate}
\end{lemma}

\begin{proof}
We may first replace $S$ by $S_g$ for some $g \in S$, $g \not \in \mathfrak q$
and assume that $R \to S$ is unramified.
The base change $S \otimes_R \kappa(\mathfrak p)$
is unramified over $\kappa(\mathfrak p)$ by
Lemma \ref{lemma-unramified}.
By
Lemma \ref{lemma-characterize-smooth-over-field}
it is smooth hence \'etale over $\kappa(\mathfrak p)$.
Hence we see that
$S \otimes_R \kappa(\mathfrak p) =
(R \setminus \mathfrak p)^{-1} S/\mathfrak pS$
is a product of finite separable field extensions of
$\kappa(\mathfrak p)$ by Lemma \ref{lemma-etale-over-field}.
This implies the lemma.
\end{proof}

\begin{lemma}
\label{lemma-unramified-quasi-finite}
Let $R \to S$ be a finite type ring map.
Let $\mathfrak q$ be a prime of $S$.
If $R \to S$ is unramified at $\mathfrak q$ then
$R \to S$ is quasi-finite at $\mathfrak q$.
In particular, an unramified ring map is quasi-finite.
\end{lemma}

\begin{proof}
An unramified ring map is of finite type.
Thus it is clear that the second statement follows from the first.
To see the first statement apply the characterization of
Lemma \ref{lemma-isolated-point-fibre} part (2) using
Lemma \ref{lemma-unramified-at-prime}.
\end{proof}

\begin{lemma}
\label{lemma-characterize-unramified}
Let $R \to S$ be a ring map. Let $\mathfrak q$ be a prime of $S$
lying over a prime $\mathfrak p$ of $R$. If
\begin{enumerate}
\item $R \to S$ is of finite type,
\item $\mathfrak p S_{\mathfrak q}$ is the maximal ideal
of the local ring $S_{\mathfrak q}$, and
\item the field extension $\kappa(\mathfrak p) \subset \kappa(\mathfrak q)$
is finite separable,
\end{enumerate}
then $R \to S$ is unramified at $\mathfrak q$.
\end{lemma}

\begin{proof}
By Lemma \ref{lemma-unramified} (8) it suffices to show that
$\Omega_{S \otimes_R \kappa(\mathfrak p) / \kappa(\mathfrak p)}$
is zero when localized at $\mathfrak q$. Hence we may replace $S$
by $S \otimes_R \kappa(\mathfrak p)$ and $R$ by $\kappa(\mathfrak p)$.
In other words, we may assume that $R = k$ is a field and $S$
is a finite type $k$-algebra.
In this case the hypotheses imply that
$S_{\mathfrak q} \cong \kappa(\mathfrak q)$
and hence $S = \kappa(\mathfrak q) \times S'$ (see
Lemma \ref{lemma-isolated-point}).
Hence $(\Omega_{S/k})_{\mathfrak q} = \Omega_{\kappa(\mathfrak q)/k}$
which is zero as desired.
\end{proof}

\begin{proposition}
\label{proposition-unramified-locally-standard}
Let $R \to S$ be a ring map. Let $\mathfrak q \subset S$ be a prime.
If $R \to S$ is unramified at $\mathfrak q$, then there exist
\begin{enumerate}
\item a $g \in S$, $g \not \in \mathfrak q$,
\item a standard \'etale ring map $R \to S'$, and
\item a surjective $R$-algebra map $S' \to S_g$.
\end{enumerate}
\end{proposition}

\begin{proof}
This proof is the ``same'' as the proof of
Proposition \ref{proposition-etale-locally-standard}.
The proof is a little roundabout and there may be ways to
shorten it.

\medskip\noindent
Step 1. By Definition \ref{definition-unramified}
there exists a $g \in S$, $g \not \in \mathfrak q$
such that $R \to S_g$ is unramified. Thus we may assume that $S$ is
unramified over $R$.

\medskip\noindent
Step 2. By Lemma \ref{lemma-unramified}
there exists an unramified ring map $R_0 \to S_0$
with $R_0$ of finite type over $\mathbf{Z}$, and a ring map
$R_0 \to R$ such that $S$ is a quotient of $R \otimes_{R_0} S_0$. Denote
$\mathfrak q_0$ the prime of $S_0$ corresponding to $\mathfrak q$.
If we show the result for $(R_0 \to S_0, \mathfrak q_0)$ then the
result follows for $(R \to S, \mathfrak q)$ by base change. Hence
we may assume that $R$ is Noetherian.

\medskip\noindent
Step 3.
Note that $R \to S$ is quasi-finite by
Lemma \ref{lemma-unramified-quasi-finite}.
By Lemma \ref{lemma-quasi-finite-open-integral-closure}
there exists a finite ring map $R \to S'$, an $R$-algebra map
$S' \to S$, an element $g' \in S'$ such that
$g' \not \in \mathfrak q$ such that $S' \to S$ induces
an isomorphism $S'_{g'} \cong S_{g'}$.
(Note that $S'$ may not unramified over $R$.)
Thus we may assume that (a) $R$ is Noetherian, (b) $R \to S$ is finite
and (c) $R \to S$ is unramified at $\mathfrak q$
(but no longer necessarily unramified at all primes).

\medskip\noindent
Step 4. Let $\mathfrak p \subset R$ be the prime corresponding
to $\mathfrak q$. Consider the fibre ring
$S \otimes_R \kappa(\mathfrak p)$. This is a finite algebra over
$\kappa(\mathfrak p)$. Hence it is Artinian
(see Lemma \ref{lemma-finite-dimensional-algebra}) and
so a finite product of local rings
$$
S \otimes_R \kappa(\mathfrak p) = \prod\nolimits_{i = 1}^n A_i
$$
see Proposition \ref{proposition-dimension-zero-ring}. One of the factors,
say $A_1$, is the local ring $S_{\mathfrak q}/\mathfrak pS_{\mathfrak q}$
which is isomorphic to $\kappa(\mathfrak q)$,
see Lemma \ref{lemma-unramified-at-prime}. The other factors correspond to
the other primes, say $\mathfrak q_2, \ldots, \mathfrak q_n$ of
$S$ lying over $\mathfrak p$.

\medskip\noindent
Step 5. We may choose a nonzero element $\alpha \in \kappa(\mathfrak q)$ which
generates the finite separable field extension
$\kappa(\mathfrak p) \subset \kappa(\mathfrak q)$ (so even if the
field extension is trivial we do not allow $\alpha = 0$).
Note that for any $\lambda \in \kappa(\mathfrak p)^*$ the
element $\lambda \alpha$ also generates $\kappa(\mathfrak q)$
over $\kappa(\mathfrak p)$. Consider the element
$$
\overline{t} =
(\alpha, 0, \ldots, 0) \in
\prod\nolimits_{i = 1}^n A_i =
S \otimes_R \kappa(\mathfrak p).
$$
After possibly replacing $\alpha$ by $\lambda \alpha$ as above
we may assume that $\overline{t}$ is the image of $t \in S$.
Let $I \subset R[x]$ be the kernel of the $R$-algebra
map $R[x] \to S$ which maps $x$ to $t$. Set $S' = R[x]/I$,
so $S' \subset S$. Here is a diagram
$$
\xymatrix{
R[x] \ar[r] & S' \ar[r] & S \\
R \ar[u] \ar[ru] \ar[rru] & &
}
$$
By construction the primes $\mathfrak q_j$, $j \geq 2$ of $S$ all
lie over the prime $(\mathfrak p, x)$ of $R[x]$, whereas
the prime $\mathfrak q$ lies over a different prime of $R[x]$
because $\alpha \not = 0$.

\medskip\noindent
Step 6. Denote $\mathfrak q' \subset S'$ the prime of $S'$
corresponding to $\mathfrak q$. By the above $\mathfrak q$ is
the only prime of $S$ lying over $\mathfrak q'$. Thus we see that
$S_{\mathfrak q} = S_{\mathfrak q'}$, see
Lemma \ref{lemma-unique-prime-over-localize-below} (we have
going up for $S' \to S$ by Lemma \ref{lemma-integral-going-up}
since $S' \to S$ is finite as $R \to S$ is finite).
It follows that $S'_{\mathfrak q'} \to S_{\mathfrak q}$ is finite
and injective as the localization of the finite injective ring map
$S' \to S$. Consider the maps of local rings
$$
R_{\mathfrak p} \to S'_{\mathfrak q'} \to S_{\mathfrak q}
$$
The second map is finite and injective. We have
$S_{\mathfrak q}/\mathfrak pS_{\mathfrak q} = \kappa(\mathfrak q)$,
see Lemma \ref{lemma-unramified-at-prime}.
Hence a fortiori
$S_{\mathfrak q}/\mathfrak q'S_{\mathfrak q} = \kappa(\mathfrak q)$.
Since
$$
\kappa(\mathfrak p) \subset \kappa(\mathfrak q') \subset \kappa(\mathfrak q)
$$
and since $\alpha$ is in the image of $\kappa(\mathfrak q')$ in
$\kappa(\mathfrak q)$
we conclude that $\kappa(\mathfrak q') = \kappa(\mathfrak q)$.
Hence by Nakayama's Lemma \ref{lemma-NAK} applied to the
$S'_{\mathfrak q'}$-module map $S'_{\mathfrak q'} \to S_{\mathfrak q}$,
the map $S'_{\mathfrak q'} \to S_{\mathfrak q}$ is surjective.
In other words,
$S'_{\mathfrak q'} \cong S_{\mathfrak q}$.

\medskip\noindent
Step 7. By Lemma \ref{lemma-local-isomorphism} there exists a $g' \in S'$,
$g' \not \in \mathfrak q'$ such that $S'_{g'} \cong S_{g'}$.
As $R$ is Noetherian the ring $S'$ is finite over $R$ as it is an $R$-submodule
of the finite $R$-module $S$. Hence after replacing $S$ by $S'$ we may
assume that (a) $R$ is Noetherian, (b) $S$ finite over $R$, (c)
$S$ is unramified over $R$ at $\mathfrak q$, and (d) $S = R[x]/I$.

\medskip\noindent
Step 8. Consider the ring
$S \otimes_R \kappa(\mathfrak p) = \kappa(\mathfrak p)[x]/\overline{I}$
where $\overline{I} = I \cdot \kappa(\mathfrak p)[x]$ is the ideal generated
by $I$ in $\kappa(\mathfrak p)[x]$. As $\kappa(\mathfrak p)[x]$ is a PID
we know that $\overline{I} = (\overline{h})$ for some monic
$\overline{h} \in \kappa(\mathfrak p)$. After replacing $\overline{h}$
by $\lambda \cdot \overline{h}$ for some $\lambda \in \kappa(\mathfrak p)$
we may assume that $\overline{h}$ is the image of some $h \in R[x]$.
(The problem is that we do not know if we may choose $h$ monic.)
Also, as in Step 4 we know that
$S \otimes_R \kappa(\mathfrak p) = A_1 \times \ldots \times A_n$ with
$A_1 = \kappa(\mathfrak q)$ a finite separable extension of
$\kappa(\mathfrak p)$ and $A_2, \ldots, A_n$ local. This implies
that
$$
\overline{h} = \overline{h}_1 \overline{h}_2^{e_2} \ldots \overline{h}_n^{e_n}
$$
for certain pairwise coprime irreducible monic polynomials
$\overline{h}_i \in \kappa(\mathfrak p)[x]$ and certain
$e_2, \ldots, e_n \geq 1$. Here the numbering is chosen so that
$A_i = \kappa(\mathfrak p)[x]/(\overline{h}_i^{e_i})$ as
$\kappa(\mathfrak p)[x]$-algebras. Note that $\overline{h}_1$ is
the minimal polynomial of $\alpha \in \kappa(\mathfrak q)$ and hence
is a separable polynomial (its derivative is prime to itself).

\medskip\noindent
Step 9. Let $m \in I$ be a monic element; such an element exists
because the ring extension $R \to R[x]/I$ is finite hence integral.
Denote $\overline{m}$ the image in $\kappa(\mathfrak p)[x]$.
We may factor
$$
\overline{m} = \overline{k}
\overline{h}_1^{d_1} \overline{h}_2^{d_2} \ldots \overline{h}_n^{d_n}
$$
for some $d_1 \geq 1$, $d_j \geq e_j$, $j = 2, \ldots, n$ and
$\overline{k} \in \kappa(\mathfrak p)[x]$ prime to all the $\overline{h}_i$.
Set $f = m^l + h$ where $l \deg(m) > \deg(h)$, and $l \geq 2$.
Then $f$ is monic as a polynomial over $R$. Also, the image $\overline{f}$
of $f$ in $\kappa(\mathfrak p)[x]$ factors as
$$
\overline{f} =
\overline{h}_1 \overline{h}_2^{e_2} \ldots \overline{h}_n^{e_n}
+
\overline{k}^l \overline{h}_1^{ld_1} \overline{h}_2^{ld_2}
\ldots \overline{h}_n^{ld_n}
=
\overline{h}_1(\overline{h}_2^{e_2} \ldots \overline{h}_n^{e_n}
+
\overline{k}^l
\overline{h}_1^{ld_1 - 1} \overline{h}_2^{ld_2} \ldots \overline{h}_n^{ld_n})
= \overline{h}_1 \overline{w}
$$
with $\overline{w}$ a polynomial relatively prime to $\overline{h}_1$.
Set $g = f'$ (the derivative with respect to $x$).

\medskip\noindent
Step 10. The ring map $R[x] \to S = R[x]/I$ has the properties:
(1) it maps $f$ to zero, and
(2) it maps $g$ to an element of $S \setminus \mathfrak q$.
The first assertion is clear since $f$ is an element of $I$.
For the second assertion we just have to show that $g$ does
not map to zero in
$\kappa(\mathfrak q) = \kappa(\mathfrak p)[x]/(\overline{h}_1)$.
The image of $g$ in $\kappa(\mathfrak p)[x]$ is the derivative
of $\overline{f}$. Thus (2) is clear because
$$
\overline{g} =
\frac{\text{d}\overline{f}}{\text{d}x} =
\overline{w}\frac{\text{d}\overline{h}_1}{\text{d}x} +
\overline{h}_1\frac{\text{d}\overline{w}}{\text{d}x},
$$
$\overline{w}$ is prime to $\overline{h}_1$ and
$\overline{h}_1$ is separable.

\medskip\noindent
Step 11.
We conclude that $\varphi : R[x]/(f) \to S$ is a surjective ring map,
$R[x]_g/(f)$ is \'etale over $R$ (because it is standard \'etale,
see Lemma \ref{lemma-standard-etale}) and $\varphi(g) \not \in \mathfrak q$.
Thus the map $(R[x]/(f))_g \to S_{\varphi(g)}$ is the desired
surjection.
\end{proof}



\begin{lemma}
\label{lemma-etale-makes-unramfied-closed-at-prime}
Let $R \to S$ be a ring map.
Let $\mathfrak q$ be a prime of $S$ lying over $\mathfrak p \subset R$.
Assume that $R \to S$ is of finite type and unramified at $\mathfrak q$.
Then there exist
\begin{enumerate}
\item an \'etale ring map $R \to R'$,
\item a prime $\mathfrak p' \subset R'$ lying over $\mathfrak p$.
\item a product decomposition
$$
R' \otimes_R S = A \times B
$$
\end{enumerate}
with the following properties
\begin{enumerate}
\item $R' \to A$ is surjective, and
\item $\mathfrak p'A$ is a prime of $A$ lying over $\mathfrak p'$ and
over $\mathfrak q$.
\end{enumerate}
\end{lemma}

\begin{proof}
We may replace $(R \to S, \mathfrak p, \mathfrak q)$
with any base change $(R' \to R'\otimes_R S, \mathfrak p', \mathfrak q')$
by a \'etale ring map $R \to R'$ with a prime $\mathfrak p'$
lying over $\mathfrak p$, and a choice of $\mathfrak q'$ lying over
both $\mathfrak q$ and $\mathfrak p'$. Note also that given
$R \to R'$ and $\mathfrak p'$ a suitable $\mathfrak q'$ can always
be found.

\medskip\noindent
The assumption that $R \to S$ is of finite type means that we may apply
Lemma \ref{lemma-etale-makes-quasi-finite-finite-variant}. Thus we may
assume that $S = A_1 \times \ldots \times A_n \times B$, that
each $R \to A_i$ is finite with exactly one prime $\mathfrak r_i$
lying over $\mathfrak p$ such that
$\kappa(\mathfrak p) \subset \kappa(\mathfrak r_i)$ is purely inseparable
and that $R \to B$ is not quasi-finite at any prime lying over $\mathfrak p$.
Then clearly $\mathfrak q = \mathfrak r_i$ for some $i$, since
an unramified morphism is quasi-finite
(see Lemma \ref{lemma-unramified-quasi-finite}).
Say $\mathfrak q = \mathfrak r_1$.
By Lemma \ref{lemma-unramified-at-prime} we see that
$\kappa(\mathfrak p) \subset \kappa(\mathfrak r_1)$
is separable hence the trivial field extension, and that
$\mathfrak p(A_1)_{\mathfrak r_1}$ is the maximal ideal.
Also, by Lemma \ref{lemma-unique-prime-over-localize-below}
(which applies to $R \to A_1$ because a finite ring map satisfies going up by
Lemma \ref{lemma-integral-going-up})
we have $(A_1)_{\mathfrak r_1} = (A_1)_{\mathfrak p}$.
It follows from Nakayama's Lemma \ref{lemma-NAK}
that the map of local rings
$R_{\mathfrak p} \to (A_1)_{\mathfrak p} = (A_1)_{\mathfrak r_1}$
is surjective. Since $A_1$ is finite over $R$ we see that there
exists a $f \in R$, $f \not \in \mathfrak p$ such that
$R_f \to (A_1)_f$ is surjective. After replacing $R$ by $R_f$ we win.
\end{proof}

