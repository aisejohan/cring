\chapter{Dedekind domains}

\section{Dedekind rings}


\begin{example} 
Let $R$ be a noetherian local domain whose prime ideals are $(0)$ and the maximal
ideal 
$\mathfrak{m} \neq 0$. In this condition, I claim:
\begin{proposition} 
TFAE:
\begin{enumerate}
\item $R$ is factorial.
\item $\mathfrak{m}$ is principal.
\item  $R$ is integrally closed.
\item $R$ is a valuation ring with value group $\mathbb{Z}$. 
\end{enumerate}
\end{proposition} 
\begin{definition} 
A ring satisfying these conditions is called a \textbf{discrete valuation
ring} (\textbf{DVR}).
A discrete valuation ring necessarily has only two prime ideals. In fact, a
valuation ring with value group $\mathbb{Z}$ satisfies all the above conditions. 
\end{definition} 
\begin{proof} 
Suppose $R$ is factorial. Then every prime ideal of height one is principal.
But $\mathfrak{m}$ is the only prime that can be height one (it's minimal over
any nonzero nonunit of $R$. Thus 1 implies 2, and similarly 2 implies 1.

1 implies 3 is true for any $R$: factorialness implies integrally closedness.
This is either either homework on the problem set or an easy exercise one can
do for yourself. 

4 implies 2 because one chooses an element $x \in R$ such that the valuation of
$x$ is one. Then, it is easy to see that $x$ generates $\mathfrak{m}$: if $y
\in \mathfrak{m}$, then the valuation of $y$ is at least one, so $y/x \in R$
and $y = (y/x)x \in (x)$. 

The implication 2 implies 4 was essentially done last time. Suppose
$\mathfrak{m}$ is principal, generated by $t$. Last time, we saw that
\emph{all} nonzero ideals of $R$ have the form $(t^n)$ for some $n>0$.  If $x
\in R$, we define the valuation of $x$ to be $n$ if $(x) = (t^n)$. One can
easily check that this is a valuation on $R$ which extends to the quotient
field by additivity.

The interesting part of the argument is the claim that 3
implies 2. Suppose $R$ is integrally closed; I claim that $\mathfrak{m}$ is
principal. Choose $x \in \mathfrak{m} - \left\{0\right\}$. If $(x) =
\mathfrak{m}$, we're done. Otherwise, we can look at $\mathfrak{m}/(x) \neq
0$.  We have a finitely generated module over a noetherian ring which is
nonzero, so it has an associated prime. That associated prime is either zero or
$\mathfrak{m}$. But $0$ is not an associated prime because every element in the
module is killed by $x$. So $\mathfrak{m}$ is an associated prime.

Thus, there is $y \in \mathfrak{m}$ such that $y \notin (x)$ and $\mathfrak{m}y
\subset (x)$. In particular, $y/x \in K(R) - R$ but
\[ (y/x) \mathfrak{m} \subset R.  \]
There are two cases:
\begin{enumerate}
\item Suppose $(y/x) \mathfrak{m}  = R$. Then we can write $\mathfrak{m} =
R(x/y)$. So $\mathfrak{m}$ is principal. (This argument shows that $x/y \in R$.)	
\item The other possibility is that $y/x \mathfrak{m} \subsetneq R$. In this
case, this is an ideal, so 
\[ (y/x) \mathfrak{m} \subset \mathfrak{m}.  \]
In particular, multiplication by $y/x$ carries $\mathfrak{m}$ to itself.  So
multiplication by $y/x$ stabilizes the finitely generated module
$\mathfrak{m}$. By the usual characteristic polynomial argument, we see that
$y/x$ is integral over $R$. In particular, $y/x \in R$, as $R$
was integrally closed, a contradiction as $y \notin (x)$. 
\end{enumerate}
\end{proof} 

We now introduce a closely related notion.  
\begin{definition} 
A \textbf{Dedekind ring} is a noetherian domain $R$ such that 
\begin{enumerate}
\item $R$ is integrally closed.
\item Every nonzero prime ideal of $R$ is maximal. 
\end{enumerate}
\end{definition} 
\end{example} 


\begin{remark} 
If $R$ is Dedekind, then any nonzero element is height one. This is evident
since every nonzero prime is maximal. 

If $R$ is Dedekind, then $R$ is locally factorial. In fact, the localization of
$R$ at a nonzero prime $\mathfrak{p}$ is a DVR.
\begin{proof} 
$R_{\mathfrak{p}}$ has precisely two prime ideals: $(0)$ and
$\mathfrak{p}R_{\mathfrak{p}}$. As a localization of an integrally closed
domain, it is integrally closed. So $R_{\mathfrak{p}}$ is a DVR by the above
result (hence
factorial).
\end{proof} 
\end{remark} 


Assume $R$ is Dedekind now.
We have an exact sequence
\[ 0 \to R^* \to K(R)^* \to \cart(R) \to \pic(R) \to 0.  \]
Here $\cart(R) \simeq \weil(R)$. But $\weil(R)$ is free on the nonzero
primes, or equivalently maximal ideals, $R$ being Dedekind. 
In fact, however, $\cart(R)$ has a simpler description.

\begin{proposition} 
Suppose $R$ is Dedekind. Then $\cart(R)$ consists of all nonzero finitely generated
submodules of $K(R)$ (i.e. \textbf{fractional ideals}). 
\end{proposition} 

This is the same thing as saying as every nonzero finitely generated submodule of $K(R)$ is
invertible.
\begin{proof} 
Suppose $M \subset K(R)$ is nonzero and finitely generated It suffices to check that $M$ is
invertible after localizing at every prime, i.e. that $M_{\mathfrak{p}}$ is
an invertible---or equivalently, trivial, $R_{\mathfrak{p}}$-module. At the
zero prime, there is nothing to check. We might as well assume that
$\mathfrak{p}$ is maximal. Then $R_{\mathfrak{p}}$ is a DVR and
$M_{\mathfrak{p}}$ is a finitely generated submodule of $K(R_{\mathfrak{p}}) = K(R)$. 

Let $S$ be the set of integers $n$ such that there exists $ x \in
M_{\mathfrak{p}}$ with $v(x) = n$, for $v$ the valuation of $R_{\mathfrak{p}}$.
By finite generation of $M$, $S$ is bounded below. Thus $S$ has a least element
$k$. There is an element of $M_{\mathfrak{p}}$, call it $x$, with valuation $k$.

It is easy to check that $M_{\mathfrak{p}}$ is generated by $x$, and is in fact free with
generator $x$. The reason is simply that $x$ has the smallest valuation of
anything in $M_{\mathfrak{p}}$.
\end{proof} 

What's the upshot of this?

\begin{theorem}
If $R$ is a Dedekind ring, then any nonzero ideal $I \subset R$ is invertible,
and therefore uniquely described as a product of powers of (nonzero) prime ideals, $I =
\prod \mathfrak{p}_i^{n_i}$.
\end{theorem} 
\begin{proof} 
This is simply because $I$ is in $\cart(R) = \weil(R)$ by the above result. 
\end{proof} 

This is Dedekind's generalization of unique factorization.

We now give the standard examples:
\begin{example} 
\begin{enumerate}
\item Any PID is Dedekind.  
\item If $K$ is a finite extension of $\mathbb{Q}$, and set $R $ to be the
integral closure of $\mathbb{Z}$ in $K$, then $R$ is a Dedekind ring. The ring
of integers in any number field is a Dedekind ring. 
\item If $R$ is the coordinate ring of an algebraic variety which is smooth and
irreducible of dimension one, then $R$ is Dedekind. 
\item  Let $X$ be a compact Riemann surface, and let $S \subset X$ be a
nonempty finite subset. Then the ring of meromorphic functions on $X$ with
poles only in $S$ is
Dedekind. The maximal ideals in this ring are precisely those corresponding to
points of $X-S$. 
\end{enumerate}
\end{example} 

