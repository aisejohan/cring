\setcounter{chapter}{-1}
\chapter{Categories}
\label{categorychapter}


The language of categories is not strictly necessary to understand the basics
of  commutative
algebra. Nonetheless, it is extremely convenient and powerful. It will clarify
many of the constructions made in the future when we can freely use terms like
``universal property'' or ``adjoint functor.'' As a result, we begin this book
with a brief introduction to their theory. We only scratch the surface; the
interested reader can pursue further study in \cite{Ma98}.


Nonetheless, the reader is advised not to take the present chapter too
seriously; skipping it for the moment to chapter 1 and returning here as a
reference could be quite reasonable.

\section{Introduction}
\newcommand{\ob}{\operatorname{ob}}

\subsection{Definitions}

Categories are supposed to be places where mathematical objects live.
Intuitively, to any given type of structure (e.g. groups, rings, etc.),
there should be a
category of objects with that structure. These are not, of course, the only
type of categories, but they will be the primary ones of concern to us in this
book.


The basic idea of a category is that there should be objects, and that one
should be able to map between objects. These mappings could be functions, and
they often are, but they don't have to be. One has to be able to compose
mappings. Nothing else is required.

\begin{definition}
A \textbf{category} $\mathcal{C}$ consists of:
\begin{enumerate}
\item  A collection of \textbf{objects},
$\ob \mathcal{C}$.
\item For each pair of objects $X, Y \in
\ob \mathcal{C}$, a set
of \textbf{morphisms} $\hom_{\mathcal{C}}(X, Y)$ (abbreviated $\hom(X,Y)$).
\item For each object $X \in \ob\mathcal{C}$, there is an \textbf{identity
morphism}
$1 \in \hom_{\mathcal{C}}(X, X)$.
\item Next, there is a \textbf{composition law}
$\hom_{\mathcal{C}}(X, Y) \times \hom_{\mathcal{C}}(Y, Z) \to
\hom_{\mathcal{C}}(X, Z), (g, f) \to g
\circ f$ for every
triple $X, Y, Z$ of objects.
\item  The composition law is unital (that is, left or right composition with
the identity has no effect) and associative.
\end{enumerate}
\end{definition}

We write $X \to Y$ to denote an element of $\hom_{\mathcal{C}}(X, Y)$.

$\mathcal{C}$ is the storehouse for mathematical objects: groups, Lie algebras,
rings, etc.
\begin{example}
\begin{enumerate}
\item $\mathcal{C}  = \mathbf{Sets}$; the objects are sets, and the morphisms
are maps of sets.
\item $\mathcal{C} = \mathbf{Grps}$; the objects are groups, and the morphisms
are maps of groups (i.e. homomorphisms).
\item $\mathcal{C} = \mathbf{LieAlg}$; the objects are Lie algebras, and the
morphisms are maps of Lie algebras (i.e. homomorphisms).\footnote{Feel free to
omit if the notion of Lie algebra is unfamiliar.}
\item  $\mathcal{C} = \mathbf{Vect}_k$; the objects are vector spaces over a
field $k$, and the morphisms are linear maps.
\end{enumerate}
\end{example}



In general, the objects of a category don not have to form a set; they can
be too large for
that.
For instance, the collection of objects in $\mathbf{Sets}$ does not form a set.

The standard examples of categories are the ones above: structured sets
together with structure-preserving maps. Nonetheless, one can easily give
other examples that are not of this form.

\begin{example}
Let $G$ be a finite group. Then we can make a category $B_G$ where the objects
just consist of one point $\ast$ and the maps $\ast \to \ast$ are the elements
of $G$. The identity is the identity of $G$ and composition is multiplication
in the group.

In this case, the category doesn't represent  a class of objects, but
instead we think of the composition law as the key thing. So a group is a
special kind of category.
\end{example}

\begin{example}
A monoid is precisely a category with one object. Recall that a \textbf{monoid}
has an associative and unital multiplication (not necessarily inverses).
\end{example}

There is, however, a major difference between category theory and set theory.
There is \textbf{nothing} in the language of categories that lets you look
\emph{inside} an object. We think of vector spaces having elements, spaces
having points, etc.
By contrast, categories treat these kinds of things as invisible. There
is nothing ``inside'' of an object $X \in \mathcal{C}$. The only way to
understand $X$ is
to understand the ways one can map into and out of $X$.

\begin{example}
In the category of topological spaces, one can in fact recover the
``underlying set'' of a topological space via the hom-sets. Namely, for each
topological space, the points of $X$ are the same thing as the mappings from a
one-point space into $X$. Later we will say that the functor assigning to each
space its underlying set is \emph{corepresentable.}
\end{example}

\subsection{The language of commutative diagrams}

While the language of categories is, of course, purely algebraic, it will be
convenient for psychological reasons to visualize categorical arguments
through diagrams.
We shall introduce this notation here.

Let $\mathcal{C}$ be a category, and let $X, Y$ be objects in $\mathcal{C}$.
If $f \in \hom(X, Y)$, we shall sometimes write $f$ as an arrow
\[ X \to Y  \]
or
\[ X \stackrel{f}{\to} Y \]
as if $f$ were an actual function.
If $X \stackrel{f}{\to} Y$ and $Y \stackrel{g}{\to} Z$ are morphisms,
composition $g \circ f: X \to Z$ can be visualized by the picture
\[ X \stackrel{f}{\to} Y \stackrel{g}{\to} Z.\]

Finally, when we work with several objects, we shall often draw collections of
morphisms into diagrams, where arrows indicate morphisms between two objects.
A diagram will be said to \textbf{commute} if whenever one goes from one
object in the diagram to another by following the arrows in the right order,
one obtains the same morphism.
For instance, the commutativity of the diagram
\[ \xymatrix{
X \ar[d]^f \ar[r]^{f'} &  W \ar[d]^g \\
Y \ar[r]^{g'} &  Z
}\]
is equivalent to the assertion that
\[ g \circ f' = g' \circ f \in \hom(X, Z).  \]


\section{Functors}
Let $\mathcal{C}, \mathcal{D}$ be categories. If $\mathcal{C}, \mathcal{D}$
are categories of structured sets (of possibly different types), there may be a
way to associate objects in $\mathcal{D}$ to objects in $\mathcal{C}$. For
instance, to every group $G$ we can associate its \emph{group ring}
$\mathbb{Z}[G]$
 (\add{definition or reference}).
In many cases, given a map between objects in $c\mathcal{C}$ preserving the
relevant structure, there will be an induced map on the corresponding objects
in $\mathcal{D}$. It is from here that we define a \emph{functor.}

\begin{definition}
A \textbf{functor} $F: \mathcal{C} \to \mathcal{D}$ consists of a function $F:
\mathcal{C} \to  \mathcal{D}$ (that is, a rule that assigns to each object
in $\mathcal{C}$ an object of $\mathcal{D}$) and, for each pair $X, Y \in
\mathcal{C}$,
a map
$F: \hom_{\mathcal{C}}(X, Y) \to \hom_{\mathcal{D}}(FX, FY)$, which preserves
the identity
maps and composition.
\end{definition}


\begin{example}
There is a functor from $\mathbf{Sets} \to \mathbf{AbelianGrp}$ sending a set
$S$ to a free abelian group on the set.
\end{example}

\begin{example}
There is a functor from $\mathbf{TopSpaces} \to \mathbf{GradedAbGrp}$
(categories of topological spaces and graded abelian groups) sending a
space $X$ to its homology groups $H_*(X)$. We know that given a map of spaces,
we get a map of graded abelian groups.
\end{example}

\begin{exercise}
What is the category of categories? (Ignore set-theoretic issues.)
\end{exercise}


\begin{example}
What is a functor $B_G \stackrel{}{\to} \mathbf{Sets}$? Here $B_G$ is the
category alluded to above.

The unique object $\ast$ goes to some set $X$. For each element $g \in G$, we
get a map $g: \ast \to \ast$ and thus a map $X \to X$. This is supposed to
preserve the composition law (which in $G$ is just multiplication), as well as
identities.

In particular, we get maps $i_G: X \to X$ corresponding to each $g \in G$, such
that the following diagram commutes:
\[ \xymatrix{
X \ar[r]^{i_{g_1}} \ar[rd]_{i_{g_1g_2}} & X \ar[d]^{i_{g_2}} \\ & X
}\]
So a functor $B_G \to \mathbf{Sets}$ is just a left $G$-action on a set $X$.
\end{example}

Sometimes these are called \textbf{covariant functors}. Indeed:

\begin{definition}
A \textbf{contravariant functor} from $\mathcal{C}
\stackrel{F}{\to}\mathcal{D}$ is similar
data except that now a map $X \to Y$ now goes to a map $FY \to FX$. Composites
are required to be preserved, albeit in the other direction.
\end{definition}

As you might guess:

\begin{example}
A \textbf{contravariant} functor from $B_G$ to $\mathbf{Sets}$ corresponds to a
set with a \emph{right} $G$-action.
\end{example}

\begin{example}
On the category $\mathbf{Vect}$ of vector spaces over a field $k$, we
have
the contravariant functor
\[ V \to V^{\ast}.  \]
sending a vector space to its dual $V^{\ast} = \hom(V,k)$.
Given a map $V \to W$ of vector spaces, there is an induced map
\[ W^{\ast} \to V^{\ast}  \]
given by the transpose.
\end{example}

\begin{example}
If we map $B_G \to B_G$ sending $\ast \to \ast$ and $g \to g^{-1}$, we get a
contravariant functor.
\end{example}

\begin{exercise}
Let $\mathcal{C}$ be a category. Define the \textbf{opposite category}
$\mathcal{C}^{op}$ of $\mathcal{C}$ to have the same objects as
$\mathcal{C}$  but such that the morphsims between $X,Y$ in
$\mathcal{C}^{op}$
are those between $Y$ and $X$ in $\mathcal{C}$.
Show that there is a contravariant functor $\mathcal{C} \to
\mathcal{C}^{op}$.
\end{exercise}

There is room, nevertheless, for something else. You could have something that
sent an object to a map.
This is, I think, the reason for Maclane and Eilenberg to describe the next
property.



\section{Natural transformations}

The original paper of Eilenberg and Maclane was called ``On a general theory of
natural transformations.''


Suppose $F, G: \mathcal{C} \to \mathcal{D}$ are functors.

\begin{definition}
A \textbf{natural transformation} $T: F \to G$ consists of the following data.
For each $X \in C$, there is a morphism $TX: FX \to GX$ satisfying the
following
condition. Whenever $f: X \to Y$ is a morphism, the following diagram must
commute:
\[ \xymatrix{
FX \ar[d]^{TX }\ar[r] &  FY \ar[d]^{TY}  \\
GX \ar[r] &  GY
}.\]
\end{definition}

When we say that things are ``natural'' in the future, we will mean that the
transformation between functors is natural in this sense.
Some people don't like this. They don't like to use the language of categories.
If you really try to go in and examine things, it can be hard to figure out
what things really mean.
However, we will use it to state theorems conveniently.


\begin{exercise}
Work this out for yourselves. Suppose you have two functors $B_G \to
\mathbf{Sets}$, i.e. $G$-sets. What's a natural transformation between them?
\end{exercise}

Now we want to prove a theorem.
\begin{theorem}
If $f: X \to Y$ is a map in $\mathcal{C}$, and $F: \mathcal{C} \to \mathcal{D}$
is a functor, then $F(f): FX \to FY$ is an isomorphism.
\end{theorem}
This is going to have a really stupid proof, but there is an important point
lurking here.
\begin{example}
Let $\mathcal{C}$ be the homotopy category of topological spaces
$\mathbf{hoT}$. The objects are topological spaces and the morphisms between
$X, Y$ are the continuous maps $X \to Y$ modulo the relation of being
homotopic. Homology is actually a functor from $\mathbf{hoT}$ to the category
of graded abelian groups.
\end{example}


Hold on. Wait a second. Do we even know what an isomorphism in a category even
is? No, we don't.

\begin{definition}
An \textbf{isomorphism} between objects $X, Y$ in a category $\mathcal{C}$ is a
map $f: X \to Y$ such that there exists $g: Y \to X$ with
\[ g \circ f = 1_X, \quad f \circ g = 1_Y.  \]
\end{definition}

This is more correct than the idea of being one-to-one and onto. A bijection of
topological spaces is not necessarily a homeomorphism.


\begin{proof}
If we have maps $f: X \to Y$ and $g : Y \to X$ such that the composites both
ways are identities, then we can apply the functor $F$ to this, and we find
that since
\[ f \circ g = 1_Y, \quad g \circ f = 1_X,   \]
that
\[ F(f) \circ F(g) = 1_{F(Y)}, \quad F(g) \circ F(f) = 1_{F(X)}.  \]
We have used the fact that functors preserve composition and identities. This
implies that $F(f)$ is an isomorphism.
\end{proof}

Categories have a way of making things so general that they're trivial. Hence,
it is called general abstract nonsense. The things that become meaningful in
category theory are \textbf{not} the proofs. They are the \textbf{definitions}.
What we just did is very much in the spirit I was describing of categories. The
notion of isomorphism was defined in terms of properties of maps, not in terms
of things intrinsic (like injections and surjections).

What's important here is not the theorem, but the \emph{definition of an
isomorphism.}



\section{Various universal constructions}

Last time, we introduced the idea of a category, and showed that a functor
takes isomorphisms to isomorphisms.  This was an amazing result with a trivial
proof.  Today, we will characterize objects in terms of maps.
\subsection{Initial and terminal objects}


\begin{definition}
Let $\mathcal{C}$ be a category. An \textbf{initial object} in a category is an
object $X \in \mathcal{C}$ with the property that $\mathcal{C}(X, Y)$ has one
element for all $Y \in \mathcal{C}$.

So there is a unique map out of $X$ into each $Y \in \mathcal{C}$.
\end{definition}


\begin{example}
If $\mathcal{C}$ is $\mathbf{Sets}$, then the empty set $\emptyset$ is an
initial object. The empty set is the set for indecisive people. To map out of
the indecisive set, you never have to decide where anything goes---it just
goes. There is a unique map from the empty set into any other set.
\end{example}

It seems too abstract to be useful. But it is.

There is a dual notion, called a \textbf{terminal object}, where every object
can map into it in precisely one way.
\begin{definition}
A \textbf{terminal object} in a category $\mathcal{C}$ is an object $Y \in
\mathcal{C}$ such that $\mathcal{C}(X, Y) = \ast$ for each $X \in \mathcal{C}$.
\end{definition}

\begin{example}
The one point set is a terminal object in $\mathbf{Sets}$.
\end{example}

The important thing about the next ``theorems'' is the conceptual framework.
\begin{theorem}
Any two initial (resp. terminal) objects in $\mathcal{C}$ are isomorphic by a
unique isomorphism.
\end{theorem}
\begin{proof}
The proof is really easy. We do it for terminal objects. Say $Y, Y'$ are
terminal objects. Then $\mathcal{C}(Y, Y')$ and $\mathcal{C}(Y', Y)$ are one
point sets. So there are unique maps $Y \to Y', Y' \to Y$, whose composites
must be the identities: we know that $\mathcal{C}(Y, Y) , \mathcal{C}(Y', Y')$
are one-point sets. This means that the maps $Y \to Y', Y' \to Y$ are
isomorphisms.
\end{proof}

There is a philosophical point to be made here. We have characterized an object
uniquely in terms of mapping properties. We have characterized it
\emph{uniquely up to unique isomorphism,} which is really the best you can do
in mathematics. Two sets aren't generally the ``same,'' but they may be
isomorphic up to unique isomorphism. They're different (unless you're Luke),
but the sets are isomorphic up
to unique isomorphism.

Now we're going to talk about a bunch of other examples, which can all be
phrased via initial or terminal objects in some weird category. This,
therefore, is the proof for \emph{everything} we will do in this section.

Say we have a diagram
\[
\xymatrix{
A \ar[d] \ar[r] &  B \ar[d] \\
C \ar[r] &  X}.
\]
We can say what it means for this to be a \textbf{push-out}.

\begin{definition}
A square like this,
\[
\xymatrix{
A \ar[d] \ar[r] &  B \ar[d] \\
C \ar[r] &  X}.
\]
is a \textbf{pushout square} (and $X$ is called the \textbf{push-out}) if,
given a diagram
\[ \xymatrix{
A \ar[r] \ar[d]  &  B \ar[dd] \\
C \ar[rd] & \\
& Y
}\]
there is a unique map $X \to Y$ making the following diagram commute:
\[ \xymatrix{
A \ar[r] \ar[d]  &  B \ar[d] \\
C \ar[rd] \ar[r] & X  \ar[d]  \\
& Y
}.\]
\end{definition}

\begin{example}
The following is a pushout square in the category of abelian groups:
\[ \xymatrix{
\mathbb{Z}/2 \ar[r] \ar[d]  &  \mathbb{Z}/4 \ar[d]  \\
\mathbb{Z}/6 \ar[r] &  \mathbb{Z}/12
}.\]
In the category of groups, the push-out is actually
$\mathrm{SL}_2(\mathbb{Z})$---this is a cool theorem. The point is that being a
push-out is actually dependent on the category.
\end{example}

\begin{proposition}
If the push-out of
\[ \xymatrix{
A \ar[d] \ar[r] & B \\
C
}\]
exists, it is unique up to unique isomorphism.
\end{proposition}
\begin{proof}
We can prove this in two ways. One is that suppose I had two pushout squares
\[
\xymatrix{
A \ar[d] \ar[r] &  B \ar[d] \ar[rdd] \\
C \ar[r] \ar[rrd] &  X \\
& & X'}.
\]
Then there are unique maps $X \to X', X' \to X$ from the universal property,
which have to be isomorphisms.

Alternatively, we can phrase push-outs in terms of initial objects. We could
consider the category of all cartesian diagrams as above with $A,B,C$ and
mapping into something else; then the initial
object in this category is the push-out.
\end{proof}

Now we abstract on this idea further.

\subsection{Colimits}


We now want to generalize the push-out.
Instead of a shape with $A,B,C$, we do something more general.

Start with a \textbf{small} category $I$: this is not meant in a pejorative
sense, but that the objects of $I$ form a set. What you're supposed to picture
is that $I$ is something like the category
\[
\xymatrix{
\ast \ar[d] \ar[r] &  \ast \\
\ast
}
\]
or the category
\[ \ast \rightrightarrows \ast.  \]
We will formulate the notion of a \textbf{colimit} which will specialize to the
push-out when $I$ is the first case. $I$ is to be called  the  \textbf{indexing
category}.


So we will look at functors
\[ F: I \to \mathcal{C},  \]
which in the case of the three-element category, will just
 correspond to
diagrams
\[ \xymatrix{A \ar[d]  \ar[r] &  B \\ C}.  \]

We will call a \textbf{cone} on $F$ (this is an ambiguous term) an object $X
\in \mathcal{C}$ equip'd\footnote{Imagine I wrote that with an English accent.}
with maps $F_i \to X, \forall i \in I$ such that for all maps $i \to
i' \in I$, the diagram below commutes:
\[ \xymatrix{
F_i \ar[d] \ar[r] &  X \\
F_{i'} \ar[ru]
}.\]

An example would be a cone on the three-element category above: then
this is just a commutative diagram
\[ \xymatrix{
A \ar[r]\ar[d]  &  B \ar[d]  \\
C \ar[r] &  D
}.\]

\newcommand{\colim}{\mathrm{colim}}

\begin{definition}
The \textbf{colimit} of the diagram $F: I \to \mathcal{C}$, written as $\colim
F$ or $\colim_I F $ or $\varinjlim_I F$, if it exists, is a cone $F \to X$ with
the property that if $F \to Y$ is any other cone, then there is a unique map $X
\to Y$ making the diagram
\[ \xymatrix{
F  \ar[rd] \ar[r] &  X \ar[d]  \\
& Y
}\]
commute. (This means that the corresponding diagram with $F_i$ replacing $F$
commutes for each $i \in I$.)
\end{definition}

We could think of some weird category where cones are objects and the colimit
is initial. In any case, we see:

\begin{proposition}
$\colim F$, if it exists, is unique up to unique isomorphism.
\end{proposition}

Let us go through some examples. We already looked at push-outs.

\begin{example}
Consider the category $I$ described by
\[ \ast, \ast, \ast, \ast.  \]
A functor $F: I \to \mathbf{Sets}$ is just a list of four sets $A, B, C, D$.
The colimit is just the disjoint union $A \sqcup B \sqcup C \sqcup D$. This is
the universal property of the disjoint union. To hom out of the disjoint union
is the same thing as homming out of each piece.
\end{example}


\begin{example}
Suppose we had the same category $I$ but we went into abelian groups. Then $F$
corresponds, again, to a list of four abelian groups. The colimit is the direct
sum. Again, the direct sum is characterized by the same universal property.
\end{example}

\begin{example}
Suppose we had the same $I$ ($\ast, \ast, \ast, \ast$) but the category of
groups was $\mathcal{C}$. Then the colimit is the
free product of the four groups.
\end{example}

\begin{example}
Suppose we had the same $I$ and the category $\mathcal{C}$ was of commutative
rings with unit. Then the colimit is the tensor product.
\end{example}

So the idea unifies a whole bunch of constructions.
Now let us take a different example.

\begin{example}
Take
\[ I = \ast \rightrightarrows \ast.  \]
So a functor $I \to \mathbf{Sets}$ is a diagram
\[ A \rightrightarrows B.  \]
Call the two maps $f,g: A \to B$. To get the colimit, we take $B$ and mod out
by the equivalence relation generated by $f(a) \sim g(a)$.
To hom out of this is the same thing as homming  out of $B$ such that the
pullbacks to $A$ are the same.

This is the relation \textbf{generated} as above, not just as above. It can get
tricky.
\end{example}

\begin{definition}
When $I$ is just a bunch of points  $\ast, \ast, \ast, \dots$ with no
nonidentity morphisms, then the
colimit over $I$ is called the \textbf{coproduct}.
\end{definition}

We use the coproduct to mean things like direct sums, disjoint unions, and
tensor products.

\begin{definition}
When $I$ is $\ast \rightrightarrows \ast$, the colimit is called the
\textbf{coequalizer}.
\end{definition}

\begin{theorem}
If $\mathcal{C}$ has all coproducts and coequalizers, then it has all colimits.
\end{theorem}

\begin{proof}
Exercise. \add{proof} \end{proof}

\subsection{Filtered colimits}


\emph{Filtered colimits} are colimits
over special indexing categories $I$ which look like totally ordered sets.
These have several convenient properties as compared to general colimits.
For instance, in the category of \emph{modules} over a ring (to be studied in
\rref{foundations}), we shall see that filtered colimits actually
preserve injections and surjections. In fact, they are \emph{exact.} This is
not true in more general categories which are similarly structured.



\begin{definition}
An indexing category is \textbf{filtered} if the following hold:
\begin{enumerate}
\item Given $i_0, i_1 \in I$, there is a third object $i \in I$ such that both
$i_0, i_1$ map into $i$.
\item Given any two maps $i_0 \rightrightarrows i_1$, there exists $i$ and $i_1
\to i$ such that the two maps $i_0 \rightrightarrows i$ are equal. Any two ways
of pushing an object into another can be made into the same eventually.
\end{enumerate}
\end{definition}

\begin{example}
If $I$ is the category
\[ \ast \to \ast \to \ast \to \dots,  \]
i.e. the category generated by the poset $\mathbb{Z}_{\geq 0}$, then that is
filtered.
\end{example}


\begin{example}
If $G$ is a torsion-free abelian group, the category $I$ of finitely generated
subgroups of $G$ and inclusion maps is filtered. We don't actually need the
lack of torsion.
\end{example}

\begin{definition}
Colimts over a filtered category are called \textbf{filtered colimits}.
\end{definition}

\begin{example}
Any torsion-free abelian group is the filtered colimit of its finitely
generated subgroups, which are free abelian groups.
\end{example}
This gives a simple approach for showing that a torsion-free abelian group is
flat.

\begin{proposition}
If $I$ is filtered\footnote{Some people say filtering.} and $\mathcal{C} =
\mathbf{Sets}, \mathbf{Abgrp}, \mathbf{Grps}$, etc., and $F: I \to \mathcal{C}$
is a functor, then $\colim_I F$ exists and is given by the disjoint union of
$F_i, i \in I$ modulo the relation $x \in F_i$ is equivalent to $x' \in F_{i'}$
if $x$ maps to $x'$ under $F_i \to F_{i'}$. This is already an equivalence
relation.
\end{proposition}

The fact that the relation given above is transitive uses the filtering of the
indexing set. Otherwise, we would need to use the relation generated by it.

\begin{example}
Take $\mathbb{Q}$. This is the filtered colimit of the free submodules
$\mathbb{Z}(1/n)$.

Alternatively, choose a sequence of numbers $m_1 , m_2, \dots, $ such that for
all $p, n$, we have $p^n \mid m_i$ for $i \gg 0$. Then we have a sequence of
maps
\[ \mathbb{Z} \stackrel{m_1}{\to} \mathbb{Z} \stackrel{m_2}{\to}\mathbb{Z}
\to \dots.   \]
The colimit of this is $\mathbb{Q}$. There is a quick way of seeing this, which
is left to the reader.
\end{example}

When we have a functor $F: I \to \mathbf{Sets}, \mathbf{Grps},
\mathbf{Modules}$ taking values in a ``nice'' category (e.g. the category of
sets, modules, etc.), you can construct the colimit by taking the union of the
$F_i, i \in I$ and quotienting by the equivalence relation $x \in F_i \sim x'
\in F_{i'}$ if $f: i \to i'$ sends $x$ into $x'$. This is already an
equivalence relation, as one can check.

Another way of saying this is that we have the disjoint union of the $F_i$
modulo the relation that $a \in F_i$ and $b \in F_{i'}$ are equivalent if and
only if there is a later $i''$ with maps $i \to i'', i' \to i''$ such that
$a,b$ both map to the same thing in $F_{i''}$.


