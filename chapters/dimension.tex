\chapter{Dimension theory}

\label{chdimension}
\textbf{Dimension theory} assigns to each commutative ring---say,
noetherian---an invariant called the dimension. The most standard definition,
that of Krull dimension (which we shall not adopt at first), defines the
dimension in terms of the maximal lengths of ascending chains of prime ideals. 
In general, however, the geometric intuition behind dimension is that it
should assign to an affine ring---say, one of the form $\mathbb{C}[x_1, \dots,
X_n]/I$---something like the ``topological dimension'' of the affine variety
in $\mathbb{C}^n$ cut out by the ideal $I$. 

In this chapter, we shall obtain three different expressions for the dimension
of a noetherian local ring $(R, \mathfrak{m})$, each of which will be useful
at different times in proving results. 

\section{The Hilbert function and the dimension of a local ring}
\subsection{Integer-valued polynomials}

It is now necessary to do a small amount of general algebra.

Let $P \in \mathbb{Q}[t]$. We consider the question of when $P$ maps the
integers $\mathbb{Z}$, or more generally the sufficiently large integers, into
$\mathbb{Z}$. Of course, any polynomial in $\mathbb{Z}[t]$ will do this, but
there are others: consider $\frac{1}{2}(t^2 -t)$, for instance.

\begin{proposition}\label{integervalued} 
Let $P \in \mathbb{Q}[t]$. Then $P(m)$ is an integer for $m \gg 0$ integral if and only if
$P$ can be written in the form
\[ P(t) = \sum_n c_n \binom{t}{n}, \quad c_n \in \mathbb{Z}.  \]
In particular, $P(\mathbb{Z}) \subset \mathbb{Z}$.
\end{proposition} 
So $P$ is a $\mathbb{Z}$-linear function of binomial coefficients. 
\begin{proof} 
Note that the set $\left\{\binom{t}{n}\right\}_{n \in \mathbb{Z}_{\geq 0}}$ forms a basis for the set of
polynomials $\mathbb{Q}[t]$. It is thus clear that $P(t)$ can be written as
a rational combination $\sum c_n \binom{t}{n}$ for the $c_n \in \mathbb{Q}$.
We need to argue that the $c_n \in \mathbb{Z}$ in fact.

Consider the operator $\Delta$ defined on functions $\mathbb{Z} \to
\mathbb{C}$ as follows:
\[( \Delta f)(m) = f(m) - f(m-1).  \]
It is obvious that if $f$ takes integer values for $m \gg 0$, then so does
$\Delta f$. It is also easy to check that $\Delta \binom{t}{n} =
\binom{t}{n-1}$.

By looking at the
function $\Delta P = \sum c_n \binom{t}{n-1}$ (which takes values in $\mathbb{Z}$), it is easy to see that the $c_n \in \mathbb{Z}$ by induction
on the degree. 
It is also easy to see directly that the binomial coefficients take values in
$\mathbb{Z}$ at \emph{all} arguments.
\end{proof} 



\subsection{Definition and examples}
Let $R$ be a   ring. 
\begin{question} 
What is a good definition for $\dim(R)$? Actually, more generally, 
what is the dimension of $R$ at a given ``point'' (i.e. prime ideal)?
\end{question} 

Geometrically, think of $\spec R$, for any ring; pick some point corresponding to a maximal
ideal $\mathfrak{m} \subset R$. We want to define the \textbf{dimension of $R$}
at $\mathfrak{m}$. This is to be thought of kind of like ``dimension over the
complex numbers,'' for algebraic varieties defined over $\mathbb{C}$. But it
should be purely algebraic. 
What might you do?


Here  is an idea. For a topological space $X$ to be $n$-dimensional at $x \in
X$, there should be $n$ coordinates at the point $x$. In other words, the
point $x$ should be uniquely  defined
by the zero locus of $n$ points on the space.
Motivated by this, we could try defining $\dim_{\mathfrak{m}} R$
to be the number of generators of $\mathfrak{m}$.
However, this is a bad definition, as $\mathfrak{m}$ may not have the same number of
generators as $\mathfrak{m}R_{\mathfrak{m}}$. In other words, it is not a
truly \emph{local} definition.
\begin{example} 
Let $R$ be a noetherian integrally closed domain which is not a UFD. Let $\mathfrak{p}
\subset R$ be a prime ideal which is minimal over a principal ideal but which
is not itself principal. Then $\mathfrak{p}R_{\mathfrak{p}}$ is generated by
one element, as we will eventually see, but $\mathfrak{p}$ is not.
\end{example} 

We want our definition of dimension to be
local.
So this leads us to:
\begin{definition} 
If $R$ is a (noetherian) \emph{local} ring with maximal ideal $\mathfrak{m}$,
then the \textbf{embedding dimension} of $R$, denoted $\emdim R$ is
the minimal number of generators for $\mathfrak{m}$. If $R$ is a  noetherian
ring and $\mathfrak{p} \subset R$ a prime ideal, then the \textbf{embedding
dimension at $\mathfrak{p}$} is that of the local ring $R_{\mathfrak{p}}$.
\end{definition} 

In the above definition, it is clearly sufficient to study what happens for
local rings, and we impose that restriction for now. By Nakayama's lemma, the
embedding dimension is the minimal number of generators of
$\mathfrak{m}/\mathfrak{m}^2$, or the $R/\mathfrak{m}$-dimension of that vector
space:
\[ \emdim R = \dim_{R/\mathfrak{m}} \mathfrak{m}/ \mathfrak{m}^2.  \]


In general, however, the embedding dimension is not going to coincide with the
intuitive ``geometric'' dimension of an algebraic
variety.

\begin{example} 
Let $R = \mathbb{C}[t^2, t^3] \subset \mathbb{C}[t]$, which is the coordinate
ring of a cubic curve $y^2 =x^3$ as $R \simeq \mathbb{C}[x,y]/(x^2 - y^3)$
via $x = t^3, y = t^2$. Let us localize at the prime ideal $\mathfrak{p} = (t^2,
t^3)$: we get $R_{\mathfrak{p}}$. 

Now $\spec R$ is singular at the origin. In fact, as a result, $\mathfrak{p}
R_{\mathfrak{p}} \subset R_{\mathfrak{p}}$ needs two generators, but the
variety it corresponds to is one-dimensional.
\end{example} 

So the embedding dimension is the smallest dimension into which you can embed
$R$ into a smooth space.
But for singular varieties this is not the dimension we want. 

So instead of considering simply $\mathfrak{m}/\mathfrak{m}^2$, let us
consider the \emph{sequence} of finite-dimensional vector spaces
\[ \mathfrak{m}^k/\mathfrak{m}^{k+1}.  \]
Computing these dimensions as a function of $k$ gives some invariant that describes the local
geometry of $\spec R$.

We shall eventually prove:
\begin{theorem} \label{hilbfnispolynomial}
Let $(R, \mathfrak{m})$ be a local noetherian ring. Then there exists a
polynomial
$f \in \mathbb{Q}[t]$ such that
\[ f(n) =  \ell(R/\mathfrak{m}^n) = \sum_{i=0}^{n-1} \dim
\mathfrak{m}^i/\mathfrak{m}^{i+1} \quad \forall n \gg 0.  \]

Moreover, $\deg f \leq \dim \mathfrak{m}/\mathfrak{m}^2$.
\end{theorem}


Note that this polynomial is well-defined, as any two polynomials agreeing for large $n$
coincide. Note also that $R/\mathfrak{m}^n$ is artinian so of finite length,
and that we have used the fact that the length is additive for short exact
sequences. We would have liked to write $\dim R/\mathfrak{m}^n$, but we can't,
in general, so we use the substitute of the length. 

Based on this, we define:
\begin{definition} 
The \textbf{dimension} of the local ring $R$ is the degree of the polynomial
$f$ above. For an arbitrary noetherian ring $R$, we define $\dim R =
\sup_{\mathfrak{p} \in \spec R} \dim (R_{\mathfrak{p}})$.
\end{definition} 


Let us now do a few example computations. 

\begin{example}[The affine line] \label{easydimcomputation}
\label{dimaffineline}
Consider the local ring $(R, \mathfrak{m}) = \mathbb{C}[t]_{(t)}$. Then $\mathfrak{m} = (t)$ and
$\mathfrak{m}^k/\mathfrak{m}^{k+1}$ is one-dimensional, generated by $t^k$. In
particular, the ring has dimension one. 
\end{example} 

\begin{example}[A singular curve] Consider $R = \mathbb{C}[t^2, t^3]_{(t^2, t^3)}$, the local ring of $y^2 = x^3$
at zero. Then $\mathfrak{m}^n$ is generated by $t^{2n}, t^{2n+1}, \dots$.
$\mathfrak{m}^{n+1}$ is generated by $t^{2n+2}, t^{2n+3}, \dots$. So the
quotients all have dimension two. The dimension of these quotients is a little
larger than in \rref{easydimcomputation}, but they do not grow. The ring still has dimension one. 
\end{example} 

\begin{example}[The affine plane] \label{dimaffineplane}
Consider $R = \mathbb{C}[x,y]_{(x,y)}$. Then $\mathfrak{m}^k$ is generated by
polynomials in $x,y$ that are homogeneous in degree $k$. So $\mathfrak{m}^k/\mathfrak{m}^{k+1}$
has dimensions that \emph{grow} linearly in $k$. This is a genuinely two-dimensional
example. 
\end{example} 

It is this difference between constant linear and quadratic growth in
$R/\mathfrak{m}^n$ as $n \to \infty$, and not the size of the initial terms,
that we want for our definition of dimension.

Let us now generalize \rref{dimaffineline} and \rref{dimaffineplane}
above to affine spaces of arbitrary dimension. 
\begin{example}[Affine space] 
Consider $R = \mathbb{C}[x_1, \dots, x_n]_{(x_1, \dots, x_n)}$. 
This represents the variety $\mathbb{C}^n = \mathbb{A}^n_{\mathbb{C}}$ near the origin geometrically, so
it should intuitively have dimension $n$. Let us check that it does. 

Namely, we need to compute the polynomial $f$ above. Here $R/\mathfrak{m}^k$ looks like the set of
polynomials of degree $<k$ over $\mathbb{C}$. The dimension as a vector space
of this is
given by some binomial coefficient $\binom{n+k-1}{n}$. This is a polynomial in
$k$ of degree $n$. In particular, $\ell(R/\mathfrak{m}^k)$ grows like $k^n$.
So $R$ is $n$-dimensional. 
\end{example} 


Finally, we offer one more example, showing that DVRs have dimension one. In
fact, among noetherian integrally closed local domains, DVRs are
\emph{characterized} by this property (\rref{} of \rref{}).

\begin{example}[The dimension of a DVR]
Let $R$ be a DVR. Then $\mathfrak{m}^k/\mathfrak{m}^{k+1}$ is of length one for
each $k$. So $R/\mathfrak{m}^k$ has length $k$. Thus we can take $f(t) = t$, so
$R$ has dimension one.
\end{example} 

\subsection{The Hilbert function is a polynomial} 

While we have given a definition of dimension and computed various examples,
we have yet to check that our definition is well-defined. 
Namely, we have to prove \rref{hilbfnispolynomial}.

\begin{proof}[Proof of  \rref{hilbfnispolynomial}]
Fix a noetherian local ring $(R, \mathfrak{m})$. We are to show that
$\ell(R/\mathfrak{m}^n)$ is a polynomial for $n \gg 0$. We also have to bound
this degree by $\dim_{R/\mathfrak{m}} \mathfrak{m}/\mathfrak{m}^2$, the
embedding dimension. We will do this by reducing to a general fact about
graded modules over a polynomial ring.

Let $S = \bigoplus_n  \mathfrak{m}^n/\mathfrak{m}^{n+1}$. Then $S$ has a
natural grading, and in fact it is a graded ring in a natural way from the
multiplication map 
\[ \mathfrak{m}^{n_1} \times \mathfrak{m}^{n_2} \to \mathfrak{m}^{n_1 + n_2}.  \]
In fact, $S$ is the \emph{associated graded ring} of the $\mathfrak{m}$-adic filtration.
Note that $S_0 = R/\mathfrak{m}$ is a field, which we will denote by  $k$.
So $S$ is a graded $k$-algebra.

\begin{lemma} 
$S$ is a finitely generated $k$-algebra. In fact, $S$ can be generated by at
most $\emdim(R)$ elements.
\end{lemma} 
\begin{proof} 
Let $x_1, \dots, x_r$ be generators for $\mathfrak{m} $ with $r = \emdim(R)$.
They  (or rather, their images) are thus a $k$-basis
for $\mathfrak{m}/\mathfrak{m}^2$.
Then their images in $\mathfrak{m}/\mathfrak{m}^2 \subset S$ generate $S$.
This follows because $S_1$ generates $S$ as an $S_0$-algebra: the products of
the elements in $\mathfrak{m}$ generate the higher powers of $\mathfrak{m}$.
\end{proof} 

So $S$ is a graded quotient of the polynomial ring $k[t_1, \dots, t_r]$, with
$t_i $ mapping to $x_i$. In particular, $S$ is a finitely generated, graded $k[t_1,
\dots, t_r]$-module.
Note that also $\ell(R/\mathfrak{m}^n)  = \dim_{k}(S_0) + \dots +
\dim_{k}(S_{n-1})$ for any $n$, thanks to the filtration. This is the
invariant we are interested in.

It will now suffice to prove the following more general proposition.
\begin{proposition} \label{hilbfngeneral}
Let $M$ be any finitely generated graded module over the polynomial ring
$k[x_1, \dots, x_r]$. Then
there exists a polynomial $f_M^+ \in \mathbb{Q}[t]$ of degree $ \leq r$, such that
\[ f_M^+(t) = \sum_{s \leq t} \dim M_s \quad t \gg 0.   \]
\end{proposition} 
Applying this to $M = S$ will give the desired result. We can forget about
everything else, and look at this problem over graded polynomial rings.

This function is called the \textbf{Hilbert function}.

\begin{proof}[Proof of \rref{hilbfngeneral}]
Note that if we have an exact sequence of graded modules over the polynomial
ring,
\[ 0 \to M' \to M \to M'' \to 0,   \]
and polynomials $f_{M'}, f_{M''}$ as in the proposition, then $f_M$ exists and 
\[ f_M = f_{M'} + f_{M''}.  \] This is obvious from the definitions.
Next, we observe that if $M$ is a finitely generated graded module, over two
different polynomial rings, but with the same grading, then the existence (and
value) of $f_M$ is independent of which polynomial ring one considers.
Finally, we observe that it is sufficient to prove that $f_M(t) = \dim M_t$ is a
polynomial in $t$ for $t \gg 0$.

We will
use these three observations and induct on $n$.

If $n  = 0$, then $M$ is a finite-dimensional graded vector space over
$k$, and the grading must be concentrated in finitely many degrees. Thus
the result is evident as $f_{M}(t)$ will just equal $\dim M$ (which will be the
appropriate dimension for $t \gg 0$). 

Suppose $n > 0$.  Then
consider the filtration of $M$
\[ 0 \subset \ker( x_1: M \to M) \subset \ker (x_1^2: M \to M) \subset \dots
\subset M.  \]
This must stabilize by noetherianness at some $M' \subset M$. Each of the
quotients $\ker( x_1^i)/\ker (x_1^{i+1})$ is a finitely generated module over
$k[x_1, \dots,
x_n]/(x_1)$, which is a smaller polynomial ring.  So each of these quotients
$\ker (x_1^{i+1})/\ker (x_1^{i})$ has a Hilbert function of degree $\leq n-1$ by
the inductive hypothesis. 

Climbing up the filtration, we see that  $M'$ has a Hilbert function which is the sum of the Hilbert functions of
these quotients $\ker(x_1^{i+1})/\ker(x_1^{i})$. In particular, $f_{M'}$ exists. If we show that $f_{M/M'}$
exists, then $f_M$ necessarily exists. So we might as well show that the
Hilbert function $f_M$ exists when $x_1$ is  a non-zerodivisor on $M$. 

So, we have reduced to the case where $M \stackrel{x_1}{\to} M$
is injective.
Now $M$ has a filtration
\[ M \supset x_1 M \supset x_1^2 M \supset \dots  \]
which is an exhaustive filtration of $M$ in that nothing can be divisible by
powers of $x_1$ over and over, or the degree would not be finite. So it
follows that  $\bigcap x_1^m M = 0$. 

Let $N = M/x_1 M $, which is isomorphic to $ x_1^m M/x_1^{m+1} M$ since $M \stackrel{x_1}{\to} M$ is
injective. Here $N$ is a finitely generated graded module over $k[x_2, \dots, x_n]$, and by the
inductive hypothesis on $n$, we see that there is a polynomial $f_N^+$ of degree $\leq n-1$ such that
\[ f_N^+(t) = \sum_{t' \leq t} \dim N_{t'}, \quad t \gg 0.  \]

Fix $t \gg 0$ and consider the $k$-vector space $M_t$, which has a finite filtration 
\[ M_t \supset (x_1 M)_t \supset (x_1^2 M)_t \supset \dots  \]
which has successive quotients that are the  graded pieces of $N \simeq
M/x_1 M \simeq x_1 M/x_1^2 M \simeq \dots$ in dimensions $t, t-1, \dots$. We
find that
\[ (x_1^2 M)_t/(x_1^3 M)_t \simeq N_{t-2},  \]
for instance. Summing this, we find that
\[ \dim M_t = \dim N_t + \dim N_{t-1} + \dots . \]
The sum above is actually finite. In fact, by finite generation, there is $K
\gg 0 $ such that $\dim N_q  = 0$ for $q< -K$. From this, we find that 
\[ \dim M_t = \sum_{t' = -K}^{t} \dim N_{t'},  \]
which implies that $\dim M_t$ is a polynomial for $t \gg 0$. This completes
the proof.
\end{proof} 
\end{proof}


Let $(R, \mathfrak{m})$ a noetherian local ring and $M$ a finitely generated
$R$-module. 
\begin{proposition} \label{hilbertlocalring} 
$\ell(M/\mathfrak{m}^m M)$ is a polynomial for $m \gg 0$. 
\end{proposition} 
\begin{proof} 
This follows from \rref{hilbfngeneral}, and in fact we have
essentially seen the argument above. Indeed, we consider the
associated graded module
\[ N = \bigoplus \mathfrak{m}^k M/\mathfrak{m}^{k+1}M , \]
which is finitely generated over the associated graded ring
\[ \bigoplus \mathfrak{m}^k/\mathfrak{m}^{k+1}.  \]
Consequently, the graded pieces of $N$ have dimensions growing polynomially
for large degrees. This implies the result. 

\end{proof} 
\begin{definition} 
We define the \textbf{Hilbert function} $H_M(m)$  to be the unique polynomial
such that 
\[ H_M(m) = \ell(M/ \mathfrak{m}^m M), \quad m \gg 0.  \]
\end{definition} 

It is clear, incidentally, that $H_M$ is integer-valued, so we see by
\rref{integervalued} that $H_M$ is a $\mathbb{Z}$-linear
combination of binomial coefficients.

\subsection{The dimension of a module}
Let $R $ be a local noetherian ring with maximal ideal $\mathfrak{m}$. We
have seen (\rref{hilbertlocalring}) that  there is a polynomial $H(t)$ with
\[ H(t) = \ell(R/\mathfrak{m}^t), \quad t \gg 0.  \]
Earlier, we defined the \textbf{dimension} of $R$ is the degree of $f_M^+$. 
Since the degree of the Hilbert function is at most the number of generators
of the polynomial ring, we saw that 
\[ \dim R \leq \emdim R.  \]

Armed with the machinery of the Hilbert function, we can extend this
definition to modules.
\begin{definition} 
If $R$ is local noetherian, and $N$ a finite $R$-module, then $N$ has a
Hilbert polynomial $H_N(t)$ which when evaluated at $t \gg 0$ gives
the length $\ell(N/\mathfrak{m}^t N)$.
We say that the \textbf{dimension of
$N$} is the degree of this Hilbert polynomial.
\end{definition} 

Clearly, the dimension of the \emph{ring} $R$ is the same thing as that of the
\emph{module} $R$.

We next show that the dimension behaves well with respect to short exact
sequences. This is actually slightly subtle since, in general, tensoring with
$R/\mathfrak{m}^t$ is not exact; it turns out to be \emph{close} to being
exact by the Artin-Rees lemma. 
On the other hand, the corresponding fact for modules over a \emph{polynomial
ring} is very easy, as 
no tensoring was involved in the definition.

\begin{proposition} 
Suppose we have an exact sequence
\[ 0 \to M' \to M \to M'' \to 0 \]
of graded modules over a polynomial ring $k[x_1, \dots, x_n]$.	Then
\[ f_M(t) = f_{M'}(t) + f_{M''}(t), \quad  f_M^+(t) = f_{M'}^+(t) +
f_{M''}^+(t). \]
As a result, $\deg f_M = \max \deg f_{M'}, \deg f_{M''}$.
\end{proposition} 
\begin{proof} The first part is obvious as the dimension is additive on vector
spaces. The second part follows because Hilbert functions have nonnegative
leading coefficients.
\end{proof} 

\begin{proposition} \label{dimexactseq}
Fix an exact sequence
\[ 0 \to N' \to N \to N'' \to 0  \]
of finite $R$-modules. Then $\dim N = \max (\dim N', \dim N'')$.
\end{proposition} 
\begin{proof} 
We have an exact sequence
\[ 0 \to K \to N/\mathfrak{m}^t N \to N''/\mathfrak{m}^t N'' \to 0  \]
where $K$ is the kernel. Here $K = (N' + \mathfrak{m}^t N)/ \mathfrak{m}^t N
= N'/( N' \cap \mathfrak{m}^t N)$. This is not quite $N'/\mathfrak{m}^t N'$,
but it's pretty close. 
We have a surjection
\[ N'/\mathfrak{m}^t N \twoheadrightarrow N'/(N' \cap \mathfrak{m}^t N) = K. \]
In particular, 
\[ \ell(K) \leq \ell(N'/\mathfrak{m}^t N').  \]
On the other hand, we have the Artin-Rees lemma, which gives an inequality in
the opposite direction. We have a containment
\[ \mathfrak{m}^t N' \subset N' \cap \mathfrak{m}^t N \subset
\mathfrak{m}^{t-c} N'  \]
for some $c$. This implies that $\ell(K) \geq \ell( N'/\mathfrak{m}^{t-c} N')$. 

Define $M = \bigoplus \mathfrak{m}^t N/\mathfrak{m}^{t+1} N$, and define $M',
M''$ similarly in terms of $N', N''$. Then we have seen that 
\[ \boxed{f_M^+(t-c) \leq \ell(K) \leq f_M^+(t).}  \]
We also know that the length of $K$ plus the length of $N''/\mathfrak{m}^t N''$
is $f_M^+(t)$, i.e.
\[ \ell(K) + f_{M''}^+(t) = f_M^+(t).  \]
Now the length of $K$ is a polynomial in $t$ which is pretty similar to
$f_{M'}^+$, in that the leading coefficient is the same. So we have an
approximate equality $f_{M'}^+(t) + f_{M''}^+(t) \simeq f_M^+(t)$. This implies the
result since the degree of $f_M^+$ is $\dim N$ (and similarly for the others). 
\end{proof} 



\begin{proposition} 
$\dim R$ is the same as $\dim R/\rad R$.
\end{proposition} 
I.e., the dimension doesn't change when you kill off nilpotent elements, which
is what you would expect, as nilpotents don't affect $\spec (R)$.
\begin{proof} 
For this, we need a little more information about Hilbert functions.
We thus digress substantially. 


Finally, let us return to the claim about dimension and nilpotents. Let $R$ be
a local noetherian ring and $I = \rad (R)$. Then $I$ is a finite $R$-module. In
particular, $I$ is nilpotent, so $I^n  = 0$ for $n \gg 0$. We will show that
\[ \dim R/I = \dim R/I^2 = \dots \]
which will imply the result, as eventually the powers become zero. 

In particular, we have to show for each $k$, 
\[ \dim R/I^k  = \dim R/I^{k+1}.  \]
There is an exact sequence
\[ 0 \to I^k/I^{k+1} \to R/I^{k+1} \to R/I^k \to 0.  \]
The dimension of these rings is the same thing as the dimensions as
$R$-modules. So we can use this short exact sequence of modules. By the
previous result, we are reduced to showing that
\[ \dim I^k/I^{k+1} \leq \dim R/I^k.  \]
Well, note that $I$ kills $I^k/I^{k+1}$. In particular, $I^k/I^{k+1}$ is a finitely generated
$R/I^k$-module. There is an exact sequence
\[ \bigoplus_N R/I^k \to I^k/I^{k+1} \to 0  \]
which implies that $\dim I^k/I^{k+1} \leq \dim \bigoplus_N R/I^k = \dim R/I^k$.
\end{proof} 

\begin{example} 
Let $\mathfrak{p} \subset \mathbb{C}[x_1, \dots, x_n]$ and let $R =
(\mathbb{C}[x_1,\dots, x_n]/\mathfrak{p})_{\mathfrak{m}}$ for some maximal
ideal $\mathfrak{m}$. What is $\dim R$? 
What does dimension mean for coordinate rings over $\mathbb{C}$?

Recall by the Noether normalization theorem that there exists a polynomial ring
$\mathbb{C}[y_1, \dots, y_m]$ contained in $S=\mathbb{C}[x_1,\dots,
x_n]/\mathfrak{p}$ and $S$ is a finite integral extension over this polynomial ring. 
We claim that 
\[ \dim R = m.  \]
There is not sufficient time for that today. 
\end{example} 

\subsection{Dimension depends only on the support}

Let $(R, \mathfrak{m})$ be a local noetherian ring. Let $M$ be a finitely generated
$R$-module. 	We defined the \textbf{Hilbert polynomial} of $M$ to be the
polynomial which evaluates at $t \gg 0$ to $\ell(M/\mathfrak{m}^tM)$. We proved
last time that such a polynomial always exists, and called its degree the
\textbf{dimension of $M$}. However, 
we shall now see that $\dim M$ really depends only on the support\footnote{
Recall that $\supp M = \left\{\mathfrak{p}: M_{\mathfrak{p}}\neq 0\right\}$.} $\supp M$.
In this sense, the dimension is really a statement about the \emph{topological
space} $\supp M \subset \spec R$, not about $M$ itself. 


In other words, we will prove:
\begin{proposition} 
$\dim M$ depends only on $\supp M$.
\end{proposition} 

In fact, we shall show:

\begin{proposition} 
$\dim M = \max_{\mathfrak{p} \in \supp M} \dim R/\mathfrak{p}$.
\end{proposition} 
\begin{proof} 
By \rref{filtrationlemma} in \rref{noetherian}, there is a finite filtration 
\[ 0 = M_0 \subset M_1 \subset \dots \subset M_m = M,  \]
such that each of the successive quotients is isomorphic to $R/\mathfrak{p}_i
\subset R$
for some prime ideal $\mathfrak{p}_i$. Given a short exact sequence
of modules, we know that the dimension in the middle is the maximum of the dimensions at the
two ends (\rref{dimexactseq}). Iterating this, we see that the
dimension of $M$ is the maximum of the
dimension of the successive quotients $M_i/M_{i-1}$.

But the $\mathfrak{p}_i$'s that occur
are all in $\supp M$, so we find 
\[ \dim M = \max_{\mathfrak{p}_i} R/\mathfrak{p}_i \leq \max_{\mathfrak{p} \in \supp M} \dim R/\mathfrak{p}.  \]
We must show the reverse inequality. But fix any prime $\mathfrak{p} \in \supp
M$. Then $M_{\mathfrak{p}} \neq 0$, so one of the $R/\mathfrak{p}_i$ localized
at  $\mathfrak{p}$ must be nonzero, as localization is an exact functor. Thus
$\mathfrak{p}$ must contain some $\mathfrak{p}_i$. So $R/\mathfrak{p}$ is a
quotient of $R/\mathfrak{p}_i$. In particular,
\[ \dim R/\mathfrak{p} \leq \dim R/\mathfrak{p}_i.  \]
\end{proof} 

Having proved this, we throw out the notation $\dim M$, and henceforth write
instead $\dim \supp M$.



%% N. B. This following material on affine rings should be added in 
\begin{comment}
\subsection{The dimension of an affine ring} Last time, we made a claim. If $R$
is a domain and a finite module over a polynomial ring $k[x_1, \dots, x_n]$,
then $R_{\mathfrak{m}}$ for any maximal $\mathfrak{m} \subset R$ has dimension
$n$. This connects the dimension with the transcendence degree. 

First, let us talk about finite extensions of rings. Let $R$ be a commutative
ring and let $R \to R'$ be a morphism that makes $R'$ a finitely generated $R$-module (in
particular, integral over $R$). Let $\mathfrak{m}' \subset R'$ be maximal. Let
$\mathfrak{m}$ be the pull-back to $R$, which is also maximal (as $R \to R'$ is
integral). 
Let $M$ be a finitely generated $R'$-module, hence also a finitely generated $R$-module. 

We can look at $M_{\mathfrak{m}}$ as an $R_{\mathfrak{m}}$-module or
$M_{\mathfrak{m}'}$ as an $R'_{\mathfrak{m}'}$-module. Either of these will be
finitely generated. 

\begin{proposition} 
$\dim \supp M_{\mathfrak{m}}  \geq \dim \supp M_{\mathfrak{m}'}$.
\end{proposition} 
Here $M_{\mathfrak{m}}$ is an $R_{\mathfrak{m}}$-module, $M_{\mathfrak{m}'}$ is
an $R'_{\mathfrak{m}'}$-module. 

\begin{proof} 
Consider $R/\mathfrak{m} \to R'/\mathfrak{m} R' \to R'/\mathfrak{m}'$. Then we
see that $R'/\mathfrak{m} R'$ is a finite $R/\mathfrak{m}$-module, so a
finite-dimensional $R/\mathfrak{m}$-vector space. In particular,
$R'/\mathfrak{m} R'$ is of finite length as an $R/\mathfrak{m}$-module, in
particular an artinian ring. It is thus a product of local artinian rings.
These artinian rings are the localizations of $R'/\mathfrak{m}R'$ at ideals of
$R'$ lying over $\mathfrak{m}$. One of these ideals is $\mathfrak{m}'$. 
So in particular
\[ R'/\mathfrak{m}R \simeq R'/\mathfrak{m}'\times \mathrm{other \ factors}.  \]
The nilradical of an artinian ring being nilpotent, we see that
$\mathfrak{m}'^c R'_{\mathfrak{m}'} \subset \mathfrak{m} R'_{\mathfrak{m}}$ for
some $c$.

OK, I'm not following this---too tired. Will pick this up someday.
\end{proof} 


\begin{proposition} 
$\dim \supp M_{\mathfrak{m}} = \max_{\mathfrak{m}' \mid \mathfrak{m}} \dim
\supp M_{\mathfrak{m}'}$.
\end{proposition} 

This means $\mathfrak{m}'$ lies over $\mathfrak{m}$.
\begin{proof} 
Done similarly, using artinian techniques. I'm kind of tired.
\end{proof} 
\end{comment}
\begin{example} 
Let $R' = \mathbb{C}[x_1, \dots, x_n]/\mathfrak{p}$. Noether normalization says
that there exists a finite injective map $\mathbb{C}[y_1, \dots, y_a] \to R'$.
The claim is that
\[ \dim R'_{\mathfrak{m}} =a  \]
for any maximal ideal $\mathfrak{m} \subset R'$. We are set up to prove a
slightly weaker definition. In particular (see below for the definition of the
dimension of a non-local ring), by the proposition, we
find the weaker claim
\[ \dim R' = a,  \]
as the dimension of a polynomial ring $\mathbb{C}[y_1, \dots, y_a]$ is $a$.
(\textbf{I don't think we have proved this yet.})
\end{example} 



\section{Other definitions and characterizations of dimension}

\subsection{The topological characterization of dimension} We now want a topological
characterization of dimension. So, first, we want to study how dimension
changes as we do things to a module. Let $M$ be a finitely generated $R$-module over a local
noetherian ring $R$. Let $x \in \mathfrak{m}$ for $\mathfrak{m}$ as the maximal
ideal.
You might ask
\begin{quote}
What is the relation between $\dim \supp M$ and $\dim \supp M/xM$?
\end{quote}
Well, $M$ surjects onto $M/xM$, so we have the inequality $\geq$. But we think
of dimension as describing the number of parameters you need to describe
something. The number of parameters shouldn't change too much with going from
$M$ to $M/xM$. Indeed, as one can check,
\[ \supp M/xM = \supp M \cap V(x)  \]
and intersecting $\supp M$ with the ``hypersurface'' $V(x)$ should shrink the
dimension by one. 


We thus make:
\begin{prediction}
\[ \dim \supp M/xM = \dim \supp M - 1.  \]
\end{prediction}
Obviously this is not always true, e.g. if $x$ acts by zero on $M$. But we want
to rule that out. 
Under reasonable cases, in fact, the prediction is correct:

\begin{proposition} \label{dimdropsbyone}
Suppose $x \in \mathfrak{m}$ is a nonzerodivisor on $M$. Then 
\[ \dim \supp M/xM = \dim \supp M - 1.  \]
\end{proposition} 
\begin{proof} 
To see this, we look at Hilbert polynomials. Let us consider the exact sequence
\[ 0 \to xM \to M \to M/xM \to 0  \]
which leads to an exact sequence for each $t$,
\[ 0 \to xM/(xM \cap \mathfrak{m}^t M) \to M/\mathfrak{m}^t M \to M/(xM  +
\mathfrak{m}^t M) \to 0 . \]
For $t$ large, the lengths of these things are given by Hilbert polynomials,
as the thing on the right is $M/xM \otimes_R R/\mathfrak{m}^t$. 
We have
\[ f_M^+(t) = f_{M/xM}^+(t) + \ell(xM/ (x M \cap \mathfrak{m}^t M), \quad t
\gg 0.  \]
In particular, $\ell( xM/ (xM \cap \mathfrak{m}^t M))$ is a polynomial in $t$.
What can we say about it? Well, $xM \simeq M$ as $x$ is a nonzerodivisor. In
particular
\[ xM / (xM \cap \mathfrak{m}^t M) \simeq M/N_t  \]
where
\[ N_t = \left\{a \in M: xa \in \mathfrak{m}^t M\right\} . \]
In particular, $N_t \supset \mathfrak{m}^{t-1} M$. This tells us that
$\ell(M/N_t) \leq \ell(M/\mathfrak{m}^{t-1} M) = f_M^+(t-1)$ for $t \gg 0$.
Combining this with the above information, we learn that
\[ f_M^+(t) \leq f_{M/xM}^+(t) + f_M^+(t-1),   \]
which implies that $f_{M/xM}^+(t)$ is at least the successive difference
$f_M^+(t) - f_M^+(t-1)$. This last polynomial has degree $\dim \supp M -1$. In
particular, $f_{M/xM}^+(t)$ has degree at least $\dim \supp M -1 $. This gives
us one direction, actually the hard one. We showed that intersecting something with codimension one
doesn't drive the dimension down too much. 

Let us now do the other direction. We essentially did this last time via the
Artin-Rees lemma. We know that $N_t = \left\{a \in M: xa \in
\mathfrak{m}^t\right\}$. The Artin-Rees lemma tells us that there is a constant
$c$ such that $N_{t+c} \subset \mathfrak{m}^t M$ for all $t$. Therefore,
$\ell(M/N_{t+c}) \geq \ell(M/\mathfrak{m}^t M) = f_M^+(t), t \gg 0$. Now
remember the exact sequence $0 \to M/N_t \to M/\mathfrak{m}^t M \to M/(xM +
\mathfrak{m}^t M) \to 0$. We see from this that
\[ \ell(M/ \mathfrak{m}^t M) = \ell(M/N_t) + f_{M/xM}^+(t) \geq f_M^+(t-c) +
f_{M/xM}^+(t), \quad t \gg 0,  \]
which implies that
\[ f_{M/xM}^+(t) \leq f_M^+(t) - f_M^+(t-c),  \]
so the degree must go down. And we find that $\deg f_{M/xM}^+ < \deg f_{M}^+$.
\end{proof} 

This gives us an algorithm of computing the dimension of an $R$-module $M$. 
First, it reduces to computing $\dim R/\mathfrak{p}$ for $\mathfrak{p} \subset
R$ a prime ideal. We may assume that $R$ is a domain and that we are looking
for $\dim R$. Geometrically, this
corresponds to taking an irreducible component of $\spec R$.

Now choose any $x
\in R$ such that $x$ is nonzero but noninvertible. If there is no such element,
then $R$ is a field and has dimension zero. Then compute $\dim R/x$
(recursively) and add one.

Notice that this algorithm said nothing about Hilbert polynomials, and only
talked about the structure of prime ideals.

\subsection{Recap}
Last time, we were talking about dimension theory. 
Recall that $R$ is a local noetherian ring with maximal ideal $\mathfrak{m}$,
$M$ a finitely generated $R$-module. We can look at the lengths $\ell(M/\mathfrak{m}^t M)$
for varying $t$; for $t \gg 0$ this is a polynomial function. The degree of
this polynomial is called the \textbf{dimension} of $\supp M$. 

\begin{remark} 
If $M = 0$, then we define $\dim \supp M = -1$ by convention.
\end{remark} 

Last time, we showed that if $M \neq 0$ and $x \in \mathfrak{m}$ such that $x$
is a nonzerodivisor on $M$ (i.e. $M \stackrel{x}{\to} M$ injective), then 
\[ \boxed{ \dim \supp M/xM = \dim \supp M - 1.}\]
Using this, we could give a recursion for calculating the dimension. 
To compute $\dim R = \dim \spec R$, we note three properties:
\begin{enumerate}
\item $\dim R = \sup_{\mathfrak{p} \ \mathrm{a \ minimal \ prime}}
R/\mathfrak{p}$. Intuitively, this says that a variety which is the union of
irreducible components has dimension equal to the maximum of these irreducibles.
\item $\dim R = 0$ for $R$  a field. This is obvious from the definitions.
\item If $R$ is a domain, and $x \in \mathfrak{m} - \left\{0\right\}$, then
$\dim R/(x) +1 = \dim R $. This is obvious from the boxed formula as $x$ is a nonzerodivisor.
\end{enumerate}

These three properties \emph{uniquely characterize} the dimension invariant. 

\textbf{More precisely, if
$d: \left\{\mathrm{local \ noetherian \ rings}\right\} \to \mathbb{Z}_{\geq 0}$
satisfies the above three properties, then $d = \dim $. }
\begin{proof} 
Induction on $\dim R$. It is clearly sufficient to prove this for $R$ a domain. 
If $R$ is a field, then it's clear; if $\dim R>0$, the third condition lets us
reduce to a case covered by the inductive hypothesis (i.e. go down).
\end{proof} 

Let us rephrase 3 above:
\begin{quote}
3': If $R$ is a domain and not a field, then 
\[ \dim R = \sup_{x \in \mathfrak{m} - 0} \dim R/(x) + 1. \]
\end{quote}
Obviously 3' implies 3, and it is clear by the same argument that 1,2, 3'
characterize the notion of dimension.

\subsection{Krull dimension} We shall now define another notion of
dimension, and show that it is equivalent to the older one by showing that it
satisfies these axioms.

\begin{definition} 
Let $R$ be a commutative ring. A \textbf{chain of prime ideals} in $R$ is a finite
sequence
\[ \mathfrak{p}_0 \subsetneq \mathfrak{p}_1 \subsetneq \dots \subsetneq
\mathfrak{p}_n.  \]
This chain is said to have \textbf{length $n$.}
\end{definition} 

\begin{definition} 
The \textbf{Krull dimension} of $R$ is equal to the maximum length of any chain
of prime ideals. This might be $\infty$, but we will soon see this cannot
happen for $R$ local and noetherian.
\end{definition} 

\begin{remark} 
For any maximal chain $\left\{\mathfrak{p}_i, 0 \leq i \leq n\right\}$ of primes (i.e. which can't be expanded), we must have
that $\mathfrak{p}_0$ is minimal prime and $\mathfrak{p}_n$ a maximal ideal.
\end{remark} 

\begin{theorem} 
For a noetherian local ring $R$, the Krull dimension of $R$ exists and is equal
to the usual $\dim R$.
\end{theorem}
\begin{proof} 
We will show that the Krull dimension satisfies the above axioms. For now,
write $\krdim$ for Krull dimension.

\begin{enumerate}
\item First, note that $\krdim(R) = \max_{\mathfrak{p} \in R \
\mathrm{minimal}}  \krdim(R/\mathfrak{p})$. This is because any chain of prime
ideals in $R$ contains a minimal prime. So any chain of prime ideals in $R$ can
be viewed as a chain in \emph{some} $R/\mathfrak{p}$, and conversely.
\item Second, we need to check that $\krdim(R) = 0$ for $R$ a field. This is
obvious, as there is precisely one prime ideal.
\item The third condition is interesting. We must check that for $(R,
\mathfrak{m})$ a local
domain, 
\[ \krdim(R) = \max_{x \in \mathfrak{m} - \left\{0\right\}} \krdim(R/(x)) + 1.  \]
If we prove this, we will have shown that condition 3' is satisfied by the
Krull dimension. It will follow by the inductive argument above that $\krdim(R)
= \dim (R)$ for any $R$. 
There are two inequalities to prove. First, we must show
\[ \krdim(R) \geq \krdim(R/x) +1, \quad \forall x \in \mathfrak{m} - 0.  \]
So suppose $k = \krdim(R/x)$. We want to show that there is a chain of prime
ideals of length $k+1$ in $R$. So say $\mathfrak{p}_0 \subsetneq \dots
\subsetneq \mathfrak{p}_k$ is a chain of length $k$ in $R/(x)$. The inverse
images in $R$ give a proper chain of primes in $R$ of  length $k$, all of which
contain $(x)$ and thus properly contain $0$. Thus adding zero will give a chain
of primes in $R$ of length $k+1$. 

Conversely, we want to show that if there is a chain of primes in $R$ of
length  $k+1$, then there is a chain of length $k$ in $R/(x)$ for some $x \in
\mathfrak{m} - \left\{0\right\}$. Let us write the chain  of length $k+1$:
\[ \mathfrak{q}_{-1} \subset \mathfrak{q}_0 \subsetneq \dots \subsetneq
\mathfrak{q}_k \subset R . \]
Now evidently $\mathfrak{q}_0$ contains some $x \in \mathfrak{m} - 0$. Then the
chain $\mathfrak{q}_0 \subsetneq \dots \subsetneq \mathfrak{q}_k$ can be
identified with a chain in $R/(x)$ for this $x$. So for this $x$, we have that
$\krdim R \leq \sup \krdim R/(x) + 1$.
\end{enumerate}
\end{proof} 

There is thus a combinatorial definition of definition.

Geometrically, let $X = \spec R$ for $R$ an affine ring over $\mathbb{C}$ (a
polynomial ring mod some ideal). Then $R$ has Krull dimension $\geq k$ iff there is a
chain of irreducible subvarieties of $X$,
\[ X_0 \supset X_1 \supset \dots \supset X_k . \]
You will meet justification for this in \rref{subsectiondimension} below.

\begin{remark}[\textbf{Warning!}] Let $R$ be a local noetherian ring of dimension $k$. This
means that there is a chain of prime ideals of length $k$, and no longer
chains. Thus there is a maximal chain whose length is $k$. However, not all
maximal chains in $\spec R$ have length $k$. 
\end{remark} 

\begin{example} 
Let $R =( \mathbb{C}[X,Y,Z]/(XY,XZ))_{(X,Y,Z)}$. It is left as an
exercise to the reader to see that there are maximal chains of
length not two.

There are more complicated local noetherian \emph{domains} which have maximal
chains of prime ideals not of the same length. These examples are not what you
would encounter in daily experience, and are necessarily complicated. This
cannot happen for finitely generated domains over a field.
\end{example} 

\begin{example} 
An easier way all maximal chains could fail to be of the same length is if
$\spec R$ has two components (in which case $R = R_0 \times R_1$ for rings
$R_0, R_1$). 
\end{example} 


\subsection{Yet another definition}
Let's start by thinking about the definition of a module. Recall that if $(R,
\mathfrak{m})$ is
a local noetherian ring and $M$ a finitely generated $R$-module, and $x \in \mathfrak{m}$ is
a nonzerodivisor on $M$, then
\[ \dim \supp M/xM = \dim \supp M -1.  \]

\begin{question} 
What if $x$ is  a zerodivisor? 
\end{question} 

This is not necessarily true (e.g. if $x \in \ann(M)$). Nonetheless, we claim
that even in this case:
\begin{proposition} 
For any $x \in \mathfrak{m}$,
\[ \boxed{ \dim \supp M \geq \dim \supp M/xM \geq \dim \supp M -1 .}\]
\end{proposition} 
The upper bound on $\dim M/xM$ is obvious as $M/xM$ is a quotient of $M$. The
lower bound is trickier. 

\begin{proof} 
Let $N = \left\{a \in M: x^n a = 0 \ \mathrm{for \ some \ } n \right\}$. We can
construct an exact sequence
\[ 0 \to N \to M \to M/N \to 0.  \]
Let $M'' = M/N$.
Now $x$ is a nonzerodivisor on $M/N$ by construction. We claim that 
\[ 0 \to N/xN \to M/xM \to M''/xM'' \to 0  \]
is exact as well. For this we only need to see exactness at the beginning,
i.e. injectivity of $N/xN \to M/xM$. So
we need to show that if $a \in N$ and $a \in xM$, then $a \in x N$.

To see this, suppose $a = xb$ where $b \in M$. Then if $\phi: M \to M''$, then
$\phi(b) \in M''$ is killed by $x$ as $x \phi(b) = \phi(bx) = \phi(a)$.
This means that $\phi(b)=0$ as $M'' \stackrel{x}{\to} M''$ is injective. Thus
$b \in N$ in fact. So $a \in xN$ in fact.

From the exactness, we see that (as $x$ is a nonzerodivisor on $M''$)
\begin{align*} \dim M/xM & = \max (\dim M''/xM'', \dim N/xN) \geq \max(\dim M'' -1, \dim
N)\\ &  \geq \max( \dim M'', \dim N)-1  .  \end{align*}
The reason for the last claim is that $\supp N/xN = \supp N$ as $N$ is
$x$-torsion, and the dimension depends only on the support. But the thing on the right is just $\dim M -1$. 
\end{proof} 

As a result, we find:

\begin{proposition} 
$\dim \supp M$ is the minimal integer $n$ such that there exist elements $x_1,
\dots, x_n \in \mathfrak{m}$ with $M/(x_1 , \dots, x_n) M$ has finite length. 
\end{proposition} 
Note that $n$ always exists, since we can look at a bunch of generators of the
maximal ideal, and $M/\mathfrak{m}M $ is a finite-dimensional vector space and
is thus of finite length.
\begin{proof} 
Induction on $\dim \supp M$. Note that $\dim \supp(M)=0$ if and only if the
Hilbert polynomial has degree zero, i.e. $M$ has finite length or that $n=0$
($n$ being defined as in the statement). 

Suppose $\dim \supp M > 0$. \begin{enumerate}
\item  We first show that there are $x_1, \dots, x_{\dim M}$
with $M/(x_1, \dots, x_{\dim M})M$ have finite length. 
Let $M' \subset M$ be the maximal submodule having finite length. There
is an exact sequence
\[ 0 \to M' \to M \to M'' \to 0  \]
where $M'' = M/M'$ has no finite length submodules. In this case, we can
basically ignore $M'$, and replace $M$ by $M''$. The reason is that modding out
by $M'$ doesn't affect either $n$ or the dimension. 

So let us replace $M$ with
$M''$ and thereby assume that $M$ has no finite length submodules. In
particular, $M$ does not contain a copy of $R/\mathfrak{m}$, i.e. $\mathfrak{m}
\notin \ass(M)$. 
By prime avoidance, this means that there is $x_1 \in \mathfrak{m}$ that acts as
a nonzerodivisor on $M$. Thus
\[ \dim M/x_1M = \dim M -1.  \]
The inductive hypothesis says that there are $x_2, \dots, x_{\dim M}$ with
$$(M/x_1 M)/(x_2, \dots, x_{\dim M}) (M/xM) \simeq M/(x_1, \dots, x_{\dim M})M $$
of finite length. This shows the claim.
\item Conversely, suppose that there $M/(x_1, \dots, x_n)M$ has finite length.
Then we claim that $n \geq \dim M$. This follows because we had the previous
result that modding out by a single element can chop off the dimension by at
most $1$. Recursively applying this, and using the fact that $\dim$ of a
finite length module is zero, we find
\[ 0 = \dim M/(x_1 , \dots, x_n )M \geq \dim M -n. \]
\end{enumerate}
\end{proof} 


\begin{corollary} 
Let $(R, \mathfrak{m})$ be a local noetherian ring. Then $\dim R$ is equal to the minimal $n$
such that there exist $x_1, \dots, x_n \in R$ with $R/(x_1, \dots, x_n) R$ is
artinian. Or, equivalently, such that $(x_1, \dots, x_n)$ contains a power of
$\mathfrak{m}$.
\end{corollary}


\begin{remark} 
We manifestly have here that the dimension of $R$ is at most the embedding
dimension. Here, we're not worried about generating the maximal ideal, but
simply something containing a power of it.
\end{remark} 
\lecture{11/5}

We have been talking about dimension. Let $R$ be a local noetherian ring with
maximal ideal $\mathfrak{m}$. Then, as we have said in previous lectures, $\dim R$ can be characterized by:
\begin{enumerate}
\item The minimal $n$ such that there is an $n$-primary ideal generated by $n$
elements $x_1, \dots, x_n \in \mathfrak{m}$. That is, the closed point
$\mathfrak{m}$ of
$\spec R$ is cut out \emph{set-theoretically} by the intersection $\bigcap
V(x_i)$. This is one way of saying that the closed point can be defined by $n$
parameters. 
\item The \emph{maximal} $n$ such that there exists a chain of prime ideals
\[ \mathfrak{p}_0 \subset \mathfrak{p}_1 \subset \dots \subset \mathfrak{p}_n. \]
\item The degree of the Hilbert polynomial $f^+(t)$, which equals
$\ell(R/\mathfrak{m}^t)$ for $t \gg 0$.
\end{enumerate}


\subsection{Krull's Hauptidealsatz}


Let $R$ be a local noetherian ring.
The following is now clear from what we have shown:

\begin{theorem} \label{hauptv1}
$R$ has dimension $1$ if and only if there is a nonzerodivisor $x \in \mathfrak{m}$ such that
$R/(x)$ is artinian.
\end{theorem} 



\begin{remark} 
Let $R$ be a domain. We said that a nonzero prime $\mathfrak{p} \subset R$ is
\textbf{height one} if $\mathfrak{p}$ is minimal among the prime ideals
containing some nonzero $x \in R$. 

According to Krull's Hauptidealsatz, $\mathfrak{p}$ has height one \textbf{if
and only if $\dim R_{\mathfrak{p}} = 1$.}
\end{remark} 


We can generalize the notion of $\mathfrak{p}$ as follows.
\begin{definition} 
Let $R$ be a noetherian ring (not necessarily local), and $\mathfrak{p} \in
\spec R$. Then we define the \textbf{height} of $\mathfrak{p}$, denoted
$\het(\mathfrak{p})$, as $\dim R_{\mathfrak{p}}$.
We know that this is the length of a maximal chain of primes in
$R_{\mathfrak{p}}$. This is thus the maximal length of prime ideals of $R$, 
\[ \mathfrak{p}_0 \subset \dots \subset \mathfrak{p}_n = \mathfrak{p}  \]
that ends in $\mathfrak{p}$. This is the origin of the term ``height.''
\end{definition} 

\begin{remark} 
Sometimes, the height is called the \textbf{codimension}. This corresponds to
the codimension in $\spec R$ of the corresponding irreducible closed subset of
$\spec R$.
\end{remark} 

\begin{theorem}[Krull's Hauptidealsatz]  Let $R$ be a noetherian ring, and $x
\in R$ a nonzerodivisor. If $\mathfrak{p} \in \spec R$ is minimal over $x$,
then $\mathfrak{p}$ has height one.
\end{theorem} 
\begin{proof} 
Immediate from \cref{hauptv1}.
\end{proof} 

\begin{theorem}[Artin-Tate]
Let $A$ be a noetherian domain. Then the following are equivalent:
\begin{enumerate}
\item There is $f \in A-\left\{0\right\} $ such that $A_f$ is a field.
\item $A$ has finitely many maximal ideals and has dimension at most 1.
\end{enumerate}
\end{theorem} 
\begin{proof} We follow \cite{EGA}.

Suppose first that there is $f$ with $A_f$ a field. 
Then all nonzero prime ideals of $A$ contain $f$. 
We need to deduce that $A$ has dimension $\leq 1$. Without loss of generality,
we may assume that $A$ is not a field.

There are finitely many primes $\mathfrak{p}_1,\dots, \mathfrak{p}_k$ which
are minimal over $f$; these are all height one. The claim is that any maximal ideal of $A$ is of this
form. Suppose $\mathfrak{m}$ were maximal and not one of the $\mathfrak{p}_i$.
Then by prime avoidance, there is $g \in \mathfrak{m}$ which
lies in no $\mathfrak{p}_i$. A minimal prime $\mathfrak{P}$ of $g$ has height
one, so by our assumptions contains $f$. However, it is then one of the
$\mathfrak{p}_i$; this is a contradiction as $g \in \mathfrak{P}$.
\end{proof} 


\subsection{Further remarks}

We can recast earlier notions in terms of dimension.
\begin{remark} 
A noetherian ring has dimension zero if and only if $R$ is artinian. Indeed,
$R$ has dimension zero iff all primes are maximal.
\end{remark} 


\begin{remark} 
A noetherian domain has dimension zero iff it is a field. Indeed, in this case
$(0)$ is maximal.
\end{remark} 

\begin{remark} 
$R$ has dimension $\leq 1$ if and only if every non-minimal prime of $R$ is
maximal. That is, there are no chains of length $\geq 2$.
\end{remark} 

\begin{remark} 
A (noetherian) domain  $R$ has dimension $\leq 1$ iff every nonzero prime ideal
is maximal.
\end{remark} 

In particular,
\begin{proposition} 
$R$ is Dedekind iff it is a noetherian, integrally closed domain of dimension
$1$. 
\end{proposition} 


\section{Further topics}

\subsection{Change of rings}
Let $f: R \to R'$ be  a map of noetherian rings. 

\begin{question} 
What is the relationship between $\dim R$ and $\dim R'$?
\end{question} 

A map $f$ gives a map $\spec R' \to \spec R$, where $\spec R'$ is the union
of various fibers over the points of $\spec R$. You might imagine that the
dimension is the dimension of $R$ plus the fiber dimension. This is sometimes
true.

Now assume that $R, R'$ are \emph{local}  with maximal ideals $\mathfrak{m},
\mathfrak{m}'$. Assume furthermore that $f$ is local, i.e. $f(\mathfrak{m})
\subset \mathfrak{m}'$.

\begin{theorem} 
$\dim R' \leq \dim R +  \dim R'/\mathfrak{m}R'$. Equality holds if $f: R \to
R'$ is flat.
\end{theorem} 

Here $R'/\mathfrak{m}R'$ is to be interpreted as the ``fiber'' of $\spec R'$
above $\mathfrak{m} \in \spec R$. The fibers can behave weirdly as the
basepoint varies in $\spec R$, so we can't
expect equality in general.

\begin{remark} 
Let us review flatness as it has been a while. An $R$-module $M$ is \emph{flat} iff
the operation of tensoring with $M$ is an exact functor. The map $f: R \to R'$
is \emph{flat} iff $R'$ is a flat $R$-module. Since the construction of taking
fibers is a tensor product (i.e. $R'/\mathfrak{m}R' = R' \otimes_R
R/\mathfrak{m}$), perhaps the condition of flatness here is not as surprising as
it might be.
\end{remark} 

\begin{proof} 
Let us first prove the inequality. Say $$\dim R = a,  \ \dim R'/\mathfrak{m}R'
= b.$$ We'd like to see that
\[ \dim R' \leq a+b.  \]
To do this, we need to find $a+b$ elements in the maximal ideal $\mathfrak{m}'$
that generate a $\mathfrak{m}'$-primary ideal of $R'$. 

There are elements $x_1, \dots, x_a \in \mathfrak{m}$ that generate an
$\mathfrak{m}$-primary ideal $I = (x_1, \dots, x_a)$ in $R$. There is a surjection $R'/I R'
\twoheadrightarrow R'/\mathfrak{m}R'$.
The kernel $\mathfrak{m}R'/IR'$ is nilpotent since $I$ contains a power of
$\mathfrak{m}$. 	We've seen that nilpotents \emph{don't} affect the dimension.
In particular, 
\[ \dim R'/IR' = \dim R'/\mathfrak{m}R' = b.  \]
There are thus elements $y_1, \dots, y_b \in \mathfrak{m}'/IR'$ such that the
ideal $J = (y_1, \dots, y_b) \subset R'/I R'$ is $\mathfrak{m}'/IR'$-primary.
The inverse image of $J$ in $R'$, call it $\overline{J} \subset R'$, is
$\mathfrak{m}'$-primary. However, $\overline{J}$ is generated by the $a+b$
elements
\[ f(x_1), \dots, f(x_a), \overline{y_1}, \dots, \overline{y_b}  \]
if the $\overline{y_i}$ lift $y_i$. 

But we don't always have equality. Nonetheless, if all the fibers are similar,
then we should expect that the dimension of the ``total space'' $\spec R'$ is
the dimension of the ``base'' $\spec R$ plus the ``fiber'' dimension $\spec
R'/\mathfrak{m}R'$.  
\emph{The precise condition of $f$ flat articulates the condition that the fibers
 ``behave well.'' }
Why this is so is something of a mystery, for now.
But for some evidence, take the present result about fiber dimension.

Anyway, let us now prove equality for flat $R$-algebras. As before, write $a =
\dim R, b = \dim R'/\mathfrak{m}R'$. We'd like to show that
\[ \dim R' \geq a+b.  \]
By what has been shown, this will be enough.
This is going to be tricky since we now need to give \emph{lower bounds} on the
dimension; finding a sequence $x_{1}, \dots, x_{a+b}$ such that the quotient
$R/(x_1, \dots, x_{a+b})$ is artinian would bound \emph{above} the dimension.

So our strategy will be to find a chain of primes of length $a+b$. Well, first
we know that there are primes
\[ \mathfrak{q}_0 \subset \mathfrak{q}_1 \subset \dots \subset \mathfrak{q}_b
\subset R'/\mathfrak{m}R'.  \]
Let $\overline{\mathfrak{q}_i}$ be the inverse images in $R'$. Then the
$\overline{\mathfrak{q}_i}$ are a strictly ascending chain of primes in $R'$ where
$\overline{\mathfrak{q}_0}$ contains $\mathfrak{m}R'$. So we have a chain of
length $b$; we need to extend this by additional terms.

Now $f^{-1}(\overline{\mathfrak{q}_0})$ contains $\mathfrak{m}$, hence is
$\mathfrak{m}$. Since $\dim R = a$, there is a chain
$\left\{\mathfrak{p}_i\right\}$ of prime ideals of length
$a$ going down from $f^{-1}(\overline{\mathfrak{q}_0}) = \mathfrak{m}$. We are 
now going to find primes $\mathfrak{p}_i' \subset R'$ forming a chain such that
$f^{-1}(\mathfrak{p}_i') = \mathfrak{p}_i$. In other words, we are going to
\emph{lift} the chain $\mathfrak{p}_i$ to $\spec R'$. We can do this at the
first stage for $i=a$, where $\mathfrak{p}_a = \mathfrak{m}$ and we can set
$\mathfrak{p}'_a = \overline{\mathfrak{q}_0}$. If we can indeed do this
lifting, and catenate the chains $\overline{\mathfrak{q}_j}, \mathfrak{p}'_i$,
then we will have a chain of the appropriate length.

We will proceed by descending induction. Assume that we have
$\mathfrak{p}_{i+1}' \subset R'$ and $f^{-1}(\mathfrak{p}_{i+1}') =
\mathfrak{p}_{i+1} \subset R$. We want to find $\mathfrak{p}_i' \subset
\mathfrak{p}'_{i+1}$ such that $f^{-1}(\mathfrak{p}_i') = \mathfrak{p}_i$. The
existence of that prime is a consequence of the following general fact.

\begin{theorem}[Going down] Let $f: R \to R'$ be a flat map of
noetherian commutative
rings. Suppose $\mathfrak{q} \in \spec R'$, and let $\mathfrak{p}
=f^{-1}(\mathfrak{q})$. Suppose $\mathfrak{p}_0 \subset \mathfrak{p}$ is a
prime of $R$. Then there is a prime $\mathfrak{q}_0 \subset \mathfrak{q}$ with 
\[ f^{-1}(\mathfrak{q}_0) = \mathfrak{p}_0.  \]
\end{theorem} 
\begin{proof} 
We may replace $R'$ with $R'_{\mathfrak{q}}$. There is still a map
\[ R \to R_{\mathfrak{q}}'  \]
which is flat as localization is flat. The maximal ideal in $R'_{\mathfrak{q}}$
has inverse image $\mathfrak{p}$. So the problem now reduces to finding
\emph{some} $\mathfrak{p}_0$ in the localization that pulls back appropriately.

Anyhow, throwing out the old $R$ and replacing with the localization, we may
assume that $R'$ is local and $\mathfrak{q}$ the maximal ideal. (The condition
$\mathfrak{q}_0 \subset \mathfrak{q}$ is now automatic.) 

The claim now is that we can replace $R$ with $R/\mathfrak{p}_0$ and $R'$ with
$R'/\mathfrak{p}_0 R' = R' \otimes R/\mathfrak{p}_0$. We can do this because
base change preserves flatness (see below), and in this case we can reduce to the case of
$\mathfrak{p}_0 = (0)$---in particular, $R$ is a domain. 
Taking these quotients just replaces $\spec R, \spec R'$ with closed subsets
where all the action happens anyhow.

Under these replacements, we now have:
\begin{enumerate}
\item $R'$ is local with maximal ideal $\mathfrak{q}$ 
\item $R$ is a domain and $\mathfrak{p}_0 = (0)$.
\end{enumerate}
We want a prime of $R'$ that pulls back to $(0)$ in $R$. I claim that any
minimal prime of $R'$ will work. 
Suppose otherwise. Let $\mathfrak{q}_0 \subset R'$ be a minimal prime, and
suppose $x \in R \cap f^{-1}(\mathfrak{q}_0) - \left\{0\right\}$. But
$\mathfrak{q}_0 \in \ass(R')$. So $f(x)$ is
a zerodivisor on $R'$. Thus multiplication by $x$ on $R'$ is not injective. 

But, $R$ is a domain, so $R \stackrel{x}{\to} R$ is injective. Tensoring with
$R'$ must preserve this, implying that $R' \stackrel{x}{\to} R'$ is injective
because $R'$ is flat. This is a contradiction.
\end{proof} 

We used:
\begin{lemma} 
Let $R \to R'$ be a flat map, and $S$ an $R$-algebra. Then $S \to S \otimes_R
R'$ is a flat map.
\end{lemma} 
\begin{proof} 
The construction of taking an $S$-module with $S \otimes_R R'$ is an exact
functor, because that's the same thing as taking an $S$-module, restricting to
$R$, and tensoring with $R'$.
\end{proof} 
The proof of the fiber dimension theorem is now complete.

\end{proof} 



\subsection{The dimension of a polynomial ring}

Adding an indeterminate variable corresponds geometrically to taking the
product with the affine line, and so should increase the dimension by one. We
show that this is indeed the case.
\label{dimpoly}
\begin{theorem} 
Let $R$ be a noetherian ring. Then $\dim R[X] = \dim R+1$.
\end{theorem} 

Interestingly, this is \emph{false} if $R$ is non-noetherian, cf. \cite{}.
Let $R$ be a ring of dimension $n$. 

\begin{lemma} 
$\dim R[x] \geq \dim R+1$.
\end{lemma} 
\begin{proof} 
Let $\mathfrak{p}_0 \subset \dots \subset \mathfrak{p}_n$ be a chain of primes of
length $n = \dim R$. Then $\mathfrak{p}_0 R[x] \subset \dots \subset
\mathfrak{p}_n R[x] \subset (x, \mathfrak{p}_n)R[x]$ is a chain of primes in
$R[x]$ of length $n+1$ because of the following fact: if $\mathfrak{q} \subset
R$ is prime, then so is $\mathfrak{q}R[x] \subset R[x]$.\footnote{This is
because $R[x]/\mathfrak{q}R[x] = (R/\mathfrak{q})[x]$ is a domain.} Note also
that as $\mathfrak{p}_n \subsetneq R$, we have that $\mathfrak{p}_n R[x]
\subsetneq (x, \mathfrak{p}_n)$. So this is indeed a legitimate chain. 
\end{proof} 

Now we need only show:
\begin{lemma} 
Let $R$ be noetherian of dimension $n$. Then $\dim R[x] \leq \dim R+1$.
\end{lemma} 
\begin{proof} 
Let $\mathfrak{q}_0 \subset \dots \subset \mathfrak{q}_m \subset R[x]$ be a chain of primes
in $R[x]$. Let $\mathfrak{m} = \mathfrak{q}_m \cap R$. Then if we localize and
replace $R$ with $R_{\mathfrak{m}}$, we get a chain of primes of length $m$ in
$R_{\mathfrak{m}}[x]$. 
In fact, we get more. We get a chain of primes of length $m$ in
$(R[x])_{\mathfrak{q}_m}$, and a \emph{local } inclusion of noetherian local rings
\[ R_{\mathfrak{m}} \hookrightarrow (R[x])_{\mathfrak{q}_m} . \]
To this we can apply the fiber dimension theorem. In particular, this implies
that
\[ m \leq \dim (R[x])_{\mathfrak{q}_m} \leq \dim R_{\mathfrak{m}} + \dim
(R[x])_{\mathfrak{q}_m} /\mathfrak{m} (R[x])_{\mathfrak{q}_m}. \]
Here $\dim R_{\mathfrak{m}} \leq \dim R = n$. So if we show that $\dim
(R[x])_{\mathfrak{q}_m} /\mathfrak{m} (R[x])_{\mathfrak{q}_m} \leq 1$, we will
have seen that $m \leq n+1$, and will be done. But this last ring is a
localization of $(R_{\mathfrak{m}}/\mathfrak{m}R_{\mathfrak{m}})[x]$, which is
a PID by the euclidean algorithm for polynomial rings over a field, and thus of
dimension  $\leq 1$.
\end{proof} 

\subsection{A refined fiber dimension theorem}

Let $R$ be a local noetherian domain, and let $R \to S$ be an injection of
rings making $S$ into an $R$-algebra. Suppose $S$ is also a local domain, such
that the morphism $R \to S$ is local. This is essentially the setup of
\cref{dimpoly}, but in this section, we make the refining assumption that $S$
is \emph{essentially of finite type} over $R$; in other words, $S$ is the
localization of a finitely generated $R$-algebra.

Let $k$ be the residue field of $R$, and $k'$ that of $S$; because $R \to S$ is
local, there is induced a morphism of fields $k \to k'$. 
We shall prove, following \cite{EGA}:
\newcommand{\trdeg}{\mathrm{tr.deg.}}
\begin{theorem}[Dimension formula]
\begin{equation}\label{strongfiberdim} \dim S + \trdeg S/R \leq \dim R + \trdeg
k'/k.  \end{equation}
\end{theorem} 
Here $\trdeg B/A$ is more properly the transcendence degree of the quotient
field of $B$ over that of $A$.
Geometrically, it corresponds to the dimension of the ``generic'' fiber. 

\begin{proof} Let $\mathfrak{m} \subset R$ be the maximal ideal.
We know that $S$ is a localization of an algebra of the form
$(R[x_1, \dots, x_k])/\mathfrak{p}$ where $\mathfrak{p} \subset R[x_1, \dots,
x_n]$ is a prime ideal $\mathfrak{q}$.
We induct on $k$.

Since we can ``d\'evissage'' the extension $R \to S$ as the
composite
\[ R \to (R[x_1, \dots, x_{k-1}]/(\mathfrak{p} \cap R[x_1, \dots,
x_{k-1}])_{\mathfrak{q'}}  \to S, \]
(where $\mathfrak{q}' \in \spec R[x_1, \dots, x_{k-1}]/(\mathfrak{p} \cap R[x_1, \dots,
x_{k-1}]$ is the pull-back of $\mathfrak{q}$),
we see that it suffices to prove \eqref{strongfiberdim} when $k=1$, that is $S$
is the localization of a quotient of $R[x]$.

So suppose $k=1$. Then we have $S = (R[x])_{\mathfrak{q}}/\mathfrak{p}$ where
$\mathfrak{q} \subset R[x]$ is another prime ideal lying over $\mathfrak{m}$.
Let us start by considering the case where $\mathfrak{q} = 0$.

\begin{lemma} Let $(R, \mathfrak{m})$ be a local noetherian domain as above.
Let $S = R[x]_{\mathfrak{q}}$ where $\mathfrak{q} \in \spec R[x]$ is a prime
lying over $\mathfrak{m}$. Then \eqref{strongfiberdim} holds with equality.
\end{lemma} 
\begin{proof}
In this case, $\trdeg S/R = 1$. Now $\mathfrak{q}$
could be $\mathfrak{m} R[x]$ or a prime ideal containing that, which is then
automatically
maximal, as we know from the proof of \cref{dimpoly}. Indeed, primes
containing $\mathfrak{m}R[x]$ are
in bijection with primes of $R/\mathfrak{m}[x]$, and these come in two forms:
zero, and those generated by one element. (Note that in the former case, the
residue field is the field of rational functions $k(x)$ and in the second, the residue field is finite over
$k$.)

\begin{enumerate}
\item 
In the first case, $\dim S = \dim R[x]_{\mathfrak{m}R[x]} = \dim R$ and but the
residue field extension is $(R[x]_{\mathfrak{m}R[x]})/\mathfrak{m}
R[x]_{\mathfrak{m}R[x]} = k(x)$, so $\trdeg k'/k = 1$ and the formula is
satisfied. 
\item  In the second case, $\mathfrak{q}$ properly contains $\mathfrak{m}
R[x]$.
Then $\dim R[x]_{\mathfrak{q}} = \dim R + 1$, but the residue field extension
is finite. So in this case too, the formula is satisfied.
\end{enumerate}
\end{proof}


Now, finally, we have to consider the case where $\mathfrak{p} \subset R[x]$ is
not zero, and we have $S = (R[x]/\mathfrak{p})_{\mathfrak{q}}$ for
$\mathfrak{q} \in \spec R[x]/\mathfrak{p}$ lying over $\mathfrak{m}$.
In this case, $\trdeg S/R = 0$. So we need to prove
\[ \dim S \leq \dim R + \trdeg k'/k.  \]
Let us, by abuse of notation, identify $\mathfrak{q}$ with its preimage in
$R[x]$.
(Recall that $\spec R[x]/\mathfrak{p}$ is canonically identified as a closed
subset of $\spec R[x]$.)
Then we know that
\( \dim ( R[x]/\mathfrak{p})_{\mathfrak{q}}  \)
is the largest chain of primes in $R[x]$ between $\mathfrak{p}, \mathfrak{q}$.
In particular, it is at most $\dim R[x]_{\mathfrak{q}} - \mathrm{height}
\mathfrak{p}
\leq \dim R + 1 - 1 = \dim R$. So the result is clear.
\end{proof} 

In \cite{EGA}, this is used to prove the geometric result that if $\phi:X  \to
Y$ is a morphism of varieties over an algebraically closed field (or a morphism
of finite type between nice schemes), then the local dimension (that is, the
dimension at $x$) of
the fiber $\phi^{-1}(\phi(x))$ is an upper semi-continuous function of $x \in X$.
\subsection{An infinite-dimensional noetherian ring}

We shall now present an example, due to Nagata, of an infinite-dimensional
noetherian ring. Note that such a ring cannot be \emph{local}.

Consider the ring $R=\mathbb{C}[\{x_{i,j}\}_{0 \leq i \leq j}]$ of polynomials in 
infinitely many variables $x_{i,j}$. 
This is clearly an infinite-dimensional ring, but it is also not noetherian.
We will localize it suitably to make it noetherian.

Let $\mathfrak{p}_n \subset R$ be the
ideal  $(x_{1,n}, x_{2,n}, \dots, x_{n,n})$ for all $i \leq n$.
Let $S = R - \bigcup \mathfrak{p}_n$; this is a multiplicatively closed set. 

\begin{theorem}[Nagata] The ring $S^{-1}R$ is noetherian and has infinite
dimension.
\end{theorem} 

We start with
\begin{proposition} 
The ring in the statement of the problem is noetherian.
\end{proposition} 

The proof is slightly messy, so we first prove a few lemmas.

Let $R' = S^{-1}R$ as in the problem statement. We start by proving that every ideal in $R'$ is contained
in one of the $\mathfrak{p}_n$ (which, by abuse of notation, we identify with
their localizations in $R' = S^{-1}R$). 
In particular, the $\mathfrak{p}_n$ are the maximal ideals in $R'$.

\begin{lemma} 
The $\mathfrak{p}_n$ are the maximal ideals in $R'$.
\end{lemma} 
\begin{proof} 
We start with an observation:
\begin{quote}
If $f \neq 0 $, then $f$ belongs to only finitely many $\mathfrak{p}_n$.
\end{quote}
To see this, let us use the following notation. If $M$ is a monomial, we let
$S(M)$ denote the set of subscripts $x_{a,b}$ that occur and $S_2(M)$ the set
of second subscripts (i.e. the $b$'s). 
For $f \in R$, we define $S(f)$ to be the intersection of the $S(M)$ for $M$ a
monomial occurring nontrivially in $f$. Similarly we define $S_2(f)$.

Let us prove:
\begin{lemma} 
$f \in \mathfrak{p}_n$ iff $n \in S_2(f)$. Moreover, $S(f)$ and $S_2(f)$ are
finite for any $f \neq 0$.
\end{lemma} 
\begin{proof} 
Indeed, $f \in \mathfrak{p}_n$ iff every monomial in $f$ is divisible by some
$x_{i,n}, i \leq n$, as $\mathfrak{p}_n  = (x_{i,n})_{i \leq n}$. From this the first assertion is clear. The second too,
because $f$ will contain a nonzero monomial, and that can be divisible by only
finitely many $x_{a,b}$.
\end{proof} 
From this, it is clear how to define $ S_2(f)$ for any element in $R'$,
not necessarily a polynomial in $R$. Namely, it is the set of $n$ such that $f
\in \mathfrak{p}_n$. 
It is now clear, from the second statement of the lemma, that any $f \neq 0$ lies in \emph{only finitely many
$\mathfrak{p}_n$}. In particular, the observation has been proved. 

Let $\mathcal{T} = \left\{ S_2(f), f \in I - 0\right\}$. \emph{I claim that
$\emptyset \in \mathcal{T}$ iff $I = (1)$.} For $\emptyset \in \mathcal{T}$ iff
there is a polynomial lying in no $\mathfrak{p}_n$. Since the union $\bigcup
\mathfrak{p}_n$ is the set of non-units (by construction), we find that the
assertion is clear.


\begin{lemma} 
$\mathcal{T}$ is closed under finite intersections.
\end{lemma} 
\begin{proof} 
Suppose $T_1, T_2 \in \mathcal{T}$. Without loss of generality, there are
\emph{polynomials} $F_1, F_2 \in R$ such that $S_2(F_1) = T_1, S_2(F_2) = T_2$.
A generic linear combination $a F_1 + bF_2$ will involve no cancellation for
$a, b \in \mathbb{C}$, and
the monomials in this linear combination will be the union of those in $F_1$
and those in $F_2$ (scaled appropriately). So $S_2(aF_1 + bF_2) = S_2(F_1) \cap S_2(F_2)$.
\end{proof} 

Finally, we can prove the result that the $\mathfrak{p}_n$ are the only maximal
ideals. Suppose $I$ was contained in no $\mathfrak{p}_n$, and form the set
$\mathcal{T}$ as above. This is a collection of finite sets. Since $I
\not\subset \mathfrak{p}_n$ for each $n$, we find that $n \notin \bigcap_{T \in
\mathcal{T}} T$. This intersection is thus empty. It follows that there is a
\emph{finite} intersection of sets in $\mathcal{T}$
which is empty as $\mathcal{T}$ consists of finite sets. But $\mathcal{T}$ is closed under intersections. There is thus
an element in $I$ whose $S_2$ is empty, and is thus a unit. Thus $I = (1)$.
\end{proof} 

We have proved that the $\mathfrak{p}_n$ are the only maximal ideals. This is
not enough, though. We need:
\begin{lemma} 
$R'_{\mathfrak{p}_n}$ is noetherian for each $n$. 
\end{lemma} 
\begin{proof} 
Indeed, any polynomial involving the variables $x_{a,b}$ for $ b \neq n$ is
invertible in this ring. We see that this ring contains the field
\[ \mathbb{C}(\{x_{a,b}, b \neq n\}),  \]
and over it is contained in the field $\mathbb{C}(\left\{x_{a,b}, \forall
a,b\right\})$. It is a localization of the algebra $\mathbb{C}(\{x_{a,b}, b
\neq n\})[x_{1,n} , \dots, x_{n,n}]$ and is consequently noetherian by
Hilbert's basis theorem.
\end{proof} 

The proof will be completed with:
\begin{lemma} 
Let $R$ be a ring. Suppose every element $x \neq 0$ in the ring belongs to only
finitely many maximal ideals, and suppose that $R_{\mathfrak{m}}$ is noetherian
for each $\mathfrak{m} \subset R$ maximal. Then $R$ is noetherian. 
\end{lemma} 
\begin{proof} 
Let $I \subset R$ be a nonzero ideal. We must show that it is finitely generated. We
know that $I$ is contained in only finitely many maximal ideals $\mathfrak{m}_1
, \dots , \mathfrak{m}_k$. 
At each of these maximal ideals, we know that $I_{\mathfrak{m}_i}$ is finitely
generated. Clearing denominators, we can choose a finite set of generators in
$R$. So we can collect them together and get a finite set $a_1, \dots, a_N
\subset I$
which generate $I_{\mathfrak{m}_i} \subset R_{\mathfrak{m}_i}$ for each $i$. It
is not necessarily true that $J = (a_1, \dots, a_N) = I$, though we do have
$\subset$. However, $I_{\mathfrak{m}} = J_{\mathfrak{m}}$ except at finitely
many maximal ideals $\mathfrak{n}_1, \dots, \mathfrak{n}_M$ because a nonzero
element is a.e. a unit. However, these $\mathfrak{n}_j$ are not among the
$\mathfrak{m}_i$. In particular, for each $j$, there is $b_j \in I -
\mathfrak{n}_j$ as $I \not\subset \mathfrak{n}_j$. Then we find that the ideal
\[ (a_1, \dots, a_N, b_1, \dots, b_M) \subset I \]
becomes equal to $I$ in all the localizations. So it is $I$, and $I$ is finitely generated 

\end{proof} 

We need only see that the ring $R'$ has infinite dimension. But for each $n$, there
is a chain of primes $(x_{1,n}) \subset (x_{1,n}, x_{2,n}) \subset
\dots \subset (x_{1,n}, \dots, x_{n,n})$ of length $n-1$. The supremum of the
lengths is thus infinite.

\subsection{Catenary rings}
 \begin{definition}
   A ring $R$ is \emph{catenary} if given any two primes $\mathfrak{p}\subsetneq \mathfrak{p}'$, any two
   maximal prime chains from $\mathfrak{p}$ to $\mathfrak{p}'$ have the same length.
 \end{definition}
 Nagata showed that there are noetherian domains which are not catenary. We
 shall see that \emph{affine rings}, or rings finitely generated over a field,
 are always catenary.

 \begin{definition}
   If $\mathfrak{p}\in \spec R$, then $\dim \mathfrak{p}:= \dim R/\mathfrak{p}$.
 \end{definition}


\begin{lemma}
   Let $S$ be a $k$-affine domain with $tr.d._k S=n$, and let $\mathfrak{p}\in
	\spec S$ be height one. Then
   $tr.d._k (S/\mathfrak{p})=n-1$.
 \end{lemma}
 \begin{proof}
   \underline{Case 1}: assume $S=k[x_1,\dots, x_n]$ is a polynomial algebra. In this
   case, height 1 primes are principal, so $\mathfrak{p}=(f)$ for some $f$. Say $f$ has positive
   degree with respect to $x_1$, so $f = g_r(x_2,\dots, x_n)x_1^r + \cdots$. We have
   that $k[x_2,\dots, x_n]\cap (f)=(0)$ (just look at degree with respect to $x_1$). It
   follows that $k[x_2,\dots, x_n]\hookrightarrow S/(f)$, so $\bar x_2,\dots, \bar x_n$
   are algebraically independent in $S/\mathfrak{p}$. By $\bar x_1$ is algebraic over $Q(k[\bar
   x_2,\dots, \bar x_n])$ as witnessed by $f$. This, $tr.d._k S/\mathfrak{p}=n-1$.

   \underline{Case 2}: reduction to case 1. Let $R=k[x_1,\dots, x_n]$ be a Noether
   normalization for $S$, and let $\mathfrak{p}_0=\mathfrak{p}\cap R$. Observe that Going Down applies
   (because $S$ is a domain and $R$ is normal). It follows that $ht_R(\mathfrak{p}_0)=ht_S(\mathfrak{p})=1$.
   By case 1, we get that $tr.d. (R/\mathfrak{p}_0)=n-1$. By $(\ast)$, we get that $tr.d.
   R/\mathfrak{p}_0=tr.d. (S/\mathfrak{p})$.
 \end{proof}


 \begin{theorem}
   Any $k$-affine algebra $S$ is catenary (even if $S$ is not a domain). In fact, any
   saturated prime chain from $\mathfrak{p}$ to $\mathfrak{p}'$ has length $\dim \mathfrak{p} - \dim \mathfrak{p}'$. If $S$ is a
   domain, then all maximal ideals have the same height.
 \end{theorem}
 \begin{proof}
   Consider any chain $\mathfrak{p}\subsetneq \mathfrak{p}_1\subsetneq \cdots \subsetneq \mathfrak{p}_r = \mathfrak{p}'$. Then we
   get the chain
   \[
    S/\mathfrak{p} \twoheadrightarrow S/\mathfrak{p}_1 \twoheadrightarrow \cdots \twoheadrightarrow S/\mathfrak{p}_r
    = S/\mathfrak{p}'
   \]
   Here $\mathfrak{p}_i/\mathfrak{p}_{i-1}$ is height 1 in $S/\mathfrak{p}_{i-1}$, so each arrow decreases the
   transcendence degree by exactly 1. Therefore, $tr.d._k S/\mathfrak{p}' = tr.d._k S/\mathfrak{p} -r$.
   \[
    r = tr.d._k S/\mathfrak{p} - tr.d._k S/\mathfrak{p}' = \dim S/\mathfrak{p} - \dim S/\mathfrak{p}' = \dim \mathfrak{p}-\dim \mathfrak{p}'.
   \]
   To get the last statement, take $\mathfrak{p}=0$ and $\mathfrak{p}'=\mathfrak{m}$. Then we get that $ht(\mathfrak{m})=\dim S$.
 \end{proof}
 Note that the last statement fails in general.
 \begin{example}
   Take $S=k\times k[x_1,\dots, x_n]$. Then $ht(0\times k[x_1,\dots, x_n])=0$, but
   $ht\bigl(k\times (x_1,\dots, x_n)\bigr) = n$.
 \end{example}
 But that example is not connected.
 \begin{example}
   $S = k[x,y,z]/(xy,xz)$.
 \end{example}
 But this example is not a domain. In general, for any prime $\mathfrak{p}$ in any ring $S$, we
 have
 \[
    ht(\mathfrak{p}) + \dim \mathfrak{p} \le \dim S.
 \]
 \begin{theorem}
   Let $S$ be an affine algebra, with  minimal primes $ \{\mathfrak{p}_1,\dots, \mathfrak{p}_r\}$. Then the following
   are equivalent.
   \begin{enumerate}
     \item $\dim \mathfrak{p}_i$ are all equal.
     \item $ht(\mathfrak{p})+\dim \mathfrak{p} =\dim S$ for all primes $\mathfrak{p}\in \spec S$. In particular, if $S$
     is a domain, we get this condition.
   \end{enumerate}
 \end{theorem}
 \begin{proof}
   $(1\Rightarrow 2)$ $ht(\mathfrak{p})$ is the length of some saturated prime chain from $\mathfrak{p}$ to
   some minimal prime $\mathfrak{p}_i$. This length is $\dim \mathfrak{p}_i - \dim \mathfrak{p} = \dim S - \dim \mathfrak{p}$ (by
   condition 1). Thus, we get $(2)$.

   $(2\Rightarrow 1)$ Apply (2) to the minimal prime $\mathfrak{p}_i$ to get $\dim \mathfrak{p}_i=\dim S$ for
   all $i$.
 \end{proof}
 We finish with a (non-affine) noetherian domain $S$ with maximal ideals of different
 heights. We need the following fact.\\
 \underline{Fact}: If $R$ is a ring with $a\in R$, then there is a canonical $R$-algebra
 isomorphism $R[x]/(ax-1) \cong R[a^{-1}]$, $x\leftrightarrow a^{-1}$.
 \begin{example}
   Let $\bigl(R,(\mathfrak{p}i)\bigr)$ be a DVR with quotient field $K$. Let $S=R[x]$, and assume
   for now that we know that $\dim S=2$. Look at $\mathfrak{m}_2=(\mathfrak{p}i,x)$ and $\mathfrak{m}_1=(\mathfrak{p}i x-1)$.
   Note that $\mathfrak{m}_1$ is maximal because $S/\mathfrak{m}_1 = K$. It is easy to show that
   $ht(\mathfrak{m}_1)=1$. However, $\mathfrak{m}_2\supsetneq (x)\supsetneq (0)$, so $ht(\mathfrak{m}_2)=2$.
 \end{example}

\subsection{Dimension theory for topological spaces}
\label{subsectiondimension}
The present subsection (which consists mostly of exercises) is a digression   that may illuminate the notion of
Krull dimension.

\begin{definition} 
Let $X$ be a topological space.\footnote{We do not include the empty space.} Recall that $ X$ is
\textbf{irreducible} if  cannot be written as the union of
two proper closed subsets $F_1, F_2 \subsetneq X$.

We say that a subset of $X$ is irreducible if it is irreducible with respect
to the induced topology.
\end{definition} 

In general, this notion is not valid from the topological spaces familiar from
analysis. For instance:

\begin{exercise} 
Points are the only irreducible subsets of $\mathbb{R}$.
\end{exercise} 

Nonetheless, irreducible sets behave very nicely with respect to certain
operations. As you will now prove, if $U \subset X$ is an open subset, then
the irreducible closed subsets of $U$ are in bijection with the irreducible
closed subsets of $X$ that intersect $U$. 
\begin{exercise} \label{irredifeveryopenisdense}
A space is irreducible if and only if every open set is dense, or
alternatively if every open set is connected.
\end{exercise} 

\begin{exercise} 
Let $X$ be a space, $Y \subset X$ an irreducible subset. Then 
$\overline{Y} \subset X$ is irreducible.
\end{exercise} 

\begin{exercise} 
Let $X$ be a space, $U \subset X$ an open subset. 
Then the map $Z \to Z \cap U$ gives a bijection between the irreducible
closed subsets of $X$ meeting $U$ and the irreducible closed subsets of $U$.
The inverse is given by $Z' \to \overline{Z'}$.
\end{exercise} 

As stated above, the  notion of irreducibility is useless for spaces
like manifolds. In fact, by \rref{irredifeveryopenisdense}, a
Hausdorff space cannot be irreducible unless it consists of one point.
However, for the highly non-Hausdorff spaces encountered in algebraic geometry, this notion is very
useful.

Let $R$ be a commutative ring, and $X = \spec R$.

\begin{exercise} 
A closed subset $F \subset \spec R$ is irreducible if and only if it can be
written in the form $F = V(\mathfrak{p})$ for $\mathfrak{p} \subset R$ prime.
In particular, $\spec R$ is irreducible if and only if $R$ has one minimal
prime. 
\end{exercise} 

In fact, spectra of rings are particularly nice: they are \textbf{sober
spaces.}
\begin{definition} 
A space $X$ is called \textbf{sober} if to every irreducible closed $F \subset
X$, there is a unique point $\xi \in F$ such that $F = \overline{ \left\{\xi\right\}}$. 
This point is called the \textbf{generic point.}
\end{definition} 

\begin{exercise} 
Check that if $X$ is any topological space and $\xi  \in X$, then the closure
$\overline{\left\{\xi\right\}}$ of the point $\xi$ is irreducible.
\end{exercise} 

\begin{exercise} 
Show that $\spec R$ for $R$ a ring is sober.
\end{exercise} 

\begin{exercise} 
Let $X$ be a space with a cover $\left\{X_\alpha\right\}$ by open subsets,
each of which is a sober space. Then $X$ is a sober space. (Hint: any
irreducible closed subset must intersect one of the $X_\alpha$, so is the
closure of its intersection with that one.)
\end{exercise} 

We now come to the main motivation of this subsection, and the reason for its
inclusion here.

\begin{definition} 
Let $X$ be a topological space. Then the \textbf{dimension} (or
\textbf{combinatorial dimension}) of $X$ is the maximal $k$ such that a chain
\[ F_0 \subsetneq F_1 \subsetneq \dots \subsetneq F_k \subset X  \]
with the $F_i$ irreducible, exists. This number is denoted $\dim X$ and may be
infinite.
\end{definition} 

\begin{exercise} 
What is the Krull dimension of $\mathbb{R}$?
\end{exercise} 

\begin{exercise} 
Let $X = \bigcup X_i$ be the finite union of subsets $X_i \subset X$. 

\end{exercise} 

\begin{exercise} 
Let $R$ be a ring. Then $\dim \spec R$ is equal to the Krull dimension of $R$. 
\end{exercise} 

Most of the spaces one wishes to work with in standard algebraic geometry have a
strong form of compactness. Actually, compactness is the wrong word, since the
spaces of algebraic geometry are not Hausdorff.

\begin{definition} 
A space is \textbf{noetherian} if every descending sequence of closed subsets
$F_0 \supset F_1 \supset \dots$ stabilizes.
\end{definition} 

\begin{exercise} 
If $R$ is noetherian, $\spec R$ is noetherian as a topological space.
\end{exercise} 


