\chapter{Dimension theory}


Today, we'd like to start talking about dimension theory. But first we need a
little something else.

\subsection{Some definitions}

Let $R$ be a commutative ring, $M$ an $R$-module. 

\begin{definition} 
$M$ is \textbf{simple} if $M \neq 0$ and $M$ has no nontrivial submodules.
\end{definition} 

\begin{definition} 
$M$ is \textbf{finite length} if there is a finite filtration $0 \subset M^0
\subset \dots \subset M^n = M$ where each $M^i/M^{i-1}$ is simple.
\end{definition} 

\begin{remark} 
$M$ is simple iff it is isomorphic $R/\mathfrak{m}$ for $\mathfrak{m} \subset
R$ an ideal. Why? Well, it must contain a cyclic submodule generated by $x \in
M - \left\{0\right\}$. So it must contain a submodule isomorphic to $R/I$, and
simplicity implies that $M \simeq R/I$ for some $I$. If $I$ is not maximal,
then we will get a nontrivial submodule of $R/I$. Conversely, it's easy to see
that $R/\mathfrak{m}$ is simple for $\mathfrak{m}$ maximal.
\end{remark} 

\begin{proposition} 
$M$ is finite length iff $M$ is both noetherian and artinian.
\end{proposition} 
\begin{proof} 
Any simple module is obviously both noetherian and artinian---there are two
submodules. So if $M$ is finite length, then the finite filtration with simple
quotients implies that $M$ is noetherian and artinian, since these two
properties are stable under extensions.

Suppose $M \neq 0$ is noetherian and artinian. Let $M_1 \subset M$ be a minimal
nonzero submodule, possible by artinianness. This is necessarily simple. Then we have a filtration
\[ 0 = M_0 \subset M_1.  \]
If $M_1 = M$, then the filtration goes up to $M$, and we have that $M$ is of
finite length. If not, find a minimal $M_2$ containing $M_1$; then the quotient
$M_2/M_1$ is simple. We have the filtration
\[ 0 = M_0 \subset M_1 \subset M_2,  \]
which we can keep continuing until at some point we hit $M$. Note that since
$M$ is noetherian, we cannot continue this strictly ascending chain forever. 
\end{proof} 

\begin{proposition} 
In this case, the length of the filtration is well-defined. That is, any two filtrations
on $M$ with simple quotients have the same length.
\end{proposition} 
\begin{definition} 
This number is called the \textbf{length} of $M$ and is denoted $\ell(M)$.
\end{definition} 
\begin{proof} 
Let us introduce a temporary definition: $l(M)$ is the length of the
\emph{minimal} filtration on $M$. A priori, we don't know that $\ell(M)$ makes
any sense. \textbf{We will show that any filtration is of length
$l(M)$.} This is the proposition in another form.

The proof of this claim is by induction on $l(M)$. Suppose we have a filtration
\[ 0 = M_0 \subset M_1 \subset \dots \subset M_n = M  \]
with simple quotients. We'd like to show that $n  = l(M)$. By definition of
$l(M)$, there is another filtration
\[ 0 = N_0 \subset \dots \subset N_{l(M)} = M.  \]
If $l(M) = 0,1$, then $M$ is zero or simple, which will imply that $n=0,1$
respectively. So we can assume $l(M)  \geq 2$.
There are two cases:
\begin{enumerate}
\item $M_{n-1} = N_{l(M) -1 } $. Then $M_{n-1} = N_{l(M)-1}$ has $l$ at most
$l(M)-1$. Thus by the inductive hypothesis any two filtrations on $M_{n-1}$
have the same length, so $n-1 = l(M) -1$ implying what we want. 
\item We have $M_{n-1} \cap N_{l(M) - 1} \subsetneq M_{n-1}, N_{l(M)-1}$. 
Call this intersection $K$. 

Now we can replace the filtrations of $M_{n-1}, N_{l(M)-1}$ such that the next
term after that is $K$, because any two filtrations on these proper submodules
have the same length. So we find that $n-1 = l(K) +1$ and $l(M)-1 = l(K)+1$ by
the inductive hypothesis. This implies what we want.
\end{enumerate}
\end{proof} 


\subsection{Introduction to dimension theory}

Let $R$ be a   ring. 
\begin{question} 
What is a good definition for $\dim(R)$? Actually, more generally, we want the dimension at a
point.
\end{question} 

Geometrically, think of $\spec R$, for any ring; pick some point corresponding to a maximal
ideal $\mathfrak{m} \subset R$. We want to define the \textbf{dimension of $R$}
at $\mathfrak{m}$. This is to be thought of kind of like ``dimension over the
complex numbers,'' for algebraic varieties defined over $\mathbb{C}$. But it
should be purely algebraic. 

What might you do?


Here's an idea. For a topological space $X$ to be $n$-dimensional at $x \in X$, this should
mean that there are $n$ coordinates at the point $x$. The point $x$ is defined
by the zero locus of $n$ points on the space.

\begin{definition}[Proposal] We could try defining $\dim_{\mathfrak{m}} R$
to be the number of generators of $\mathfrak{m}$.
\end{definition} 

This is a bad definition, as $\mathfrak{m}$ may not have the same number of
generators as $\mathfrak{m}R_{\mathfrak{m}}$.	We want our definition to be
local.
So this leads us to:
\begin{definition} 
If $R$ is a (noetherian) \emph{local} ring with maximal ideal $\mathfrak{m}$, then the \textbf{embedding dimension} of $R$ is
the minimal number of generators for $\mathfrak{m}$.
\end{definition} 

By Nakayama's lemma, this is the minimal number of generators of
$\mathfrak{m}/\mathfrak{m}^2$, or the $R/\mathfrak{m}$-dimension of that vector
space.
However, this isn't going to coincide with the dimension of an algebraic
variety.

\begin{example} 
Let $R = \mathbb{C}[t^2, t^3] \subset \mathbb{C}[t]$, which is the coordinate
ring of a cubic curve. Let us localize at the prime ideal $\mathfrak{p} = (t^2,
t^3)$: we get $R_{\mathfrak{p}}$. 

Now $\spec R$ is singular at the origin. In fact, as a result, $\mathfrak{p}
R_{\mathfrak{p}} \subset R_{\mathfrak{p}}$ needs two generators, but the
variety it corresponds to is one-dimensional.
\end{example} 

So the embedding dimension is the smallest dimension into which you can embed
$R$ into a smooth space.
But for singular varieties this is not the dimension we want. 

Well, we can consider the sequence of finite-dimensional vector spaces
\[ \mathfrak{m}^k/\mathfrak{m}^{k+1}.  \]
Computing these dimensions gives some invariant that describes the local
geometry of $\spec R$.

\begin{example} 
Consider the local ring $(R, \mathfrak{m}) = \mathbb{C}[t]_{(t)}$. Then $\mathfrak{m} = (t)$ and
$\mathfrak{m}^k/\mathfrak{m}^{k+1}$ is one-dimensional, generated by $t^k$.
\end{example} 

\begin{example} 
Consider $R = \mathbb{C}[t^2, t^3]_{(t^2, t^3)}$, the local ring of $y^2 = x^3$
at zero. Then $\mathfrak{m}^n$ is generated by $t^{2n}, t^{2n+1}, \dots$.
$\mathfrak{m}^{n+1}$ is generated by $t^{2n+2}, t^{2n+3}, \dots$. So the
quotients all have dimension two. The dimension of these quotients is a little
larger than we expected, but they don't grow.
\end{example} 

\begin{example} 
Consider $R = \mathbb{C}[x,y]_{(x,y)}$. Then $\mathfrak{m}^k$ is generated by
polynomials homogeneous in degree $k$. So $\mathfrak{m}^k/\mathfrak{m}^{k+1}$
has dimensions that \emph{grow} in $k$. This is a genuinely two-dimensional
example. 
\end{example} 

This is the difference that we want to quantify to be the dimension.

\begin{proposition} 
Let $(R, \mathfrak{m})$ be a local noetherian ring. Then there exists a
polynomial
$f \in \mathbb{Q}[t]$ such that
\[ \ell(R/\mathfrak{m}^n) = \sum_{i=0}^{n-1} \dim
\mathfrak{m}^i/\mathfrak{m}^{i+1} =  f(n)\quad \forall n \gg 0.  \]

Moreover, $\deg f \leq \dim \mathfrak{m}/\mathfrak{m}^2$.
\end{proposition} 


Note that this polynomial is well-defined, as any two polynomials agreeing for large $n$
coincide. Note also that $R/\mathfrak{m}^n$ is artinian so of finite length,
and that we have used the fact that the length is additive for short exact
sequences. We would have liked to write $\dim R/\mathfrak{m}^n$, but we can't,
in general, so we use the substitute of the length. 

Based on this, we define
\begin{definition} 
The \textbf{dimension} of $R$ is the degree of the polynomial $f$ above.
\end{definition} 

\begin{example} 
Consider $R = \mathbb{C}[x_1, \dots, x_n]_{(x_1, \dots, x_n)}$. What is the
polynomial $f$ above? Well, $R/\mathfrak{m}^k$ looks like the set of
polynomials of degree $<k$ in $\mathbb{C}$. The dimension as a vector space is
given by some binomial coefficient $\binom{n+k-1}{n}$. This is a polynomial in
$k$ of degree $n$.  So $R$ is $n$-dimensional. 
Which is what we wanted.
\end{example} 

\begin{example} 
Let $R$ be a DVR. Then $\mathfrak{m}^k/\mathfrak{m}^{k+1}$ is of length one for
each $k$. So $R/\mathfrak{m}^k$ has length $k$. Thus we can take $f(t) = t$ so
$R$ has dimension one.
\end{example} 

Now we have to prove the proposition, i.e. that there is always such a
polynomial.
\begin{proof} 
Let $S = \bigoplus_n  \mathfrak{m}^n/\mathfrak{m}^{n+1}$. Then $S$ has a
natural grading, and in fact it is a graded ring in a natural way from the map 
\[ \mathfrak{m}^{n_1} \times \mathfrak{m}^{n_2} \to \mathfrak{m}^{n_1 + n_2}.  \]
(It is the associated graded ring of the $\mathfrak{m}$-adic filtration.)
Note that $S_0 = R/\mathfrak{m}$ is a field.

Choose $n$ generators $x_1, \dots, x_n \in \mathfrak{m}$, where $n$ is what we
called the embedding dimension of $R$. So these $n$ generators give generators
of $S_1$ as an $S_0$-vector space. In fact, they generate $S$ as an
$S_0$-algebra because $S$ is generated by degree one terms over $S_0$.
So $S$ is a graded quotient of the polynomial ring $\kappa[t_1, \dots, t_n]$
for $\kappa = R/\mathfrak{m}$.
Note that $\ell(R/\mathfrak{m}^a)  = \dim_{\kappa}(S_0) + \dots +
\dim_{\kappa}(S_{a-1})$ for any $a$, thanks to the filtration. 

It will now suffice to prove the following more general proposition.
\begin{proposition} 
Let $M$ be any finitely generated graded module over the polynomial ring $\kappa[x_1, \dots, x_n]$. Then
there exists a polynomial $f_M \in \mathbb{Q}[t]$ of degree $ \leq n$, such that
\[ f_M(t) = \sum_{s \leq t} \dim M_s \quad t \gg 0.   \]
\end{proposition} 
Applying this to $M = S$ will give the desired result. We can forget about
everything else, and look at this problem over graded polynomial rings.

This function is called the \textbf{Hilbert function}.

\begin{proof} 
Note that if we have an exact sequence of gaded modules over the polynomial
ring,
\[ 0 \to M' \to M \to M'' \to 0,   \]
and $f_{M'}, f_{M''}$ exist as polynomials, then $f_M$ exists and 
\[ f_M = f_{M'} + f_{M''}.  \] This is obvious from the definitions. We will
induct on $n$. 

If $n  = 0$, then $M$ is a finite-dimensional graded vector space over
$\kappa$, and the grading must be concentrated in finitely many degrees. Thus
the result is evident as $f_{M}(t)$ will just equal $\dim M$ (which will be the
appropriate dimension for $t \gg 0$). 

Suppose $n > 0$. Let $x$ be one of the variables $x_1, \dots, x_n$. Then
consider
\[ 0 \subset \ker( x: M \to M) \subset \ker (x^2: M \to M) \subset \dots.  \]
This must stabilize by noetherianness at some $M' \subset M$. Each of the
quotients $\ker( x^i)/\ker (x^{i+1})$ is a finitely generated module over $\kappa[x_1, \dots,
x_n]/(x)$, which is a smaller polynomial ring.  So each of these subquotients
$\ker (x^i)/\ker (x^{i+1})$ has a Hilbert function of degree $\leq n-1$. 

Thus $M'$ has a Hilbert function which is the sum of the Hilbert functions of
these subquotients. In particular, $f_{M'}$ exists. If we show that $f_{M/M'}$
exists, then $f_M$ necessarily exists. So we might as well show that the
Hilbert function $f_M$ exists when $x$ is  a non-zerodivisor on $M$. 

We are out of time, so next time we will finish the proof. 
\end{proof} 
\end{proof} 
\lecture{10/29}

We started last time talking about the dimension theory about local noetherian
rings. 

\subsection{Hilbert polynomials}
Last time, we were in the middle of the proof of a lemma. 

Suppose $S = k[x_1, \dots, x_n]$ is a polynomial ring over a field $k$. It is a
graded ring; the $m$-th graded piece is the set of polynomials homogeneous of
degree $m$. Let $M$ be a finitely generated graded $S$-module. 

\begin{definition} 
The \textbf{Hilbert function} $H_M$ of $M$ is defined via $H_M(m) = \dim_k
M_m$. This is always finite for $M$ a finitely generated graded $S$-module, as $M$ is a
quotient of copies of $S$ (or twisted pieces).

Similarly, we define
\[ H_M^+(m) = \sum_{m' \leq m} H_M(m').  \]
This measures the dimension of $\deg m$ and below.
\end{definition} 

What we were proving last time was that:

\begin{proposition} 
There exist polynomials $f_M(t), f_M^+(t) \in \mathbb{Q}[t]$ such that $f_M(t) = H_M(t)$ and
$f_M^+(t) = H_M^+(t)$ for sufficiently large $t$. Moreover, $\deg f_M \leq n-1,
\deg f_M^+ \leq n$.
\end{proposition} 

In other words, the Hilbert functions eventually become polynomials. 

These polynomials don't generally have integer coefficients, but they are
close, as they take integer values at large values.
In fact, they take integer values everywhere.

\begin{remark} 
A function $f: \mathbb{Z} \to \mathbb{Z}$ is polynomial iff 
\[ f(t) = \sum_n c_n \binom{t}{n}, \quad c_n \in \mathbb{Z}.  \]
So $f$ is a $\mathbb{Z}$-linear function of binomial coefficients. 
\end{remark} 
\begin{proof} 
Note that the set $\left\{\binom{t}{n}\right\}$ forms a basis for the set of
polynomials, that is $\mathbb{Q}[t]$. It is clear that $f(t)$ can be written as
$\sum c_n \binom{t}{n}$ for the $c_n \in \mathbb{Q}$. By looking at the
function $\Delta f (t) = f(t) - f(t-1)$ (which takes values in $\mathbb{Z}$) and the fact that $\Delta \binom{t}{n}
= \binom{t}{n-1}$, it is easy to see that the $c_n \in \mathbb{Z}$ by induction
on the degree. 
It is also easy to see that the binomial coefficients take values in
$\mathbb{Z}$.
\end{proof} 

\begin{remark} 
The same remark applies if $f$ is polynomial and $f(t) \in \mathbb{Z}$ for $t
\gg 0$, by the same argument. It follows that $f(t) \in \mathbb{Z}$ for all $t$. 
\end{remark} 

Let us now prove the proposition.
\begin{proof} 
I claim, first, that the polynomiality of $H_M(t)$ (for $t$ large) is
equivalent to that of $H_M^+(t)$ (for $t$ large). This is because $H_M$ is the successive
difference of $H_M^+$, i.e. $H_M(t) = H_M^+(t) - H_M^+(t-1)$. 
Similarly
\[ H_M^+(t) = \sum_{t' \leq t} H_M(t), \]
and the successive sums of a polynomial form a polynomial. 

So if $f_M$ exists as in the proposition, then $f_M^+$ exists. Let us now show that $f_M$ exists.
Moreover, we will show that $f_M$ has degree $\leq n-1$, which will prove the
result, since $f_M^+$ has degree one higher.

Induction on $n$. 
When $n=0$, this is trivial, since $H_M(t) = 0$ for $t \gg 0$. In the general
case, we reduced to the case of $M$ having no $x_1$-torsion. The argument for
this reduction can be found in the previous lecture. 

So $M$ has a filtration
\[ M \supset x_1 M \supset x_1^2 M \supset \dots  \]
which is an exhaustive filtration of $M$ in that nothing can be divisible by
powers of $x_1$ over and over, for considerations of degree. Multiplication by
$x_1$ raises the degree by one. This states that $\bigcap x_1^m M = 0$. 

Let $N = M/x_1 M \simeq x_1^m M/x_1^{m+1} M$ since $M \stackrel{x_1}{\to} M$ is
injective. Now $N$ is a graded module over $k[x_2, \dots, x_n]$, and by the
inductive hypothesis on $n$
So there is a polynomial $f_N^+$ of degree $\leq n-1$ such that
\[ f_N^+(t) = \sum_{t' \leq t} \dim N_{t'}, \quad t \gg 0.  \]

Let's look at $M_t$, which has a finite filtration 
\[ M_t \supset (x_1 M)_t \supset (x_1^2 M)_t \supset \dots  \]
which has successive quotients that are the  graded pieces of $N \simeq
M/x_1 M \simeq x_1 M/x_1^2 M \simeq \dots$ in dimensions $t, t-1, \dots$. We
find that
\[ (x_1^2 M)_t/(x_1^3 M)_t \simeq N_{t-2},  \]
for instance. We find that
\[ \dim M_t = \dim N_t + \dim N_{t-1} + \dots  \]
which implies that $f_M(t)$ exists and coincides with $f_N^+$. 
\end{proof} 

\subsection{Back to dimension theory}
\begin{example} 
Let $R $ be a local noetherian ring with maximal ideal $\mathfrak{m}$. Then we
have the module $M = \bigoplus \mathfrak{m}_1^k/\mathfrak{m}_1^{k+1}$ over the ring
$(R/\mathfrak{m})[x_1, \dots, x_n]$ where $x_1, \dots, x_n$ are generators of
$\mathfrak{m}$.

The upshot is that 
\[ f_M^+(t) = \ell(R/\mathfrak{m}^t), \quad t \gg 0.  \]
This is a polynomial of degree $\leq n$.
\end{example} 

\begin{definition} 
The \textbf{dimension} of $R$ is the degree of $f_M^+$. 
\end{definition} 

\begin{remark} 
As we have seen, the dimension is at most the number of generators of
$\mathfrak{m}$. So the dimension is at most the embedding dimension.
\end{remark} 


\begin{definition} 
If $R$ is local noetherian, $N$ a finite $R$-module, define 
\[ M = \bigoplus \mathfrak{m}^a N / \mathfrak{m}^{a+1} N, \]
which is a module over the associated graded ring $\bigoplus
\mathfrak{m}^a/\mathfrak{m}^{a+1}$, which in turn is a quotient of a polynomial
ring. It too has a Hilbert polynomial. We say that the \textbf{dimension of
$N$} is the degree of the Hilbert polynomial $f_{M}^+$. Evaluated at $t \gg 0$,
this gives the length $\ell(N/ \mathfrak{m}^t N)$.
\end{definition} 

\begin{proposition} 
$\dim R$ is the same as $\dim R/\rad R$.
\end{proposition} 
I.e., the dimension doesn't change when you kill off nilpotent elements, which
is what you would expect, as nilpotents don't affect $\spec (R)$.
\begin{proof} 
For this, we need a little more information about Hilbert functions.
We thus digress substantially. 

\begin{proposition} 
Suppose we have an exact sequence
\[ 0 \to M' \to M \to M'' \to 0 \]
of gaded modules over a polynomial ring $k[x_1, \dots, x_n]$.	Then
\[ f_M(t) = f_{M'}(t) + f_{M''}(t), \quad  f_M^+(t) = f_{M'}^+(t) +
f_{M''}^+(t). \]
As a result, $\deg f_M = \max \deg f_{M'}, \deg f_{M''}$.
\end{proposition} 
\begin{proof} The first part is obvious as the dimension is additive on vector
spaces. The second part follows because Hilbert functions have nonnegative
leading coefficients.
\end{proof} 
In particular, 
\begin{corollary} 
Say we have an exact sequence
\[ 0 \to N' \to N \to N'' \to 0  \]
of finite $R$-modules. Then $\dim N = \max (\dim N', \dim N'')$.
\end{corollary} 
\begin{proof} 
We have an exact sequence
\[ 0 \to K \to N/\mathfrak{m}^t N \to N''/\mathfrak{m}^t N'' \to 0  \]
where $K$ is the kernel. Here $K = (N' + \mathfrak{m}^t N)/ \mathfrak{m}^t N
= N'/( N' \cap \mathfrak{m}^t N)$. This is not quite $N'/\mathfrak{m}^t N'$,
but it's pretty close. 
We have a surjection
\[ N'/\mathfrak{m}^t N \twoheadrightarrow N'/(N' \cap \mathfrak{m}^t N) = K. \]
In particular, 
\[ \ell(K) \leq \ell(N'/\mathfrak{m}^t N').  \]
On the other hand, we have the Artin-Rees lemma, which gives an inequality in
the opposite direction. We have a containment
\[ \mathfrak{m}^t N' \subset N' \cap \mathfrak{m}^t N \subset
\mathfrak{m}^{t-c} N'  \]
for some $c$. This implies that $\ell(K) \geq \ell( N'/\mathfrak{m}^{t-c} N')$. 

Define $M = \bigoplus \mathfrak{m}^t N/\mathfrak{m}^{t+1} N$, and define $M',
M''$ similarly in terms of $N', N''$. Then we have seen that 
\[ \boxed{f_M^+(t-c) \leq \ell(K) \leq f_M^+(t).}  \]
We also know that the length of $K$ plus the length of $N''/\mathfrak{m}^t N''$
is $f_M^+(t)$, i.e.
\[ \ell(K) + f_{M''}^+(t) = f_M^+(t).  \]
Now the length of $K$ is a polynomial in $t$ which is pretty similar to
$f_{M'}^+$, in that the leading coefficient is the same. So we have an
approximate equality $f_{M'}^+(t) + f_{M''}^+(t) \simeq f_M^+(t)$. This implies the
result since the degree of $f_M^+$ is $\dim N$ (and similarly for the others). 
\end{proof} 

Finally, let us return to the claim about dimension and nilpotents. Let $R$ be
a local noetherian ring and $I = \rad (R)$. Then $I$ is a finite $R$-module. In
particular, $I$ is nilpotent, so $I^n  = 0$ for $n \gg 0$. We will show that
\[ \dim R/I = \dim R/I^2 = \dots \]
which will imply the result, as eventually the powers become zero. 

In particular, we have to show for each $k$, 
\[ \dim R/I^k  = \dim R/I^{k+1}.  \]
There is an exact sequence
\[ 0 \to I^k/I^{k+1} \to R/I^{k+1} \to R/I^k \to 0.  \]
The dimension of these rings is the same thing as the dimensions as
$R$-modules. So we can use this short exact sequence of modules. By the
previous result, we are reduced to showing that
\[ \dim I^k/I^{k+1} \leq \dim R/I^k.  \]
Well, note that $I$ kills $I^k/I^{k+1}$. In particular, $I^k/I^{k+1}$ is a finitely generated
$R/I^k$-module. There is an exact sequence
\[ \bigoplus_N R/I^k \to I^k/I^{k+1} \to 0  \]
which implies that $\dim I^k/I^{k+1} \leq \dim \bigoplus_N R/I^k = \dim R/I^k$.
\end{proof} 

\begin{example} 
Let $\mathfrak{p} \subset \mathbb{C}[x_1, \dots, x_n]$ and let $R =
(\mathbb{C}[x_1,\dots, x_n]/\mathfrak{p})_{\mathfrak{m}}$ for some maximal
ideal $\mathfrak{m}$. What is $\dim R$? 
What does dimension mean for coordinate rings over $\mathbb{C}$?

Recall by the Noether normalization theorem that there exists a polynomial ring
$\mathbb{C}[y_1, \dots, y_m]$ contained in $S=\mathbb{C}[x_1,\dots,
x_n]/\mathfrak{p}$ and $S$ is a finite integral extension over this polynomial ring. 
We claim that 
\[ \dim R = m.  \]
There is not sufficient time for that today. 
\end{example} 


\lecture{11/1}

Last time, we were talking about the dimension theory of local noetherian rings.

\subsection{Recap}

Let $(R, \mathfrak{m})$ be a local noetherian ring. Let $M$ be a finitely generated
$R$-module. 	We defined the \textbf{Hilbert polynomial} of $M$ to be the
polynomial which evaluates at $t \gg 0$ to $\ell(M/\mathfrak{m}^tM)$. We proved
last time that such a polynomial always exists, and called its degree the
\textbf{dimension of $M$}. More accurately, we shall start calling it $\dim
\supp M$. 

Recall that $\supp M = \left\{\mathfrak{p}: M_{\mathfrak{p}}\neq 0\right\}$. To
make sense of this, we must show:

\begin{proposition} 
$\dim M$ depends only on $\supp M$.
\end{proposition} 

In fact, we shall show:

\begin{proposition} 
$\dim M = \max_{\mathfrak{p} \in \supp M} \dim R/\mathfrak{p}$.
\end{proposition} 
\begin{proof} 
There is a finite filtration 
\[ 0 = M_0 \subset M_1 \subset \dots \subset M_m = M,  \]
such that each of the successive quotients is isomorphic to $R/\mathfrak{p}_i$
for some prime ideal $\mathfrak{p}_i$. But if you have a short exact sequence
of modules, the dimension in the middle is the maximum of the dimensions at the
two ends. Iterating this, we see that the dimension of $M$ is the sup of the
dimension of the successive quotients. But the $\mathfrak{p}_i$'s that occur
are all in $\supp M$, so we find 
\[ \dim M = \max_{\mathfrak{p}_i} R/\mathfrak{p}_i \leq \max_{\mathfrak{p} \in \supp M} \dim R/\mathfrak{p}.  \]
We must show the reverse inequality. But fix any prime $\mathfrak{p} \in \supp
M$. Then $M_{\mathfrak{p}} \neq 0$, so one of the $R/\mathfrak{p}_i$ localized
at  $\mathfrak{p}$ must be nonzero, as localization is an exact functor. Thus
$\mathfrak{p}$ must contain some $\mathfrak{p}_i$. So $R/\mathfrak{p}$ is a
quotient of $R/\mathfrak{p}_i$. In particular,
\[ \dim R/\mathfrak{p} \leq \dim R/\mathfrak{p}_i.  \]
\end{proof} 

Having proved this, we throw out the notation $\dim M$, and henceforth write
instead $\dim \supp M$.

\subsection{The dimension of an affine ring} Last time, we made a claim. If $R$
is a domain and a finite module over a polynomial ring $k[x_1, \dots, x_n]$,
then $R_{\mathfrak{m}}$ for any maximal $\mathfrak{m} \subset R$ has dimension
$n$. This connects the dimension with the transcendence degree. 

First, let us talk about finite extensions of rings. Let $R$ be a commutative
ring and let $R \to R'$ be a morphism that makes $R'$ a finitely generated $R$-module (in
particular, integral over $R$). Let $\mathfrak{m}' \subset R'$ be maximal. Let
$\mathfrak{m}$ be the pull-back to $R$, which is also maximal (as $R \to R'$ is
integral). 
Let $M$ be a finitely generated $R'$-module, hence also a finitely generated $R$-module. 

We can look at $M_{\mathfrak{m}}$ as an $R_{\mathfrak{m}}$-module or
$M_{\mathfrak{m}'}$ as an $R'_{\mathfrak{m}'}$-module. Either of these will be
finitely generated. 

\begin{proposition} 
$\dim \supp M_{\mathfrak{m}}  \geq \dim \supp M_{\mathfrak{m}'}$.
\end{proposition} 
Here $M_{\mathfrak{m}}$ is an $R_{\mathfrak{m}}$-module, $M_{\mathfrak{m}'}$ is
an $R'_{\mathfrak{m}'}$-module. 

\begin{proof} 
Consider $R/\mathfrak{m} \to R'/\mathfrak{m} R' \to R'/\mathfrak{m}'$. Then we
see that $R'/\mathfrak{m} R'$ is a finite $R/\mathfrak{m}$-module, so a
finite-dimensional $R/\mathfrak{m}$-vector space. In particular,
$R'/\mathfrak{m} R'$ is of finite length as an $R/\mathfrak{m}$-module, in
particular an artinian ring. It is thus a product of local artinian rings.
These artinian rings are the localizations of $R'/\mathfrak{m}R'$ at ideals of
$R'$ lying over $\mathfrak{m}$. One of these ideals is $\mathfrak{m}'$. 
So in particular
\[ R'/\mathfrak{m}R \simeq R'/\mathfrak{m}'\times \mathrm{other \ factors}.  \]
The nilradical of an artinian ring being nilpotent, we see that
$\mathfrak{m}'^c R'_{\mathfrak{m}'} \subset \mathfrak{m} R'_{\mathfrak{m}}$ for
some $c$.

OK, I'm not following this---too tired. Will pick this up someday.
\end{proof} 


\begin{proposition} 
$\dim \supp M_{\mathfrak{m}} = \max_{\mathfrak{m}' \mid \mathfrak{m}} \dim
\supp M_{\mathfrak{m}'}$.
\end{proposition} 

This means $\mathfrak{m}'$ lies over $\mathfrak{m}$.
\begin{proof} 
Done similarly, using artinian techniques. I'm kind of tired.
\end{proof} 

\begin{example} 
Let $R' = \mathbb{C}[x_1, \dots, x_n]/\mathfrak{p}$. Noether normalization says
that there exists a finite injective map $\mathbb{C}[y_1, \dots, y_a] \to R'$.
The claim is that
\[ \dim R'_{\mathfrak{m}} =a  \]
for any maximal ideal $\mathfrak{m} \subset R'$. We are set up to prove a
slightly weaker definition. In particular (see below for the definition of the
dimension of a non-local ring), by the proposition, we
find the weaker claim
\[ \dim R' = a,  \]
as the dimension of a polynomial ring $\mathbb{C}[y_1, \dots, y_a]$ is $a$.
(\textbf{I don't think we have proved this yet.})
\end{example} 


\subsection{Dimension in general}
\begin{definition} 
If $R$ is a noetherian ring, we define $\dim (R) = \sup_{\mathfrak{p}}
R_{\mathfrak{p}}$ for $\mathfrak{p} \in \spec(R)$ maximal. This may be infinite. The
localizations can grow arbitrarily large in dimension, but these examples are
kind of pathological.
\end{definition} 

\subsection{A topological characterization} We now want a topological
characterization of dimension. So, first, we want to study how dimension
changes as we do things to a module. Let $M$ be a finitely generated $R$-module over a local
noetherian ring $R$. Let $x \in \mathfrak{m}$ for $\mathfrak{m}$ as the maximal
ideal.
You might ask
\begin{quote}
What is the relation between $\dim \supp M$ and $\dim \supp M/xM$?
\end{quote}
Well, $M$ surjects onto $M/xM$, so we have the inequality $\geq$. But we think
of dimension as describing the number of parameters you need to describe
something. The number of parameters shouldn't change too much with going from
$M$ to $M/xM$. Indeed, as one can check,
\[ \supp M/xM = \supp M \cap V(x)  \]
and intersecting $\supp M$ with the ``hypersurface'' $V(x)$ should shrink the
dimension by one. 


We thus make:
\begin{prediction}
\[ \dim \supp M/xM = \dim \supp M - 1.  \]
\end{prediction}
Obviously this is not always true, e.g. if $x$ acts by zero on $M$. But we want
to rule that out. 
Under reasonable cases, in fact, the prediction is correct:

\begin{proposition} 
Suppose $x \in \mathfrak{m}$ is a nonzerodivisor on $M$. Then 
\[ \dim \supp M/xM = \dim \supp M - 1.  \]
\end{proposition} 
\begin{proof} 
To see this, we look at Hilbert polynomials. Let us consider the exact sequence
\[ 0 \to xM \to M \to M/xM \to 0  \]
which leads to an exact sequence for each $t$,
\[ 0 \to xM/(xM \cap \mathfrak{m}^t M) \to M/\mathfrak{m}^t M \to M/(xM  +
\mathfrak{m}^t M) \to 0 . \]
For $t$ large, the lengths of these things are given by Hilbert polynomials,
as the thing on the right is $M/xM \otimes_R R/\mathfrak{m}^t$. 
We have
\[ f_M^+(t) = f_{M/xM}^+(t) + \ell(xM/ (x M \cap \mathfrak{m}^t M), \quad t
\gg 0.  \]
In particular, $\ell( xM/ (xM \cap \mathfrak{m}^t M))$ is a polynomial in $t$.
What can we say about it? Well, $xM \simeq M$ as $x$ is a nonzerodivisor. In
particular
\[ xM / (xM \cap \mathfrak{m}^t M) \simeq M/N_t  \]
where
\[ N_t = \left\{a \in M: xa \in \mathfrak{m}^t M\right\} . \]
In particular, $N_t \supset \mathfrak{m}^{t-1} M$. This tells us that
$\ell(M/N_t) \leq \ell(M/\mathfrak{m}^{t-1} M) = f_M^+(t-1)$ for $t \gg 0$.
Combining this with the above information, we learn that
\[ f_M^+(t) \leq f_{M/xM}^+(t) + f_M^+(t-1),   \]
which implies that $f_{M/xM}^+(t)$ is at least the successive difference
$f_M^+(t) - f_M^+(t-1)$. This last polynomial has degree $\dim \supp M -1$. In
particular, $f_{M/xM}^+(t)$ has degree at least $\dim \supp M -1 $. This gives
us one direction, actually the hard one. We showed that intersecting something with codimension one
doesn't drive the dimension down too much. 

Let us now do the other direction. We essentially did this last time via the
Artin-Rees lemma. We know that $N_t = \left\{a \in M: xa \in
\mathfrak{m}^t\right\}$. The Artin-Rees lemma tells us that there is a constant
$c$ such that $N_{t+c} \subset \mathfrak{m}^t M$ for all $t$. Therefore,
$\ell(M/N_{t+c}) \geq \ell(M/\mathfrak{m}^t M) = f_M^+(t), t \gg 0$. Now
remember the exact sequence $0 \to M/N_t \to M/\mathfrak{m}^t M \to M/(xM +
\mathfrak{m}^t M) \to 0$. We see from this that
\[ \ell(M/ \mathfrak{m}^t M) = \ell(M/N_t) + f_{M/xM}^+(t) \geq f_M^+(t-c) +
f_{M/xM}^+(t), \quad t \gg 0,  \]
which implies that
\[ f_{M/xM}^+(t) \leq f_M^+(t) - f_M^+(t-c),  \]
so the degree must go down. And we find that $\deg f_{M/xM}^+ < \deg f_{M}^+$.
\end{proof} 

This gives us an algorithm of computing the dimension of an $R$-module $M$. 
First, it reduces to computing $\dim R/\mathfrak{p}$ for $\mathfrak{p} \subset
R$ a prime ideal. We may assume that $R$ is a domain and that we are looking
for $\dim R$. Geometrically, this
corresponds to taking an irreducible component of $\spec R$.

Now choose any $x
\in R$ such that $x$ is nonzero but noninvertible. If there is no such element,
then $R$ is a field and has dimension zero. Then compute $\dim R/x$
(recursively) and add one.

Notice that this algorithm said nothing about Hilbert polynomials, and only
talked about the structure of prime ideals.
\lecture{11/3}

\subsection{Recap}
Last time, we were talking about dimension theory. 
Recall that $R$ is a local noetherian ring with maximal ideal $\mathfrak{m}$,
$M$ a finitely generated $R$-module. We can look at the lengths $\ell(M/\mathfrak{m}^t M)$
for varying $t$; for $t \gg 0$ this is a polynomial function. The degree of
this polynomial is called the \textbf{dimension} of $\supp M$. 

\begin{remark} 
If $M = 0$, then we define $\dim \supp M = -1$ by convention.
\end{remark} 

Last time, we showed that if $M \neq 0$ and $x \in \mathfrak{m}$ such that $x$
is a nonzerodivisor on $M$ (i.e. $M \stackrel{x}{\to} M$ injective), then 
\[ \boxed{ \dim \supp M/xM = \dim \supp M - 1.}\]
Using this, we could give a recursion for calculating the dimension. 
To compute $\dim R = \dim \spec R$, we note three properties:
\begin{enumerate}
\item $\dim R = \sup_{\mathfrak{p} \ \mathrm{a \ minimal \ prime}}
R/\mathfrak{p}$. Intuitively, this says that a variety which is the union of
irreducible components has dimension equal to the maximum of these irreducibles.
\item $\dim R = 0$ for $R$  a field. This is obvious from the definitions.
\item If $R$ is a domain, and $x \in \mathfrak{m} - \left\{0\right\}$, then
$\dim R/(x) +1 = \dim R $. This is obvious from the boxed formula as $x$ is a nonzerodivisor.
\end{enumerate}

These three properties \emph{uniquely characterize} the dimension invariant. 

\textbf{More precisely, if
$d: \left\{\mathrm{local \ noetherian \ rings}\right\} \to \mathbb{Z}_{\geq 0}$
satisfies the above three properties, then $d = \dim $. }
\begin{proof} 
Induction on $\dim R$. It is clearly sufficient to prove this for $R$ a domain. 
If $R$ is a field, then it's clear; if $\dim R>0$, the third condition lets us
reduce to a case covered by the inductive hypothesis (i.e. go down).
\end{proof} 

Let us rephrase 3 above:
\begin{quote}
3': If $R$ is a domain and not a field, then 
\[ \dim R = \sup_{x \in \mathfrak{m} - 0} \dim R/(x) + 1. \]
\end{quote}
Obviously 3' implies 3, and it is clear by the same argument that 1,2, 3'
characterize the notion of dimension.

\subsection{Another notion of dimension} We shall now define another notion of
dimension, and show that it is equivalent to the older one by showing that it
satisfies these axioms.

\begin{definition} 
Let $R$ be a commutative ring. A \textbf{chain of prime ideals} in $R$ is a finite
sequence
\[ \mathfrak{p}_0 \subsetneq \mathfrak{p}_1 \subsetneq \dots \subsetneq
\mathfrak{p}_n.  \]
This chain is said to have \textbf{length $n$.}
\end{definition} 

\begin{definition} 
The \textbf{Krull dimension} of $R$ is equal to the maximum length of any chain
of prime ideals. This might be $\infty$, but we will soon see this cannot
happen for $R$ local and noetherian.
\end{definition} 

\begin{remark} 
For any maximal chain $\left\{\mathfrak{p}_i, 0 \leq i \leq n\right\}$ of primes (i.e. which can't be expanded), we must have
that $\mathfrak{p}_0$ is minimal prime and $\mathfrak{p}_n$ a maximal ideal.
\end{remark} 

\begin{theorem} 
For a noetherian local ring $R$, the Krull dimension of $R$ exists and is equal
to the usual $\dim R$.
\end{theorem}
\begin{proof} 
We will show that the Krull dimension satisfies the above axioms. For now,
write $\krdim$ for Krull dimension.

\begin{enumerate}
\item First, note that $\krdim(R) = \max_{\mathfrak{p} \in R \
\mathrm{minimal}}  \krdim(R/\mathfrak{p})$. This is because any chain of prime
ideals in $R$ contains a minimal prime. So any chain of prime ideals in $R$ can
be viewed as a chain in \emph{some} $R/\mathfrak{p}$, and conversely.
\item Second, we need to check that $\krdim(R) = 0$ for $R$ a field. This is
obvious, as there is precisely one prime ideal.
\item The third condition is interesting. We must check that for $(R,
\mathfrak{m})$ a local
domain, 
\[ \krdim(R) = \max_{x \in \mathfrak{m} - \left\{0\right\}} \krdim(R/(x)) + 1.  \]
If we prove this, we will have shown that condition 3' is satisfied by the
Krull dimension. It will follow by the inductive argument above that $\krdim(R)
= \dim (R)$ for any $R$. 
There are two inequalities to prove. First, we must show
\[ \krdim(R) \geq \krdim(R/x) +1, \quad \forall x \in \mathfrak{m} - 0.  \]
So suppose $k = \krdim(R/x)$. We want to show that there is a chain of prime
ideals of length $k+1$ in $R$. So say $\mathfrak{p}_0 \subsetneq \dots
\subsetneq \mathfrak{p}_k$ is a chain of length $k$ in $R/(x)$. The inverse
images in $R$ give a proper chain of primes in $R$ of  length $k$, all of which
contain $(x)$ and thus properly contain $0$. Thus adding zero will give a chain
of primes in $R$ of length $k+1$. 

Conversely, we want to show that if there is a chain of primes in $R$ of
length  $k+1$, then there is a chain of length $k$ in $R/(x)$ for some $x \in
\mathfrak{m} - \left\{0\right\}$. Let us write the chain  of length $k+1$:
\[ \mathfrak{q}_{-1} \subset \mathfrak{q}_0 \subsetneq \dots \subsetneq
\mathfrak{q}_k \subset R . \]
Now evidently $\mathfrak{q}_0$ contains some $x \in \mathfrak{m} - 0$. Then the
chain $\mathfrak{q}_0 \subsetneq \dots \subsetneq \mathfrak{q}_k$ can be
identified with a chain in $R/(x)$ for this $x$. So for this $x$, we have that
$\krdim R \leq \sup \krdim R/(x) + 1$.
\end{enumerate}
\end{proof} 

There is thus a combinatorial definition of definition.

Geometrically, let $X = \spec R$ for $R$ an affine ring over $\mathbb{C}$ (a
polynomial ring mod some ideal). Then $R$ has Krull dimension $\geq k$ iff there is a
chain of irreducible subvarieties of $X$,
\[ X_0 \supset X_1 \supset \dots \supset X_k . \]
You will meet justification for this in \ref{subsectiondimension} below.

\begin{remark}[\textbf{Warning!}] Let $R$ be a local noetherian ring of dimension $k$. This
means that there is a chain of prime ideals of length $k$, and no longer
chains. Thus there is a maximal chain whose length is $k$. However, not all
maximal chains in $\spec R$ have length $k$. 
\end{remark} 

\begin{example} 
Let $R =( \mathbb{C}[X,Y,Z]/(XY,XZ))_{(X,Y,Z)}$. It is left as an
exercise to the reader to see that there are maximal chains of
length not two.

There are more complicated local noetherian \emph{domains} which have maximal
chains of prime ideals not of the same length. These examples are not what you
would encounter in daily experience, and are necessarily complicated. This
cannot happen for finitely generated domains over a field.
\end{example} 

\begin{example} 
An easier way all maximal chains could fail to be of the same length is if
$\spec R$ has two components (in which case $R = R_0 \times R_1$ for rings
$R_0, R_1$). 
\end{example} 

\subsection{Dimension theory for topological spaces}
\label{subsectiondimension}
The present subsection (which consists mostly of exercises) is a digression   that may illuminate the notion of
Krull dimension.

\begin{definition} 
Let $X$ be a topological space.\footnote{We do not include the empty space.} Recall that $ X$ is
\textbf{irreducible} if  cannot be written as the union of
two proper closed subsets $F_1, F_2 \subsetneq X$.

We say that a subset of $X$ is irreducible if it is irreducible with respect
to the induced topology.
\end{definition} 

In general, this notion is not valid from the topological spaces familiar from
analysis. For instance:

\begin{exercise} 
Points are the only irreducible subsets of $\mathbb{R}$.
\end{exercise} 

Nonetheless, irreducible sets behave very nicely with respect to certain
operations. As you will now prove, if $U \subset X$ is an open subset, then
the irreducible closed subsets of $U$ are in bijection with the irreducible
closed subsets of $X$ that intersect $U$. 
\begin{exercise} \label{irredifeveryopenisdense}
A space is irreducible if and only if every open set is dense, or
alternatively if every open set is connected.
\end{exercise} 

\begin{exercise} 
Let $X$ be a space, $Y \subset X$ an irreducible subset. Then 
$\overline{Y} \subset X$ is irreducible.
\end{exercise} 

\begin{exercise} 
Let $X$ be a space, $U \subset X$ an open subset. 
Then the map $Z \to Z \cap U$ gives a bijection between the irreducible
closed subsets of $X$ meeting $U$ and the irreducible closed subsets of $U$.
The inverse is given by $Z' \to \overline{Z'}$.
\end{exercise} 

As stated above, the  notion of irreducibility is useless for spaces
like manifolds. In fact, by Exercise \ref{irredifeveryopenisdense}, a
Hausdorff space cannot be irreducible unless it consists of one point.
However, for the highly non-Hausdorff spaces encountered in algebraic geometry, this notion is very
useful.

Let $R$ be a commutative ring, and $X = \spec R$.

\begin{exercise} 
A closed subset $F \subset \spec R$ is irreducible if and only if it can be
written in the form $F = V(\mathfrak{p})$ for $\mathfrak{p} \subset R$ prime.
In particular, $\spec R$ is irreducible if and only if $R$ has one minimal
prime. 
\end{exercise} 

In fact, spectra of rings are particularly nice: they are \textbf{sober
spaces.}
\begin{definition} 
A space $X$ is called \textbf{sober} if to every irreducible closed $F \subset
X$, there is a unique point $\xi \in F$ such that $F = \overline{ \left\{\xi\right\}}$. 
This point is called the \textbf{generic point.}
\end{definition} 

\begin{exercise} 
Check that if $X$ is any topological space and $\xi  \in X$, then the closure
$\overline{\left\{\xi\right\}}$ of the point $\xi$ is irreducible.
\end{exercise} 

\begin{exercise} 
Show that $\spec R$ for $R$ a ring is sober.
\end{exercise} 

\begin{exercise} 
Let $X$ be a space with a cover $\left\{X_\alpha\right\}$ by open subsets,
each of which is a sober space. Then $X$ is a sober space. (Hint: any
irreducible closed subset must intersect one of the $X_\alpha$, so is the
closure of its intersection with that one.)
\end{exercise} 

We now come to the main motivation of this subsection, and the reason for its
inclusion here.

\begin{definition} 
Let $X$ be a topological space. Then the \textbf{dimension} (or
\textbf{combinatorial dimension}) of $X$ is the maximal $k$ such that a chain
\[ F_0 \subsetneq F_1 \subsetneq \dots \subsetneq F_k \subset X  \]
with the $F_i$ irreducible, exists. This number is denoted $\dim X$ and may be
infinite.
\end{definition} 

\begin{exercise} 
What is the Krull dimension of $\mathbb{R}$?
\end{exercise} 

\begin{exercise} 
Let $X = \bigcup X_i$ be the finite union of subsets $X_i \subset X$. 

\end{exercise} 

\begin{exercise} 
Let $R$ be a ring. Then $\dim \spec R$ is equal to the Krull dimension of $R$. 
\end{exercise} 




\subsection{Yet another definition}
Let's start by thinking about the definition of a module. Recall that if $(R,
\mathfrak{m})$ is
a local noetherian ring and $M$ a finitely generated $R$-module, and $x \in \mathfrak{m}$ is
a nonzerodivisor on $M$, then
\[ \dim \supp M/xM = \dim \supp M -1.  \]

\begin{question} 
What if $x$ is  a zerodivisor? 
\end{question} 

This is not necessarily true (e.g. if $x \in \ann(M)$). Nonetheless, we claim
that even in this case:
\begin{proposition} 
For any $x \in \mathfrak{m}$,
\[ \boxed{ \dim \supp M \geq \dim \supp M/xM \geq \dim \supp M -1 .}\]
\end{proposition} 
The upper bound on $\dim M/xM$ is obvious as $M/xM$ is a quotient of $M$. The
lower bound is trickier. 

\begin{proof} 
Let $N = \left\{a \in M: x^n a = 0 \ \mathrm{for \ some \ } n \right\}$. We can
construct an exact sequence
\[ 0 \to N \to M \to M/N \to 0.  \]
Let $M'' = M/N$.
Now $x$ is a nonzerodivisor on $M/N$ by construction. We claim that 
\[ 0 \to N/xN \to M/xM \to M''/xM'' \to 0  \]
is exact as well. For this we only need to see exactness at the beginning,
i.e. injectivity of $N/xN \to M/xM$. So
we need to show that if $a \in N$ and $a \in xM$, then $a \in x N$.

To see this, suppose $a = xb$ where $b \in M$. Then if $\phi: M \to M''$, then
$\phi(b) \in M''$ is killed by $x$ as $x \phi(b) = \phi(bx) = \phi(a)$.
This means that $\phi(b)=0$ as $M'' \stackrel{x}{\to} M''$ is injective. Thus
$b \in N$ in fact. So $a \in xN$ in fact.

From the exactness, we see that (as $x$ is a nonzerodivisor on $M''$)
\begin{align*} \dim M/xM & = \max (\dim M''/xM'', \dim N/xN) \geq \max(\dim M'' -1, \dim
N)\\ &  \geq \max( \dim M'', \dim N)-1  .  \end{align*}
The reason for the last claim is that $\supp N/xN = \supp N$ as $N$ is
$x$-torsion, and the dimension depends only on the support. But the thing on the right is just $\dim M -1$. 
\end{proof} 

As a result, we find:

\begin{proposition} 
$\dim \supp M$ is the minimal integer $n$ such that there exist elements $x_1,
\dots, x_n \in \mathfrak{m}$ with $M/(x_1 , \dots, x_n) M$ has finite length. 
\end{proposition} 
Note that $n$ always exists, since we can look at a bunch of generators of the
maximal ideal, and $M/\mathfrak{m}M $ is a finite-dimensional vector space and
is thus of finite length.
\begin{proof} 
Induction on $\dim \supp M$. Note that $\dim \supp(M)=0$ if and only if the
Hilbert polynomial has degree zero, i.e. $M$ has finite length or that $n=0$
($n$ being defined as in the statement). 

Suppose $\dim \supp M > 0$. \begin{enumerate}
\item  We first show that there are $x_1, \dots, x_{\dim M}$
with $M/(x_1, \dots, x_{\dim M})M$ have finite length. 
Let $M' \subset M$ be the maximal submodule having finite length. There
is an exact sequence
\[ 0 \to M' \to M \to M'' \to 0  \]
where $M'' = M/M'$ has no finite length submodules. In this case, we can
basically ignore $M'$, and replace $M$ by $M''$. The reason is that modding out
by $M'$ doesn't affect either $n$ or the dimension. 

So let us replace $M$ with
$M''$ and thereby assume that $M$ has no finite length submodules. In
particular, $M$ does not contain a copy of $R/\mathfrak{m}$, i.e. $\mathfrak{m}
\notin \ass(M)$. 
By prime avoidance, this means that there is $x_1 \in \mathfrak{m}$ that acts as
a nonzerodivisor on $M$. Thus
\[ \dim M/x_1M = \dim M -1.  \]
The inductive hypothesis says that there are $x_2, \dots, x_{\dim M}$ with
$$(M/x_1 M)/(x_2, \dots, x_{\dim M}) (M/xM) \simeq M/(x_1, \dots, x_{\dim M})M $$
of finite length. This shows the claim.
\item Conversely, suppose that there $M/(x_1, \dots, x_n)M$ has finite length.
Then we claim that $n \geq \dim M$. This follows because we had the previous
result that modding out by a single element can chop off the dimension by at
most $1$. Recursively applying this, and using the fact that $\dim$ of a
finite length module is zero, we find
\[ 0 = \dim M/(x_1 , \dots, x_n )M \geq \dim M -n. \]
\end{enumerate}
\end{proof} 


\begin{corollary} 
Let $(R, \mathfrak{m})$ be a local noetherian ring. Then $\dim R$ is equal to the minimal $n$
such that there exist $x_1, \dots, x_n \in R$ with $R/(x_1, \dots, x_n) R$ is
artinian. Or, equivalently, such that $(x_1, \dots, x_n)$ contains a power of
$\mathfrak{m}$.
\end{corollary}


\begin{remark} 
We manifestly have here that the dimension of $R$ is at most the embedding
dimension. Here, we're not worried about generating the maximal ideal, but
simply something containing a power of it.
\end{remark} 
\lecture{11/5}

We have been talking about dimension. Let $R$ be a local noetherian ring with
maximal ideal $\mathfrak{m}$. Then, as we have said in previous lectures, $\dim R$ can be characterized by:
\begin{enumerate}
\item The minimal $n$ such that there is an $n$-primary ideal generated by $n$
elements $x_1, \dots, x_n \in \mathfrak{m}$. That is, the closed point
$\mathfrak{m}$ of
$\spec R$ is cut out \emph{set-theoretically} by the intersection $\bigcap
V(x_i)$. This is one way of saying that the closed point can be defined by $n$
parameters. 
\item The \emph{maximal} $n$ such that there exists a chain of prime ideals
\[ \mathfrak{p}_0 \subset \mathfrak{p}_1 \subset \dots \subset \mathfrak{p}_n. \]
\item The degree of the Hilbert polynomial $f^+(t)$, which equals
$\ell(R/\mathfrak{m}^t)$ for $t \gg 0$.
\end{enumerate}


\subsection{Consequences of the notion of dimension}


Let $R$ be a local noetherian ring.
The following is now clear from what we have shown:

\begin{theorem}[Krull's Hauptidealsatz]
$R$ has dimension $1$ if and only if there is a nonzerodivisor $x \in \mathfrak{m}$ such that
$R/(x)$ is artinian.
\end{theorem} 



\begin{remark} 
Let $R$ be a domain. We said that a nonzero prime $\mathfrak{p} \subset R$ is
\textbf{height one} if $\mathfrak{p}$ is minimal among the prime ideals
containing some nonzero $x \in R$. 

According to Krull's Hauptidealsatz, $\mathfrak{p}$ has height one \textbf{if
and only if $\dim R_{\mathfrak{p}} = 1$.}
\end{remark} 


We can generalize the notion of $\mathfrak{p}$ as follows.
\begin{definition} 
Let $R$ be a noetherian ring (not necessarily local), and $\mathfrak{p} \in
\spec R$. Then we define the \textbf{height} of $\mathfrak{p}$, denoted
$\het(\mathfrak{p})$, as $\dim R_{\mathfrak{p}}$.
We know that this is the length of a maximal chain of primes in
$R_{\mathfrak{p}}$. This is thus the maximal length of prime ideals of $R$, 
\[ \mathfrak{p}_0 \subset \dots \subset \mathfrak{p}_n = \mathfrak{p}  \]
that ends in $\mathfrak{p}$. This is the origin of the term ``height.''
\end{definition} 

\begin{remark} 
Sometimes, the height is called the \textbf{codimension}. This corresponds to
the codimension in $\spec R$ of the corresponding irreducible closed subset of
$\spec R$.
\end{remark} 

\subsection{Further remarks}

We can recast earlier notions in terms of dimension.
\begin{remark} 
A noetherian ring has dimension zero if and only if $R$ is artinian. Indeed,
$R$ has dimension zero iff all primes are maximal.
\end{remark} 


\begin{remark} 
A noetherian domain has dimension zero iff it is a field. Indeed, in this case
$(0)$ is maximal.
\end{remark} 

\begin{remark} 
$R$ has dimension $\leq 1$ if and only if every non-minimal prime of $R$ is
maximal. That is, there are no chains of length $\geq 2$.
\end{remark} 

\begin{remark} 
A (noetherian) domain  $R$ has dimension $\leq 1$ iff every nonzero prime ideal
is maximal.
\end{remark} 

In particular,
\begin{proposition} 
$R$ is Dedekind iff it is a noetherian, integrally closed domain of dimension
$1$. 
\end{proposition} 

\subsection{Change of rings}
Let $f: R \to R'$ be  a map of noetherian rings. 

\begin{question} 
What is the relationship between $\dim R$ and $\dim R'$?
\end{question} 

A map $f$ gives a map $\spec R' \to \spec R$, where $\spec R'$ is the union
of various fibers over the points of $\spec R$. You might imagine that the
dimension is the dimension of $R$ plus the fiber dimension. This is sometimes
true.

Now assume that $R, R'$ are \emph{local}  with maximal ideals $\mathfrak{m},
\mathfrak{m}'$. Assume furthermore that $f$ is local, i.e. $f(\mathfrak{m})
\subset \mathfrak{m}'$.

\begin{theorem} 
$\dim R' \leq \dim R +  \dim R'/\mathfrak{m}R'$. Equality holds if $f: R \to
R'$ is flat.
\end{theorem} 

Here $R'/\mathfrak{m}R'$ is to be interpreted as the ``fiber'' of $\spec R'$
above $\mathfrak{m} \in \spec R$. The fibers can behave weirdly as the
basepoint varies in $\spec R$, so we can't
expect equality in general.

\begin{remark} 
Let us review flatness as it has been a while. An $R$-module $M$ is \emph{flat} iff
the operation of tensoring with $M$ is an exact functor. The map $f: R \to R'$
is \emph{flat} iff $R'$ is a flat $R$-module. Since the construction of taking
fibers is a tensor product (i.e. $R'/\mathfrak{m}R' = R' \otimes_R
R/\mathfrak{m}$), perhaps the condition of flatness here is not as surprising as
it might be.
\end{remark} 

\begin{proof} 
Let us first prove the inequality. Say $$\dim R = a,  \ \dim R'/\mathfrak{m}R'
= b.$$ We'd like to see that
\[ \dim R' \leq a+b.  \]
To do this, we need to find $a+b$ elements in the maximal ideal $\mathfrak{m}'$
that generate a $\mathfrak{m}'$-primary ideal of $R'$. 

There are elements $x_1, \dots, x_a \in \mathfrak{m}$ that generate an
$\mathfrak{m}$-primary ideal $I = (x_1, \dots, x_a)$ in $R$. There is a surjection $R'/I R'
\twoheadrightarrow R'/\mathfrak{m}R'$.
The kernel $\mathfrak{m}R'/IR'$ is nilpotent since $I$ contains a power of
$\mathfrak{m}$. 	We've seen that nilpotents \emph{don't} affect the dimension.
In particular, 
\[ \dim R'/IR' = \dim R'/\mathfrak{m}R' = b.  \]
There are thus elements $y_1, \dots, y_b \in \mathfrak{m}'/IR'$ such that the
ideal $J = (y_1, \dots, y_b) \subset R'/I R'$ is $\mathfrak{m}'/IR'$-primary.
The inverse image of $J$ in $R'$, call it $\overline{J} \subset R'$, is
$\mathfrak{m}'$-primary. However, $\overline{J}$ is generated by the $a+b$
elements
\[ f(x_1), \dots, f(x_a), \overline{y_1}, \dots, \overline{y_b}  \]
if the $\overline{y_i}$ lift $y_i$. 

But we don't always have equality. Nonetheless, if all the fibers are similar,
then we should expect that the dimension of the ``total space'' $\spec R'$ is
the dimension of the ``base'' $\spec R$ plus the ``fiber'' dimension $\spec
R'/\mathfrak{m}R'$.  
\emph{The precise condition of $f$ flat articulates the condition that the fibers
 ``behave well.'' }
Why this is so is something of a mystery, for now.
But for some evidence, take the present result about fiber dimension.

Anyway, let us now prove equality for flat $R$-algebras. As before, write $a =
\dim R, b = \dim R'/\mathfrak{m}R'$. We'd like to show that
\[ \dim R' \geq a+b.  \]
By what has been shown, this will be enough.
This is going to be tricky since we now need to give \emph{lower bounds} on the
dimension; finding a sequence $x_{1}, \dots, x_{a+b}$ such that the quotient
$R/(x_1, \dots, x_{a+b})$ is artinian would bound \emph{above} the dimension.

So our strategy will be to find a chain of primes of length $a+b$. Well, first
we know that there are primes
\[ \mathfrak{q}_0 \subset \mathfrak{q}_1 \subset \dots \subset \mathfrak{q}_b
\subset R'/\mathfrak{m}R'.  \]
Let $\overline{\mathfrak{q}_i}$ be the inverse images in $R'$. Then the
$\overline{\mathfrak{q}_i}$ are a strictly ascending chain of primes in $R'$ where
$\overline{\mathfrak{q}_0}$ contains $\mathfrak{m}R'$. So we have a chain of
length $b$; we need to extend this by additional terms.

Now $f^{-1}(\overline{\mathfrak{q}_0})$ contains $\mathfrak{m}$, hence is
$\mathfrak{m}$. Since $\dim R = a$, there is a chain
$\left\{\mathfrak{p}_i\right\}$ of prime ideals of length
$a$ going down from $f^{-1}(\overline{\mathfrak{q}_0}) = \mathfrak{m}$. We are 
now going to find primes $\mathfrak{p}_i' \subset R'$ forming a chain such that
$f^{-1}(\mathfrak{p}_i') = \mathfrak{p}_i$. In other words, we are going to
\emph{lift} the chain $\mathfrak{p}_i$ to $\spec R'$. We can do this at the
first stage for $i=a$, where $\mathfrak{p}_a = \mathfrak{m}$ and we can set
$\mathfrak{p}'_a = \overline{\mathfrak{q}_0}$. If we can indeed do this
lifting, and catenate the chains $\overline{\mathfrak{q}_j}, \mathfrak{p}'_i$,
then we will have a chain of the appropriate length.

We will proceed by descending induction. Assume that we have
$\mathfrak{p}_{i+1}' \subset R'$ and $f^{-1}(\mathfrak{p}_{i+1}') =
\mathfrak{p}_{i+1} \subset R$. We want to find $\mathfrak{p}_i' \subset
\mathfrak{p}'_{i+1}$ such that $f^{-1}(\mathfrak{p}_i') = \mathfrak{p}_i$. The
existence of that prime is a consequence of the following general fact.

\begin{theorem}[Going down] Let $f: R \to R'$ be a flat map of
noetherian commutative
rings. Suppose $\mathfrak{q} \in \spec R'$, and let $\mathfrak{p}
=f^{-1}(\mathfrak{q})$. Suppose $\mathfrak{p}_0 \subset \mathfrak{p}$ is a
prime of $R$. Then there is a prime $\mathfrak{q}_0 \subset \mathfrak{q}$ with 
\[ f^{-1}(\mathfrak{q}_0) = \mathfrak{p}_0.  \]
\end{theorem} 
\begin{proof} 
We may replace $R'$ with $R'_{\mathfrak{q}}$. There is still a map
\[ R \to R_{\mathfrak{q}}'  \]
which is flat as localization is flat. The maximal ideal in $R'_{\mathfrak{q}}$
has inverse image $\mathfrak{p}$. So the problem now reduces to finding
\emph{some} $\mathfrak{p}_0$ in the localization that pulls back appropriately.

Anyhow, throwing out the old $R$ and replacing with the localization, we may
assume that $R'$ is local and $\mathfrak{q}$ the maximal ideal. (The condition
$\mathfrak{q}_0 \subset \mathfrak{q}$ is now automatic.) 

The claim now is that we can replace $R$ with $R/\mathfrak{p}_0$ and $R'$ with
$R'/\mathfrak{p}_0 R' = R' \otimes R/\mathfrak{p}_0$. We can do this because
base change preserves flatness (see below), and in this case we can reduce to the case of
$\mathfrak{p}_0 = (0)$---in particular, $R$ is a domain. 
Taking these quotients just replaces $\spec R, \spec R'$ with closed subsets
where all the action happens anyhow.

Under these replacements, we now have:
\begin{enumerate}
\item $R'$ is local with maximal ideal $\mathfrak{q}$ 
\item $R$ is a domain and $\mathfrak{p}_0 = (0)$.
\end{enumerate}
We want a prime of $R'$ that pulls back to $(0)$ in $R$. I claim that any
minimal prime of $R'$ will work. 
Suppose otherwise. Let $\mathfrak{q}_0 \subset R'$ be a minimal prime, and
suppose $x \in R \cap f^{-1}(\mathfrak{q}_0) - \left\{0\right\}$. But
$\mathfrak{q}_0 \in \ass(R')$. So $f(x)$ is
a zerodivisor on $R'$. Thus multiplication by $x$ on $R'$ is not injective. 

But, $R$ is a domain, so $R \stackrel{x}{\to} R$ is injective. Tensoring with
$R'$ must preserve this, implying that $R' \stackrel{x}{\to} R'$ is injective
because $R'$ is flat. This is a contradiction.
\end{proof} 

We used:
\begin{lemma} 
Let $R \to R'$ be a flat map, and $S$ an $R$-algebra. Then $S \to S \otimes_R
R'$ is a flat map.
\end{lemma} 
\begin{proof} 
The construction of taking an $S$-module with $S \otimes_R R'$ is an exact
functor, because that's the same thing as taking an $S$-module, restricting to
$R$, and tensoring with $R'$.
\end{proof} 
The proof of the fiber dimension theorem is now complete.

\end{proof} 



