\chapter{Homological theory of local rings}

We will then apply the general theory to commutative algebra proper. The use
of homological machinery provides a new and elegant characterization of regular local
rings (among noetherian local rings, they are the ones with finite global
dimension) and leads to proofs of several difficult results about them.
For instance, we will be able to prove the rather important result (which one
repeatedly uses in algebraic geometry) that a
regular local ring is a UFD.
As another example, the aforementioned criterion leads to a direct proof of
the otherwise non-obvious that a localization of a regular local ring at a
prime ideal is still a regular local ring.

\textbf{Note: right now, the material on regular local rings is still missing!
It should be added.}

\section{Depth}
Throughout, let $(R, \mathfrak{m})$ be  a noetherian
local ring. Let $k = R/\mathfrak{m}$ be the residue field. 


\subsection{Depth over local rings}
Let $M \neq 0$ be a finitely generated $R$-module. We are going to define an
arithmetic invariant of $M$, called the \emph{depth}, that will measure in
some sense the torsion of $M$.

\newcommand{\depth}{\operatorname{depth}}
\begin{definition}
The \textbf{depth} of $M$ is equal to the smallest integer $i$
such that
$\ext^i(k,M) \neq 0$.
\end{definition}

We shall give another characterization of this shortly that makes no reference
to $\ext$ functors, and is purely elementary.
Note that
contained in this definition is an assertion: that there is such
an $i$ (at least if $M \neq 0$).

\begin{example} Let us characterize when a module $M$ has depth zero. 
Depth zero is equivalent to saying that $\ext^0(k,M) = \hom_R(k, M) \neq 0$,
i.e. that there is a
nontrivial morphism
\[ k \to M.  \]
As $k = R/\mathfrak{m}$, the existence of such a map is
equivalent to the existence of a nonzero $x$
such that $\ann(x) = \mathfrak{m}$, i.e. $\mathfrak{m} \in
\ass(M)$. So depth
zero is equivalent to having $\mathfrak{m} \in \ass(M)$.
\end{example}

Suppose now that $\depth(M) \neq 0$. In particular,
$\mathfrak{m} \notin
\ass(M)$. Since $\ass(M)$ is finite, prime avoidance implies that
$\mathfrak{m}
\not\subset \bigcup_{\mathfrak{p} \in \ass(M)} \mathfrak{p}$.
Thus
$\mathfrak{m}$ contains an element which is a nonzerodivisor on
$M$ (see \cref{assmdichotomy}). So we find:

\begin{proposition} \label{depthzero}
$M$ has depth zero iff every element in $\mathfrak{m}$ is a
zerodivisor on $M$.
\end{proposition}

Now suppose $\depth M \neq 0$. There is $a \in \mathfrak{m}$
which is a
nonzerodivisor on $M$, i.e.  such that there is
an exact sequence
\[ 0 \to M \stackrel{a}{\to} M \to M/aM \to 0.  \]
For each $i$, there is an  exact sequence in $\ext$ groups:
\begin{equation} \label{extlongextdepth}\ext^{i-1}(k,M) \to \ext^i(k,M) \stackrel{a}{\to} \ext^i(k,M)
\to \ext^i(k,
M/aM) \to \ext^{i+1}(k,M)  .\end{equation}
However, the map $a: \ext^i(k,M) \to \ext^i(k,M)$ is zero as
multiplication by $a$
kills $k$. (As we said last time, if $a$ kills a module $N$,
then it kills
$\ext^*(N,M)$ for all $M$.) We see from this that
\[ \ext^i(k,M) \hookrightarrow \ext^i(k,M/aM)  \]
is injective, and
\[ \ext^{i-1}(k, M/aM) \twoheadrightarrow \ext^i(k,M)  \]
is surjective.

\begin{corollary} \label{depthdropsbyone}
If $a \in \mathfrak{m}$ is a nonzerodivisor on $M$, then
\[ \depth(M/aM) = \depth M -1.  \]
\end{corollary}
\begin{proof}
When $\depth M = \infty$, this is easy (left to the reader) from
the exact
sequence. Suppose $\depth(M) = n$. We would like to see that
$\depth M/aM =
n-1$. That is, we want to see that $\ext^{n-1}(k,M/aM) \neq 0$,
but
$\ext^i(k,M/aM) =
0$ for $i < n-1$. This is direct from the sequence \eqref{extlongextdepth} above.
In fact, surjectivity of $\ext^{n-1}(k,M/aM) \to \ext^n(k,M)$
shows that
$\ext^{n-1}(k,M/aM) \neq 0$. Now let $i < n-1$. 
Then in \eqref{extlongextdepth}, $\ext^i(k, M/aM)$ is sandwiched between two
zeros, so it is zero.
\end{proof}

The moral of the above discussion is that one quotients out by a nonzerodivisor, the depth drops by one.
In fact, we have described a recursive algorithm for computing
$\depth(M)$.
\begin{enumerate}
\item If $\mathfrak{m}  \in \ass(M)$, output zero.
\item If $\mathfrak{m} \notin \ass(M)$, choose an element $a
\in\mathfrak{m}$
which is a nonzerodivisor on $M$. Output $\depth(M/aM) +1$.
\end{enumerate}


If one wished to apply this in practice, one would probably start by
looking for a
nonzerodivisor $a_1 \in \mathfrak{m}$ on $M$, and then looking for
one on $M/a_1
M$, etc.
From this we make:

\begin{definition}
Let $(R, \mathfrak{m})$ be a local noetherian ring, $M$ a finite
$R$-module. A
sequence $a_1, \dots, a_n \in \mathfrak{m}$ is said to be
\textbf{$M$-regular} iff:
\begin{enumerate}
\item $a_1$ is a nonzerodivisor on $M$
\item $a_2$ is a nonzerodivisor on $M/a_1 M$
\item  $\dots$
\item $a_i$ is a nonzerodivisor on $M/(a_1, \dots, a_{i-1})M$
for all $i$.
\end{enumerate}
A regular sequence $a_1, \dots, a_n$ is \textbf{maximal } if it
can be extended
no further, i.e. there is no $a_{n+1}$ such that $a_1, \dots,
a_{n+1}$ is
$M$-regular.
\end{definition}

\begin{corollary} \label{depthregular}
$\depth(M)$ is the length of every maximal $M$-regular
sequence. In particular,
all $M$-regular sequences have the same length.
\end{corollary}

\begin{proof}
If $a_1, \dots, a_n$ is $M$-regular, then
\[ \depth M/(a_1, \dots, a_i)M = \depth M -i  \]
for each $i$, by an easy induction on $i$ and the \cref{depthdropsbyone}.
From this the result is clear, because depth zero occurs precisely when
$\mathfrak{m}$ is an associated prime (\cref{depthzero}). But it is also clear 
that a regular sequence $a_1, \dots, a_n$ is maximal precisely when every
element of $\mathfrak{m}$ acts as a zerodivisor on $M/(a_1, \dots, a_n) M$,
that is, $\mathfrak{m} \in \ass(M/(a_1, \dots, a_n)M)$.
\end{proof}

\begin{remark}
We could \emph{define} the depth via the length of a maximal
$M$-regular sequence.
\end{remark}

Finally, we conclude by bounding the depth in terms of the dimension. 

\begin{corollary} Let $M \neq 0$. Then the depth of $M$ is finite. In fact,
\begin{equation} \label{depthbound} \depth M \leq \dim \supp M.  \end{equation}
\end{corollary}
\begin{proof}
If $\depth M = 0$, then we are done.

In general, we induct on $\dim \supp M$, which we know is
finite. Otherwise,
there is $ a \in \mathfrak{m}$ which is a nonzerodivisor on $M$.
We know that
\[ \depth M/aM = \depth M -1  \]
and (by \cref{dimdropsbyone})
\[ \dim \supp M/aM = \dim \supp M -1.  \]
By induction, we have that $\depth M/aM \leq \dim \supp M/aM$.
From this the
induction step is clear, because $\depth$ and $\dim$ both drop by one after
quotienting.
\end{proof}

Generally, the depth is not the dimension.
\begin{example}
Given any $M$, if you add $k$ to it, then you make the depth
zero: $M \oplus k$
has $\mathfrak{m}$ as an associated prime. But the dimension
 does not
jump to zero just by adding a copy of $k$. If $M$ is a direct sum of pieces of
differing dimensions, then the bound \eqref{depthbound} does not exhibit
equality.
\end{example}

\subsection{Depth in the non-local case}

In the previous subsection, we defined the notion of \emph{depth} of a
finitely generated module over a noetherian local ring using the $\ext$
functors. We then showed that the depth was the length of a maximal regular
sequence. 

Now, although it will not be necessary for the main results in this chapter, we want to generalize this to the case of a non-local ring. Most of the
same arguments go through, though there are some subtle differences. For
instance, regular sequences remain regular under permutation in the local
case, but not in general. Since there will be some repetition, we shall try to
be brief.

We start by generalizing the idea of a regular sequence which is not required
to be contained in the maximal ideal of a local ring.
\begin{definition} 
A sequence $x_1, \dots, x_n \in M$ is \textbf{$M$-regular} (or is an
\textbf{$M$-sequence} if for each $k \leq n$, $x_k$ is a nonzerodivisor on the
$R$-module $M/(x_1, \dots, x_{k-1}) M$ and also $(x_1, \dots, x_n) M \neq M$. 	\end{definition} 

So $x_1$ is a nonzerodivisor on $M$, by the first part. That is, the homothety
$M \stackrel{x_1}{\to} M$ is injective. 
The last condition is also going to turn out to be necessary for us. In the
previous subsection, it was automatic as $\mathfrak{m}M \neq M$ (unless $M =
0$) by Nakayama's lemma as $M$ was assumed finitely generated.	


The property of being a regular sequence is inherently an inductive one. Note
that $x_1, \dots, x_n$ is a regular sequence on $M$ if and only if $x_1$ is a
zerodivisor on $M$ and $x_2, \dots, x_n$ is an $M/x_1 M$-sequence.


\begin{definition} 
If $M$ is an $R$-module and $I \subset R$ an ideal, then we write $\depth_I M$
for the length of the length-maximizing $M$-sequence contained in $I$.
When $R$ is local and $I \subset R$ the maximal ideal, then we just write
$\depth M$ as before.
\end{definition} 

While we will in fact have a similar characterization of $\depth$ in terms of
$\ext$, in this section we \emph{define} it via regular sequences.

\begin{example} 
The basic example one is supposed to keep in mind is the polynomial ring $R =
R_0[x_1, \dots, x_n]$ and $M = R$. Then the sequence $x_1, \dots, x_n$ is
regular in $R$.
\end{example} 

As before, we have a simple characterization of depth zero:
\begin{proposition} Let $R$ be noetherian, $M$ finitely generated.
If $M$ is an $R$-module with $IM \neq M$, then $M$ has depth zero if and only
if $I $ is contained in an element of $\ass(M)$.
\end{proposition} 
\begin{proof} 
This is analogous to \cref{depthzero}. Note than an ideal consists of
zerodivisors on $M$ if and only if it is contained in an associated prime
(\cref{assmdichotomy}).
\end{proof} 

The above proof used \cref{assmdichotomy}, a key fact which will be used
repeatedly in the sequel.
This is one reason the theory of depth works best for finitely generated
modules over noetherian rings.

The first observation to make is that regular sequences are \textbf{not}
preserved by permutation. This is one nice characteristic that we would like
but is not satisfied.

\begin{example} Let $k $ be a field.
Consider $R=k[x,y]/((x-1)y, yz)$. Then $x,z$ is a regular sequence on $R$. Indeed,
$x$ is a nonzerodivisor and $R/(x) = k[z]$.  However, $z,
x$ is not a regular sequence because $z$ is a zerodivisor in $R$.
\end{example} 

Nonetheless, they \emph{are} preserved by permutation for local rings under
suitable noetherian hypotheses: 
\begin{proposition} 
Let $R$ be a noetherian local ring and $M$ a finite $R$-module. Then if $x_1,
\dots, x_n$ is a $M$-sequence contained in the maximal ideal, so is any permutation $x_{\sigma(1)}, \dots,
x_{\sigma(n)}$.
\end{proposition} 
\begin{proof} 
It is clearly enough to check this for a transposition. Namely, if we have an
$M$-sequence
\[ x_1, \dots, x_i, x_{i+1}, \dots x_n  \]
we would like to check that so is
\[ x_1, \dots, x_{i+1}, x_i, \dots, x_n.  \]
It is here that we use the inductive nature. Namely, all we need to do is check
that
\[ x_{i+1}, x_i, \dots,x_n  \]
is regular on $M/(x_1, \dots,x_{i-1}) M$, since the first part of the sequence
will automatically be regular. Now $x_{i+2}, \dots, x_n$ will automatically be
regular on $M/(x_1, \dots, x_{i+1})M$. So all we need to show is that
$x_{i+1}, x_i$ is regular on $M/(x_1, \dots, x_{i-1})M$.

The moral of the story is that we have reduced to the following lemma.

\begin{lemma} 
Let $R$ be a noetherian local ring. Let $N$
be a finite $R$-module and 
$a,b \in R$ an $N$-sequence contained in the maximal ideal. Then so is $b,a$.
\end{lemma} 
We can prove this as follows. First, $a$ will be a nonzerodivisor on $N/bN$.
Indeed, if not then we can write
\[ an = bn'  \]
for some $n,n' \in N$ with $n \notin bN$. But $b$ is a nonzerodivisor on
$N/aN$, which means that $bn' \in aN$ implies $n' \in aN$. Say $n' = an''$. So
$an = ba n''$. As $a$ is a nonzerodivisor on $N$, we see that $n = bn''$. Thus
$n \in bN$, contradiction.
This part has not used the fact that $R$ is local.

Now I claim that $b$ is a nonzerodivisor on $N$. Suppose $n \in N$ and $bn =
0$. Since $b$ is a nonzerodivisor on $N/aN$, we have that $n \in aN$, say $n =
an'$. Thus 
\[ b(an') = a(bn') = 0.  \]
The fact that $N \stackrel{a}{\to} N$ is injective implies that $bn' = 0$. So
we can do the same and get $n' = an''$, $n'' = a n^{(3)}, n^{(3)} =a n^{(4)}$, and
so on. It follows that $n$ is a multiple of $a, a^2,a^3, \dots$, and hence in
$\mathfrak{m}^j N$ for each $j$ where $\mathfrak{m} \subset R$ is the maximal
ideal. The Krull intersection theorem now implies that $n = 0$. 

Together, these arguments imply that $b,a$ is an $N$-sequence, proving the
lemma.
\end{proof} 


One might wonder what goes wrong, and why permutations do not preserve
regular sequences in general; after all, oftentimes we can reduce results
to their analogs for local rings. Yet the fact that regularity is preserved by
permutations for local rings does not extend to arbitrary rings.
The problem is that regular sequences do \emph{not} localize. Well, they almost
do, but the final condition that $(x_1, \dots, x_n) M \neq M$ doesn't get
preserved.
We can state:

\begin{proposition} 
Suppose $x_1, \dots, x_n$ is an $M$-sequence. Let $N$ be a flat $R$-module.
Then if $(x_1, \dots, x_n)M \otimes N \neq M \otimes N$, then $x_1, \dots, x_n$
is an $M \otimes N$-sequence.
\end{proposition} 
\begin{proof} 
This is actually very easy now. The fact that $x_i: M/(x_1, \dots, x_{i-1})M
\to M/(x_1, \dots, x_{i-1})M$ is injective is preserved when $M$ is replaced by
$M \otimes N$ because the functor $- \otimes N$ is exact. 
\end{proof} 

In particular, it follows that if we have a good reason for supposing that
$(x_1,\dots, x_n) M \otimes N \neq M \otimes N$, then we'll already be
done. For instance, if $N$ is the localization of $R$ at a prime ideal
containing the $x_i$. Then we see that automatically $x_1, \dots, x_n$ is an
$M_{\mathfrak{p}} = M \otimes_R R_{\mathfrak{p}}$-sequence. 

Finally, we have an analog of the previous correspondence between depth and
the vanishing of $\ext$. Since the argument is analogous to
\cref{depthregular}, we omit it.
\begin{theorem} Let $R$ be a ring. Suppose $M$ is an $R$-module and $IM \neq M$.
All maximal $M$-sequences in $I$ have the same length. This length is the
smallest value of $r$ such that $\mathrm{Ext}^r(R/I, M) \neq 0$.
\end{theorem} 

\begin{exercise} 
Suppose $I$ is an ideal in $R$. Let $M$ be an $R$-module such that $IM \neq
M$. Show that $\depth_I M \geq 2$ if and only if the natural map
\[ M \simeq \hom(R, M) \to \hom(I, M)  \]
is an isomorphism.
\end{exercise} 


\subsection{Powers of regular sequences}

Regular sequences don't necessarily behave well with respect to permutation or
localization without additional hypotheses. However, in all cases they behave
well with respect to taking powers. The upshot of this is that the invariant
called \textbf{depth} that we will soon introduce is invariant under passing to
the radical.

We shall deduce this from the following easy fact.
\begin{lemma} 
Suppose we have an exact sequence of $R$-modules
\[  0 \to M' \to M \to M'' \to 0.  \]
Suppose the sequence $x_1, \dots, x_n \in R$ is $M'$-regular and $M''$-regular.
Then it is $M$-regular.
\end{lemma} 
The converse is not true, of course.
\begin{proof} 
Morally, this is the snake lemma. For instance, the fact that multiplication by
$x_1$ is injective on $M', M''$ implies by the snake diagram that $M
\stackrel{x_1}{\to} M$ is injective. However, we don't a priori know that a
simple inductive argument on $n$ will work to prove this. The reason is that it needs
to be seen that quotienting each term by $(x_1, \dots, x_{n-1})$ will preserve
exactness. However, a general fact will tell us that this is indeed the case.
See below.

Anyway, this general fact now lets us induct on $n$. If we assume
that $x_1, \dots, x_{n-1}$ is $M$-regular, we need only prove that $x_{n}:
M/(x_1, \dots, x_{n-1})M
\to M/(x_1, \dots, x_{n-1})$ is injective. (It is not surjective or the
sequence would not be $M''$-regular.) But we have the exact sequence by the
next lemma,
\[ 0 \to M'/(x_1 \dots x_{n-1})M' \to M/(x_1 \dots x_{n-1})M \to  M''/(x_1
\dots x_{n-1})M'' \to 0 \]
and the injectivity of $x_n$ on the two ends implies it at the middle by the
snake lemma.
\end{proof} 

So we need to prove:
\begin{lemma} 
Suppose $0 \to M' \to M \to M' \to 0$ is a short exact sequence. Let $x_1,
\dots, x_m$ be an $M''$-sequence. Then the sequence
\[ 0 \to M'/(x_1 \dots x_m)M' \to M/(x_1 \dots x_m)M \to   M''/(x_1 \dots
x_m)M'' \to 0\]
is exact as well.
\end{lemma} 
One argument here uses the fact that the Tor functors vanish when you have a
regular sequence like this. We can give a direct argument. It is really just
pure diagram-chasing though...
\begin{proof} 
By induction, this needs only be proved when $m=1$, since we have the recursive
description of regular sequences: in general, $x_2 \dots x_m$ will be regular
on $M''/x_1 M''$. 
In any case, we have exactness except possibly at the left as the tensor
product is right-exact. So let $m' \in M'$; suppose $m'$ maps to a multiple of
$x_1$ in $M$. We need to show that $m'$ is a multiple of $x_1$ in $M'$. 

Suppose $m'$ maps to $x_1 m$. Then $x_1m$ maps to zero in $M''$, so by regularity $m$
maps to zero in $M''$. Thus $m$ comes from something, $\overline{m}'$, in $M'$. In particular
$m' - x_1 \overline{m}'$ maps to zero in $M$, so it is zero in $M'$. Thus
indeed $m'$ is a multiple of $x_1$ in $M'$.
\end{proof} 
So here is the result:

\begin{proposition} 
Let $M$ be an $R$-module and $x_1, \dots, x_n$ an $M$-sequence. Then $x_1^{a_1}
,\dots, x_n^{a_n}$ is an $M$-sequence for any $a_1, \dots, a_n \in
\mathbb{Z}_{>0}$.
\end{proposition} 

\begin{proof}

The clearest way I know to see this is to use the following lemma, which tells
you that regular sequences are stable under certain reasonable actions.

\begin{lemma} 
Suppose $x_1, \dots, x_i, \dots, x_n$ and $x_1, \dots, x_i', \dots, x_n$ are
$M$-sequences for some $M$. Then so is $x_1, \dots, x_i x_i', \dots, x_n$.
\end{lemma} 

\begin{proof} 
As usual, we can mod out by $(x_1 \dots x_{i-1})$ and thus assume that $i=1$.
We have to show that if $x_1, \dots, x_n$ and $x_1', \dots, x_n$ are
$M$-sequences, then so is $x_1 x_1', \dots, x_n$.

We have an exact sequence
\[ 0 \to x_1 M/x_1 x_1' M \to M/x_1 x_1' M \to  M/x_1  M \to 0.  \]
Now $x_2, \dots, x_n$ is regular  on the last term by assumption, and also on
the first term, which is isomorphic to $M/x_1' M$ as $x_1$ acts as a
nonzerodivisor on $M$. So $x_2, \dots, x_n$ is regular on both ends, and thus
in the middle. This means that 
\[ x_1 x_1', \dots, x_n  \]
is $M$-regular. That proves the lemma. 
\end{proof} 

So we now can prove the proposition. It is trivial if $\sum a_i = n$ (i.e. if
all are $1$) it is clear. In general, we can use complete induction on $\sum
a_i$. Suppose we know the result for smaller values of $\sum a_i$. We can
assume that some $a_j >1$. 
Then  the sequence
\[ x_1^{a_1}, \dots x_j^{a_j} , \dots x_n^{a_n} \]
is obtained from the sequences
\[  x_1^{a_1}, \dots,x_j^{a_j - 1}, \dots, x_n^{a_n} \]
and
\[  x_1^{a_1}, \dots,x_j^{1}, \dots, x_n^{a_n} \]
by multiplying the middle terms. But the complete induction hypothesis implies
that both those two sequences are $M$-regular, so we can apply the lemma. 
\end{proof} 

In general, the product of two regular sequences is not a regular sequence. For
instance, consider a regular sequence $x,y$ in some f.g. module $M$ over a
noetherian local ring. Then $y,x$ is regular, but the product sequence $xy, xy$
is \emph{never} regular.


\subsection{Depth}

Constructing regular sequences sequences is a useful task. We often want to ask
how long we can make them subject to some constraint. For instance, 

\begin{definition} 
Suppose $I$ is an ideal such that $IM \neq M$. Then we define the
\textbf{$I$-depth of $M$} to be the maximum length of a maximal $M$-sequence contained
in $I$. When $R$ is a local ring and $I$ the maximal ideal, then that number is
simply called the \textbf{depth} of $M$.

The \textbf{depth} of a proper ideal $I \subset R$ is its depth on $R$.
\end{definition} 


The definition is slightly awkward, but it turns out that all maximal
$M$-sequences in $I$ have the same length. So we can use any of them to compute
the depth. 

The first thing we can prove using the above machinery is that depth is really
a ``geometric'' invariant, in that it depends only on the radical of $I$.

\begin{proposition} 
Let $R$ be a ring, $I \subset R$ an ideal, and $M$ an $R$-module
with $IM \neq M$. Then $\mathrm{depth}_I M = \mathrm{depth}_{\mathrm{Rad}(I)} M$.
\end{proposition} 
\begin{proof} 
The inequality $\mathrm{depth}_I M \leq \mathrm{depth}_{\mathrm{Rad} I} M$ is trivial, so we need only
show that if $x_1, \dots, x_n$ is an $M$-sequence in $\mathrm{Rad}(I)$, then there is
an $M$-sequence of length $n$ in $I$. For this we just take a high power
\[ x_1^N, \dots, x_n^{N}  \]
where $N$ is large enough such that everything is in $I$. We can do this as
powers of $M$-sequences are $M$-sequences.
\end{proof} 

This was a fairly easy consequence of the above result on powers of regular
sequences. On the other hand, we want to give another proof, because it will
let us do more. Namely, we will show that depth is really a function of prime
ideals.

For convenience, we set the following condition: if $IM = M$, we define
\[ \mathrm{depth}_I (M) = \infty.  \]

\begin{proposition} 
Let $R$ be a noetherian ring, $I \subset R$ an ideal, and $M$ a f.g. $R$-module. 
Then
\[ \mathrm{depth}_I M = \min_{\mathfrak{p} \in V(I)} \mathrm{depth}_{\mathfrak{p}} M .  \]
\end{proposition} 

So the depth of $I$ on $M$ can be calculated  if you know the depths at each
prime containing $I$. In this sense, it is clear that $\mathrm{depth}_I (M)$ depends
only on $V(I)$ (and the depths on those primes), so clearly it depends only on
$I$ \emph{up to radical}.

\begin{proof} 
In this proof, we shall \textbf{use the fact that the length of every maximal
$M$-sequence is the same}, something which we will prove below.

It is obvious that we have an inequality
\[ \mathrm{depth}_I \leq  \min_{\mathfrak{p} \in V(I)} \mathrm{depth}_{\mathfrak{p}} M \]
as each of those primes contains $I$. 
We are to prove that there is 
a prime $\mathfrak{p}$ containing $I$ with
\[ \mathrm{depth}_I M = \mathrm{depth}_{\mathfrak{p}} M . \]
But we shall actually prove the stronger statement that there is $\mathfrak{p}
\supset I$ with $\mathrm{depth}_{\mathfrak{p}} M_{\mathfrak{p}} = \mathrm{depth}_I M$. Note
that localization at a prime can only increase depth because an $M$-sequence in
$\mathfrak{p}$ leads to an $M$-sequence in $M_{\mathfrak{p}}$ thanks to
Nakayama's lemma and the flatness of localization.

So let $x_1, \dots, x_n \in I$ be a $M$-sequence of maximum length. Then $I$
acts by zerodivisors on 
$M/(x_1 , \dots, x_n) M$ or we could extend the sequence further. 
In particular, $I$ is contained in an associated prime of $M/(x_1, \dots, x_n)
M$ by elementary commutative algebra (basically, prime avoidance).

Call this associated prime $\mathfrak{p} \in V(I)$. Then $\mathfrak{p}$ is an
associated prime of $M_{\mathfrak{p}}/(x_1, \dots, x_n) M_{\mathfrak{p}}$,
and in particular acts only by zerodivisors on this module. 
Thus the $M_{\mathfrak{p}}$-sequence $x_1, \dots, x_n$ can be extended no
further in $\mathfrak{p}$. In particular, since as we will see soon, the depth
can be computed as the length of any maximal $M_{\mathfrak{p}}$-sequence,
\[ \mathrm{depth}_{\mathfrak{p}} M_{\mathfrak{p}} = \mathrm{depth}_I M. \]
\end{proof} 

Perhaps we should note a corollary of the argument above:
\begin{corollary} 
Hypotheses as above, we have $\mathrm{depth}_I M  \leq \mathrm{depth}_\mathfrak{p} M_{\mathfrak{p}}$ for
any prime $\mathfrak{p} \supset I$. However, there is at least one $\mathfrak{p}
\supset I$ where equality holds. \end{corollary}

\subsection{Depth and dimension}

Consider an $R$-module $M$, which is always assumed to be finitely generated.
Let $I \subset R$ be an ideal with $IM \neq M$. We know
that if $x \in I$ is a nonzerodivisor on $M$, then $x$ is part of a maximal
$M$-sequence in $I$, which has length $\mathrm{depth}_I M$ necessarily.
It follows that $M/xM$ has a $M$-sequence of length $\mathrm{depth}_I M - 1$ (because
the initial $x$ is thrown out) which can be extended no further.

In particular, we find
\begin{proposition} 
Hypotheses as above, let $x \in I$ be a nonzerodivisor on $M$. Then 
\[ \mathrm{depth}_I (M/xM) = \mathrm{depth} M - 1.  \]
\end{proposition} 

This is strikingly analogous to the dimension of the module $M$. 
Recall that $\dim M$ is defined to be the Krull dimension of the topological
space  $\supp M = V( \mathrm{Ann} M)$ for $\mathrm{Ann} M$ the annihilator of $M$.
But the ``generic points'' of the topological space $V(\mathrm{Ann} M)$, or the
smallest primes in $\supp M$, are precisely the associated primes of $M$. 
So if $x$ is a nonzerodivisor on $M$, we have that $x$ is not contained
in any associated primes of $M$, so that $\supp(M/xM)$ must have smaller
dimension than $\supp M$. That is,
\[ \dim M/xM \leq \dim M - 1.  \]
But I claim that we have in fact equality. 

\begin{lemma} 
For any f.g. $R$-module $M$, we have $\dim M /xM \geq \dim M - 1$.
\end{lemma} 
\begin{proof} We have $\supp(M/xM) = \supp M \cap V(x)$. Indeed, this is easily
seen by localization: if $\mathfrak{p}$ is such that $M_{\mathfrak{p}}/x
M_{\mathfrak{p}} \neq 0$, then $M_{\mathfrak{p}} \neq 0$ and $\mathfrak{p} \in
\supp M$. Similarly, $x$ is a non-unit in $R_\mathfrak{p}$ and thus $x \in
\mathfrak{p}$, so $\mathfrak{p} \in V(x)$. The converse is proved the same way
using Nakayama's lemma.

But we know that Krull's principal ideal theorem says that the dimension of a
closed set intersected with a ``hypersurface'' like $V(x)$ is at least the
initial dimension minus one. So this gives the other inequality. 
\end{proof} 

In particular, we deduce:
\begin{proposition} 
Let $M$ be a f.g. module over the noetherian ring $R$. Then
\[ \mathrm{depth}_I M \leq \dim M  \]
for any ideal $I \subset R$ with $IM \neq M$.
\end{proposition} 
\begin{proof} 
Indeed, if $x_1, \dots, x_r$ is a maximal $M$-sequence in $I$, then 
\[ \dim M/(x_1 , \dots, x_r) M = \dim M - r  \]
by the above remarks. 
This implies that $r \leq \dim M$. That proves the result. 
\end{proof} 

This does not tell us much about how $\mathrm{depth}_I M$ depends on $I$, though; it
just says something about how it depends on $M$. In particular, it is not very
helpful when trying to estimate $\mathrm{depth} I = \mathrm{depth}_I R$.
Nonetheless, there is a somewhat stronger result, which we will need in the
future.

\begin{proposition} 
Hypotheses as above, $\mathrm{depth}_I M$ is at most the length of every  chain
of primes in $\mathrm{Spec} R$ that starts at an associated prime of $M$ and
ends at a prime containing $I$.
\end{proposition} 

\begin{proof} Consider a chain of primes $\mathfrak{p}_0 \subset \dots \subset
\mathfrak{p}_k$ where $\mathfrak{p}_0$ is an associated prime and
$\mathfrak{p}_k$ contains $I$. 
The goal is to show that 
\[ k \leq \mathrm{depth}_I M.  \]
By localization, we can assume that $\mathfrak{p}_k$ is the maximal ideal of
$R$; recall that localization can only increase the depth.

In this case, the argument has become:
\begin{lemma} 
Let $(R,\mathfrak{m})$ be a noetherian local ring. Let $M$ be a finite
$R$-module. Then the depth of $\mathfrak{m}$ on $M$ is at most the dimension of
$R/\mathfrak{p}$ for $\mathfrak{p}$ an associated prime of $M$.
\end{lemma} 

To prove this, first assume that the depth is zero. In that case, the result is
immediate. We shall now argue inductively.
Assume that that this is true for modules of smaller depth. 
We will quotient out appropriately to shrink the
support and change the associated 
primes. Namely, choose a $M$-regular (nonzerodivisor on $M$) $x \in R$. 
Then $\mathrm{depth}_I M/xM = \mathrm{depth}_I M -1$. 

Let $\mathfrak{p}_0$ be an associated prime of $R$.
I claim that $\mathfrak{p}_0$ is properly contained in an associated prime of
$M/xM$. Indeed, $x \notin \mathfrak{p}_0$, so $\mathfrak{p}_0$ cannot itself be
an associated prime. 
However, $\mathfrak{p}_0$ annihilates a nonzero element of $M/xM$. To see this,
consider maximal principal submodule of $M$ annihilated by $\mathfrak{p}_0$.
Let this submodule be $Rz$ for some $z \in M$. Then if $z$ is a multiple of
$x$, say $z = xz'$, then $Rz'$ would be a larger
submodule of $M$ annihilated by $\mathfrak{p}_0$---here we are using the fact
that $x$ is a nonzerodivisor on $M$. So the image of this $z$ in $M/xM$ is
nonzero and is clearly annihilated by $\mathfrak{p}_0$. 
Thus $\mathfrak{p}_0$ is contained in an associated prime of $M/xM$. Call this
prime $\mathfrak{q}_0$. 

Now we know that $\mathrm{depth}_I M/xM = \mathrm{depth}_I M -1$. Also, by the inductive
hypothesis, we know that $\dim R/\mathfrak{q}_0 \geq \mathrm{depth}_I M/xM = \mathrm{depth}_I M
-1$. But the dimension of $R/\mathfrak{q}_0$ is strictly greater than that of
$R/\mathfrak{p}_0$, so at least $\dim R/\mathfrak{q}_0 +1 = \mathrm{depth}_I M$. This
proves the lemma.
\end{proof} 




\section{Cohen-Macaulayness}

\begin{definition}
Let $(R, \mathfrak{m})$ be a noetherian local ring. Then
we set $\depth R$ to be the its depth as an $R$-module.
\end{definition}




\begin{example}
If $R$ is regular, then $\depth R = \dim R$.
\end{example}
\begin{proof}
Induction on $\dim R$. If $\dim R=0$, then this is obvious by
the inequality
$\leq $ which is always true.

Suppose $\dim R = 0$. Then $\mathfrak{m} \neq 0$ and in
particular
$\mathfrak{m}/\mathfrak{m}^2 \neq 0$. Choose $x \in
\mathfrak{m}-
\mathfrak{m}^2$. Let $R'=R/(x)$. We know that $\dim R' = \dim
R-1$ as $x$ is a
nonzerodivisor (by regularity). On the other hand, the embedding
dimension of $R'$ also drops
by one, as we have divided out by something in $\mathfrak{m} -
\mathfrak{m}^2$.
In particular, $R'$ is regular local too. So the inductive
hypothesis states
that
\[\depth R-1 =  \depth R' = \dim R' = \dim R -1.   \]

Differently phrased, we could choose $x_1, \dots, x_n \in
\mathfrak{m}$ which forms a basis for
$\mathfrak{m}/\mathfrak{m}^2$; this is a
\emph{regular sequence} (that is, an $R$-regular sequence) by
this argument. It
is maximal as $x_1, \dots, x_n$ generate $\mathfrak{m}$ and
$R/(x_1, \dots,
x_n)$ clearly has depth zero.
\end{proof}

More generally:
\begin{definition}
A noetherian local ring $(R, \mathfrak{m})$ is called
\textbf{Cohen-Macaulay}
if $\dim R = \depth R$. A general noetherian ring $R$ is
\textbf{Cohen-Macaulay} if
$R_{\mathfrak{p}}$ is Cohen-Macaulay for all $\mathfrak{p} \in
\spec R$.
\end{definition}
For instance, any regular local ring is Cohen-Macaulay, as is
any local
artinian ring (because the dimension is zero for an artinian
ring).

We shall eventually prove:

\begin{proposition}
Let $R = \mathbb{C}[X_1, \dots, X_n]/\mathfrak{p}$ for
$\mathfrak{p}$ prime.
Choose an injective map $\mathbb{C}[y_1, \dots, y_n]
\hookrightarrow R$ making $R$ a
finite module. Then $R$ is Cohen-Macaulay iff $R$ is projective
as a module
over $\mathbb{C}[y_1, \dots, y_n]$.\footnote{In fact, this is
equivalent to
freeness, although we will not prove it. Any projective finite
module over a
polynomial ring over a field is free, though this is a hard
theorem.}
\end{proposition}

The picture is that the inclusion $\mathbb{C}[y_1, \dots, y_m ]
\hookrightarrow
\mathbb{C}[x_1, \dots, x_n]/\mathfrak{p}$ corresponds to a map
\[ X \to \mathbb{C}^m  \]
for $X = V(\mathfrak{p}) \subset \mathbb{C}^n$. This statement
of freeness is a
statement about how the fibers of this finite map stay similar
in some sense.

\begin{example}
Consider $\mathbb{C}[x,y]/(xy)$, the coordinate ring of the
union of two axes
intersecting at the origin. This is Cohen-Macaulay (but not
regular, as it
is not a domain). Indeed, we can project the associated variety
$X = V(xy)$
onto the affine line by adding the coordinates. This corresponds
to the map
\[ \mathbb{C}[z] \to \mathbb{C}[x,y]/(xy)  \]
sending $z \to x+y$. This makes $\mathbb{C}[x,y]/(xy)$ into a
free
$\mathbb{C}[z]$-module of rank two (with generators $1, x$), as
one can check.
So by the previous result (strictly speaking, its extension to
non-domains),
the ring in question is Cohen-Macaulay.
\end{example}

\begin{example}
$R=\mathbb{C}[x,y,z]/(xy, xz)$ is not Cohen-Macaulay (at the
origin). The associated variety looks
geometrically like the union of the plane $x=0$ and the line
$y=z=0$ in affine
3-space. Here there are two components of different dimensions
intersecting.
Let's choose a regular sequence (that is, regular after
localization at the
origin). The dimension at the origin is clearly two because of
the plane.
First, we need a nonzerodivisor in this ring, which vanishes at
the origin, say
$ x+y+z$. (\textbf{Exercise:} Check this.) When we quotient by
this, we get
\[ S=\mathbb{C}[x,y,z]/(xy,xz, x+y+z) = \mathbb{C}[y,z]/(
(y+z)y, (y+z)z). \]

The claim is that $S$ localized at the ideal corresponding to
$(0,0)$ has depth
zero. We have $y+z \neq 0$, which is killed by both $y,z$, and
hence by the
maximal ideal at zero. In particular the maximal ideal at zero
is an associated
prime, which implies the claim about the depth.
\end{example}

As it happens, a Cohen-Macaulay variety is always
equidimensional. The rough
reason is that each irreducible piece puts an upper bound on the
depth given by
the dimension of the piece. If any piece is too small, the total
depth will be
too small.

Anyway, we shall not say much more about Cohen-Macaulayness, but
instead focus
on understanding regular local rings. We want, for next time,
to understand the relationship
between depth and lengths of projective resolutions.
We will prove:

\begin{theorem}[Auslander-Buchsbaum formula] Let $(R,
\mathfrak{m})$ be a
noetherian local ring and $M$ a finite $R$-module. Suppose $M$
has a finite
projective resolution of length $d$, where $d$ is minimal.

Then
\[ d = \depth(R) - \depth(M).  \]
\end{theorem}
So in a sense, depth measures how far $M$ is from being a free
module. If the
depth is large, then you need a lot of projective modules to
resolve $M$.
\lecture{11/22}

Last time we were talking about depth. Let's use this to
reformulate a few
definitions made earlier.

\subsection{Reduced rings}
Recall that a noetherian ring is \textbf{reduced} iff:
\begin{enumerate}
\item For any minimal prime $\mathfrak{p} \subset R$,
$R_{\mathfrak{p}}$ is a
field.
\item Every associated prime of $R$ is minimal.
\end{enumerate}

Condition 1 can be reduced as follows. To say that
$\mathfrak{p}\subset R$ is
minimal is to say that it is zero-dimensional, and that is
regular iff it is a
field. So the first condition is that \emph{for every height
zero prime,
$R_{\mathfrak{p}}$ is regular.} For the second condition,
$\mathfrak{p} \in
\ass(R)$ iff $\mathfrak{p} \in \ass(R_{\mathfrak{p}})$, which is
equivalent to
$\depth R_{\mathfrak{p}} = 0$.

Namely, the two conditions are:
\begin{enumerate}
\item For every height zero prime $\mathfrak{p} $,
$R_{\mathfrak{p}}$ is
regular.
\item For every prime $\mathfrak{p}$ of height $>0$, $\depth
R_{\mathfrak{p}} >
0$.
\end{enumerate}

Condition two is always satisfied in a Cohen-Macaulay ring.

\subsection{Serre's criterion again}

Recall that
\begin{definition}
A noetherian ring is \textbf{normal} iff it is a finite direct
product of
integrally closed domains.
\end{definition}

In the homework, we showed:
\begin{proposition}
A reduced ring $R$ is normal iff
\begin{enumerate}
\item For every height one prime $\mathfrak{p}  \in \spec R$,
$R_{\mathfrak{p}}$ is a DVR (i.e. regular).
\item For every nonzerodivisor $x \in R$, every associated prime
of $R/x$ is
minimal.
\end{enumerate}
\end{proposition}
(We had proved this for \emph{domains} earlier.)
These conditions are equivalent to:
\begin{enumerate}
\item For every prime $\mathfrak{p}$ of height $\leq 1$,
$R_{\mathfrak{p}} $ is regular.
\item For every prime $\mathfrak{p}$ of height $\geq 1$,
$\depth R_{\mathfrak{p}} \geq 1$ (necessary for reducedness)
\item $\depth R_{\mathfrak{p}} \geq 2$ for $\mathfrak{p}$ not
minimal over any
principal ideal $(x)$ for $x$ a nonzerodivisor. Condition three
is the last
condition of the proposition as quotienting out by $x$ drops the
depth by one.
\end{enumerate}

The first and third conditions imply the second. In particular,
we find:

\begin{theorem}[Serre's criterion] A noetherian ring is normal
iff:
\begin{enumerate}
\item For every prime $\mathfrak{p}$ of height $\leq 1$,
$R_{\mathfrak{p}} $ is regular.
\item $\depth R_{\mathfrak{p}} \geq 2$ for $\mathfrak{p}$ not
minimal over any
principal ideal $(x)$ for $x$ a nonzerodivisor. \end{enumerate}
\end{theorem}
For a Cohen-Macaulay ring, the last condition is automatic, as
the depth is the
codimension.

\subsection{Projective dimension}

\newcommand{\pr}{\mathrm{pd}}
Let $R$ be a commutative ring, $M$ an $R$-module.

\begin{definition}
The \textbf{projective dimension} of $M$ is the largest integer
$n$ such that
there exists  a module $N$ with
\[ \ext^n(M,N) \neq 0.  \]
(If no such $n,N$ exist, then we say that the projective
dimension is $\infty$.)
We write $\pr(M)$ for the projective dimension.
\end{definition}

\begin{remark}
$\pr(M) = 0$ iff $M$ is projective. Indeed, we have seen that
the $\ext$ groups
$\ext^i(M,N), i >0$
vanish always.
\end{remark}

If you wanted to compute the projective dimension, you could go
as follows.
Take any $M$. Choose a surjection $P \twoheadrightarrow M$ with
$P$ projective;
call the kernel $K$ and draw a short exact sequence
\[ 0 \to K \to P \to M \to 0.  \]
For any $R$-module $N$, we have a long exact sequence
\[ \ext^{i-1}(P,N) \to \ext^{i-1}(K,N) \to \ext^i(M,N) \to
\ext^i(P, N). \]
If $i >0$, the right end vanishes; if $i >1$, the left end
vanishes. So if $i
>1$, this map $\ext^{i-1}(K,N) \to \ext^i(M,N)$ is an
\emph{isomorphism}.

Suppose that $\pr(K) = d \geq 0$. We find that
$\ext^{i-1}(K,N)=0$ for $i-1
> d$.
This implies that $\ext^i(M,N) = 0$ for such $i > d+1$. In
particular, $\pr(M)
\leq d+1$.
This argument is completely reversible if $d >0$.
Then we see from these isomorphisms that
\[ \boxed{\pr(M) = \pr(K)+1}, \quad \mathrm{unless} \ \pr(M)=0
\]
If $M$ is projective, the sequence $0 \to K \to P \to M \to 0$
splits, and
$\pr(K)=0$ too.

The upshot is that \textbf{we can compute projective dimension
by choosing a
projective resolution.}
\begin{proposition}
Let $M$ be an $R$-module. Then $\pr(M) \leq n$ iff there exists
a finite
projective resolution of $M$ having $n+1$ terms,
\[ 0 \to P_n \to \dots \to P_1 \to P_0 \to M \to 0.  \]
\end{proposition}
\begin{proof}
Induction on $n$. When $n = 0$, $M$ is projective, and we can
use the
resolution $0 \to M \to M \to 0$.

Suppose $\pr(M) \leq n$, where $n >0$. We can get a short exact
sequence
\[ 0 \to K \to P_0 \to M \to 0  \]
with $P_0$ projective, so $\pr(K) \leq n-1$. The inductive
hypothesis implies
that there is a projective resolution of $K$ of length $\leq
n-1$. We can
splice this in with the short exact sequence to get a projective
resolution of
$M$ of length $n$.

The argument is reversible. Choose any projective resolution
\[  0 \to P_n \to \dots \to P_1 \to P_0 \to M \to 0 \]
and split into short exact sequences, and argue inductively.
\end{proof}


Let $\pr(M) = n$. Choose any projective resolution $\dots \to
P_2 \to P_1 \to
P_0 \to M$. Choose $K_i = \ker(P_i \to P_{i-1})$ for each $i$.
Then there is a short exact sequence $0 \to K_0 \to P_0 \to M
\to 0$. Moreover,
there are exact sequences
\[ 0 \to K_i \to P_i \to K_{i-1} \to 0  \]
for each $i$. From these, we see that the projective dimensions
of the $K_i$
drop by one as $i$ increments. So $K_{n-1}$ is projective if
$\pr(M) = n$ as
$\pr(K_{n-1})=0$. In particular, we can get a projective
resolution
\[ 0 \to K_{n-1} \to P_{n-1} \to \dots \to P_0 \to M \to 0  \]
which is of length $n$.
In particular, if you ever start trying to write a projective
resolution of
$M$, you can stop after going out $n$ terms, because the kernels
will become
projective.


\subsection{Minimal projective resolutions}
Usually projective resolutions are non-unique. But sometimes
they kind of are.
Let $(R, \mathfrak{m})$ be a local noetherian ring, $M$ a
finitely generated $R$-module.

\begin{definition}
A projective resolution $P_* \to M$ of finitely generated
modules is \textbf{minimal} if for each $i$, the
induced map $P_i \otimes R/\mathfrak{m} \to P_{i-1} \otimes
R/\mathfrak{m}$ is
zero, and same for $P_0 \otimes R/\mathfrak{m} \to
M/\mathfrak{m}M$.
\end{definition}

This is equivalent to saying that for each $i$, the map $P_i
\to\ker(P_{i-1}
\to P_{i-2})$ is an isomorphism modulo $\mathfrak{m}$.

\begin{proposition}
Every $M$ (over a local noetherian ring) has a minimal
projective resolution.
\end{proposition}
\begin{proof}
Start with a module $M$. Then $M/\mathfrak{m}M$ is a
finite-dimensional vector
space over $R/\mathfrak{m}$, of dimension say $d_0$. We can
choose a basis for that vector space, which
we can lift to $M$. That determines a map of free modules
\[ R^{d_0} \to M,  \]
which is a surjection by Nakayama's lemma. It is by construction
an
isomorphism modulo $\mathfrak{m}$. Then define $K =
\ker(R^{d_0}\to M)$; this
is finitely generated by noetherianness, and we
can do the same thing for $K$, and repeat to get a map $R^{d_1}
\twoheadrightarrow K$ which is an isomorphism modulo
$\mathfrak{m}$. Then
\[ R^{d_1} \to R^{d_0} \to M \to 0  \]
is exact, and minimal; we can continue this by the same
procedure.
\end{proof}


\begin{proposition}
Minimal projective resolutions are unique up to isomorphism.
\end{proposition}
\begin{proof}
Suppose we have one minimal projective resolution:
\[ \dots \to P_2 \to P_1 \to P_0 \to M \to 0  \]
and another:
\[ \dots \to Q_2 \to Q _1 \to Q_0 \to M \to 0  .\]
There is always a map of projective resolutions $P_* \to Q_*$ by
general
homological algebra. There is, equivalently, a commutative
diagram
\[\xymatrix{ \dots \ar[d] \ar[r] & P_2\ar[d] \ar[r] & P_1
\ar[d]\ar[r]
& P_0 \ar[d] \ar[r] & M \ar[d]^{\mathrm{id}} \ar[r] & 0 \\
 \dots  \ar[r] &   Q_2  \ar[r] &  Q_1   \ar[r]
&  Q_0   \ar[r] &   M  \ar[r] &   0 } \]
If both resolutions are minimal, the claim is that this map is
an isomorphism.
That is, $\phi_i: P_i \to Q_i$ is an isomorphism, for each $i$.

To see this, note that $P_i, Q_i$ are finite free
$R$-modules.\footnote{We are
using the fact that a finite projective module over a local ring
is
\emph{free}.} So $\phi_i$ is an isomorphism iff $\phi_i$ is an
isomorphism
modulo the maximal ideal, i.e. if
\[ P_i/\mathfrak{m}P_i \to Q_i/\mathfrak{m}Q_i  \]
is an isomorphism. Indeed, if $\phi_i$ is an isomorphism, then
its tensor
product with $R/\mathfrak{m}$ obviously is an isomorphism.
Conversely suppose
that the reductions mod $\mathfrak{m}$ make an isomorphism. Then
the ranks of
$P_i, Q_i$ are the same, and $\phi_i$ is an $n$-by-$n$ matrix
whose determinant
is not in the maximal ideal, so is invertible. This means that
$\phi_i$ is invertible by the
usual formula for the inverse matrix.

So we are to check that $P_i / \mathfrak{m}P_i \to Q_i /
\mathfrak{m}Q_i$ is an
isomorphism for each $i$. This is equivalent to the assertion
that
\[ (Q_i/\mathfrak{m}Q_i)^{\vee} \to
(P_i/\mathfrak{m}P_i)^{\vee}\]
is an isomorphism. But this is the map
\[ \hom_R(Q_i, R/\mathfrak{m}) \to \hom_R(P_i, R/\mathfrak{m}).
\]
If we look at the chain complexes $\hom(P_*, R/\mathfrak{m}),
\hom(Q_*,
R/\mathfrak{m})$, the cohomologies
compute the $\ext$ groups of $(M, R/\mathfrak{m})$. But all the
maps in this
chain complex are zero because the resolution is minimal, and we
have that the
image of $P_i$ is contained in $\mathfrak{m}P_{i-1}$ (ditto for
$Q_i$). So the
cohomologies are just the individual terms, and the maps
$ \hom_R(Q_i, R/\mathfrak{m}) \to \hom_R(P_i, R/\mathfrak{m})$
correspond to
the identities on $\ext^i(M, R/\mathfrak{m})$. So these are
isomorphisms.\footnote{We are sweeping under the rug the
statement that $\ext$
can be computed via \emph{any} projective resolution. More
precisely, if you
take any two projective resolutions, and take the induced maps
between the
projective resolutions, hom them into $R/\mathfrak{m}$, then the
maps on
cohomology are isomorphisms.}
\end{proof}


\begin{corollary}
If $\dots \to P_2 \to P_1 \to P_0 \to M$ is a minimal projective
resolution of
$M$, then the ranks $\mathrm{rank}(P_i)$ are well-defined (i.e.
don't depend
on the choice of the minimal resolution).
\end{corollary}
\begin{proof}
Immediate from the proposition. In fact, the ranks are the
dimensions (as
$R/\mathfrak{m}$-vector spaces) of $\ext^i(M, R/\mathfrak{m})$.
\end{proof}

Let us advertise the goal for next time. We would like to prove
Serre's
criterion for regularity.

\begin{theorem}
Let $(R, \mathfrak{m})$ be a local noetherian ring. Then $R$ is
regular iff
$R/\mathfrak{m}$ has finite projective dimension. In this case,
$\pr(R/\mathfrak{m}) = \dim R$.
\end{theorem}
\lecture{11/24}

\subsection{The Auslander-Buchsbaum formula}

Today, we shall start by proving:

\begin{theorem}[Auslander-Buschsbaum formula]
Let $R$ be a local noetherian ring, $M$ a f.g. $R$-module of
finite
projective dimension. If $\pr(R) <
\infty$, then $\pr(M) = \depth(R) - \depth(M)$.
\end{theorem}

\begin{proof}
Induction on $\pr(M)$. When $\pr(M)=0$, then $M$ is projective,
so isomorphic
to $R^n$ for some $n$. Thus $\depth(M) = \depth(R)$.

Assume $\pr(M) > 0$.
Choose a surjection $P \twoheadrightarrow M$ and write an exact
sequence
\[ 0 \to K \to P \to M \to 0,  \]
where $\pr(K) = \pr(M)-1$. We also know by induction that
\[ \pr(K) = \depth R - \depth(K).  \]
What we want to prove is that
\[ \depth R - \depth M = \pr(M) = \pr(K)+1.  \]
This is equivalent to wanting know that $\depth(K) = \depth (M)
+1$.
In general, this may not be true, though, but we will prove it
under
minimality hypotheses.

Without loss of generality, we can choose that $P$ is
\emph{minimal}, i.e.
becomes an isomorphism modulo the maximal ideal $\mathfrak{m}$.
This means
that the rank of $P$ is $\dim M/\mathfrak{m}M$.
So $K = 0$ iff $P \to M$ is an isomorphism; we've assumed that
$M$ is not
free, so $K \neq 0$.

Recall that the depth of $M$ is the smallest value $i$ such
that$\ext^i(R/\mathfrak{m}, M) \neq 0$. So we should look at the long exact
sequence from the above short exact sequence:
\[ \ext^i(R/\mathfrak{m}, P) \to  \ext^i(R/\mathfrak{m},M)  \to
\ext^{i+1}(R/\mathfrak{m}, K) \to \ext^{i+1}(R/\mathfrak{m},
P).\]
Now $P$ is just a direct sum of copies of $R$, so
$\ext^i(R/\mathfrak{m}, P)$
and $\ext^{i+1}(R/\mathfrak{m}, P)$ are zero if $i+1< \depth R$.
In
particular, if $i+1< \depth R$, then the map $
\ext^i(R/\mathfrak{m},M) \to
\ext^{i+1}(R/\mathfrak{m}, K) $ is an isomorphism.
So we find that $\depth M + 1 = \depth K$ in this case.

We have seen that \emph{if $\depth K < \depth R$, then } by
taking $i$ over
all integers $< \depth K$, we find that
\[ \ext^{i}(R/\mathfrak{m}, M) = \begin{cases}
0 & \mathrm{if \ } i+1 < \depth K \\
\ext^{i+1}(R/\mathfrak{m},K) & \mathrm{if \ } i+1 = \depth K
\end{cases}. \]
In particular, we are \textbf{done} unless $\depth K \geq \depth
R$.
By the inductive hypothesis, this is equivalent to saying that
$K$ is
projective.

So let us consider the case where $K$ is projective, i.e.
$\pr(M)=1$.
We want to show that $\depth M = d-1$ if $d = \depth R$.
We need a
slightly different argument in this case. Let $d = \depth(R) =
\depth (P) =
\depth(K)$ since $P,K$ are free. We have a short exact sequence
\[ 0 \to K \to P \to M \to 0  \]
and a long exact sequence of $\ext$ groups:
\[ 0 \to \ext^{d-1}(R/\mathfrak{m}, M) \to
\ext^d(R/\mathfrak{m}, K) \to \ext^d(R/\mathfrak{m}, P) .\]
We know that $\ext^d(R/\mathfrak{m}, K)$ is nonzero as $K$ is
free and $R$ has
depth $d$. However, $\ext^i(R/\mathfrak{m}, K) =
\ext^i(R/\mathfrak{m}, P)=0$
for $i<d$. This implies that $\ext^{i-1}(R/\mathfrak{m}, M)=0$
for $i<d$.

We will show:
\begin{quote}
The map $\ext^d(R/\mathfrak{m}, K) \to \ext^{d}(R/\mathfrak{m},
P)$ is zero.
\end{quote}
This will imply that the depth of $M$ is \emph{precisely} $d-1$.
This is because the matrix $K \to P$ is given by multiplication
by a matrix
with coefficients in $\mathfrak{m}$ as $K/\mathfrak{m}K \to
P/\mathfrak{m}P$
is zero. In particular, the map on the $\ext$ groups is zero,
because it is
annihilated by $\mathfrak{m}$.
\end{proof}

\begin{example}
Let $R = \mathbb{C}[x_1, \dots, x_n]/\mathfrak{p}$ for
$\mathfrak{p}$ prime.
Choose an injection $R' \to R$ where $R' = \mathbb{C}[y_1,
\dots, y_m]$ and
$R$ is a f.g. $R'$-module. This exists by the Noether
normalization lemma.

We wanted to show:

\begin{theorem}
$R$ is Cohen-Macaulay\footnote{That is, its localizations at any
prime---or,
though we haven't proved yet, at any maximal ideal---are.} iff
$R$ is a
projective $R'$-module.
\end{theorem}

We shall use the fact that projectiveness can be tested locally
at every
maximal ideal.

\begin{proof}
Choose a maximal ideal $\mathfrak{m} \subset R'$. We will show
that
$R_{\mathfrak{m}}$ is a free $R'_{\mathfrak{m}}$-module via the
injection of
rings $R'_{\mathfrak{m}} \hookrightarrow R_{\mathfrak{m}}$
(where
$R_{\mathfrak{m}}$ is defined as $R$ localized at the
multiplicative subset
of elements of $R' - \mathfrak{m}$) at each $\mathfrak{m}$ iff
Cohen-Macaulayness holds.

Now $R'_{\mathfrak{m}}$ is a regular local ring, so its depth is
$m$. By the
Auslander-Buchsbaum formula, $R_{\mathfrak{m}}$ is projective as
an
$R'_{\mathfrak{m}}$-module iff
\[ \depth_{R'_{\mathfrak{m}}} R_{\mathfrak{m}} = m.  \]
Now $R$ is a projective module iff the above condition holds for
all maximal
ideals $\mathfrak{m} \subset R'$. The claim is that this is
equivalent to
saying that $\depth R_{\mathfrak{n}} = m = \dim
R_{\mathfrak{n}}$
for every maximal ideal $\mathfrak{n} \subset R$ (depth over
$R$!).

These two statements are almost the same, but one is about the
depth of $R$ as
an $R$-module, and another as an $R'$-module.

\begin{quote}
Issue: There may be several maximal ideals of $R$ lying over the
maximal ideal
$\mathfrak{m} \subset R'$.
\end{quote}

The problem is that $R_{\mathfrak{m}}$ is not generally local,
and not
generally equal to $R_{\mathfrak{n}}$ if $\mathfrak{n}$ lies
over
$\mathfrak{m}$. Fortunately, depth makes sense even over
semi-local rings
(rings with finitely many maximal ideals).

Let us just assume that this does not occur, though. Let us
assume that
$R_{\mathfrak{m}}$ is a local ring for every maximal ideal
$\mathfrak{m}
\subset R$. Then we are reduced to showing that if $S =
R_{\mathfrak{m}}$,
then the depth of $S$ as an $R'_{\mathfrak{m}}$-module is the
same as the
depth as an $R_{\mathfrak{m}}$-module. That is, the depth
doesn't depend too
much on the ring, since $R'_{\mathfrak{m}}, R_{\mathfrak{m}}$
are ``pretty
close.'' If you believe this, then you believe the theorem, by
the first
paragraph.


Let's prove this claim in a more general form:

\begin{proposition}
Let $\phi: S' \to S$ be a local\footnote{I.e. $\phi$ sends
non-units into
non-units.} map of local noetherian rings such that $S$ is a
f.g.
$S'$-module. Then, for any finitely generated $S$-module $M$,
\[ \depth_S M = \depth_{S'} M.  \]
\end{proposition}
With this, the theorem will be proved.

\begin{remark}
This result generalizes to the semi-local case, which is how
one side-steps
the issue above.
\end{remark}

\begin{proof}
By induction on $\depth_{S'} M$. There are two cases.

Let $\mathfrak{m}', \mathfrak{m}$ be the maximal ideals of $S',
S$.
If $\depth_{S'}(M) >0$, then there is an element $a$ in
$\mathfrak{m}'$ such
that
\[ M \stackrel{\phi(a)}{\to} M \]
is injective. Now $\phi(a) \in \mathfrak{m}$. So $\phi(a)$ is a
nonzerodivisor, and we have an exact sequence
\[ 0 \to M \stackrel{\phi(a)}{\to} M \to M/\phi(a) M \to 0.  \]
Thus we find
\[ \depth_{S} M > 0 . \]
Moreover, we find that $\depth_S M = \depth_S (M/\phi(a) M) +1$
and
$\depth_{S'} M = \depth_{S'}(M/\phi(a) M))+1$. The inductive
hypothesis now
tells us that
\[ \depth_S M = \depth_{S'}M.  \]

The hard case is where $\depth_{S'} M = 0$. We need to show that
this is
equivalent to $\depth_{S} M = 0$. So we know at first that
$\mathfrak{m}' \in
\ass(M)$. That is, there is an element $x \in M$ such that
$\ann_{S'}(x) =
\mathfrak{m}'$.
Now $\ann_S(x) \subsetneq S$ and contains $\mathfrak{m}' S$.

$Sx \subset M$ is a submodule, surjected onto by $S$ by the map
$a \to ax$.
This map actually, as we have seen, factors through
$S/\mathfrak{m}' S$. Here
$S$ is a finite $S'$-module, so $S/\mathfrak{m}'S$ is a finite
$S'/\mathfrak{m}'$-module. In particular, it is a
finite-dimensional vector space
over a field. It is thus a local artinian ring. But $Sx$ is a
module over this
local artinian ring. It must have an associated prime, which is
a maximal
ideal in $S/\mathfrak{m}'S$. The only maximal ideal can be
$\mathfrak{m}/\mathfrak{m}'S$. It follows that $\mathfrak{m}
\in\ass(Sx)
\subset \ass(M)$.

In particular, $\depth_S M = 0$ too, and we are done.
\end{proof}

\end{proof}
\end{example}
