\chapter{Noetherian rings and modules}
\label{noetherian}

The finiteness condition of a noetherian ring is necessary for much of
commutative algebra; many of the results we prove after this will apply only (or mostly) to the
noetherian case. In algebraic geometry, the noetherian condition guarantees
that the topological space associated to the ring (the $\spec $) has all its
sets quasi-compact; this condition can be phrased as saying that the space
itself is noetherian in a certain sense. 

We shall start by proving the basic properties of noetherian rings. These are
fairly standard and straightforward; they could have been placed after
chapter~\ref{foundations}, in fact. More subtle is the structure theory for
finitely generated modules over a noetherian ring. While there is nothing as
concrete as there is for PIDs (there, one has a very explicit descrition for
the isomorphism classes), one can still construct a so-called ``primary
decomposition.'' This will be the primary focus after the basic properties of
noetherian rings and modules have been established. Finally, we finish with an
important subclass of noetherian rings, the \emph{artinian} ones.


\section{Basics}

\subsection{The noetherian condition}


\begin{definition} 
Let $R$ be a commutative ring and $M$ an $R$-module. We say that $M$ is
\textbf{noetherian} if every submodule of $M$ is finitely generated.
\end{definition} 


There is a convenient
reformulation of the finiteness hypothesis above in terms of the
\emph{ascending chain condition}.

\begin{proposition} $M$ is a module over $R$.
The following are equivalent:
\begin{enumerate}
\item $M$ is noetherian. 
\item Every chain of submodules  $M_0 \subset M_1 \subset \dots \subset M$,
eventually stabilizes at some $M_N$. (Ascending chain condition.)
\item Every nonempty collection of submodules of $M$ has a maximal element.
\end{enumerate}
\end{proposition} 
\begin{proof} 
Say $M$ is noetherian and we have such a chain
\[ M_0 \subset M_1 \subset \dots.  \]
Write
\[ M' = \bigcup M_i \subset M,  \]
which is finitely generated since $M$ is noetherian. Let it be generated by
$x_1, \dots,x_n$. Each of these finitely many elements is in the union, so
they are all contained in some $M_N$. This means that
\[ M' \subset M_N, \quad \mathrm{so} \quad M_N = M'  \]
and the chain stabilizes.

For the converse, assume the ACC.  Let $M' \subset M$ be any submodule.  Define
a chain of submodules $M_0 \subset M_1 \subset  \dots \subset M'$ inductively as follows. First, just take
$M_0 = \left\{0\right\}$. Take $M_{n+1}$ to be $M_n + Rx$ for  some $x \in
M' - M_n$, if such an $x$ exists; if not take $M_{n+1}=M_n$.  
So $M_0$ is zero,
$M_1$ is generated by some nonzero element of $M'$, $M_2$ is $M_1$ together
with some element of $M'$ not in $M_1$, and so on, until (if ever) the chain
stabilizes.

However, by construction, we have an ascending
chain, so it stabilizes at some finite place by the ascending chain condition.
Thus, at some point, it is
impossible to choose something in $M'$ that does not belong to  $M_N$. In
particular, $M'$ is generated by $N$ elements, since $M_N$ is and $M' = M_N$.
This proves the reverse implication. Thus the equivalence of 1 and 2 is clear.
The equivalence of 2 and 3 is left to the reader. 
\end{proof} 


Working with noetherian modules over non-noetherian rings can be a little
funny, though, so normally this definition is combined with:


\begin{definition} 
The ring $R$ is \textbf{noetherian} if $R$ is noetherian as an $R$-module.
Equivalently phrased, $R$ is noetherian if all of its ideals are finitely generated.
\end{definition} 

We start with the basic examples:

\begin{example} 
\begin{enumerate}
\item Any field is noetherian. There are two ideals: $(1)$ and $(0)$. 
\item Any PID is noetherian: any ideal is generated by one element. So
$\mathbb{Z}$ is noetherian.
\end{enumerate}
\end{example} 

The first basic result we want to prove is that over a noetherian ring, the
noetherian modules are precisely the finitely generated ones.  This will
follow from \cref{exactnoetherian} in the next subsection. So the defining 
property of noetherian rings is that a submodule of a finitely generated
module is finitely generated. (Compare
\cref{noetherianiffg}.)

\begin{exercise} 
The ring $\mathbb{C}[X_1, X_2, \dots]$ of polynomials in infinitely many
variables is not noetherian. Note that the ring itself is finitely generated
(by the element $1$), but there are ideals that are not finitely generated.
\end{exercise} 

\begin{exercise} 
Let $R$ be a ring such that every \emph{prime} ideal is finitely generated.
Then $R$ is noetherian. (Hint: show that, in any ring, an ideal maximal with respect to the
condition of being not finitely generated is prime.)
\end{exercise} 

\subsection{Stability properties}
\begin{proposition} \label{exactnoetherian}
If 
\[ 0 \to M' \to  M \to M'' \to 0  \]
is an exact sequence of modules, then $M$ is noetherian if and only if $M',
M''$ are.
\end{proposition} 

One direction states that noetherianness is preserved under subobjects and
quotients. The other direction states that noetherianness is preserved under
extensions. 
\begin{proof} 
If $M$ is noetherian, then every submodule of $M'$ is a submodule of $M$, so is
finitely generated. So $M'$ is noetherian too. Now we show that $M''$ is
noetherian. Let $N \subset M''$ and let
$\widetilde{N} \subset M$ the inverse image. Then $\widetilde{N}$ is finitely generated, so
$N$---as the homomorphic image of $\widetilde{N}$---is finitely generated 
So $M''$ is noetherian.

Suppose $M', M''$ noetherian. We prove $M$ noetherian.
Let's verify the ascending chain condition. Consider
\[ M_1 \subset M_2 \subset \dots \subset M.  \]
Let $M_i''$ denote the image of $M_i$ in $M''$ and let $M'_i$ be the
intersection of $M_i$ with $M'$. Here we think of $M'$ as a submodule of $M$.
These are ascending chains of submodules of $M', M''$, respectively, so they
stabilize by noetherianness.
So for some $N$, we have
that $n \geq N$ implies 
\[ M'_n = M'_{n+1}, \quad M''_n = M''_{n+1}.  \]

We claim that this implies, for such $n$, 
\[ M_n = M_{n+1}.  \]
Why? Say $x \in M_{n+1} \subset M$. Then $x$ maps into something in $M''_{n+1} = M''_n$.  
So there is something in $M_n$, call it $y$, such that $x,y$ go to the same
thing in $M''$. In particular, 
\[ x - y \in M_{n+1} \]
goes to zero in $M''$, so $x-y \in M'$. Thus $x-y \in M'_{n+1} = M'_n$. In
particular, 
\[ x = (x-y) + y \in M'_n + M_n = M_n.  \]
So $x \in M_n$, and 
\[ M_n = M_{n+1} . \]
This proves the result.
\end{proof} 

The class of noetherian modules is thus ``robust.'' We can get from that the
following.

\begin{proposition} 
If $\phi: A \to B$ is a surjection of commutative rings and $A$ is noetherian, then $B$ is
noetherian too.
\end{proposition} 
\begin{proof} 
Indeed, $B$ is noetherian as an $A$-module; indeed, it is the quotient of a
noetherian $A$-module (namely, $A$). However, it is easy to see that the
$A$-submodules of $B$ are just the $B$-modules in $B$, so $B$ is noetherian as a
$B$-module too. So $B$ is noetherian.  
\end{proof} 

Another easy stability property:

\begin{proposition} 
Let $R$ be a commutative ring, $S \subset R$ a multiplicatively closed subset.   If
$R$ is noetherian, then $S^{-1}R$ is noetherian.
\end{proposition} 
I.e., the class of noetherian rings is closed under localization.
\begin{proof} 
Say $\phi: R \to S^{-1}R$ is the canonical map. Let $I \subset S^{-1}R$ be an
ideal. Then $\phi^{-1}(I) \subset R$ is an ideal, so finitely generated. It
follows that
\[ \phi^{-1}(I)( S^{-1}R )\subset S^{-1}R  \]
is finitely generated as an ideal in $S^{-1}R$; the generators are the images
of the generators of $\phi^{-1}(I)$.

Now we claim that
\[  \phi^{-1}(I)( S^{-1}R ) = I . \]
The inclusion $\subset$ is trivial. For the latter inclusion, if $x/s \in I$,
then $x \in \phi^{-1}(I)$, so 
\[ x = (1/s) x \in (S^{-1}R) \phi^{-1}(I).  \] This proves the claim and
implies that $I$ is finitely generated.
\end{proof} 

\subsection{The basis theorem}
Let us now prove something a little less formal.

\begin{theorem}[Hilbert basis theorem]
If $R$ is a noetherian ring, then the polynomial ring $R[X]$ is noetherian.
\end{theorem} 
\begin{proof} 
Let $I \subset R[X]$ be an ideal. We prove that it is finitely generated. For
each $m \in \mathbb{Z}_{\geq 0}$, let $I(n)$ be the collection of elements 
$ a\in R$ consisting of the coefficients of $x^n$ of elements of $I$ of degree
$\leq n$.
This is an ideal, as is easily seen.

In fact, we claim that
\[ I(1) \subset I(2) \subset \dots  \]
which follows because if $ a\in I(1)$, there is an element $aX + \dots$ in $I$.
Thus $X(aX + \dots) = aX^2 + \dots \in I$, so $a \in I(2)$. And so on.

Since $R$ is noetherian, this chain stabilizes at some $I(N). $
Also, because $R$ is noetherian, each $I(n)$ is generated by finitely many
elements $a_{n,1}, \dots, a_{n, m_n} \in I(n)$. All of these come from polynomials
$P_{n,i} \in I$ such that $P_{n,i} = a_{n,i} X^n + \dots$.

The claim is that the $P_{n,i}$ for $n \leq N$ and $i \leq m_n$ generate $I$. 
This is a finite set of polynomials, so if we prove the claim, we will have
proved the basis theorem. Let $J$ be the ideal generated by
$\left\{P_{n,i}, n \leq N, i \leq m_n \right\}$. We know $J \subset I$. We must
prove $I \subset J$.

We will show that any element $P(X) \in I$ of degree $n$ belongs to $J$ by
induction on $n$. The degree is the largest nonzero coefficient. In particular,
the zero polynomial does not have a degree, but the zero polynomial is
obviously in $J$.

There are two cases. In the first case, $n \geq N$. Then we write
\[ P(X) = a X^n + \dots .  \] By definition, $a \in I(n) = I(N)$ since the
chain of ideals $I(n)$ stabilized. Thus we can write $a$ in terms of the
generators:  $a = \sum a_{N, i} \lambda_i$ for some
$\lambda_i \in R$. Define the polynomial
\[ Q = \sum \lambda_i P_{N, i} x^{n-N} \in J.  \] Then $Q$ has degree $n$ and
the leading term is just $a$.  In particular, 
\[ P - Q  \]
is in $I$ and has degree less than $n$. By the inductive hypothesis, this
belongs to $J$, and since $Q \in J$, it follows that $P \in J$. 

Now consider the case of $n < N$. 
Again, we write $P(X) = a X^n + \dots$. Then $a \in I(n)$.  We can write 
\[ a = \sum a_{n,i} \lambda_i, \quad \lambda_i \in R.  \]
But the $a_{n,i} \in I(n)$. The polynomial
\[ Q = \sum \lambda_i P_{n,i}   \]
belongs to $J$ since $n  < N$. In the same way, $P-Q \in I$ has a lower degree.
Induction as before implies that $P \in J$. 
\end{proof} 


\begin{example} 
Let $k$ be a field. Then $k[x_1, \dots, x_n]$ is noetherian for any $n$, by the
Hilbert basis theorem and induction on $n$. 
\end{example} 


\begin{example} 
Any finitely generated commutative ring $R$ is noetherian. Indeed, then there
is a surjection
\[ \mathbb{Z}[x_1, \dots, x_n] \twoheadrightarrow R  \]
where the $x_i$ get mapped onto generators in $R$. The former is noetherian by
the basis theorem, and $R$ is as a quotient noetherian. 
\end{example} 


\begin{corollary} 
Any ring $R$ can be obtained as a filtered direct limit of noetherian rings.
\end{corollary} 
\begin{proof} 
Indeed, $R$ is the filtered direct limit of its finitely generated subrings. 
\end{proof} 
This observation is sometimes useful in commutative algebra and algebraic
geometry, in order to reduce questions about arbitrary commutative rings to
noetherian rings. Noetherian rings have strong finiteness hypotheses that let
you get numerical invariants that may be useful. For instance, we can do things
like inducting on the dimension for noetherian local rings.

\begin{example} 
Take $R = \mathbb{C}[x_1, \dots, x_n]$. For any algebraic variety $V$ defined
by polynomial equations, we know that $V$ is  the vanishing locus of some ideal
$I \subset R$. Using the Hilbert basis theorem, we have shown that $I$ is
finitely generated. This implies that $V$ can be described by \emph{finitely}
many polynomial equations. 
\end{example} 

\subsection{More on noetherian rings}
Let $R$ be a noetherian ring.  
\begin{proposition} \label{noetherianiffg} 
An $R$-module $M$ is noetherian if and only if $M$ is finitely generated.
\end{proposition} 
\begin{proof} 
The only if direction is obvious. A module is noetherian if and only if every
submodule is finitely generated. 

For the if direction, if $M$ is finitely generated, then there is  a surjection
of $R$-modules
\[ R^n \to M  \]
where $R$ is noetherian. So $R^n$ is noetherian because it is a successive
extension of copies of $R$ and an extension of two noetherian modules is also
noetherian. So $M$ is a quotient of a noetherian module and is noetherian.

\end{proof} 

\section{Associated primes}

We shall now begin the structure theory for noetherian modules. The first step
will be to associate to each module a collection of primes, called the
\emph{associated primes}, which lie in a bigger collection of primes where the
localizations are nonzero.

\subsection{The support}
 Let $R$ be a  noetherian ring.  An $R$-module $M$ is supposed to be thought of as somehow
spread out over the topological space $\spec R$, whatever this might mean. If $\mathfrak{p} \in \spec R$, then  let
\( \k(\mathfrak{p}) = \mathrm{fr.  \ field \ } R/\mathfrak{p}  ,\)
which is the residue field of $R_{\mathfrak{p}}$. If $M$ is any $R$-module, we
can consider $M \otimes_R \k(\mathfrak{p})$ for each $\mathfrak{p}$; it is a
vector space over $\k(\mathfrak{p})$. If $M$ is finitely generated, then $M \otimes_R
\k(\mathfrak{p})$ is a finite-dimensional vector space.

\begin{definition} 
Let $M$ be a finitely generated $R$-module. Then $\supp M$, the
\textbf{support} of $M$,  is defined to be the set of primes
$\mathfrak{p} \in \spec R$ such that
\( M \otimes_R \k(\mathfrak{p}) \neq 0.  \)
\end{definition} 

One is  supposed to think of a module $M$ as something like a vector bundle
over the topological space
$\spec R$. At each $\mathfrak{p} \in \spec R$, we associate the vector space $M
\otimes_R \k(\mathfrak{p})$; this is the ``fiber.'' Of course, the intuition
of  $M$'s being a vector bundle is somewhat limited, since the fibers
do not generally have  the same dimension.
Nonetheless, we can talk about the support, i.e. the set of spaces where the
``fiber'' is not zero.

Note that $\mathfrak{p} \in \supp M$ if and only if $M_{\mathfrak{p}} \neq 0$. This is
because
\[ (M \otimes_R R_{\mathfrak{p}})/( \mathfrak{p} R_{\mathfrak{p}} (M \otimes_R
R_{\mathfrak{p}}))  = M_{\mathfrak{p}}
\otimes_{R_{\mathfrak{p}}} \k(\mathfrak{p})  \]
and we can use Nakayama's lemma over the local ring $R_{\mathfrak{p}}$.  (We
are using the fact that $M$ is finitely generated.)

A vector bundle whose support is empty is zero. Thus the following easy
proposition is intuitive:

\begin{proposition} 
$M = 0$ if and only if $\supp M = \emptyset$. 
\end{proposition}
\begin{proof} 
Indeed, $M=0$ if and only if $M_{\mathfrak{p}} = 0$ for all primes
$\mathfrak{p} \in \spec R$. This is equivalent to $\supp M = \emptyset$.
\end{proof} 
 
\begin{exercise} 
Let $0 \to M' \to M \to M'' \to 0$ be exact. Then 
\[ \supp M = \supp M' \cup \supp M''.  \]
\end{exercise} 


We will see soon that that $\supp M$ is closed in $\spec R$. One imagines that
$M$ lives on this closed subset $\supp M$, in some sense.



\subsection{Associated primes}
Throughout this section, $R$ is a noetherian ring. The \emph{associated
primes} of a module $M$ will refer to primes that arise as the annihilators of
elements in $M$. As we shall see, the support of  a module is determined by
the associated primes. Namely, the associated primes contain the ``generic
points'' (that is, the minimal primes) of the support. In some cases, however,
they may contain more. 

\add{We are currently using the notation $\ann(x)$ for the annihilator of $x
\in M$. This has not been defined before. Should we add this in a previous
chapter?}

\begin{definition} 
Let $M$ be a finitely generated $R$-module.  The prime ideal $\mathfrak{p}$ is said to be
\textbf{associated} to $M$ if there exists an element $x \in M$ such that
$\mathfrak{p}$ is the annihilator of $x$.  The set of associated primes is
$\ass(M)$.
\end{definition} 

Note that the annihilator of an element $x \in M$ is not necessarily prime, but
it is possible that the annihilator might be prime, in which case it is
associated.

\begin{exercise} 
Show that $\mathfrak{p} \in \ass(M)$ if and only if there is an injection
$R/\mathfrak{p} \hookrightarrow M$.
\end{exercise} 

\begin{exercise} 
Let $\mathfrak{p} \in \spec R$. Then $\ass(R/\mathfrak{p}) =
\left\{\mathfrak{p}\right\}$.
\end{exercise} 

We shall start by proving that $\ass(M) \neq \emptyset$ for nonzero modules. 
\begin{proposition} \label{assmnonempty} 
If $M \neq 0$, then $M$ has an associated prime.
\end{proposition} 
\begin{proof}  Consider the collection of ideals in $R$ that arise as the
annihilator of a nonzero element in $M$. 
Let $I \subset R$ be a maximal element among this collection.  The existence of $I$ is guaranteed thanks to the noetherianness of
$R$.
Then $I = \ann(x)$ for some $x \in M$, so  $1 \notin I$ because the annihilator of a nonzero element is not the full
ring.

I claim that
$I$ is prime,  and hence $I \in \ass(M)$.  
Indeed, suppose $ab \in I$ where $a,b \in R$. This means that
\[ (ab)x = 0.  \]
Consider the annihilator $\ann(bx)$ of $bx$. This contains the annihilator of $x$, so $I$;
it also contains $a$.

There are two cases. If $bx = 0$, then $ b \in I$ and we are done. Suppose to
the contrary $bx \neq 0$. In this case, $\ann(bx)$ contains $(a) + I$, which
 contains $I$. By maximality, it must happen that $\ann(bx) = I$ and $ a \in
 I$. 

 In either case, we find that one of $a,b $ belongs to $I$, so that $I$ is
 prime. 

\end{proof} 

\begin{example} 
Without the noetherian hypothesis, \cref{assmnonempty} is
\emph{false}. Consider $R = \mathbb{C}[x_1, x_2, \dots]$, the polynomial ring
over $\mathbb{C}$ in infinitely many variables, and the ideal $I = (x_1,
x_2^2, x_3^3, \dots) \subset R$.
The claim is that 
\[ \ass(R/I ) = \emptyset.  \]
To see this, suppose a prime $\mathfrak{p}$ was the annihilator of some
$\overline{f}\in R/I$.  Then $\overline{f}$ lifts to $f \in R$; it follows
that $\mathfrak{p}$ is precisely the set of $g \in R$ such that $fg \in I$.
Now $f$ contains only finitely many of the variables $x_i$, say $x_1, \dots,
x_n$. It is then clear that $x_{n+1}^{n+1} f \in I$  (so $x_{n+1}^{n+1} \in
\mathfrak{p}$), but $x_{n+1} f \notin I$ (so $x_{n+1} \notin \mathfrak{p}$).
It follows that $\mathfrak{p}$ is not a prime, a contradiction.
\end{example} 

We shall now show that the associated primes are finite in number.

\begin{proposition} \label{finiteassm}
If $M$ is finitely generated, then $\ass(M)$ is finite.
\end{proposition}

The idea is going to be to use the fact that $M$ is finitely generated to build
$M$ out of finitely many pieces, and use that to bound the number of associated
primes to each piece. For this, we need:

\begin{lemma} \label{assexact} 
Suppose we have an exact sequence of finitely generated $R$-modules
\[ 0 \to M' \to M \to M'' \to 0.  \]
Then 
\[\ass(M') \subset \ass(M) \subset \ass(M') \cup \ass(M'')  \]
\end{lemma} 
\begin{proof} 
The first claim is obvious. If $\mathfrak{p}$ is the annihilator of
in $x \in M'$, it is an annihilator of something in $M$ (namely the image of
$x$), because
$M' \to M$ is injective. So $\ass(M') \subset \ass(M)$. 

The hard direction is the other inclusion. Suppose $\mathfrak{p} \in \ass(M)$.
Then there is $x \in M$ such that
\[ \mathfrak{p} = \ann(x).  \]
Consider the submodule $Rx \subset M$.  If $Rx \cap M' \neq 0$, then we can
choose $y \in Rx \cap M' - \left\{0\right\}$. I claim that $\ann(y) =
\mathfrak{p}$ and so $\mathfrak{p} \in \ass(M')$.
To see this, $ y = ax$ for some $a \in R$. The annihilator of $y$ is the set of elements
$b \in R$ such that
\[ abx = 0  \]
or, equivalently, the set of $b \in R$ such that $ab \in \mathfrak{p} =
\ann(x)$. But $y = ax \neq 0$, so $a \notin \mathfrak{p}$. As a
result, the condition $b \in \ann(y)$ is the same as $b \in \mathfrak{p}$. In
other words, 
\[ \ann(y) = \mathfrak{p}  \]
which proves the claim.

Suppose now that  $Rx \cap M' = 0$. Let $\phi: M \twoheadrightarrow M''$
be the surjection. I claim that $\mathfrak{p} = \ann(\phi(x))$ and
consequently that
$\mathfrak{p} \in \ass(M'')$.  The proof is as follows. Clearly $\mathfrak{p}$
annihilates $\phi(x)$ as it annihilates $x$. Suppose $a \in \ann(\phi(x))$.
This means that $\phi(ax) = 0$, so $ax \in \ker \phi=M'$; but $\ker \phi \cap Rx =
0$. So $ax = 0$ and consequently $a \in \mathfrak{p}$. It follows $\ann(\phi(x)) = \mathfrak{p}$. 
\end{proof} 

The next step in the proof of \cref{finiteassm} is that any
finitely generated module
admits a filtration each of whose quotients are of a particularly nice form.
This result is quite useful and will be referred to in the future.

\begin{proposition} \label{filtrationlemma}
For any finitely generated $R$-module $M$, there exists a finite filtration
\[ 0 = M_0 \subset M_1 \subset \dots \subset M_k = M  \]
such that the successive quotients $M_{i+1}/M_i$ are isomorphic to various
$R/\mathfrak{p}_i$ with the $\mathfrak{p}_i \subset R$ prime.
\end{proposition} 
\begin{proof} 
Let $M' \subset M$ be maximal among submodules for which such a filtration
(ending with $M'$)
exists. We would like to show that $M' = M$.  Now $M'$ is well-defined since
$0$ has such a filtration and $M$ is
noetherian.  

There is a filtration
\[ 0 = M_0 \subset M_1 \subset \dots \subset M_l = M' \subset M  \]
where the successive quotients, \emph{except} possibly the last $M/M'$, are of
the form $R/\mathfrak{p}_i $ for $\mathfrak{p}_i \in \spec R$.
If $M' = M$, we are done. Otherwise, consider
the quotient $M/M' \neq 0$. There is an associated prime of $M/M'$. So there is
a prime $\mathfrak{p}$ which is the annihilator of $x \in M/M'$. This means
that there is an injection 
\[ R/\mathfrak{p} \hookrightarrow M/M'.  \]
Now, take $M_{l+1}$ as the inverse image in $M$
of $R/\mathfrak{p} \subset M/M'$. 
Then, we can consider the finite filtration
\[ 0 = M_0 \subset M_1 \subset \dots \subset M_{l+1} , \]
all of whose successive quotients are of the form $R/\mathfrak{p}_i$; this is
because $M_{l+1}/M_l = M_{l+1}/M'$ is of this form by construction.
We have thus extended this filtration one
step further,  a contradiction since
$M'$ was assumed to be maximal.
\end{proof} 

Now we are in a position to meet the goal, and prove that $\ass(M)$ is
always a finite set. 
\begin{proof}[Proof of \cref{finiteassm}]
Suppose $M$ is finitely generated Take our filtration
\[ 0 = M_0 \subset M_1 \subset \dots \subset M_k = M.  \]
By induction, we show that $\ass(M_i)$ is finite for each $i$. It is obviously
true for $i=0$. Assume now that $\ass(M_i)$ is finite; we prove the same for
$\ass(M_{i+1})$. We have an exact sequence
\[ 0 \to M_i \to M_{i+1} \to R/\mathfrak{p}_i \to 0  \]
which implies that, by \cref{assexact},
\[ \ass(M_{i+1}) \subset \ass(M_i) \cup \ass(R/\mathfrak{p}_i) = \ass(M_i)
\cup \left\{\mathfrak{p}_i\right\} , \]
so $\ass(M_{i+1})$ is also finite.
By induction, it is now clear that $\ass(M_i)$ is finite for every $i$.

This proves the proposition; it also shows that the number of
associated primes is at most the length of the filtration. 
\end{proof} 

\subsection{Associated primes and localization}

It turns out to be extremely convenient that the construction $M  \to \ass(M)$
behaves about as nicely with respect to localization as we could possibly
want. This lets us, in fact, reduce arguments to the case of a local ring,
which is a significant simplification.

So, as usual, let $R $ be noetherian, and $M$ a finitely generated $R$-module. 
Let further $S \subset R$ be a multiplicative subset. 
Then $S^{-1}M$ is a finitely generated module over the noetherian ring
$S^{-1}M$. So it makes sense to consider both $\ass(M) \subset \spec R$ and
$\ass(S^{-1}M) \subset \spec S^{-1}R$. But we also know that $\spec S^{-1}R
\subset \spec R$ is just the set of primes of $R$ that do not intersect $S$.
Thus, we can directly compare $\ass(M)$ and $\ass(S^{-1}M)$, and one might
conjecture (correctly, as it happens) that $\ass(S^{-1}M) = \ass(M) \cap \spec
S^{-1}R$.
\begin{proposition} \label{assmlocalization}
Let $R$ noetherian, $M$ finitely generated and $S \subset R$ multiplicatively closed. 
Then 
\[ \ass(S^{-1}M)  = \left\{S^{-1}\mathfrak{p}: \mathfrak{p} \in \ass(M),
\mathfrak{p}\cap S  = \emptyset \right\} . \]
\end{proposition} 
\begin{proof} 
We first prove the easy direction, namely that $\ass(S^{-1}M)$
\emph{contains} primes in $\spec S^{-1}R \cap \ass(M)$. 

Suppose $\mathfrak{p} \in \ass(M)$ and
$\mathfrak{p} \cap S = \emptyset$. Then $\mathfrak{p} = \ann(x)$ for some $x
\in M$. Then the annihilator of $x/1 \in S^{-1}M$ is just $S^{-1}\mathfrak{p}$, as one
can directly check. Thus $S^{-1}\mathfrak{p} \in \ass(S^{-1}M)$.
So we get the easy inclusion.

Let us now do the harder inclusion.  
Call the localization map $R \to S^{-1}R  $ as $\phi$.
Let $\mathfrak{q} \in \ass(S^{-1}M)$. By definition, this means that $\mathfrak{q} =
\ann(x/s)$ for some $x \in M$, $s \in S$. We want to see that
$\phi^{-1}(\mathfrak{q}) \in \ass(M) \subset \spec R$.
By definition $\phi^{-1}(\mathfrak{q})$ is the set of elements $a \in R$ such that
\[ \frac{ax}{s} = 0 \in S^{-1}M . \]
In other words, by definition of the localization, this is 
\[  \phi^{-1}(\mathfrak{q}) = \bigcup_{t \in S} \left\{a \in R: atx = 0 \in M\right\} = \bigcup \ann(tx)
\subset R.\]
We know, however, that among elements of the form $\ann(tx)$, there is a
\emph{maximal} element $I=\ann(t_0 x)$ for some $t_0 \in S$, since $R$ is
noetherian.  The claim is that $I = \phi^{-1}(\mathfrak{q})$, so
$\phi^{-1}(\mathfrak{q}) \in \ass(M)$.

Indeed,  any other annihilator $I' = \ann(tx)$ (for $t \in S$) must be contained in $\ann(t_0 t x)$. However, 
\( I \subset \ann(t_0 x)  \)
and $I$ is maximal, so $I = \ann(t_0 t x)$ and
\( I' \subset I.  \) In other words, $I$ contains all the other annihilators
$\ann(tx)$ for $t \in S$.
In particular, the big union above, i.e. $\phi^{-1}(\mathfrak{q})$, is just
\( I = \ann(t_0 x).  \)
In particular, $\phi^{-1}(\mathfrak{q}) = \ann(t_0x)$ is in $\ass(M)$.
This means that every associated prime
of $S^{-1}M$ comes from an associated prime of $M$, which completes the proof.
\end{proof} 





\subsection{Associated primes determine the support}
The next claim is that the support and the associated primes are related.

\begin{proposition}\label{supportassociated} The support is the closure of the associated primes:
\[ \supp M  = \bigcup_{\mathfrak{q} \in \ass(M)}
\overline{\left\{\mathfrak{q}\right\}} \]
\end{proposition} 

By definition of the Zariski topology, this means that a prime $\mathfrak{p}
\in \spec R$ belongs to $\supp M$ if and only if it contains an associated
prime. 

\begin{proof} 
First,  we show that $\supp(M)$ contains the set of primes
$\mathfrak{p} \in \spec R$ containing an associated prime; this will imply
that $\supp(M) \supset \bigcup_{\mathfrak{q} \in \ass(M)}
\overline{\left\{\mathfrak{q}\right\}}$. So let $\mathfrak{q}$ be an
associated prime and $\mathfrak{p} \supset \mathfrak{q}$. We need to show that
\[ \mathfrak{p} \in \supp M, \ \text{i.e.} \ M_{\mathfrak{p}} \neq 0.  \]
But, since $\mathfrak{q} \in \ass(M)$,  there is an injective map
\[ R/\mathfrak{q} \hookrightarrow M , \]
so localization gives an injective map
\[ (R/\mathfrak{q})_{\mathfrak{p}} \hookrightarrow M_{\mathfrak{p}}.  \]
Here, however, the first object $(R/\mathfrak{q})_{\mathfrak{p}}$ is nonzero since nothing nonzero in $R/\mathfrak{q}$ can be
annihilated by something outside $\mathfrak{p}$. So $M_{\mathfrak{p}} \neq
0$, and $\mathfrak{p} \in \supp M$. 

Let us now prove the converse inclusion. Suppose that $\mathfrak{p} \in \supp M$. We
have to show that $\mathfrak{p}$ contains an associated prime.  
By assumption, $M_{\mathfrak{p}} \neq 0$, and $M_{\mathfrak{p}}$ is a finitely generated
module over the noetherian ring $R_{\mathfrak{p}}$. So $M_{\mathfrak{p}}$ has
an associated prime.
It follows by \cref{assmlocalization} that $\ass(M) \cap \spec
R_{\mathfrak{p}}$ is nonempty. Since the primes of $R_{\mathfrak{p}}$
correspond to the primes contained in $\mathfrak{p}$, it follows that there
is  a prime contained in $\mathfrak{p}$ that lies in $\ass(M)$. This is
precisely what we wanted to prove.
\end{proof} 


\begin{corollary} \label{suppisclosed} For $M$ finitely generated,  
$\supp M$ is closed. Further, every minimal element of $\supp M$ lies in
$\ass(M)$.
\end{corollary} 
\begin{proof} 
Indeed, the above result says that
\[ \supp M  = \bigcup_{\mathfrak{q} \in \ass(M)}
\overline{\left\{\mathfrak{q}\right\}}. \]
Since $\ass(M)$ is finite, it follows that $\supp M$ is closed.
The above equality also shows that any minimal element of $\supp M$ must be an
associated prime.
\end{proof} 

\begin{example} 
\cref{suppisclosed} is \emph{false} for modules that are not finitely
generated. Consider for instance the abelian group $\bigoplus_p \mathbb{Z}/p$.
The support of this as a $\mathbb{Z}$-module is precisely the set of all
closed points (i.e., maximal ideals) of $\spec \mathbb{Z}$, and is
consequently is not closed. 
\end{example} 

\begin{corollary} 
The ring $R$ has finitely many minimal prime ideals. 
\end{corollary} 
\begin{proof} 
Clearly, $\supp R = \spec R$. Thus every prime ideal of $R$
contains an associated prime of $R$ by \cref{supportassociated}.
\end{proof} 

So $\spec R$ is the finite union of the  irreducible closed  pieces
$\overline{\mathfrak{q}}$ if $R$ is noetherian.
\add{I am not sure if ``irreducibility'' has already been defined. Check this.}

We have just seen that $\supp M$ is a closed subset of $\spec R$ and is a union
of finitely many irreducible subsets.  More precisely, 
\[ \supp M = \bigcup_{\mathfrak{q} \in \ass(M)}
\overline{\left\{\mathfrak{q}\right\}}  \]
though there might be some redundancy in this expression. Some associated prime might be contained
in others.  

\begin{definition} 
A prime $\mathfrak{p} \in \ass(M)$ is an \textbf{isolated} associated prime of
$M$ if it is minimal (with respect to the ordering on $\ass(M)$); it is
\textbf{embedded} otherwise. 
\end{definition} 

So the embedded primes are not needed to describe the support of $M$. 

\add{Examples of embedded primes}

\begin{remark} 
It follows that in a noetherian ring, every minimal prime consists of
zerodivisors. Although we shall not use this in the future, the same is true
in non-noetherian rings as well.  Here is an argument.

Let $R$ be a ring and $\mathfrak{p} \subset R$ a minimal prime. Then
$R_{\mathfrak{p}}$ has precisely one prime ideal.
We now use:

\begin{lemma} 
If a ring $R$ has precisely one prime ideal $\mathfrak{p}$, then any $x \in
\mathfrak{p}$ is nilpotent.
\end{lemma} 
\begin{proof} 
Indeed, it suffices to see that $R_x = 0$ (Exercise~\ref{nilpcriterion} in
chapter~\ref{spec}) if $x \in
\mathfrak{p}$. But $\spec R_x$
consists of the primes of $R$ not containing $x$. However, there are no such
primes. Thus $\spec R_x = \emptyset$, so $R_x = 0$.
\end{proof} 

It follows that every element in $\mathfrak{p}$ is a zerodivisor in
$R_{\mathfrak{p}}$.
As a result, if $x \in \mathfrak{p}$, there is $\frac{s}{t} \in
R_{\mathfrak{p}}$ such that $xs/t = 0$ but $\frac{s}{t} \neq 0$.
In particular, there is $t' \notin \mathfrak{p}$ with 
\[ xst' = 0, \quad st' \neq 0,  \]
so that $x$ is a zerodivisor. 
\end{remark}



\subsection{Primary modules}

A primary modules are ones that has only one associated prime. It is equivalent
to say that any homothety is either injective or nilpotent. 
As we will see in the next section, any module has a ``primary
decomposition:'' in fact, it embeds as a submodule of a sum of primary
modules.

\begin{definition} 
Let $\mathfrak{p} \subset R$ be prime, $M$ a finitely generated  $R$-module. Then $M$ is
\textbf{$\mathfrak{p}$-primary} if 
\[ \ass(M) = \left\{\mathfrak{p}\right\}.  \]

A module is \textbf{primary} if it is $\mathfrak{p}$-primary for some
prime $\mathfrak{p}$, i.e., has precisely one associated prime. 
\end{definition} 

\begin{proposition} \label{whenisprimary}
Let $M$ be a finitely generated $R$-module. Then $M$ is \textbf{$\mathfrak{p}$}-primary if
and only if, for every $m \in M - \left\{0\right\}$, 
the annihilator $\ann(m)$ has radical $\mathfrak{p}$.
\end{proposition} 
\begin{proof} 
We first need a small observation.

\begin{lemma} 
If $M$ is $\mathfrak{p}$-primary, then any nonzero submodule $M' \subset M$ is
$\mathfrak{p}$-primary.
\end{lemma} 
\begin{proof} 
Indeed, we know that $\ass(M') \subset \ass(M)$ by \cref{assexact}.
Since $M' \neq 0$, we also know that $M'$ has an associated prime
(\cref{assmnonempty}). Thus $ \ass(M') = \{\mathfrak{p}\}$, so
$M'$ is $\mathfrak{p}$-primary.
\end{proof} 

Let us now return to the proof of the main result,
\cref{whenisprimary}.
Assume first that $M$ is $\mathfrak{p}$-primary. Let $x \in M$, $x \neq 0$. Let
$I = \ann(x)$; we are to show that $\rad(I)  =\mathfrak{p}$. By definition, there is an injection
\[ R/I \hookrightarrow M  \]
sending $1 \to x$. As a result, $R/I$ is $\mathfrak{p}$-primary by the above
lemma. We want to know that $\mathfrak{p}  = \rad(I)$. 
We saw that the support $\supp R/I = \left\{\mathfrak{q}: \mathfrak{q}
\supset I\right\}$ is the union of the closures of the associated primes. In
this case, 
\[ \supp(R/I) = \left\{\mathfrak{q}: \mathfrak{q} \supset \mathfrak{p}\right\}
.\]
But we know that $\rad(I) = \bigcap_{\mathfrak{q} \supset I} \mathfrak{q}$,
which by the above is just $\mathfrak{p}$. This proves that $\rad(I) =
\mathfrak{p}$.
We have shown that if $R/I$ is primary, then $I$ has radical $\mathfrak{p}$.

The converse is easy. 
Suppose the condition holds and $\mathfrak{q} \in \ass(M)$, so $\mathfrak{q} =
\ann(x)$ for $x \neq 0$. But then $\rad(\mathfrak{q}) = \mathfrak{p}$, so 
\[ \mathfrak{q} = \mathfrak{p}  \] and $\ass(M) = \left\{\mathfrak{p}\right\}$.
\end{proof} 

We have another characterization.

\begin{proposition} \label{whenisprimary2}
Let $M \neq 0$ be a finitely generated $R$-module. Then $M$ is primary if and
only if for each $a \in
R$, then the homothety $ M \stackrel{a}{\to} M$ is either injective or nilpotent.
\end{proposition}
\begin{proof} 
Suppose first that $M$ is $\mathfrak{p}$-primary. Then multiplication by anything in
$\mathfrak{p}$ is nilpotent because the annihilator of everything nonzero has
radical $\mathfrak{p}$ by \cref{whenisprimary}. But if $a \notin \mathfrak{p}$, then $\ann(x)$ for
$x \in M - \left\{0\right\}$ has radical $\mathfrak{p}$ and cannot contain $a$. 

Let us now do the other direction. Assume that every element of $a$ acts either injectively or nilpotently on $M$.
Let $I \subset R$ be the collection of elements $a \in R$ such that $a^n M = 0$
for $n$ large. Then $I$ is an ideal, since it is closed under addition by the
binomial formula: if $a, b \in I$ and $a^n, b^n$ act by zero, then $(a+b)^{2n}$
acts by zero as well.


I claim that $I$ is actually prime. If $a,b \notin I$, then  $a,b$ act by
multiplication injectively on $M$. So $a: M \to M, b: M \to M$ are injective.
However, a composition of injections is injective, so $ab$ acts injectively and
$ab \notin I$. So $I$ is prime.

We need now to check that if $x \in M$ is nonzero, then $\ann(x)$ has radical
$I$. Indeed, if $a \in R$   annihilates $x$,
then the homothety  $M \stackrel{a}{\to} M$ cannot be injective, so it must be
nilpotent (i.e. in $I$). Conversely, if $a \in I$, then a power of $a$ is
nilpotent, so a power of $a$ 
must kill $x$. 
It follows that $\ann(x) = I$. Now, by \cref{whenisprimary}, we see
that $M$ is $I$-primary.
\end{proof} 

We now have this notion of a primary module. The idea is that all the torsion is
somehow concentrated in some prime.

\begin{example} 
If $R$ is a noetherian ring and $\mathfrak{p} \in \spec R$, then
$R/\mathfrak{p}$ is $\mathfrak{p}$-primary. More generally, if $I \subset R$
is an ideal, then $R/I$ is ideal if and only if $\rad(I) $ is prime. This
follows from \cref{whenisprimary2}.
\end{example} 

\begin{exercise} 
If $0 \to M' \to M \to M'' \to 0$ is an exact sequence with $M', M, M''$
nonzero and finitely generated, then $M$ is $\mathfrak{p}$-primary if and only if $M', M''$ are.
\end{exercise} 

\begin{exercise} 
Let $M$ be a finitely generated $R$-module. Let $\mathfrak{p} \in \spec R$. Show that the sum of two
$\mathfrak{p}$-primary  submodules is $\mathfrak{p}$-primary. Deduce that
there is a $\mathfrak{p}$-primary submodule of $M$ which contains every
$\mathfrak{p}$-primary submodule. 
\end{exercise} 

\begin{exercise}[Bourbaki]
Let $M$ be a finitely generated $R$-module. Let $T \subset \ass(M)$ be a
subset of the associated primes. Prove that there is a submodule $N \subset M$
such that 
\[ \ass(N) = T, \quad \ass(M/N) = \ass(M) - T.  \]

\end{exercise} 

\section{Primary decomposition} This is the structure theorem for modules
over a noetherian ring, in some sense.
Throuoghout, we fix a noetherian ring $R$.

\subsection{Irreducible and coprimary modules}

\begin{definition} 
Let $M$ be a finitely generated $R$-module. A submodule $N \subset M$ is
\textbf{$\mathfrak{p}$-coprimary} if $M/N$ is $\mathfrak{p}$-primary.

Similarly, we can say that $N \subset M$ is \textbf{coprimary} if it is
$\mathfrak{p}$-coprimary for some $\mathfrak{p} \in \spec R$.
\end{definition} 

We shall now show we can represent any submodule of $M$ as an intersection of
coprimary submodules. In order to do this, we will define a submodule of $M$ to be
\emph{irreducible} if it cannot be written as a nontrivial intersection of
submodules of $M$. It
will follow by general nonsense that any submodule is an intersection of
irreducible submodueles. We will then see that any irreducible submodule is
coprimary. 

\begin{definition} 
The submomdule $N \subsetneq M$ is \textbf{irreducible} if whenever $N = N_1 \cap N_2$ for $N_1,
N_2 \subset M$ submodules, then either one of $N_1, N_2$ equals $N$. In other words, it is not
 the intersection of larger submodules. 
\end{definition} 

\begin{proposition} \label{irrediscoprimary}
An irreducible submodule $N \subset M$ is coprimary.
\end{proposition} 
\begin{proof} 
Say $a \in R$. We would like to show that the homothety 
\[ M/N \stackrel{a}{\to} M/N  \]
is either injective or nilpotent.
Consider  the following submodules of $M/N$:
\[ K(n) =  \left\{x \in M/N: a^n x = 0\right\} . \]
Then clearly $K(0) \subset K(1) \subset \dots$; this chain stabilizes  as
the quotient module is noetherian.
In particular, $K(n) = K(2n)$ for large $n$. 

It follows that if $x \in M/N$ is divisible by $a^n$ ($n$ large) and nonzero, then $a^n x$
is also nonzero. Indeed, say $x = a^n y \neq 0$; then $y \notin K(n)$, so $a^{n}x =
a^{2n}y \neq 0$ or we would have $y \in K(2n) = K(n)$. In $M/N$, the submodules
\[ a^n(M/N) \cap \ker(a^n)  \]
are equal to zero for large $n$. But our assumption was that $N$ is
irreducible.  So one of these submodules of $M/N$ is zero. That is, either
$a^n(M/N) = 0$ or $\ker a^n = 0$. We get either injectivity or nilpotence on
$M/N$. This proves the result.
\end{proof} 
 
\subsection{Irreducible and primary decompositions}

We shall now show that in a finitely generated module over a noetherian ring,
we can write $0$ as an intersection of coprimary modules. This decomposition,
which is called a \emph{primary decomposition}, will be deduced from purely
general reasoning.

\begin{definition} 
An \textbf{irreducible decomposition} of the module $M$ is a representation
$N_1 \cap N_2 \dots \cap N_k  = 0$, where the $N_i \subset M$ are irreducible
submodules.
\end{definition} 

\begin{proposition} 
If $M$ is finitely generated, then $M$ has an irreducible decomposition. There exist finitely many irreducible
submodules $N_1, \dots, N_k$ with
\[  N_1 \cap \dots \cap N_k = 0. \]
\end{proposition} 
In other words,
\[  M \to \bigoplus M/N_i  \]
is injective.
So a finitely generated module over a noetherian ring can be imbedded in a direct sum of
primary modules, since by \cref{irrediscoprimary} the $M/N_i$ are
primary. 

\begin{proof} This is now purely formal. 

Among the submodules of $M$, some may be expressible as intersections of
finitely many irreducibles, while some may not be. Our goal is to show that
$0$ is such an intersection.
Let $M' \subset M$ be a maximal submodule of $M$ such that $M'$ \emph{cannot} be
written as such an intersection. If no such
$M'$ exists, then we are done, because then $0$ can be written as an
intersection of finitely many irreducible submodules.

Now $M'$ is not irreducible, or it would be the intersection of one irreducible
submodule.
It follows $M'$ can be written as $M'=M_1' \cap M_2'$ for two strictly
larger submodules of $M$.  But by maximality, $M_1', M_2'$ admit decompositions as
intersections of irreducibles. So $M'$ admits such a decomposition as well, a contradiction. 
\end{proof} 

\begin{corollary} 
For any finitely generated $M$, there exist coprimary submodules $N_1, \dots,
N_k \subset M$ such that $N_1 \cap \dots \cap N_k  = 0$.
\end{corollary} 
\begin{proof} 
Indeed, every irreducible submodule is coprimary.
\end{proof} 


For any $M$, we have an \textbf{irreducible decomposition}
\[ 0 = \bigcap N_i  \]
for the $N_i$ a finite set of irreducible (and thus coprimary) submodules. 
This decomposition here is highly non-unique and non-canonical. Let's try to
pare it down to something which is a lot more canonical.

The first claim is that we can collect together modules which are coprimary for
some prime. 
\begin{lemma} 
Let $N_1, N_2 \subset M$ be $\mathfrak{p}$-coprimary submodules. Then $N_1 \cap
N_2$ is also $\mathfrak{p}$-coprimary.
\end{lemma} 
\begin{proof} 
We have to show that $M/N_1 \cap N_2$ is $\mathfrak{p}$-primary. Indeed, we have an injection
\[ M/N_1 \cap N_2 \rightarrowtail  M/N_1 \oplus M/N_2  \]
which implies that $\ass(M/N_1 \cap N_2) \subset \ass(M/N_1) \cup \ass(M/N_2) =
\left\{\mathfrak{p}\right\}$. So we are done. 
\end{proof} 

In particular, if we do not want irreducibility but only primariness in the
decomposition
\[ 0 = \bigcap N_i,  \]
we can assume that each $N_i$ is $\mathfrak{p}_i$ coprimary for some
 prime
$\mathfrak{p}_i$ with the $\mathfrak{p}_i$ \emph{distinct}.

\begin{definition} 
Such a decomposition of zero, where the different modules $N_i$ are
$\mathfrak{p}_i$-coprimary for different $\mathfrak{p}_i$, is called a \textbf{primary decomposition}.
\end{definition} 



\subsection{Uniqueness questions}

In general, primary decomposition is \emph{not} unique. Nonetheless, we shall
see that a limited amount of uniqueness does hold. For instance, the primes
that occur are determined.

Let $M$ be a finitely generated module over a noetherian ring $R$, and suppose
$N_1 \cap \dots \cap N_k = 0$ is a primary decomposition.
Let us assume that the decomposition is
\emph{minimal}: that is, if we dropped one of the $N_i$, the intersection would no
longer be zero.
This implies that
\[ N_i \not\supset \bigcap_{j \neq i} N_j  \]
or we could omit one of the $N_i$. Then the decomposition is called a \textbf{reduced primary decomposition}.

Again, what this tells us is that $M \rightarrowtail  \bigoplus M/N_i$. What we
have shown is that $M$ can be imbedded in a sum of pieces, each of which is
$\mathfrak{p}$-primary for some prime, and the different primes are distinct.

This is \textbf{not} unique. However, 

\begin{proposition} 
The primes $\mathfrak{p}_i$ that appear in a reduced primary decomposition of zero are
uniquely determined. They are the associated primes of $M$.
\end{proposition} 
\begin{proof} 
All the associated primes of $M$ have to be there, because we have the injection
\[ M \rightarrowtail  \bigoplus M/N_i  \]
so the associated primes of $M$ are among those of $M/N_i$ (i.e. the
$\mathfrak{p}_i$).

The hard direction is to see that each $\mathfrak{p}_i$ is an associated prime.
I.e. if $M/N_i$ is $\mathfrak{p}_i$-primary, then $\mathfrak{p}_i \in \ass(M)$;
we don't need to use primary modules except for primes in the associated primes. 
Here we need to use the fact that our decomposition has no redundancy.  Without
loss of generality, it suffices to show that $\mathfrak{p}_1$, for instance,
belongs to $\ass(M)$. We will use the fact that
\[ N_1 \not\supset N_2 \cap \dots .  \]
So this tells us that $N_2 \cap N_3 \cap \dots$ is not equal to zero, or we
would have a containment. We have a map
\[ N_2 \cap \dots \cap N_k \to M/N_1;  \]
it is injective, since the kernel is $N_1 \cap N_2 \cap \dots \cap N_k = 0$ as
this is a decomposition.
However, $M/N_1$ is $\mathfrak{p}_1$-primary, so $N_2 \cap \dots \cap N_k$ is
$\mathfrak{p}_1$-primary. In particular, $\mathfrak{p}_1$ is an associated
prime of $N_2 \cap \dots \cap N_k$, hence of  $M$.
\end{proof} 

The primes are determined. The factors are not. However, in some cases they are.

\begin{proposition} 
Let $\mathfrak{p}_i$ be a minimal associated prime of $M$, i.e. not containing
any smaller associated prime. Then the submodule $N_i$  corresponding to
$\mathfrak{p}_i$ in the reduced primary decomposition is uniquely determined:
it is the kernel of 
\[ M \to M_{\mathfrak{p}_i}.  \]
\end{proposition} 

\begin{proof} 
We have that $\bigcap N_j = \left\{0\right\} \subset M$. When we localize at
$\mathfrak{p}_i$, we find that
\[ (\bigcap N_j)_{\mathfrak{p}_i} = \bigcap (N_j)_{\mathfrak{p}_i} =0 \]
as localization is an exact functor. If $j \neq i$, then $M/N_j$ is
$\mathfrak{p}_j$ primary, and has only $\mathfrak{p}_j$ as an associated prime.
It follows that $(M/N_j)_{\mathfrak{p}_i}$ has no associated primes, since the
only associated prime could be $\mathfrak{p}_j$, and that's not contained in
$\mathfrak{p}_j$.
In particular, $(N_j)_{\mathfrak{p}_i} = M_{\mathfrak{p}_i}$.

Thus, when we localize the primary decomposition at $\mathfrak{p}_i$, we get
a trivial primary decomposition: most of the factors are the full
$M_{\mathfrak{p}_i}$.  It follows that $(N_i)_{\mathfrak{p}_i}=0$. When we draw
a commutative diagram
\[ 
\xymatrix{
N_i \ar[r] \ar[d]  &  (N_i)_{\mathfrak{p}_i} = 0 \ar[d]  \\
M \ar[r] &  M_{\mathfrak{p}_i}.
}
\]
we find that $N_i$ goes to zero in the localization.

Now if $x \in \ker(M \to M_{\mathfrak{p}_i}$, then $sx = 0$ for some $s \notin
\mathfrak{p}_i$. When we take the map $M \to M/N_i$, $sx$ maps to zero; but $s$
acts injectively on $M/N_i$, so $x$ maps to zero in $M/N_i$, i.e. is zero in
$N_i$.
\end{proof} 

This has been abstract, so:
\begin{example} Let $ R = \mathbb{Z}$.
Let $M = \mathbb{Z} \oplus \mathbb{Z}/p$. Then zero can be written as 
\[ \mathbb{Z} \cap \mathbb{Z}/p  \]
as submodules of $M$. But $\mathbb{Z}$ is $\mathfrak{p}$-coprimary, while
$\mathbb{Z}/p$ is $(0)$-coprimary. 

This is not unique. We could have considered 
\[ \{(n,n), n \in \mathbb{Z}\} \subset M.  \]
However, the zero-coprimary part has to be the $p$-torsion. This is because
$(0)$ is the minimal ideal. 

The decomposition is always unique, in general, if
we have no inclusions among the prime ideals. For $\mathbb{Z}$-modules, this
means that primary decomposition is unique for torsion modules. 
Any torsion group is a direct sum of the $p$-power torsion over all primes $p$. 
\end{example} 


\section{Artinian rings and modules}

The notion of an \emph{artinian ring}  appears to be dual to that of a
noetherian ring, since the chain condition is simply reversed in the
definition. However, the artinian condition is much stronger than the
noetherian one. In fact, 
artinianness actually implies noetherianness, and much more. 
Artinian modules over non-artinian rings are frequently of interest as well;
for instance, if $R$ is a noetherian ring and $\mathfrak{m}$ is a maximal
ideal, then for any finitely generated $R$-module $M$, the module
$M/\mathfrak{m}M$ is artinian. 

\subsection{Definitions}

\begin{definition} 
A commutative ring $R$ is \textbf{Artinian} every descending chain of ideals
$I_0 \supset I_1 \supset I_2 \supset \dots$
stabilizes.
\end{definition} 

\begin{definition}
The same definition makes sense for modules. We can define an $R$-module $M$ to
be \textbf{Artinian} if every descending chain of submodules stabilizes. 
\end{definition}

In fact, as we shall see when we study dimension theory, we actually often do
want to study artinian modules over non-artinian rings, so this definition is
useful. 

\begin{exercise} 
A module is artinian if and only if every nonempty collection of submodules
has a minimal element.
\end{exercise} 
\begin{exercise} 
A ring which is a finite-dimensional algebra over a field is artinian.
\end{exercise} 
\begin{proposition}  \label{exactartinian}
If $0 \to M' \to M \to M'' \to 0$ is an exact sequence, then $M$ is Artinian
if and only if $M', M''$ are. 
\end{proposition}

This is proved in the same way as for noetherianness.

\begin{corollary} 
Let $R$ be artinian. Then every finitely generated $R$-module is artinian.
\end{corollary} 
\begin{proof} 
Standard.
\end{proof} 

\subsection{The main result}
This definition is obviously dual to the notion of noetherianness, but it is
much more restrictive.  
The main result is:

\begin{theorem} \label{artinianclassification}
A commutative ring $R$ is artinian if and only if:
\begin{enumerate}
\item $R$ is noetherian. 
\item Every prime ideal of $R$ is maximal.\footnote{This is much different from
the Dedekind ring condition---there, zero is not maximal. An artinian domain is
necessarily a field, in fact.}
\end{enumerate}
\end{theorem} 


So artinian rings are very simple---small in some sense.
They all look kind of like fields.

We shall prove this result in a series of small pieces. We begin with a piece
of the forward implication in \cref{artinianclassification}:
\begin{lemma} Let $R$ be artinian. 
Every prime $\mathfrak{p} \subset R$ is maximal.
\end{lemma} 
\begin{proof} 
Indeed, if $\mathfrak{p} \subset R$ is a prime ideal, $R/\mathfrak{p}$ is
artinian, as it is a quotient of an artinian ring. We want to show that
$R/\mathfrak{p}$ is a field,
which is the same thing as saying that $\mathfrak{p}$ is maximal.
(In particular, we are essentially proving that an artinian \emph{domain} is a
field.)

Let $x \in
R/\mathfrak{p}$ be nonzero. We have a descending chain
\[ R/\mathfrak{p} \supset (x) \supset (x^{2}) \dots  \]
which necessarily stabilizes. Then we have $(x^n) = (x^{n+1})$ for some $n$. In
particular, we have $x^n = y x^{n+1}$ for some $y \in R/\mathfrak{p}$. But $x$
is a nonzerodivisor, and  we find $ 1 = xy$. So $x$ is invertible. Thus
$R/\mathfrak{p}$ is a field.
\end{proof} 

Next, we claim there are only a few primes in an artinian ring:
\begin{lemma} 
If $R$ is artinian, there are only finitely many maximal ideals.
\end{lemma} 
\begin{proof} 
Assume otherwise. Then we have an infinite sequence
\[ \mathfrak{m}_1,  \mathfrak{m}_2, \dots  \]
of distinct maximal ideals. Then we have the descending chain
\[ R \supset \mathfrak{m}_1 \supset \mathfrak{m}_1 \cap \mathfrak{m}_2 \supset \dots.  \]
This necessarily stabilizes. So for some $n$, we have that $\mathfrak{m}_1 \cap \dots \cap 
\mathfrak{m}_n \subset \mathfrak{m}_{n+1}$. However, this means that
$\mathfrak{m}_{n+1}$ contains one of the $\mathfrak{m}_1, \dots,
\mathfrak{m}_n$ since these are prime ideals (a familiar argument).  Maximality
and distinctness of the $\mathfrak{m}_i$ give a contradiction.
\end{proof} 

In particular, we see that $\spec R$ for an artinian ring is just a finite set.
In fact, since each point is closed, as each prime is maximal, the set has the
\emph{discrete topology.} As a result, $\spec R$ for an artinian ring is
\emph{Hausdorff}. (There are very few other cases.)

This means that $R$ factors as a product of rings. Whenever $\spec R$ can be
written as a disjoint union of components, there is a factoring of $R$ into a
product $\prod R_i$. So $R = \prod R_i$ where each $R_i$ has
only one maximal ideal. Each $R_i$, as a homomorphic image of $R$, is artinian. We find, as a result, 

\add{mention that disconnections of $\spec R$ are the same thing as
idempotents.}

\begin{proposition} 
Any artinian ring is a finite product of local artinian rings.
\end{proposition} 

Now, let us continue our analysis. We may as well assume that we are working
with \emph{local} artinian rings $R$ in the future. In particular, $R$ has a unique
prime $\mathfrak{m}$, which must be the radical of $R$ as the radical is the
intersection of all primes. 

We shall now see that the unique prime ideal $\mathfrak{m} \subset R$ is
nilpotent by:
\begin{lemma} \label{radnilpotentartinian}
If $R$ is artinian (not necessarily local), then $\rad (R) $ is nilpotent.
\end{lemma} 

It is, of course, always true that any \emph{element} of the radical $\rad(R)$
is nilpotent, but it is not true for a general ring $R$ that $\rad(R)$ is
nilpotent as an \emph{ideal}.

\begin{proof} 
Call $J = \rad(R)$. Consider the decreasing filtration
\[ R \supset J \supset J^2 \supset J^3 \supset \dots.  \]
We want to show that this stabilizes at zero. A priori, we know that it
stabilizes \emph{somewhere}. For some $n$, we have
\[ J^n = J^{n'}, \quad n' \geq n.  \]
Call the eventual stabilization of these ideals $I$. Consider ideals $I'$ such
that
\[ II' \neq 0.  \]
There are now two cases:
\begin{enumerate}
\item No such $I'$ exists. Then $I = 0$, and we are done: the powers of
$J^n$ stabilize at zero. 
\item Otherwise,  there is a
\emph{minimal} such $I'$ (minimal for satisfying $II' \neq 0$) as $R$ is
artinian. Necessarily $I'$ is nonzero, and furthermore there is $x \in I'$ with $x I \neq
0$. 

It follows by minimality that
\[ I' = (x) , \]
so $I'$ is principal. Then $xI \neq 0$; observe
that $xI$ is also $(xI)I $ as $I^2  = I$ from the definition of $I$. Since
$(xI) I \neq 0$, it follows again by minimality that
\[ xI = (x).  \] Hence, there is $y \in I$ such that $xy = x$; but now, by construction $I \subset J = \rad (R)$, implying that $y $ is nilpotent.
It follows that
\[ x = xy = xy^2 = \dots = 0  \]
as $y$ is nilpotent. However, $x \neq 0$ as $xI \neq 0$. This is a
contradiction, which implies that the second case cannot occur. 
\end{enumerate}
We have now proved the lemma.
\end{proof} 

Finally, we may prove:

\begin{lemma} 
A local artinian ring $R$ is noetherian.
\end{lemma} 
\begin{proof} 
We have the filtration $R \supset \mathfrak{m} \supset \mathfrak{m}^2 \supset
\dots$. This eventually stabilizes at zero by \cref{radnilpotentartinian}. I
claim that $R$ is noetherian as an $R$-module. To prove this, it suffices to
show that $\mathfrak{m}^k/\mathfrak{m}^{k+1}$ is noetherian as an $R$-module.
But of course, this is annihilated by $\mathfrak{m}$, so it is really a vector
space over the field $R/\mathfrak{m}$. But $\mathfrak{m}^k/\mathfrak{m}^{k+1}$
is a subquotient of an artinian module, so is artinian itself. We have to show
that it is noetherian. 
It suffices to show now that if $k$ is a field, and $V$ a $k$-vector space,
then TFAE:
\begin{enumerate}
\item $V$ is artinian. 
\item $V$ is noetherian.
\item $V$ is finite-dimensional.
\end{enumerate}
This is evident by linear algebra. 	
\end{proof} 

Now, finally, we have shown that an artinian ring is noetherian. We have to
discuss the converse. Let us assume now that $R$ is noetherian and has only
maximal prime ideals. We show that $R$ is artinian. Let us consider $\spec R$;
there are only finitely many minimal primes by the theory of associated
primes: every prime ideal is minimal in this case. Once again, we learn that $\spec R$
is finite and has the discrete topology. This means that $R$ is a product of
factors $\prod R_i$ where each $R_i$ is a local noetherian ring with a unique
prime ideal. We might as well now prove:

\begin{lemma} 
Let $(R, \mathfrak{m})$ be a local noetherian ring with one prime ideal. Then
$R$ is artinian.
\end{lemma} 
\begin{proof} 
We know that $\mathfrak{m} = \mathrm{rad}(R)$. So $\mathfrak{m}$ consists of
nilpotent elements, so by finite generatedness it is nilpotent.  Then we have a
finite filtration
\[ R \supset \mathfrak{m} \supset \dots \supset \mathfrak{m}^k = 0.  \]
Each of the quotients are finite-dimensional vector spaces, so artinian; this
implies that $R$ itself is artinian. 
\end{proof} 



The theory of artinian rings is thus a special case of the theory of noetherian
rings.

