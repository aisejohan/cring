\chapter{Completions}
Completions are used in geometry and number theory in order to give a finer picture of local structure; for example, taking completions of rings allows for the recovery of a topology that looks more like the Euclidean topology as it has more open sets than the Zariski topology. Completions are also used in algebraic number theory to allow for the study of fields around a prime number (or prime ideal). 

\subsection{Completions of Local Rings}
\begin{definition} Let $R$ be a local ring and $\mathfrak{m}$ its maximal ideal. Then the completion of $R$ with respect to $\mathfrak{m}$ denoted $\hat{R}$ is the inverse limit 
\begin{equation}
\hat{R}=\lim_{\leftarrow}(R/\mathfrak{m}^nR)\end{equation} We then topologize $\hat{R}$ by setting powers of $\mathfrak{m}$ to be basic open sets around $0$. The topology formed by these basic open sets is called the ``Krull'' or ``$\mathfrak{m}$-adic topology.''
\end{definition}

\subsection{Number Fields}
Recall that in algebraic number theory, a number field is a 
finite dimensional algebraic extension of $\mathbb{Q}$. 
Sitting inside of $\mathbb{Q}$ is the ring of integers, $\mathbb{Z}$. For any prime number $p\in \mathbb{Z}$, we can localize $\mathbb{Z}$ to the 
 prime ideal $(p)$ giving us a local ring $\mathbb{Z}_(p)$. 
 If we take the completion of this local ring we get the $p$-adic numbers $\mathbb{Q}_p$. Notice that since $\mathbb{Z}_(p)/p^n\cong\mathbb{Z}/p$, this is really the same as taking the inverse limit $\lim_{\leftarrow}\mathbb{Z}/p^n$.

\subsection{Motivation}
Say you have a commutative ring $R$. We can draw $\spec R$, a picture whose
points are the primes of $R$. Consider a maximal ideal $\mathfrak{m} \in \spec
R$. If you think of $\spec R$ as a space, and $R$ as a collection of functions
on that space, then $R_{\mathfrak{m}}$ is the collection of ``germs'' of
functions defined near the point $\mathfrak{m}$. 

The Zariski topology is kind of lousy as far as topologies go, and you can't
really find small neighborhoods of $\mathfrak{m}$.

\begin{example} 
Let $X$ be a compact Riemann surface, and let $x \in X$. Let $R$ be the ring of
holomorphic functions on $X - \left\{x\right\}$ which are meromorphic at $x$.
In this case, $\spec R$ has the ideal $(0)$ and maximal ideals corresponding to
the zero locus at some point in $X - \left\{x\right\}$. So $\spec R$ is $X -
\left\{x\right\}$ together with a ``generic'' point.

Let's just look at the closed points.  If we pick $y \in X - \left\{x\right\}$,
then we can consider the local ring $R_y = \left\{s^{-1}r, s(y) \neq
0\right\}$. This ring is a direct limit of the rings of holomorphic functions
on sets $U$ that extend meromorphically to $X$, where $U$ ranges over open
subsets of $X$ containing $y$ which are the nonzero loci of things in $R$. However, $U$ really ranges over complements of
finite subsets. It does not range over open sets in the \emph{complex} topology.

Near $y$, $X$ looks like the complex numbers in the complex plane. In the Zariski topology, this
isn't the case. Each $R_y$ actually remembers the whole Riemann surface. The
reason is that the quotient field of $R_y$ is the rational function field of
$X$, which recovers $X$. Thus $R_y$ remembers too much. 
\end{example} 

We would like a variant of localization that would remember much less about the
global topology.

\subsection{Definition}

\begin{definition} 
Let $R$ be a commutative ring and $I \subset R$ an ideal. Then we define the
\textbf{completion of $R$ at $I$}
\[ \hat{R}_I = \varprojlim R/I^n.  \]
So this is the inverse limit of the quotients $R/I^n$. There is a tower of
commutative rings
\[ \dots \to  R/I^3 \to R/I^2 \to R/I   \]
whose inverse limit is $R^{\vee}_I$. 
\end{definition} 

Let us give some examples.
\begin{example} 
Let $R = \mathbb{Z}, I = (p)$. Then the completion is denoted $\mathbb{Z}_p$
and is called the ring of \textbf{$p$-adic integers.} This is the inverse limit
of the rings $\mathbb{Z}/p^i$. 
\end{example} 

\begin{example} 
Let $X$ be a Riemann surface. Let $ x \in X$ be as before, and let $R$ be as
before: the ring of meromorphic functions on $X$ with poles only at $x$. We can
complete $R$ at the ideal $\mathfrak{m}_y \subset R$ corresponding to $y \in X - \left\{x\right\}$. This
is always isomorphic to a power series ring
\[ \mathbb{C}[[t]]  \]
where $t$ is a holomorphic coordinate at $y$. 

The reason is that if you consider $R/\mathfrak{m}_y^n$, you always get
$\mathbb{C}[t]/(t^n)$, where $t$ corresponds to a local coordinate at $y$. Thus
\emph{these} rings don't remember much about the Riemann surface. They're all
isomorphic, for instance. 
\end{example} 

\begin{remark} 
Usually, we will complete $R$ at \emph{maximal} ideals.
If we wanted to study $R$ near a prime $\mathfrak{p} \in \spec R$, we might
first approximate $R$ by $R_{\mathfrak{p}}$, which is a local ring; we might
make another approximation by completing $R_{\mathfrak{p}}$. Then we get a
\emph{really} local structure. 
\end{remark} 

\begin{remark} 
There is always a map $R \to \hat{R}_I$ by taking the limit of the maps $R/I^i$.
\end{remark} 

A priori, you might think you get  a big mess. The amazing thing is that for
noetherian rings, you get well-behaved stuff.

\subsection{Properties of completions}

\begin{proposition} 
Let $R$ be noetherian, $I \subset R$ an ideal. Then $\hat{R}_I$ is noetherian.
\end{proposition} 
\begin{proof} 
Choose generators $x_1, \dots, x_n \in I$. This can be done as $I$ is finitely generated
Consider a power series ring 
\[ R[[t_1, \dots, t_n]] ; \]
the claim is that there is a map $R[[t_1\dots t_n]] \to \hat{R}_I$ sending each
$t_i$ to $x_i  \in \hat{R}_I$. This is not trivial, since we aren't talking
about a polynomial ring, but a power series ring. 

To build this map, we  want a compatible family of maps
\[ R[[t_1, \dots, t_n]] \to R[t_1, \dots, t_n]/(t_1, \dots, t_n)^k  \to R/I^k.  \]
where the second ring is the polynomial ring where you have killed homogeneous
polynomials of degree $\geq k$. There is a map from $R[[t_1, \dots, t_n]]$ to
the second ring that kills monomials of degree $ \geq k$. The second map
$R[t_1, \dots, t_n]/(t_1, \dots, t_n)^k \to R/I^k$ sends $t_i \to x_i$ and is
obviously well-defined.

So we get the map
\[ \phi:  R[[t_1, \dots, t_n]] \to \hat{R}_I ,  \]
which I claim is surjective. Let us prove this. Suppose $a \in \hat{R}_I$. Then $a$ can be thought
of as a collection of elements $(a_k) \in R/I^k$ which are compatible with one
another.  We can lift each $a_k$ to some $\overline{a_k} \in R$ in a
compatible manner, such that
\[ \overline{a_{k+1}} = \overline{a_k} + b_k, \quad b_k \in I^k.  \]
Since $b_k \in I^k$, we can write it as
\[ b_k = f_k(x_1, \dots, x_n)  \]
for $f_k$ a polynomial in $R$ of degree $k$, by definition of the generators in
$I^k$. 

I claim now that 
\[ a = \phi\left( \sum f_k(t_1, \dots, t_n)  \right).  \]
The proof is just to check modulo $I^k$ for each $k$. This we  do by induction.
When you reduce  modulo $I^k$, one checks easily that you get $a_k$.

As we have seen, $\hat{R}_I$ is the quotient of a power series ring. In the
homework, it was seen that $R[[t_1, \dots, t_n]]$ is noetherian; this is a
variant of the Hilbert basis theorem proved in class. So $\hat{R}_I$ is
noetherian.
\end{proof} 

We want to think of completion as analogous to localization. We would like to
think of it as an inoffensive operation. One way to say this for localization
was that localization is an exact functor. This is true for completions too.
But first, we need a variant: completions of modules.

\begin{definition} 
Let $R$ be a ring, $M$ an $R$-module, $I \subset R$ an ideal. We define the
\textbf{completion of $M$ at $I$} as 
\[ \hat{M}_I =  \varprojlim M/I^n M.  \]

This is an inverse limit of $R$-modules, so it is an $R$-module. Furthermore,
it is even an $\hat{R}_I$-module, as one easily checks. It is also functorial.
\end{definition} 

\begin{proposition} \label{completionisexact}
If $R$ is noetherian and $I \subset R$ an ideal, then the construction $M \to
\hat{M}_I$ is exact when restricted to finitely generated modules.
\end{proposition} 


Let's be more precise.  If $M$ is finitely generated, and 
\( 0 \to M' \to M \to M'' \to 0  \)
is an exact sequence,\footnote{The ends are finitely generated by noetherianness.} then
\[ 0 \to \hat{M'}_I \to \hat{M}_I \to \hat{M''}_I \to 0  \]
is also exact. 

For a moment, let us step back and think about exact sequences of inverse
limits of abelian groups. Say we have a tower of exact sequences of abelian
groups
\[ 
\xymatrix{
0 \ar[r] &  \vdots \ar[d]  \ar[r] &  \vdots \ar[d]  \ar[r] & \vdots \ar[d]
\ar[r] &  0 \\
0 \ar[r] &  A_2 \ar[d]  \ar[r] &  B_2 \ar[d]  \ar[r] &  C_2 \ar[d]  \ar[r] &  0
\\
0 \ar[r] &  A_1 \ar[d]  \ar[r] &  B_1 \ar[d]  \ar[r] &  C_1 \ar[d]  \ar[r] &  0
\\
0 \ar[r] &  A_0 \ar[r] &  B_0 \ar[r] &  C_0 \ar[r] &  0
}.
\]
Then we get a sequence
\[ 0 \to \varprojlim A_n \to \varprojlim B_n \to \varprojlim C_n \to 0.  \]
In general, it is \emph{not} exact. But it is left-exact. 

\begin{proposition} 
Hypotheses as above, $ 0 \to \varprojlim A_n \to \varprojlim B_n \to
\varprojlim C_n$ is exact. 
\end{proposition} 
\begin{proof} 
It is obvious that $\phi \circ \psi = 0$. 

Let us first show that $\phi: \varprojlim A_n \to \varprojlim B_n$ is
injective. So suppose $a $ is in the projective limit, represented by a
compatible sequence of elements $(a_k )\in A_k$. If $\phi$ maps to zero, all
the $a_k$ go to zero in $B_k$. Injectivity of $A_k \to B_k$ implies that each
$a_k$ is zero. This implies $\phi$ is injective. 

Now let us show exactness at the next step. Let $\psi:  \varprojlim B_n \to
\varprojlim C_n$ and let $b = (b_k)$ be in $\ker \psi$. This means that each
$b_k$ gets killed when it maps to $C_k$. This means that each $b_k$ comes from
something in $a_k$. These $a_k$ are unique by injectivity of $A_k \to B_k$. It
follows that the $a_k$ have no choice but to be compatible. Thus $(a_k)$ maps
into $(b_k)$. So $b$ is in the image of $\phi$. 
\end{proof} 

So far, so good. We get some level of exactness. But the map on the end is not
necessarily surjective. Nonetheless:

\begin{proposition} 
$\psi: \varprojlim B_n \to \varprojlim C_n$ is surjective if each $A_{n+1} \to
A_n$ is surjective.
\end{proposition} 
\begin{proof} 
Say $c \in \varprojlim C_n$, represented by a compatible family $(c_k)$. We
have to show that there is a compatible family $(b_k) \in \varprojlim B_n$
which maps into $c$. It is easy to choose the $b_k$ \emph{individiually} since
$B_k \to C_k$ is surjective. The problem is that a priori we may not get
something compatible.

We construct $b_k$ by induction on then, therefore. Assume that $b_k$ which
lifts $c_k$ has been constructed. 
We know that $c_k$ receives  a map from $c_{k+1}$. 
\[ \xymatrix{
& & c_{k+1 } \ar[d] \\
& b_k \ar[r] &  c_k
}.\]
Choose any $x \in B_{k+1}$ which maps to $c_{k+1}$. However, $x$ might not map
down to $b_k$, which would screw up the compatibility conditions. Next, we try to adjust $x$. 
Consider $x' \in B_k$ to be the image of $x$ under $B_{k+1} \to B_k$. We know
that $x' - b_k$ maps to zero in $C_k$, because $c_{k+1}$ maps to $c_k$.
So $x' - b_k$ comes from something in $A_k$, call it $a$. 
\[ \xymatrix{
& x \ar[r] &  c_{k+1 } \ar[d] \\
& b_k \ar[r] &  c_k
}.\]
But $a$ comes from some $\overline{a} \in A_{k+1}$. Then we define
\[ b_{k+1} = x - \overline{a},  \]
which adjustment doesn't change the fact that $b_{k+1}$ maps to $c_{k+1}$.
However, this adjustment makes $b_{k+1}$ compatible with $b_k$. Then we
construct the family $b_k$ by induction. We have seen surjectivity.
\end{proof} 

Now, let us study the exactness of completions.
\begin{proof}[Proof of Proposition~\ref{completionisexact}]

Let us try to apply the general remarks above to studying the sequence 
\[ 0 \to \hat{M'}_I \to \hat{M}_I \to \hat{M''}_I \to 0.  \]
Now $\hat{M}_I = \varprojlim M/I^n$. We can construct  surjective maps
\[ M/I^n \twoheadrightarrow M''/I^n  \]
whose inverse limits lead to $\hat{M}_I \to \hat{M''}_I$. The image is
$M/(M' + I^n M)$. What is the kernel?
Well, it is $M' + I^n M/ I^n M$.  This is equivalently
\[ M'/M' \cap I^n M.  \]
So we get an exact sequence
\[ 0 \to M'/M' \cap I^n M \to M/I^n M \to M''/I^n M'' \to 0.  \]
By the above analysis of exactness of inverse limits, we get an exact sequence
\[ 0 \to \varprojlim M'/(I^n M \cap M') \to \hat{M}_I \to \hat{M''}_I \to 0.  \]
We of course have surjective maps $M'/I^n M' \to M'/(I^n M \cap M') $ though
these are generally not isomorphisms. Something ``divisible by $I^n$'' in $M$ but
in $M'$ is generally not divisible by $I^n$ in $M'$.
Anyway, we get a map
\[ \varprojlim M'/I^n M' \to \varprojlim M'/I^n M \cap M'  \]
where the individual maps are not necessarily isomorphisms. Nonetheless, I
claim that the map on inverse limits is an isomorphism. This will imply that
completion is indeed an exact functor.

Essentially, to say that this map is an isomorphism is to say that the
filtrations $I^n M'$ and $I^n M \cap M'$ on $M'$ are \emph{comparable.} To
prove this, and to complete the proof of exactness, it suffices to show:
\begin{proposition}[Artin-Rees lemma] Let $R$ be noetherian, $I \subset R$ an
ideal. Suppose $M$ is a finitely generated $R$-module and $M' \subset M$ a submodule. Then
there is a  constant $c$ such that
\[ I^{n+c} M \cap M' \subset I^n M'.  \]
So the two filtrations $I^n M \cap M', I^n M'$ on $M$ are equivalent up to a
shift.
\end{proposition} 
\begin{proof} 
Define a new ring $R' = R \oplus It \oplus I^2 t^2 + \dots$, which is a subring of
$R[t]$. The coefficient of $t^n $ is required to belong to $I^n$. 

\begin{lemma} 
$R'$ is noetherian.
\end{lemma} 
\begin{proof} 
Choose generators $x_1, \dots, x_n \in I$; then there is a map $R[y_1, \dots,
y_n] \to R'$ sending $y_i \to x_i t$. This is surjective. Hence by the basis
theorem, $R'$ is noetherian.
\end{proof} 

Let $N = M \oplus  IM t \oplus I^2 M [t] \oplus \dots \subset M[t] = M
\otimes_R R[t]$. Note that $N$ is an $R'$-module. It is in fact a finitely
generated $R'$-module, hence noetherian, since $M$ was finitely generated over
$R$. Let $N' = N \cap M'[t]$. In particular
\[ N' = M' \oplus (M' \cap IM ) t \oplus \dots.  \]
So $N' \subset N$ is finitely generated. Choose generators for $N'$, and let
$c$ be the largest degree (exponent of $t$) that occurs.
I claim that $c$ works.  This is easy to check, but we're out of time. We'll
talk about this more next week. 
\end{proof} 
\end{proof} 
\lecture{10/25}

Last time, we were talking about completions. We showed that if $R$ is
noetherian and $I \subset R$ an ideal, an exact sequence
\[ 0 \to M' \to M \to M \to 0   \]
of finitely generated $R$-modules leads to a sequence
\[ 0 \to \hat{M'}_I \to \hat{M}_I \to \hat{M;}_I \to 0  \]
which is also exact. We showed this using the Artin-Rees lemma.

\begin{remark}
In particular, completion is an \textbf{exact functor}: if $A \to B \to C$ is
exact, so is the sequence of completions. This can be seen by drawing in
kernels and cokernels, and using the fact that completions preserve short
exact sequences. 
\end{remark}

\subsection{Completions and flatness}

Suppose that $M$ is a finitely generated $R$-module. Then there is a surjection $R^n
\twoheadrightarrow M$, whose kernel is also finitely generated as $R$ is
noetherian. It follows that
$M$ is finitely presented. In particular, there is a sequence
\[ R^m \to R^n \to M \to 0.  \]
We get an exact sequence
\[ \hat{R}^m \to \hat{R}^n \to \hat{M} \to 0  \]
where the second map is just multiplication by the same $m$-by-$n$ matrix as in
the first case.

\begin{corollary} 
If $M$ is finitely generated, there is a canonical isomorphism
\[ \hat{M}_I \simeq M \otimes_R \hat{R}_I.  \]
\end{corollary} 

\begin{proof} 
We know that there is a map $M \to \hat{M}_I$, so the canonical morphism
$\phi_M: M \otimes_R \hat{R}_{I} \to \hat{M}_I$ exists
(because this induces a map from $M \otimes_R \hat{R}_I$). We
need to check that it is an isomorphism. 

If there is an exact sequence $M' \to M \to M'' \to 0$, there is a commutative
diagram
\[ \xymatrix{
M' \otimes_R \hat{R}_I \ar[d]^{\phi_{M'}}  \ar[r] & M \otimes_R \hat{R}_I
\ar[d]^{\phi_M}  \ar[r] &
M'' \otimes_R \hat{R}_I \ar[d] \ar[r] &  0 \\
\hat{M'}_I \ar[r] &  \hat{M}_I \ar[r] &  \hat{M''}_I \ar[r] &  0
}.\]
Exactness of completion and right-exactness of $\otimes$ implies that this
diagram is exact. It follows that if $\phi_M, \phi_{M'}$ are isomorphisms, so
is $\phi_{M''}$. 

But any $M''$ appears at the end of such a sequence with $M', M$ are free by
the finite presentation argument above. So it suffices to prove $\phi$ an
isomorphism for finite frees, which reduces to the case of $\phi_R$ an
isomorphism. That is obvious. 
\end{proof} 

\begin{corollary} 
$\hat{R}_I$ is a flat $R$-module.
\end{corollary} 
\begin{proof} 
Indeed, tensoring with $\hat{R}_I$ is exact (because it is completion, and
completion is exact) on the category of finitely generated $R$-modules.
Exactness on the category of all $R$-modules follows by taking direct limits,
since every module is a direct limit of finitely generated modules, and
direct limits preserve exactness. 
\end{proof} 


\begin{remark} 
Warning: $\hat{M}_I$ is, in general, not $M \otimes_R \hat{R}_I$ when $M$ is
not finitely generated. One example to think about is $M  = \mathbb{Z}[t]$,
$R = \mathbb{Z}$. The
completion of $M$ at $I = (p)$ is the completion of $\mathbb{Z}[t]$ at $p
\mathbb{Z}[t]$, which contains elements like
\[ 1 + pt + p^2 t^2 + \dots,  \]
which belong to the completion but not to $\hat{R}_I \otimes M = \mathbb{Z}_p
[t]$. 
\end{remark} 

\subsection{The Krull intersection theorem}
We lied earlier. What we called the Krull intersection theorem is not actually
called the Krull theorem.

\begin{theorem}[Krull] Let $R$ be a local noetherian ring with maximal ideal
$\mathfrak{m}$.\footnote{This is a favorite time to complete $R$.} Then the map
$R \to \hat{R}_{\mathfrak{m}}$ is injective. Alternatively,
\[ \bigcap \mathfrak{m}^i = (0).  \]
\end{theorem} 

\begin{proof} 
Let $I = \ker(R \to \hat{R}_{\mathfrak{m}})$. We have $I \hookrightarrow R$;
thus there is a map
\[ \hat{I}_{\mathfrak{m}}  \hookrightarrow \hat{R}_{\mathfrak{m}}. \]
Now $\hat{I}_{\mathfrak{m}}$ is generated by $I$ as an
$R_{\mathfrak{m}}$-module. It follows that the map $\hat{I}_{\mathfrak{m}}
\hookrightarrow R_{\mathfrak{m}}$ is the zero map. 

So $\hat{I}_{\mathfrak{m}} = 0$. But this is the inverse limit of  a tower of
abelian groups $I/\mathfrak{m}^i I $ where the maps are surjections; to say
that this limit is zero is to say that $I = \mathfrak{m}I$. So $I = 0$ by
Nakayama. 
\end{proof} 

\subsection{Hensel's lemma} One thing that you might be interested in doing is solving
Diophantine equations. Say $R = \mathbb{Z}$; you want to find solutions to a
polynomial $f(X) \in \mathbb{Z}[X]$. Generally, it is very hard to find
solutions. However, there are easy tests you can do that will tell you if there
are no solutions. For instance, reduce mod a prime. One way you can prove that
there are no solutions is to show that there are no solutions mod 2. 

But there might be solutions mod 2 and yet you might not be sure about
solutions in $\mathbb{Z}$. So you might try mod 4, mod 8, and so on---you get a
whole tower of problems to consider. If you manage to solve all these equations
, you can solve the equations in the 2-adic integers $\mathbb{Z}_2 =
\hat{\mathbb{Z}}_{(2)}$.

But the Krull intersection theorem implies that $\mathbb{Z} \to \mathbb{Z}_2$
is injective. So if you expected that there was a unique solution in
$\mathbb{Z}$, you might try looking at the solutions in $\mathbb{Z}_2$ to be
the solutions in $\mathbb{Z}$.



The moral is that solving an equation over $\mathbb{Z}_2$ is intermediate in
difficulty between $\mathbb{Z}/2$ and $\mathbb{Z}$. Nonetheless, it turns out
that solving an equation mod $\mathbb{Z}/2$ is very close to solving it over
$\mathbb{Z}_2$, thanks to

\begin{theorem}[Hensel's Lemma]
Let $R$ be a noetherian ring, $I \subset R$ an ideal. Let $f(X) \in R[X]$ be a
polynomial such that the equation $f(X)=0$ has a solution $ a \in R/I$.
Suppose, moreover, that $f'(a)$ is invertible in $R/I$. 

Then $a$ lifts uniquely to a solution of the equation $f(X) = 0$ in $\hat{R}_I$.
\end{theorem} 

\begin{example} 
Let $R = \mathbb{Z}, I = (5)$. Consider the equation $f(x) = x^2 + 1 = 0$ in $R$. This
has a solution modulo five, namely $2$. Then $f'(2) = 4$ is invertible in
$\mathbb{Z}/5$. So the equation $x^2 + 1 = 0$ has a solution in $\mathbb{Z}_5$.
In other words, $\sqrt{-1} \in \mathbb{Z}_5$.
\end{example} 

Let's prove Hensel's lemma.
\begin{proof} 
Now we have $a \in R/I$ such that $f(a) = 0 \in R/I$ and $f'(a)$ is invertible. 
The claim is going to be that for each $m \geq 1$, there is a \emph{unique}
element $a_n \in R/I^n$ such that
\[ a_n \to a \ (I), \quad f(a_n)  = 0 \in R/I^n.  \]
Uniqueness implies that this sequence $(a_n)$ is compatible, and thus gives the
required element of the completion.
It will be a solution of $f(X) = 0$ since it is a solution at each element of
the tower.

Let us now prove the claim.
For $n=1$, $a_1 = a$ necessarily. 
The proof is induction on $n$. Assume that $a_n$ exists and is unique. We would like to show
that $a_{n+1}$ exists and is unique. Well, if it is going to exist, when we
reduce $a_{n+1}$ modulo $I^n$, we must get $a_n$ or uniqueness at the $n$-th
step would fail. 

So let $\overline{a}$ be any lifting of $a_n$ to $R/I^{n+1}$.  Then $a_{n+1}$
is going to be that lifting plus some $\epsilon \in I^n/I^{n+1}$. We want
\[ f(\overline{a} + \epsilon) = 0 \in R/I^{n+1}.  \]
But this is 
\[ f(\overline{a}) +  \epsilon f'(\overline{a})  \]
because $\epsilon^2 = 0 \in R/I^{n+1}$. However, this lets us solve for
$\epsilon$, because then necessarily $\epsilon =
\frac{-f(\overline{a})}{f'(\overline{a})} \in I^n$.
Note that $f'(\overline{a}) \in R/I^{n+1}$ is invertible. If you believe this
for a moment, then we have seen that $\epsilon$ exists and is unique; note
that $\epsilon \in I^n$ because $f(\overline{a}) \in I^n$. 


\begin{lemma} 
$f'(\overline{a}) \in R/I^{n+1}$ is invertible.
\end{lemma} 
\begin{proof} 
If we reduce this modulo $R/I$, we get the invertible element $f'(a) \in R/I$.
Note also that the $I/I^{n+1}$ is a nilpotent ideal in $R/I^{n+1}$. So we are
reduced to showing, more generally:

\begin{lemma} 
Let $A$ be a ring,\footnote{E.g. $R/I^{n+1}$.} $J$ a nilpotent
ideal.\footnote{E.g. $J = I/I^{n+1}$.} Then an element $x \in A$ is invertible
if and only if its reduction in $A/J$ is invertible.
\end{lemma} 
\begin{proof} 
One direction is obvious. For the converse, say $x \in A$  has an invertible
image. This implies that there is $y \in A$ such that $xy \equiv 1 \mod J$. Say
$$xy = 1+m,$$ where $m \in J$. But $1+m$ is invertible because 
\[ \frac{1}{1+m} = 1 - m + m^2 \pm \dots.  \]
The expression makes sense as the high powers of $m$ are zero. 
So this means that $y(1+m)^{-1}$ is the inverse to $x$. 
\end{proof} 
\end{proof} 
\end{proof} 

This was one of many versions of Hensel's lemma. There are many ways you can
improve on a statement. The above version says something about
``nondegenerate'' cases, where the derivative is invertible. There are better
versions which handle degenerate cases.

\begin{example} 
Consider $x^2 - 1$; let's try to solve this in $\mathbb{Z}_2$. Well,
$\mathbb{Z}_2$ is a domain, so the only solutions can be $\pm 1$. But these
have the same reduction in $\mathbb{Z}/2$. The lifting of the solution is
non-unique.

The reason why Hensel's lemma fails is that $f'(\pm 1) = \pm 2$ is not
invertible in $\mathbb{Z}/2$. But it is not far off. If you go to
$\mathbb{Z}/4$, we do get two solutions, and the derivative is at least nonzero
at those places.
\end{example} 

One possible extension of Hensel's lemma is to allow the derivative to be
noninvertible, but at least to bound the degree to which it is noninvertible.
From this you can get interesting information. 
But then you may have to look at equations $R/I^n$ instead of just $R/I$, where
$n$ depends on the level of noninvertibility.

Let us describe the multivariable Hensel lemma.

\begin{theorem} 
Let $f_1, \dots, f_n$ be polynomials in $n$ variables over the ring $R$. Let
$J$ be the Jacobian matrix $( \frac{\partial f_i}{\partial x_j})$. Suppose
$\Delta = \det J \in R[x_1, \dots, x_n]$. 

If the system $\left\{f_i(x) = 0\right\}$ has a solution $a \in (R/I)^n$ in $R/I$ for some
ideal $I$ satisfying the condition that $\Delta(a)$ is invertible, then there
is a unique solution of $\left\{f_i(x) =0\right\}$ in $\hat{R}_I^n$ which lifts $a$. 
\end{theorem} 
The proof is the same idea: successive approximation, using the invertibility
of $\Delta$. 

