\chapter{The $\spec$ of a ring}


The notion of the $\spec$ of a ring is fundamental in modern algebraic
geometry. It is the scheme-theoretic analog of classical affine schemes. The
identification occurs when one identifies the maximal ideals of the polynomial
ring $k[x_1, \dots, x_n]$ (for $k$ an algebraically closed field) with the
points of the classical variety $\mathbb{A}^n_k = k^n$. In modern algebraic
geometry, one adds the ``non-closed points'' given by the other prime ideals.
Just as general varieties were classically defined by gluing affine varieties, a
scheme is defined by gluing open affines. 

This is not a book on schemes, but it will nonetheless be convenient to
introduce the $\spec$ construction, outside of the obvious benefits of
including preparatory material for algebraic geometry. First of all, it will provide a convenient
notation. Second, and more importantly, it will provide a convenient geometric
intuition. For example, an $R$-module can be thought of as a kind of ``vector
bundle''---technically, a sheaf---over the space $\spec R$, with the caveat
that the rank might not be locally constant (which is, however, the case when the module
is projective).

\section{The spectrum of a ring}

We shall now associate to every commutative ring a topological space $\spec R$
in a functorial manner.
That is, there will be a contravariant functor
\[\spec:  \mathbf{CRing} \to \mathbf{Top}  \]
where $\mathbf{Top}$ is the category of topological spaces.
This construction is the basis for scheme-theoretic
algebraic geometry and will be used frequently in the sequel. 

\subsection{Definition and examples}

We start simply by defining $\spec$ as a set. We will next construct the
Zariski topology and later the functoriality.
\begin{definition} 
Let $R$ be a commutative ring.  The \textbf{spectrum} of $R$, denoted $\spec R$, is
the set of prime ideals of $R$.
\end{definition} 

We shall now make $\spec R$ into a topological space. First, we describe a
collection of sets which will become the closed sets. 
If $I \subset R$ is an ideal, let
\[ V(I) = \left\{\mathfrak{p}: \mathfrak{p} \supset I\right\} \subset \spec R.
\]

\begin{proposition} 
There is a topology on $\spec R$ such that the closed subsets are of the form
$V(I)$ for $I \subset R$ an ideal.
\end{proposition} 

\begin{proof} 
Indeed:
\begin{enumerate}
\item $\emptyset = V((1))$ because $(1)$ is not prime. So $\emptyset$ is closed. 
\item $\spec R = V((0))$ because any ideal contains zero. So $\spec R$ is
closed.
\item  We show the closed sets are stable under intersections.  Let
$K_{\alpha} = V(I_{\alpha})$ be closed subsets of $\spec R$ for $\alpha$
ranging over some index set.  Let $I
= \sum I_{\alpha}$. Then 
\[ V(I) = \bigcap K_{\alpha} = \bigcap V(I_{\alpha}),  \]
which follows because $I$ is the smallest ideal containing each $I_{\alpha}$,
so a prime contains every $I_{\alpha}$ iff it contains $I$.  
\item The union of two closed sets is closed. Indeed, if $K,K' \subset \spec
R$ are closed, we show $K \cup K'$ is closed.  Say $K= V(I), K' = V(I')$. Then
we claim: 
\[ K \cup K'  = V(II').  \]
Here $II'$ is the ideal generated by products $ii', i \in I, i' \in I'$. If
$\mathfrak{p}$ is \textbf{prime} and contains $II'$, it must contain one of $I$, $I'$;
this implies the displayed equation above and implies the result.
\end{enumerate}
\end{proof} 
\begin{definition} 
The topology on $\spec R$ defined above is called the \textbf{Zariski
topology}. With it,  $\spec R$ is now a topological space.
\end{definition} 

In order to see the geometry of this construction, let us work several examples.

\begin{example} 
Let $R = \mathbb{Z}$, and consider $\spec \mathbb{Z}$. Then every prime is generated by one element, since
$\mathbb{Z}$ is a PID. We have that $\spec \mathbb{Z} = \{(0)\} \cup \bigcup_{p \
\mathrm{prime}} \{ (p)\}$.  The picture is that one has all the familiar primes $(2), (3),
(5), \dots, $ and then a special point $(0)$.

Let us now describe the closed subsets. These are of the form $V(I)$ where $I
\subset \mathbb{Z}$ is an ideal, so $I = (n)$ for some $n \in \mathbb{Z}$. 

\begin{enumerate}
\item If $n=0$, the closed subset is all of $\spec \mathbb{Z}$. 
\item If $n \neq 0$, then $n$ has finitely many prime divisors. So $V((n))$ consists
of the prime ideals corresponding to these prime divisors.  
\end{enumerate}

The only closed subsets besides the entire space are the finite subsets
that exclude $(0)$.  
\end{example} 

\begin{example} \label{twovarpoly}
Say $R = \mathbb{C}[x,y]$ is a polynomial ring in two variables.  What is
$\spec R$? We won't give a complete answer. But we will write down several
prime ideals.

\begin{enumerate}
\item For every pair of complex numbers $s,t \in \mathbb{C}$, the collection of polynomials
$f \in R$ such that $f(s,t) = 0$ is a prime ideal $\mathfrak{m} _{s,t}$.  In
fact, it is maximal, as the residue field is all of $\mathbb{C}$.  Indeed,
$R/\mathfrak{m}_{s,t} \simeq \mathbb{C}$ under the map $f \to f(s,t)$.  

In fact, 
\begin{theorem}
The $\mathfrak{m}_{s,t}$ are all the maximal ideals in $R$.
\end{theorem} 
This will follow from the \emph{Hilbert Nullstellensatz} to be proved later
(Theorem~\ref{nullstellensatz}).
\item $(0) \subset R$ is a prime ideal since $R$ is a domain. 
\item  If $f(x,y) \in R$ is an irreducible polynomial, then $(f)$ is a prime
ideal.  This is equivalent to unique factorization in $R$.\footnote{To be
proved later \ref{}.}  
\end{enumerate}

To draw $\spec R$, we start by drawing $\mathbb{C}^2$, the collection of
maximal ideals.  $\spec R$ has additional (non-closed) points too, as
described above, but for now let us
consider the topology induced on $\mathbb{C}^2$ as a subspace of $\spec R$. 

The closed subsets of $\spec R$ are subsets $V(I)$ where $I$ is an ideal,
generated by some polynomials $\left\{f_{\alpha}(x,y)\right\}$.  
It is of interest to determine  the  subset of $\mathbb{C}^2$ that $V(I)$
induces. In other words, we ask:
\begin{quote}
What points of $\mathbb{C}^2$ (with $(s,t)$ identified with
$\mathfrak{m}_{s,t}$) lie in $V(I)$?
\end{quote}
Now we know that $(s,t)$ corresponds to a point of $I$ if and only if  $I
\subset \mathfrak{m}_{s,t}$.
This is true iff all the
$f_{\alpha} $ lie in $ \mathfrak{m}_{s,t}$, i.e. if $f_{\alpha}(s,t) =0$ for all
$\alpha$.  So the closed subsets of $\mathbb{C}^2$ (with the induced Zariski
topology) are precisely the subsets
that can be defined by polynomial equations.  This is \textbf{much} coarser
than the usual topology.  For instance, $\left\{(z_1,z_2): \Re(z_1) \geq 0\right\}$ is
not Zariski-closed. 

The Zariski topology is so coarse because one has only algebraic data (namely,
polynomials, or elements of $R$) to define the topology.
\end{example} 

\begin{exercise} 
Let $R_1, R_2$ be commutative rings. Give $R_1 \times R_2$ a natural structure
of a ring, and describe $\spec( R_1 \times R_2)$ in terms of $\spec R_1$ and
$\spec R_2$.
\end{exercise} 


\begin{exercise} 
Let $X$ be a compact Hausdorff space, $C(X)$ the ring of real continuous
functions $X \to \mathbb{R}$. 
The maximal ideals in $\spec C(X)$ are in bijection with the points of $X$,
and the topology induced on $X $ (as a subset of $\spec C(X)$) is just the usual topology.
\end{exercise}

\begin{exercise}
Prove the following result: if $X, Y$ are compact Hausdorff spaces and $C(X),
C(Y)$ the associated rings of continuous functions, if $C(X), C(Y)$ are
isomorphic as $\mathbb{R}$-algebras, then $X$ is homeomorphic to $Y$.
\end{exercise} 


\subsection{The radical ideal-closed subset correspondence}

We now return to the case of an arbitrary  commutative ring $R$. If $I \subset R$, we get a closed
subset $V(I) \subset \spec R$.  It is called $V(I)$ because one is supposed to
think of it as the places where the elements of $I$ ``vanish,'' as the
elements of $R$ are something like ``functions.'' This analogy is perhaps best
seen in the example of a polynomial ring over an algebraically closed field,
e.g. Example~\ref{twovarpoly} above.

The map from ideals into closed sets is very far from being injective in
general, though by definition it is surjective.

\begin{example} 
If $R = \mathbb{Z}$ and $p$ is prime, then $I = (p), I' = (p^2)$ define the
same subset (namely, $\left\{(p)\right\}$) of
$\spec R$. 
\end{example} 

We now ask the question of why the map from ideals to closed subsets fails to
be injective. As we shall see, the entire problem disappears if we restrict to
\emph{radical} ideals.

\begin{definition} 
If $I$ is an ideal, then the \textbf{radical} $\rad(I)  $ or $ \sqrt{I}$ is
defined as $$\rad(I) =
\left\{x \in R: x^n \in I \ \mathrm{for} \ \mathrm{some} \ n \right\}.$$
An ideal is \textbf{radical} if it is equal to its radical. (This is
equivalent to the earlier Definition~\ref{}.) 
\end{definition} 

Before proceeding, we must check:
\begin{lemma} 
If $I$ an ideal, so is $\rad(I)$.
\end{lemma} 
\begin{proof} 
Clearly $\rad(I)$ is closed under multiplication since $I$ is.  
Suppose $x,y \in \rad(I)$; we show $x+y \in \rad(I)$.  Then $x^n, y^n \in I$
for some $n$ (large) and thus for all larger $n$. The binomial expansion now
gives
\[ (x+y)^{2n} = x^{2n} + \binom{2n}{1} x^{2n-1}y + \dots + y^{2n},  \]
where every term contains either $x,y$ with power $ \geq n$, so every term
belongs to $I$.  Thus $(x+y)^{2n} \in I$ and, by definition, we see then that $x+y \in \rad(I)$. 
\end{proof} 

The map $I \to V(I)$ does in fact depend only on the radical of $I$. In fact, if $I,J$ have the same radical $\rad(I) = \rad(J)$, then $V(I) = V(J)$. 
Indeed, $V(I) = V(\rad(I)) = V(\rad(J)) = V(J)$ by:
\begin{lemma} 
For any $I$, $V(I) = V(\rad(I))$.
\end{lemma} 
\begin{proof} 
Indeed, $I \subset \rad(I)$ and therefore obviously $V(\rad(I)) \subset V(I)$. We have to show the
converse inclusion. Namely, we must prove:
\begin{quote}
If $\mathfrak{p} \supset I$, then $\mathfrak{p} \supset  \rad(I).$
\end{quote}
So suppose $\mathfrak{p} \subset I$ is prime and $x \in \rad(I)$; then $x^n \in I \subset \mathfrak{p}$ for some $n$.
But $\mathfrak{p}$ is prime, so whenever a  product of things belongs to
$\mathfrak{p}$, a factor does.  Thus since $x^n = x . x \dots . x$, we must
have $x \in \mathfrak{p}$. So
\[ \rad(I) \subset \mathfrak{p}  \]
proving the quoted claim, and thus the lemma.
\end{proof} 

There is a converse to this remark:
\begin{proposition} 
If $V(I) = V(J)$, then $\rad(I) = \rad(J)$. 
\end{proposition} 
So two ideals define the same closed subset iff they have the same radical.
\begin{proof} 
We write down a formula for $\rad(I)$ that will imply this at once.
\begin{lemma} For a commutative ring $R$ and an ideal $I \subset R$, 
\[ \rad(I) = \bigcap_{\mathfrak{p} \supset I} \mathfrak{p}.  \]
\end{lemma} 
From this, it follows that $V(I)$ determines $\rad(I)$.  This will thus imply
the proposition.  
We now prove the lemma:
\begin{proof} 
\begin{enumerate}
\item We show $\rad(I) \subset \bigcap_{\mathfrak{p} \in V(I)} \mathfrak{p} $.  In
particular, this follows if we show that if a prime contains $I$, it contains $\rad(I)$; but we have already
discussed this above.  
\item If $x \notin \rad(I)$, we will show that there is a prime ideal $\mathfrak{p}
\supset I$ not containing $x$. This will imply the reverse inclusion and the
lemma.  
\end{enumerate}


We want to find $\mathfrak{p}$ not containing $x$, more generally not
containing any power of $x$.  In particular, we want $\mathfrak{p} \cap \left\{1,
x, x^2 \dots, \right\} = \emptyset$.  This set $S = \left\{1, x, \dots\right\}$
is multiplicatively closed, in that it contains 1 and is closed under
finite products. Right now, it does not hit $I$; we want to find a
\emph{prime} containing $I$ that does not hit $\left\{x^n, n \geq 0\right\}$.


More generally, we will prove:

\begin{sublemma}
Let $S$ be multiplicatively closed set in any ring $R$ and let $I$ be any ideal with $I \cap S  =
\emptyset$.  There is a prime ideal $\mathfrak{p} \supset I$ and does not
intersect $S$.  
\end{sublemma}
In English, any ideal missing $S$ can be enlarged to a prime ideal missing $S$.
This is actually fancier version of a previous argument. We showed earlier that any ideal not
containing the multiplicatively closed subset $\left\{1\right\}$ can be
contained in a prime ideal not containing $1$ in \ref{}.

Note that the sublemma clearly implies the lemma when applied to $S =
\left\{1, x, \dots\right\}.$

\begin{proof}[Proof of the sublemma]
Let $P = \left\{J: J \supset I, J \cap S = \emptyset \right\}$. Then $P$ is a
poset with respect to  inclusion. Note that $P \neq \emptyset$ because $I \in P$.  Also,
for any nonempty linearly ordered subset of $P$, the union is in $P$ (i.e. there is an
upper bound).  
We can invoke Zorn's lemma to get a maximal element of $P$.  This element is an
ideal $\mathfrak{p} \supset I$ with $\mathfrak{p} \cap S = \emptyset$.  I claim
that $\mathfrak{p}$ is prime.

First of all, $1 \notin \mathfrak{p}$ because $1 \in S$.  We need only check
that if $xy \in \mathfrak{p}$, then $x \in \mathfrak{p}$ or $y \in
\mathfrak{p}$. Suppose otherwise, so $x,y \notin \mathfrak{p}$.  Then $(x,\mathfrak{p}) \notin P$ or
$\mathfrak{p}$ would not be maximal. Ditto for $(y, \mathfrak{p})$. 

In particular, we have that these bigger ideals both intersect $S$.  This means
that there are 
\[  a \in \mathfrak{p} , r \in R \quad \mathrm{such that }\quad a+rx \in S \]
and 
\[  b \in \mathfrak{p} , r' \in R \quad \mathrm{such that}\quad b+r'y \in S .\]
Now $S$ is multiplicatively closed, so multiply $(a+rx)(b+r'y) \in S$.
We find:
\[ ab + ar'y+brx+rr'xy \in S.  \]
Now $a,b \in \mathfrak{p}$ and $xy \in \mathfrak{p}$, so all the terms above
are in $\mathfrak{p}$, and the sum is too. But this contradicts $\mathfrak{p}
\cap S = \emptyset$. 
\end{proof}
\end{proof} 
\end{proof} 

The upshot of the previous lemmata is :
\begin{proposition}
There is a bijection between the closed subsets of $\spec R$ and radical ideals
$I \subset R$.
\end{proposition}

\subsection{Functoriality of $\spec$}
 The construction $R \to \spec R$ is functorial in $R$ in a
contravariant sense.   That is,  if $f: R \to R'$, there is a continuous map $\spec
R' \to \spec R$. This map sends $\mathfrak{p} \subset R'$ to
$f^{-1}(\mathfrak{p}) \subset R$, which is easily seen to be a prime ideal
in $R$.   Call this map $F: \spec R' \to \spec R$. So far, we have seen that
$\spec R$ induces a contravariant  functor from $\mathbf{Rings} \to \mathbf{Sets}$.

\begin{exercise} 
A contravariant functor $F: \mathcal{C} \to \mathbf{Sets}$ (for some category
$\mathcal{C}$) is called \textbf{representable} if it is naturally isomorphic
to a functor of the form $X \to \hom(X, X_0)$ for some $X_0 \in \mathcal{C}$,
or equivalently if the induced covariant functor on
$\mathcal{C}^{\mathrm{op}}$ is corepresentable. 

The functor $R \to \spec R $ is not representable. (Hint: Indeed, a representable
functor must send the initial object into a one-point set.)
\end{exercise} 

Next, we check that the morphisms induced on $\spec$'s from a
ring-homomorphism are in fact \emph{continuous} maps of topological spaces. 

\begin{proposition} 
$\spec $ induces a contravariant functor from $\mathbf{Rings}$ to the category
$\mathbf{Top}$ of topological spaces.
\end{proposition} 
\begin{proof} Let $f : R \to R'$. 
We need to check that this map $ \spec R' \to \spec R$, which we call $F$, is
continuous.
That is, we must check that $F^{-1}$ sends closed
subsets of $\spec R$ to closed subsets of $\spec R'$.  

More precisely, if $I \subset
R$ and we take the inverse image $F^{-1}(V(I)) \subset \spec R'$, it is just
the closed set $V(f(I))$.  This is best left to the reader, but here is the justification.  If $\mathfrak{p} \in \spec R'$, then $F(\mathfrak{p}) = f^{-1}(\mathfrak{p})
\supset I$ if and only if $\mathfrak{p} \supset f(I)$. So $F(\mathfrak{p}) \in
V(I)$ if and only if $\mathfrak{p} \in V(f(I))$.

\end{proof} 



\begin{example} 
Let $R$ be a commutative ring, $I \subset R$ an ideal, $f: R \to R/I$. There is a map
of topological spaces
\[ F: \spec (R/I) \to \spec R  .\]
This map is a closed embedding whose image is $V(I)$.  Most of this follows because
there is a bijection between ideals of $R$ containing $I$ and
ideals of $R/I$, and this bijection preserves primality.

\begin{exercise} 
Show that this map $\spec R/I \to \spec R$ is indeed a homeomorphism from $\spec R/I
\to V(I)$.  
\end{exercise} 
\end{example} 

\section{Basic open sets}

\subsection{A basis for the Zariski topology}
In the previous section, we were talking about the Zariski topology. If $R$ is a
commutative ring, we recall that $\spec R$ is defined to be the collection of
prime ideals in $R$. This has a topology where the closed sets are the sets of
the form
\[ V(I) = \left\{\mathfrak{p} \in \spec R: \mathfrak{p} \supset I\right\} . \]
There is another way to describe the Zariski topology in terms of
\emph{open} sets.  

\begin{definition} 
If $f \in R$, we let 
\[ U_f = \left\{\mathfrak{p}: f \notin \mathfrak{p}\right\}  \]
so that $U_f$ is the subset of $\spec R$ consisting of primes not containing
$f$. This is the complement of $V((f))$, so it is open.  
\end{definition} 

\begin{proposition} 
The sets $U_f$ form a basis for the Zariski topology. 
\end{proposition} 

\begin{proof} 
Suppose $U \subset \spec R$ is open.  We claim that $U$ is a union of basic
open sets $U_f$. 

Now $U = \spec R - V(I)$ for some ideal $I$.  Then
\[ U = \bigcup_{f \in I} U_f  \]
because if an ideal is not in $V(I)$, then it fails to contain some $f \in I$,
i.e. is in $U_f$ for that $f$. Alternatively, we could take complements, whence
the above statement becomes
\[ V(I) = \bigcap_{f \in I} V((f))  \]
which is clear.
\end{proof} 

The basic open sets have nice properties.
\begin{enumerate}
\item $U_1  = \spec R$ because prime ideals are not allowed to contain the
unit element. 
\item $U_0 = \emptyset$ because every prime ideal contains $0$.
\item $U_{fg} = U_f \cap U_g$ because $fg$ lies in a prime $\mathfrak{p}$ if and only if one
of $f,g$ does.
\end{enumerate}

Now let us describe what the Zariski topology has to do with localization.

\begin{example} 
Let $R$ be a ring and $f \in R$.  Consider $S = \left\{1, f, f^2, \dots
\right\}$; this is a multiplicatively closed subset. Last week, we defined
$S^{-1}R$.
\end{example}

\begin{definition} 
For $S$ the powers of $f$, we write $R[f^{-1}]=S^{-1}R$. 
\end{definition} 

There is  a map $\phi: R \to R[f^{-1}]$ and a corresponding map
\[ \spec R[f^{-1}] \to \spec R  \]
sending a prime $\mathfrak{p} \subset R[f^{-1}]$ to $\phi^{-1}(\mathfrak{p})$.

\begin{proposition} 
This map induces a homeomorphism of $\spec R[f^{-1}]$ onto $U_f \subset \spec
R$. 
\end{proposition} 

So if you take a commutative ring and invert an element, you just get an open
subset of $\spec$. This is why it's called localization: you are restricting to
an open subset on the $\spec $ level when you invert something.

\begin{proof} 
\begin{enumerate}
\item First, we show that $\spec R[f^{-1}] \to \spec R$ lands in $U_f$. If
$\mathfrak{p} \subset R[f^{-1}]$, then we must show that the inverse image
$\phi^{-1}(\mathfrak{p})$ can't contain $f$. If otherwise, that would imply that
$\phi(f) \in \mathfrak{p}$; however, $\phi(f)$ is invertible, and then
$\mathfrak{p}$ would be $(1)$.  
\item Let's show that the map surjects onto $U_f$. If $\mathfrak{p} \subset R$ is a prime
ideal not containing $f$, i.e. $\mathfrak{p} \in U_f$. We want to construct a
corresponding prime in the ring $R[f^{-1}]$ whose inv. image is $\mathfrak{p}$.

Let $\mathfrak{p}[f^{-1}]$ be the collection of all fractions
\[ \{\frac{x}{f^n}, x \in \mathfrak{p}\} \subset R[f^{-1}],  \]
which is evidently an ideal. Note that whether the numerator is in
$\mathfrak{p}$ is \textbf{independent} of the
representing fraction $\frac{x}{f^n}$ used.\footnote{Suppose $\frac{x}{f^n} =
\frac{y}{f^k}$ for $y \in \mathfrak{p}$. Then there is $N$ such that
$f^N(f^k x - f^n y) = 0 \in \mathfrak{p}$; since $y \in \mathfrak{p}$ and $f
\notin \mathfrak{p}$, it follows that $x \in \mathfrak{p}$.}
In fact, $\mathfrak{p}[f^{-1}]$ is a prime ideal. Indeed, suppose 
\[  \frac{a}{f^m} \frac{b}{f^n} \in \mathfrak{p}[f^{-1}] .\]
Then $\frac{ab}{f^{m+n}}$ belongs to this ideal, which means $ab \in
\mathfrak{p}$; so one of $a,b \in \mathfrak{p}$ and one of the two fractions
$\frac{a}{f^m}, \frac{b}{f^n}$ belongs to $\mathfrak{p}[f^{-1}]$. Also, $1/1
\notin \mathfrak{p}[f^{-1}]$.

It is clear that the inverse image of $\mathfrak{p}[f^{-1}]$ is $\mathfrak{p}$,
because the image of $x \in R$ is $x/1$, and this belongs to
$\mathfrak{p}[f^{-1}]$ precisely wehn $x \in \mathfrak{p}$.
\item The map $\spec R[f^{-1}] \to \spec R$ is injective. Suppose
$\mathfrak{p}, \mathfrak{p'}$ are prime ideals in the localization and the
inverse images are the same.  
We must show that $\mathfrak{p} = \mathfrak{p'}$.

Suppose $\frac{x}{f^n} \in \mathfrak{p}$.  Then $x/1 \in \mathfrak{p}$, so $x
\in \phi^{-1}(\mathfrak{p}) = \phi^{-1}(\mathfrak{p}')$.  This means that $x/1
\in \mathfrak{p}'$, so 
$\frac{x}{f^n} \in \mathfrak{p}'$ too.  So a fraction that belongs to
$\mathfrak{p}$ belongs to $\mathfrak{p}'$. By symmetry the two ideals must be
the same.  
\item We now know that the map $\psi: \spec R[f^{-1}] \to U_f$ is a continuous
bijection. It is left to see that it is a homeomorphism.  We will show that it
is open.  
In particular, we have to show that a basic open set on the left side is mapped
to an open set on the right side.
If $y/f^n \in R[f^{-1}]$, we have to show that $U_{y/f^n} \subset \spec
R[f^{-1}]$ has open image under $\psi$.  We'll in fact show what open set it is
.

I claim that
\[ \psi(U_{y/f^n}) = U_{fy} \subset \spec R.  \]
To see this, $\mathfrak{p}$ is contained in $U_{f/y^n}$. This mean that
$\mathfrak{p}$ doesn't contain $y/f^n$.  In particular, $\mathfrak{p}$ doesn't
contain  the multiple $yf/1$.  So $\psi(\mathfrak{p})$ doesn't contain $yf$.
This proves the inclusion $\subset$.  

To complete the proof of the claim, and
the result, we must show that if $\mathfrak{p} \subset \spec R[f^{-1}]$ and
$\psi(\mathfrak{p}) = \phi^{-1}(\mathfrak{p}) \in U_{fy}$, then $y/f^n$ doesn't
belong to $\mathfrak{p}$.  (This is kosher and dandy because we have a bijection.) But the hypothesis implies that $fy \notin
\phi^{-1}(\mathfrak{p})$, so $fy/1 \notin \mathfrak{p}$.  Dividing by $f^{n+1}$
implies that
\[ y/f^{n} \notin \mathfrak{p}  \]
and $\mathfrak{p} \in U_{f/y^n}$. 
\end{enumerate}
\end{proof} 

If $\spec R$ is a space, and $f$ is thought of as a ``function'' defined on
$\spec R$, the space $U_f$ is to be thought of as the set of points where $f$
``doesn't vanish'' or ``is invertible.''
Thinking about rings in terms of their spectra is a very useful idea, though we
don't make too much use of it. 

We will bring it up when appropriate.  

\begin{remark} 
The construction $R \to R[f^{-1}]$ as discussed above is an instance of
localization.  More generally, we can define $S^{-1}R$ for $S \subset R$
multiplicativelly closed. We can define maps
\[ \spec S^{-1}R \to \spec R . \]
How can you think about the construction in general? You can think of it as
\[ \varinjlim_{f \in S} R[f^{-1}]  \]
which is a direct limit when you invert more and more elements.  

As an example, consider $S = R - \mathfrak{p}$ for a prime $\mathfrak{p}$, and for
simplicity that $R$ is countable. We can write $S =
S_0 \cup S_1 \cup \dots$, where each $S_k$ is generated by a finite number of
elements $f_0, \dots, f_k$.  Then $R_{\mathfrak{p}} = \varinjlim S_k^{-1} R$.
So we have
\[ S^{-1}R = \varinjlim_k R[f_0^{-1} , f_1^{-1}, \dots, f_k^{-1}  ] = \varinjlim
R[(f_0\dots f_k)^{-1}]. \]
The functions we invert in this construction are precisely those which do not
contain $\mathfrak{p}$, or where ``the functions don't vanish.''  The idea is
that to construct $\spec S^{-1}R = \spec R_{\mathfrak{p}}$, we keep cutting out
from $\spec R$ vanishing locuses of various functions that do not
intersect $\mathfrak{p}$.  In the end, you don't restrict to an open set, but
to a direct limit of this.
\end{remark} 

