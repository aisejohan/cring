\usepackage{amscd}
\chapter{Homological Algebra} 

Though strictly homological algebra doesn't
belong to the study of commutative algbra, it is so widely used in fields
such as geometry and algebra that it would be a waste not to write a short
exposition on it. Throughout this chapter we will assume that our objects are
modules over a commutative ring though the results hold for any ``abelian
category.'' Historically homological algebra developed out of algebraic
topology; originally homology was considered a series of numbers that keep
track of what we now call the ``Betti numbers.'' The section on history can
be skipped without impairing understanding but should be read if the reader
is looking for the motivation for some of the ideas we will study.

\subsection{Origins of Homology: Simplicial Homology}

The first homology theory
to be developed was simplicial homology - the study of homology of simplicial
complexes. To be simple, we will not develop the general theory and instead
motivate our definitions with a few basic examples.  

\begin{example} Suppose
our simplicial complex has one line segment with both ends identified at
one point $p$. Call the line segment $a$. The $n$-th homology group of this
space roughly counts how many ``different ways'' there are of finding $n$
dimensional sub-simplices that have no boundary that aren't the boundary of
any $n+1$ dimensional simplex. For the circle, notice that for each integer,
we can find such a way (namely the simplex that wraps counter clockwise that
integer number of times). The way we compute this is we look at the free abelian group generated by $0$ simplices, and $1$ simplices (there are no simplices of
dimension $2$ or higher so we can ignore that). We call these groups $C_0$ and
$C_1$ respectively. There is a boundary map $\partial_1: C_1\rightarrow C_0$.
This boundary map takes a $1$-simplex and associates to it its end vertex minus
its starting vertex (considered as an element in the free abelian group on
vertices of our simplex). In the case of the circle, since there is only one
$1$-simplex and one $0$-simplex, this map is trivial. We then get our homology
group by looking at $\ker(\partial_1)$. In the case that there is a nontrivial
boundary map $\partial_2: C_2\rightarrow C_1$ (which can only happen when our
simplex is at least $2$-dimensional), we have to take the quotient
$\ker(\partial_1)/\ker(\partial_2)$. This motivates us to define homology in a
general setting.
\end{example}

\subsection{Chain Complexes}
The chain complex is the most fundamental construction in homological algebra.
\begin{definition} A chain complex is a collection of modules over a ring $C_i$
for integer values of $i$. Usually we will find that $C_i=0$ for negative valuesof $i$. Equipped with the $C_i$ are boundary operators
$\partial_i:C_i\rightarrow C_{i-1}$ with the property that
$\partial_{i-1}\partial_i=0$. Sometimes the boundary map is also called the
``differential.'' Often, notation is abused and the indices for the boundary map are dropped.\end{definition}
\begin{example} All exact sequences are chain complexes. \end{example}
\begin{example} Any sequence of abelian groups for which the boundary operators
are all $0$ is a chain complex. \end{example}
We will see plenty of more examples in due time. 
\begin{definition} Let $C_*$ be a chain complex with boundary map $\partial_*$.
We define homology objects $h_i(C_*)=\ker(\partial_i)/Im(\partial_{i+1})$.
\end{definition}
\begin{example} In a chain complex $C_*$ where all the boundary maps are
trivial, $h_i(C_*)=C_i$. \end{example}

Often we will bundle all the modules $C_i$ of a chain complex together to form a graded module $C_*=\bigoplus_i C_i$. In this case, the boundary operator is a
endomorphism that takes elements from degree $i$ to degree $i-1$. Similarly, we
often bundle together all the homology modules to give a graded homology module
$h_*(C_*)=\bigoplus_i h_i(C_*)$.

\subsection{Functoriality}
It turns out that chain complexes form a category when morphisms are appropriately defined.
\begin{definition} A map of chain complexes $f:C_*\rightarrow D_*$, or a chain map, is a sequence of maps $f_i:C_i\rightarrow D_i$ such that $f\partial = \partial' f$ where $\partial$ is the boundary map of $C_*$ and $\partial'$ of $D_*$ (again we are abusing notation and dropping indices). \end{definition}
\begin{theorem} Chain maps induce maps in homology in the same direction; in other words, homology is a covariant functor from the category of chain complexes to the category of graded modules.\end{theorem}
\begin{proof}
Let $f:C_*\rightarrow D_*$ be a chain map. Let $\partial$ and $\partial'$ be the differentials for $C_*$ and $D_*$ respectively. Then we have a commutative diagram:

\begin{equation}
\begin{CD}
C_{i+1} @>\partial_{i+1}>> C_i @>>\partial_i> C_{i-1}\\
@VV f_{i+1} V          @VV f_i V             @VVf_{i-1} V\\
D_{i+1} @>\partial'_{i+1}>> D_i @>>\partial'_i> D_{i-1}
\end{CD}
\end{equation}

Now, in order to check that a chain map $f$ induces a map $f_*$ on homology, we need to check that $f_*(Im(\partial))\subseteq Im(\partial')$ and $f_*(\ker(\partial))\subseteq \ker(\partial)$. We first check the condition on images: we want to look at $f_i(Im(\partial_{i+1}))$. By commutativity of $f$ and the bounary maps, this is equal to $\partial'_{i+1}(Im(f_{i+1})$. Hence we have $f_i(Im(\partial_{i+1}))\subseteq Im(\partial_{i+1}')$. For the condition on kernels, let $x\in \ker(\partial_i)$. Then by commutativity, $\partial'_i(f_i(x))=f_{i-1}\partial_i(x)=0$. Thus we have that $f$ induces for each $i$ a homomorphism $f_i:h_i(C_*)\rightarrow h_i(D_*)$ and hence it induces a homomorphism on homology as a graded module. \end{proof}

As seen above, in homological algebra, one often deals with the images and kernels of boundary maps; we will simplify notation by introducing the following:
\begin{definition} The submodule of cycles $Z_i\subseteq C_i$ is the kernel $\ker(\partial_i)$. The submodule of boundaries $B_i\subseteq C_i$ is the image $Im(\partial_{i+1})$. Thus homology is said to be ``cycles mod boundaries,'' i.e. $Z_i/B_i$.\end{definition}
To further simplify notation, often all differentials regardless of what chain complex they are part of are denoted $\partial$, thus the commutativity relation on chain maps is $f\partial=\partial f$ with indices and differention between the boundary operators dropped. 

\subsection{Cochain Complexes}
Cochain complexes are much like chain complexes except the arrows point in the
opposite direction.
\begin{definition} A cochain complex is a series of modules $C_i$ for integer
values of $i$ (again, it is common that $C_i=0$ for negative values of $i$). The
cochain complex is also equipped with coboundary operators, also called
differentials, $\partial_i:C_i\rightarrow C_{i+1}$ such that
$\partial_{i+1}\partial_i=0$. Given a cochain complex $C^*$, we define the
cohomology objects to be $h^i(C^*)=\ker(\partial_i)/Im(\partial_{i-1})$.
\end{definition}
Similarly, we can also turn cochain complexes and cohomology modules into a
graded module.

\subsection{Chain Homotopies}
Returning to algebraic topology as a motivation, originally homology was
intended to be a homotopy invariant meaning that space with the same homotopy
type would have isomorphic homology modules. In fact, any homotopy induces what
is now known as a chain homotopy on the simplicial chain complexes; it can be
shown that when such a chain homotopy exists, the homology modules are
isomorphic.

\subsection{Resolutions}
\begin{definition} A projective object in an abelian category is an object $P$
such that for any map $P\rightarrow M$ and a surjection $N\rightarrow M$, there
is a lift $P\rightarrow N$ making the diagram commute. \end{definition}
\begin{example} All free objects are projective. \end{example}

Often it is useful to study how projective objects map into a particular module.
This motivates the following definition.
\begin{definition} Let $M$ be an arbitrary module, a projective resolution of
$M$ is an exact sequence
\begin{equation} \cdots\rightarrow P_i\rightarrow P_{i-1}\rightarrow
P_{i-2}\cdots\rightarrow P_1\rightarrow P_0\rightarrow M \end{equation} where
the $P_i$ are projective modules. \end{definition}
\begin{theorem}An abelian category has enough projectives if each object has a
projective resolution. In particular, the category of modules over a ring has
enough projectives. \end{theorem}
\begin{proof} Pick a generating set $S_0$ that generates $M$. Then take the free
module generated by $S_0$, since free objects are projective, this free module
is a projective object - call it $P_0$. Then we look at the kernel
$\ker(P_0\rightarrow M)$. We pick a generating set for this call it $S_1$. Look
at the free module generated by $S_1$, call it $P_1$; this object is also
projective and maps onto $\ker(P_0\rightarrow M)$. Inductively repeating this
construction gives a series of projcetive objects $P_i$ and an exact sequence
with the last term being $M$. Hence we have constructed a projective resolution
of $M$. \end{proof}
\begin{example} The abelian group $\mathbb{Z}/2$ has the free resolution $0\rightarrow\cdots 0\rightarrow\mathbb{Z}\rightarrow\mathbb{Z}\rightarrow\mathbb{Z}/2$.
Similarly, since any finitely generated abelian group can be decomposed into the direct sum of torsion subgroups and free subgroups, all finitely generated abelian groups admit a resolution of the type shown above.\end{example}

\subsection{Derived Functors}
Often in homological algebra, we see that ``short exact sequences induce long exact sequences.'' Using the theory of derived functors, we can make this formal.
\begin{definition} A right-exact functor is an additive functor between two abelian categories $F$ such that for every short exact sequence $0\rightarrow A\rightarrowB\rightarrow C\rightarrow 0$, we get a exact sequence $F(A)\rightarrow F(B)\rightarrow F(C)\rightarrow 0$. \end{definition}
\begin{example} The example that we will extensively study is the tensory product, $-\otimes M$ for any $R$-module $M$ is right-exact.\end{example}
Now, what would be nice is if we could continue the exact sequence $F(A)\rightarrow F(B)\rightarrow F(C)\rightarrow 0$ on the left and make a long exact sequence; derived functors allow for this to happen.

\begin{definition} Let $M$ be an object in an abelian category that has enough projectives. Let $P_*\rightarrow M$ be a resolution. Then the left derived functors $L_iF(-)$ are defined to be the homology objects $h_i(F(P_*))$ (the degree $0$ term of this complex is the last term to the left of $M$ in the resolution).\end{definition}

