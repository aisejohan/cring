\chapter{Foundations}

We begin with a few historical remarks.  Fermat's last theorem states that the
equation
  \[ \label{ft} x^n  + y^n = z^n \]
has no nontrivial solutions in the integers, for $n \ge 3$.  We could try to
prove this by factoring the expression on the left hand side. We can write
  \[ (x+y)(x+ \zeta y) (x+ \zeta^2y) \dots (x+ \zeta^{n-1}y) = z^n, \]
where $\zeta$ is a primitive $n$th root of unity.  Unfortunately, the factors
lie in $\mathbb{Z}[\zeta]$, not the ring of integers $\mathbb{Z}$.  Though
$\mathbb{Z}[\zeta]$ is still a ring where we have notions of primes and
factorization, just as in $\mathbb{Z}$, we will see that prime factorization
is not always unique in $\mathbb{Z}[\zeta]$.

Here is another case where unique factorization fails.  Consider the ring
$\mathbb{Z}[\sqrt{-5}]$ of complex numbers of the form $a + b\sqrt{-5}$, where
$a, b \in \mathbb{Z}$.  Then we have the two factorizations
  \[ 6 = 2 \cdot 3 = (1 + \sqrt{-5})(1 - \sqrt{-5}). \]
Both of these are factorizations of 6 into irreducible factors, but they
are fundamentally different.

\section{Commutative rings and their ideals}

\begin{definition} 
A \textbf{commutative ring} is a set $R$ with an addition map
$+ : R \times R \to R$ and a multiplication map $\times : R \times R \to R$
that satisfy the following conditions.

\begin{enumerate}
  \item $R$ is a group under addition.
  \item The multiplication map is commutative and distributes over addition.
  \item There is a \textbf{unit} (or \textbf{identity element}), denoted by
        $1$, such that $1 \times x = x$ for all $x \in R$.
\end{enumerate}

Given a ring, a \textbf{subring} is a subset that contains the identity
element and is closed under addition and multiplication.
\end{definition} 

\begin{example} 
$\mathbb{Z}$ is the simplest example of a ring.
\end{example} 

\begin{example}
For any ring $R$, we can consider the polynomial ring $R[x_1, \ldots, x_n]$
which consists of the polynomials in $n$ variables with coefficients in $R$.
\end{example}

\subsection{Ideals}

\begin{definition}
Let $R$ be a ring.  An \textbf{ideal} in $R$ is a subset $I \subset R$ that
satisfies the following.

\begin{enumerate}
  \item $0 \in I$.
  \item If $x, y \in I$, then $x + y \in I$.
  \item If $x \in I$ and $y \in R$, then $xy \in I$.
\end{enumerate}
\end{definition}

Given some $x_1, \ldots, x_n \in R$, we denote by $(x_1, \ldots, x_n) \subset
R$ the subset of linear combinations $\sum r_i x_i$, where $r_i \in R$.  This
is clearly an ideal, and in fact the smallest one containing all $x_i$.  It is
called the ideal \textbf{generated} by $x_1, \ldots, x_n$.  A
\textbf{principal ideal} $(x)$ is one generated by a single $x \in R$.

\begin{example}
The ideal generalizes the notion of divisibility in $\mathbb{Z}$.  Note that
in $\mathbb{Z}$, the set of elements divisible by $n \in \mathbb{Z}$ forms the
ideal $I = n\mathbb{Z} = (n)$.
\end{example}

\begin{exercise} 
Show that the ideal $(2, 1 + \sqrt{-5}) \subset \mathbb{Z}[\sqrt{-5}]$ is not
principal.
\end{exercise}

\paragraph{Operations on ideals.}

\begin{definition}
The sum $I + J$ of two ideals $I, J \subset R$ is defined as the set
  \[ \left\{ x_i + y_i : x_i \in I, y_i \in J \right\}. \]
\end{definition}

\begin{definition}
The project $IJ$ of two ideals $I, J \subset R$ is defined as the smallest
ideal containing the products $xy$ for all $x \in I, y \in J$. This is just
the set
  \[ \left\{ \sum x_i y_i : x_i \in I, y_i \in J \right\}. \]
\end{definition}

\begin{exercise}
Given ideals $I, J \subset R$, verify the following.

\begin{enumerate}
  \item $I + J$ is the smallest ideal containing $I$ and $J$.
  \item $I \cap J$ is an ideal.
\end{enumerate}
\end{exercise}

\begin{example}
In $\mathbb{Z}$, we have the following for any $m, n$.

\begin{enumerate}
  \item $(m) + (n) = (\gcd\{ m, n \})$,
  \item $(m)(n) = (mn)$,
  \item $(m) \cap (n) = (\mathrm{lcm}\{ m, n \})$.
\end{enumerate}
\end{example}

\begin{proposition}
For ideals $I, J, K \subset R$, we have the following.

\begin{enumerate}
  \item Distributivity: $I(J + K) = IJ + IK$.
  \item $I \cap (J + K) = I \cap J + I \cap K$ if $I \supseteq J$ or $I \supseteq K$.
  \item If $I + J = R$, $I \cap J = IJ$.
\end{enumerate}

\begin{proof}
1 and 2 are clear.  For 3, note that $(I + J)(I \cap J) = I(I \cap J)
+ J(I \cap J) \subseteq IJ$.  Since $IJ \subseteq I \cap J$, the result
follows.
\end{proof}
\end{proposition}


\subsection{The categorical structure}
The class of rings forms a category. Its morphisms are called ring homomorphisms.


\begin{definition}
A \textbf{ring homomorphism} between two rings $R$ and $S$ as a map
$f : R \to S$ that respects addition and multiplication. That is,

\begin{enumerate}
  \item $f(1_R) = 1_S$, where $1_R$ and $1_S$ are the respective identity
        elements.
  \item $f(a + b) = f(a) + f(b)$ for $a, b \in R$.
  \item $f(ab) = f(a)f(b)$ for $a, b \in R$.
\end{enumerate}
\end{definition}

The philosophy of Grothendieck, as expounded in his EGA, is that one should
always do things in a relative context. This means that instead of working
with objects, one should work with \emph{morphisms} of objects. Motivated by
this, we introduce:

\begin{definition} 
Given a ring $A$, an \textbf{$A$-algebra} is a ring $R$ together with a
morphism of rings (a \emph{structure morphism}) $A \to R$. There is a category of $A$-algebras, where a
morphism between $A$-algebras is required to commute with the structure
morphisms. 
\end{definition} 

\begin{example} 
Every ring is a $\mathbb{Z}$-algebra in a natural and unique way.
\end{example}

After this series of technical definitions and discussion of unique
factorization, we shall give several more  examples.

\begin{exercise}\label{polynomial} Let $R$ be a commutative ring.
Show that the set of polynomials in one variable over $R$ is a commutative
ring $R[x]$, in fact an $R$-algebra. Give a rigorous definition of this.
\end{exercise} 
\begin{exercise} 
If $R$ is a commutative ring, recall that an \textbf{invertible element} (or, somewhat
confusingly, a \textbf{unit}) $u \in R$ is an element such
that there exists $v \in R$ with $uv = 1$. Prove that $v$ is necessarily
unique.
\end{exercise}
\begin{exercise} 
More generally, if $R$ is a ring and $G$ a commutative monoid, then the set
$R[G]$ of formal finite sums $\sum r_i g_i$ with $r_i \in R, g_i \in G$ is a
commutative ring, called the \textbf{group ring}. The case of $G =
\mathbb{Z}_{\geq 0}$ is the polynomial ring. 
\end{exercise} 

\begin{exercise}
\label{integersinitial}
The ring $\mathbb{Z}$ is an \emph{initial object} in the category of rings.
That is, for any ring $R$, there is a \emph{unique} morphism of rings
$\mathbb{Z} \to R$.
\end{exercise} 

\begin{exercise} 
The ring where $0=1$ (the \textbf{zero ring}) is a \emph{final object} in the category of rings. That
is, every ring admits a unique map to the zero ring.	
\end{exercise} 

\begin{exercise} 
Let $X$ be a set and $R$ a ring. The set $R^X$ of functions $f:X \to R$ is a
ring. If $S \subset X$, then the set of functions $f: X \to R$ that vanish on
$S$ is an ideal.
\end{exercise}

\begin{exercise}
\label{corepresentable}
Let $\mathcal{C}$ be a category and $F: \mathcal{C} \to \mathbf{Sets}$  a
covariant functor. Recall that $F$ is said to be \textbf{corepresentable} if
$F$ is naturally isomorphic to $X \to \hom_{\mathcal{C}}(U, X)$ for some
object $U \in \mathcal{C}$. For instance, the functor sending everything to a
one-point set is corepresentable if and only if $\mathcal{C}$ admits an
initial object.

Prove that the functor  $\mathbf{Rings} \to \mathbf{Sets}$ assigning to each ring its underlying set is
representable. (Hint: use a suitable polynomial ring.)
\end{exercise} 

\section{Further examples}

We now illustrate a few important examples of 
commutative rings. The section is in large measure an advertisement for why
one might care about commutative algebra; nonetheless, the reader is
encouraged at least to skim this section.

\subsection{Rings of holomorphic functions}

The following subsection may be omitted without impairing understanding.

There is a fruitful analogy in number theory between the rings $\mathbb{Z}$ and
$\mathbb{C}[t]$, the latter being the polynomial ring over $\mathbb{C}$ in one
variable (Exercise \ref{polynomial}).  Why are they analogous? Both of these rings have a theory of unique
factorization:  that is, factorization into primes or irreducible polynomials. (In the
latter, the irreducible polynomials have degree one.)
Indeed we know:
\begin{enumerate}
\item Any nonzero integer factors as a product of primes (possibly times $-1$). 
\item Any  nonzero polynomial factors as a product of an element of
$\mathbb{C}^* =\mathbb{C} - \left\{0\right\}$ and polynomials of the form $t -
a, a \in \mathbb{C}$.
\end{enumerate}


There is another way of thinking of $\mathbb{C}[t]$ in terms of complex
analysis.  This is equal to the ring of holomorphic functions on $\mathbb{C}$
which are meromorphic at infinity.  
Alternatively, consider the Riemann sphere $\mathbb{C} \cup \{ \infty\}$; then the ring $\mathbb{C}[t]$
consists of meromorphic functions on the sphere whose poles (if any) are at
$\infty$. 

This description admits generalizations. 
Let $X$ be a
Riemann surface.  (Example: take the complex numbers modulo a lattice, i.e. an
elliptic curve.)
Suppose that $x \in X$. Define $R_x$ to be the ring of meromorphic functions on $X$
which are allowed poles only at $x$ (so are everywhere else holomorphic). 

\begin{example} Fix the notations of the previous discussion. 
Fix $y \neq x \in X$. Let $R_x$ be the ring of meromorphic functions on the
Riemann surface $X$ which are holomorphic on $X - \left\{x\right\}$, as before.
Then the collection of functions that vanish at $y$ forms an
\emph{ideal} in $R_x$. 

There are lots of other ideals. For instance, fix two
points $y_0, y_1 \neq x$; we look at the ideal of $R_x$ that vanish at both $y_0, y_1$.

\end{example} 


\textbf{For any Riemann surface $X$, the conclusion of Dedekind's theorem  \ref{ded1} applies.  } In other
words, the ring  $R_x$ as defined in the example admits  unique factorization of
ideals. We shall call such rings \textbf{Dedekind domains} in the future.

\begin{example} Keep the preceding notation.

Let $f \in R_x$, nonzero. By definition, $f$ may have a pole at $x$, but no poles elsewhere. $f$ vanishes
at finitely many points $y_1, \dots, y_m$. When $X$ was the Riemann sphere,
knowing the zeros of $f$ told us something about $f$. Indeed, in this case
$f$ is just a
polynomial, and we have a nice factorization of $f$ into functions in $R_x$ that vanish only
at one point. In general Riemann surfaces, this
is not generally possible.  This failure turns out to be very interesting.

Let $X = \mathbb{C}/\Lambda$ be an elliptic curve (for $\Lambda \subset
\mathbb{C}^2$ a lattice), and suppose $x = 0$. Suppose we
are given $y_1, y_2, \dots, y_m \in X$ that are nonzero; we ask whether there
exists a function $f \in R_x$ having simple zeros at $y_1, \dots, y_m$ and nowhere else.
The answer is interesting, and turns out to recover the group structure on the
lattice.

\begin{proposition} 
A function $f \in R_x$ with simple zeros only at the $\left\{y_i\right\}$ exists if and only if $y_1 + y_2 + \dots + y_n = 0$ (modulo $\Lambda$).

\end{proposition} 
So this problem of finding a function with specified zeros is equivalent to
checking that the specific zeros add up to zero with the group structure.

In any case, there might not be such a nice function, but we have at least an
ideal $I$ of functions that have zeros (not necessarily simple) at $y_1, \dots,
y_n$.  This ideal has unique factorization into the ideals of functions
vanishing at $y_1$, functions vanishing at $y_2$, so on.  
\end{example} 


\subsection{Ideals and varieties}

We saw in the previous subsection that ideals can be thought of as the
vanishing of functions. This, like divisibility, is another interpretation,
which is particularly interesting in algebraic geometry.


Recall the  ring $\mathbb{C}[t]$ of complex polynomials discussed in the
last subsection. More generally, if $R$ is a ring,  we saw in
Exercise~\ref{polynomial} that the set $R[t]$ of polynomials with coefficients
in $R$
is a ring.  This is a construction that
can be iterated to get a polynomial ring in several variables over $R$.

\begin{example} 
Consider the polynomial ring $\mathbb{C}[x_1, \dots, x_n]$. Recall that before
we thought of the ring $\mathbb{C}[t]$ as a ring of meromorphic functions.  
Similarly each element of the polynomial ring $\mathbb{C}[x_1, \dots, x_n]$
gives a function $\mathbb{C}^n \to \mathbb{C}$; we can think of the polynomial
ring as sitting inside the ring of all functions $\mathbb{C}^n \to \mathbb{C}$.

A question you might ask: What are the ideals in this ring?  One way to get an
ideal is to pick a point $x=(x_1, \dots, x_n) \in \mathbb{C}^n$; consider the
collection of all functions $f \in \mathbb{C}[x_1, \dots, x_n]$ which vanish on
$x$; by the usual argument, this is an ideal.

There are, of course, other ideals. More generally, if $Y \subset
\mathbb{C}^n$, consider the collection of polynomial functions $f:
\mathbb{C}^n \to \mathbb{C}$ such that $f \equiv 0$ on
$Y$.  This is easily seen to be an ideal in the polynomial ring. We thus have a
way of taking a subset of $\mathbb{C}^n$ and producing an ideal.
Let $I_Y$ be the ideal corresponding to $Y$.  

This construction is not injective. One can have $Y \neq Y'$ but $I_Y = I_{Y'}$. For instance, if $Y$ is dense in
$\mathbb{C}^n$, then $I_Y = (0)$, because the only way a continuous function on
$\mathbb{C}^n$ can vanish on $Y$ is for it to be zero.

There is a much closer connection in the other direction. You might ask whether
all ideals can arise in this way. The quick answer is no---not even when $n=1$. The ideal $(x^2) \subset \mathbb{C}[x]$ cannot be obtained
in this way.  It is easy to see that the only way we could get this as $I_Y$ is
for $Y=\left\{0\right\}$, but $I_Y$ in this case is just $(x)$, not $(x^2)$.
What's going wrong in this example is that $(x^2)$ is not a \emph{radical}
ideal.
\end{example} 

\begin{definition}\label{def-radical-ideal} 
An ideal $I \subset R$ is \textbf{radical} if whenever $x^2 \in I$, then $x \in
I$.
\end{definition} 

The ideals $I_Y$ in the polynomial ring are all radical.  This is obvious.
You might now ask whether this is the only obstruction. We now state a theorem
that we will prove later in this class.

\begin{theorem}[Hilbert's Nullstellensatz] If $I \subset \mathbb{C}[x_1, \dots,
x_n]$ is a radical ideal, then $I = I_Y$ for some $Y \subset \mathbb{C}^n$. In
fact, the canonical choice of $Y$ is the set of points where all the functions
in $Y$ vanish.\footnote{Such a subset is called an algebraic variety.}
\end{theorem} 


This will be one of the highlights of the present course. But before we can
get to it, there is much to do.

\begin{exercise} 
Assuming the Nullstellensatz, show that any \emph{maximal} ideal in the
polynomial ring $\mathbb{C}[x_1, \dots, x_n]$ is of the form 
$(x_1-a_1, \dots, x_n-a_n)$ for $a_1, \dots, a_n \in \mathbb{C}$. An ideal of a
ring is called \textbf{maximal} if the only ideal that contains it is the
whole ring (and it itself is not the whole ring).
\end{exercise} 

\section{Modules over a commutative ring}



We will now establish some basic terminology about modules.

\subsection{Definitions}
Suppose $R$ is a commutative ring.  

\begin{definition} 
An \textbf{$R$-module $M$} is an abelian group $M$ with a map $R \times M \to
M$ (written $(a,m) \to am$) such that
\begin{enumerate}[\textbf{M} 1]
\item  $(ab) m = a(bm)$ for $a,b \in R, m \in M$, i.e. there is an associative law. 
\item $1m
= m$; the unit acts as the identity. 
\item There are distributive laws
on both sides:
$(a+b)m = am + bm$ and $a(m+n) = am + an$ for $a,b \in R, \ m,n \in M$.

\end{enumerate} \end{definition} 

Another definition can be given  as follows.
\begin{definition} 
If $M$ is an abelian group, $End(M)$ is the set of homomorphisms $f: M \to M$.  
This can be made into a (noncommutative) \emph{ring}.\footnote{A
noncommutative ring is one satisfying all the usual axioms of a ring except
that multiplication is not required to be commutative.} Addition is defined pointwise, and
multiplication is by composition. The identity element is the identity
function $1_M$.
\end{definition} 

We made the following definition earlier for commutative rings, but for
clarity we re-state it:
\begin{definition} 
If $R, R'$ are rings (possibly noncommutative) then a function $f: R \to R'$ is a
\textbf{ring-homomorphism}  or \textbf{morphism} if it is compatible with the
ring structures, i.e
\begin{enumerate}
\item  $f(x+y) = f(x) + f(y)$
\item $f(xy) = f(x)f(y)$
\item  $f(1) = 1$.
\end{enumerate}
\end{definition} 

The last condition is not redundant because otherwise the zero map would
automatically be a homomorphism.
The alternative definition of a module is left to the reader in the following
exercise.
\begin{exercise}
If $R$ is a ring and $R \to End(M)$ a homomorphism, then $M$ is made into an
$R$-module, and vice versa.  
\end{exercise}


\begin{example} 
if $R$ is a ring, then $R$ is an $R$-module by multiplication on the left. 
\end{example} 
\begin{example} 
A $\mathbb{Z}$-module is the same thing as an abelian group.
\end{example} 

\begin{definition} 
If $M$ is an $R$-module, a subset $M_0 \subset M$ is a \textbf{submodule} if it
is a subgroup (closed under addition and inversion) and is closed under
multiplication by elements of $R$, i.e. $aM_0 \subset M_0$ for $a \in R$. A
submodule is a module in its own right. If $M_0 \subset M$ is a submodule,
there is a commutative diagram:
\[ \xymatrix{
R \times M_0 \ar[d] \ar[r] &  M_0 \ar[d] \\ R \times M \ar[r] &  M
}.\]
Here the horizontal maps are multiplication.
\end{definition} 

\begin{example} 
Let $R$ be a (\textbf{commutative}) ring; then an ideal in $R$ is the same thing as a
submodule of $R$.
\end{example} 

\begin{example} 
If $A$ is a ring, an $A$-algebra is an $A$-module in an obvious way. More
generally, if $A$ is a ring and $R$ is an $A$-algebra, any $R$-module becomes
an $A$-module by pulling back the multiplication map via $A \to R$. 
\end{example} 



Dual to submodules  is the notion of a \emph{quotient module}, which we define
next:
\begin{definition} Suppose $M$ is an $R$-module and $M_0$  a
submodule.  Then the abelian group $M/M_0$ (of cosets)  is an $R$-module,
called the \textbf{quotient module} by $M_0$.  

Multiplication is as follows. If
one has a coset $x  + M_0 \in M/M_0$, one  multiplies this by $a \in R$ to
get the coset $ax
+ M_0$. This does not depend on the coset representative.  
\end{definition} 


\subsection{Quotient rings}
We shall next see that when one quotients by an \emph{ideal}, one gets not
merely a module but actually another \emph{ring.}


If $R$ is a ring and $I \subset R$ an ideal, then $R/I$ is an $R$-module. The
multiplication is $a(b+I) = ab + I$. 

As one easily checks, this descends further to a multiplication 
\[ R/I \times R/I \to R/I  \] such that there is a commutative diagram
\[ \xymatrix{
R \times R/I \ar[rd] \ar[rr] & &  R/I \\
& R/I \times R/I \ar[ru] 
}.\]
In particular, $R/I$ is a ring, under multiplication $(a+I)(b+I) = ab+I$. 
\begin{definition} 
$R/I$ is called the \textbf{quotient ring} by the ideal $I$.
\end{definition} 

The
reduction map $\phi \colon R \to R/I$ is a ring-homomorphism with a
\emph{universal
property}.
Namely, for any ring $B$, there is a map
 \[ \hom(R/I, B) \to \hom(R, B)  \]
 on the hom-sets
 by composing with the ring-homomorphism $\phi$; this map is injective and the
 image consists of all homomorphisms $R \to B$ which vanish on $I$.  
Stated alternatively, to map out of $R/I$ (into some ring $B$) is the same thing as mapping out of
$R$ while killing the ideal $I \subset R$.

This is best thought out for oneself, but here is the detailed justification.
The reason is that any map $R/I \to B$ pulls back to a map $R \to R/I \to B$
which annihilates $I$ since $R \to R/I$ annihilates $I$. Conversely, if we have
a map 
\[ f: R \to B  \]
killing $I$, then we can define $R/I \to B$ by sending $a+I$ to $f(a)$; this is
uniquely defined since $f$ annihilates $I$.

\begin{exercise} 
 If $R$ is a commutative
ring, an element $e \in R$ is said to be \textbf{idempotent} if $e^2 =
e$. Define a covariant functor $\mathbf{Rings} \to \mathbf{Sets}$ sending a
ring to its idempotents. Prove that it is corepresentable. (Answer: the
corepresenting object is $\mathbb{Z}[X]/(X - X^2)$.)
\end{exercise} 

\begin{exercise} 
Show that the functor assigning to each ring the set of elements annihilated
by 2 is corepresentable. 
\end{exercise} 

\begin{exercise}
If $I \subset J \subset R$, then $J/I$ is an ideal of $R/I$, and there is a
canonical isomorphism
\[ (R/I)/(J/I) \simeq R/J.  \]
\end{exercise} 


\subsection{The categorical structure on modules}
So far, we have talked about modules, but we have not discussed morphisms
between modules, and have yet to make the class of modules over a given ring
into a category. This we do next.

Let us introduce a  few more basic notions.

\begin{definition} 
Let $R$ be a ring.  Suppose $M,N$ are $R$-modules.  A map $f: M \to N$
is a \textbf{module-homomorphism} if it preserves all the relevant structures.

Namely, it must be a homomorphism of abelian groups, $f(x+y) = f(x) + f(y)$,
and second it must
preserve multiplication:
$f(ax)  = af(x)$ for $a \in R, x \in M$. 
\end{definition}


Thus, for any commutative ring $R$, the class of $R$-modules and
module-homomorphisms forms a \textbf{category.}

\begin{example} 
If $a \in R$, then multiplication by $a$ is a module-homomorphism $M
\stackrel{a}{\to} M$ for any $R$-module $M$.
\end{example} 

\begin{definition} Let $f: M \to N$ be a module homomorphism.
In this case, the \textbf{kernel} $\ker f$ of $f$ is  the set of elements $m
\in M$ with $f(m)=0$. This is
a submodule of $M$, as is easy to see. 

The \textbf{image} $\im f$ of $f$ (the set-theoretic
image, i.e. the collection of all $f(x), x \in M$) is also a submodule of $N$. 

The
\textbf{cokernel} of $f$ is defined by
\(  N/\im(f).  \) 
\end{definition}

\begin{exercise} \label{univpropertykernel} 
The universal property of the kernel is as follows. Let $M \stackrel{f}{\to }
N$ be a morphism with kernel $K \subset M$. Let $T \to M$ be a map. Then $T \to M$ factors through the
kernel $K \to M$ if and only if its composition with $f$ (a morphism $T \to N$) is zero. 
In particular, if we think of the hom-sets as abelian groups (i.e.
$\mathbb{Z}$-modules)
\[ \hom( T,K) = \ker\left( \hom(T, M) \to \hom(T, N) \right). \]
\end{exercise} 
\begin{exercise} 
What is the universal property of the cokernel?
\end{exercise} 

\begin{exercise} 
On the category of modules, the functor assigning to each module $M$ its
underlying set is corepresentable (cf.\ Exercise \ref{corepresentable}).
\end{exercise} 

We shall now introduce the notions of \emph{direct sum} and \emph{direct
product}. Let $I$ be a set, and suppose that for each $i \in I$, we are given
an $R$-module $M_i$.

\begin{definition} 
The \textbf{direct product} $\prod M_i$ is set-theoretically the cartesian product. It is given
the structure of an $R$-module by addition and multiplication pointwise on
each factor. 
\end{definition} 
\begin{definition} 
The \textbf{direct sum} $\bigoplus_I M_i$ is the set of elements in the direct
product such that all but finitely many entries are zero. The direct sum is a
submodule of the direct product.
\end{definition} 


\begin{exercise} 
The direct product is a product in the category of modules, and the direct sum
is a coproduct.
\end{exercise} 

\add{filtered colimits}

\subsection{Exactness}
Finally, we introduce the notion of \emph{exactness}. 
\begin{definition} 
Let $f: M \to N$ be a morphism of $R$-modules.  Suppose $g: N \to P$ is another morphism of
$R$-modules.  

The pair of maps is a \textbf{complex} if $g \circ f = 0$. So $M \to N \to P$
is zero.
In particular, $\im(f) \subset \ker(g)$. 

This complex is \textbf{exact} (or exact at $N$) if $\im(f) = \ker(g)$. So
anything that is killed when you map to $P$ actually comes from something in
$M$. 

\end{definition} 


We shall often write pairs of maps as sequences
\[ A \stackrel{f}{\to} B \stackrel{g}{\to} C  \]
and say that the sequence is \textbf{exact} if the image of the map $f:A
\to B$ is the same as the kernel of the map $g: B \to C$.
A longer (possibly infinite) sequence of modules
\[ A_0 \to A_1 \to A_2 \to \dots  \]
will be called \textbf{exact} if it is exact at each step.

\begin{example} 
The sequence $0 \to A \stackrel{f}{\to} B$ is exact if and only if the map $f$
is injective. Similarly, $A \stackrel{f}{\to} B \to 0$ is exact if and only if
$f$ is surjective. 
\end{example} 

One typically sees this definition applied to sequences of the form
\[ 0 \to M'\stackrel{f}{ \to} M \stackrel{g}{\to} M'' \to 0,  \]
which, if exact, is called a \textbf{short exact sequence}. 
Exactness here means that $f$ is injective, $g$ is surjective, and $f$ maps
onto the image of $g$.  So $M''$ can be thought of as the quotient $M/M'$.

Suppose you have a functor  $F$ from the category of $R$-modules to the
category of  $S$-modules, where $R, S$ are rings.  Then:

\begin{definition} 
\begin{enumerate}
\item  $F$ is called \textbf{additive} if $F$ preserves direct sums.  
\item  $F$ is called \textbf{exact} if $F$ is additive and preserves exact sequences.  
\item  $F$ is called \textbf{left exact} if $F$ is additive and preserves exact sequences of the form
$0 \to M' \to M \to M''$.  In particular, $F$ preserves kernels.  
\item  $F$ is \textbf{right exact} if $F$ is additive and $F$ preserves exact
sequences of the form $M' \to M \to M'' \to 0$, i.e. $F$ preserves cokernels.  
\end{enumerate}
\end{definition} 

A functor is exact if and only if it is both left and right exact.  


\add{further explanation, exactness of filtered colimits}

\subsection{Free modules}

\begin{definition} 
A module $M$ is \textbf{free} if it is isomorphic to $\bigoplus_I R$ for some
index set $I$. 
\end{definition} 




\section{Ideals}

The notion of an \emph{ideal} has already been defined. Now we will introduce additional terminology related to the theory of ideals.

\subsection{Prime and maximal ideals}

If $R$ is any commutative ring, there are two obvious ideals. These obvious
ones are the zero ideal $(0)$
consisting only of the zero element, and the unit element $(1)$ consisting of all of
$R$.


The notion of a ``prime ideal'' is intended to generalize the familiar idea of a prime
number.

\begin{definition} 
An ideal $I \subset R$ is said to be \textbf{prime} if
\begin{enumerate}[\textbf{P} 1]
\item  $1 \notin I$ (by convention, 1 is not a prime number)
\item If $xy \in I$, either $x \in I$ or $y \in I$.
\end{enumerate}
\end{definition} 

\begin{example}
\label{integerprimes}
If $R = \mathbb{Z}$ and $p \in R$, then $(p) \subset \mathbb{Z}$ is a prime ideal iff $p$ is a
prime number (or the opposite of a prime number) or zero.
\end{example} 

\begin{definition} 
An ideal $I \subset R$ is called \textbf{maximal}\footnote{Maximal with
respect to not being the unit ideal.} if 
\begin{enumerate}[\textbf{M} 1]
\item  $1 \notin I$
\item Any larger ideal contains $1$ (i.e., is all of $R$).
\end{enumerate}
\end{definition} 

So a maximal ideal is a maximal element in the partially ordered set of proper
ideals (an ideal is \textbf{proper} if it does not contain 1).

\begin{exercise} 
Find the maximal ideals in $\mathbb{C}[t]$.
\end{exercise} 


\begin{proposition} 
A maximal ideal is prime.
\end{proposition} 
\begin{proof} 
First, a maximal ideal does not contain 1.

Let $I \subset R$ be a maximal ideal.
We need to show that if $xy \in I$,
then one of $x,y \in I$.  If $x \notin I$, then $(I,x) = I + (x)$ (the ideal
generated by $I$ and $x$) strictly contains $I$, so by maximality contains
$1$.  In particular, $1 \in I+(x)$, so we can write
\[ 1 = a + xb  \]
where $a \in I, b \in R$. Multiply both sides by $y$:
\[ y = ay  + bxy.  \]
Both terms on the right here are in $I$ ($a \in I$ and $xy \in I$), so we find
that $y \in I$.

\end{proof} 

Given a ring $R$, what can we say about the collection of ideals in $R$?
There
are two obvious ideals in $R$, namely $(0)$ and $ (1)$.  These are the same if and
only if $0=1$, i.e. $R$ is the zero ring.
So for any nonzero commutative ring, we have at least two distinct ideals.  

Next, we show that maximal ideals always \emph{do} exist, except in the case
of the zero ring. 
\begin{proposition} \label{anycontainedinmaximal}
Let $R$ be a commutative ring. Let $I \subset R$ be a proper ideal.  Then $I$
is contained in a maximal ideal.
\end{proposition} 

\begin{proof} 
This requires the axiom of choice in the form of Zorn's lemma.  Let
$P$ be the collection of all ideals $J \subset R$ such that $I
\subset J$ and $J \neq R$.  Then $P$ is a poset with respect to  inclusion.  $P$ is
nonempty because it contains $I$.  Note that given a (nonempty) linearly ordered
collection of ideals $J_{\alpha} \in P$, the union $\bigcup J_{\alpha} \subset
R$ is an ideal: this is easily seen in view of the linear ordering (if $x,y
\in \bigcup J_{\alpha}$, then both $x,y$ belong to some $J_{\gamma}$, so $x+y
\in J_{\gamma}$; multiplicative closure is even easier). The union is not all
of $R$ because it does not contain $1$.  

This implies that $P$ has a maximal element by Zorn's lemma.  This maximal element may
be called $\mathfrak{M}$; it's a proper element containing $I$. I claim that
$\mathfrak{M}$ is a maximal ideal, because if it were contained in a larger
 ideal, that would  be in $P$ (which can't happen by maximality) unless it were all of $R$.
\end{proof} 

\begin{corollary} 
Let $R $ be a nonzero commutative ring.  Then $R$ has a maximal ideal.
\end{corollary} 
\begin{proof} 
Apply the lemma to the zero ideal.  
\end{proof} 

\begin{corollary} 
Let $R$ be a nonzero commutative ring. Then $x \in R$ is invertible if and
only if it belongs to no maximal ideal $\mathfrak{m} \subset R$.
\end{corollary} 
\begin{proof} 
Indeed, $x$ is invertible if and only if $(x) = 1$. That is, if and only if
$(x)$ is not a proper ideal; now Proposition~\ref{anycontainedinmaximal}
finishes the argument.
\end{proof} 

\subsection{Fields and integral domains}

Recall:

\begin{definition} 
A commutative ring $R$ is called a  \textbf{field} if $1 \neq 0$ and for every $x \in R -
\left\{0\right\}$ there exists an \textbf{inverse} $x^{-1} \in R$ such that $xx^{-1} =
1$.


\end{definition}


This condition has an obvious interpretation in terms of ideals.
\begin{proposition} 
A commutative ring with $1 \neq 0$ is a field iff it has only the two ideals $(1),
(0)$.
\end{proposition} 

Alternatively, a ring is a field if and only if $(0)$ is a maximal ideal.

\begin{proof} 
Assume $R$ is a field.  Suppose $I \subset R$.  If $I \neq (0)$, then there is
a nonzero $x \in I$. Then there is an inverse $x^{-1}$. We have $x^{-1} x =1
\in I$, so $I = (1)$.
In a field, there is thus 	no room for ideals other than $(0)$ and $(1)$.

To prove the converse, assume every ideal of $R$ is $(0)$ or $(1)$. Then for
each $x \in R$, $(x) = (0)$ or $(1)$. If $x \neq 0$, the first can't happen, so
that means that the ideal generated by $x$ is the unit ideal. So $1$ is a
multiple of $x$, implying that $x$ has a multiplicative inverse.
\end{proof} 

So fields also have an uninteresting ideal structure.

\begin{corollary} 
If $R$ is a ring and $I \subset R$ is an ideal, then $I$ is maximal if and only
if $R/I$ is a field.
\end{corollary} 

\begin{proof}
The basic point here is that there is a bijection between the ideals of $R/I$
and ideals of $R$ containing $I$. 

Denote  by $\phi: R \to R/I$ the reduction map. There is a
construction mapping ideals of $R/I$ to ideals of $R$. This sends an ideal in
$R/I$ to
its inverse image.  This is easily seen to map to ideals of $R$ containing $I$.
The map from ideals of $R/I$ to ideals of $R$ containing $I$ is a bijection,
as one checks easily.

It follows that $R/I$ is a field precisely if
$R/I$ has precisely two ideals, i.e. precisely if there are precisely two
ideals in $R$ containing $I$. These ideals must be $(1)$ and $I$, so this
holds if and only if $I$ is maximal.
\end{proof} 

There is a similar characterization of prime ideals.

\begin{definition} 
A commutative ring $R$ is an \textbf{integral domain} if for all $ x,y \in R$,
$x \neq 0 $ and $y \neq 0$ imply $xy \neq 0$.
\end{definition} 

\begin{proposition}\label{primeifdomain} 
An ideal $I \subset R$ is prime iff $R/I$ is a domain.
\end{proposition} 

\begin{exercise} 
Prove Proposition~\ref{primeifdomain}.
\end{exercise} 

Any field is an integral domain. This is because in a field, nonzero elements
are invertible, and the product of two invertible elements is invertible. This
statement translates in ring theory to the statement that a maximal ideal is
prime.

\begin{exercise} 
Let $R$ be a domain. Consider the set of formal quotients $a/b, a, b \in R$
with $b \neq 0$. Define addition and multiplication using usual rules. Show
that the resulting object $K(R)$ is a ring, and in fact a \emph{field}. The
natural map $R \to K(R)$, $r \to r/1$, has a universal property. If $R
\hookrightarrow L$ is an injection of $R$ into a field $L$, then there is a
unique morphism $K(R) \to L$ of fields extending $R \to L$. This construction
will be generalized when we consider \emph{localization.}

Note that a non-injective map $R\to L$ will \emph{not} factor through the
quotient field!
\end{exercise} 


\begin{exercise}\label{Jacobson} 
Let $R$ be a commutative ring. Then the \textbf{Jacobson radical} of $R$ is
the intersection $\bigcap \mathfrak{m}$ of all maximal ideals $\mathfrak{m}
\subset R$. Prove that an element $x$ is in the Jacobson radical if and only
if $1 - yx$ is invertible for all $y \in R$.
\end{exercise} 

\subsection{Principal ideal domains}

\begin{definition} 
A ring $R$ is a \textbf{principal ideal domain} or \textbf{PID} if $R \neq 0$, $R$ is not a
field, $R$ is a domain, and every ideal of $R$ is principal.
\end{definition} 

These have the next simplest theory of ideals.
Each ideal is very simple---it's principal---though there might be a lot of ideals.

\begin{example} 
$\mathbb{Z}$ is a PID. The only nontrivial fact to check here is that:
\begin{proposition} 
Any nonzero ideal $I \subset \mathbb{Z}$ is principal.
\end{proposition} 
\begin{proof} 
If $I = (0)$, then this is obvious.  Else there is $n \in I -
\left\{0\right\}$; we can assume $n>0$.  Choose $n \in I$ as small as possible and
positive. Then  I claim that the ideal $I$ is generated by $(n)$. Indeed, we have $(n)
\subset I$ obviously. If $m \in I$ is another integer, then divide $m$ by $n$,
to find $m = nb + r$ for $r \in [0, n)$. We find that $r \in I$ and $0 \leq r <
n$, so $r=0$, and $m$ is divisible by $n$. And $I \subset (n)$. 

So $I = (n)$.
\end{proof} 
\end{example} 


\begin{exercise} 
A domain $R$ is \textbf{euclidean} if there is a map $v: R \to
\mathbb{Z}_{\geq 0}$
such that $v(xy) = v(x)v(y)$ and such that a type of division algorithm
holds: if $a,b \in R$ with $b \neq 0$, there are $q,r \in R$ such that
\[ a = qb + r  \]
and $v(r)< v(b)$. Prove that a euclidean domain is principal.
\end{exercise} 

\begin{exercise} 
Prove that $\mathbb{Z}[i]$ is principal. (Use the previous exercise.)
\end{exercise} 

\begin{exercise} \label{polyringisprincipal}
Prove that the polynomial ring $F[t]$ for $F$ a field is principal. 
\end{exercise} 

\section{Basic properties of modules}

\subsection{Free modules}

\subsection{Finitely generated modules}

\begin{definition} 
An $R$-module $M$ is \textbf{finitely generated} if there exists a surjection
$R^n \to M$ for some $n$. In other words, it has a finite number of elements
whose ``span'' contains $M$.
\end{definition} 


