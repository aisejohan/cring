\setcounter{chapter}{-1} \chapter{Categories}

N.B. This is a very rough, preliminary chapter. Currently it consists of notes
taken in an algebraic topology class. Serious editing is still needed.

\section{Introduction} \newcommand{\ob}{\mathrm{ob}} Categories are supposed
to be places where mathematical objects live.

\begin{definition} A \textbf{category} $\mathcal{C}$ consists of a collection
of \textbf{objects}, $\mathrm{ob}\mathcal{C}$, and for each pair of objects
$X, Y \in \mathrm{ob}\mathcal{C}$, a set of \textbf{morphisms} $\mathcal{C}(X,
Y)$.

For each object $X \in \ob\mathcal{C}$, there is an \textbf{identity morphism}
$1 \in \mathcal{C}(X, X)$. Next, there is a \textbf{composition law}
$\mathcal{C}(X, Y) \times \mathcal{C}(Y, Z) \to \mathcal{C}(X, Z), (g, f) \to
g \circ f$ for every triple $X, Y, Z$ of objects, which is unital and
associative.

We write $X \to Y$ to denote an element of $\mathcal{C}(X, Y)$.
\end{definition}

$\mathcal{C}$ is the storehouse for mathematical objects: groups, Lie
algebras, rings, etc.

\begin{remark} Some people use $\hom(X, Y)$ to denote the set of morphisms
between $X, Y$. \end{remark}

\begin{remark} When we write $X \in \mathcal{C}$, it means $X \in \ob
\mathcal{C}$. This is a convenient convention. \end{remark}

\begin{remark} The objects don't have to form a set; they can be large. But
the things $\mathcal{C}(X, Y)$ are sets. \end{remark}

Now, we do a bunch of examples.

\begin{example} \begin{enumerate} \item $\mathcal{C} = \mathbf{Sets}$; the
objects are sets, and the morphisms are maps of sets. \item $\mathcal{C} =
\mathbf{Grps}$; the objects are groups, and the morphisms are maps of groups
(i.e. homomorphisms). \item $\mathcal{C} = \mathbf{LieAlg}$; the objects are
Lie algebras, and the morphisms are maps of Lie algebras (i.e. homomorphisms).
\end{enumerate} \end{example}

There is \textbf{nothing} in the language of categories that lets you look
\emph{inside} an object. We think of vector spaces having elements, spaces
having points, etc. Categories treat these kinds of things as invisible. There
is nothing ``inside'' of $X \in \mathcal{C}$. The only way to understand $X$
is to understand the homs into and out of $X$.

We will elaborate on this in the future.

\begin{example} Let $G$ be a finite group. Then we an make a category $B_G$
where the objects just consist of one point $\ast$ and the maps $\ast \to
\ast$ are the elements of $G$. The identity is the identity of $G$ and
composition is multiplication in the group.

In this case, the category doesn't represent so much of a class of objects,
but instead we think of the composition law as the key thing. So a group is a
special kind of category. \end{example}

\begin{example} A monoid is precisely a category with one object. Recall that
a \textbf{monoid} has an associative and unital multiplication (not
necessarily inverses). \end{example}

Figures... ADD THEM

\begin{remark} A lot of people said that they've seen this before in
commutative algebra, but not necessarily thoroughly. So we will keep talking
about this. \end{remark}

\subsection{Functors} Let $\mathcal{C}, \mathcal{D}$ be categories.

\begin{definition} A \textbf{functor} $F: \mathcal{C} \to \mathcal{D}$
consists of a function $F: \ob \mathcal{C} \to \ob \mathcal{D}$ and, for each
pair $X, Y \in \mathcal{C}$, a map $F: \mathcal{C}(X, Y) \to \mathcal{D}(FX,
FY)$, which preserves the identity maps and composition. \end{definition}

\begin{example} There is a functor from $\mathbf{Sets} \to
\mathbf{AbelianGrp}$ sending a set $S$ to a free abelian group on the set.
\end{example}

\begin{example} There is a functor from $\mathbf{TopSpaces} \to
\mathbf{GradedAbGrp}$ (categories of topological spaces and graded abelian
groups) sending a space $X$ to its homology groups $H_*(X)$. We know that
given a map of spaces, we get a map of graded abelian groups. \end{example}

\begin{example} What is a functor $B_G \stackrel{}{\to} \mathbf{Sets}$? Here
$B_G$ is the category alluded to above.

The unique object $\ast$ goes to some set $X$. For each element $g \in G$, we
get a map $g: \ast \to \ast$ and thus a map $X \to X$. This is supposed to
preserve the composition law (which in $G$ is just multiplication), as well as
identities.

In particular, we get maps $i_G: X \to X$ corresponding to each $g \in G$,
such that the following diagram commutes: \[ \xymatrix{ X \ar[r]^{i_{g_1}}
\ar[rd]^{i_{g_1g_2}} & X \ar[d]^{i_{g_2}} \\ & X }\] So a functor $B_G \to
\mathbf{Sets}$ is just a left $G$-action on a set $X$. \end{example}

``I never liked the idea of left and right action. What about aliens on
another planet that didn't have left and right hands?''

Sometimes these are called \textbf{covariant functors}. Indeed:

\begin{definition} A \textbf{contravariant functor} from $\mathcal{C}
\stackrel{F}{\to}\mathcal{D}$ is similar data except that now a map $X \to Y$
now goes to a map $FY \to FX$. Composites are required to be preserved, albeit
in the other direction. \end{definition}

``The reason we have shiny objects in nature is so that we will think of
contravariant functors. If you look in a mirror, it's like applying a
contravariant functor to yourself. This is kind of a myopic view of mankind.''

As you might guess:

\begin{example} A \textbf{contravariant} functor from $B_G$ to $\mathbf{Sets}$
corresponds to a set with a \emph{right} $G$-action. \end{example}

We will, in a week or so, define a contravariant version of homology when we
start studying cohomology.

\begin{example} On the category $\mathbf{Vect}$ of vector spaces, we have the
contravariant functor \[ V \to V^{\ast}. \] \end{example}

\begin{example} If we map $B_G \to B_G$ sending $\ast \to \ast$ and $g \to
g^{-1}$, we get a contravariant functor. \end{example}

There is room, nevertheless, for something else. You could have something that
sent an object to a map. This is, I think, the reason for Maclane and
Eilenberg to describe the next property.

\subsection{Natural transformations}

The original paper of Eilenberg and Maclane was called ``On a general theory
of natural transformations.'' Maclane was a great guy, incidentally, besides
inventing homological algebra; there was a picture in his office of someone
holding a ray gun like a sci-fi movie yelling Tor, Tor, Tor.

Suppose $F, G: \mathcal{C} \to \mathcal{D}$ are functors.

\begin{definition} A \textbf{natural transformation} $T: F \to G$ consists of
the following data. For each $X \in C$, there is a morphism $TX: FX \to GX$
satisfying the following condition. Whenever $f: X \to Y$ is a morphism, the
following diagram must commute: \[ \xymatrix{ FX \ar[d]^{TX }\ar[r] & FY
\ar[d]^{TY} \\ GX \ar[r] & GY }.\] \end{definition}

When we say that things are ``natural'' in the future, we will mean that the
transformation between functors is natural in this sense.

\begin{example} The \textbf{connecting homomorphism} $H_n(X, A) \to
H_{n-1}(A)$ is natural. This is going to be a little rocky, but let's say what
this means.

If we have pairs $(X, A) \to (Y, B)$, then the following diagram commutes \[
\xymatrix{ H_n(X, A) \ar[d] \ar[r] & H_{n-1}(A) \ar[d] \\ H_n(Y, B) \ar[r] &
H_{n-1}(B) }.\] This identity is very important in the axiomatic
characterization of homology, due to Eilenberg-Steenrod.

This is a little funny, and one has to think about which category we're
talking about. We can use the category of \textbf{pairs of topological
spaces}. So the objects here are pairs $(X, A)$ and morphisms are morphisms of
pairs. \end{example}

\begin{remark} ``I'm here to put the 'funk' in functor.'' (Dick Gross came in
to say hello.) \end{remark}

Some people don't like this. They don't like to use the language of
categories. If you really try to go in and examine things, it can be hard to
figure out what things really mean. However, we will use it to state theorems
conveniently.

\begin{exercise} Work this out for yourselves. Suppose you have two functors
$B_G \to \mathbf{Sets}$, i.e. $G$-sets. What's a natural transformation
between them? \end{exercise}

Now I want to prove a theorem. \begin{theorem} If $f: X \to Y$ is a map in
$\mathcal{C}$, and $F: \mathcal{C} \to \mathcal{D}$ is a functor, then $F(f):
FX \to FY$ is an isomorphism. \end{theorem} This is going to have a really
stupid proof, but there is an important point lurking here. \begin{example}
Let $\mathcal{C}$ be the homotopy category of topological spaces
$\mathbf{hoT}$. The objects are topological spaces and the morphisms between
$X, Y$ are the continuous maps $X \to Y$ modulo the relation of being
homotopic. Homology is actually a functor from $\mathbf{hoT}$ to the category
of graded abelian groups. \end{example}

Hold on. Wait a second. Do we even know what an isomorphism in a category even
is? No, we don't.

\begin{definition} An \textbf{isomorphism} between objects $X, Y$ in a
category $\mathcal{C}$ is a map $f: X \to Y$ such that there exists $g: Y \to
X$ with \[ g \circ f = 1_X, \quad f \circ g = 1_Y. \] \end{definition}

This is more correct than the idea of being one-to-one and onto. A bijection
of topological spaces is not necessarily a homeomorphism.

\begin{proof} If we have maps $f: X \to Y$ and $g : Y \to X$ such that the
composites both ways are identities, then we can apply the functor $F$ to the
whole dog and pony show, and we find that since \[ f \circ g = 1_Y, \quad g
\circ f = 1_X, \] that \[ F(f) \circ F(g) = 1_{F(Y)}, \quad F(g) \circ F(f) =
1_{F(X)}. \] We have used the fact that functors preserve composition and
identities. This implies that $F(f)$ is an isomorphism. \end{proof}

Categories have a way of making things so general that they're trivial. Hence,
it is called general abstract nonsense. The things that become meaningful in
category theory are \textbf{not} the proofs. They are the
\textbf{definitions}. What we just did is very much in the spirit I was
describing of categories. The notion of isomorphism was defined in terms of
properties of maps, not in terms of things intrinsic (like injections and
surjections).

What's important here is not the theorem, but the \emph{definition of an
isomorphism.}

A lot of it for this course, though, is just getting used to the language and
the definitions.

\section{Various universal constructions}

Last time, we introduced the idea of a category, and showed that a functor
takes isomorphisms to isomorphisms. This was an amazing result with a trivial
proof. Today, we will characterize objects in terms of maps.
\subsection{Initial and terminal objects}

\begin{definition} Let $\mathcal{C}$ be a category. An \textbf{initial object}
in a category is an object $X \in \mathcal{C}$ with the property that
$\mathcal{C}(X, Y)$ has one element for all $Y \in \mathcal{C}$.

So there is a unique map out of $X$ into each $Y \in \mathcal{C}$.
\end{definition}

\begin{example} If $\mathcal{C}$ is $\mathbf{Sets}$, then the empty set
$\emptyset$ is an initial object. The empty set is the set for indecisive
people. To map out of the indecisive set, you never have to decide where
anything goes---it just goes. There is a unique map from the empty set into
any other set. \end{example}

It seems too abstract to be useful. But it is.

There is a dual notion, called a \textbf{terminal object}, where every object
can map into it in precisely one way. \begin{definition} A \textbf{terminal
object} in a category $\mathcal{C}$ is an object $Y \in \mathcal{C}$ such that
$\mathcal{C}(X, Y) = \ast$ for each $X \in \mathcal{C}$. \end{definition}

\begin{example} The one point set is a terminal object in $\mathbf{Sets}$.
\end{example}

The important thing about the next ``theorems'' is the conceptual framework.
\begin{theorem} Any two initial (resp. terminal) objects in $\mathcal{C}$ are
isomorphic by a unique isomorphism. \end{theorem} \begin{proof} The proof is
really easy. We do it for terminal objects. Say $Y, Y'$ are terminal objects.
Then $\mathcal{C}(Y, Y')$ and $\mathcal{C}(Y', Y)$ are one point sets. So
there are unique maps $Y \to Y', Y' \to Y$, whose composites must be the
identities: we know that $\mathcal{C}(Y, Y) , \mathcal{C}(Y', Y')$ are
one-point sets. This means that the maps $Y \to Y', Y' \to Y$ are
isomorphisms. \end{proof}

There is a philosophical point to be made here. We have characterized an
object uniquely in terms of mapping properties. We have characterized it
\emph{uniquely up to unique isomorphism,} which is really the best you can do
in mathematics. Two sets aren't generally the ``same,'' but they may be
isomorphism up to unique isomorphism. Like the sets of your father and Darth
Vader: they're different (unless you're Luke), but the sets are isomorphic up
to unique isomorphism.

Now we're going to talk about a bunch of other examples, which can all be
phrased via initial or terminal objects in some weird category. This,
therefore, is the proof for \emph{everything} we will do today.

Say we have a diagram \[ \xymatrix{ A \ar[d] \ar[r] & B \ar[d] \\ C \ar[r] &
X}. \] We can say what it means for this to be a \textbf{push-out}.

\begin{definition} A square like this, \[ \xymatrix{ A \ar[d] \ar[r] & B
\ar[d] \\ C \ar[r] & X}. \] is a \textbf{pushout square} (and $X$ is called
the \textbf{push-out}) if, given a diagram \[ \xymatrix{ A \ar[r] \ar[d] & B
\ar[dd] \\ C \ar[rd] & \\ & Y }\] there is a unique map $X \to Y$ making the
diagram \[ \xymatrix{ A \ar[r] \ar[d] & B \ar[d] \\ C \ar[rd] \ar[r] & X
\ar[d] \\ & Y }.\] \end{definition}

\begin{example} The following is a pushout square in the category of abelian
groups: \[ \xymatrix{ \mathbb{Z}/2 \ar[r] \ar[d] & \mathbb{Z}/4 \ar[d] \\
\mathbb{Z}/6 \ar[r] & \mathbb{Z}/12 }.\] In the category of groups, the
push-out is actually $\mathrm{SL}_2(\mathbb{Z})$---this is a cool theorem. The
point is that being a push-out is actually dependent on the category.
\end{example}

\begin{proposition} If the push-out of \[ \xymatrix{ A \ar[d] \ar[r] & B \\ C
}\] exists, it is unique up to unique isomorphism. \end{proposition}
\begin{proof} We can prove this in two ways. One is that suppose I had two
pushout squares \[ \xymatrix{ A \ar[d] \ar[r] & B \ar[d] \ar[rdd] \\ C \ar[r]
\ar[rrd] & X \\ & & X'}. \] Then there are unique maps $X \to X', X' \to X$
from the universal property, which have to be isomorphisms.

Alternatively, we can phrase push-outs in terms of initial objects. We could
consider the category of all cartesian diagrams as above with $A,B,C$ and
mapping into something else; then the initial object in this category is the
push-out. \end{proof}

Now we abstract on this idea further.

\subsection{Colimits}

We now want to generalize the push-out. Instead of a shape with $A,B,C$, we do
something more general.

Start with a \textbf{small} category $I$: this is not meant in a pejorative
sense, but that the objects of $I$ form a set. What you're supposed to picture
is that $I$ is something like the category \[ \xymatrix{ \ast \ar[d] \ar[r] &
\ast \\ \ast } \] or the category \[ \ast \rightrightarrows \ast. \] We will
formulate the notion of a \textbf{colimit} which will specialize to the
push-out when $I$ is the first case. $I$ is to be called the \textbf{indexing
category}.

So we will look at functors \[ F: I \to \mathcal{C}, \] which in the case of
the three-element category, will just correspond to diagrams \[ \xymatrix{A
\ar[d] \ar[r] & B \\ C}. \]

We will call a \textbf{cone} on $F$ (this is an ambiguous term) an object $X
\in \mathcal{C}$ equip'd\footnote{Imagine I wrote that with an English
accent.} with maps $F_i \to X, \forall i \in I$ such that for all maps $i \to
i' \in I$, the diagram below commutes: \[ \xymatrix{ F_i \ar[d] \ar[r] & X \\
F_{i'} \ar[ru] }.\]

An example would be a cone on the three-element category above: then this is
just a commutative diagram \[ \xymatrix{ A \ar[r]\ar[d] & B \ar[d] \\ C \ar[r]
& D }.\]

\newcommand{\colim}{\mathrm{colim}}

\begin{definition} The \textbf{colimit} of the diagram $F: I \to \mathcal{C}$,
written as $\colim F$ or $\colim_I F $ or $\varinjlim_I F$, if it exists, is a
cone $F \to X$ with the property that if $F \to Y$ is any other cone, then
there is a unique map $X \to Y$ making the diagram \[ \xymatrix{ F \ar[rd]
\ar[r] & X \ar[d] \\ & Y }\] commute. (This means that the corresponding
diagram with $F_i$ replacing $F$ commutes for each $i \in I$.)
\end{definition}

We could think of some weird category where cones are objects and the colimit
is initial. In any case, we see:

\begin{proposition} $\colim F$, if it exists, is unique up to unique
isomorphism. \end{proposition}

Let us go through some examples. We already looked at push-outs.

\begin{example} Consider the category $I$ described by \[ \ast, \ast, \ast,
\ast. \] A functor $F: I \to \mathbf{Sets}$ is just a list of four sets $A, B,
C, D$. The colimit is just the disjoint union $A \sqcup B \sqcup C \sqcup D$.
This is the universal property of the disjoint union. To hom out of the
disjoint union is the same thing as homming out of each piece. \end{example}

\begin{example} Suppose we had the same category $I$ but we went into abelian
groups. Then $F$ corresponds, again, to a list of four abelian groups. The
colimit is the direct sum. Again, the direct sum is characterized by the same
universal property. \end{example}

\begin{example} Suppose we had the same $I$ ($\ast, \ast, \ast, \ast$) but the
category of groups was $\mathcal{C}$. Then the colimit is the free product of
the four groups. \end{example}

\begin{example} Suppose we had the same $I$ and the category $\mathcal{C}$ was
of commutative rings with unit. Then the colimit is the tensor product.
\end{example}

So the idea unifies a whole bunch of constructions.

Now let us take a different example.

\begin{example} Take \[ I = \ast \rightrightarrows \ast. \] So a functor $I
\to \mathbf{Sets}$ is a diagram \[ A \rightrightarrows B. \] Call the two maps
$f,g: A \to B$. To get the colimit, we take $B$ and mod out by the equivalence
relation generated by $f(a) \sim g(a)$. To hom out of this is the same thing
as homming out of $B$ such that the pullbacks to $A$ are the same.

This is the relation \textbf{generated} as above, not just as above. It can
get tricky. \end{example}

\begin{definition} When $I$ is just a bunch of points $\ast, \ast, \ast,
\dots$ with no nonidentity morphisms, then the colimit over $I$ is called the
\textbf{coproduct}. \end{definition}

We use the coproduct to mean things like direct sums, disjoint unions, and
tensor products.

\begin{definition} When $I$ is $\ast \rightrightarrows \ast$, the colimit is
called the \textbf{coequalizer}. \end{definition}

\begin{theorem} If $\mathcal{C}$ has all coproducts and coequalizers, then it
has all colimits. \end{theorem}

\begin{proof} Exercise. It's not too hard, but it is---I don't know, I'll talk
about it on Monday. It's worth racking your brain over. \end{proof}

One of the reasons I talked about colimits is that we can talk about it in
class and use the language. Also, there are a lot of examples we haven't done
in class as we haven't studied filtered colimits.

\subsection{Filtered colimits}

These are really useful, especially in algebraic topology. These are colimits
over special $I$.

\begin{definition} An indexing category is \textbf{filtered} if the following
hold: \begin{enumerate} \item Given $i_0, i_1 \in I$, there is a third object
$i \in I$ such that both $i_0, i_1$ map into $i$. \item Given any two maps
$i_0 \rightrightarrows i_1$, there exists $i$ and $i_1 \to i$ such that the
two maps $i_0 \rightrightarrows i$ are equal. Any two ways of pushing an
object into another can be made into the same eventually. \end{enumerate}
\end{definition}

\begin{example} If $I$ is the category \[ \ast \to \ast \to \ast \to \dots, \]
i.e. the category generated by the poset $\mathbb{Z}_{\geq 0}$, then that is
filtered. \end{example}

\begin{example} If $G$ is a torsion-free abelian group, the category $I$ of
finitely generated subgroups of $G$ and inclusion maps is filtered. We don't
actually need the lack of torsion. \end{example}

\begin{definition} Colimts over a filtered category are called
\textbf{filtered colimits}. \end{definition}

\begin{example} Any torsion-free abelian group is the filtered colimit of its
finitely generated subgroups, which are free abelian groups. \end{example}
This gives a simple approach for showing that a torsion-free abelian group is
flat.

\begin{proposition} If $I$ is filtered\footnote{Some people say filtering.}
and $\mathcal{C} = \mathbf{Sets}, \mathbf{Abgrp}, \mathbf{Grps}$, etc., and
$F: I \to \mathcal{C}$ is a functor, then $\colim_I F$ exists and is given by
the disjoint union of $F_i, i \in I$ modulo the relation $x \in F_i$ is
equivalent to $x' \in F_{i'}$ if $x$ maps to $x'$ under $F_i \to F_{i'}$. This
is already an equivalence relation. \end{proposition}

The fact that the relation given above is transitive uses the filtering of the
indexing set. Otherwise, we would need to use the relation generated by it.

\begin{example} Take $\mathbb{Q}$. This is the filtered colimit of the free
submodules $\mathbb{Z}(1/n)$.

Alternatively, choose a sequence of numbers $m_1 , m_2, \dots, $ such that for
all $p, n$, we have $p^n \mid m_i$ for $i \gg 0$. Then we have a sequence of
maps \[ \mathbb{Z} \stackrel{m_1}{\to} \mathbb{Z}
\stackrel{m_2}{\to}\mathbb{Z} \to \dots. \] The colimit of this is
$\mathbb{Q}$. There is a quick way of seeing this, which is left to the
reader. \end{example} \lecture{10/18}

\subsection{Filtered colimits} Last time, we talked about something called
\emph{filtered colimits}. In this, we had a special property of the indexing
category $I$. It had the property that given any two $i_0, i_1 \in I$, there
was a third one into which they mapped; moreover, given any two maps $i_0
\rightrightarrows i_1$, there was a third one which coequalized them. Filtered
colimits were defined as colimits over a filtered category.

When we have a functor $F: I \to \mathbf{Sets}, \mathbf{Grps},
\mathbf{Modules}$ taking values in a ``nice'' category (e.g. the category of
sets, modules, etc.), you can construct the colimit by taking the union of the
$F_i, i \in I$ and quotienting by the equivalence relation $x \in F_i \sim x'
\in F_{i'}$ if $f: i \to i'$ sends $x$ into $x'$. This is already an
equivalence relation, as one can check.

Another way of saying this is that we have the disjoint union of the $F_i$
modulo the relation that $a \in F_i$ and $b \in F_{i'}$ are equivalent if and
only if there is a later $i''$ with maps $i \to i'', i' \to i''$ such that
$a,b$ both map to the same thing in $F_{i''}$.

Suppose $F: I \to \mathbf{Ch}$ is a functor from a filtered category $I$ to
the category of chain complexes. For instance, $I$ could be the category $\ast
\to \ast \to \ast \to \dots$, leading to a sequence of chain complexes
$C_*^{(0)}\to C_{*}^{(1)} \to \dots$. This is the standard example you're
supposed to keep in mind.

Then: \begin{proposition} The homology of the colimit $\varinjlim_I F$ is the
colimit of the homologies $H(F_i)_{i \in F}$. \end{proposition}

\begin{proof} This is easy to prove. The deep idea is the formulation, not the
proof.

We will first prove that the natural map \[ \colim_I (H_*F) \to H_* \colim_I F
\] is onto. Suppose we have something in $H_*(\colim F)$. Then this element
$x$ is represented by a $n$-cycle $z$ in $(\colim_I F)_n$ for some $n$. The
colimit $(\colim_I F)_n$ is just $\sqcup (F_i)_n$ modulo the equivalence
relation. So $z$ is represented by some $z' \in (F_i)_n$. We don't, a priori,
know that $z'$ is a cycle, i.e. that $dz' = 0$. If this were the case, then we
would have a class in $\colim_I (H_* F)$ mapping onto $x$.

However, $dz'$ does go to zero in the colimit $\colim_I F$ as $z'$ is a cycle
in this colimit. Because it is filtered, we know that there is a map $f:i \to
i'$ such that $dz'$ goes to zero in $i'$. In $F_{i'}$, $z'$ becomes a cycle.
So the homology class of $x$ is in the image of $H_n(F_{i'})$, which maps into
the colimit $\colim_I H_n(F_i)$, which in turn maps into the homology of the
colimit. We have thus seen that \[ \colim_I H_n(F_i) \to H_n(\colim_I F_i) \]
is surjective.

Now let us prove that it is one-to-one. Suppose $x \in \colim_I H_n(F)$ goes
to zero in the homology of the colimit $\colim_I F$. So $x$ is represented by
some cycle $z \in Z_n(F_i)$. In the colimit $\colim_I F$, $x$ is a boundary $x
= dy$. There is thus $\overline{y} \in F_{i'}$ representing $y$. By pushing
forward into some mutually larger $i''$, we might as well suppose that $x = d
\overline{y}$ in $F_i$ itself. This means that $x$ was zero in $H_n(F_i)$
itself. \end{proof}

I hope that made sense. If it didn't, it's one of those things that's more
complicated when you say it out loud than when you think it through for
yourself. I can't remember whether this was in Hatcher or not. But then you'll
just get what I said here in a less entertaining way. I find these kinds of
things hard to digest when someone is standing there telling it to me.

But anyway, this is one of the main uses of filtered colimits---or directed
colimits, as some people say. \lecture{[Section] 10/18}

The key observation made in class is that any diagram of the form \[
\xymatrix{ X \ar[r]\ar[rd] & Y \ar[d] \\ & Z } \] can be interpreted as a
\emph{functor} from a suitable \emph{diagram category}. A \emph{cone} on a
functor $F: I \to \mathcal{C}$ can be defined as a collection of maps $Fi \to
Z$. There is a category of cones one can define, and in this category, the
initial object is the \emph{colimit}.

Colimits don't have to exist.

\begin{remark} Given a functor $F: I \to \mathcal{C}$ where $\mathcal{C}$ has
a terminal object, you can always consider the trivial cone over the functor
mapping each object $Fi, i \in I$ into the terminal object. \end{remark}

For limits, one reverses the arrows and defines a \emph{co-cone} over a
functor and considers the terminal object in the category of co-cones.

As we saw in class, given a functor $G$, we can always define a natural map \[
G(\colim_I F) \to \colim_I GF. \] Here is an example. Given the category $J:
\ast \to \ast$, a functor $J \to \mathbf{Top}$ is just a morphism $X \to Y$. A
colimit of this $X \to Y$ is just a space (the cone) $C_F$ with maps \[ X \to
C_F, Y \to C_F. \] If $G$ is a functor from $\mathbf{Top}$ to some other
category, we have a commutative diagram \[ \xymatrix{ G(X) \ar[rd] \ar[rr] & &
G(Y) \ar[ld] \\ & G(C_F) \ar[ru] }\]

From this, we get a map from this cone into the universal cone $C_{GF}$ over
$CG(X) \to CG(Y)$. In particular, we get a map \[ G(C_F) \to C_{GF}. \]
