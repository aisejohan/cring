\chapter{Fields and Extensions}
Recall that a field is an integral domain for which all non-zero elements are
invertible. Equivalently, the only two ideals of a field are $(0)$ and $(1)$
since any nonzero element is a unit.


\begin{example}
One of the most familiar examples of a field is $\mathbb{Q}$.
\end{example}

\begin{example}
Another common field class of fields is the class of finite fields, for example
$\mathbb{Z}/(2)$.
\end{example}

\begin{exercise} In fact, for any maximal ideal $\mathfrak{m}\subseteq R$, the
quotient $R/\mathfrak{m}$ is a field. 
\end{exercise}

\begin{example} If $F$ is a field, then any ring homomorphism $f:F\rightarrow
R$is either injective or the zero map; this is because $ker(f)$ is an ideal in
$F$but there are only two ideals, $(0)$ and $(1)$.
\end{example}

\begin{definition} If $F$ is a field contained in a field $G$, then $G$ is said
to be a field extension of $F$.
\end{definition}

There are plenty of field extensions coming from both number theory and
geometry.

\begin{example} For example let $X$ be a Riemann surface. Let $k(X)$ denote the
set of meromorphic functions on $X$; clearly $k(X)$ is a ring under
multiplication and addition of functions. It turns out that in fact $k(X)$ is a
field, this is because if $f(z)$ is meromorphic, so is $1/f(z)$. For example,
let $\mathbb{C}_{\infty}$ be the Riemann sphere; then we know from complex
analysis that the ring of meromorphic functions $k(\mathbb{C}_{\infty}$ is the
field of rational functions $\mathbb{C}(z)$. More is true. In fact, there is
always a holomorphic function $X\rightarrow \mathbb{C}_{\infty}$ obtained by
taking a meromorphic function $f\in k(X)$ and considering it as a holomorphic
function $f:X\rightarrow\mathbb{C}_{\infty}$. In this case, $f$ induces a map
$k(\mathbb{C}_{\infty})\rightarrow k(X)$ by composing with $f$. In particular,
this map is an injection meaning that $k(X)$ is a field extension of
$\mathbb{C}[z]$.
\end{example}

\begin{example} The previous example is a more specific case of a general
phenomenon in geometry; let $C_1$ and $C_2$ be smooth projective curves (over a
field $k$). Then a morphism $f:C_1\rightarrow C_2$ makes $k(C_1)$ into a field
extension $k(C_2)$ 
\end{example}

\begin{definition}An element $\alpha\in F$ is said to be algebraic over $E$ if
$\alpha$ is the root of some polynomial with coefficients in $E$. If all
elements of $F$ are algebraic then $F$ is said to be an algebraic extension. If
$E\subseteq F$ is a field extension then $F$ is also a vector space over $E$
(the scalar action is just multiplication in $F$). The dimension of $F$
considered as an $E$-vector space is called the degree of the extension and is
denoted $[F:E]$. If $[F:E]<\infty$ then $F$ is said to be a finite extension
(note that all finite extensions are algebraic). Field extensions are sometimes
denoted $F/E$ for $E\subseteq F$.
\end{definition}

Algebraic number theory is in fact the study of algebraic extensions, in
particular, a field extensions $K$ of $\mathbb{Q}$ is said to be a number field
if it is a finite extension of $\mathbb{Q}$.

Since field extensions $F/E$ are always vector spaces, there is a basis $B$ for
$F$ such that any element of $F$ is the linear combination of elements of $B$
with coefficients in $E$. Returning to the previous example, if $K$ is the
smallest field that contains $\mathbb{Q}$ and an algebraic number $\alpha$, then
$[K:\mathbb{Q}]$ is the degree of the minimal polynomial $f(x)$ of $\alpha$.

\subsection{Transcendental Extensions}
There is a distinguished type of transcendental extension: those that are
``purely transcendental.'' 

