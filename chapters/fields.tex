\chapter{Fields and Extensions}
Recall that a field is an integral domain for which all non-zero elements are
invertible. Equivalently, the only two ideals of a field are $(0)$ and $(1)$
since any nonzero element is a unit.

\section{General remarks}

\begin{example}
One of the most familiar examples of a field is $\mathbb{Q}$.
\end{example}

\begin{example}
Another common field class of fields is the class of finite fields, for example
$\mathbb{Z}/(2)$.
\end{example}

\begin{exercise} In fact, for any maximal ideal $\mathfrak{m}\subseteq R$, the
quotient $R/\mathfrak{m}$ is a field. 
\end{exercise}

\begin{example} If $F$ is a field and $R$ is any ring, then any ring homomorphism $f:F\rightarrow
R$ is either injective or the zero map (in which case $R=0$); this is because $ker(f)$ is an ideal in
$F$ but there are only two ideals, $(0)$ and $(1)$. If $f$ is identically
zero, then $1=0$ in $R$ so $R=0$ too.
\end{example}

\begin{definition} If $F$ is a field contained in a field $G$, then $G$ is said
to be a field extension of $F$.
\end{definition}

There are plenty of field extensions coming from both number theory and
geometry.

\begin{example} For example let $X$ be a Riemann surface. Let $k(X)$ denote the
set of meromorphic functions on $X$; clearly $k(X)$ is a ring under
multiplication and addition of functions. It turns out that in fact $k(X)$ is a
field, this is because if $f(z)$ is meromorphic, so is $1/f(z)$. For example,
let $\mathbb{C}_{\infty}$ be the Riemann sphere; then we know from complex
analysis that the ring of meromorphic functions $k(\mathbb{C}_{\infty}$ is the
field of rational functions $\mathbb{C}(z)$. More is true. In fact, there is
always a holomorphic function $X\rightarrow \mathbb{C}_{\infty}$ obtained by
taking a meromorphic function $f\in k(X)$ and considering it as a holomorphic
function $f:X\rightarrow\mathbb{C}_{\infty}$. In this case, $f$ induces a map
$k(\mathbb{C}_{\infty})\rightarrow k(X)$ by composing with $f$. In particular,
this map is an injection meaning that $k(X)$ is a field extension of
$\mathbb{C}[z]$.
\end{example}

\begin{example} The previous example is a more specific case of a general
phenomenon in geometry; let $C_1$ and $C_2$ be smooth projective curves (over a
field $k$). Then a morphism $f:C_1\rightarrow C_2$ makes $k(C_1)$ into a field
extension $k(C_2)$ 
\end{example}

\begin{definition}An element $\alpha\in F$ is said to be algebraic over $E$ if
$\alpha$ is the root of some polynomial with coefficients in $E$. If all
elements of $F$ are algebraic then $F$ is said to be an algebraic extension. If
$E\subseteq F$ is a field extension then $F$ is also a vector space over $E$
(the scalar action is just multiplication in $F$). The dimension of $F$
considered as an $E$-vector space is called the degree of the extension and is
denoted $[F:E]$. If $[F:E]<\infty$ then $F$ is said to be a finite extension
(note that all finite extensions are algebraic). Field extensions are sometimes
denoted $F/E$ for $E\subseteq F$.
\end{definition}

Algebraic number theory is in fact the study of algebraic extensions, in
particular, a field extensions $K$ of $\mathbb{Q}$ is said to be a number field
if it is a finite extension of $\mathbb{Q}$.

Since field extensions $F/E$ are always vector spaces, there is a basis $B$ for
$F$ such that any element of $F$ is the linear combination of elements of $B$
with coefficients in $E$. Returning to the previous example, if $K$ is the
smallest field that contains $\mathbb{Q}$ and an algebraic number $\alpha$,
then$[K:\mathbb{Q}]$ is the degree of the minimal polynomial $f(x)$ of $\alpha$.


\subsection{The characteristic of a field}
\label{more-fields}

\noindent
Let $F$ be a field. As $\mathbf{Z}$ is the intial object of the category of
rings, there is a ring map $f : \mathbf{Z} \to F$, see
Exercise~\ref{integersinitial}.
The image of this ring map is an integral domain (as a subring of a field)
hence the kernel of $f$ is a prime ideal in $\mathbf{Z}$, see
\cref{primeifdomain}.
Hence the kernel of $f$ is either $(0)$ or $(p)$ for some prime number $p$, see
Example~\ref{integerprimes}.

In the first case we see that $f$ is injective, and in this case
we think of $\mathbf{Z}$ as a subring of $F$. Moreover, since every
nonzero element of $F$ is invertible we see that it makes sense to
talk about $p/q \in F$ for $p, q \in \mathbf{Z}$ with $q \not = 0$.
Hence in this case we may and we do think of $\mathbf{Q}$ as a subring of $F$.
You can easily see that this is the smallest subfield of $F$ in this case.

In the second case, i.e., when $\text{Ker}(f) = (p)$ we see that
$\mathbf{Z}/(p) = \mathbf{Z}/p\mathbf{Z}$ is a subring of $F$. Since
$\mathbf{Z}$ is a principal ideal domain the ideal $(p)$ is maximal
and $\mathbf{Z}/p\mathbf{Z}$ is a field too. This field is always denoted
$\mathbf{F}_p$ and as before we think of $\mathbf{F}_p \subset F$
as a subfield. Clearly it is the smallest subfield of $F$.

Arguing in this way we see that every field contains a smallest subfield
which is either $\mathbf{Q}$ or finite equal to $\mathbf{F}_p$ for some
prime number $p$.

\begin{definition}
The \textbf{characteristic} of a field $F$ is $0$ if
$\mathbf{Z} \subset F$, or is a prime $p$ if $p = 0$ in $F$.
The \textbf{prime subfield of $F$} is the smallest subfield of $F$
which is either $\mathbf{Q} \subset F$ if the characteristic is zero, or
$\mathbf{F}_p \subset F$ if the characteristic is $p > 0$.
\end{definition}

\section{Field extensions}



\subsection{Transcendental Extensions}
There is a distinguished type of transcendental extension: those that are
``purely transcendental.'' 
\begin{definition} A field extension $E'/E$ is purely transcendental if it is
obtained by adjoining a set $B$ of algebraically independent elements. A set of
elements is algebraically independent over $E$ if there is no nonzero polynomial$P$
with coefficients in $E$ such
that $P(b_1,b_2,\cdots b_n)=0$ for any finite subset of elements $b_1, \dots,
b_n \in B$.
\end{definition}

\begin{example} The field $\mathbb{Q}(\pi)$ is purely transcendental; in
particular, $\mathbb{Q}(\pi)\cong\mathbb{Q}(x)$ with the isomorphism fixing
$\mathbb{Q}$. \end{example}
Similar to the degree of an algebraic extension, there is a way of keeping
trackof the number of algebraically independent generators that are required to
generate a purely transcendental extension.
\begin{definition} Let $E'/E$ be a purely transcendental extension generated by
some set of algebraically independent elements $B$. Then the transcendence
degree $trdeg(E'/E)=\#(B)$ and $B$ is called a transcendence basis for $E'/E$
(we will see later that $trdeg(E'/E)$ is independent of choice of basis).
\end{definition}
In general, let $F/E$ be a field extension, we can always construct an
intermediate extension $F/E'/E$ such that $F/E'$ is algebraic and $E'/E$ is
purely transcendental. Then if $B$ is a transcendence basis for $E'$, it is
alsocalled a transcendence basis for $F$. Similarly, $trdeg(F/E)$ is defined to
be
$trdeg(E'/E)$.
\begin{theorem} Let $F/E$ be a field extension, a transcendence basis exists.
\end{theorem}
\begin{proof} Let $A$ be an algebraically independent subset of $F$. Now pick a
subset $G\subseteq F$ that generates $F/E$, we can find a transcendence basis
$B$ such that $A\subseteq B\subseteq G$. Define a collection of algebraically
independent sets $\mathcal{B}$ whose members are subsets of $G$ that contain
$A$. The set can be partially ordered inclusion and contains at least one
element, $A$. The union of elements of $\mathcal{B}$ is algebraically
independent since any algebraic dependence relation would have occured in one
ofthe elements of $\mathcal{B}$ since the polynomial is only allowed to be over
finitely many variables. The union also satisfies $A\subseteq
\bigcup\mathcal{B}\subseteq G$ so by Zorn's lemma, there is a maximal element
$B\in\mathcal{B}$. Now we claim $F$ is algebraic over $E(B)$. This is because
ifit wasn't then there would be a transcendental element $f\in G$ (since
$E(G)=F$)such that $B\cup\{f\}$ wold be algebraically independent contradicting
the
maximality of $B$. Thus $B$ is our transcendence basis. \end{proof}
Now we prove that the transcendence degree of a field extension is independent
of choice of basis.
\begin{theorem} Let $F/E$ be a field extension. Any two transcendence bases for
$F/E$ have the same cardinality. This shows that the $trdeg(E/F)$ is well
defined. \end{theorem}
\begin{proof} 
Let $B$ and $B'$ be two transcendence bases. Without loss of generality, we can
assume that $\#(B')\leq \#(B)$. Now we divide the proof into two cases: the
first case is that $B$ is an infinite set. Then for each $\alpha\in B'$, there
is a finite set $B_{\alpha}$ such that $\alpha$ is algebraic over
$E(B_{\alpha})$ since any algebraic dependence relation only uses finitely many
indeterminates. Then we define $B^*=\bigcup_{\alpha\in B'} B_{\alpha}$. By
construction, $B^*\subseteq B$, but we claim that in fact the two sets are
equal. To see this, suppose that they are not equal, say there is an element
$\beta\in B\setminus B^*$. We know $\beta$ is algebraic over $E(B')$ which is
algebraic over $E(B^*)$. Therefor $\beta$ is algebraic over $E(B^*)$, a
contradiction. So $\#(B)\leq \sum_{\alpha\in B'} \#(B_{\alpha})$. Now if $B'$ is
finite, then so is $B$ so we can assume $B'$ is infinite; this means
\begin{equation} \#(B)\leq \sum_{\alpha\in B'}\#(B_{\alpha})=\#(\coprod
B_{\alpha})\leq \#(B'\times\mathbb{Z})=\#(B')\end{equation} with the inequality $\#(\coprod
B_{\alpha}) \leq \#(B'\times \mathbb{Z})$ given by the correspondence
$b_{\alpha_i}\mapsto (\alpha,i)\in B'\times \mathbb{Z}$ with $B_\alpha =
\{b_{\alpha_1},b_{\alpha_2}\cdots b_{\alpha_{n_\alpha}}\}$ Therefore in the
infinite case, $\#(B)=\#(B')$.

Now we need to look at the case where $B$ is finite. In this case, $B'$ is also
finite, so suppose $B=\{\alpha_1,\cdots\alpha_n\}$ and
$B'=\{\beta_1,\cdots\beta_m\}$ with $m\leq n$. We perform induction on $m$: if
$m=0$ then $F/E$ is algebraic so $B=\null$ so $n=0$, otherwise there is an
irreducible polynomial $f\in E[x,y_1,\cdots y_n]$ such that
$f(\beta_1,\alpha_1,\cdots \alpha_n) = 0$. Since $\beta_1$ is not algebraic over
$E$, $f$ must involve some $y_i$ so without loss of generality, assume $f$ uses
$y_1$. Let $B^*=\{\beta_1,\alpha_2,\cdots\alpha_n\}$. We claim that $B^*$ is a
basis for $F/E$. To prove this claim, we see that we have a tower of algebraic
extensions $F/E(B^*,\alpha_1)/E(B^*)$ since $\alpha_1$ is algebraic over
$E(B^*)$. Now we claim that $B^*$ (counting multiplicity of elements) is
algebraically independent over $E$ because if it weren't, then there would be an
irreducible $g\in E[x,y_2,\cdots y_n]$ such that
$g(\beta_1,\alpha_2,\cdots\alpha_n)=0$ which must involve $x$ making $\beta_1$
algebraic over $E(\alpha_2,\cdots \alpha_n)$ which would make $\alpha_1$
algebraic over $E(\alpha_2,\cdots \alpha_n)$ which is impossible. So this means
that $\{\alpha_2,\cdots\alpha_n\}$ and $\{\beta_2,\cdots\beta_m\}$ are bases for
$F$ over $E(\beta_1)$ which means by induction, $m=n$. \end{proof}

\begin{example} Consider the field extension $\mathbb{Q}(e,\pi)$ formed by
adjoining the numbers $e$ and $\pi$. This field extension has transcendence
degree at least $1$ since both $e$ and $\pi$ are transcendental over the
rationals. However, this field extension might have transcendence degree $2$ if
$e$ and $\pi$ are algebraically independent. Whether or not this is true is
unknown and the problem of determining $trdeg(\mathbb{Q}(e,\pi))$ is an open
problem.\end{example}
\begin{example} let $E$ be a field and $F=E(t)/E$. Then $\{t\}$ is a
transcendence basis since $F=E(t)$. However, $\{t^2\}$ is also a transcendence
basis since $E(t)/E(t^2)$ is algebraic. This illustrates that while we can
always decompose an extension $F/E$ into an algebraic extension $F/E'$ and a
purely transcendental extension $E'/E$, this decomposition is not unique and
depends on choice of transcendence basis. \end{example}
\begin{exercise} If we have a tower of fields $G/F/E$, then $trdeg(G/E)=trdeg(F/E)+trdeg(G/F)$. \end{exercise}

\subsection{Linearly Disjoint Field Extensions}
Let $k$ be a field, $K$ and $L$ field extensions of $k$. Suppose also that $K$ and $L$ are embedded in some larger field $\Omega$. 

\begin{definition} The compositum of $K$ and $L$ written $KL$ is $k(K\cup L)=L(K)=K(L)$. 
\end{definition}



\begin{definition} $K$ and $L$ are said to be linearly disjoint over $k$ if the following map is injective:
\begin{equation} \theta: K\otimes_k L\rightarrow KL \end{equation} defined by $x\otimes y\mapsto xy$. 
\end{definition}


\section{Algebraic closure}

\begin{definition} 
Let $F$ be a field. An \textbf{algebraic closure} of $F$ is a field
$\overline{F}$ containing $F$ such that:
\begin{enumerate}[\textbf{AC} 1]
\item $\overline{F} $ is algebraic over $F$.
\item $\overline{F}$ is \textbf{algebraically closed} (that is, every
non-constant polynomial in $\overline{F}[X]$ has a root in $\overline{F}$).
\end{enumerate}
\end{definition} 

\begin{theorem}
Every field has an algebraic closure.
\end{theorem}

\begin{proof}
Let $ K$ be a field and $ \Sigma$ be the set of all monic irreducibles in $ K[x]$. Let $ A = K[\{x_f : f \in \Sigma\}]$ be the polynomial ring generated by indeterminates $ x_f$, one for each $ f \in \Sigma$. Then let $ \mathfrak{a}$ be the ideal of $ A$ generated by polynomials of the form $ f(x_f)$ for each $ f \in \Sigma$.

\emph{Claim 1}. $ \mathfrak{a}$ is a proper ideal.

\emph{Proof of claim 1}. Suppose $ \mathfrak{a} = (1)$, so there exist finitely many polynomials $ f_i \in \Sigma$ and $ g_i \in A$ such that $ 1 = f_1(x_{f_1}) g_1 + \dotsb + f_k(x_{f_k}) g_k$. Each $ g_i$ uses some finite collection of indeterminates $ V_i \{x_{f_{i_1}}, \dotsc, x_{f_{i_{k_i}}}\}$. This notation is ridiculous, so we simplify it.

We can take the union of all the $ V_i$, together with the indeterminates $ x_{f_1}, \dotsc, x_{f_k}$ to get a larger but still finite set of indeterminates $ V = \{x_{f_1}, \dotsc, x_{f_n}\}$ for some $ n \geq k$ (ordered so that the original $ x_{f_1}, \dotsc, x_{f_k}$ agree the first $ k$ elements of $ V$). Now we can regard each $ g_i$ as a polynomial in this new set of indeterminates $ V$.
Then, we can write $ 1 = f_1(x_{f_1}) g_1 + \dotsb + f_n(x_{f_n}) g_n$ where for each $ i > k$, we let $ g_i = 0$ (so that we've adjoined a few zeroes to the right hand side of the equality).
Finally, we define $ x_i = x_{f_i}$, so that we have
$ 1 = f_1(x_1)g_1(x_1, \dotsc, x_n) + \dotsb + f_n(x_n) g_n(x_1, \dotsc, x_n)$.

Suppose $ n$ is the minimal integer such that there exists an expression of this form, so that

\[ \mathfrak{b} = (f_1(x_1), \dotsc, f_{n-1}(x_{n-1})) \]

is a proper ideal of $ B = K[x_1, \dotsc, x_{n-1}]$, but

\[ (f_1(x_1), \dotsc, f_n(x_n)) \]

is the unit ideal in $ B[x_n]$. Let $ \hat{B} = B/\mathfrak{b}$ (observe that this ring is nonzero). We have a composition of maps

\[ B[x_n] \to \hat{B}[x_n] \to \hat{B}[x_n]/(\widehat{f_n(x_n)}) \]

where the first map is reduction of coefficients modulo $ \mathfrak{b}$, and the second map is the quotient by the principal ideal generated by the image $ \widehat{f_n(x_n)}$ of $ f_n(x_n)$ in $ \hat{B}[x_n]$. We know $ \hat{B}$ is a nonzero ring, so since $ f_n$ is monic, the top coefficient of $ \widehat{f_n(x_n)}$ is still $ 1 \in \hat{B}$. In particular, the top coefficient cannot be nilpotent. Furthermore, since $ f_n$ was irreducible, it is not a constant polynomial, so by the characterization of units in polynomial rings, $ \widehat{f_n(x_n)}$ is not a unit, so it does not generate the unit ideal. Thus the quotient $ \hat{B}[x_n]/(\widehat{f_n(x_n)})$ should not be the zero ring.

On the other hand, observe that each $ f_i(x_i)$ is in the kernel of this composition, so in fact the entire ideal $ (f_1(x_1), \dotsc, f_n(x_n))$ is contained in the kernel. But this ideal is the unit ideal, so all of $ B[x_n]$ is in the kernal of this composition. In particular, $ 1 \in B[x_n]$ is in the kernal, and since ring maps preserve identity, this forces $ 1 = 0$ in $ \hat{B}[x_n]/(\widehat{f_n(x_n)})$, which makes this the the zero ring. This contradicts our previous observation, and proves the claim that $ \mathfrak{a}$ is a proper ideal.

Now, given claim 1, there exists a maximal ideal $ \mathfrak{m}$ of $ A$ containing $ \mathfrak{a}$. Let $ K_1 = A/\mathfrak{m}$. This is an extension field of $ K$ via the inclusion given by

\[ K \to A \to A/\mathfrak{m} \]

(this map is automatically injective as it is a map between fields). Furthermore every $ f \in \Sigma$ has a root in $ K_1$. Specifically, the coset $ x_f + \mathfrak{m}$ in $ A/\mathfrak{m} = K_1$ is a root of $ f$ since

\[ f(x_f + \mathfrak{m}) = f(x_f) + \mathfrak{m} = 0. \]

Inductively, given $ K_n$ for some $ n \geq 1$, repeat the construction with $ K_n$ in place of $ K$ to get an extension field $ K_{n+1}$ of $ K_n$ in which every irreducible $ f \in K_n[x]$ has a root. Let $ L = \bigcup_{n = 1}^{\infty} K_n$.

\emph{Claim 2}. Every $ f \in L[x]$ splits completely into linear factors in $ L$.

\emph{Proof of claim 2}. We induct on the degree of $ f$. In the base case, when $ f$ itself is linear, there is nothing to prove. Inductively, suppose every polynomial in $ L[x]$ of degree less than $ n$ splits completely into linear factors, and suppose

\[ f = a_0 + a_1x + \dotsb + a_nx^n \in L[x] \]

has degree $ n$. Then each $ a_i \in K_{n_i}$ for some $ n_i$, so let $ n = \max n_i$ and regard $ f$ as a polynomial in $ K_n[x]$. If $ f$ is reducible in $ K_n[x]$, then we have a factorization $ f = gh$ with the degree of $ g, h$ strictly less than $ n$. Therefore, inductively, they both split into linear factors in $ L[x]$, so $ f$ must also. On the other hand, if $ f$ is irreducible, then by our construction, it has a root $ a\in K_{n+1}$, so we have $ f = (x - a) g$ for some $ g \in K_{n+1}[x]$ of degree $ n - 1$. Again inductively, we can split $ g$ into linear factors in $ L$, so clearly we can do the same with $ f$ also. This completes the proof of claim 2.

Let $ \bar{K}$ be the set of algebraic elements in $ L$. Clearly $ \bar{K}$ is an algebraic extension of $ K$. If $ f \in \bar{K}[x]$, then we have a factorization of $ f$ in $ L[x]$ into linear factors

\[ f = b(x - a_1)(x - a_2) \dotsb (x - a_n). \]

for $ b \in \bar{K}$ and, a priori, $ a_i \in L$. But each $ a_i$ is a root of $ f$, which means it is algebraic over $ \bar{K}$, which is an algebraic extension of $ K$; so by transitivity of "being algebraic," each $ a_i$ is algebraic over $ K$. So in fact we conclude that $ a_i \in \bar{K}$ already, since $ \bar{K}$ consisted of all elements algebraic over $ K$. Therefore, since $ \bar{K}$ is an algebraic extension of $ K$ such that every $ f \in \bar{K}[x]$ splits into linear factors in $ \bar{K}$, $ \bar{K}$ is the algebraic closure of $ K$.

\end{proof}

\section{Galois theory}
\subsection{Definitions}

Throughout, $F \subset K$ is a finite field extension.  We fix once and for
all an algebraic closure $M$ for both and an embedding of $F$ in $M$.  When
necessary, we write $K = F(\alpha_1, \dots, \alpha_n)$, and $K_0 = F, K_i =
F(\alpha_1, \dots, \alpha_i)$, $q_i$ the minimal polynomial of $\alpha_i$ over
$F_{i - 1}$, $Q_i$ that over $F$.

\begin{definition} $\Aut(K/F)$ denotes the group of automorphisms of $K$ which fix
$F$ (pointwise!).  $\Emb(K/F)$ denotes the set of embeddings of $K$ into $M$
respecting the chosen embedding of $F$.
\label{def:gal}
\end{definition}

\begin{definition} For an element $\alpha \in K$ with minimal polynomial $q$, we say
$q$ and $\alpha$ are separable if $q$ has distinct roots, and we say $K$ is
separable if this holds for all $\alpha$; conversely, we say they are purely
inseparable if $q$ has only one root.  We say $K$ is splitting if each $q$
splits in $K$.
\label{def:sepsplit}
\end{definition}

\begin{definition} By $\deg(K/F)$ we mean the dimension of $K$ as an $F$-vector
space.  We denote $K_s/F$ the set of elements of $K$ whose minimal polynomials
over $F$ have distinct roots; by \ref{sep_subfield} this is a subfield, and
$\deg(K_s/F) = \deg_s(K/F)$ and $\deg(K/K_s) = \deg_i(K/F)$ by definition.
\label{def:sep}
\end{definition}

\begin{definition} If $K = F(\alpha)$ for some $\alpha$ with minimal polynomial
$q(x) \in F[x]$, then by \ref{sep_poly}, $q(x) = r(x^{p^d})$, where $p =
\Char{F}$ (or $1$ if $\Char{F} = 0$) and $r$ is separable; in this case we
also denote $\deg_s(K/F) = \deg(r), \deg_i(K/F) = p^d$.  \label{def:prim_sep}
\end{definition}

\subsection{Theorems}

\begin{lemma} $q(x) \in F[x]$ is separable if and only if $\gcd(q, q') = 1$,
where $q'$ is the formal derivative of $q$. 
\label{der_poly}
\end{lemma}

\begin{proof} Passing to $M$, we may factor $q$:
\begin{equation*}
q(x) = \prod_{i = 1}^n (x - r_i)^{m_i}
\end{equation*}
and by the product rule,
\begin{equation*}
q'(x) = \sum_{j = 1}^n m_j (x - r_j)^{m_j - 1} \prod_{i \neq j} (x -
	r_i)^{m_i}.
\end{equation*}
$\gcd(q, q')$ is independent of the field since $K[x]$ is a PID, hence is the
product of the common factors of $q$ and $q'$ over $M$.  $q$ and $q'$ have a
common factor in $M$ if and only if they have a common root, since they both
split, and therefore if and only if $q'$ vanishes at some $r_j$.  Given the
above representation, this holds if and only if $m_j > 1$. \end{proof}

\begin{lemma} If $\Char{F} = 0$ then $K_s = K$.  If $\Char{F} = p > 0$, then for
any irreducible $q(x) \in K[x]$, there is some $d \geq 0$ and polynomial $r(x)
\in K[x]$ such that $q(x) = r(x^{p^d})$, and $r$ is separable and irreducible.
\label{sep_poly}
\end{lemma}

\begin{proof} By formal differentiation, $q'(x)$ has positive degree unless
each exponent is a multiple of $p$; in characteristic zero this never occurs.
If this is not the case, since $q$ is irreducible, it can have no factor in
common with $q'$ and therefore has distinct roots by \ref{der_poly}.

If $p > 0$, let $d$ be the largest integer such that each exponent of $q$ is a
multiple of $p^d$, and define $r$ by the above equation.  Then by
construction, $r$ has at least one exponent which is not a multiple of $p$,
and therefore has distinct roots. \end{proof}

\begin{corollary} In the statement of \ref{sep_poly}, $q$ and $r$ have the same
number of roots.
\label{sep_roots}
\end{corollary}

\begin{proof} $\alpha$ is a root of $q$ if and only if $\alpha^{p^d}$ is a
root of $r$; i.e. the roots of $q$ are the roots of $x^{p^d} - \beta$, where
$\beta$ is a root of $r$.  But if $\alpha$ is one such root, then $(x -
\alpha)^{p^d} = x^{p^d} - \alpha^{p^d} = x^{p^d} - \beta$ since $\Char{K} =
p$, and therefore $\alpha$ is the only root of $x^{p^d} - \beta$. \end{proof}

\begin{lemma} The correspondence which to each $g \in \Emb(K/F)$ assigns the
$n$-tuple $(g(\alpha_1), \dots, g(\alpha_n))$ of elements of $M$ is a
bijection from $\Emb(K/F)$ to the set of tuples of $\beta_i \in M$, such that
$\beta_i$ is a root of $q_i$ over $K(\beta_1, \dots, \beta_{i - 1})$.
\label{emb_roots}
\end{lemma}

\begin{proof} First take $K = F(\alpha) = F[x]/(q)$, in which case the maps $g
\colon K \to M$ over $F$ are identified with the elements $\beta \in M$ such
that $q(\beta) = 0$ (where $g(\alpha) = \beta$).

Now, considering the tower $K = K_n / K_{n - 1} / \dots / K_0 = F$, each
extension of which is primitive, and a given embedding $g$, we define
recursively $g_1 \in \Emb(K_1/F)$ by restriction and subsequent $g_i$ by
identifying $K_{i - 1}$ with its image and restricting $g$ to $K_i$.  By the
above paragraph each $g_i$ corresponds to the image $\beta_i = g_i(\alpha_i)$,
each of which is a root of $q_i$.  Conversely, given such a set of roots of
the $q_i$, we define $g$ recursively by this formula. \end{proof}

\begin{corollary} $|\Emb(K/F)| = \prod_{i = 1}^n \deg_s(q_i)$.
\label{emb_size}
\end{corollary}

\begin{proof} This follows immediately by induction from \ref{emb_roots} by
\ref{sep_roots}. \end{proof}

\begin{lemma} For any $f \in \Emb(K/F)$, the map $\Aut(K/F) \to \Emb(K/F)$ given
by $\sigma \mapsto f \circ \sigma$ is injective.  
\label{aut_inj}
\end{lemma}

\begin{proof} This is immediate from the injectivity of $f$. \end{proof}

\begin{corollary} $\Aut(K/F)$ is finite.
\label{aut_fin}
\end{corollary}

\begin{proof} By \ref{aut_inj}, $\Aut(K/F)$ injects into $\Emb(K/F)$, which by
\ref{emb_size} is finite. \end{proof}

\begin{proposition} The inequality
\begin{equation*}
|\Aut(K/F)| \leq |\Emb(K/F)|
\end{equation*}
is an equality if and only if the $q_i$ all split in $K$.
\label{aut_ineq}
\end{proposition}

\begin{proof} The inequality follows from \ref{aut_inj} and from \ref{aut_fin}.
Since both sets are finite, equality holds if and only if the injection of
\ref{aut_inj} is surjective (for fixed $f \in \Emb(K/F)$).

If surjectivity holds, let $\beta_1, \dots, \beta_n$ be arbitrary roots of
$q_1, \dots, q_n$ in the sense of \ref{emb_roots}, and extract an embedding $g
\colon K \to M$ with $g(\alpha_i) = \beta_i$.  Since the correspondence $f
\mapsto f \circ \sigma$ ($\sigma \in \Aut(K/F)$) is a bijection, there is some
$\sigma$ such that $g = f \circ \sigma$, and therefore $f$ and $g$ have the
same image.  Therefore the image of $K$ in $M$ is canonical, and contains
$\beta_1, \dots, \beta_n$ for any choice thereof.

If the $q_i$ all split, let $g \in \Emb(K/F)$ be arbitrary, so the
$g(\alpha_i)$ are roots of $q_i$ in $M$ as in \ref{emb_roots}.  But the $q_i$
have all their roots in $K$, hence in the image $f(K)$, so $f$ and $g$ again
have the same image, and $f^{-1} \circ g \in \Aut(K/F)$.  Thus $g = f \circ
(f^{-1} \circ g)$ shows that the map of \ref{aut_inj} is surjective.
\end{proof}

\begin{corollary} Define
\begin{equation*}
D(K/F) = \prod_{i = 1}^n \deg_s(K_i/K_{i - 1}).
\end{equation*}
Then the chain of equalities and inequalities
\begin{equation*}
|\Aut(K/F)| \leq |\Emb(K/F)| = D(K/F) \leq \deg(K/F)
\end{equation*}
holds; the first inequality is an equality if and only if each $q_i$ splits in
$K$, and the second if and only if each $q_i$ is separable.
\label{large_aut_ineq}
\end{corollary}

\begin{proof} The statements concerning the first inequality are just
\ref{aut_ineq}; the interior equality is just \ref{emb_size}; the latter
inequality is obvious from the multiplicativity of the degrees of field
extensions; and the deduction for equality follows from the definition of
$\deg_s$. \end{proof}

\begin{corollary} The $q_i$ respectively split and are separable in $K$ if and only
if the $Q_i$ do and are.
\label{absolute_sepsplit}
\end{corollary}

\begin{proof} The ordering of the $\alpha_i$ is irrelevant, so we may take
each $i = 1$ in turn.  Then $Q_1 = q_1$ and if either of the equalities in
\ref{large_aut_ineq} holds then so does the corresponding statement here.
Conversely, clearly each $q_i$ divides $Q_i$, so splitting or separability
for the latter implies that for the former. \end{proof}

\begin{corollary} Let $\alpha \in K$ have minimal polynomial $q$; if the $Q_i$ are
respectively split, separable, and purely inseparable over $F$ then $q$ is as
well.
\label{global_sepsplit}
\end{corollary}

\begin{proof} We may take $\alpha$ as the first element of an alternative
generating set for $K/F$.  The numerical statement of \ref{large_aut_ineq}
does not depend on the particular generating set, hence the conditions given
hold of the set containing $\alpha$ if and only if they hold of the canonical
set ${\alpha_1, \dots, \alpha_n}$.

For purely inseparable, if the $Q_i$ all have only one root then $|\Emb(K/F)|
= 1$ by \ref{large_aut_ineq}, and taking $\alpha$ as the first element of a
generating set as above shows that $q$ must have only one root as well for
this to hold. \end{proof}

\begin{corollary} $K_s$ is a field and $\deg(K_s/F) = D(K/F)$.
\label{sep_subfield}
\end{corollary}

\begin{proof} Assume $\Char{F} = p > 0$, for otherwise $K_s = K$.  Using
\ref{sep_poly}, write each $Q_i = R_i(x^{p^{d_i}})$, and let $\beta_i =
\alpha_i^{p^{d_i}}$.  Then the $\beta_i$ have $R_i$ as minimal polynomials and
the $\alpha_i$ satisfy $s_i = x^{p^{d_i}} - \beta_i$ over $K' = F(\beta_1,
\dots, \beta_n)$.  Therefore the $\alpha_i$ have minimal polynomials over $K'$
dividing the $s_i$ and hence those polynomials have but one distinct root.

By \ref{global_sepsplit}, the elements of $K'$ are separable, and those of
$K'$ purely inseparable over $K'$.  In particular, since these minimal
polynomials divide those over $F$, none of these elements is separable, so $K'
= K_s$.

The numerical statement follows by computation:
\begin{equation*}
\deg(K/K') = \prod_{i = 1}^n p^{d_i}
	= \prod_{i = 1}^n \frac{\deg(K_i/K_{i - 1})}{\deg_s(K_i/K_{i - 1})}
	= \frac{\deg(K/F)}{D(K/F)}. 
	\end{equation*}
\end{proof}

\begin{theorem} The following inequality holds:
\begin{equation*}
|\Aut(K/F)| \leq |\Emb(K/F)| = \deg_s(K/F) \leq \deg(K/F).
\end{equation*}
Equality holds on the left if and only if $K/F$ is splitting; it holds on the
right if and only if $K/F$ is separable.
\label{galois_size}
\end{theorem}

\begin{proof} The numerical statement combines \ref{large_aut_ineq} and
\ref{sep_subfield}.  The deductions combine \ref{absolute_sepsplit} and
\ref{global_sepsplit}. \end{proof}

\subsection{Definitions}

Throughout, we will denote as before $K/F$ a finite field extension, and $G =
\Aut(K/F)$, $H$ a subgroup of $G$.  $L/F$ is a subextension of $K/F$.

\begin{definition} When $K/F$ is separable and splitting, we say it is Galois and
write $G = \Gal(K/F)$, the Galois group of $K$ over $F$.
\label{defn:galois_extension}
\end{definition}

\begin{definition} The fixed field of $H$ is the field $K^H$ of elements fixed by
the action of $H$ on $K$.  Conversely, $G_L$ is the fixing subgroup of $L$,
the subgroup of $G$ whose elements fix $L$.
\label{defn:fixing}
\end{definition}

\subsection{Theorems}

\begin{lemma} A polynomial $q(x) \in K[x]$ which splits in $K$ lies in
$K^H[x]$ if and only if its roots are permuted by the action of $H$.  In this
case, the sets of roots of the irreducible factors of $q$ over $K^H$ are the orbits
of the action of $H$ on the roots of $q$ (counting multiplicity).
\label{root_action}
\end{lemma}

\begin{proof} Since $H$ acts by automorphisms, we have $\sigma q(x) = q(\sigma
x)$ as a functional equation on $K$, so $\sigma$ permutes the roots of $q$.
Conversely, since the coefficients of $\sigma$ are the elementary symmetric
polynomials in its roots, $H$ permuting the roots implies that it fixes the
coefficients.

Clearly $q$ is the product of the polynomials $q_i$ whose roots are the orbits
of the action of $H$ on the roots of $q$, counting multiplicities, so it
suffices to show that these polynomials are defined over $K^H$ and are
irreducible.  Since $H$ acts on the roots of the $q_i$ by construction, the
former is satisfied.  If some $q_i$ factored over $K^H$, its factors would
admit an action of $H$ on their roots by the previous paragraph.  The roots of
$q_i$ are distinct by construction, so its factors do not share roots; hence
the action on the roots of $q_i$ would not be transitive, a contradiction.
\end{proof}

\begin{corollary} Let $q(x) \in K[x]$; if it is irreducible, then $H$ acts
transitively on its roots; conversely, if $q$ is separable and $H$ acts
transitively on its roots, then $q(x) \in K^H[x]$ is irreducible.
\label{sep_irred}
\end{corollary}

\begin{proof} Immediate from \ref{root_action}. \end{proof}

\begin{lemma} If $K/F$ is Galois, so is $K/L$, and $\Gal(K/L) = G_L$..
\label{sub_galois}
\end{lemma}

\begin{proof} $K/F$ Galois means that the minimal polynomial over $F$ of every
element of $K$ is separable and splits in $K$; the minimal polynomials over $L
= K^H$ divide those over $F$, and therefore this is true of $K/L$ as well;
hence $K/L$ is likewise a Galois extension. $\Gal(K/L) = \Aut(K/L)$ consists
of those automorphisms $\sigma$ of $K$ which fix $L$; since $F \subset L$ we
have \emph{a fortiori} that $\sigma$ fixes $F$, hence $\Gal(K/L) \subset G$
and consists of the subgroup which fixes $L$; i.e. $G_L$. \end{proof}

\begin{corollary} If $K/F$ and $L/F$ are Galois, then the action of $G$ on elements of $L$
defines a surjection of $G$ onto $\Gal(L/F)$.  Thus $G_L$ is normal in $G$ and $\Gal(L/F) \cong G/G_L$.  Conversely, if $N \subset G$ is normal, then $K^N/F$ is Galois.
\label{normal}
\end{corollary}

\begin{proof} $L/F$ is splitting, so by \ref{root_action} the elements of $G$
act as endomorphisms (hence automorphisms) of $L/F$, and the kernel of this action is $G_L$.  By
\ref{sub_galois}, we have $G_L = \Gal(K/L)$, so $|G_L| = |\Gal(K/L)| = [K : L] = [K : F] / [L : F]$,
or rearranging and using that $K/F$ is Galois, we get $|G|/|G_L| = [L : F] =
|\Gal(L/F)|$.  Thus the map $G \to \Gal(L/F)$ is surjective and thus the induced map $G/G_L \to
\Gal(L/F)$ is an isomorphism.

Conversely, let $N$ be normal and take $\alpha \in K^N$.  For any conjugate $\beta$ of $\alpha$, we
have $\beta = g(\alpha)$ for some $g \in G$; let $n \in N$.  Then $n(\beta) = (ng)(\alpha) =
g(g^{-1} n g)(\alpha) = g(\alpha) = \beta$, since $g^{-1} n g \in N$ by normality of $N$.  Thus
$\beta \in K^N$, so $K^N$ is splitting, i.e., Galois. \end{proof}

\begin{proposition} If $K/F$ is Galois and $H = G_L$, then $K^H = L$.
\label{fixed_field}
\end{proposition}

\begin{proof} By \ref{sub_galois}, $K/L$ and $K/K^H$ are both Galois.  By
definition, $\Gal(K/L) = G_L = H$; since $H$ fixes $K^H$ we certainly have
$H < \Gal(K/K^H)$, but since $L \subset K^H$ we have \emph{a fortiori} that
$\Gal(K/K^H) < \Gal(K/L) = H$, so $\Gal(K/K^H) = H$ as well.  It follows
from \ref{galois_size} that $\deg(K/L) = |H| = \deg(K/K^H)$, so that $K^H =
L$. \end{proof}

\begin{lemma} If $K$ is a finite field, then $K^\ast$ is cyclic.
\label{fin_cyclic}
\end{lemma}

\begin{proof} $K$ is then a finite extension of $\mathbb{F}_p$ for $p =
\Char{K}$, hence has order $p^n$, $n = \deg(K/\mathbb{F}_p)$.  Thus
$\alpha^{p^n} = \alpha$ for all $\alpha \in K$, since $|K^\ast| = p^n - 1$.
It follows that every element of $K$ is a root of $q_n(x) = x^{p^n} - x$.  For
any $d < n$, the elements of order at most $p^d - 1$ satisfy $q_d(x)$, which has
$p^d$ roots.  It follows that there are at least $p^n(p - 1) > 0$ elements of
order exactly $p^n - 1$, so $K^\ast$ is cyclic. \end{proof}

\begin{corollary} If $K$ is a finite field, then $\Gal(K/F)$ is cyclic, generated by
the Frobenius automorphism.
\label{fin_gal_cyclic}
\end{corollary}

\begin{proof} First take $F = \mathbb{F}_p$.  Then the map $f_i(\alpha) =
\alpha^{p^i}$ is an endomorphism, injective since $K$ is a field, and
surjective since it is finite, hence an automorphism.  Since every $\alpha$
satisfies $\alpha^{p^n} = \alpha$, $f_n = 1$, but by \ref{fin_cyclic}, $f_{n -
1}$ is nontrivial (applied to the generator).  Since $n = \deg(K/F)$, $f =
f_1$ generates $\Gal(K/F)$.

If $F$ is now arbitrary, by \ref{fixed_field} we have $\Gal(K/F) =
\Gal(K/\mathbb{F}_p)_F$, and every subgroup of a cyclic group is cyclic.
\end{proof}

\begin{corollary} If $K$ is finite, $K/F$ is primitive.
\label{fin_prim_elt}
\end{corollary}

\begin{proof} No element of $G$ fixes the generator $\alpha$ of $K^\ast$, so
it cannot lie in any proper subfield.  Therefore $F(\alpha) = K$. \end{proof}

\begin{proposition} If $F$ is infinite and $K/F$ has only finitely many subextensions, then it is
primitive.
\label{gen_prim_elt}
\end{proposition}

\begin{proof} We proceed by induction on the number of generators of $K/F$.

If $K = F(\alpha)$ we are done.  If not, $K = F(\alpha_1, \dots, \alpha_n) =
F(\alpha_1, \dots, \alpha_{n - 1})(\alpha_n) = F(\beta, \alpha_n)$ by
induction, so we may assume $n = 2$.  There are infinitely many subfields
$F(\alpha_1 + t \alpha_2)$, with $t \in F$, hence two of them are equal, say for $t_1$ and
$t_2$.  Thus, $\alpha_1 + t_2 \alpha_2 \in F(\alpha_1 + t_1 \alpha_2)$.  Then
$(t_2 - t_1)\alpha_2 \in F(\alpha_1 + t_1 \alpha_2)$, hence $\alpha_2$ lies in
this field, hence $\alpha_1$ does.  Therefore $K = F(\alpha_1 + t_1
\alpha_2)$. \end{proof}

\begin{corollary} If $K/F$ is separable, it is primitive, and the generator may be
taken to be a linear combination of any finite set of generators of $K/F$.
\label{prim_elt}
\end{corollary}

\begin{proof} We may embed $K/F$ in a Galois extension $M/F$ by adjoining all
the conjugates of its generators.  Subextensions of $K/F$ are as well subextensions
of $K'/F$ and by \ref{fixed_field} the map $H \mapsto (K')^H$ is a surjection
from the subgroups of $G$ to the subextensions of $K'/F$, which are hence
finite in number.  By \ref{fin_prim_elt} we may assume $F$ is infinite.  The
result now follows from \ref{gen_prim_elt}. \end{proof}

\begin{corollary}
 If $K/F$ is Galois and $H \subset G$, then if $L = K^H$, we have $H = G_L$.
 \label{fixing_subgroup}
\end{corollary}

\begin{proof}
 Let $\alpha$ be a primitive element for $K/L$.  The polynomial $\prod_{h \in H} (x - h(\alpha))$ is fixed by $H$, and therefore has coefficients in $L$, so $\alpha$ has $|H|$ conjugate roots over $L$.  But since $\alpha$ is primitive, we have $K = L(\alpha)$, so the minimal polynomial of $\alpha$ has degree $\deg(K/L)$, which is the same as the number of its roots.  Thus $|H| = \deg(K/L)$.  Since $H \subset G_L$ and $|G_L| = \deg(K/L)$, we have equality.
\end{proof}


\begin{theorem} The correspondences $H \mapsto K^H$, $L \mapsto G_L$ define
inclusion-reversing inverse maps between the set of subgroups of $G$ and the
set of subextensions of $K/F$, such that normal subgroups and Galois subfields
correspond.
\label{fundamental_theorem}
\end{theorem}

\begin{proof} This combines \ref{fixed_field}, \ref{fixing_subgroup}, and \ref{normal}.
\end{proof}


