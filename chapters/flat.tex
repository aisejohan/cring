\chapter{Flatness revisited}

In the past, we have already encountered the notion of \emph{flatness}. We
shall now study it in more detail.

\section{Faithful flatness}

\subsection{Faithfully flat modules}
Let $R$ be a commutative ring.

\begin{definition} 
The $R$-module $M$ is \textbf{faithfully flat} if  any complex $N' \to N
\to N''$ of $R$-modules is exact if and only if the tensored sequence $N'
\otimes_R M \to N \otimes_R M \to N'' \otimes_R M$ is exact.
\end{definition} 

Clearly, a faithfully flat module is flat.


\begin{example} 
The direct sum of faithfully flat modules is faithfully flat.
\end{example} 
\begin{example} 
A (nonzero)  free module is faithfully flat, because $R$ itself is flat
(tensoring with $R$ is the identity functor).
\end{example} 

We shall now prove several useful criteria about faithfully flat modules.

\begin{proposition}  \label{easyffcriterion}
An $R$-module $M$ is faithfully flat if and only if it is flat and if $M
\otimes_R N = 0$ implies $N=0$ for any $N$.
\end{proposition} 
\begin{proof} Suppose $M$ f.f.
Then $M$ is flat, clearly. In addition, if $N$ is any $R$-module, consider the
sequence
\[ 0 \to N \to 0;  \]
it is exact if and only if
\[ 0 \to M \otimes_R N \to 0  \]
is exact. Thus $N=0$ if and only if $M \otimes_R N = 0$.

Conversely, suppose $M$ is flat and satisfies the additional condition. We
need to show that if $N'
\otimes_R M \to N \otimes_R M \to N'' \otimes_R M$ is exact, so is $N' \to N
\to N''$. Since $M$ is flat, taking homology commutes with tensoring with $M$.
In particular, if $H$ is the homology of $N' \to N \to N''$, then $H \otimes_R
M$ is the homology of 
$N'
\otimes_R M \to N \otimes_R M \to N'' \otimes_R M$. It follows that $H
\otimes_R M = 0$, so $H=0$, and the initial complex is exact.
\end{proof} 


\begin{exercise} 
The direct sum of a flat module and a faithfully flat module is faithfully flat.
\end{exercise} 

A functor $F$ between two categories is said to be \textbf{faithful} if the
induced map on the hom-sets $\hom(x,y) \to \hom(Fx, Fy)$ is always injective.
The following result explains the use of the term ``faithful.''

\begin{proposition} 
A module $M$ is f.f. if and only if it is flat and the functor $N \to N
\otimes_R M$ is faithful.
\end{proposition} 
\begin{proof} Let $M$ be flat.
We need to check that $M$ is f.f. if and only if the natural map
\[ \hom_R(N, N') \to \hom_R(N \otimes_R M, N' \otimes_R M)  \]
is injective.
Suppose first $M$ is faithfully flat and $f: N \to N'$ goes to zero $f \otimes
1_M: N \otimes_R M \to  N' \otimes_R M$. We know by flatness that
\[ \im(f) \otimes_R M = \im(f \otimes 1_M)  \]
so that if $f \otimes 1_M = 0$, then $\im(f) \otimes M = 0$. Thus by faithful
flatness, $\im(f) = 0$ by Proposition~\ref{easyffcriterion}.

Conversely, let us suppose $M$ flat and the functor $N \to N \otimes_M$
faithful. Let $N \neq 0$; then $1_N \neq 0$ as maps $N \to N$. 
It follows that $1_N \otimes 1_M$ and $0 \otimes 1_M = 0$ are different as
endomorphisms of $M \otimes_R N$. Thus $M \otimes_R N \neq 0$. By
Proposition~\ref{easyffcriterion}, we are done again.
\end{proof} 


Finally, we prove one last criterion:

\begin{proposition} \label{ffmaximal} 
$M$ is f.f. if and only if $M$ is flat and $\mathfrak{m}M \neq M$ for all
maximal ideals $\mathfrak{m} \subset R$.
\end{proposition} 
\begin{proof} 
If $M$ is f.f., then $M$ is flat, and $M \otimes_R R/\mathfrak{m} =
M/\mathfrak{m}M \neq 0$ for all $\mathfrak{m}$ as $R/\mathfrak{m} \neq 0$, by
Proposition~\ref{easyffcriterion}. So we get one direction.

Alternatively, suppose $M$ is flat and $M \otimes_R R/\mathfrak{m} \neq 0$ for
all maximal $\mathfrak{m}$. Since every proper ideal is contained in a maximal
ideal, it follows that $M \otimes_R R/I \neq 0$ for all proper ideals $I$. We
shall use this and Proposition~\ref{easyffcriterion} to prove that $M$ is f.f.

Let $N$ now be any nonzero module. Then $N$ contains a \emph{cyclic} submodule, i.e.
one isomorphic to $R/I$ for some proper $I$. The injection
\[ R/I \hookrightarrow N  \]
becomes an injection
\[ R/I \otimes_R M \hookrightarrow N \otimes_R M,  \]
and since $R/I \otimes_R M \neq 0$, we find that $N \otimes_R M \neq 0$. By
Proposition~\ref{easyffcriterion}, it follows that $M$ is f.f.
\end{proof} 

\begin{corollary} 
A finitely generated flat module over a \emph{local} ring is faithfully flat.
\end{corollary} 

A \emph{finitely presented} flat module is in fact free, but we do not prove
this.
\begin{proof} 
Indeed, let $R$ be a local ring with maximal ideal $\mathfrak{m}$, and $M$ a
finitely generated flat $R$-module. Then by Nakayama's lemma, $M/\mathfrak{m}M
\neq 0$, so that $M$ must be faithfully flat.
\end{proof} 

\begin{proposition} 
Faithfully flat modules are closed under direct sums and tensor products.
\end{proposition} 

\begin{proof} 
Exercise.
\end{proof} 


\subsection{Faithfully flat algebras}

Let $\phi: R \to S$ be a morphism of rings, making $S$ into an $R$-algebra.

\begin{definition} 
$S$ is a \textbf{faithfully flat $R$-algebra} if it is faithfully flat as an
$R$-module.
\end{definition} 

\begin{proposition} 
If $S$ is a f.f. $R$-algebra, then the structure map $R \to S$ is injective.
\end{proposition} 
\begin{proof} 
Indeed, let us tensor the map $R \to S $ with $S$, over $R$. We get a morphism
of $S$-modules
\[ S \to S \otimes_R S ,  \]
sending $s \to  1 \otimes s$.
This morphism has an obvious section $S \otimes_R S \to S$ sending $a \otimes b
\to ab$. Since it has a section, it is injective. But faithful flatness says
that the original map must be injective itself.
\end{proof} 


\begin{proposition} 
A flat local homomorphism of local rings is faithfully flat.
\end{proposition} 
\begin{proof} 
Let $\phi: R \to S$ be a local homomorphism of local rings with maximal ideals
$\mathfrak{m}, \mathfrak{n}$. Then by definition $\phi(\mathfrak{m}) \subset
\mathfrak{n}$. It follows that $S \neq \phi(\mathfrak{m})S$, so by
Proposition~\ref{ffmaximal} we win.
\end{proof} 


\begin{example} 
The map $R \to R[x]$ from a ring into its polynomial ring is always faithfully
flat. This is clear.
\end{example} 

There are many topological consequences of faithful flatness on the $\spec$'s. These are
explored in detail in volume 4-2 of \cite{EGA}. We shall only scratch the
surface.

\begin{proposition} 
Let $R \to S$ be a faithfully flat morphism of rings. Then the map $\spec S
\to \spec R$ is surjective.
\end{proposition} 

\begin{proof} Since $R \to S$ is injective, we may regard $R$ as a subring of $S$.
We shall first show that:

\begin{lemma} 
If $I \subset R$ is any ideal, then $R \cap IS = I$.
\end{lemma}
\begin{proof} 
To see this, note that the morphism
\[ R/I \to S/IS  \]
is faithfully flat, since faithful flatness is preserved by base-change, and
this is the base-change of $R \to S$ via $R \to R/I$.
In particular, it is injective. Thus $IS \cap R = I$.
\end{proof} 


Now to see surjectivity, we use a general criterion:

\begin{lemma} 
Let $\phi: R \to S$ be a morphism of rings and suppose $\mathfrak{p} \in \spec
R$. Then $\mathfrak{p}$ is in the image of $\spec S \to \spec R$ if and only if 
$\phi^{-1}( \phi(\mathfrak{p}) S) = \mathfrak{p}$.
\end{lemma} 

This lemma will prove the proposition.
\begin{proof} 
Suppose first that $\mathfrak{p}$ is in the image of $\spec S \to \spec R$. In
this case, there is $\mathfrak{q} \in \spec S$ such that
\[ \mathfrak{p} = \phi^{-1}(\mathfrak{q}).  \]
In particular, $\mathfrak{q} \supset \phi(\mathfrak{p})S$, so that
\[ \mathfrak{p} \subset \phi^{-1}(\phi(\mathfrak{p}) S),  \]
while the other inclusion is obviously true.

Conversely, suppose that $\mathfrak{p} \subset \phi^{-1}(\phi(\mathfrak{p})
S)$. In this case, we know that 
\[ \phi(R  - \mathfrak{p}) \cap \phi(\mathfrak{p})S = \emptyset.  \]
Now $T = \phi(R - \mathfrak{p})$ is a multiplicatively closed subset.
There is a morphism
\begin{equation} \label{randomequationwhichidonthaveanamefor}
R_{\mathfrak{p}} \to T^{-1}S 
\end{equation} 
which sends elements of $\mathfrak{p}$ into non-units, by
\eqref{randomequationwhichidonthaveanamefor} so it is a \emph{local}
homomorphism. The maximal ideal of $T^{-1} S$ pulls back to that of
$R_{\mathfrak{p}}$. By the usual commutative diagrams, it follows that
$\mathfrak{p}$ is the preimage of something in $\spec S$.
\end{proof} 
\end{proof} 

In fact, one can show that the morphism $\spec S \to \spec R$ is actually an
\emph{identification,} that is, a quotient map. This is true more generally
for faithfully flat and quasi-compact morphisms of schemes; see \cite{EGA},
volume 4-2.

\begin{exercise} 
Let $R \to S$ be a faithfully flat morphism of rings. If $S$ is noetherian, so
is $R$. The converse is false!
\end{exercise} 




\section{Interpretation via Tor}

Recall that the derived functors of the tensor product functor are denoted by $\mathrm{Tor} _R^i, i \geq 0$.  So $\mathrm{Tor} _R^0(M,N) = M \otimes N$.  These functors are symmetric in both variables, and  given an exact sequence
\[ 0 \to N' \to N \to N'' \to 0 \]
we have a long exact sequence
\[ \mathrm{Tor} ^i(N',M) \to \mathrm{Tor} ^i(N,M) \to \mathrm{Tor} ^i(N'',M ) \to \mathrm{Tor} ^{i-1}(N',M) \to \dots \]
For this, cf. books on homological algebra.

One of the basic applications of this is that for a flat module $M$, the tor-functors vanish for $i \geq 1$ (whatever be $N$).
Indeed, recall that $\mathrm{Tor} (M,N)$ is computed by taking a projective resolution of $N$,
\[ \dots \to P_2 \to P_1 \to P_0 \to M \to 0 \]
tensoring with $M$, and taking the homology.  But tensoring with $M$ is exact if we have flatness, so the higher $\mathrm{Tor} $ modules vanish.

Conversely, suppose $\mathrm{Tor} ^i(M,N) = 0$ for all $N$ and $i>0$.  Then given an exact sequence,
\[ 0 \to N'   \to N \to N'' \to 0 \]
we get exactness of
\[  \mathrm{Tor} _1(N'',M)  \to N' \otimes M \to N \otimes M \to N'' \otimes M \to 0 \]
so the $\mathrm{Tor} $-vanishing (as we saw, we only needed it for $i=1$) gives flatness.
By taking direct (filtered) limits, which preserve exactness, we can reduce to the case of checking that $\mathrm{Tor} ^1(M,N) = 0$ for $N$ finitely generated.  By using a filtration on $N$ whose quotients are of the form $R/I$ for $I \subset R$ an ideal, we find that:

\begin{proposition} $M$ is flat iff $\mathrm{Tor} ^1(M,R/I)=0$ for all finitely generated ideals $I \subset R$.
\end{proposition}

Note that there is an exact sequence $0 \to I \to R \to R/I \to 0$ and
so
\[ \mathrm{Tor} _1(M,R)=0 \to \mathrm{Tor} _1(M,R/I) \to I \otimes M \to M \]
is exact, and it suffices to show that
\[ I \otimes M \to M \]
is injective for all $I$.

\begin{theorem} If $R$ is noetherian local and $M$ finitely generated and flat, then $M$ is free.
\end{theorem}
Indeed, let $\mathfrak{m} \subset R$ be the maximal ideal and $k$ the residue field.  Now $M/\mathfrak{m} M$ is a finitely generated $k$-vector space; choose a basis.  This induces an isomorphism
\[ k^n \to M/\mathfrak{m} M \]
which we can lift to a map
\[ R^n \to M ,\]
which is surjective by Nakayama's lemma, and which becomes an isomorphism upon being tensored with $k$.  Let $Q$ be the kernel:
\[ 0 \to Q \to R^n \to M \to 0;\]
then we have an exact sequence
\[ \mathrm{Tor} ^1(k,M)=0 \to Q \otimes k \to k^n \to M/\mathfrak{m} M \to 0.\]
As a result $Q \otimes k = Q/\mathfrak{m} Q = 0$, which implies $Q=0$ by Nakayama, so $R^n \to M$ is an isomorphism, q.e.d.

\section{Flatness over a local ring}

Let $R$ be a noetherian local ring with maximal ideal $\mathfrak{m}$ and residue field $k$, $S$ be a local finitely generated $R$-algebra with $\mathfrak{m}S \subset \mathfrak{n}$ for $\mathfrak{n}$ the maximal ideal of $S$, and $M$ a finitely generated $S$-module.

\begin{theorem} $M$ is flat over $R$ iff
\[ \mathrm{Tor} ^1_R( k, M) = 0.\]
\end{theorem}

\begin{proof} 
Necessity is immediate.  What we have to prove is sufficiency.

First, I make the following claim. If $N$ is an $R$-module of finite length, then
\[ \mathrm{Tor} ^1_R( N, M)=0.\]
This is because $N$ has by devissage a filtration $N_i$ whose quotients are of the form $R/\mathfrak{p}$ for $\mathfrak{p}$ prime and (by finite length hypothesis) $\mathfrak{p}= \mathfrak{m}$.  
Then the result is true $N_0=0 \subset N$ trivially.  We climb up the filtration piece by piece inductively; if $\mathrm{Tor} ^1_R(N_i, M)=0$, then the exact sequence
\[ 0 \to N_i \to N_{i+1} \to k \to 0 \]
yields
\[ \mathrm{Tor} ^1_R(N_i, M) \to \mathrm{Tor} ^1_R(N_{i+1}, M) \to 0 \]
from the long exact sequence of $\mathrm{Tor} $ and the hypothesis on $M$.
The claim is proved.


The idea behind this proof is to show that $I \otimes_RM \to M$ is injective for any $I \subset R$.  We will use some diagram chasing and the Krull intersection theorem on the kernel $K$ of this map, to interpolate between it and various quotients by powers of $\mathfrak{m}$.
First we write some exact sequences.

We have an exact sequence
\[ 0 \to \mathfrak{m}^t \cap I \to I \to I/I \cap \mathfrak{m}^t \to 0\]
which we tensor with $M$:
\[   \mathfrak{m}^t \cap I \otimes M \to I \otimes M \to I/I \cap \mathfrak{m}^t \otimes M \to 0.\]

The sequence
\[ 0 \to  I/I  \cap \mathfrak{m}^t \to R/\mathfrak{m}^t \to R/(I+\mathfrak{m}^t) \to 0\]
is also exact, and tensoring with $M$ yields an exact sequence:
\[ 0 \to  I/I  \cap \mathfrak{m}^t \otimes M  \to M/\mathfrak{m}^tM  \to M/(\mathfrak{m}^t  + I) M \to 0\]
because $\mathrm{Tor} ^1_R(M,   R/(I+\mathfrak{m}^t))=0$ (finite length hypothesis.


Let us draw the following commutative diagram:\\
\xymatrix{
& & 0 \ar[d] \\
\mathfrak{m}^t \cap I \otimes M \ar[r] & I \otimes M \ar[r] & I/I \cap \mathfrak{m}^t \otimes M \ar[d] \\
& & M/\mathfrak{m}^t M 
}

where the column and the row are exact.
As a result, if an element in $I \otimes M$ goes to zero in $M$ (a fortiori in  $M/\mathfrak{m}^tM$) it must come from $\mathfrak{m}^t \cap I \otimes M$ for all $t$.  Thus it belongs to $\mathfrak{m}^t(I \otimes M)$ for all $t$, and the Krull intersection theorem (applied to $S$, since $\mathfrak{m}S \subset \mathfrak{n}$) implies it is zero.

\end{proof} 

\subsection{The infinitesimal criterion}

\begin{theorem} Let $M$ be a module over a noetherian local ring $R$.  Then $M$ is flat iff $M/\mathfrak{m}^tM$ is flat over $R/\mathfrak{m}^t$ for all $t>0$.
\end{theorem}
\begin{proof} 
One direction is easy, because if $M$ is a flat $R$-module and $S$ an $R$-algebra, then $M \otimes_R S$ is flat over $S$. 
For the other direction, take the same commutative diagram as before:\\
\xymatrix{
& & 0 \ar[d] \\
\mathfrak{m}^t \cap I \otimes M \ar[r] & I \otimes M \ar[r] & I/I \cap \mathfrak{m}^t \otimes M \ar[d] \\
& & M/\mathfrak{m}^t M 
}

The horizontal sequence was always exact.  The vertical sequence can be argued to be exact by tensoring the exact sequence 
\[ 0 \to  I/I  \cap \mathfrak{m}^t \to R/\mathfrak{m}^t \to R/(I+\mathfrak{m}^t) \to 0\]
of $R/\mathfrak{m}^t$-modules with $M/\mathfrak{m}^tM$, and using flatness.
Thus we get flatness of $M$ as before.

\end{proof} 
Incidentally, if we combine the local and infinitesimal criteria fo flatness, we get a little more.

