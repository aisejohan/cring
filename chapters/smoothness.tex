\chapter{Differentials and smoothness}


In this chapter, we shall introduce two notions. First, we shall discuss
\emph{regular} local rings. On varieties over an algebraically closed field,
regularity corresponds to nonsingularity of the variety at that point.
(Over non-algebraically closed fields, the connection is more subtle.) This
will be a continuation of the local algebra done earlier in the chapter \cref{}
on dimension theory.
We shall next introduce the module of \emph{K\"ahler differentials} of a
morphism of rings $A \to B$, which itself can measure smoothness (though this
connection will not be elucidated until a later chapter).

\section{Regular local rings}
We shall start by introducing the concept of a \emph{regular local} ring, which
is one where the embedding dimension and Krull dimension coincide.
\subsection{Regular local rings}

Let $A$ be a local noetherian ring with maximal ideal $\mathfrak{m} \subset A$
and residue field $k = A/\mathfrak{m}$.
Endow $A$ with the $\mathfrak{m}$-adic topology, so that there is a natural
$k$-algebra $\gr(A) = \bigoplus \mathfrak{m}^i/\mathfrak{m}^{i+1}$.
This is a finitely generated $k$-algebra; indeed, a system of generators for
the ideal $\mathfrak{m}$ generates $\gr(A)$ over $k$.
As a result, we have a natural surjective map of $k$-algebras.
\begin{equation} \label{reglocringmap} \Sym_k \mathfrak{m}/\mathfrak{m}^2 \to
\gr(A).  \end{equation}
Here $\Sym$ denotes the \emph{symmetric algebra.}
\begin{definition} A local ring is called \textbf{regular} if the above map is
an isomorphism, or equivalently if the embedding dimension of $A$ is equal to
the Krull dimension.
\end{definition}

We want to show the ``equivalently'' in the definition is justified.
One direction is easy: if \eqref{reglocringmap} is an isomorphism, then
$\gr(A)$ is a polynomial ring with $\dim_k \mathfrak{m}/\mathfrak{m}^2$
generators. But the dimension of $A$ was defined in terms of the growth of
$\dim_k \mathfrak{m}^i/\mathfrak{m}^{i+1}  = (\gr A)_i$. 
For a polynomial ring on $k$ generators, however, the $i$th graded piece has
dimension a degree-$k$ polynomial in $i$ (easy verification). 
As a result, we get the claim.

However, we still have to show that if the embedding dimension equals the Krull
dimension, then \eqref{reglocringmap} is an isomorphism. This will follow from
the next lemma.

\begin{lemma} If the inequality \[\dim(A) \leq \dim_{A/m}(\mathfrak{m}/\mathfrak{m}^2)\]
is an equality, then \eqref{reglocringmap} is an isomorphism. 
\end{lemma}
\begin{proof}
Suppose \eqref{reglocringmap} is not an isomorphism. 
Then there is an element $f \in \Sym_k \mathfrak{m}/\mathfrak{m}^2$ which is
not zero and which maps to zero in $\gr(A)$; we can assume $f$ homogeneous,
since the map of graded rings is graded.

Now the claim is that if $k[x_1, \dots, x_n]$ is a polynomial ring and $f \neq
0$ a homogeneous element, then the Hilbert polynomial of $k[x_1, \dots,
x_n]/(f)$ is less than $n$. This will easily imply the lemma, since
\eqref{reglocringmap} is always a surjection. 
Now if $\deg f = d$, then the dimension of $(k[x_1, \dots, x_n]/f)_i$ (where
$i$ is a degree) is $\dim (k[x_1, \dots, x_n])_i = \dim (k[x_1, \dots,
x_n])_{i-d}$. It follows that if $P$ is the Hilbert polynomial of the
polynomial ring, that of the quotient is $P(\cdot) - P(\cdot - d)$, which has a
strictly smaller degree. 
\end{proof}

\begin{remark} Let $A$ be a local ring and $\hat{A}$ its completion. Then $\dim(A)=\dim(\hat{A})$, because $A/\mathfrak{m}^n=\hat{A}/\hat{\mathfrak{m}}^n$, so the Hilbert functions are the same. Similarly, $\gr(A)=\gr(\hat{A})$. However, by  $\hat{A}$ is also a local ring. So applying the above lemma, we see  $A$ is regular if and only if $\hat{A}$ is regular.
\end{remark}


\begin{lemma} A regular local ring is a domain.
\label{reg loc means domain}
\end{lemma}
This is a formal consequence of the fact that $\gr(A)$ is a domain and the
filtration on $A$ is complete.
\begin{proof} Let $a,b \neq 0$. Note that $\bigcap \mathfrak{m}^n=0$ by the
Krull intersection theorem (\cref{krullint}), so there are $k_1$ and $k_2$ such that 
$a \in \mathfrak{m}^{k_1} - \mathfrak{m}^{k_1 + 1}$ and $b \in
\mathfrak{m}^{k_2} - \mathfrak{m}^{k_2 + 1}$.
Let $\overline{a}, \overline{b}$ be the images of $a,b$ in $\gr(A)$ (in
degrees $k_1, k_2$); neither is
zero.
But then $\bar{a}\bar{b} \neq 0 \in \gr(A)$, because $\gr(A)=\Sym(\mathfrak{m}/\mathfrak{m}^2)$ is a domain. So $ab \neq 0$, as desired.
\end{proof}


We now study quotients of regular local rings. 
\begin{lemma}
If $A$ is a regular local ring, and $f \in \mathfrak{m}$ is such that $f \in
\mathfrak{m}- \mathfrak{m}^2$. Then $A'=A/fA$ is also regular of dimension $\dim(A)-1$.
\label{reg loc mod f still reg loc}
\end{lemma}
\begin{proof} First let us show the dimension part of the statement. We know from \ref{local dim drop} that the dimension has to drop. On the other hand if the images of $\rad(f_1, \ldots f_n)=\mathfrak{m}/(f)$ in $A/fA$, then $\rad(f,f_1, \ldots f_n)=\mathfrak{m}$ in $A$, giving us the other direction.

Now we want to show that it is regular. This amounts to showing that $\dim_{A'/\mathfrak{m}_{A'}}(\mathfrak{m}_{A'}/\mathfrak{m}_{A'}^2)=\dim(A')$, or in other words that \[\dim_{A'/\mathfrak{m}_{A'}}(\mathfrak{m}_{A'}/\mathfrak{m}_{A'}^2) \leq \dim_{A/\mathfrak{m}_A}(\mathfrak{m}_{A}/\mathfrak{m}_A^2)-1.\]
But note that we have a surjection
\[A \twoheadrightarrow A'=A/fA\]
and $\mathfrak{m}_{A'}=\mathfrak{m}_A/f$, so we also have a surjection
\[\mathfrak{m}_A \twoheadrightarrow \mathfrak{m}_{A'}\]
and since we have
\[\mathfrak{m}_A^2 \to \mathfrak{m}_{A'}^2\]
the map factors as
\[\alpha: \mathfrak{m}_A/\mathfrak{m}_A^2 \twoheadrightarrow \mathfrak{m}_{A'}/\mathfrak{m}_{A'}^2.\]
However, all of these maps are maps of $A$ modules, so this last map is a map of $A/\mathfrak{m}_A$ vector spaces (note that $A/\mathfrak{m}_A=A'/\mathfrak{m}_{A'}$ because $f$ is in $\mathfrak{m}_A$).

Also note that $f \neq 0 \in \mathfrak{m}_A/\mathfrak{m}_A^2$, but $\alpha$ maps the image of $f$ to 0, so $\alpha$ is not injective as a map of vector spaces. It is however, surjective, so the dimension must drop, as desired.
\end{proof}



\begin{corollary}  Consider elements $f_1, \ldots f_m$ in $\mathfrak{m}$ such that $\bar{f_1}, \ldots \bar{f_m} \in \mathfrak{m}/\mathfrak{m}^2$ are independent. Then $A/(f_1, \ldots f_m)$ is regular with $\dim(A/(f_1, \ldots f_m))=\dim(A)-m$
\label{reg local mod fs still reg loc}
\end{corollary}
\begin{proof} This follows from Lemma \ref{reg loc mod f still reg loc} by induction. One just needs to check that in $A_1=A/(f_1)$, $\mathfrak{m}_1=\mathfrak{m}/(f_1)$, we have that $f_2, \ldots f_m$ are still linearly independent in $\mathfrak{m}_1/\mathfrak{m}_1^2$. This is easy to check.
\end{proof}



\begin{theorem} Let $A_0$ be a regular local ring of dimension $n$, and consider a short exact sequence
\[0 \to I \to A_0 \to A \to 0\]
where $A$ is a ring. Note that $\spec(A)$ is a closed subset of $\spec(A_0)$ so $A$ is also local. Then the following are equivalent
\begin{enumerate}
\item $A$ is regular
\item There are elements $f_1, \ldots f_m \in I$ such that $\bar{f_1}, \ldots \bar{f_m}$ are linearly independent in $\mathfrak{m}_{A_0}/\mathfrak{m}_{A_0}^2$ where $m=n-\dim(A)$ such that $(f_1, \ldots f_m)=I$.
\end{enumerate}
\label{reg loc main thm}
\end{theorem}
\begin{proof} \textbf{(2) $\Rightarrow$ (1)} This is exactly the statement of Corollary \ref{reg local mod fs still reg loc}

\noindent \textbf{(1) $\Rightarrow$ (2)} Just as in the proof of Lemma \ref{reg loc mod f still reg loc}, i.e, using Problems 2a and 5a in Problem Set 12, we see that there is exact sequence
\[I \otimes_{A_0} A_0/\mathfrak{m}_{A_0} \to \mathfrak{m}_{A_0}/\mathfrak{m}_{A_0}^2 \to \mathfrak{m}_{A}/\mathfrak{m}_{A}^2  \to 0.\]
When we proved this in Lemma \ref{reg loc mod f still reg loc}, we used $f \in \mathfrak{m}_A$, but that still works because $I \subset \mathfrak{m}_{A_0}$ because $A_0$ is local with maximal ideal $\mathfrak{m}_{A_0}$ so all ideals are in $\mathfrak{m}_{A_0}$.

By assumption $A_0$ and $A$ are regular local, and we know $A_0/\mathfrak{m}_{A_0}=A/\mathfrak{m}_A$, because $A=A_0/I$, $\mathfrak{m}_A=\mathfrak{m}_{A_0}/I$ so
\[\dim_{A_0/\mathfrak{m}_{A_0}}(\mathfrak{m}_{A_0}/\mathfrak{m}_{A_0}^2)=\dim(A_0)=n\]
and
\[\dim_{A_0/\mathfrak{m}_{A_0}}(\mathfrak{m}_{A}/\mathfrak{m}_{A}^2)=\dim(A)\]
so the image of $I_{A_0}\otimes A_0/\mathfrak{m}_{A_0}$ in $\mathfrak{m}_{A_0}/\mathfrak{m}_{A_0}^2$ has dimension $m=n-\dim(A)$. Let $\bar{f}_1, \ldots \bar{f}_m$ be a set of linearly independent generators of $\mathfrak{m}_{A_0}/\mathfrak{m}_{A_0}^2$, and let $f_1, \ldots f_m$ be liftings to $I$.

Let $I'$ be the ideal generated by $f_1, \ldots f_m$ and consider $A'=A_0/I'$. Then by Corollary \ref{reg local mod fs still reg loc}, we know that $A'$ is a regular local ring with dimension $n-m=\dim(A)$. Also $I' \subset I$ so we have
\[0 \to J \to A' \to A \to 0\]
But Lemma \ref{reg loc means domain}, this means that it is a domain with the same dimension as $A$. But we know from Lemma \ref{local dim drop} that if we mod a domain out by a nonzero ideal, the dimension strictly drops, so since the dimensions are the same, $J=0$, so $I$ is generated by $f_1, \ldots f_m$, as desired.

\end{proof}

\subsection{Examples of regular local rings}
We now give several examples.
\begin{example} 
If $\dim(R)=0$, i.e. $R$ is artinian, then $R$ is regular iff the maximal ideal
is zero, i.e. if $R$ is a field.
\end{example} 

\begin{example} 
If $\dim(R) =1$, then it is regular iff $\mathfrak{m}$ is principal. In a
noetherian local ring, the maximal ideal is principal iff $R$ is a DVR.
This is \emph{likely} already proved in these notes. 
\end{example} 
We find:
\begin{proposition} 
A one-dimensional regular local ring is the same thing as a DVR.
\end{proposition} 

\newcommand{\maxspec}{\mathrm{MaxSpec}}
\begin{example} 
Let $R$ be  be the coordinate ring $ \mathbb{C}[x_1, \dots, x_n]/I$ of an algebraic
variety. Let $\mathfrak{m}$ be  a maximal ideal corresponding to the origin.
Then $\maxspec R \subset \spec R$ is a subvariety of $\mathbb{C}^n$, and $0$ is in this subvariety.

Then I claim:

\begin{proposition} 
$R_{\mathfrak{m}}$ is regular iff $\maxspec R$ is a smooth submanifold near $0$.
\end{proposition}
\begin{proof} 
We will show that regularity implies smoothness. The other direction is omitted.

We have a surjection $\mathbb{C}[x_1, \dots, x_n ] \twoheadrightarrow R$, with
kernel $I$. There is a maximal ideal $\mathfrak{m}' \subset \mathbb{C}[x_1,
\dots, x_n]$ defined as $(x_1, \dots, x_n)$. Then we have a surjection
\[ \mathfrak{m}'/\mathfrak{m}'^2 \twoheadrightarrow \mathfrak{m}/\mathfrak{m}^2  \]
whose kernel is $I + \mathfrak{m}'^2/\mathfrak{m}'^2 $. We find that
\[ \mathfrak{m}/\mathfrak{m}^2 = \mathfrak{m}'/(I + \mathfrak{m}'^2).  \]
Note that $\mathbb{C}[x_1, \dots, x_n]_{\mathfrak{m}'}$ is a regular local
ring of dimension $n$.

The first claim is that $R_{\mathfrak{m}}$ is regular if and only if, after
localizing the polynomial ring at the maximal ideal $\mathfrak{m}'$, the ideal
$I$ is generated by $n - \dim(R)$ functions having linearly independent
derivatives.	Granting this claim, say $I_{\mathfrak{m}'}$ is generated by
elements $f_1, \dots, f_m \in I$; then there is a map
\[ \mathbb{C}^n \stackrel{(f_1, \dots, f_m)}{\to} \mathbb{C}^m  \]
which is a submersion at the origin as the derivatives $\nabla f_i$ are
linearly independent at the origin. The implicit function theorem tells us that
the inverse image of zero, i.e. $\maxspec R$, is locally a submanifold. 

Now we need to verify the claim made earlier. Namely, we will show that
regularity of $R$ implies that $I_{\mathfrak{m}}$ is generated by elements
whose derivatives are linearly independent. However, we will postpone this
until next time.
\end{proof} 
\end{example} 


\subsection{Regular local rings look alike}
So, as we've seen, regularity corresponds to smoothness. Complex analytically,
all smooth points are the same though---they're locally manifolds.  We'd like
an algebraic version of this. The vague
claim is that all regular local rings of the same dimension ``look alike.''


Let $(R, \mathfrak{m})$ be a noetherian  local ring. Consider the graded ring
\[ S = R/\mathfrak{m} \oplus \mathfrak{m}/\mathfrak{m}^2 \oplus \dots.  \]
If we write $k = R/\mathfrak{m}$ be the residue field, it is easy to see that this is a finitely
generated $k$-algebra. If we choose elements $x_1, \dots, x_n \in
\mathfrak{m}/\mathfrak{m}^2$ generating this vector space, then they generate
$S$ as an algebra. 

\begin{proposition} 
$R$ is regular if and only if $S$ is isomorphic to the polynomial ring $k[x_1,
\dots, x_n]$, i.e. for every $f \in k[x_1, \dots, x_n]$, if $f$ maps to zero
in $S$, then $f=0$.
\end{proposition} 
\begin{proof} 
Suppose first that $k[x_1, \dots, x_n] \twoheadrightarrow S$ isn't injective.
Then there exists a $f \neq 0$ in this polynomial ring which maps to zero in
$S$. Then $S$ is not just a quotient of this polynomial ring, but a quotient of
$k[x_1, \dots, x_n]/(f) \twoheadrightarrow S$. As this is a map of graded
rings, we can assume that $f$ is homogeneous. 

In particular, the Hilbert
function of $S$ is less than or equal to the Hilbert function of $k[x_1, \dots,
x_n]/(f)$. In particular, the degree of the Hilbert function of $S$, namely
the dimension of $R$, is at most
the degree of the Hilbert function of this quotient---and quotienting by $f$
will reduce the degree of the Hilbert function so that it is $<n$. So $\dim(R) <n$.

If $S$ is isomorphic to a polynomial ring, then we can just read off what the
Hilbert function of $R$ will be, and we find that its degree is $n$.
\end{proof} 

As we have seen, regularity is equivalent to a statement about the associated
graded of $R$. Now we would like to transfer this to statements about things
closer to $R$. 

\textbf{Assume now for simplicity that the residue field of $k=R/\mathfrak{m}$
maps back into $R$.} This is always true in complex algebraic geometry, as the
residue field is just $\mathbb{C}$. Choose generators $y_1, \dots, y_m \in
\mathfrak{m}$ where $n = \dim_k \mathfrak{m}/\mathfrak{m}^2$ is the embedding
dimension. We get a map in the other direction
\[ \phi:k[y_1, \dots, y_m] \to R  \]
thanks to the section $k \to R$. This map from the polynomial ring is maybe not
an isomorphism, but if we let $\mathfrak{m} \subset R$ be the maximal ideal,
and $\mathfrak{n} = (y_1, \dots, y_m)$, the maps on associated gradeds will be
the same.

We find, by the previous result:

\begin{proposition} 
$R$ is regular iff  $\phi$ induces an isomorphism on the associated graded,
i.e. if $\mathfrak{n}^t/\mathfrak{n}^{t+1} \to
\mathfrak{m}^t/\mathfrak{m}^{t+1}$ is an isomorphism.
\end{proposition} 

That is, $\phi$ induces an isomorphism
\[ k[y_1, \dots,y_m]/\mathfrak{n}^t \simeq R/\mathfrak{m}^t  \]
for all $t$, because it is an isomorphism on the associated graded level.
So this in turn is equivalent, upon taking inverse limits, to the statement that
$\phi$ induces an isomorphism
\[ k[[y_1, \dots, y_m ]] \to \hat{R} \]
at the level of completions.

We can now conclude:
\begin{theorem} 
Let $R$ be a regular local ring of dimension $m$. Suppose $R$ contains a copy
of its residue field $k$.\footnote{I.e. there is a section of the map $R
\twoheadrightarrow R/\mathfrak{m}$.} Then $\hat{R} \simeq k[[x_1, \dots, x_m]]$.
\end{theorem} 


Let us now state this informally. First, note that:
\begin{proposition} 
For any local noetherian ring $R$, we have $\dim(R) = \dim(\hat{R})$.
\end{proposition} 
\begin{proof} 
Immediate from the expression of dimension via Hilbert polynomials.
\end{proof} 

On a similar note, the \emph{embedding dimension} of $R$ is the same as that of
the completion, because $\mathfrak{m}/\mathfrak{m}^2$ is regular.
So:
\begin{proposition} 
$R$ is regular local iff $\hat{R}$ is regular local.
\end{proposition} 

Finally:
\begin{corollary} 
A complete noetherian regular local ring that contains a copy of its residue
field $k$ is a power series ring over $k$.
\end{corollary} 

It now makes sense to say:
\begin{quote}
\textbf{All \emph{complete} regular local rings of the same dimension look
alike.} (More precisely, this is true when $R$ is assumed to contain a copy of
its residue field, but this is not a strong assumption in practice. One can
show that this will be satisfied if $R$ contains \emph{any}
field.\footnote{This is not always satisfied---take the $p$-adic integers, for instance.})
\end{quote}

We won't get into the precise statement of the general structure theorem, when
the ring is not assumed to contain its residue field, but a safe
intuition to take away from this is the above bolded statement.


\subsection{Regular local rings are domains}


Here is one nice property of regular local rings.

\begin{proposition} 
If $R$ is a regular local (noetherian, as always) ring, then $R$ is a domain.
\end{proposition} 

Geometrically, this is saying that smooth points are locally irreducible.
\begin{proof} 
Say $xy=0$ in $R$. We want to prove that one of $x,y$ is zero. Let us invoke
the Krull intersection theorem, which states that $(0) = \bigcap
\mathfrak{m}^i$. Then if $x \neq 0$, $x \in \mathfrak{m}^t -
\mathfrak{m}^{t+1}$ for some $t$. Same for $y$, if it is not zero: we can
choose $y \in \mathfrak{m}^u - \mathfrak{m}^{u+1}$. Then $x,y$ correspond to
elements $\overline{x}, \overline{y}$ in the associated graded ring (in the
$t$th and $u$th pieces) which are nonzero. Their product is nonzero in the
associated graded ring because that is a polynomial ring, hence a domain. So
$\overline{x} \overline{y} \neq 0$ in $\mathfrak{m}^{s+t}/\mathfrak{m}^{s+t+1}$.

Thus $xy \neq 0$, contradiction.

\end{proof} 

Later we will prove much more. In fact, a regular local ring is a factorial
ring. This is something we're not ready to prove yet, but one consequence of
that will be the following algebro-geometric fact. Let $X = \spec
\mathbb{C}[X_1, \dots, X_n]/I$ for some ideal $I$; so $X$ is basically a subset
of $\mathbb{C}^n$ plus some nonclosed points. Then if $X$ is smooth, we find
that $\mathbb{C}[X_1, \dots, X_n]/I$ is locally factorial. Indeed, smoothness
implies regularity, hence local factoriality. The whole apparatus of Weil and
Cartier divisors now kicks in.
\lecture{11/10}


\subsection{Regularity and algebraic geometry}
We were talking about the theory of regular local rings. Recall an assertion
made last time.

Take the ring $\mathbb{C}[X_1, \dots, X_n]$, and $\mathfrak{m} = (X_1, \dots,
X_n)$ the maximal ideal at zero. Let $R = (\mathbb{C}[X_1, \dots,
X_n]_{\mathfrak{m}})/I$ for some ideal $I$. Let $\phi: \mathbb{C}[X_1, \dots,
X_n] \to R$ be the canonical map. The maximal ideal $\mathfrak{n}$ of $R$ is
generated by $\phi(\mathfrak{m})$.

Last time, we claimed:
\begin{proposition} 
$R$ is regular local iff $I$ is generated by functions $f_1, \dots, f_m$ which
have linearly independent derivatives at zero.
\end{proposition} 

\begin{proof} 
Let's first think about what the condition of having linearly independent
derivatives means. 

If we consider $\mathbb{C}[X_1, \dots, X_n]/\mathfrak{m}$,
this is isomorphic to $\mathbb{C}$, the isomorphism being given by evaluation
at zero. 
Now $\mathfrak{m}/\mathfrak{m}^2 = \mathbb{C}^n$ having a basis given
by the images of $X_1, \dots, X_n$. A more canonical way of describing this is
as the \textbf{cotangent space} of $\mathbb{C}^n$ at the origin. The idea is
that any polynomial $f$ corresponds to the 1-form $df = \sum \frac{\partial
f}{\partial X_i} dX_i$. The evaluation of this 1-form at the origin gives a
formal linear combination of the symbols $dX_i$. It is easy to see that $df|_0$
vanishes if $f$ is constant or is in $\mathfrak{m}^2$.
Restricting to $\mathfrak{m}$, we get a map
\[ \mathfrak{m}/\mathfrak{m}^2 \to \mathbb{C}^n, \quad f \to df|_0,  \]
which is obviously an isomorphism.

Consider $f_1, \dots, f_a \in \mathfrak{m}$. We have seen that \emph{the derivatives (or gradients) are
linearly independent at the origin iff the images of $f_1, \dots, f_a$ are
linearly independent in $\mathfrak{m}/\mathfrak{m}^2$. }

If we consider $\mathbb{C}[X_1, \dots,X_n]_{\mathfrak{m}}$, last time we
mentioned that it was a regular local ring. The result will now follow from
\begin{lemma} 
Let $R $ be a quotient of a regular local ring $S$, say $R = S/I$ for some
$I$. Let $\mathfrak{m} \subset S$ be the maximal ideal. Then $R$ is regular iff
$I$ is generated by elements $f_1, \dots, f_a$ which are linearly independent
in $\mathfrak{m}/\mathfrak{m}^2$.
\end{lemma} 
\begin{proof} 
First, the easy direction. Say $I = (f_1,\dots, f_a)$ where $f_1, \dots, f_a$
are linearly independent in $\mathfrak{m}/\mathfrak{m}^2$. $S$ is regular, so
the dimension is equal to the embedding dimension of $S$. We want to show the
same thing for $R$.

Now $\dim R = \dim S/(f_1, \dots, f_a)$. We would expect that the dimension
drops by $a$; we can't immediately conclude this, but at least can argue that
\[ \dim R \geq \dim S - a  \]
by the principal ideal theorem. Let now $\mathfrak{n} \subset R$ be the maximal
ideal. The embedding dimension of $R$ is the dimension of
$\mathfrak{n}/\mathfrak{n}^2 \simeq \mathfrak{m}/(I + \mathfrak{m}^2)$. This is
a quotient of $\mathfrak{m}/\mathfrak{m}^2$, so its dimension is the dimension
of $\mathfrak{m}/\mathfrak{m}^2$ minus the image of $I$ in
$\mathfrak{m}/\mathfrak{m}^2$. This is precisely the embedding dimension of $S$
minus $a$, i.e. $\dim S - a$. We learn that
\[ \dim R \geq \dim S - a = \mathrm{embedding \ dim} R,  \]
which implies that $R$ is local, as the converse implication is true in any
noetherian local ring.

Now we want to do the converse. Say that $R$ is regular of dimension $\dim S
-a$. We want to find elements $f_1, \dots, f_a$.
So far, we know that the embedding dimension of $R$ is equal to the embedding
dimension of $S$ minus $a$. In particular,
\[ \dim \mathfrak{n}/\mathfrak{n}^2 = \dim \mathfrak{m}/(\mathfrak{m}^2+I) =
\dim \mathfrak{m}/\mathfrak{m}^2 - a.  \]
We can choose $f_1, \dots, f_a \in I$ such that their images in
$\mathfrak{m}/\mathfrak{m}^2$ are a basis for  the image of $I$. 
We have maps
\[ S \twoheadrightarrow S/(f_1, \dots, f_a) \twoheadrightarrow S/I = R.  \]
What can we say about the intermediate ring $R'=S/(f_1, \dots, f_a)$? It is
obtained from a regular local ring by killing elements linearly independent in
$\mathfrak{m}/\mathfrak{m}^2$. In particular, $R'$ is regular local of
dimension $\dim (S) -a$. 

We want to prove that $I = (f_1, \dots, f_a)$, i.e. $R = R'$. Suppose not. Then
$R  = R'/J$ for some ideal $J \neq 0$. Choose any $x \in J$ which is not zero.
Then $x$ is  a nonzerodivisor on $R'$ because $R'$ is regular. In particular, $R'/(x)$ has dimension
$\dim R' -1$. Since $R$ is a quotient of this, we have that $\dim R < \dim R' =
\dim S -a$. This is a contradiction from our earlier assumptions.
\end{proof} 
\end{proof} 


The upshot of this is that in algebraic geometry, regularity has something to
do with smoothness. 

\begin{remark}[Warning] This argument proves that if $R \simeq K[x_1, \dots,
x_n]/I$ for $K$ algebraically closed, then $R_{\mathfrak{m}}$ is regular local for some maximal ideal
$\mathfrak{m}$ if the corresponding algebraic variety is smooth at the
corresponding point. We proved this in the special case $K  = \mathbb{C}$ and
$\mathfrak{m}$ the ideal of the origin.

If $K$ is not algebraically closed, we \textbf{can't assume} that any maximal
ideal corresponds to a point in the usual sense. Moreover, if $K$ is not
perfect, regularity does \textbf{not} imply smoothness. We have not quite
defined smoothness, but here's a definition: smoothness means that the local
ring you get by base-changing $K$ to the algebraic closure is regular. So what
this means is that 
regularity of affine rings over a field $K$ is not preserved under
base-change from $K$ to $\overline{K}$. 
\end{remark} 

\begin{example} Let $K$ be non-perfect of characteristic $p$. Let $a$ not have
a $p$th root.
Consider $K[x]/(x^p -a)$. This is a regular local ring of dimension zero, i.e.
is a field. If $K$ is replaced by its algebraic closure, then we get
$\overline{K}[x]/(x^p - a)$, which is $\overline{K}[x]/(x- a^{1/p})^p$. This is
still zero-dimensional but is not a field. Over the algebraic closure, the ring
fails to be regular.
\end{example} 




\section{K\"ahler differentials}
\subsection{Derivations and K\"ahler differentials} Let $R$ be a ring with the maximal ideal
$\mathfrak{m}$. Then there is a $R/\mathfrak{m}$-vector space
$\mathfrak{m}/\mathfrak{m}^2$. This is what we would like to think of as the
``{cotangent space}'' of $\spec R$ at $\mathfrak{m}$. Intuitively, the
cotangent space is what you get by differentiating functions which vanish at
the point, but
differentiating functions that vanish twice should give zero. This is the moral
justification.
(Recall that on a smooth manifold $M$, if $\mathcal{O}_p$ is the local ring of
smooth functions defined in a neighborhood of $p \in M$, and $\mathfrak{m}_p
\subset \mathcal{O}_p$ is the maximal ideal consisting of ``germs'' vanishing
at $p$, then the cotangent space $T_p^* M$ is naturally
$\mathfrak{m}_p/\mathfrak{m}_p^2$.)

A goal might be to generalize this. What if you wanted to think about all
points at once? We'd like to describe the ``cotangent bundle'' to $\spec R$ in
an analogous way. Let's try and describe what would be a section to this
cotangent bundle. A section of $\Omega^*_{\spec R}$ should be the same
thing as a ``1-form'' on $\spec R$. We don't know what a 1-form is yet, but at
least we can give some examples. If $f \in R$, then $f$ is a ``function'' on
$\spec R$, and its ``differential'' should be a 1-form. So there should be a
``$df$'' which should be a 1-form. 
This is analogous to the fact that if $g$ is a real-valued function on the
smooth manifold $M$, then there is a 1-form $dg$.

We should expect the rules $d(fg)= df+dg$ and $d(fg) = f(dg) + g(df)$ as the
usual rules of differentiation. For this to make sense, 1-forms should be an
$R$-module. 
Before defining the appropriate object, we start with:

\begin{definition} 
Let $R$ be a commutative ring, $M$ an $R$-module. A \textbf{derivation} from
$R$ to $M$ is a map $D: R \to M$ such that the two identities below hold:
\begin{gather} D(fg)= Df + Dg  \\
 D(fg) = f(Dg) + g(Df).  \end{gather}
\end{definition} 
These equations make sense as $M$ is an $R$-module.

Whatever a 1-form on $\spec R$ might be, there should be a derivation
\[ d: R \to \left\{\text{1--forms}\right\}.  \]
An idea would be to \emph{define} the 1-forms or the ``cotangent bundle''
$\Omega_R$ by a
universal property. It should be universal among $R$-modules with a derivation.

To make this precise:
\begin{proposition} 
There is an $R$-module $\Omega_R$ and a derivation $d_{\mathrm{univ}} : R \to
\Omega_R$ satisfying the following universal property. For all $R$-modules
$M$, there is a canonical isomorphism 
\[ \hom_{R}(\Omega_R, M) \simeq \mathrm{Der}(R, M)  \]
given by composing the universal $d_{\mathrm{univ}}$ with a map $\Omega_R \to M$.
\end{proposition} 

That is, any derivation $d: R \to M$ factors through this universal derivation
in a unique way. Given the derivation $d: R \to M$, we can make the following diagram
commutative in a unique way such that $\Omega_R \to M$ is a morphism of
$R$-modules:
\[ 
\xymatrix{
R \ar[r]^d \ar[d]  &  M \\
\Omega_R \ar[ru]^{d_{\mathrm{univ}}}
}
\]

\begin{definition} 
$\Omega_R$ is called the module of \textbf{K\"ahler differentials} of $R$.
\end{definition} 

Let us now verify this proposition.
\begin{proof} 
This is like the verification of the tensor product. Namely, build a free
gadget and quotient out by whatever you need.

Let $\Omega_R$ be the quotient of the free $R$-module generated by elements
$da$ for $a \in R$ by enforcing the relations
\begin{enumerate}
\item $d(a+b) =da + db$. 
\item $d(ab) = adb + bda$.
\end{enumerate}
By construction, the map $a \to da$ is a derivation $R \to \Omega_R$. 
It is easy to see that it is universal. Given a derivation $d': R \to M$, we get a
map $\Omega_R \to M$ sending $da \to d'(a), a \in R$.
\end{proof} 

The philosophy of Grothendieck says that we should do this, as with everything,
in a relative context.
Indeed, we are going to need a slight variant, for the case of a \emph{morphism} of
rings.

\subsection{Relative differentials}

On a smooth manifold $M$, the derivation $d$ from smooth functions to 1-forms
satisfies an additional property: it maps the constant functions to zero. 
This is the motivation for the next definition:

\begin{definition} 
Let $f: R \to R'$ be a ring-homomorphism. Let $M$ be an $R'$-module. A
derivation $d: R' \to M$ is \textbf{$R$-linear if $d(f(a)) = 0, a \in R$.}
This is equivalent to saying that $d$ is an $R$-homomorphism by the Leibnitz
rule.
\end{definition} 

Now we want to construct an analog of the ``cotangent bundle'' taking into
account linearity. 

\begin{proposition} 
Let $R'$ be an $R$-algebra.
Then there is a universal $R$-linear derivation $R'
\stackrel{d_{\mathrm{univ}}}{\to} \Omega_{R'/R}$.
\end{proposition} 
\begin{proof} 
Use the same construction as in the absolute case. We get a map $R' \to
\Omega_{R'}$ as before. This is not generally $R$-linear, so one has to
quotient out by the images of $d(f(r)), r \in R$.
In other words, $\Omega_{R'/R}$ is the quotient of the free $R'$-module on
symbols $\left\{dr', r' \in R'\right\}$ with the relations:
\begin{enumerate}
\item $d(r_1' r_2') = r'_1 d(r_2') + d(r'_1) r_2'$. 
\item $d(r_1' + r_2') = dr_1' + dr_2'$.
\item  $dr = 0$ for $r \in R$ (where we identify $r$ with its image $f(r)$ in
$R'$, by abuse of notation).
\end{enumerate}
\end{proof} 

\begin{definition} 
$\Omega_{R'/R}$ is called the module of \textbf{relative K\"ahler
differentials,} or simply K\"ahler differentials.
\end{definition} 

Here $\Omega_{R'/R}$ also corepresents a simple functor on the category of
$R'$-modules: given an $R'$-module $M$, we have
\[ \hom_{R'}(\Omega_{R'/R}, M) = \mathrm{Der}_R(R', M),  \]
where $\mathrm{Der}_R$ denotes $R$-derivations.
This is a \emph{subfunctor} of the functor $\mathrm{Der}_R(R', \cdot)$, and so
by Yoneda's lemma there is a natural map $\Omega_{R'} \to \Omega_{R'/R}$.
We shall expand on this in the future.

\subsection{The case of a polynomial ring}
Let us do an example to make this more concrete.

\begin{example} \label{polynomialringdiff}
Let $R' = \mathbb{C}[x_1, \dots, x_n], R = \mathbb{C}$. In this case, the claim
is that there is an isomorphism
\[ \Omega_{R'/R} \simeq R'^n.  \]
More precisely, $\Omega_{R'/R}$ is free on $dx_1, \dots,dx_n$. So the cotangent
bundle is ``free.'' In general, the module $\Omega_{R'/R}$ will not be free, or
even projective, so the intuition that it is a vector bundle will be rather
loose. (The projectivity will be connected to \emph{smoothness} of $R'/R$.)

\begin{proof} 
The construction $f \to \left( \frac{\partial f}{\partial x_i}  \right)$ gives
a map $R' \to R'^n$. By elementary calculus, this is a derivation, even an
$R$-linear derivation.  We get a map
\[ \phi:\Omega_{R'/R} \to R'^n  \]
by the universal property of the K\"ahler differentials. The claim is that this
map is an isomorphism. The map is characterized by sending $df$ to $\left(
\frac{\partial f}{\partial x_i}\right)$. Note that $dx_1, \dots, dx_n$ map to a
basis of $R'^n$ because differentiating $x_i$ gives 1 at $i$ and zero at $j
\neq i$. So we see that $\phi$ is surjective. 

There is a map $\psi: R'^n \to \Omega_{R'/R}$ sending $\left(a_i  \right)$ to
$\sum a_i dx_i$. It is easy to check that $\phi \circ \psi = 1$ from the
definition of $\phi$. What we still need to show is that $\psi \circ \phi =1$.
Namely, for any $f$, we want to show that $\psi \circ \phi(df) = df$ for $f \in
R'$. This is precisely the claim that $df = \sum \frac{\partial f}{\partial
x_i} dx_i$. Both sides are additive in $f$, indeed are derivations, and
coincide on monomials of degree one, so they are equal.
\end{proof} 

\end{example} 

\subsection{Exact sequences of K\"ahler differentials}
We now want to prove a few basic properties of K\"ahler differentials, which
can be seen either from the explicit construction or in terms of the functors
they represent, by formal nonsense.
These results will be useful in computation.

 Recall that if
$\phi: A \to B$ is a map of rings, we can define a $B$-module
\( \Omega_{B/A}\)  which is generated by formal symbols $ dx|_{x \in
B}$ and subject to the relations $d(x+y) = dx+dy$, $d(a)=0, a \in A$,
and $d(xy) = xdy + ydx$.
By construction, $\Omega_{B/A}$ is the receptacle from the universal $A$-linear
derivation into a $B$-module.

Let $A \to B \to C$ be a triple of maps of rings. There is an obvious map $dx \to dx$
\[ \Omega_{C/A} \to \Omega_{C/B}  \]
where both sides have the same generators, except with a few additional
relations on $\Omega_{C/B}$. We have to quotient by $db, b \in B$. In
particular, there is a map $\Omega_{B/A} \to \Omega_{C/A}$, $dx \to dx$, whose images
generate the kernel. This induces a map
\[ C \otimes_B \Omega_{B/A} \to \Omega_{C/A}.  \]
The image is the $C$-module generated by $db|_{b \in B}$, and this is the
kernel of the previous map.
We have proved:
\begin{proposition}[First exact sequence] Given a sequence $A \to B \to C$ of rings, there is an exact sequence 
\[  C \otimes_B \Omega_{B/A} \to \Omega_{C/A} \to \Omega_{C/B} \to 0 .\]
\end{proposition} 
\begin{proof}[Second proof]
There is, however, a more functorial means of seeing this sequence, which we
now describe. 
Namely, let us consider the category of $C$-modules, and the functors
corepresented by these three objects. We have, for a $C$-module $M$:
\begin{gather*} 
\hom_C(\Omega_{C/B}, M) = \mathrm{Der}_B(C, M) \\
\hom_C(\Omega_{C/A}, M) = \mathrm{Der}_A(C, M) \\
\hom_C(C \otimes_B \Omega_{B/A}, M) = \hom_B(\Omega_{B/A}, M) = \mathrm{Der}_A(B, M).
\end{gather*} 
By Yoneda's lemma, we know that a map of modules is the same thing as a natural
transformation between the corresponding corepresentable functors, in the
reverse direction. 
It is easy to see that there are natural transformations
\[ \mathrm{Der}_B(C, M) \to \mathrm{Der}_A(C, M), \quad \mathrm{Der}_A(C, M) \to \mathrm{Der}_A(B, M)  \]
obtained by restriction in the second case, and by doing nothing in the first
case (a $B$-derivation is automatically an $A$-derivation). 
The induced maps on the modules of differentials are precisely those described
before; this is easy to check (and we could have defined the maps by these
functors if we wished). Now to say that the sequence is right exact is to say
that for each $M$, there is an exact sequence of abelian groups
\[ 0 \to \mathrm{Der}_B(C, M) \to \mathrm{Der}_A(C, M) \to \mathrm{Der}_A(B, M).  \]
But this is obvious from the definitions: an $A$-derivation is a $B$-derivation
if and only if the restriction to $B$ is trivial.
This establishes the claim. 
\end{proof} 



Next, we are interested in a second exact sequence. In the past
(\cref{polynomialringdiff}), we computed the module of K\"ahler differentials
of a \emph{polynomial} algebra. While this was a special case, any algebra is a
quotient of a polynomial algebra. As a result, it will be useful to know how
$\Omega_{B/A}$ behaves with respect to quotienting $B$.

 Let $A
\to  B$ be a  homomorphism of rings and $I \subset B $ an ideal. We would like
to describe $\Omega_{B/I/A}$. There is a map
\[ \Omega_{B/A} \to \Omega_{B/I/A}  \]
sending $dx$ to $d \overline{x}$ for $\overline{x}$ the reduction of $x$ in
$B/I$. This is surjective on generators, so it is surjective. It is not
injective, though. In $\Omega_{B/I/A}$, the generators $dx, dx'$ are identified
if $x \equiv x' \mod I$.  Moreover, $\Omega_{B/I/A}$ is a 
$B/I$-module. 
This means that there will be additional relations for that. To remedy this, we
can tensor and consider the morphism
\[ \Omega_{B/A} \otimes_B B/I \to \Omega_{B/I/A} \to 0.  \]

Let us now define a map 
\[ \phi: I /I^2 \to \Omega_{B/A} \otimes_B B/I,  \]
which we claim will generate the kernel. Given $x \in I$, we define $\phi(x) =
dx$. If $x \in I^2$, then $dx \in I \Omega_{B/A}$ so $\phi$ is indeed a map of
abelian groups
$I/I^2 \to \Omega_{B/A} \otimes_B B/I$. Let us check that this is a
$B/I$-module homorphism. We would like to check that $\phi(xy) = y \phi(x)$
for $x \in I$ in
$\Omega_{B/A}/I \Omega_{B/A}$. This follows from the Leibnitz rule, $\phi(xy) =
y \phi(x) + xdy \equiv x \phi(x) \mod I \Omega_{B/A}$. So $\phi$ is also
defined. Its image is the submodule of $\Omega_{B/A}/I \Omega_{B/A}$ generated
by $dx, x \in I$. This is precisely what one has to quotient out by to get
$\Omega_{B/I/A}$. In particular:

\begin{proposition}[Second exact sequence] Let $B$ be an $A$-algebra and $I \subset B$ an ideal.
There is an exact sequence
\[ I/I^2 \to \Omega_{B/A} \otimes_B B/I \to \Omega_{B/I/A} \to 0.  \]
\end{proposition} 

These results will let us compute the module of K\"ahler differentials in cases
we want.

\begin{example} 
Let $B = A[x_1, \dots, x_n]/I$ for $I$ an ideal. We will compute $\Omega_{B/A}$.

First, $\Omega_{A[x_1, \dots, x_n]/A} \otimes B \simeq B^n$ generated by
symbols $dx_i$. There is a surjection of
\[ B^n \to \Omega_{B/A} \to 0  \]
whose kernel is generated by $dx, x \in I$, by the second exact sequence above.
If $I = (f_1, \dots, f_m)$, then the kernel is generated by 
$\left\{df_i\right\}$.
It follows that $\Omega_{B/A}$ is the cokernel of the map
\[ B^m \to B^n  \]
that sends the $i$th generator of $B^m$ to $df_i$ thought of as an element in
the free $B$-module $B^n$ on generators $dx_1, \dots, dx_n$. Here, thanks to
the Leibnitz rule, $df_i$ is
given by formally differentiating the polynomial, i.e.
\[ df_i = \sum_j \frac{\partial f_i}{\partial x_j} dx_j. \] We have thus
explicitly represented $\Omega_{B/A}$ as the cokernel of the matrix $\left(
\frac{\partial f_i}{\partial x_j}\right)$.
\end{example} 

Last time, we were talking about the connection of K\"ahler differentials and
the cotangent bundle.
\begin{example} 
Let $R = \mathbb{C}[x_1, \dots, x_n]/I$ be the coordinate ring of an algebraic
variety. Let $\mathfrak{m} \subset R$ be the maximal ideal. Then 
$\Omega_{R/\mathbb{C}}$ is what you should think of as containing information
of the cotangent bundle of $\spec R$. You might ask what the fiber over a point
$\mathfrak{m} \in \spec R$ is, though. That is, we might ask what
\[ \Omega_{R/\mathbb{C}} \otimes_R R/\mathfrak{m}  \]
is. To see this, we note that there are maps
\[ \mathbb{C} \to R \to R/\mathfrak{m} \simeq \mathbb{C}.  \]
There is now an exact sequence by our general properties
\[ \mathfrak{m}/\mathfrak{m}^2 \to \Omega_{R/\mathbb{C}} \otimes_R
R/\mathfrak{m} \to \Omega_{\mathbb{R}/\mathfrak{m}/\mathbb{C}} \to 0  \]
where the last thing is zero as $R/\mathfrak{m} \simeq \mathbb{C} $ by the
Nullstellensatz.
The upshot is that $\Omega_{R/\mathbb{C}} \otimes_R R/\mathfrak{m}$ is a
quotient of $\mathfrak{m}/\mathfrak{m}^2$. Let's leave it there for now.
\end{example} 

\subsection{K\"ahler differentials and base change}

We now want to show that the formation of $\Omega$ is compatible with base
change. Namely, let $B$ be an $A$-algebra, visualized by a morphism $ A \to B$.
If $A \to A'$ is any morphism of rings, we can think of the \emph{base-change}
$A' \to A' \otimes_A B$; we often write $B' =  A' \otimes_A B$.

\begin{proposition} \label{basechangediff} With the above notation, there is a canonical isomorphism
of $B'$-modules:
\[ \Omega_{B/A} \otimes_A A' \simeq \Omega_{B'/A'}.  \]
\end{proposition} 
Note that, for a $B$-module, the functors $\otimes_A A'$ and $\otimes_B B'$ are
the same. So we could have as well written $\Omega_{B/A} \otimes_B B' \simeq
\Omega_{B'/A'}$.
\begin{proof} 
We will use the functorial approach. Namely, for a $B'$-module $M$, we will
show that there is a canonical isomorphism
\[ \hom_{B'}( \Omega_{B/A} \otimes_A A', M) \simeq 
\hom_{B'}( \Omega_{B'/A'}, M) .
\]
The right side represents $A'$-derivations $B' \to M$, or $\mathrm{Der}_{A'}(B', M)$.
The left side represents $\hom_B(\Omega_{B/A}, M)$, or $\mathrm{Der}_A(B, M)$.
Here the natural map of modules corresponds by Yoneda's lemma to the restriction
\[ \mathrm{Der}_{A'}(B', M) \to \mathrm{Der}_A(B, M).  \]
We need to see that this restriction map is an isomorphism. But given an
$A$-derivation $B \to M$, this is to say that extends in a \emph{unique} way to
an $A'$-linear derivation $B' \to M$. This is easy to verify directly.
\end{proof} 


\subsection{Differentials and localization}
We now show that localization behaves \emph{extremely} nicely with respect to
the formation of K\"ahler differentials. This is important in algebraic
geometry for knowing that the ``cotangent bundle'' can be defined locally.

\begin{proposition} \label{localizationdiff}
Let $f: A \to B$ be a map of rings. Let $S \subset  B$ be multiplicatively
closed. Then the natural map
\[ S^{-1}\Omega_{B/A} \to \Omega_{S^{-1}B/A}  \]
is an isomorphism.
\end{proposition} 
So the formation of K\"ahler differentials commutes with localization.

\begin{proof} 
We could prove this by the calculational definition, but perhaps it is better
to prove it via the universal property. If $M$ is any $S^{-1}B$-module, then 
we can look at 
\[ \hom_{S^{-1}B}( \Omega_{S^{-1}B/A}, M)  \]
which is given by the group of $A$-linear derivations $S^{-1}B \to M$, by the
universal property. 

On the other hand, 
\[ \hom_{S^{-1}B}( S^{-1} \Omega_{B/A}, M)  \]
is the same thing as the set of $B$-linear maps $\Omega_{B/A} \to M$, i.e. the
set of $A$-linear derivations $B \to M$. 

We want to show that these two are the same thing. Given an $A$-derivation
$S^{-1}B \to M$, we get an $A$-derivation $B \to M$ by pulling back. We want to
show that any $A$-linear derivation $B \to M$ arises in this way. So we need to
show that any $A$-linear derivation $d: B \to M$ extends uniquely to an $A$-linear
$\overline{d}: S^{-1}B \to M$.
Here are two proofs:
\begin{enumerate}
\item (Lowbrow proof.) For $x/s \in S^{-1}B$, with $x \in B, s \in S$, we
define $\overline{d}(x/s) = dx/s - xds/s^2$ as in calculus. The claim is that
this works, and is the only thing that works. One should check
this---\textbf{exercise}.
\item (Highbrow proof.) We start with a digression. Let $B$ be a commutative
ring, $M$ a $B$-module. Consider $B \oplus M$, which is a  $B$-module. We can
make it into a ring (via \textbf{square zero multiplication}) by multiplying
\[ (b,x)(b',x') = (bb', bx'+b'x).  \]
This is compatible with the $B$-module structure on $M \subset B \oplus
M$. Note that $M$ is an ideal in this ring with square zero.  Then the
projection $\pi: B \oplus M \to B$ is a ring-homomorphism as well.
There is also a ring-homomorphism in the other direction $b \to (b,0)$, which
is a section of $\pi$. There may be other homomorphisms $B \to B \oplus M$.

You might ask what all the right inverses to $\pi$ are, i.e. ring-homomorphisms
$\phi:  B \to B \oplus M $ such that $\pi \circ \phi = 1_{B}$. This must be of
the form $\phi: b \to (b, db)$ where $d: B \to M$ is some map. It is easy to check
that $\phi$ is a homomorphism precisely when $d$ is a derivation.

Suppose now $A \to B$ is a morphism of rings making $B$ an $A$-algebra. Then
$B \oplus M$ is an $A$-algebra via the inclusion $a \to (a, 0)$. Then
you might ask when $\phi: b \to (b, db), B \to B \oplus M$ is an
$A$-homomorphism. The answer is clear: when $d$ is an $A$-derivation.

Recall that we were in the situation of $f: A \to B$  a morphism of rings, $S
\subset B$ a multiplicatively closed subset, and $M$ an $S^{-1}B$-module. The
claim was that any $A$-linear derivation $d: B \to M$ extends uniquely to
$\overline{d}: S^{-1} B \to M$.
We can draw a diagram
\[ \xymatrix{
& B \oplus M \ar[d] \ar[r] &  S^{-1}B \oplus M \ar[d] \\
A \ar[r] &  B \ar[r] &  S^{-1}B
}.\]
This is a cartesian diagram. So given a section of $A$-algebras $B \to B \oplus M$, we have to
construct a section of $A$-algebras $S^{-1}B \to S^{-1}B \oplus M$. We can do this by the
universal property of localization, since $S$ acts by invertible elements on
$S^{-1}B \oplus M$. (To see this, note that $S$ acts by invertible elements on
$S^{-1}B$, and $M$ is a nilpotent ideal.)
\end{enumerate}
\end{proof} 

Finally, we note that there is an even slicker argument. (We learned this from
\cite{Qu}.) 
Namely, it suffices to show that $\Omega_{S^{-1}B/B} =0 $, by the exact
sequences.
But this is a $S^{-1}B$-module, so we have
\[  \Omega_{S^{-1}B/B} = \Omega_{S^{-1}B/B} \otimes_B S^{-1}B, \]
because tensoring with $S^{-1}B$ localizes at $S$, but this does nothing to a
$S^{-1}B$-module! By the base change formula (\cref{basechangediff}), we have
\[ \Omega_{S^{-1}B/B} \otimes_B S^{-1}B = \Omega_{S^{-1}B/S^{-1}B} = 0,  \]
where we again use the fact that $S^{-1} B \otimes_B S^{-1} B \simeq S^{-1}B$.


\section{Introduction to smoothness}
\subsection{K\"ahler differentials for fields}

Let us start with the simplest examples---fields.

\begin{example} 
Let $k$ be a field, $k'/k$ an extension. 
\begin{question} 
What does $\Omega_{k'/k}$ look like? When does it vanish?
\end{question} 
$\Omega_{k'/k}$ is a $k'$-vector space.

\begin{proposition} 
Let $k'/k$ be a separable algebraic extension of fields. Then $\Omega_{k'/k} = 0$.
\end{proposition} 
\begin{proof} 
We will need a formal property of K\"ahler differentials that is easy to check,
namely that they are compatible with filtered colimits. If $B = \varinjlim
B_\alpha$ for $A$-algebras $B_\alpha$, then there is a canonical isomorphism
\[ \Omega_{B/A} \simeq \varinjlim \Omega_{B_{\alpha}/A}.  \]
One can check this on generators and relations, for instance.

Given this, we can reduce to the case of $k'/k$ finite and separable. 
\begin{remark} 
Given a sequence of fields and morphisms $k \to k' \to k''$, then there is an
exact sequence
\[ \Omega_{k'/k} \otimes k'' \to \Omega_{k''/k} \to \Omega_{k''/k'} \to 0.  \]
In particular, if $\Omega_{k'/k} = \Omega_{k''/k'} =0 $, then $\Omega_{k''/k} =
0$. This is a kind of d\'evissage argument.
\end{remark} 

Anyway, recall that we have a finite separable extension $k'/k$ where $k' =
k(x_1, \dots, x_n)$.\footnote{We can take $n=1$ by the primitive element
theorem, but shall not need this.} We will show that
\[ \Omega_{k(x_1, \dots, x_i)/k(x_1, \dots, x_{i-1})} =0 \quad \forall i,  \]
which will imply by the devissage argument that $\Omega_{k'/k} = 0$.
In particular, we are reduced to showing the proposition when $k'$ is generated
over $k$ by a \emph{single element} $x$. Then we have that
\[ k' \simeq k[X]/(f(X))  \]
for $f(X)$ an irreducible polynomial. Set $I = (f(X))$. We have an exact sequence
\[ I/I^2 \to \Omega_{k[X]/k} \otimes_{k[X]} k' \to \Omega_{k'/k} \to 0 \]
The middle term is a copy of $k'$ and the first term is isomorphic to $k[X]/I
\simeq k'$. So there is an exact sequence
\[ k' \to k' \to \Omega_{k'/k} \to 0.  \]
The first term is, as we have computed, multiplication by $f'(x)$; however
this is nonzero by separability. Thus we find that $\Omega_{k'/k} =0$.
\end{proof} 
\end{example} 

\begin{remark} 
The above result is \textbf{not true} for inseparable extensions in general. 
\end{remark} 
\begin{example} 
Let $k$ be an imperfect field of characteristic $p>0$. There is $x \in k$ such
that $x^{1/p} \notin k$, by definition. Let $k' = k(x^{1/p})$. As a ring, this
looks like
$k[t]/(t^p - x)$. In writing the exact sequence, we find that $\Omega_{k'/k} =
k'$ as this is the cokernel of the map $k' \to k'$ given by multiplication
$\frac{d}{dt}|_{x^{1/p}} (t^p - x)$. That polynomial has identically vanishing
derivative, though. We find that a generator of $\Omega_{k'/k}$ is $dt$ where
$t$ is a $p$th root of $x$, and $\Omega_{k'/k } \simeq k$.
\end{example} 

Now let us consider transcendental extensions. Let $k' = k(x_1, \dots, x_n)$ be
a purely transcendental extension, i.e. the field of rational functions of
$x_1, \dots, x_n$.

\begin{proposition} 
If $k' = k(x_1, \dots, x_n)$, then $\Omega_{k'/k}$ is a free $k'$-module on the
generators $dx_i$.
\end{proposition} 
This extends to an \emph{infinitely generated} purely transcendental extension,
because K\"ahler differentials commute with filtered colimits. 
\begin{proof} 
We already know this for the polynomial ring $k[x_1, \dots, x_n]$. However, the
rational function field is just a localization of the polynomial ring at the
zero ideal.  So the result will follow from \cref{localizationdiff}.
\end{proof} 

We have shown that separable algebraic extensions have no K\"ahler
differentials, but that purely transcendental extensions have a free module of
rank equal to the transcendence degree.

We can deduce from this:
\begin{corollary} 
Let $L/K$ be a field extension of fields of char 0. Then 
\[ \dim_L \Omega_{L/K} = \mathrm{trdeg}(L/K).  \]
\end{corollary} 
\begin{proof}[Partial proof] 
Put the above two facts together. Choose a transcendence basis $\{x_\alpha\}$
for $L/K$. This means that $L$ is algebraic over $K(\left\{x_\alpha\right\})$
and the $\left\{x_\alpha\right\}$ are algebraically independent.
Moreover $L/K(\left\{x_\alpha\right\})$ is \emph{separable} algebraic.  Now let
us use a few things about these cotangent complexes. There is an exact sequence:
\[ \Omega_{K(\left\{x_\alpha\right\})}
\otimes_{K(\left\{x_\alpha\right\})} L \to \Omega_{L/K} \to \Omega_{L/K(\left\{x_\alpha\right\})}  \to 0 \]
The last thing is zero, and we know what the first thing is; it's free on the
$dx_\alpha$. So we find that $\Omega_{L/K}$ is generated by
the elements $dx_\alpha$. If we knew that the $dx_\alpha$ were linearly
independent, then we would be done. But we don't, yet. 
\end{proof}

This is \textbf{not true} in characteristic $p$. If $L = K(\alpha^{1/p})$ for
$\alpha \in K$ and $\alpha^{1/p} \notin K$, then $\Omega_{L/K} \neq 0$.

\subsection{Regularity, smoothness, and K\"ahler differentials}
From this, let us revisit a statement made last time. 
Let $K$ be an algebraically closed field, let $R = k[x_1, \dots, x_n]/I$ and
let $\mathfrak{m} \subset R$ be a maximal ideal. Recall that the
Nullstellensatz implies that $R/\mathfrak{m} \simeq k$. We were studying 
\[ \Omega_{R/k}.  \]
This is an $R$-module, so $\Omega_{R/k} \otimes_R k$ makes sense. There is a
surjection
\[ \mathfrak{m}/\mathfrak{m}^2 \to \Omega_{R/k} \otimes_R k \to 0,  \]
that sends $x \to dx$.
\begin{proposition} 
This map is an isomorphism.
\end{proposition} 
\begin{proof} 
We construct a map going the other way. Call the map $\mathfrak{m}/\mathfrak{m}^2 \to
\Omega_{R/k} \otimes_R k$ as $\phi$. We want to construct
\[ \psi: \Omega_{R/k} \otimes_R k \to \mathfrak{m}/\mathfrak{m}^2.  \]
This is equivalent to giving an $R$-module map 
\[ \Omega_{R/k} \to \mathfrak{m}/\mathfrak{m}^2,  \]
that is a derivation $\partial: R \to \mathfrak{m}/\mathfrak{m}^2$. This acts
via $\partial(\lambda + x) = x$ for $\lambda \in k, x \in \mathfrak{m}$. Since
$k+\mathfrak{m} = R$, this is indeed well-defined. We must check that
$\partial$ is a derivation. That is, we have to compute
$\partial((\lambda+x)(\lambda' + x'))$.
But this is 
\[ \partial(\lambda\lambda' + (\lambda x' + \lambda' x) + xx').  \]
The definition of $\partial$ is to ignore the constant term and look at the
nonconstant term mod $\mathfrak{m}^2$. So this becomes
\[ \lambda x' + \lambda' x = (\partial (\lambda+x)) (x'+\lambda') + (\partial (\lambda'+
x')) (x+\lambda)  \]
because $xx' \in \mathfrak{m}^2$, and because $\mathfrak{m}$ acts trivially on
$\mathfrak{m}/\mathfrak{m}^2$. Thus we get the map $\psi$ in the inverse
direction, and one checks that $\phi, \psi$ are inverses. This is because
$\phi$ sends $x \to dx$ and $\psi$ sends $dx \to x$.
\end{proof} 

\begin{corollary} 
Let $R$ be as before. Then $R_{\mathfrak{m}}$ is regular iff $\dim
R_{\mathfrak{m}} = \dim_k \Omega_{R/k} \otimes_R R/\mathfrak{m}$.
\end{corollary} 
In particular, the modules of K\"ahler differentials detect regularity for
certain rings.

\begin{definition} 
Let $R$ be a noetherian ring. We say that $R$ is \textbf{regular} if
$R_{\mathfrak{m}}$ is regular for every maximal ideal $\mathfrak{m}$. (This
actually implies that $R_{\mathfrak{p}}$ is regular for all primes
$\mathfrak{p}$, though we are not ready to see this. It will follow from the
fact that the localization of a regular local ring at a prime ideal is regular.)
\end{definition} 

Let $R = k[x_1, \dots, x_n]/I$ be an affine ring over an algebraically closed
field $k$. 
Then:

\begin{proposition} 
TFAE:
\begin{enumerate}
\item $R$ is regular. 
\item ``$R$ is smooth over $k$'' (to be defined)
\item  $\Omega_{R/k}$ is  a projective module over $R$ of rank $\dim R$.
\end{enumerate}
\end{proposition} 
A finitely generated projective module is locally free. So the last statement is that
$(\Omega_{R/k})_{\mathfrak{p}}$ is free of rank $\dim R$ for each prime
$\mathfrak{p}$.

\begin{remark} 
A projective module does not necessarily have a well-defined rank as an integer. For
instance, if $R = R_1 \times R_2$ and $M = R_1 \times 0$, then $M$ is a summand
of $R$, hence is projective. But there are two candidates for what the rank
should be. The problem is that $\spec R$ is disconnected into two pieces, and
$M$ is of rank one on one piece, and of rank zero on the other.
But in this case, it does not happen.
\end{remark}

\begin{remark} 
The smoothness condition states that locally on $\spec R$, we have an isomorphism with
$k[y_1, \dots, y_n]/(f_1, \dots, f_m)$ with the gradients $\nabla f_i$ linearly
independent. Equivalently, if $R_{\mathfrak{m}}$ is the localization of $R$ at
a maximal ideal  $\mathfrak{m}$, then $R_{\mathfrak{m}}$ is a regular local
ring, as we have seen.
\end{remark} 

\begin{proof} 
We have already seen that 1 and 2 are equivalent. The new thing is that they
are equivalent to 3. First, assume 1 (or 2). 
First, note that $\Omega_{R/k}$ is a finitely generated $R$-module; that's a general
observation:

\begin{proposition} 
If $f: A \to B$ is a map of rings that makes $B$ a finitely generated $A$-algebra, then
$\Omega_{B/A}$ is a finitely generated $B$-module.
\end{proposition} 
\begin{proof} 
We've seen this is true for polynomial rings, and we can use the exact
sequence. If $B$ is a quotient of a polynomial ring, then $\Omega_{B/A}$ is a
quotient of the K\"ahler differentials of the polynomial ring.
\end{proof} 
Return to the main proof. In particular, $\Omega_{R/k}$ is projective if and
only if $(\Omega_{R/k})_{\mathfrak{m}}$ is projective for every maximal ideal
$\mathfrak{m}$.  According to the second assertion, we have that
$R_{\mathfrak{m}}$ looks like $(k[y_1, \dots, y_n]/(f_1, \dots,
f_m))_{\mathfrak{n}}$ for some maximal ideal $\mathfrak{n}$, with the
gradients $\nabla f_i$ linearly independent. Thus
$(\Omega_{R/k})_{\mathfrak{m}} = \Omega_{R_{\mathfrak{m}}/k}$ looks like the cokernel of 
\[ R_{\mathfrak{m}}^m \to R_{\mathfrak{m}}^n  \]
where the map is multiplication by the Jacobian matrix $\left(\frac{\partial
f_i}{\partial y_j}  \right)$. By assumption this matrix has full rank. We see
that there is a left inverse of the reduced  map $k^m \to k^n$. 
We can lift this to a map $R_{\mathfrak{m}}^n \to R_{\mathfrak{m}}^m$. Since
this is a left inverse mod $\mathfrak{m}$, the composite is at least an
isomorphism (looking at determinants). Anyway, we see that $\Omega_{R/k}$ is
given by the cokernel of a map of free module that splits, hence is projective.
The rank is $n-m = \dim R_{\mathfrak{m}}$.

Finally, let us prove that 3 implies 1. Suppose $\Omega_{R/k}$ is projective of
rank $\dim R$. So this means that $\Omega_{R_{\mathfrak{m}}/k}$ is free of
dimension $\dim R_{\mathfrak{m}}$. But this implies that $(\Omega_{R/k})
\otimes_R R/\mathfrak{m}$ is free of the appropriate rank, and that is---as we
have seen already---the embedding dimension $\mathfrak{m}/\mathfrak{m}^2$. So
if 3 holds, the embedding dimension equals the usual dimension, and we get
regularity.
\end{proof} 

\begin{corollary} 
Let $R = \mathbb{C}[x_1, \dots, x_n]/\mathfrak{p}$ for $\mathfrak{p}$ a prime.
Then there is a nonzero $f \in R$ such that $R[f^{-1}]$ is regular.
\end{corollary} 
Geometrically, this says the following. $\spec R$ is some algebraic variety,
and $\spec R[f^{-1}]$ is a Zariski open subset. What we are saying is that, in
characteristic zero, any algebraic variety has a nonempty open smooth locus.
The singular locus is always smaller than the entire variety.

\begin{proof} 
$\Omega_{R/\mathbb{C}}$ is a finitely generated $R$-module. Let $K(R) $ be the fraction field of $R$.
Now
\[ \Omega_{R/\mathbb{C}} \otimes_R K(R) = \Omega_{K(R)/\mathbb{C}}  \]
is a finite $K(R)$-vector space. The dimension is
$\mathrm{trdeg}(K(R)/\mathbb{C})$. That is also $d=\dim R$, as we have seen.
Choose elements $x_1, \dots, x_d \in \Omega_{R/\mathbb{C}}$ which form a basis
for $\Omega_{K(R)/\mathbb{C}}$. There is a map
\[ R^d \to \Omega_{R/\mathbb{C}}  \]
which is an isomorphism after localization at $(0)$. This implies that there is
$f \in R$ such that the map is an isomorphism after localization at
$f$.\footnote{There is an inverse defined over the fraction field, so it is
defined over some localization.} We find that $\Omega_{R[f^{-1}]/\mathbb{C}}$
is free of rank $d$ for some $f$, which is what we wanted.
\end{proof} 

This argument works over any algebraically closed field of characteristic
zero, or really any field of characteristic zero.
\begin{remark}[Warning] Over imperfect fields in characteristic $p$, two things can happen:
\begin{enumerate}
\item Varieties need not be generically smooth 
\item $\Omega_{R/k}$ can be projective with the wrong rank
\end{enumerate}
(Nothing goes wrong for \textbf{algebraically closed fields} of characteristic
$p$.)
\begin{example} 
Here is a silly example. Say $R = k[y]/(y^p-x)$ where $x \in K$ has no $p$th
root. We know that $\Omega_{R/k}$ is free of rank one. However, the rank is
wrong: the variety has dimension zero.
\end{example} 
\end{remark} 

Last time, were trying to show that $\Omega_{L/K}$ is free on a transcendence
basis if $L/K$ is an extension in characteristic zero. So we had a tower of fields
\[ K \to K' \to L,  \]
where $L/K'$ was separable algebraic. 
We claim in this case that
\[ \Omega_{L/K} \simeq \Omega_{K'/K} \otimes_{K'} L.  \]
This will prove the result. But we had not done this yesterday.
\begin{proof} 
This doesn't follow directly from the previous calculations. Without loss of generality, $L$ is
finite over $K'$, and in particular, $L = K'[x]/(f(x))$ for $f$ separable. The claim is that
\[ \Omega_{L/K} \simeq (\Omega_{K'/K}\otimes_{K'}L \oplus K' dx)/f'(x)dx + \dots  \]
When we kill the vector $f'(x) dx + \dots$, we kill the second component. 
\end{proof} 



