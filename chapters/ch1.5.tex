\chapter{Fields and Extensions}
Recall that a field is an integral domain for which all non-zero elements are
invertible. Equivalently, the only two ideals of a field are $(0)$ and $(1)$
since any nonzero element is a unit.
\begin{example}
One of the most familiar examples of a field is $\mathbb{Q}$. \end{example}
\begin{example}
Another common field class of fields is the class of finite fields, for example
$\mathbb{Z}/(2)$.\end{example}
\begin{exercise} In fact, for any maximal ideal $\mathfrak{m}\subseteq R$, the
quotient $R/\mathfrak{m}$ is a field. \end{exercise}
\begin{example} If $F$ is a field, then any ring homomorphism $f:F\rightarrow
R$is either injective or the zero map; this is because $ker(f)$ is an ideal in
$F$but there are only two ideals, $(0)$ and $(1)$. \end{example}

\begin{definition} If $F$ is a field contained in a field $G$, then $G$ is said
to be a field extension of $F$. \end{definition}
There are plenty of field extensions coming from both number theory and
geometry.
\begin{example} For example let $X$ be a Riemann surface. Let $k(X)$ denote the
set of meromorphic functions on $X$; clearly $k(X)$ is a ring under
multiplication and addition of functions. It turns out that in fact $k(X)$ is a
field, this is because if $f(z)$ is meromorphic, so is $1/f(z)$. For example,
let $\mathbb{C}_{\infty}$ be the Riemann sphere; then we know from complex
analysis that the ring of meromorphic functions $k(\mathbb{C}_{\infty}$ is the
field of rational functions $\mathbb{C}(z)$. More is true. In fact, there is
always a holomorphic function $X\rightarrow \mathbb{C}_{\infty}$ obtained by
taking a meromorphic function $f\in k(X)$ and considering it as a holomorphic
function $f:X\rightarrow\mathbb{C}_{\infty}$. In this case, $f$ induces a map
$k(\mathbb{C}_{\infty})\rightarrow k(X)$ by composing with $f$. In particular,
this map is an injection meaning that $k(X)$ is a field extension of
$\mathbb{C}[z]$.
\end{example}

\begin{example} The previous example is a more specific case of a general
phenomenon in geometry; let $C_1$ and $C_2$ be smooth projective curves (over a
field $k$). Then a morphism $f:C_1\rightarrow C_2$ makes $k(C_1)$ into a field
extension $k(C_2)$ \end{example}

\begin{definition}An element $\alpha\in F$ is said to be algebraic over $E$ if
$\alpha$ is the root of some polynomial with coefficients in $E$. If all
elements of $F$ are algebraic then $F$ is said to be an algebraic extension. If
$E\subseteq F$ is a field extension then $F$ is also a vector space over $E$
(the scalar action is just multiplication in $F$). The dimension of $F$
considered as an $E$-vector space is called the degree of the extension and is
denoted $[F:E]$. If $[F:E]<\infty$ then $F$ is said to be a finite extension
(note that all finite extensions are algebraic). Field extensions are sometimes
denoted $F/E$ for $E\subseteq F$. \end{definition}
Algebraic number theory is in fact the study of algebraic extensions, in
particular, a field extensions $K$ of $\mathbb{Q}$ is said to be a number field
if it is a finite extension of $\mathbb{Q}$.

Since field extensions $F/E$ are always vector spaces, there is a basis $B$ for$$F$ such that any element of $F$ is the linear combination of elements of $B$
with coefficients in $E$. Returning to the previous example, if $K$ is the
smallest field that contains $\mathbb{Q}$ and an algebraic number $\alpha$, then$[K:\mathbb{Q}]$ is the degree of the minimal polynomial $f(x)$ of $\alpha$.

\subsection{Transcendental Extensions}
There is a distinguished type of transcendental extension: those that are
``purely transcendental.'' 
\begin{definition} A field extension $E'/E$ is purely transcendental if it is
obtained by adjoining a set $B$ of algebraically independent elements. A set of
elements is algebraically independent if there is no nonzero polynomial such
that $P(b_1,b_2,\cdots b_n)=0$ for any finite subset of elements.
\end{definition}
\begin{example} The field $\mathbb{Q}(\pi)$ is purely transcendental; in
particular, $\mathbb{Q}(\pi)\cong\mathbb{Q}(x)$ with the isomorphism fixing
$\mathbb{Q}$. \end{example}
Similar to the degree of an algebraic extension, there is a way of keeping track
of the number of algebraically independent generators that are required to
generate a purely transcendental extension.
\begin{definition} Let $E'/E$ be a purely transcendental extension generated by
some set of algebraically independent elements $B$. Then the transcendence
degree $trdeg(E'/E)=#(B)$ and $B$ is called a transcendence basis for $E'/E$.
\end{definition}
In general, let $F/E$ be a field extension, we can always construct an
intermediate extension $F/E'/E$ such that $F/E'$ is algebraic and $E'/E$ is
purely transcendental. Then if $B$ is a transcendence basis for $E'$, it is also
called a transcendence basis for $F$.
\begin{theorem} Let $F/E$ be a field extension, a transcendence basis exists.
\end{theorem}
\begin{proof} Let $A$ be an algebraically independent subset of $F$. Now pick a
subset $G\subseteq F$ that generates $F/E$, we can find a transcendence basis
$B$ such that $A\subseteq B\subseteq G$. Define a collection of algebraically
independent sets $\mathcal{B}$ whose members are subsets of $G$ that contain
$A$. The set can be partially ordered inclusion and contains at least one
element, $A$. The union of elements of $\mathcal{B}$ is algebraically
independent since any algebraic dependence relation would have occured in one of
the elements of $\mathcal{B}$ since the polynomial is only allowed to be over
finitely many variables. The union also satisfies $A\subseteq
\bigcup\mathcal{B}\subseteq G$ so by Zorn's lemma, there is a maximal element
$B\in\mathcal{B}$. Now we claim $F$ is algebraic over $E(B)$. This is because if
it wasn't then there would be a transcendental element $f\in G$ (since $E(G)=F$)
such that $B\cup\{f\}$ wold be algebraically independent contradicting the
maximality of $B$. Thus $B$ is our transcendence basis. \end{proof}


