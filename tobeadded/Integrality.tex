\documentclass[12pt,final,notitlepage,onecolumn]{article}%
\usepackage{amsfonts}
\usepackage{amssymb}
\usepackage{graphicx}
\usepackage{amsmath}
\usepackage{color}%
\setcounter{MaxMatrixCols}{30}
%TCIDATA{OutputFilter=latex2.dll}
%TCIDATA{Version=5.50.0.2960}
%TCIDATA{CSTFile=LaTeX article (bright).cst}
%TCIDATA{Created=Wed Dec 18 14:40:10 2002}
%TCIDATA{LastRevised=Tuesday, January 04, 2011 15:59:07}
%TCIDATA{<META NAME="GraphicsSave" CONTENT="32">}
%TCIDATA{<META NAME="SaveForMode" CONTENT="1">}
%TCIDATA{BibliographyScheme=Manual}
%TCIDATA{<META NAME="DocumentShell" CONTENT="u">}
%BeginMSIPreambleData
\providecommand{\U}[1]{\protect\rule{.1in}{.1in}}
%EndMSIPreambleData
\voffset=-2.5cm
\hoffset=-2.5cm
\setlength\textheight{24cm}
\setlength\textwidth{15.5cm}
\begin{document}
\color{black}

\begin{center}
\textbf{A few facts on integrality *DETAILED VERSION*}

\textit{Darij Grinberg}

Version 6 (30 November 2010)
\end{center}

The purpose of this note is to collect some theorems and proofs related to
integrality in commutative algebra. The note is subdivided into four parts.

Part 1 (Integrality over rings) consists of known facts (Theorems 1, 4, 5) and
a generalized exercise from [1] (Corollary 3) with a few minor variations
(Theorem 2 and Corollary 6).

Part 2 (Integrality over ideal semifiltrations) merges integrality over rings
(as considered in Part 1) and integrality over ideals (a less-known but still
very useful notion; the book [2] is devoted to it) into one general notion -
that of integrality over ideal semifiltrations (Definition 9). This notion is
very general, yet it can be reduced to the basic notion of integrality over
rings by a suitable change of base ring (Theorem 7). This reduction allows to
extend some standard properties of integrality over rings to the general case
(Theorems 8 and 9).

Part 3 (Generalizing to two ideal semifiltrations) continues Part 2, adding
one more layer of generality. Its main result is a "relative" version of
Theorem 7 (Theorem 11) and a known fact generalized one more time (Theorem 13).

Part 4 (Accelerating ideal semifiltrations) generalizes Theorem 11 (and thus
also Theorem 7) a bit further by considering a generalization of powers of an ideal.

Part 5 (Generalizing a lemma by Lombardi) is about an auxiliary result
Lombardi used in [3] to prove Kronecker's Theorem\footnote{\textbf{Kronecker's
Theorem.} Let $B$ be a ring ("ring" always means "commutative ring with unity"
in this paper). Let $g$ and $h$ be two elements of the polynomial ring
$B\left[  X\right]  $. Let $g_{\alpha}$ be any coefficient of the polynomial
$g$. Let $h_{\beta}$ be any coefficient of the polynomial $h$. Let $A$ be a
subring of $B$ which contains all coefficients of the polynomial $gh$. Then,
the element $g_{\alpha}h_{\beta}$ of $B$ is integral over the subring $A$.}.
We extend this auxiliary result here.

This note is supposed to be self-contained (only linear algebra and basic
knowledge about rings, ideals and polynomials is assumed). The proofs are
constructive. However, when writing down the proofs I focussed on maximal
detail (to ensure correctness) rather than on clarity, so the proofs are
probably a pain to read. I think of making a short version of this note with
the obvious parts of proofs left out.

\textit{This is the long version of this paper, with all proofs maximally
detailed. For all practical purposes, the brief version [4] should be totally
enough, and probably better as it is much easier to read.}

\begin{center}
\color{blue} \textbf{Preludium} \color{black}
\end{center}

\textbf{Definitions and notations:}

\textbf{Definition 1.} In the following, "ring" will always mean "commutative
ring with unity". We denote the set $\left\{  0,1,2,...\right\}  $ by
$\mathbb{N}$, and the set $\left\{  1,2,3,...\right\}  $ by $\mathbb{N}^{+}$.

\textbf{Definition 2.} Let $A$ be a ring. Let $M$ be an $A$-module. If
$n\in\mathbb{N}$, and if $m_{1},$ $m_{2},$ $...,$ $m_{n}$ are $n$ elements of
$M$, then we define an $A$-submodule $\left\langle m_{1},m_{2},...,m_{n}%
\right\rangle _{A}$ of $M$ by%
\[
\left\langle m_{1},m_{2},...,m_{n}\right\rangle _{A}=\left\{  \sum
\limits_{i=1}^{n}a_{i}m_{i}\ \mid\ \left(  a_{1},a_{2},...,a_{n}\right)  \in
A^{n}\right\}  .
\]
Also, if $S$ is a finite set, and $m_{s}$ is an element of $M$ for every $s\in
S$, then we define an $A$-submodule $\left\langle m_{s}\ \mid\ s\in
S\right\rangle _{A}$ of $M$ by%
\[
\left\langle m_{s}\ \mid\ s\in S\right\rangle _{A}=\left\{  \sum\limits_{s\in
S}a_{s}m_{s}\ \mid\ \left(  a_{s}\right)  _{s\in S}\in A^{S}\right\}  .
\]
Of course, if $m_{1},$ $m_{2},$ $...,$ $m_{n}$ are $n$ elements of $M$, then
$\left\langle m_{1},m_{2},...,m_{n}\right\rangle _{A}=\left\langle m_{s}%
\ \mid\ s\in\left\{  1,2,...,n\right\}  \right\rangle _{A}$.

We notice something almost trivial:

\begin{quote}
\textbf{Module inclusion lemma.} Let $A$ be a ring. Let $M$ be an $A$-module.
Let $N$ be an $A$-submodule of $M$. If $S$ is a finite set, and $m_{s}$ is an
element of $N$ for every $s\in S$, then $\left\langle m_{s}\ \mid\ s\in
S\right\rangle _{A}\subseteq N$.\ \ \ \ \footnote{\textit{Proof.} We have%
\[
\left\langle m_{s}\ \mid\ s\in S\right\rangle _{A}=\left\{  \sum\limits_{s\in
S}a_{s}m_{s}\ \mid\ \left(  a_{s}\right)  _{s\in S}\in A^{S}\right\}
\subseteq N,
\]
since $\sum\limits_{s\in S}a_{s}m_{s}\ \in N$ for every $\left(  a_{s}\right)
_{s\in S}\in A^{S}$ (because $m_{s}\in N$ for every $s\in S$, and because $N$
is an $A$-module).}
\end{quote}

\textbf{Definition 3.} Let $A$ be a ring, and let $n\in\mathbb{N}$. Let $M$ be
an $A$-module. We say that the $A$-module $M$ is $n$\textit{-generated} if
there exist $n$ elements $m_{1},$ $m_{2},$ $...,$ $m_{n}$ of $M$ such that
$M=\left\langle m_{1},m_{2},...,m_{n}\right\rangle _{A}$. In other words, the
$A$-module $M$ is $n$-generated if and only if there exists a set $S$ and an
element $m_{s}$ of $M$ for every $s\in S$ such that $\left\vert S\right\vert
=n$ and $M=\left\langle m_{s}\ \mid\ s\in S\right\rangle _{A}$.

\textbf{Definition 4.} Let $A$ and $B$ be two rings. We say that $A\subseteq
B$ if and only if
\[
\left(  \text{the set }A\text{ is a subset of the set }B\right)  \text{ and
}\left(  \text{the inclusion map }A\rightarrow B\text{ is a ring
homomorphism}\right)  .
\]


Now assume that $A\subseteq B$. Then, obviously, $B$ is canonically an
$A$-algebra (since $A\subseteq B$). If $u_{1},$ $u_{2},$ $...,$ $u_{n}$ are
$n$ elements of $B$, then we define an $A$-subalgebra $A\left[  u_{1}%
,u_{2},...,u_{n}\right]  $ of $B$ by%
\[
A\left[  u_{1},u_{2},...,u_{n}\right]  =\left\{  P\left(  u_{1},u_{2}%
,...,u_{n}\right)  \ \mid\ P\in A\left[  X_{1},X_{2},...,X_{n}\right]
\right\}  .
\]


In particular, if $u$ is an element of $B$, then the $A$-subalgebra $A\left[
u\right]  $ of $B$ is defined by%
\[
A\left[  u\right]  =\left\{  P\left(  u\right)  \ \mid\ P\in A\left[
X\right]  \right\}  .
\]
Since $A\left[  X\right]  =\left\{  \sum\limits_{i=0}^{m}a_{i}X^{i}%
\ \mid\ m\in\mathbb{N}\text{ and }\left(  a_{0},a_{1},...,a_{m}\right)  \in
A^{m+1}\right\}  $, this becomes
\begin{align*}
A\left[  u\right]   &  =\left\{  \left(  \sum\limits_{i=0}^{m}a_{i}%
X^{i}\right)  \left(  u\right)  \ \mid\ m\in\mathbb{N}\text{ and }\left(
a_{0},a_{1},...,a_{m}\right)  \in A^{m+1}\right\} \\
&  \ \ \ \ \ \ \ \ \ \ \left(  \text{where }\left(  \sum\limits_{i=0}^{m}%
a_{i}X^{i}\right)  \left(  u\right)  \text{ means the polynomial }%
\sum\limits_{i=0}^{m}a_{i}X^{i}\text{ evaluated at }X=u\right) \\
&  =\left\{  \sum\limits_{i=0}^{m}a_{i}u^{i}\ \mid\ m\in\mathbb{N}\text{ and
}\left(  a_{0},a_{1},...,a_{m}\right)  \in A^{m+1}\right\}
\ \ \ \ \ \ \ \ \ \ \left(  \text{because }\left(  \sum\limits_{i=0}^{m}%
a_{i}X^{i}\right)  \left(  u\right)  =\sum\limits_{i=0}^{m}a_{i}u^{i}\right)
.
\end{align*}
Obviously, $uA\left[  u\right]  \subseteq A\left[  u\right]  $ (since
$A\left[  u\right]  $ is an $A$-algebra and $u\in A\left[  u\right]  $).

\begin{center}
\color{blue} \textbf{1. Integrality over rings} \color{black}
\end{center}

\begin{quote}
\textbf{Theorem 1.} Let $A$ and $B$ be two rings such that $A\subseteq B$.
Obviously, $B$ is canonically an $A$-module (since $A\subseteq B$). Let
$n\in\mathbb{N}$. Let $u\in B$. Then, the following four assertions
$\mathcal{A},$ $\mathcal{B},$ $\mathcal{C}$ and $\mathcal{D}$ are pairwise equivalent:

\textit{Assertion }$\mathcal{A}$\textit{:} There exists a monic polynomial
$P\in A\left[  X\right]  $ with $\deg P=n$ and $P\left(  u\right)  =0$.

\textit{Assertion }$\mathcal{B}$\textit{:} There exist a $B$-module $C$ and an
$n$-generated $A$-submodule $U$ of $C$ such that $uU\subseteq U$ and such that
every $v\in B$ satisfying $vU=0$ satisfies $v=0$. (Here, $C$ is an $A$-module,
since $C$ is a $B$-module and $A\subseteq B$.)

\textit{Assertion }$\mathcal{C}$\textit{:} There exists an $n$-generated
$A$-submodule $U$ of $B$ such that $1\in U$ and $uU\subseteq U$.

\textit{Assertion }$\mathcal{D}$\textit{:} We have $A\left[  u\right]
=\left\langle u^{0},u^{1},...,u^{n-1}\right\rangle _{A}$.
\end{quote}

\textbf{Definition 5.} Let $A$ and $B$ be two rings such that $A\subseteq B$.
Let $n\in\mathbb{N}$. Let $u\in B$. We say that the element $u$ of $B$ is
$n$\textit{-integral over }$A$ if it satisfies the four equivalent assertions
$\mathcal{A},$ $\mathcal{B},$ $\mathcal{C}$ and $\mathcal{D}$ of Theorem 1.

Hence, in particular, the element $u$ of $B$ is $n$-integral over $A$ if and
only if it satisfies the assertion $\mathcal{A}$ of Theorem 1. In other words,
$u$ is $n$-integral over $A$ if and only if there exists a monic polynomial
$P\in A\left[  X\right]  $ with $\deg P=n$ and $P\left(  u\right)  =0$.

\textit{Proof of Theorem 1.} We will prove the implications $\mathcal{A}%
\Longrightarrow\mathcal{C}$, $\mathcal{C}\Longrightarrow\mathcal{B}$,
$\mathcal{B}\Longrightarrow\mathcal{A}$, $\mathcal{A}\Longrightarrow
\mathcal{D}$ and $\mathcal{D}\Longrightarrow\mathcal{C}$.

\textit{Proof of the implication }$\mathcal{A}\Longrightarrow\mathcal{C}%
$\textit{.} Assume that Assertion $\mathcal{A}$ holds. Then, there exists a
monic polynomial $P\in A\left[  X\right]  $ with $\deg P=n$ and $P\left(
u\right)  =0$. Since $P\in A\left[  X\right]  $ is a monic polynomial with
$\deg P=n$, there exist elements $a_{0},$ $a_{1},$ $...,$ $a_{n-1}$ of $A$
such that $P\left(  X\right)  =X^{n}+\sum\limits_{k=0}^{n-1}a_{k}X^{k}$. Thus,
$P\left(  u\right)  =u^{n}+\sum\limits_{k=0}^{n-1}a_{k}u^{k}$, so that
$P\left(  u\right)  =0$ becomes $u^{n}+\sum\limits_{k=0}^{n-1}a_{k}u^{k}=0$.
Hence, $u^{n}=-\sum\limits_{k=0}^{n-1}a_{k}u^{k}$.

Let $U$ be the $A$-submodule $\left\langle u^{0},u^{1},...,u^{n-1}%
\right\rangle _{A}$ of $B$. Then, $U$ is an $n$-generated $A$-module (since
$u^{0},$ $u^{1},$ $...,$ $u^{n-1}$ are $n$ elements of $U$). Besides,
$1=u^{0}\in U$.

Now, $u\cdot u^{k}\in U$ for any $k\in\left\{  0,1,...,n-1\right\}  $ (since
$k\in\left\{  0,1,...,n-1\right\}  $ yields either $0\leq k<n-1$ or $k=n-1$,
but $u\cdot u^{k}=u^{k+1}\in\left\langle u^{0},u^{1},...,u^{n-1}\right\rangle
_{A}=U$ if $0\leq k<n-1$, and $u\cdot u^{k}=u\cdot u^{n-1}=u^{n}%
=-\sum\limits_{k=0}^{n-1}a_{k}u^{k}\in\left\langle u^{0},u^{1},...,u^{n-1}%
\right\rangle _{A}=U$ if $k=n-1$, so that $u\cdot u^{k}\in U$ in both cases).
Hence,%
\[
uU=u\left\langle u^{0},u^{1},...,u^{n-1}\right\rangle _{A}=\left\langle u\cdot
u^{0},u\cdot u^{1},...,u\cdot u^{n-1}\right\rangle _{A}\subseteq U
\]
(since $u\cdot u^{k}\in U$ for any $k\in\left\{  0,1,...,n-1\right\}  $).

Thus, Assertion $\mathcal{C}$ holds. Hence, we have proved that $\mathcal{A}%
\Longrightarrow\mathcal{C}$.

\textit{Proof of the implication }$\mathcal{C}\Longrightarrow\mathcal{B}%
$\textit{.} Assume that Assertion $\mathcal{C}$ holds. Then, there exists an
$n$-generated $A$-submodule $U$ of $B$ such that $1\in U$ and $uU\subseteq U$.
Every $v\in B$ satisfying $vU=0$ satisfies $v=0$ (since $1\in U$ and $vU=0$
yield $v\cdot\underbrace{1}_{\in U}\in vU=0$ and thus $v\cdot1=0$, so that
$v=0$). Set $C=B$. Then, $C$ is a $B$-module, and $U$ is an $n$-generated
$A$-submodule of $C$ (since $U$ is an $n$-generated $A$-submodule of $B$, and
$C=B$). Thus, Assertion $\mathcal{B}$ holds. Hence, we have proved that
$\mathcal{C}\Longrightarrow\mathcal{B}$.

\textit{Proof of the implication }$\mathcal{B}\Longrightarrow\mathcal{A}%
$\textit{.} Assume that Assertion $\mathcal{B}$ holds. Then, there exist a
$B$-module $C$ and an $n$-generated $A$-submodule $U$ of $C$ such that
$uU\subseteq U$ (where $C$ is an $A$-module, since $C$ is a $B$-module and
$A\subseteq B$), and such that every $v\in B$ satisfying $vU=0$ satisfies
$v=0$.

Since the $A$-module $U$ is $n$-generated, there exist $n$ elements $m_{1},$
$m_{2},$ $...,$ $m_{n}$ of $U$ such that $U=\left\langle m_{1},m_{2}%
,...,m_{n}\right\rangle _{A}$. For any $k\in\left\{  1,2,...,n\right\}  $, we
have%
\begin{align*}
um_{k}  &  \in uU\ \ \ \ \ \ \ \ \ \ \left(  \text{since }m_{k}\in U\right) \\
&  \subseteq U=\left\langle m_{1},m_{2},...,m_{n}\right\rangle _{A},
\end{align*}
so that there exist $n$ elements $a_{k,1},$ $a_{k,2},$ $...,$ $a_{k,n}$ of $A$
such that $um_{k}=\sum\limits_{i=1}^{n}a_{k,i}m_{i}$.

We introduce two notations:

\begin{itemize}
\item For any matrix $T$ and any integers $x$ and $y$, we denote by $T_{x,y}$
the entry of the matrix $T$ in the $x$-th row and the $y$-th column.

\item For any assertion $\mathcal{U}$, we denote by $\left[  \mathcal{U}%
\right]  $ the Boolean value of the assertion $\mathcal{U}$ (that is, $\left[
\mathcal{U}\right]  =\left\{
\begin{array}
[c]{c}%
1\text{, if }\mathcal{U}\text{ is true;}\\
0\text{, if }\mathcal{U}\text{ is false}%
\end{array}
\right.  $).
\end{itemize}

Clearly, the $n\times n$ identity matrix $I_{n}$ satisfies $\left(
I_{n}\right)  _{\tau,i}=\left[  \tau=i\right]  $ for every $\tau\in\left\{
1,2,...,n\right\}  $ and $i\in\left\{  1,2,...,n\right\}  $.

Note that for every $\tau\in\left\{  1,2,...,n\right\}  $, we have%
\begin{equation}
\sum\limits_{i=1}^{n}\left(  I_{n}\right)  _{\tau,i}m_{i}=m_{\tau}, \label{1}%
\end{equation}
since%
\begin{align*}
\sum\limits_{i=1}^{n}\underbrace{\left(  I_{n}\right)  _{\tau,i}}_{=\left[
\tau=i\right]  =\left[  i=\tau\right]  }m_{i}  &  =\sum\limits_{i=1}%
^{n}\left[  i=\tau\right]  m_{i}=\sum\limits_{i\in\left\{  1,2,...,n\right\}
}\left[  i=\tau\right]  m_{i}\\
&  =\sum\limits_{\substack{i\in\left\{  1,2,...,n\right\}  \\\text{such that
}i=\tau}}\underbrace{\left[  i=\tau\right]  }_{\substack{=1\text{,
since}\\i=\tau\text{ is true}}}m_{i}+\sum\limits_{\substack{i\in\left\{
1,2,...,n\right\}  \\\text{such that }i\neq\tau}}\underbrace{\left[
i=\tau\right]  }_{\substack{=0\text{, since}\\i=\tau\text{ is false,}%
\\\text{since }i\neq\tau}}m_{i}\\
&  =\sum\limits_{\substack{i\in\left\{  1,2,...,n\right\}  \\\text{such that
}i=\tau}}\underbrace{1m_{i}}_{=m_{i}}+\underbrace{\sum\limits_{\substack{i\in
\left\{  1,2,...,n\right\}  \\\text{such that }i\neq\tau}}0m_{i}}_{=0}%
=\sum\limits_{\substack{i\in\left\{  1,2,...,n\right\}  \\\text{such that
}i=\tau}}m_{i}+0\\
&  =\sum\limits_{\substack{i\in\left\{  1,2,...,n\right\}  \\\text{such that
}i=\tau}}m_{i}=\sum\limits_{i\in\left\{  \tau\right\}  }m_{i}%
\ \ \ \ \ \ \ \ \ \ \left(
\begin{array}
[c]{c}%
\text{since }\left\{  i\in\left\{  1,2,...,n\right\}  \ \mid\ i=\tau\right\}
=\left\{  \tau\right\}  ,\\
\text{because }\tau\in\left\{  1,2,...,n\right\}
\end{array}
\right) \\
&  =m_{\tau}.
\end{align*}
Hence, for every $k\in\left\{  1,2,...,n\right\}  $, we have%
\begin{align*}
\sum\limits_{i=1}^{n}\left(  u\left(  I_{n}\right)  _{k,i}-a_{k,i}\right)
m_{i}  &  =\sum\limits_{i=1}^{n}\left(  u\left(  I_{n}\right)  _{k,i}%
m_{i}-a_{k,i}m_{i}\right)  =u\underbrace{\sum\limits_{i=1}^{n}\left(
I_{n}\right)  _{k,i}m_{i}}_{\substack{=m_{k},\text{ by (1)}\\\text{(applied to
}\tau=k\text{)}}}-\sum\limits_{i=1}^{n}a_{k,i}m_{i}\\
&  =um_{k}-\sum\limits_{i=1}^{n}a_{k,i}m_{i}=0
\end{align*}
(since $um_{k}=\sum\limits_{i=1}^{n}a_{k,i}m_{i}$).

Define a matrix $S\in A^{n\times n}$ by $\left(  S_{k,i}=a_{k,i}\text{ for all
}k\in\left\{  1,2,...,n\right\}  \text{ and }i\in\left\{  1,2,...,n\right\}
\right)  $.

Define a matrix $T\in B^{n\times n}$ by $T=\operatorname{adj}\left(
uI_{n}-S\right)  $ (where $S$ is considered as an element of $B^{n\times n}$,
because $S\in A^{n\times n}$ and $A\subseteq B$).

Let $P\in A\left[  X\right]  $ be the characteristic polynomial of the matrix
$S\in A^{n\times n}$. Then, $P$ is monic, and $\deg P=n$. Besides, $P\left(
X\right)  =\det\left(  XI_{n}-S\right)  $, so that $P\left(  u\right)
=\det\left(  uI_{n}-S\right)  $. Then,%
\[
P\left(  u\right)  \cdot I_{n}=\det\left(  uI_{n}-S\right)  \cdot
I_{n}=\underbrace{\operatorname{adj}\left(  uI_{n}-S\right)  }_{=T}%
\cdot\left(  uI_{n}-S\right)  =T\cdot\left(  uI_{n}-S\right)  .
\]


Now, for every $\tau\in\left\{  1,2,...,n\right\}  $, we have%
\begin{align*}
P\left(  u\right)  m_{\tau}  &  =P\left(  u\right)  \sum\limits_{i=1}%
^{n}\left(  I_{n}\right)  _{\tau,i}m_{i}\ \ \ \ \ \ \ \ \ \ \left(
\text{since (1) yields }m_{\tau}=\sum\limits_{i=1}^{n}\left(  I_{n}\right)
_{\tau,i}m_{i}\right) \\
&  =\sum_{i=1}^{n}\underbrace{P\left(  u\right)  \cdot\left(  I_{n}\right)
_{\tau,i}}_{=\left(  P\left(  u\right)  \cdot I_{n}\right)  _{\tau,i}}%
m_{i}=\sum_{i=1}^{n}\left(  \underbrace{P\left(  u\right)  \cdot I_{n}%
}_{=T\cdot\left(  uI_{n}-S\right)  }\right)  _{\tau,i}m_{i}=\sum_{i=1}%
^{n}\underbrace{\left(  T\cdot\left(  uI_{n}-S\right)  \right)  _{\tau,i}%
}_{=\sum\limits_{k=1}^{n}T_{\tau,k}\left(  uI_{n}-S\right)  _{k,i}}m_{i}\\
&  =\sum_{i=1}^{n}\sum\limits_{k=1}^{n}T_{\tau,k}\left(  uI_{n}-S\right)
_{k,i}m_{i}=\sum\limits_{k=1}^{n}T_{\tau,k}\sum_{i=1}^{n}\underbrace{\left(
uI_{n}-S\right)  _{k,i}}_{=u\left(  I_{n}\right)  _{k,i}-S_{k,i}}m_{i}\\
&  =\sum\limits_{k=1}^{n}T_{\tau,k}\sum_{i=1}^{n}\left(  u\left(
I_{n}\right)  _{k,i}-\underbrace{S_{k,i}}_{=a_{k,i}}\right)  m_{i}%
=\sum\limits_{k=1}^{n}T_{\tau,k}\underbrace{\sum_{i=1}^{n}\left(  u\left(
I_{n}\right)  _{k,i}-a_{k,i}\right)  m_{i}}_{=0}=0.
\end{align*}
Thus,%
\begin{align*}
P\left(  u\right)  \cdot U  &  =P\left(  u\right)  \cdot\left\langle
m_{1},m_{2},...,m_{n}\right\rangle _{A}=\left\langle P\left(  u\right)  \cdot
m_{1},P\left(  u\right)  \cdot m_{2},...,P\left(  u\right)  \cdot
m_{n}\right\rangle _{A}\\
&  =\left\langle 0,0,...,0\right\rangle _{A}\ \ \ \ \ \ \ \ \ \ \left(
\text{since }P\left(  u\right)  \cdot m_{\tau}=0\text{ for any }\tau
\in\left\{  1,2,...,n\right\}  \right) \\
&  =0.
\end{align*}
This implies $P\left(  u\right)  =0$ (since every $v\in B$ satisfying $vU=0$
satisfies $v=0$). Thus, Assertion $\mathcal{A}$ holds. Hence, we have proved
that $\mathcal{B}\Longrightarrow\mathcal{A}$.

\textit{Proof of the implication }$\mathcal{A}\Longrightarrow\mathcal{D}%
$\textit{.} Assume that Assertion $\mathcal{A}$ holds. Then, there exists a
monic polynomial $P\in A\left[  X\right]  $ with $\deg P=n$ and $P\left(
u\right)  =0$. Since $P\in A\left[  X\right]  $ is a monic polynomial with
$\deg P=n$, there exist elements $a_{0},$ $a_{1},$ $...,$ $a_{n-1}$ of $A$
such that $P\left(  X\right)  =X^{n}+\sum\limits_{k=0}^{n-1}a_{k}X^{k}$. Thus,
$P\left(  u\right)  =u^{n}+\sum\limits_{k=0}^{n-1}a_{k}u^{k}$, so that
$P\left(  u\right)  =0$ becomes $u^{n}+\sum\limits_{k=0}^{n-1}a_{k}u^{k}=0$.
Hence, $u^{n}=-\sum\limits_{k=0}^{n-1}a_{k}u^{k}$.

Let $U$ be the $A$-submodule $\left\langle u^{0},u^{1},...,u^{n-1}%
\right\rangle _{A}$ of $B$. As in the Proof of the implication $\mathcal{A}%
\Longrightarrow\mathcal{C}$, we can show that $U$ is an $n$-generated
$A$-module, and that $1\in U$ and $uU\subseteq U$.

Now, we are going to show that
\begin{equation}
u^{i}\in U\ \ \ \ \ \ \ \ \ \ \text{for any }i\in\mathbb{N}. \label{2}%
\end{equation}


\textit{Proof of (2).} We will prove (2) by induction over $i$:

\textit{Induction base:} The assertion (2) holds for $i=0$ (since $u^{0}\in
U$). This completes the induction base.

\textit{Induction step:} Let $\tau\in\mathbb{N}$. If the assertion (2) holds
for $i=\tau$, then the assertion (2) holds for $i=\tau+1$ (because if the
assertion (2) holds for $i=\tau$, then $u^{\tau}\in U$, so that $u^{\tau
+1}=u\cdot\underbrace{u^{\tau}}_{\in U}\in uU\subseteq U$, so that $u^{\tau
+1}\in U$, and thus the assertion (2) holds for $i=\tau+1$). This completes
the induction step.

Hence, the induction is complete, and (2) is proven.

Thus,%
\[
A\left[  u\right]  =\left\{  \sum\limits_{i=0}^{m}a_{i}u^{i}\ \mid
\ m\in\mathbb{N}\text{ and }\left(  a_{0},a_{1},...,a_{m}\right)  \in
A^{m+1}\right\}  \subseteq U
\]
(since $\sum\limits_{i=0}^{m}a_{i}u^{i}\in U$ for any $m\in\mathbb{N}$ and any
$\left(  a_{0},a_{1},...,a_{m}\right)  \in A^{m+1}$, because $a_{i}\in A$ and
$u^{i}\in U$ for any $i\in\left\{  0,1,...,m\right\}  $ (by (2)) and $U$ is an
$A$-module). On the other hand, $U\subseteq A\left[  u\right]  $, since%
\begin{align*}
U  &  =\left\langle u^{0},u^{1},...,u^{n-1}\right\rangle _{A}=\left\{
\sum\limits_{i=0}^{n-1}a_{i}u^{i}\ \mid\ \left(  a_{0},a_{1},...,a_{n-1}%
\right)  \in A^{n}\right\} \\
&  \subseteq\left\{  \sum\limits_{i=0}^{m}a_{i}u^{i}\ \mid\ m\in
\mathbb{N}\text{ and }\left(  a_{0},a_{1},...,a_{m}\right)  \in A^{m+1}%
\right\}  =A\left[  u\right]  .
\end{align*}
Thus, $U=A\left[  u\right]  $. In other words, $\left\langle u^{0}%
,u^{1},...,u^{n-1}\right\rangle _{A}=A\left[  u\right]  $. Thus, Assertion
$\mathcal{D}$ holds. Hence, we have proved that $\mathcal{A}\Longrightarrow
\mathcal{D}$.

\textit{Proof of the implication }$\mathcal{D}\Longrightarrow\mathcal{C}%
$\textit{.} Assume that Assertion $\mathcal{D}$ holds. Then, $A\left[
u\right]  =\left\langle u^{0},u^{1},...,u^{n-1}\right\rangle _{A}$.

Let $U$ be the $A$-submodule $\left\langle u^{0},u^{1},...,u^{n-1}%
\right\rangle _{A}$ of $B$. Then, $U$ is an $n$-generated $A$-module (since
$u^{0},$ $u^{1},$ $...,$ $u^{n-1}$ are $n$ elements of $U$). Besides,
$1=u^{0}\in U$.

Also,%
\[
uU=u\cdot\left\langle u^{0},u^{1},...,u^{n-1}\right\rangle _{A}=u\cdot
A\left[  u\right]  \subseteq A\left[  u\right]  =\left\langle u^{0}%
,u^{1},...,u^{n-1}\right\rangle _{A}=U.
\]


Thus, Assertion $\mathcal{C}$ holds. Hence, we have proved that $\mathcal{D}%
\Longrightarrow\mathcal{C}$.

Now, we have proved the implications $\mathcal{A}\Longrightarrow\mathcal{D},$
$\mathcal{D}\Longrightarrow\mathcal{C},$ $\mathcal{C}\Longrightarrow
\mathcal{B}$ and $\mathcal{B}\Longrightarrow\mathcal{A}$ above. Thus, all four
assertions $\mathcal{A},$ $\mathcal{B},$ $\mathcal{C}$ and $\mathcal{D}$ are
pairwise equivalent, and Theorem 1 is proven.

\begin{quote}
\textbf{Theorem 2.} Let $A$ and $B$ be two rings such that $A\subseteq B$. Let
$n\in\mathbb{N}$. Let $v\in B$. Let $a_{0},$ $a_{1},$ $...,$ $a_{n}$ be $n+1$
elements of $A$ such that $\sum\limits_{i=0}^{n}a_{i}v^{i}=0$. Let
$k\in\left\{  0,1,...,n\right\}  $. Then, $\sum\limits_{i=0}^{n-k}a_{i+k}%
v^{i}$ is $n$-integral over $A$.
\end{quote}

\textit{Proof of Theorem 2.} Let $U$ be the $A$-submodule $\left\langle
v^{0},v^{1},...,v^{n-1}\right\rangle _{A}$ of $B$. Then, $U$ is an
$n$-generated $A$-module (since $v^{0},$ $v^{1},$ $...,$ $v^{n-1}$ are $n$
elements of $U$). Besides, $1=v^{0}\in U$.

Let $u=\sum\limits_{i=0}^{n-k}a_{i+k}v^{i}$. Then,%
\begin{align*}
0  &  =\sum\limits_{i=0}^{n}a_{i}v^{i}=\sum\limits_{i=0}^{k-1}a_{i}v^{i}%
+\sum\limits_{i=k}^{n}a_{i}v^{i}=\sum\limits_{i=0}^{k-1}a_{i}v^{i}%
+\sum\limits_{i=0}^{n-k}a_{i+k}\underbrace{v^{i+k}}_{=v^{i}v^{k}}\\
&  \ \ \ \ \ \ \ \ \ \ \left(  \text{here, we substituted }i+k\text{ for
}i\text{\ in the second sum}\right) \\
&  =\sum\limits_{i=0}^{k-1}a_{i}v^{i}+v^{k}\underbrace{\sum\limits_{i=0}%
^{n-k}a_{i+k}v^{i}}_{=u}=\sum\limits_{i=0}^{k-1}a_{i}v^{i}+v^{k}u,
\end{align*}
so that $v^{k}u=-\sum\limits_{i=0}^{k-1}a_{i}v^{i}$.

Now, we are going to show that%
\begin{equation}
uv^{t}\in U\ \ \ \ \ \ \ \ \ \ \text{for any }t\in\left\{
0,1,...,n-1\right\}  . \label{3}%
\end{equation}


\textit{Proof of (3).} Since $t\in\left\{  0,1,...,n-1\right\}  $, one of the
following two cases must hold:

\textit{Case 1:} We have $t\in\left\{  0,1,...,k-1\right\}  $.

\textit{Case 2:} We have $t\in\left\{  k,k+1,...,n-1\right\}  $.

In Case 1, we have%
\begin{align*}
uv^{t}  &  =\sum\limits_{i=0}^{n-k}a_{i+k}\underbrace{v^{i}\cdot v^{t}%
}_{=v^{i+t}}\ \ \ \ \ \ \ \ \ \ \left(  \text{since }u=\sum\limits_{i=0}%
^{n-k}a_{i+k}v^{i}\right) \\
&  =\sum\limits_{i=0}^{n-k}a_{i+k}v^{i+t}\in\left\langle v^{0},v^{1}%
,...,v^{n-1}\right\rangle _{A}\\
&  \ \ \ \ \ \ \ \ \ \ \left(
\begin{array}
[c]{c}%
\text{since }t\in\left\{  0,1,...,k-1\right\}  \text{ yields }i+t\in\left\{
0,1,...,n-1\right\}  \text{ and thus}\\
v^{i+t}\in\left\{  v^{0},v^{1},...,v^{n-1}\right\}  \text{ for any }%
i\in\left\{  0,1,...,n-k\right\}
\end{array}
\right) \\
&  =U.
\end{align*}


In Case 2, we have $t\in\left\{  k,k+1,...,n-1\right\}  $, thus $t-k\in
\left\{  0,1,...,n-k-1\right\}  $ and hence%
\begin{align*}
uv^{t}  &  =u\underbrace{v^{k+\left(  t-k\right)  }}_{=v^{k}v^{t-k}}%
=v^{k}u\cdot v^{t-k}=-\sum\limits_{i=0}^{k-1}a_{i}\underbrace{v^{i}\cdot
v^{t-k}}_{=v^{i+\left(  t-k\right)  }}\ \ \ \ \ \ \ \ \ \ \left(  \text{since
}v^{k}u=-\sum\limits_{i=0}^{k-1}a_{i}v^{i}\right) \\
&  =-\sum\limits_{i=0}^{k-1}a_{i}v^{i+\left(  t-k\right)  }\in\left\langle
v^{0},v^{1},...,v^{n-1}\right\rangle _{A}\\
&  \ \ \ \ \ \ \ \ \ \ \left(
\begin{array}
[c]{c}%
\text{since }t-k\in\left\{  0,1,...,n-k-1\right\}  \text{ yields }i+\left(
t-k\right)  \in\left\{  0,1,...,n-1\right\}  \text{ and thus}\\
v^{i+\left(  t-k\right)  }\in\left\{  v^{0},v^{1},...,v^{n-1}\right\}  \text{
for any }i\in\left\{  0,1,...,k-1\right\}
\end{array}
\right) \\
&  =U.
\end{align*}


Hence, in both cases, we have $uv^{t}\in U$. Thus, $uv^{t}\in U$ always holds,
and (3) is proven.

Now,%
\[
uU=u\left\langle v^{0},v^{1},...,v^{n-1}\right\rangle _{A}=\left\langle
uv^{0},uv^{1},...,uv^{n-1}\right\rangle _{A}\subseteq
U\ \ \ \ \ \ \ \ \ \ \left(  \text{due to (3)}\right)  .
\]


Altogether, $U$ is an $n$-generated $A$-submodule of $B$ such that $1\in U$
and $uU\subseteq U$. Thus, $u\in B$ satisfies Assertion $\mathcal{C}$ of
Theorem 1. Hence, $u\in B$ satisfies the four equivalent assertions
$\mathcal{A},$ $\mathcal{B},$ $\mathcal{C}$ and $\mathcal{D}$ of Theorem 1.
Consequently, $u$ is $n$-integral over $A$. Since $u=\sum\limits_{i=0}%
^{n-k}a_{i+k}v^{i}$, this means that $\sum\limits_{i=0}^{n-k}a_{i+k}v^{i}$ is
$n$-integral over $A$. This proves Theorem 2.

\begin{quote}
\textbf{Corollary 3.} Let $A$ and $B$ be two rings such that $A\subseteq B$.
Let $\alpha\in\mathbb{N}$ and $\beta\in\mathbb{N}$. Let $u\in B$ and $v\in B$.
Let $s_{0},$ $s_{1},$ $...,$ $s_{\alpha}$ be $\alpha+1$ elements of $A$ such
that $\sum\limits_{i=0}^{\alpha}s_{i}v^{i}=u$. Let $t_{0},$ $t_{1},$ $...,$
$t_{\beta}$ be $\beta+1$ elements of $A$ such that $\sum\limits_{i=0}^{\beta
}t_{i}v^{\beta-i}=uv^{\beta}$. Then, $u$ is $\left(  \alpha+\beta\right)
$-integral over $A$.
\end{quote}

(This Corollary 3 generalizes Exercise 2-5 in [1].)

\textit{First proof of Corollary 3.} Let $k=\beta$ and $n=\alpha+\beta$. Then,
$k\in\left\{  0,1,...,n\right\}  $. Define $n+1$ elements $a_{0},$ $a_{1},$
$...,$ $a_{n}$ of $A$ by%
\[
a_{i}=\left\{
\begin{array}
[c]{c}%
t_{\beta-i},\text{ if }i<\beta;\\
t_{0}-s_{0},\text{ if }i=\beta;\\
-s_{i-\beta},\text{ if }i>\beta;
\end{array}
\right.  \ \ \ \ \ \ \ \ \ \ \text{for every }i\in\left\{  0,1,...,n\right\}
.
\]


Then,%
\begin{align*}
\sum\limits_{i=0}^{n}a_{i}v^{i}  &  =\sum\limits_{i=0}^{\alpha+\beta}%
a_{i}v^{i}=\sum\limits_{i=0}^{\beta-1}\underbrace{a_{i}}_{\substack{=t_{\beta
-i}\text{ (by the}\\\text{definition of }a_{i}\text{,}\\\text{since }%
i<\beta\text{)}}}v^{i}+\sum\limits_{i=\beta}^{\beta}\underbrace{a_{i}%
}_{\substack{=t_{0}-s_{0}\text{ (by the}\\\text{definition of }a_{i}%
\text{,}\\\text{since }i=\beta\text{)}}}v^{i}+\sum\limits_{i=\beta+1}%
^{\alpha+\beta}\underbrace{a_{i}}_{\substack{=-s_{i-\beta}\text{ (by
the}\\\text{definition of }a_{i}\text{,}\\\text{since }i>\beta\text{)}}%
}v^{i}\\
&  =\sum\limits_{i=0}^{\beta-1}t_{\beta-i}v^{i}+\underbrace{\sum
\limits_{i=\beta}^{\beta}\left(  t_{0}-s_{0}\right)  v^{i}}%
_{\substack{=\left(  t_{0}-s_{0}\right)  v^{\beta}\\=t_{0}v^{\beta}%
-s_{0}v^{\beta}}}+\underbrace{\sum\limits_{i=\beta+1}^{\alpha+\beta}\left(
-s_{i-\beta}\right)  v^{i}}_{=-\sum\limits_{i=\beta+1}^{\alpha+\beta
}s_{i-\beta}v^{i}}\\
&  =\sum\limits_{i=0}^{\beta-1}t_{\beta-i}v^{i}+t_{0}v^{\beta}-s_{0}v^{\beta
}-\sum\limits_{i=\beta+1}^{\alpha+\beta}s_{i-\beta}v^{i}=\sum\limits_{i=0}%
^{\beta-1}t_{\beta-i}v^{i}+t_{0}v^{\beta}-\left(  s_{0}v^{\beta}%
+\sum\limits_{i=\beta+1}^{\alpha+\beta}s_{i-\beta}v^{i}\right) \\
&  =\sum\limits_{i=0}^{\beta-1}t_{\beta-i}v^{i}+t_{0}v^{\beta}-\left(
s_{0}v^{\beta}+\sum\limits_{i=1}^{\alpha}\underbrace{s_{\left(  i+\beta
\right)  -\beta}}_{=s_{i}}\underbrace{v^{i+\beta}}_{=v^{i}v^{\beta}}\right) \\
&  \ \ \ \ \ \ \ \ \ \ \left(  \text{here, we substituted }i+\beta\text{ for
}i\text{ in the second sum}\right) \\
&  =\sum\limits_{i=0}^{\beta-1}t_{\beta-i}v^{i}+t_{0}v^{\beta}-\left(
s_{0}v^{\beta}+\sum\limits_{i=1}^{\alpha}s_{i}v^{i}v^{\beta}\right) \\
&  =\sum\limits_{i=1}^{\beta}\underbrace{t_{\beta-\left(  \beta-i\right)  }%
}_{=t_{i}}v^{\beta-i}+t_{0}\underbrace{v^{\beta}}_{=v^{\beta-0}}-\left(
s_{0}\underbrace{v^{\beta}}_{=v^{0}v^{\beta}}+\sum\limits_{i=1}^{\alpha}%
s_{i}v^{i}v^{\beta}\right) \\
&  \ \ \ \ \ \ \ \ \ \ \left(  \text{here, we substituted }\beta-i\text{ for
}i\text{ in the first sum}\right) \\
&  =\sum\limits_{i=1}^{\beta}t_{i}v^{\beta-i}+t_{0}v^{\beta-0}-\left(
s_{0}v^{0}v^{\beta}+\sum\limits_{i=1}^{\alpha}s_{i}v^{i}v^{\beta}\right) \\
&  =\underbrace{\sum\limits_{i=1}^{\beta}t_{i}v^{\beta-i}+t_{0}v^{\beta-0}%
}_{=\sum\limits_{i=0}^{\beta}t_{i}v^{\beta-i}=uv^{\beta}}-\left(
\underbrace{s_{0}v^{0}+\sum\limits_{i=1}^{\alpha}s_{i}v^{i}}_{=\sum
\limits_{i=0}^{\alpha}s_{i}v^{i}=u}\right)  v^{\beta}=uv^{\beta}-uv^{\beta}=0.
\end{align*}
Thus, Theorem 2 yields that $\sum\limits_{i=0}^{n-k}a_{i+k}v^{i}$ is
$n$-integral over $A$. But%
\begin{align*}
\sum\limits_{i=0}^{n-k}a_{i+k}v^{i}  &  =\sum\limits_{i=0}^{n-\beta}%
a_{i+\beta}v^{i}=\sum\limits_{i=0}^{0}\underbrace{a_{i+\beta}}%
_{\substack{=t_{0}-s_{0}\text{ (by the}\\\text{definition of }a_{i+\beta
}\text{,}\\\text{since }i=0\text{ yields}\\i+\beta=\beta\text{)}}}v^{i}%
+\sum\limits_{i=1}^{n-\beta}\underbrace{a_{i+\beta}}_{\substack{=-s_{\left(
i+\beta\right)  -\beta}\text{ (by the}\\\text{definition of }a_{i+\beta
}\text{,}\\\text{since }i>0\text{ yields}\\i+\beta>\beta\text{)}}}v^{i}\\
&  =\underbrace{\sum\limits_{i=0}^{0}\left(  t_{0}-s_{0}\right)  v^{i}%
}_{\substack{=\left(  t_{0}-s_{0}\right)  v^{0}\\=t_{0}v^{0}-s_{0}%
v^{0}\\=t_{0}-s_{0}v^{0}}}+\sum\limits_{i=1}^{n-\beta}\left(
-\underbrace{s_{\left(  i+\beta\right)  -\beta}}_{=s_{i}}\right)  v^{i}%
=t_{0}-s_{0}v^{0}+\sum\limits_{i=1}^{n-\beta}\left(  -s_{i}\right)  v^{i}\\
&  =t_{0}-s_{0}v^{0}-\sum\limits_{i=1}^{n-\beta}s_{i}v^{i}=t_{0}-s_{0}%
v^{0}-\sum\limits_{i=1}^{\alpha}s_{i}v^{i}\ \ \ \ \ \ \ \ \ \ \left(
\text{since }n=\alpha+\beta\text{ yields }n-\beta=\alpha\right) \\
&  =t_{0}-\left(  \underbrace{s_{0}v^{0}+\sum\limits_{i=1}^{\alpha}s_{i}v^{i}%
}_{=\sum\limits_{i=0}^{\alpha}s_{i}v^{i}=u}\right)  =t_{0}-u.
\end{align*}
Thus, $t_{0}-u$ is $n$-integral over $A$. On the other hand, $-t_{0}$ is
$1$-integral over $A$ (by Theorem 5 \textbf{(a)} below, applied to $a=-t_{0}%
$). Thus, $\left(  -t_{0}\right)  +\left(  t_{0}-u\right)  $ is $n\cdot
1$-integral over $A$ (by Theorem 5 \textbf{(b)} below, applied to $x=-t_{0}$,
$y=t_{0}-u$ and $m=1$). In other words, $-u$ is $n$-integral over $A$ (since
$\left(  -t_{0}\right)  +\left(  t_{0}-u\right)  =-u$ and $n\cdot1=n$). On the
other hand, $-1$ is $1$-integral over $A$ (by Theorem 5 \textbf{(a)} below,
applied to $a=-1$). Thus, $\left(  -1\right)  \cdot\left(  -u\right)  $ is
$n\cdot1$-integral over $A$ (by Theorem 5 \textbf{(c)} below, applied to
$x=-1$, $y=-u$ and $m=1$). In other words, $u$ is $\left(  \alpha
+\beta\right)  $-integral over $A$ (since $\left(  -1\right)  \cdot\left(
-u\right)  =u$ and $n\cdot1=n=\alpha+\beta$). This proves Corollary 3.

We will provide a second proof of Corollary 3 in Part 5.

\begin{quote}
\textbf{Theorem 4.} Let $A$ and $B$ be two rings such that $A\subseteq B$. Let
$v\in B$ and $u\in B$. Let $m\in\mathbb{N}$ and $n\in\mathbb{N}$. Assume that
$v$ is $m$-integral over $A,$ and that $u$ is $n$-integral over $A\left[
v\right]  $. Then, $u$ is $nm$-integral over $A$.
\end{quote}

\textit{Proof of Theorem 4.} Since $v$ is $m$-integral over $A$, we have
$A\left[  v\right]  =\left\langle v^{0},v^{1},...,v^{m-1}\right\rangle _{A}$
(this is the Assertion $\mathcal{D}$ of Theorem 1, stated for $v$ and $m$ in
lieu of $u$ and $n$).

Since $u$ is $n$-integral over $A\left[  v\right]  $, we have $\left(
A\left[  v\right]  \right)  \left[  u\right]  =\left\langle u^{0}%
,u^{1},...,u^{n-1}\right\rangle _{A\left[  v\right]  }$ (this is the Assertion
$\mathcal{D}$ of Theorem 1, stated for $A\left[  v\right]  $ in lieu of $A$).

Let $S=\left\{  0,1,...,n-1\right\}  \times\left\{  0,1,...,m-1\right\}  $.

Let $x\in\left(  A\left[  v\right]  \right)  \left[  u\right]  $. Then, there
exist $n$ elements $b_{0}$, $b_{1}$, $...$, $b_{n-1}$ of $A\left[  v\right]  $
such that $x=\sum\limits_{i=0}^{n-1}b_{i}u^{i}$ (since $x\in\left(  A\left[
v\right]  \right)  \left[  u\right]  =\left\langle u^{0},u^{1},...,u^{n-1}%
\right\rangle _{A\left[  v\right]  }$). But for each $i\in\left\{
0,1,...,n-1\right\}  $, there exist $m$ elements $a_{i,0},$ $a_{i,1},$ $...,$
$a_{i,m-1}$ of $A$ such that $b_{i}=\sum\limits_{j=0}^{m-1}a_{i,j}v^{j}$
(because $b_{i}\in A\left[  v\right]  =\left\langle v^{0},v^{1},...,v^{m-1}%
\right\rangle _{A}$). Thus,%
\begin{align*}
x  &  =\sum\limits_{i=0}^{n-1}\underbrace{b_{i}}_{=\sum\limits_{j=0}%
^{m-1}a_{i,j}v^{j}}u^{i}=\sum\limits_{i=0}^{n-1}\sum\limits_{j=0}^{m-1}%
a_{i,j}v^{j}u^{i}=\sum\limits_{\left(  i,j\right)  \in\left\{
0,1,...,n-1\right\}  \times\left\{  0,1,...,m-1\right\}  }a_{i,j}v^{j}%
u^{i}=\sum\limits_{\left(  i,j\right)  \in S}a_{i,j}v^{j}u^{i}\\
&  \in\left\langle v^{j}u^{i}\ \mid\ \left(  i,j\right)  \in S\right\rangle
_{A}\ \ \ \ \ \ \ \ \ \ \left(  \text{since }a_{i,j}\in A\text{ for every
}\left(  i,j\right)  \in S\right)
\end{align*}
So we have proved that $x\in\left\langle v^{j}u^{i}\ \mid\ \left(  i,j\right)
\in S\right\rangle _{A}$ for every $x\in\left(  A\left[  v\right]  \right)
\left[  u\right]  $. Thus, $\left(  A\left[  v\right]  \right)  \left[
u\right]  \subseteq\left\langle v^{j}u^{i}\ \mid\ \left(  i,j\right)  \in
S\right\rangle _{A}$. Conversely, $\left\langle v^{j}u^{i}\ \mid\ \left(
i,j\right)  \in S\right\rangle _{A}\subseteq\left(  A\left[  v\right]
\right)  \left[  u\right]  $ (since $v^{j}\in A\left[  v\right]  $ for every
$\left(  i,j\right)  \in S$, and thus $\underbrace{v^{j}}_{\in A\left[
v\right]  }u^{i}\in\left(  A\left[  v\right]  \right)  \left[  u\right]  $ for
every $\left(  i,j\right)  \in S$, and therefore%
\[
\left\langle v^{j}u^{i}\ \mid\ \left(  i,j\right)  \in S\right\rangle
_{A}=\left\{  \underbrace{\sum\limits_{\left(  i,j\right)  \in S}a_{i,j}%
v^{j}u^{i}}_{\substack{\in\left(  A\left[  v\right]  \right)  \left[
u\right]  ,\text{ since}\\v^{j}u^{i}\in\left(  A\left[  v\right]  \right)
\left[  u\right]  \text{ for all }\left(  i,j\right)  \in S\\\text{and
}\left(  A\left[  v\right]  \right)  \left[  u\right]  \text{ is an
}A\text{-module}}}\ \mid\ \left(  a_{i,j}\right)  _{\left(  i,j\right)  \in
S}\in A^{S}\right\}  \subseteq\left(  A\left[  v\right]  \right)  \left[
u\right]
\]
). Hence, $\left(  A\left[  v\right]  \right)  \left[  u\right]  =\left\langle
v^{j}u^{i}\ \mid\ \left(  i,j\right)  \in S\right\rangle _{A}$. Thus, the
$A$-module $\left(  A\left[  v\right]  \right)  \left[  u\right]  $ is
$nm$-generated (since
\[
\left\vert S\right\vert =\left\vert \left\{  0,1,...,n-1\right\}
\times\left\{  0,1,...,m-1\right\}  \right\vert =\underbrace{\left\vert
\left\{  0,1,...,n-1\right\}  \right\vert }_{=n}\cdot\underbrace{\left\vert
\left\{  0,1,...,m-1\right\}  \right\vert }_{=m}=nm
\]
).

Let $U=\left(  A\left[  v\right]  \right)  \left[  u\right]  $. Then, the
$A$-module $U=\left(  A\left[  v\right]  \right)  \left[  u\right]  $ is
$nm$-generated. Besides, $U$ is an $A$-submodule of $B$, and we have
$1=u^{0}\in\left(  A\left[  v\right]  \right)  \left[  u\right]  =U$ and%
\begin{align*}
uU  &  =u\left(  A\left[  v\right]  \right)  \left[  u\right]  \subseteq
\left(  A\left[  v\right]  \right)  \left[  u\right]
\ \ \ \ \ \ \ \ \ \ \left(  \text{since }\left(  A\left[  v\right]  \right)
\left[  u\right]  \text{ is an }A\left[  v\right]  \text{-algebra and }%
u\in\left(  A\left[  v\right]  \right)  \left[  u\right]  \right) \\
&  =U.
\end{align*}


Altogether, we now know that the $A$-submodule $U$ of $B$ is $nm$-generated
and satisfies $1\in U$ and $uU\subseteq U$.

Thus, the element $u$ of $B$ satisfies the Assertion $\mathcal{C}$ of Theorem
1 with $n$ replaced by $nm$. Hence, $u\in B$ satisfies the four equivalent
assertions $\mathcal{A},$ $\mathcal{B},$ $\mathcal{C}$ and $\mathcal{D}$ of
Theorem 1, all with $n$ replaced by $nm$. Thus, $u$ is $nm$-integral over $A$.
This proves Theorem 4.

\begin{quote}
\textbf{Theorem 5.} Let $A$ and $B$ be two rings such that $A\subseteq B$.

\textbf{(a)} Let $a\in A$. Then, $a$ is $1$-integral over $A$.

\textbf{(b)} Let $x\in B$ and $y\in B$. Let $m\in\mathbb{N}$ and
$n\in\mathbb{N}$. Assume that $x$ is $m$-integral over $A,$ and that $y$ is
$n$-integral over $A$. Then, $x+y$ is $nm$-integral over $A$.

\textbf{(c)} Let $x\in B$ and $y\in B$. Let $m\in\mathbb{N}$ and
$n\in\mathbb{N}$. Assume that $x$ is $m$-integral over $A,$ and that $y$ is
$n$-integral over $A$. Then, $xy$ is $nm$-integral over $A$.
\end{quote}

\textit{Proof of Theorem 5.} \textbf{(a)} There exists a monic polynomial
$P\in A\left[  X\right]  $ with $\deg P=1$ and $P\left(  a\right)  =0$
(namely, the polynomial $P\in A\left[  X\right]  $ defined by $P\left(
X\right)  =X-a$). Thus, $a$ is $1$-integral over $A$. This proves Theorem 5
\textbf{(a)}.

\textbf{(b)} Since $y$ is $n$-integral over $A$, there exists a monic
polynomial $P\in A\left[  X\right]  $ with $\deg P=n$ and $P\left(  y\right)
=0$. Since $P\in A\left[  X\right]  $ is a monic polynomial with $\deg P=n$,
there exists a polynomial $\widetilde{P}\in A\left[  X\right]  $ with
$\deg\widetilde{P}<n$ and $P\left(  X\right)  =X^{n}+\widetilde{P}\left(
X\right)  $.

Now, define a polynomial $Q\in\left(  A\left[  x\right]  \right)  \left[
X\right]  $ by $Q\left(  X\right)  =P\left(  X-x\right)  $. Then,%
\begin{align*}
\deg Q  &  =\deg P\ \ \ \ \ \ \ \ \ \ \left(  \text{since shifting the
polynomial }P\text{ by the constant }x\text{ does not change its
degree}\right) \\
&  =n,
\end{align*}
and $Q\left(  X\right)  =P\left(  X-x\right)  $ yields $Q\left(  x+y\right)
=P\left(  \left(  x+y\right)  -x\right)  =P\left(  y\right)  =0$.

Define a polynomial $\widetilde{Q}\in\left(  A\left[  x\right]  \right)
\left[  X\right]  $ by $\widetilde{Q}\left(  X\right)  =\left(  \left(
X-x\right)  ^{n}-X^{n}\right)  +\widetilde{P}\left(  X-x\right)  $. Then,
$\deg\widetilde{Q}<n$ (since%
\begin{align*}
&  \deg\left(  \widetilde{P}\left(  X-x\right)  \right)  =\deg\left(
\widetilde{P}\left(  X\right)  \right) \\
&  \ \ \ \ \ \ \ \ \ \ \left(  \text{since shifting the polynomial
}\widetilde{P}\text{ by the constant }x\text{ does not change its
degree}\right) \\
&  =\deg\widetilde{P}<n
\end{align*}
and%
\begin{align*}
\deg\left(  \left(  X-x\right)  ^{n}-X^{n}\right)   &  =\deg\left(  \left(
\left(  X-x\right)  -X\right)  \cdot\sum\limits_{k=0}^{n-1}\left(  X-x\right)
^{k}X^{n-1-k}\right) \\
&  \leq\underbrace{\deg\left(  \left(  X-x\right)  -X\right)  }_{=\deg\left(
-x\right)  =0}+\underbrace{\deg\left(  \sum\limits_{k=0}^{n-1}\left(
X-x\right)  ^{k}X^{n-1-k}\right)  }_{\substack{\leq n-1,\text{ since}%
\\\deg\left(  \left(  X-x\right)  ^{k}X^{n-1-k}\right)  \leq n-1\\\text{for
any }k\in\left\{  0,1,...,n-1\right\}  }}\\
&  \leq0+\left(  n-1\right)  =n-1<n
\end{align*}
yield%
\begin{align*}
\deg\widetilde{Q}  &  =\deg\left(  \widetilde{Q}\left(  X\right)  \right)
=\deg\left(  \left(  \left(  X-x\right)  ^{n}-X^{n}\right)  +\widetilde{P}%
\left(  X-x\right)  \right) \\
&  \leq\max\left\{  \underbrace{\deg\left(  \left(  X-x\right)  ^{n}%
-X^{n}\right)  }_{<n},\underbrace{\deg\left(  \widetilde{P}\left(  X-x\right)
\right)  }_{<n}\right\}  <\max\left\{  n,n\right\}  =n
\end{align*}
). Thus, the polynomial $Q$ is monic (since%
\begin{align*}
Q\left(  X\right)   &  =P\left(  X-x\right)  =\left(  X-x\right)
^{n}+\widetilde{P}\left(  X-x\right)  \ \ \ \ \ \ \ \ \ \ \left(  \text{since
}P\left(  X\right)  =X^{n}+\widetilde{P}\left(  X\right)  \right) \\
&  =X^{n}+\underbrace{\left(  \left(  X-x\right)  ^{n}-X^{n}\right)
+\widetilde{P}\left(  X-x\right)  }_{=\widetilde{Q}\left(  X\right)  }%
=X^{n}+\widetilde{Q}\left(  X\right)
\end{align*}
and $\deg\widetilde{Q}<n$).

Hence, there exists a monic polynomial $Q\in\left(  A\left[  x\right]
\right)  \left[  X\right]  $ with $\deg Q=n$ and $Q\left(  x+y\right)  =0$.
Thus, $x+y$ is $n$-integral over $A\left[  x\right]  $. Thus, Theorem 4
(applied to $v=x$ and $u=x+y$) yields that $x+y$ is $nm$-integral over $A$.
This proves Theorem 5 \textbf{(b)}.

\textbf{(c)} Since $y$ is $n$-integral over $A$, there exists a monic
polynomial $P\in A\left[  X\right]  $ with $\deg P=n$ and $P\left(  y\right)
=0$. Since $P\in A\left[  X\right]  $ is a monic polynomial with $\deg P=n$,
there exist elements $a_{0},$ $a_{1},$ $...,$ $a_{n-1}$ of $A$ such that
$P\left(  X\right)  =X^{n}+\sum\limits_{k=0}^{n-1}a_{k}X^{k}$. Thus, $P\left(
y\right)  =y^{n}+\sum\limits_{k=0}^{n-1}a_{k}y^{k}$.

Now, define a polynomial $Q\in\left(  A\left[  x\right]  \right)  \left[
X\right]  $ by $Q\left(  X\right)  =X^{n}+\sum\limits_{k=0}^{n-1}x^{n-k}%
a_{k}X^{k}$. Then,%
\begin{align*}
Q\left(  xy\right)   &  =\underbrace{\left(  xy\right)  ^{n}}_{=x^{n}y^{n}%
}+\sum\limits_{k=0}^{n-1}x^{n-k}\underbrace{a_{k}\left(  xy\right)  ^{k}%
}_{\substack{=a_{k}x^{k}y^{k}\\=x^{k}a_{k}y^{k}}}=x^{n}y^{n}+\sum
\limits_{k=0}^{n-1}\underbrace{x^{n-k}x^{k}}_{=x^{n}}a_{k}y^{k}\\
&  =x^{n}y^{n}+\sum\limits_{k=0}^{n-1}x^{n}a_{k}y^{k}=x^{n}\left(
\underbrace{y^{n}+\sum\limits_{k=0}^{n-1}a_{k}y^{k}}_{=P\left(  y\right)
=0}\right)  =0.
\end{align*}
Also, the polynomial $Q\in\left(  A\left[  x\right]  \right)  \left[
X\right]  $ is monic and $\deg Q=n$ (since $Q\left(  X\right)  =X^{n}%
+\sum\limits_{k=0}^{n-1}x^{n-k}a_{k}X^{k}$). Thus, there exists a monic
polynomial $Q\in\left(  A\left[  x\right]  \right)  \left[  X\right]  $ with
$\deg Q=n$ and $Q\left(  xy\right)  =0$. Thus, $xy$ is $n$-integral over
$A\left[  x\right]  $. Hence, Theorem 4 (applied to $v=x$ and $u=xy$) yields
that $xy$ is $nm$-integral over $A$. This proves Theorem 5 \textbf{(c)}.

\begin{quote}
\textbf{Corollary 6.} Let $A$ and $B$ be two rings such that $A\subseteq B$.
Let $n\in\mathbb{N}^{+}$ and $m\in\mathbb{N}$. Let $v\in B$. Let $b_{0},$
$b_{1},$ $...,$ $b_{n-1}$ be $n$ elements of $A$, and let $u=\sum
\limits_{i=0}^{n-1}b_{i}v^{i}$. Assume that $vu$ is $m$-integral over $A$.
Then, $u$ is $nm$-integral over $A$.
\end{quote}

\textit{Proof of Corollary 6.} Define $n+1$ elements $a_{0},$ $a_{1},$ $...,$
$a_{n}$ of $A\left[  vu\right]  $ by%
\[
a_{i}=\left\{
\begin{array}
[c]{c}%
-vu,\text{ if }i=0;\\
b_{i-1},\text{ if }i>0
\end{array}
\right.  \ \ \ \ \ \ \ \ \ \ \text{for every }i\in\left\{  0,1,...,n\right\}
.
\]
Then, $a_{0}=-vu$. Let $k=1$. Then,%
\begin{align*}
\sum\limits_{i=0}^{n}a_{i}v^{i}  &  =\underbrace{a_{0}}_{=-vu}%
\underbrace{v^{0}}_{=1}+\sum\limits_{i=1}^{n}\underbrace{a_{i}}%
_{\substack{=b_{i-1},\\\text{since}\\i>0}}\underbrace{v^{i}}_{=v^{i-1}%
v}=-vu+\sum\limits_{i=1}^{n}b_{i-1}v^{i-1}v=-vu+\underbrace{\sum
\limits_{i=0}^{n-1}b_{i}v^{i}}_{=u}v\\
&  \ \ \ \ \ \ \ \ \ \ \left(  \text{here, we substituted }i\text{ for
}i-1\text{\ in the sum}\right) \\
&  =-vu+uv=0.
\end{align*}


Now, $A\left[  vu\right]  $ and $B$ are two rings such that $A\left[
vu\right]  \subseteq B$. The $n+1$ elements $a_{0},$ $a_{1},$ $...,$ $a_{n}$
of $A\left[  vu\right]  $ satisfy $\sum\limits_{i=0}^{n}a_{i}v^{i}=0$. We have
$k=1\in\left\{  0,1,...,n\right\}  .$

Hence, Theorem 2 (applied to the ring $A\left[  vu\right]  $ in lieu of $A$)
yields that $\sum\limits_{i=0}^{n-k}a_{i+k}v^{i}$ is $n$-integral over
$A\left[  vu\right]  $. But%
\[
\sum\limits_{i=0}^{n-k}a_{i+k}v^{i}=\sum\limits_{i=0}^{n-1}\underbrace{a_{i+1}%
}_{\substack{=b_{\left(  i+1\right)  -1},\\\text{since }i+1>0}}v^{i}%
=\sum\limits_{i=0}^{n-1}b_{\left(  i+1\right)  -1}v^{i}=\sum\limits_{i=0}%
^{n-1}b_{i}v^{i}=u.
\]
Hence, $u$ is $n$-integral over $A\left[  vu\right]  $. But $vu$ is
$m$-integral over $A$. Thus, Theorem 4 (applied to $vu$ in lieu of $v$) yields
that $u$ is $nm$-integral over $A$. This proves Corollary 6.

\begin{center}
\color{blue} \textbf{2. Integrality over ideal semifiltrations} \color{black}
\end{center}

\textbf{Definitions:}

\textbf{Definition 6.} Let $A$ be a ring, and let $\left(  I_{\rho}\right)
_{\rho\in\mathbb{N}}$ be a sequence of ideals of $A$. Then, $\left(  I_{\rho
}\right)  _{\rho\in\mathbb{N}}$ is called an \textit{ideal semifiltration} of
$A$ if and only if it satisfies the two conditions%
\begin{align*}
I_{0}  &  =A;\\
I_{a}I_{b}  &  \subseteq I_{a+b}\ \ \ \ \ \ \ \ \ \ \text{for every }%
a\in\mathbb{N}\text{ and }b\in\mathbb{N}.
\end{align*}


\textbf{Definition 7.} Let $A$ and $B$ be two rings such that $A\subseteq B$.
Then, we identify the polynomial ring $A\left[  Y\right]  $ with a subring of
the polynomial ring $B\left[  Y\right]  $ (in fact, every element of $A\left[
Y\right]  $ has the form $\sum\limits_{i=0}^{m}a_{i}Y^{i}$ for some
$m\in\mathbb{N}$ and $\left(  a_{0},a_{1},...,a_{m}\right)  \in A^{m+1}$, and
thus can be seen as an element of $B\left[  Y\right]  $ by regarding $a_{i}$
as an element of $B$ for every $i\in\left\{  0,1,...,m\right\}  $).

\textbf{Definition 8.} Let $A$ be a ring, and let $\left(  I_{\rho}\right)
_{\rho\in\mathbb{N}}$ be an ideal semifiltration of $A$. Consider the
polynomial ring $A\left[  Y\right]  $. Let $A\left[  \left(  I_{\rho}\right)
_{\rho\in\mathbb{N}}\ast Y\right]  $ denote the $A$-submodule $\sum
\limits_{i\in\mathbb{N}}I_{i}Y^{i}$ of the $A$-algebra $A\left[  Y\right]  $.
Then,%
\begin{align*}
&  A\left[  \left(  I_{\rho}\right)  _{\rho\in\mathbb{N}}\ast Y\right]
=\sum\limits_{i\in\mathbb{N}}I_{i}Y^{i}\\
&  =\left\{  \sum_{i\in\mathbb{N}}a_{i}Y^{i}\ \mid\ \left(  a_{i}\in
I_{i}\text{ for all }i\in\mathbb{N}\right)  \text{, and }\left(  \text{only
finitely many }i\in\mathbb{N}\text{ satisfy }a_{i}\neq0\right)  \right\} \\
&  =\left\{  P\in A\left[  Y\right]  \ \mid\ \text{the }i\text{-th coefficient
of the polynomial }P\text{ lies in }I_{i}\text{ for every }i\in\mathbb{N}%
\right\}  .
\end{align*}


Now, $1\in A\left[  \left(  I_{\rho}\right)  _{\rho\in\mathbb{N}}\ast
Y\right]  $ (because $1=\underbrace{1}_{\in A=I_{0}}\cdot Y^{0}\in I_{0}%
Y^{0}\subseteq\sum\limits_{i\in\mathbb{N}}I_{i}Y^{i}=A\left[  \left(  I_{\rho
}\right)  _{\rho\in\mathbb{N}}\ast Y\right]  $). Also, the $A$-submodule
$A\left[  \left(  I_{\rho}\right)  _{\rho\in\mathbb{N}}\ast Y\right]  $ of
$A\left[  Y\right]  $ is closed under multiplication (since%
\begin{align*}
A\left[  \left(  I_{\rho}\right)  _{\rho\in\mathbb{N}}\ast Y\right]  \cdot
A\left[  \left(  I_{\rho}\right)  _{\rho\in\mathbb{N}}\ast Y\right]   &
=\sum\limits_{i\in\mathbb{N}}I_{i}Y^{i}\cdot\sum\limits_{i\in\mathbb{N}}%
I_{i}Y^{i}=\sum\limits_{i\in\mathbb{N}}I_{i}Y^{i}\cdot\sum\limits_{j\in
\mathbb{N}}I_{j}Y^{j}\\
&  \ \ \ \ \ \ \ \ \ \ \left(  \text{here we renamed }i\text{ as }j\text{ in
the second sum}\right) \\
&  =\sum\limits_{i\in\mathbb{N}}\sum\limits_{j\in\mathbb{N}}I_{i}Y^{i}%
I_{j}Y^{j}=\sum\limits_{i\in\mathbb{N}}\sum\limits_{j\in\mathbb{N}%
}\underbrace{I_{i}I_{j}}_{\substack{\subseteq I_{i+j},\\\text{since }\left(
I_{\rho}\right)  _{\rho\in\mathbb{N}}\\\text{is an ideal}%
\\\text{semifiltration}}}\underbrace{Y^{i}Y^{j}}_{=Y^{i+j}}\\
&  \subseteq\sum\limits_{i\in\mathbb{N}}\sum\limits_{j\in\mathbb{N}}%
I_{i+j}Y^{i+j}\subseteq\sum_{k\in\mathbb{N}}I_{k}Y^{k}=\sum\limits_{i\in
\mathbb{N}}I_{i}Y^{i}\\
&  \ \ \ \ \ \ \ \ \ \ \left(  \text{here we renamed }k\text{ as }i\text{ in
the sum}\right) \\
&  =A\left[  \left(  I_{\rho}\right)  _{\rho\in\mathbb{N}}\ast Y\right]
\end{align*}
). Hence, $A\left[  \left(  I_{\rho}\right)  _{\rho\in\mathbb{N}}\ast
Y\right]  $ is an $A$-subalgebra of the $A$-algebra $A\left[  Y\right]  $.
This $A$-subalgebra $A\left[  \left(  I_{\rho}\right)  _{\rho\in\mathbb{N}%
}\ast Y\right]  $ is called the \textit{Rees algebra} of the ideal
semifiltration $\left(  I_{\rho}\right)  _{\rho\in\mathbb{N}}$.

Clearly, $A\subseteq A\left[  \left(  I_{\rho}\right)  _{\rho\in\mathbb{N}%
}\ast Y\right]  $, since $A\left[  \left(  I_{\rho}\right)  _{\rho
\in\mathbb{N}}\ast Y\right]  =\sum\limits_{i\in\mathbb{N}}I_{i}Y^{i}%
\supseteq\underbrace{I_{0}}_{=A}\underbrace{Y^{0}}_{=1}=A\cdot1=A$.

\textbf{Definition 9.} Let $A$ and $B$ be two rings such that $A\subseteq B$.
Let $\left(  I_{\rho}\right)  _{\rho\in\mathbb{N}}$ be an ideal semifiltration
of $A$. Let $n\in\mathbb{N}$. Let $u\in B$.

We say that the element $u$ of $B$ is $n$\textit{-integral over }$\left(
A,\left(  I_{\rho}\right)  _{\rho\in\mathbb{N}}\right)  $ if there exists some
$\left(  a_{0},a_{1},...,a_{n}\right)  \in A^{n+1}$ such that%
\[
\sum\limits_{k=0}^{n}a_{k}u^{k}=0,\ \ \ \ \ \ \ \ \ \ a_{n}%
=1,\ \ \ \ \ \ \ \ \ \ \text{and}\ \ \ \ \ \ \ \ \ \ a_{i}\in I_{n-i}\text{
for every }i\in\left\{  0,1,...,n\right\}  .
\]


We start with a theorem which reduces the question of $n$-integrality over
$\left(  A,\left(  I_{\rho}\right)  _{\rho\in\mathbb{N}}\right)  $ to that of
$n$-integrality over a ring\footnote{Theorem 7 is inspired by Proposition
5.2.1 in [2].}:

\begin{quote}
\textbf{Theorem 7.} Let $A$ and $B$ be two rings such that $A\subseteq B$. Let
$\left(  I_{\rho}\right)  _{\rho\in\mathbb{N}}$ be an ideal semifiltration of
$A$. Let $n\in\mathbb{N}$. Let $u\in B$.

Consider the polynomial ring $A\left[  Y\right]  $ and its $A$-subalgebra
$A\left[  \left(  I_{\rho}\right)  _{\rho\in\mathbb{N}}\ast Y\right]  $
defined in Definition 8.

Then, the element $u$ of $B$ is $n$-integral over $\left(  A,\left(  I_{\rho
}\right)  _{\rho\in\mathbb{N}}\right)  $ if and only if the element $uY$ of
the polynomial ring $B\left[  Y\right]  $ is $n$-integral over the ring
$A\left[  \left(  I_{\rho}\right)  _{\rho\in\mathbb{N}}\ast Y\right]  .$
(Here, $A\left[  \left(  I_{\rho}\right)  _{\rho\in\mathbb{N}}\ast Y\right]
\subseteq B\left[  Y\right]  $ because $A\left[  \left(  I_{\rho}\right)
_{\rho\in\mathbb{N}}\ast Y\right]  \subseteq A\left[  Y\right]  $ and we
consider $A\left[  Y\right]  $ as a subring of $B\left[  Y\right]  $ as
explained in Definition 7).
\end{quote}

\textit{Proof of Theorem 7.} In order to verify Theorem 7, we have to prove
the following two lemmata:

\textit{Lemma }$\mathcal{E}$\textit{:} If $u$ is $n$-integral over $\left(
A,\left(  I_{\rho}\right)  _{\rho\in\mathbb{N}}\right)  $, then $uY$ is
$n$-integral over $A\left[  \left(  I_{\rho}\right)  _{\rho\in\mathbb{N}}\ast
Y\right]  $.

\textit{Lemma} $\mathcal{F}$\textit{:} If $uY$ is $n$-integral over $A\left[
\left(  I_{\rho}\right)  _{\rho\in\mathbb{N}}\ast Y\right]  $, then $u$ is
$n$-integral over $\left(  A,\left(  I_{\rho}\right)  _{\rho\in\mathbb{N}%
}\right)  $.

\textit{Proof of Lemma }$\mathcal{E}$\textit{:} Assume that $u$ is
$n$-integral over $\left(  A,\left(  I_{\rho}\right)  _{\rho\in\mathbb{N}%
}\right)  $. Then, by Definition 9, there exists some $\left(  a_{0}%
,a_{1},...,a_{n}\right)  \in A^{n+1}$ such that%
\[
\sum\limits_{k=0}^{n}a_{k}u^{k}=0,\ \ \ \ \ \ \ \ \ \ a_{n}%
=1,\ \ \ \ \ \ \ \ \ \ \text{and}\ \ \ \ \ \ \ \ \ \ a_{i}\in I_{n-i}\text{
for every }i\in\left\{  0,1,...,n\right\}  .
\]


Note that $a_{k}Y^{n-k}\in A\left[  \left(  I_{\rho}\right)  _{\rho
\in\mathbb{N}}\ast Y\right]  $ for every $k\in\left\{  0,1,...,n\right\}  $
(because $\underbrace{a_{k}}_{\in I_{n-k}}Y^{n-k}\in I_{n-k}Y^{n-k}%
\subseteq\sum\limits_{i\in\mathbb{N}}I_{i}Y^{i}=A\left[  \left(  I_{\rho
}\right)  _{\rho\in\mathbb{N}}\ast Y\right]  $). Thus, we can define a
polynomial $P\in\left(  A\left[  \left(  I_{\rho}\right)  _{\rho\in\mathbb{N}%
}\ast Y\right]  \right)  \left[  X\right]  $ by $P\left(  X\right)
=\sum\limits_{k=0}^{n}a_{k}Y^{n-k}X^{k}$. This polynomial $P$ satisfies $\deg
P\leq n$, and its coefficient before $X^{n}$ is $\underbrace{a_{n}}%
_{=1}\underbrace{Y^{n-n}}_{=Y^{0}=1}=1$. Hence, this polynomial $P$ is monic
and satisfies $\deg P=n$. Also, $P\left(  X\right)  =\sum\limits_{k=0}%
^{n}a_{k}Y^{n-k}X^{k}$ yields%
\[
P\left(  uY\right)  =\sum\limits_{k=0}^{n}a_{k}Y^{n-k}\left(  uY\right)
^{k}=\sum\limits_{k=0}^{n}a_{k}Y^{n-k}u^{k}Y^{k}=\sum\limits_{k=0}^{n}%
a_{k}u^{k}\underbrace{Y^{n-k}Y^{k}}_{=Y^{n}}=Y^{n}\cdot\underbrace{\sum
\limits_{k=0}^{n}a_{k}u^{k}}_{=0}=0.
\]
Thus, there exists a monic polynomial $P\in\left(  A\left[  \left(  I_{\rho
}\right)  _{\rho\in\mathbb{N}}\ast Y\right]  \right)  \left[  X\right]  $ with
$\deg P=n$ and $P\left(  uY\right)  =0$. Hence, $uY$ is $n$-integral over
$A\left[  \left(  I_{\rho}\right)  _{\rho\in\mathbb{N}}\ast Y\right]  $. This
proves Lemma $\mathcal{E}$.

\textit{Proof of Lemma }$\mathcal{F}$\textit{:} Assume that $uY$ is
$n$-integral over $A\left[  \left(  I_{\rho}\right)  _{\rho\in\mathbb{N}}\ast
Y\right]  $. Then, there exists a monic polynomial $P\in\left(  A\left[
\left(  I_{\rho}\right)  _{\rho\in\mathbb{N}}\ast Y\right]  \right)  \left[
X\right]  $ with $\deg P=n$ and $P\left(  uY\right)  =0$. Since $P\in\left(
A\left[  \left(  I_{\rho}\right)  _{\rho\in\mathbb{N}}\ast Y\right]  \right)
\left[  X\right]  $ satisfies $\deg P=n$, there exists $\left(  p_{0}%
,p_{1},...,p_{n}\right)  \in\left(  A\left[  \left(  I_{\rho}\right)
_{\rho\in\mathbb{N}}\ast Y\right]  \right)  ^{n+1}$ such that $P\left(
X\right)  =\sum\limits_{k=0}^{n}p_{k}X^{k}$. Besides, $p_{n}=1$, since $P$ is
monic and $\deg P=n$.

For every $k\in\left\{  0,1,...,n\right\}  $, we have $p_{k}\in A\left[
\left(  I_{\rho}\right)  _{\rho\in\mathbb{N}}\ast Y\right]  =\sum
\limits_{i\in\mathbb{N}}I_{i}Y^{i}$, and thus, there exists a sequence
$\left(  p_{k,i}\right)  _{i\in\mathbb{N}}\in A^{\mathbb{N}}$ such that
$p_{k}=\sum\limits_{i\in\mathbb{N}}p_{k,i}Y^{i}$, such that $p_{k,i}\in I_{i}$
for every $i\in\mathbb{N}$, and such that only finitely many $i\in\mathbb{N}$
satisfy $p_{k,i}\neq0$. Thus, $P\left(  X\right)  =\sum\limits_{k=0}^{n}%
p_{k}X^{k}$ becomes $P\left(  X\right)  =\sum\limits_{k=0}^{n}\sum
\limits_{i\in\mathbb{N}}p_{k,i}Y^{i}X^{k}$ (since $p_{k}=\sum\limits_{i\in
\mathbb{N}}p_{k,i}Y^{i}$). Hence,
\begin{align*}
P\left(  uY\right)   &  =\sum\limits_{k=0}^{n}\sum\limits_{i\in\mathbb{N}%
}p_{k,i}Y^{i}\underbrace{\left(  uY\right)  ^{k}}_{\substack{=u^{k}%
Y^{k}\\=Y^{k}u^{k}}}=\sum\limits_{k=0}^{n}\sum\limits_{i\in\mathbb{N}}%
p_{k,i}\underbrace{Y^{i}Y^{k}}_{=Y^{i+k}}u^{k}\\
&  =\sum\limits_{k=0}^{n}\sum\limits_{i\in\mathbb{N}}p_{k,i}Y^{i+k}u^{k}%
=\sum\limits_{k\in\left\{  0,1,...,n\right\}  }\sum\limits_{i\in\mathbb{N}%
}p_{k,i}Y^{i+k}u^{k}\\
&  =\sum\limits_{\left(  k,i\right)  \in\left\{  0,1,...,n\right\}
\times\mathbb{N}}p_{k,i}Y^{i+k}u^{k}=\sum_{\ell\in\mathbb{N}}\sum
\limits_{\substack{\left(  k,i\right)  \in\left\{  0,1,...,n\right\}
\times\mathbb{N};\\i+k=\ell}}p_{k,i}\underbrace{Y^{i+k}}_{=Y^{\ell}}u^{k}\\
&  =\sum_{\ell\in\mathbb{N}}\sum\limits_{\substack{\left(  k,i\right)
\in\left\{  0,1,...,n\right\}  \times\mathbb{N};\\i+k=\ell}}p_{k,i}Y^{\ell
}u^{k}=\sum_{\ell\in\mathbb{N}}\sum\limits_{\substack{\left(  k,i\right)
\in\left\{  0,1,...,n\right\}  \times\mathbb{N};\\i+k=\ell}}p_{k,i}%
u^{k}Y^{\ell}.
\end{align*}
Hence, $P\left(  uY\right)  =0$ becomes $\sum\limits_{\ell\in\mathbb{N}}%
\sum\limits_{\substack{\left(  k,i\right)  \in\left\{  0,1,...,n\right\}
\times\mathbb{N};\\i+k=\ell}}p_{k,i}u^{k}Y^{\ell}=0$. In other words, the
polynomial $\sum\limits_{\ell\in\mathbb{N}}\underbrace{\sum
\limits_{\substack{\left(  k,i\right)  \in\left\{  0,1,...,n\right\}
\times\mathbb{N};\\i+k=\ell}}p_{k,i}u^{k}}_{\in B}Y^{\ell}\in B\left[
Y\right]  $ equals $0$. Hence, its coefficient before $Y^{n}$ equals $0$ as
well. But its coefficient before $Y^{n}$ is $\sum\limits_{\substack{\left(
k,i\right)  \in\left\{  0,1,...,n\right\}  \times\mathbb{N};\\i+k=n}%
}p_{k,i}u^{k}$. Hence, $\sum\limits_{\substack{\left(  k,i\right)  \in\left\{
0,1,...,n\right\}  \times\mathbb{N};\\i+k=n}}p_{k,i}u^{k}$ equals $0$.

Thus,%
\begin{align*}
0  &  =\sum\limits_{\substack{\left(  k,i\right)  \in\left\{
0,1,...,n\right\}  \times\mathbb{N};\\i+k=n}}p_{k,i}u^{k}=\sum\limits_{k\in
\left\{  0,1,...,n\right\}  }\sum_{\substack{i\in\mathbb{N};\\i+k=n}%
}p_{k,i}u^{k}=\sum\limits_{k\in\left\{  0,1,...,n\right\}  }p_{k,n-k}u^{k}\\
&  \ \ \ \ \ \ \ \ \ \ \left(
\begin{array}
[c]{c}%
\text{since }\left\{  i\in\mathbb{N}\text{\ }\mid\ i+k=n\right\}  =\left\{
i\in\mathbb{N}\ \mid\ i=n-k\right\}  =\left\{  n-k\right\}  \text{ (because
}n-k\in\mathbb{N}\text{,}\\
\text{since }k\in\left\{  0,1,...,n\right\}  \text{) yields }\sum
\limits_{\substack{i\in\mathbb{N};\\i+k=n}}p_{k,i}u^{k}=\sum\limits_{i\in
\left\{  n-k\right\}  }p_{k,i}u^{k}=p_{k,n-k}u^{k}%
\end{array}
\right)  .
\end{align*}


Note that%
\begin{align*}
\sum\limits_{i\in\mathbb{N}}p_{n,i}Y^{i}  &  =p_{n}\ \ \ \ \ \ \ \ \ \ \left(
\text{since }\sum\limits_{i\in\mathbb{N}}p_{k,i}Y^{i}=p_{k}\text{ for every
}k\in\left\{  0,1,...,n\right\}  \right) \\
&  =1=1\cdot Y^{0}%
\end{align*}
in $A\left[  Y\right]  ,$ and thus the coefficient of the polynomial
$\sum\limits_{i\in\mathbb{N}}p_{n,i}Y^{i}\in A\left[  Y\right]  $ before
$Y^{0}$ is $1;$ but the coefficient of the polynomial $\sum\limits_{i\in
\mathbb{N}}p_{n,i}Y^{i}\in A\left[  Y\right]  $ before $Y^{0}$ is $p_{n,0};$
hence, $p_{n,0}=1$.

Define an $\left(  n+1\right)  $-tuple $\left(  a_{0},a_{1},...,a_{n}\right)
\in A^{n+1}$ by $\left(  a_{k}=p_{k,n-k}\text{ for every }k\in\left\{
0,1,...,n\right\}  \right)  .$ Then, $a_{n}=p_{n,n-n}=p_{n,0}=1$. Besides,%
\[
\sum\limits_{k=0}^{n}a_{k}u^{k}=\sum\limits_{k=0}^{n}p_{k,n-k}u^{k}%
=\sum\limits_{k\in\left\{  0,1,...,n\right\}  }p_{k,n-k}u^{k}=0.
\]
Finally, $a_{k}=p_{k,n-k}\in I_{n-k}$ (since $p_{k,i}\in I_{i}$ for every
$i\in\mathbb{N}$) for every $k\in\left\{  0,1,...,n\right\}  $. In other
words, $a_{i}\in I_{n-i}$ for every $i\in\left\{  0,1,...,n\right\}  $.

Altogether, we now know that%
\[
\sum\limits_{k=0}^{n}a_{k}u^{k}=0,\ \ \ \ \ \ \ \ \ \ a_{n}%
=1,\ \ \ \ \ \ \ \ \ \ \text{and}\ \ \ \ \ \ \ \ \ \ a_{i}\in I_{n-i}\text{
for every }i\in\left\{  0,1,...,n\right\}  .
\]
Thus, by Definition 9, the element $u$ is $n$-integral over $\left(  A,\left(
I_{\rho}\right)  _{\rho\in\mathbb{N}}\right)  $. This proves Lemma
$\mathcal{F}$.

Combining Lemmata $\mathcal{E}$ and $\mathcal{F}$, we obtain that $u$ is
$n$-integral over $\left(  A,\left(  I_{\rho}\right)  _{\rho\in\mathbb{N}%
}\right)  $ if and only if $uY$ is $n$-integral over $A\left[  \left(
I_{\rho}\right)  _{\rho\in\mathbb{N}}\ast Y\right]  $. This proves Theorem 7.

The next theorem is an analogue of Theorem 5 for integrality over ideal semifiltrations:

\begin{quote}
\textbf{Theorem 8.} Let $A$ and $B$ be two rings such that $A\subseteq B$. Let
$\left(  I_{\rho}\right)  _{\rho\in\mathbb{N}}$ be an ideal semifiltration of
$A$.

\textbf{(a)} Let $u\in A$. Then, $u$ is $1$-integral over $\left(  A,\left(
I_{\rho}\right)  _{\rho\in\mathbb{N}}\right)  $ if and only if $u\in I_{1}$.

\textbf{(b)} Let $x\in B$ and $y\in B$. Let $m\in\mathbb{N}$ and
$n\in\mathbb{N}$. Assume that $x$ is $m$-integral over $\left(  A,\left(
I_{\rho}\right)  _{\rho\in\mathbb{N}}\right)  ,$ and that $y$ is $n$-integral
over $\left(  A,\left(  I_{\rho}\right)  _{\rho\in\mathbb{N}}\right)  $. Then,
$x+y$ is $nm$-integral over $\left(  A,\left(  I_{\rho}\right)  _{\rho
\in\mathbb{N}}\right)  $.

\textbf{(c)} Let $x\in B$ and $y\in B$. Let $m\in\mathbb{N}$ and
$n\in\mathbb{N}$. Assume that $x$ is $m$-integral over $\left(  A,\left(
I_{\rho}\right)  _{\rho\in\mathbb{N}}\right)  ,$ and that $y$ is $n$-integral
over $A$. Then, $xy$ is $nm$-integral over $\left(  A,\left(  I_{\rho}\right)
_{\rho\in\mathbb{N}}\right)  $.
\end{quote}

\textit{Proof of Theorem 8.} \textbf{(a)} In order to verify Theorem 8
\textbf{(a)}, we have to prove the following two lemmata:

\textit{Lemma }$\mathcal{G}$\textit{:} If $u$ is $1$-integral over $\left(
A,\left(  I_{\rho}\right)  _{\rho\in\mathbb{N}}\right)  $, then $u\in I_{1}$.

\textit{Lemma} $\mathcal{H}$\textit{:} If $u\in I_{1}$, then $u$ is
$1$-integral over $\left(  A,\left(  I_{\rho}\right)  _{\rho\in\mathbb{N}%
}\right)  $.

\textit{Proof of Lemma }$\mathcal{G}$\textit{:} Assume that $u$ is
$1$-integral over $\left(  A,\left(  I_{\rho}\right)  _{\rho\in\mathbb{N}%
}\right)  $. Then, by Definition 9 (applied to $n=1$), there exists some
$\left(  a_{0},a_{1}\right)  \in A^{2}$ such that%
\[
\sum\limits_{k=0}^{1}a_{k}u^{k}=0,\ \ \ \ \ \ \ \ \ \ a_{1}%
=1,\ \ \ \ \ \ \ \ \ \ \text{and}\ \ \ \ \ \ \ \ \ \ a_{i}\in I_{1-i}\text{
for every }i\in\left\{  0,1\right\}  .
\]
Thus, $a_{0}\in I_{1-0}$ (since $a_{i}\in I_{1-i}$ for every $i\in\left\{
0,1\right\}  $). Also,%
\[
0=\sum\limits_{k=0}^{1}a_{k}u^{k}=a_{0}\underbrace{u^{0}}_{=1}%
+\underbrace{a_{1}}_{=1}\underbrace{u^{1}}_{=u}=a_{0}+u,
\]
so that $u=-\underbrace{a_{0}}_{\in I_{1-0}=I_{1}}\in I_{1}$ (since $I_{1}$ is
an ideal). This proves Lemma $\mathcal{G}$.

\textit{Proof of Lemma }$\mathcal{H}$\textit{:} Assume that $u\in I_{1}$.
Then, $-u\in I_{1}$ (since $I_{1}$ is an ideal). Set $a_{0}=-u$ and $a_{1}=1$.
Then, $\sum\limits_{k=0}^{1}a_{k}u^{k}=\underbrace{a_{0}}_{=-u}%
\underbrace{u^{0}}_{=1}+\underbrace{a_{1}}_{=1}\underbrace{u^{1}}_{=u}%
=-u+u=0$. Also, $a_{i}\in I_{1-i}$ for every $i\in\left\{  0,1\right\}  $
(since $a_{0}=-u\in I_{1}=I_{1-0}$ and $a_{1}=1\in A=I_{0}=I_{1-1}$).
Altogether, we now know that $\left(  a_{0},a_{1}\right)  \in A^{2}$ and%
\[
\sum\limits_{k=0}^{1}a_{k}u^{k}=0,\ \ \ \ \ \ \ \ \ \ a_{1}%
=1,\ \ \ \ \ \ \ \ \ \ \text{and}\ \ \ \ \ \ \ \ \ \ a_{i}\in I_{1-i}\text{
for every }i\in\left\{  0,1\right\}  .
\]
Thus, by Definition 9 (applied to $n=1$), the element $u$ is $1$-integral over
$\left(  A,\left(  I_{\rho}\right)  _{\rho\in\mathbb{N}}\right)  $. This
proves Lemma $\mathcal{H}$.

Combining Lemmata $\mathcal{G}$ and $\mathcal{H}$, we obtain that $u$ is
$1$-integral over $\left(  A,\left(  I_{\rho}\right)  _{\rho\in\mathbb{N}%
}\right)  $ if and only if $u\in I_{1}$. This proves Theorem 8 \textbf{(a)}.

\textbf{(b)} Consider the polynomial ring $A\left[  Y\right]  $ and its
$A$-subalgebra $A\left[  \left(  I_{\rho}\right)  _{\rho\in\mathbb{N}}\ast
Y\right]  $. Theorem 7 (applied to $x$ and $m$ instead of $u$ and $n$) yields
that $xY$ is $m$-integral over $A\left[  \left(  I_{\rho}\right)  _{\rho
\in\mathbb{N}}\ast Y\right]  $ (since $x$ is $m$-integral over $\left(
A,\left(  I_{\rho}\right)  _{\rho\in\mathbb{N}}\right)  $). Also, Theorem 7
(applied to $y$ instead of $u$) yields that $yY$ is $n$-integral over
$A\left[  \left(  I_{\rho}\right)  _{\rho\in\mathbb{N}}\ast Y\right]  $ (since
$y$ is $n$-integral over $\left(  A,\left(  I_{\rho}\right)  _{\rho
\in\mathbb{N}}\right)  $). Hence, Theorem 5 \textbf{(b)} (applied to $A\left[
\left(  I_{\rho}\right)  _{\rho\in\mathbb{N}}\ast Y\right]  ,$ $B\left[
Y\right]  ,$ $xY$ and $yY$ instead of $A,$ $B,$ $x$ and $y$, respectively)
yields that $xY+yY$ is $nm$-integral over $A\left[  \left(  I_{\rho}\right)
_{\rho\in\mathbb{N}}\ast Y\right]  $. Since $xY+yY=\left(  x+y\right)  Y$,
this means that $\left(  x+y\right)  Y$ is $nm$-integral over $A\left[
\left(  I_{\rho}\right)  _{\rho\in\mathbb{N}}\ast Y\right]  $. Hence, Theorem
7 (applied to $x+y$ and $nm$ instead of $u$ and $n$) yields that $x+y$ is
$nm$-integral over $\left(  A,\left(  I_{\rho}\right)  _{\rho\in\mathbb{N}%
}\right)  $. This proves Theorem 8 \textbf{(b)}.

\textbf{(c)} First, a trivial observation:

\textit{Lemma }$\mathcal{I}$\textit{:} Let $A$, $A^{\prime}$ and $B^{\prime}$
be three rings such that $A\subseteq A^{\prime}\subseteq B^{\prime}$. Let
$v\in B^{\prime}$. Let $n\in\mathbb{N}$. If $v$ is $n$-integral over $A$, then
$v$ is $n$-integral over $A^{\prime}$.

\textit{Proof of Lemma }$\mathcal{I}$\textit{:} Assume that $v$ is
$n$-integral over $A$. Then, there exists a monic polynomial $P\in A\left[
X\right]  $ with $\deg P=n$ and $P\left(  v\right)  =0$. Since $A\subseteq
A^{\prime}$, we can identify the polynomial ring $A\left[  X\right]  $ with a
subring of the polynomial ring $A^{\prime}\left[  X\right]  $ (as explained in
Definition 7). Thus, $P\in A\left[  X\right]  $ yields $P\in A^{\prime}\left[
X\right]  $. Hence, there exists a monic polynomial $P\in A^{\prime}\left[
X\right]  $ with $\deg P=n$ and $P\left(  v\right)  =0$. Thus, $v$ is
$n$-integral over $A^{\prime}$. This proves Lemma $\mathcal{I}$.

Now let us prove Theorem 8 \textbf{(c)}.

Consider the polynomial ring $A\left[  Y\right]  $ and its $A$-subalgebra
$A\left[  \left(  I_{\rho}\right)  _{\rho\in\mathbb{N}}\ast Y\right]  $.
Theorem 7 (applied to $x$ and $m$ instead of $u$ and $n$) yields that $xY$ is
$m$-integral over $A\left[  \left(  I_{\rho}\right)  _{\rho\in\mathbb{N}}\ast
Y\right]  $ (since $x$ is $m$-integral over $\left(  A,\left(  I_{\rho
}\right)  _{\rho\in\mathbb{N}}\right)  $). On the other hand, Lemma
$\mathcal{I}$ (applied to $A^{\prime}=A\left[  \left(  I_{\rho}\right)
_{\rho\in\mathbb{N}}\ast Y\right]  $, $B^{\prime}=B\left[  Y\right]  $ and
$v=y$) yields that $y$ is $n$-integral over $A\left[  \left(  I_{\rho}\right)
_{\rho\in\mathbb{N}}\ast Y\right]  $ (since $y$ is $n$-integral over $A$, and
$A\subseteq A\left[  \left(  I_{\rho}\right)  _{\rho\in\mathbb{N}}\ast
Y\right]  \subseteq B\left[  Y\right]  $). Hence, Theorem 5 \textbf{(c)}
(applied to $A\left[  \left(  I_{\rho}\right)  _{\rho\in\mathbb{N}}\ast
Y\right]  ,$ $B\left[  Y\right]  $ and $xY$ instead of $A,$ $B$ and $x$,
respectively) yields that $xY\cdot y$ is $nm$-integral over $A\left[  \left(
I_{\rho}\right)  _{\rho\in\mathbb{N}}\ast Y\right]  $. Since $xY\cdot y=xyY$,
this means that $xyY$ is $nm$-integral over $A\left[  \left(  I_{\rho}\right)
_{\rho\in\mathbb{N}}\ast Y\right]  $. Hence, Theorem 7 (applied to $xy$ and
$nm$ instead of $u$ and $n$) yields that $xy$ is $nm$-integral over $\left(
A,\left(  I_{\rho}\right)  _{\rho\in\mathbb{N}}\right)  $. This proves Theorem
8 \textbf{(c)}.

The next theorem imitates Theorem 4 for integrality over ideal semifiltrations:

\begin{quote}
\textbf{Theorem 9.} Let $A$ and $B$ be two rings such that $A\subseteq B$. Let
$\left(  I_{\rho}\right)  _{\rho\in\mathbb{N}}$ be an ideal semifiltration of
$A$.

Let $v\in B$ and $u\in B$. Let $m\in\mathbb{N}$ and $n\in\mathbb{N}$.

\textbf{(a)} Then, $\left(  I_{\rho}A\left[  v\right]  \right)  _{\rho
\in\mathbb{N}}$ is an ideal semifiltration of $A\left[  v\right]
$.\ \ \ \ \footnote{Here and in the following, whenever $A$ and $B$ are two
rings such that $A\subseteq B$, whenever $v$ is an element of $B$, and
whenever $I$ is an ideal of $A$, you should read the term $IA\left[  v\right]
$ as $I\left(  A\left[  v\right]  \right)  $, not as $\left(  IA\right)
\left[  v\right]  $. For instance, you should read the term $I_{\rho}A\left[
v\right]  $ (in Theorem 9 \textbf{(a)}) as $I_{\rho}\left(  A\left[  v\right]
\right)  $, not as $\left(  I_{\rho}A\right)  \left[  v\right]  $.}

\textbf{(b)} Assume that $v$ is $m$-integral over $A,$ and that $u$ is
$n$-integral over $\left(  A\left[  v\right]  ,\left(  I_{\rho}A\left[
v\right]  \right)  _{\rho\in\mathbb{N}}\right)  $. Then, $u$ is $nm$-integral
over $\left(  A,\left(  I_{\rho}\right)  _{\rho\in\mathbb{N}}\right)  $.
\end{quote}

\textit{Proof of Theorem 9.} \textbf{(a)} More generally:

\textit{Lemma }$\mathcal{J}$\textit{:} Let $A$ and $A^{\prime}$ be two rings
such that $A\subseteq A^{\prime}$. Let $\left(  I_{\rho}\right)  _{\rho
\in\mathbb{N}}$ be an ideal semifiltration of $A$. Then, $\left(  I_{\rho
}A^{\prime}\right)  _{\rho\in\mathbb{N}}$ is an ideal semifiltration of
$A^{\prime}$.

\textit{Proof of Lemma }$\mathcal{J}$\textit{:} Since $\left(  I_{\rho
}\right)  _{\rho\in\mathbb{N}}$ is an ideal semifiltration of $A$, the set
$I_{\rho}$ is an ideal of $A$ for every $\rho\in\mathbb{N}$, and we have%
\begin{align*}
I_{0}  &  =A;\\
I_{a}I_{b}  &  \subseteq I_{a+b}\ \ \ \ \ \ \ \ \ \ \text{for every }%
a\in\mathbb{N}\text{ and }b\in\mathbb{N}.
\end{align*}


Now, the set $I_{\rho}A^{\prime}$ is an ideal of $A^{\prime}$ for every
$\rho\in\mathbb{N}$ (since $I_{\rho}$ is an ideal of $A$). Hence, $\left(
I_{\rho}A^{\prime}\right)  _{\rho\in\mathbb{N}}$ is a sequence of ideals of
$A^{\prime}$. It satisfies%
\begin{align*}
I_{0}A^{\prime}  &  =AA^{\prime}=A^{\prime};\\
I_{a}A^{\prime}\cdot I_{b}A^{\prime}  &  =I_{a}I_{b}A^{\prime}\subseteq
I_{a+b}A^{\prime}\ \left(  \text{since }I_{a}I_{b}\subseteq I_{a+b}\right)
\ \ \ \ \ \ \ \ \ \ \text{for every }a\in\mathbb{N}\text{ and }b\in\mathbb{N}.
\end{align*}
Thus, by Definition 6 (applied to $A^{\prime}$ and $\left(  I_{\rho}A^{\prime
}\right)  _{\rho\in\mathbb{N}}$ instead of $A$ and $\left(  I_{\rho}\right)
_{\rho\in\mathbb{N}}$), it follows that $\left(  I_{\rho}A^{\prime}\right)
_{\rho\in\mathbb{N}}$ is an ideal semifiltration of $A^{\prime}$. This proves
Lemma $\mathcal{J}$.

Now let us prove Theorem 9 \textbf{(a)}. In fact, Lemma $\mathcal{J}$ (applied
to $A^{\prime}=A\left[  v\right]  $) yields that $\left(  I_{\rho}A\left[
v\right]  \right)  _{\rho\in\mathbb{N}}$ is an ideal semifiltration of
$A\left[  v\right]  $. This proves Theorem 9 \textbf{(a)}.

\textbf{(b)} First, we will show a simple fact:

\textit{Lemma }$\mathcal{K}$\textit{:} Let $A$, $A^{\prime}$ and $B^{\prime}$
be three rings such that $A\subseteq A^{\prime}\subseteq B^{\prime}$. Let
$v\in B^{\prime}$. Then, $A^{\prime}\cdot A\left[  v\right]  =A^{\prime
}\left[  v\right]  $.

\textit{Proof of Lemma }$\mathcal{K}$\textit{:} We have $\underbrace{A^{\prime
}}_{\subseteq A^{\prime}\left[  v\right]  }\cdot\underbrace{A\left[  v\right]
}_{\substack{\subseteq A^{\prime}\left[  v\right]  ,\\\text{since }A\subseteq
A^{\prime}}}\subseteq A^{\prime}\left[  v\right]  \cdot A^{\prime}\left[
v\right]  =A^{\prime}\left[  v\right]  $ (since $A^{\prime}\left[  v\right]  $
is a ring). On the other hand, let $x$ be an element of $A^{\prime}\left[
v\right]  $. Then, there exists some $n\in\mathbb{N}$ and some $\left(
a_{0},a_{1},...,a_{n}\right)  \in\left(  A^{\prime}\right)  ^{n+1}$ such that
$x=\sum\limits_{k=0}^{n}a_{k}v^{k}$. Thus,%
\[
x=\sum\limits_{k=0}^{n}\underbrace{a_{k}}_{\in A^{\prime}}\underbrace{v^{k}%
}_{\in A\left[  v\right]  }\in\sum\limits_{k=0}^{n}A^{\prime}\cdot A\left[
v\right]  \subseteq A^{\prime}\cdot A\left[  v\right]
\ \ \ \ \ \ \ \ \ \ \left(  \text{since }A^{\prime}\cdot A\left[  v\right]
\text{ is an additive group}\right)  .
\]
Thus, we have proved that $x\in A^{\prime}\cdot A\left[  v\right]  $ for every
$x\in A^{\prime}\left[  v\right]  $. Therefore, $A^{\prime}\left[  v\right]
\subseteq A^{\prime}\cdot A\left[  v\right]  $. Combined with $A^{\prime}\cdot
A\left[  v\right]  \subseteq A^{\prime}\left[  v\right]  $, this yields
$A^{\prime}\cdot A\left[  v\right]  =A^{\prime}\left[  v\right]  $. Hence, we
have established Lemma $\mathcal{K}$.

Now let us prove Theorem 9 \textbf{(b)}. In fact, consider the polynomial ring
$A\left[  Y\right]  $ and its $A$-subalgebra $A\left[  \left(  I_{\rho
}\right)  _{\rho\in\mathbb{N}}\ast Y\right]  $. We have $A\left[  \left(
I_{\rho}\right)  _{\rho\in\mathbb{N}}\ast Y\right]  \subseteq A\left[
Y\right]  $, and (as explained in Definition 7) we can identify the polynomial
ring $A\left[  Y\right]  $ with a subring of $\left(  A\left[  v\right]
\right)  \left[  Y\right]  $ (since $A\subseteq A\left[  v\right]  $). Hence,
$A\left[  \left(  I_{\rho}\right)  _{\rho\in\mathbb{N}}\ast Y\right]
\subseteq\left(  A\left[  v\right]  \right)  \left[  Y\right]  $. On the other
hand, $\left(  A\left[  v\right]  \right)  \left[  \left(  I_{\rho}A\left[
v\right]  \right)  _{\rho\in\mathbb{N}}\ast Y\right]  \subseteq\left(
A\left[  v\right]  \right)  \left[  Y\right]  $.

Now, we will show that $\left(  A\left[  v\right]  \right)  \left[  \left(
I_{\rho}A\left[  v\right]  \right)  _{\rho\in\mathbb{N}}\ast Y\right]
=\left(  A\left[  \left(  I_{\rho}\right)  _{\rho\in\mathbb{N}}\ast Y\right]
\right)  \left[  v\right]  $.

In fact, Definition 8 yields%
\begin{align*}
\left(  A\left[  v\right]  \right)  \left[  \left(  I_{\rho}A\left[  v\right]
\right)  _{\rho\in\mathbb{N}}\ast Y\right]   &  =\sum\limits_{i\in\mathbb{N}%
}I_{i}A\left[  v\right]  \cdot Y^{i}=\sum\limits_{i\in\mathbb{N}}I_{i}%
Y^{i}\cdot A\left[  v\right]  =A\left[  \left(  I_{\rho}\right)  _{\rho
\in\mathbb{N}}\ast Y\right]  \cdot A\left[  v\right] \\
&  \ \ \ \ \ \ \ \ \ \ \left(  \text{since }\sum\limits_{i\in\mathbb{N}}%
I_{i}Y^{i}=A\left[  \left(  I_{\rho}\right)  _{\rho\in\mathbb{N}}\ast
Y\right]  \right) \\
&  =\left(  A\left[  \left(  I_{\rho}\right)  _{\rho\in\mathbb{N}}\ast
Y\right]  \right)  \left[  v\right]
\end{align*}
(by Lemma $\mathcal{K}$ (applied to $A^{\prime}=A\left[  \left(  I_{\rho
}\right)  _{\rho\in\mathbb{N}}\ast Y\right]  $ and $B^{\prime}=\left(
A\left[  v\right]  \right)  \left[  Y\right]  $)).

Note that (as explained in Definition 7) we can identify the polynomial ring
$\left(  A\left[  v\right]  \right)  \left[  Y\right]  $ with a subring of
$B\left[  Y\right]  $ (since $A\left[  v\right]  \subseteq B$). Thus,
$A\left[  \left(  I_{\rho}\right)  _{\rho\in\mathbb{N}}\ast Y\right]
\subseteq\left(  A\left[  v\right]  \right)  \left[  Y\right]  $ yields
$A\left[  \left(  I_{\rho}\right)  _{\rho\in\mathbb{N}}\ast Y\right]
\subseteq B\left[  Y\right]  $.

Besides, Lemma $\mathcal{I}$ (applied to $A\left[  \left(  I_{\rho}\right)
_{\rho\in\mathbb{N}}\ast Y\right]  $, $B\left[  Y\right]  $ and $m$ instead of
$A^{\prime}$, $B^{\prime}$ and $n$) yields that $v$ is $m$-integral over
$A\left[  \left(  I_{\rho}\right)  _{\rho\in\mathbb{N}}\ast Y\right]  $ (since
$v$ is $m$-integral over $A$, and $A\subseteq A\left[  \left(  I_{\rho
}\right)  _{\rho\in\mathbb{N}}\ast Y\right]  \subseteq B\left[  Y\right]  $).

Now, Theorem 7 (applied to $A\left[  v\right]  $ and $\left(  I_{\rho}A\left[
v\right]  \right)  _{\rho\in\mathbb{N}}$ instead of $A$ and $\left(  I_{\rho
}\right)  _{\rho\in\mathbb{N}}$) yields that $uY$ is $n$-integral over
$\left(  A\left[  v\right]  \right)  \left[  \left(  I_{\rho}A\left[
v\right]  \right)  _{\rho\in\mathbb{N}}\ast Y\right]  $ (since $u$ is
$n$-integral over $\left(  A\left[  v\right]  ,\left(  I_{\rho}A\left[
v\right]  \right)  _{\rho\in\mathbb{N}}\right)  $). Since $\left(  A\left[
v\right]  \right)  \left[  \left(  I_{\rho}A\left[  v\right]  \right)
_{\rho\in\mathbb{N}}\ast Y\right]  =\left(  A\left[  \left(  I_{\rho}\right)
_{\rho\in\mathbb{N}}\ast Y\right]  \right)  \left[  v\right]  $, this means
that $uY$ is $n$-integral over $\left(  A\left[  \left(  I_{\rho}\right)
_{\rho\in\mathbb{N}}\ast Y\right]  \right)  \left[  v\right]  $. Now, Theorem
4 (applied to $A\left[  \left(  I_{\rho}\right)  _{\rho\in\mathbb{N}}\ast
Y\right]  $, $B\left[  Y\right]  $ and $uY$ instead of $A$, $B$ and $u$)
yields that $uY$ is $nm$-integral over $A\left[  \left(  I_{\rho}\right)
_{\rho\in\mathbb{N}}\ast Y\right]  $ (since $v$ is $m$-integral over $A\left[
\left(  I_{\rho}\right)  _{\rho\in\mathbb{N}}\ast Y\right]  $, and $uY$ is
$n$-integral over $\left(  A\left[  \left(  I_{\rho}\right)  _{\rho
\in\mathbb{N}}\ast Y\right]  \right)  \left[  v\right]  $). Thus, Theorem 7
(applied to $nm$ instead of $n$) yields that $u$ is $nm$-integral over
$\left(  A,\left(  I_{\rho}\right)  _{\rho\in\mathbb{N}}\right)  $. This
proves Theorem 9 \textbf{(b)}.

\begin{center}
\color{blue} \textbf{3. Generalizing to two ideal semifiltrations} \color{black}
\end{center}

\begin{quote}
\textbf{Theorem 10.} Let $A$ be a ring.

\textbf{(a)} Then, $\left(  A\right)  _{\rho\in\mathbb{N}}$ is an ideal
semifiltration of $A$.

\textbf{(b)} Let $\left(  I_{\rho}\right)  _{\rho\in\mathbb{N}}$ and $\left(
J_{\rho}\right)  _{\rho\in\mathbb{N}}$ be two ideal semifiltrations of $A$.
Then, $\left(  I_{\rho}J_{\rho}\right)  _{\rho\in\mathbb{N}}$ is an ideal
semifiltration of $A$.
\end{quote}

\textit{Proof of Theorem 10.} \textbf{(a)} Clearly, $\left(  A\right)
_{\rho\in\mathbb{N}}$ is a sequence of ideals of $A$. Hence, in order to prove
that $\left(  A\right)  _{\rho\in\mathbb{N}}$ is an ideal semifiltration of
$A$, it is enough to verify that it satisfies the two conditions%
\begin{align*}
A  &  =A;\\
AA  &  \subseteq A\ \ \ \ \ \ \ \ \ \ \text{for every }a\in\mathbb{N}\text{
and }b\in\mathbb{N}.
\end{align*}
But these two conditions are obviously satisfied. Hence, $\left(  A\right)
_{\rho\in\mathbb{N}}$ is an ideal semifiltration of $A$ (by Definition 6,
applied to $\left(  A\right)  _{\rho\in\mathbb{N}}$ instead of $\left(
I_{\rho}\right)  _{\rho\in\mathbb{N}}$). This proves Theorem 10 \textbf{(a)}.

\textbf{(b)} Since $\left(  I_{\rho}\right)  _{\rho\in\mathbb{N}}$ is an ideal
semifiltration of $A$, it is a sequence of ideals of $A$, and it satisfies the
two conditions%
\begin{align*}
I_{0}  &  =A;\\
I_{a}I_{b}  &  \subseteq I_{a+b}\ \ \ \ \ \ \ \ \ \ \text{for every }%
a\in\mathbb{N}\text{ and }b\in\mathbb{N}%
\end{align*}
(by Definition 6). Since $\left(  J_{\rho}\right)  _{\rho\in\mathbb{N}}$ is an
ideal semifiltration of $A$, it is a sequence of ideals of $A$, and it
satisfies the two conditions%
\begin{align*}
J_{0}  &  =A;\\
J_{a}J_{b}  &  \subseteq J_{a+b}\ \ \ \ \ \ \ \ \ \ \text{for every }%
a\in\mathbb{N}\text{ and }b\in\mathbb{N}%
\end{align*}
(by Definition 6, applied to $\left(  J_{\rho}\right)  _{\rho\in\mathbb{N}}$
instead of $\left(  I_{\rho}\right)  _{\rho\in\mathbb{N}}$).

Now, $I_{\rho}J_{\rho}$ is an ideal of $A$ for every $\rho\in\mathbb{N}$
(since $I_{\rho}$ and $J_{\rho}$ are ideals of $A$ for every $\rho
\in\mathbb{N}$, and the product of any two ideals of $A$ is an ideal of $A$).
Hence, $\left(  I_{\rho}J_{\rho}\right)  _{\rho\in\mathbb{N}}$ is a sequence
of ideals of $A$. Thus, in order to prove that $\left(  I_{\rho}J_{\rho
}\right)  _{\rho\in\mathbb{N}}$ is an ideal semifiltration of $A$, it is
enough to verify that it satisfies the two conditions%
\begin{align*}
I_{0}J_{0}  &  =A;\\
I_{a}J_{a}\cdot I_{b}J_{b}  &  \subseteq I_{a+b}J_{a+b}%
\ \ \ \ \ \ \ \ \ \ \text{for every }a\in\mathbb{N}\text{ and }b\in\mathbb{N}.
\end{align*}
But these two conditions are satisfied, since%
\begin{align*}
\underbrace{I_{0}}_{=A}\underbrace{J_{0}}_{=A}  &  =AA=A;\\
I_{a}J_{a}\cdot I_{b}J_{b}  &  =\underbrace{I_{a}I_{b}}_{\subseteq I_{a+b}%
}\underbrace{J_{a}J_{b}}_{\subseteq J_{a+b}}\subseteq I_{a+b}J_{a+b}%
\ \ \ \ \ \ \ \ \ \ \text{for every }a\in\mathbb{N}\text{ and }b\in\mathbb{N}.
\end{align*}
Hence, $\left(  I_{\rho}J_{\rho}\right)  _{\rho\in\mathbb{N}}$ is an ideal
semifiltration of $A$ (by Definition 6, applied to $\left(  I_{\rho}J_{\rho
}\right)  _{\rho\in\mathbb{N}}$ instead of $\left(  I_{\rho}\right)  _{\rho
\in\mathbb{N}}$). This proves Theorem 10 \textbf{(b)}.

Now let us generalize Theorem 7:

\begin{quote}
\textbf{Theorem 11.} Let $A$ and $B$ be two rings such that $A\subseteq B$.
Let $\left(  I_{\rho}\right)  _{\rho\in\mathbb{N}}$ and $\left(  J_{\rho
}\right)  _{\rho\in\mathbb{N}}$ be two ideal semifiltrations of $A$. Let
$n\in\mathbb{N}$. Let $u\in B$.

We know that $\left(  I_{\rho}J_{\rho}\right)  _{\rho\in\mathbb{N}}$ is an
ideal semifiltration of $A$ (according to Theorem 10 \textbf{(b)}).

Consider the polynomial ring $A\left[  Y\right]  $ and its $A$-subalgebra
$A\left[  \left(  I_{\rho}\right)  _{\rho\in\mathbb{N}}\ast Y\right]  $.

We will abbreviate the ring $A\left[  \left(  I_{\rho}\right)  _{\rho
\in\mathbb{N}}\ast Y\right]  $ by $A_{\left[  I\right]  }$.

By Lemma $\mathcal{J}$ (applied to $A_{\left[  I\right]  }$ and $\left(
J_{\tau}\right)  _{\tau\in\mathbb{N}}$ instead of $A^{\prime}$ and $\left(
I_{\rho}\right)  _{\rho\in\mathbb{N}}$), the sequence $\left(  J_{\tau
}A_{\left[  I\right]  }\right)  _{\tau\in\mathbb{N}}$ is an ideal
semifiltration of $A_{\left[  I\right]  }$ (since $A\subseteq A_{\left[
I\right]  }$ and since $\left(  J_{\tau}\right)  _{\tau\in\mathbb{N}}=\left(
J_{\rho}\right)  _{\rho\in\mathbb{N}}$ is an ideal semifiltration of $A$).

Then, the element $u$ of $B$ is $n$-integral over $\left(  A,\left(  I_{\rho
}J_{\rho}\right)  _{\rho\in\mathbb{N}}\right)  $ if and only if the element
$uY$ of the polynomial ring $B\left[  Y\right]  $ is $n$-integral over
$\left(  A_{\left[  I\right]  },\left(  J_{\tau}A_{\left[  I\right]  }\right)
_{\tau\in\mathbb{N}}\right)  .$ (Here, $A_{\left[  I\right]  }\subseteq
B\left[  Y\right]  $ because $A_{\left[  I\right]  }=A\left[  \left(  I_{\rho
}\right)  _{\rho\in\mathbb{N}}\ast Y\right]  \subseteq A\left[  Y\right]  $
and we consider $A\left[  Y\right]  $ as a subring of $B\left[  Y\right]  $ as
explained in Definition 7.)
\end{quote}

\textit{Proof of Theorem 11.} First, note that%
\begin{align*}
\sum\limits_{\ell\in\mathbb{N}}I_{\ell}Y^{\ell}  &  =\sum\limits_{i\in
\mathbb{N}}I_{i}Y^{i}\ \ \ \ \ \ \ \ \ \ \left(  \text{here we renamed }%
\ell\text{ as }i\text{ in the sum}\right) \\
&  =A\left[  \left(  I_{\rho}\right)  _{\rho\in\mathbb{N}}\ast Y\right]
=A_{\left[  I\right]  }.
\end{align*}


In order to verify Theorem 11, we have to prove the following two lemmata:

\textit{Lemma }$\mathcal{E}^{\prime}$\textit{:} If $u$ is $n$-integral over
$\left(  A,\left(  I_{\rho}J_{\rho}\right)  _{\rho\in\mathbb{N}}\right)  $,
then $uY$ is $n$-integral over $\left(  A_{\left[  I\right]  },\left(
J_{\tau}A_{\left[  I\right]  }\right)  _{\tau\in\mathbb{N}}\right)  $.

\textit{Lemma} $\mathcal{F}^{\prime}$\textit{:} If $uY$ is $n$-integral over
$\left(  A_{\left[  I\right]  },\left(  J_{\tau}A_{\left[  I\right]  }\right)
_{\tau\in\mathbb{N}}\right)  $, then $u$ is $n$-integral over $\left(
A,\left(  I_{\rho}J_{\rho}\right)  _{\rho\in\mathbb{N}}\right)  $.

\textit{Proof of Lemma }$\mathcal{E}^{\prime}$\textit{:} Assume that $u$ is
$n$-integral over $\left(  A,\left(  I_{\rho}J_{\rho}\right)  _{\rho
\in\mathbb{N}}\right)  $. Then, by Definition 9 (applied to $\left(  I_{\rho
}J_{\rho}\right)  _{\rho\in\mathbb{N}}$ instead of $\left(  I_{\rho}\right)
_{\rho\in\mathbb{N}}$), there exists some $\left(  a_{0},a_{1},...,a_{n}%
\right)  \in A^{n+1}$ such that%
\[
\sum\limits_{k=0}^{n}a_{k}u^{k}=0,\ \ \ \ \ \ \ \ \ \ a_{n}%
=1,\ \ \ \ \ \ \ \ \ \ \text{and}\ \ \ \ \ \ \ \ \ \ a_{i}\in I_{n-i}%
J_{n-i}\text{ for every }i\in\left\{  0,1,...,n\right\}  .
\]


Note that $a_{k}Y^{n-k}\in A_{\left[  I\right]  }$ for every $k\in\left\{
0,1,...,n\right\}  $ (because $a_{k}\in I_{n-k}J_{n-k}\subseteq I_{n-k}$
(since $I_{n-k}$ is an ideal of $A$) and thus $a_{k}Y^{n-k}\in I_{n-k}%
Y^{n-k}\subseteq\sum\limits_{i\in\mathbb{N}}I_{i}Y^{i}=A_{\left[  I\right]  }%
$). Thus, we can define an $\left(  n+1\right)  $-tuple $\left(  b_{0}%
,b_{1},...,b_{n}\right)  \in\left(  A_{\left[  I\right]  }\right)  ^{n+1}$ by
$\left(  b_{k}=a_{k}Y^{n-k}\text{ for every }k\in\left\{  0,1,...,n\right\}
\right)  $. Then,%
\begin{align*}
\sum\limits_{k=0}^{n}b_{k}\cdot\left(  uY\right)  ^{k}  &  =\sum
\limits_{k=0}^{n}a_{k}Y^{n-k}\cdot\left(  uY\right)  ^{k}=\sum\limits_{k=0}%
^{n}a_{k}Y^{n-k}u^{k}Y^{k}=\sum\limits_{k=0}^{n}a_{k}u^{k}\underbrace{Y^{n-k}%
Y^{k}}_{=Y^{n}}=Y^{n}\cdot\underbrace{\sum\limits_{k=0}^{n}a_{k}u^{k}}%
_{=0}=0;\\
b_{n}  &  =\underbrace{a_{n}}_{=1}\underbrace{Y^{n-n}}_{=Y^{0}=1}=1,
\end{align*}
and%
\[
b_{i}=\underbrace{a_{i}}_{\substack{\in I_{n-i}J_{n-i}\\=J_{n-i}I_{n-i}%
}}Y^{n-i}\in J_{n-i}\underbrace{I_{n-i}Y^{n-i}}_{\substack{\subseteq
\sum\limits_{\ell\in\mathbb{N}}I_{\ell}Y^{\ell}\\=A_{\left[  I\right]  }%
}}\subseteq J_{n-i}A_{\left[  I\right]  }%
\]
for every $i\in\left\{  0,1,...,n\right\}  $.

Altogether, we now know that $\left(  b_{0},b_{1},...,b_{n}\right)  \in\left(
A_{\left[  I\right]  }\right)  ^{n+1}$ and%
\[
\sum\limits_{k=0}^{n}b_{k}\cdot\left(  uY\right)  ^{k}%
=0,\ \ \ \ \ \ \ \ \ \ b_{n}=1,\ \ \ \ \ \ \ \ \ \ \text{and}%
\ \ \ \ \ \ \ \ \ \ b_{i}\in J_{n-i}A_{\left[  I\right]  }\text{ for every
}i\in\left\{  0,1,...,n\right\}  .
\]
Hence, by Definition 9 (applied to $A_{\left[  I\right]  },$ $B\left[
Y\right]  ,$ $\left(  J_{\tau}A_{\left[  I\right]  }\right)  _{\tau
\in\mathbb{N}},$ $uY$ and $\left(  b_{0},b_{1},...,b_{n}\right)  $ instead of
$A,$ $B,$ $\left(  I_{\rho}\right)  _{\rho\in\mathbb{N}},$ $u$ and $\left(
a_{0},a_{1},...,a_{n}\right)  $), the element $uY$ is $n$-integral over
$\left(  A_{\left[  I\right]  },\left(  J_{\tau}A_{\left[  I\right]  }\right)
_{\tau\in\mathbb{N}}\right)  $. This proves Lemma $\mathcal{E}^{\prime}$.

\textit{Proof of Lemma }$\mathcal{F}^{\prime}$\textit{:} Assume that $uY$ is
$n$-integral over $\left(  A_{\left[  I\right]  },\left(  J_{\tau}A_{\left[
I\right]  }\right)  _{\tau\in\mathbb{N}}\right)  $. Then, by Definition 9
(applied to $A_{\left[  I\right]  },$ $B\left[  Y\right]  ,$ $\left(  J_{\tau
}A_{\left[  I\right]  }\right)  _{\tau\in\mathbb{N}},$ $uY$ and $\left(
p_{0},p_{1},...,p_{n}\right)  $ instead of $A,$ $B,$ $\left(  I_{\rho}\right)
_{\rho\in\mathbb{N}},$ $u$ and $\left(  a_{0},a_{1},...,a_{n}\right)  $),
there exists some $\left(  p_{0},p_{1},...,p_{n}\right)  \in\left(  A_{\left[
I\right]  }\right)  ^{n+1}$ such that%
\[
\sum\limits_{k=0}^{n}p_{k}\cdot\left(  uY\right)  ^{k}%
=0,\ \ \ \ \ \ \ \ \ \ p_{n}=1,\ \ \ \ \ \ \ \ \ \ \text{and}%
\ \ \ \ \ \ \ \ \ \ p_{i}\in J_{n-i}A_{\left[  I\right]  }\text{ for every
}i\in\left\{  0,1,...,n\right\}  .
\]
For every $k\in\left\{  0,1,...,n\right\}  $, we have%
\begin{align*}
p_{k}  &  \in J_{n-k}A_{\left[  I\right]  }=J_{n-k}\sum\limits_{i\in
\mathbb{N}}I_{i}Y^{i}\ \ \ \ \ \ \ \ \ \ \left(  \text{since }A_{\left[
I\right]  }=\sum\limits_{i\in\mathbb{N}}I_{i}Y^{i}\right) \\
&  =\sum\limits_{i\in\mathbb{N}}J_{n-k}I_{i}Y^{i}=\sum\limits_{i\in\mathbb{N}%
}I_{i}J_{n-k}Y^{i},
\end{align*}
and thus, there exists a sequence $\left(  p_{k,i}\right)  _{i\in\mathbb{N}%
}\in A^{\mathbb{N}}$ such that $p_{k}=\sum\limits_{i\in\mathbb{N}}p_{k,i}%
Y^{i}$, such that $p_{k,i}\in I_{i}J_{n-k}$ for every $i\in\mathbb{N}$, and
such that only finitely many $i\in\mathbb{N}$ satisfy $p_{k,i}\neq0$. Thus,%
\begin{align*}
\sum\limits_{k=0}^{n}p_{k}\cdot\left(  uY\right)  ^{k}  &  =\sum
\limits_{k=0}^{n}\sum\limits_{i\in\mathbb{N}}p_{k,i}Y^{i}\cdot
\underbrace{\left(  uY\right)  ^{k}}_{\substack{=u^{k}Y^{k}\\=Y^{k}u^{k}%
}}\ \ \ \ \ \ \ \ \ \ \left(  \text{since }p_{k}=\sum\limits_{i\in\mathbb{N}%
}p_{k,i}Y^{i}\right) \\
&  =\sum\limits_{k=0}^{n}\sum\limits_{i\in\mathbb{N}}p_{k,i}\underbrace{Y^{i}%
\cdot Y^{k}}_{=Y^{i+k}}u^{k}\\
&  =\sum\limits_{k=0}^{n}\sum\limits_{i\in\mathbb{N}}p_{k,i}Y^{i+k}u^{k}%
=\sum\limits_{k\in\left\{  0,1,...,n\right\}  }\sum\limits_{i\in\mathbb{N}%
}p_{k,i}Y^{i+k}u^{k}\\
&  =\sum\limits_{\left(  k,i\right)  \in\left\{  0,1,...,n\right\}
\times\mathbb{N}}p_{k,i}Y^{i+k}u^{k}=\sum_{\ell\in\mathbb{N}}\sum
\limits_{\substack{\left(  k,i\right)  \in\left\{  0,1,...,n\right\}
\times\mathbb{N};\\i+k=\ell}}p_{k,i}\underbrace{Y^{i+k}}_{=Y^{\ell}}u^{k}\\
&  =\sum_{\ell\in\mathbb{N}}\sum\limits_{\substack{\left(  k,i\right)
\in\left\{  0,1,...,n\right\}  \times\mathbb{N};\\i+k=\ell}}p_{k,i}Y^{\ell
}u^{k}=\sum_{\ell\in\mathbb{N}}\sum\limits_{\substack{\left(  k,i\right)
\in\left\{  0,1,...,n\right\}  \times\mathbb{N};\\i+k=\ell}}p_{k,i}%
u^{k}Y^{\ell}.
\end{align*}
Hence, $\sum\limits_{k=0}^{n}p_{k}\cdot\left(  uY\right)  ^{k}=0$ becomes
$\sum\limits_{\ell\in\mathbb{N}}\sum\limits_{\substack{\left(  k,i\right)
\in\left\{  0,1,...,n\right\}  \times\mathbb{N};\\i+k=\ell}}p_{k,i}%
u^{k}Y^{\ell}=0$. In other words, the polynomial $\sum\limits_{\ell
\in\mathbb{N}}\underbrace{\sum\limits_{\substack{\left(  k,i\right)
\in\left\{  0,1,...,n\right\}  \times\mathbb{N};\\i+k=\ell}}p_{k,i}u^{k}}_{\in
B}Y^{\ell}\in B\left[  Y\right]  $ equals $0$. Hence, its coefficient before
$Y^{n}$ equals $0$ as well. But its coefficient before $Y^{n}$ is
$\sum\limits_{\substack{\left(  k,i\right)  \in\left\{  0,1,...,n\right\}
\times\mathbb{N};\\i+k=n}}p_{k,i}u^{k}$. Hence, $\sum
\limits_{\substack{\left(  k,i\right)  \in\left\{  0,1,...,n\right\}
\times\mathbb{N};\\i+k=n}}p_{k,i}u^{k}$ equals $0$.

Thus,%
\begin{align*}
0  &  =\sum\limits_{\substack{\left(  k,i\right)  \in\left\{
0,1,...,n\right\}  \times\mathbb{N};\\i+k=n}}p_{k,i}u^{k}=\sum\limits_{k\in
\left\{  0,1,...,n\right\}  }\sum_{\substack{i\in\mathbb{N};\\i+k=n}%
}p_{k,i}u^{k}=\sum\limits_{k\in\left\{  0,1,...,n\right\}  }p_{k,n-k}u^{k}\\
&  \ \ \ \ \ \ \ \ \ \ \left(
\begin{array}
[c]{c}%
\text{since }\left\{  i\in\mathbb{N}\text{\ }\mid\ i+k=n\right\}  =\left\{
i\in\mathbb{N}\ \mid\ i=n-k\right\}  =\left\{  n-k\right\}  \text{ (because
}n-k\in\mathbb{N}\text{,}\\
\text{since }k\in\left\{  0,1,...,n\right\}  \text{) yields }\sum
\limits_{\substack{i\in\mathbb{N};\\i+k=n}}p_{k,i}u^{k}=\sum\limits_{i\in
\left\{  n-k\right\}  }p_{k,i}u^{k}=p_{k,n-k}u^{k}%
\end{array}
\right)  .
\end{align*}


Note that%
\begin{align*}
\sum\limits_{i\in\mathbb{N}}p_{n,i}Y^{i}  &  =p_{n}\ \ \ \ \ \ \ \ \ \ \left(
\text{since }\sum\limits_{i\in\mathbb{N}}p_{k,i}Y^{i}=p_{k}\text{ for every
}k\in\left\{  0,1,...,n\right\}  \right) \\
&  =1=1\cdot Y^{0}%
\end{align*}
in $A\left[  Y\right]  ,$ and thus the coefficient of the polynomial
$\sum\limits_{i\in\mathbb{N}}p_{n,i}Y^{i}\in A\left[  Y\right]  $ before
$Y^{0}$ is $1;$ but the coefficient of the polynomial $\sum\limits_{i\in
\mathbb{N}}p_{n,i}Y^{i}\in A\left[  Y\right]  $ before $Y^{0}$ is $p_{n,0};$
hence, $p_{n,0}=1$.

Define an $\left(  n+1\right)  $-tuple $\left(  a_{0},a_{1},...,a_{n}\right)
\in A^{n+1}$ by $\left(  a_{k}=p_{k,n-k}\text{ for every }k\in\left\{
0,1,...,n\right\}  \right)  .$ Then, $a_{n}=p_{n,n-n}=p_{n,0}=1$. Besides,%
\[
\sum\limits_{k=0}^{n}a_{k}u^{k}=\sum\limits_{k=0}^{n}p_{k,n-k}u^{k}%
=\sum\limits_{k\in\left\{  0,1,...,n\right\}  }p_{k,n-k}u^{k}=0.
\]
Finally, $a_{k}=p_{k,n-k}\in I_{n-k}J_{n-k}$ (since $p_{k,i}\in I_{i}J_{n-k}$
for every $i\in\mathbb{N}$) for every $k\in\left\{  0,1,...,n\right\}  $. In
other words, $a_{i}\in I_{n-i}J_{n-i}$ for every $i\in\left\{
0,1,...,n\right\}  $.

Altogether, we now know that%
\[
\sum\limits_{k=0}^{n}a_{k}u^{k}=0,\ \ \ \ \ \ \ \ \ \ a_{n}%
=1,\ \ \ \ \ \ \ \ \ \ \text{and}\ \ \ \ \ \ \ \ \ \ a_{i}\in I_{n-i}%
J_{n-i}\text{ for every }i\in\left\{  0,1,...,n\right\}  .
\]
Thus, by Definition 9 (applied to $\left(  I_{\rho}J_{\rho}\right)  _{\rho
\in\mathbb{N}}$ instead of $\left(  I_{\rho}\right)  _{\rho\in\mathbb{N}}$),
the element $u$ is $n$-integral over $\left(  A,\left(  I_{\rho}J_{\rho
}\right)  _{\rho\in\mathbb{N}}\right)  $. This proves Lemma $\mathcal{F}%
^{\prime}$.

Combining Lemmata $\mathcal{E}^{\prime}$ and $\mathcal{F}^{\prime}$, we obtain
that $u$ is $n$-integral over $\left(  A,\left(  I_{\rho}J_{\rho}\right)
_{\rho\in\mathbb{N}}\right)  $ if and only if $uY$ is $n$-integral over
$\left(  A_{\left[  I\right]  },\left(  J_{\tau}A_{\left[  I\right]  }\right)
_{\tau\in\mathbb{N}}\right)  $. This proves Theorem 11.

For the sake of completeness, we mention the following trivial fact (which
shows why Theorem 11 generalizes Theorem 7):

\begin{quote}
\textbf{Theorem 12.} Let $A$ and $B$ be two rings such that $A\subseteq B$.
Let $n\in\mathbb{N}$. Let $u\in B$.

We know that $\left(  A\right)  _{\rho\in\mathbb{N}}$ is an ideal
semifiltration of $A$ (according to Theorem 10 \textbf{(a)}).

Then, the element $u$ of $B$ is $n$-integral over $\left(  A,\left(  A\right)
_{\rho\in\mathbb{N}}\right)  $ if and only if $u$ is $n$-integral over $A$.
\end{quote}

\textit{Proof of Theorem 12.} In order to verify Theorem 12, we have to prove
the following two lemmata:

\textit{Lemma }$\mathcal{L}$\textit{:} If $u$ is $n$-integral over $\left(
A,\left(  A\right)  _{\rho\in\mathbb{N}}\right)  $, then $u$ is $n$-integral
over $A$.

\textit{Lemma} $\mathcal{M}$\textit{:} If $u$ is $n$-integral over $A$, then
$u$ is $n$-integral over $\left(  A,\left(  A\right)  _{\rho\in\mathbb{N}%
}\right)  $.

\textit{Proof of Lemma }$\mathcal{L}$\textit{:} Assume that $u$ is
$n$-integral over $\left(  A,\left(  A\right)  _{\rho\in\mathbb{N}}\right)  $.
Then, by Definition 9 (applied to $\left(  A\right)  _{\rho\in\mathbb{N}}$
instead of $\left(  I_{\rho}\right)  _{\rho\in\mathbb{N}}$), there exists some
$\left(  a_{0},a_{1},...,a_{n}\right)  \in A^{n+1}$ such that%
\[
\sum\limits_{k=0}^{n}a_{k}u^{k}=0,\ \ \ \ \ \ \ \ \ \ a_{n}%
=1,\ \ \ \ \ \ \ \ \ \ \text{and}\ \ \ \ \ \ \ \ \ \ a_{i}\in A\text{ for
every }i\in\left\{  0,1,...,n\right\}  .
\]


Define a polynomial $P\in A\left[  X\right]  $ by $P\left(  X\right)
=\sum\limits_{k=0}^{n}a_{k}X^{k}$. Then, $P\left(  X\right)  =\sum
\limits_{k=0}^{n}a_{k}X^{k}=\underbrace{a_{n}}_{=1}X^{n}+\sum\limits_{k=0}%
^{n-1}a_{k}X^{k}=X^{n}+\sum\limits_{k=0}^{n-1}a_{k}X^{k}$. Hence, the
polynomial $P$ is monic, and $\deg P=n$. Besides, $P\left(  u\right)  =0$
(since $P\left(  X\right)  =\sum\limits_{k=0}^{n}a_{k}X^{k}$ yields $P\left(
u\right)  =\sum\limits_{k=0}^{n}a_{k}u^{k}=0$). Thus, there exists a monic
polynomial $P\in A\left[  X\right]  $ with $\deg P=n$ and $P\left(  u\right)
=0$. Hence, $u$ is $n$-integral over $A$. This proves Lemma $\mathcal{L}$.

\textit{Proof of Lemma }$\mathcal{M}$\textit{:} Assume that $u$ is
$n$-integral over $A$. Then, there exists a monic polynomial $P\in A\left[
X\right]  $ with $\deg P=n$ and $P\left(  u\right)  =0$. Since $\deg P=n$,
there exists some $\left(  n+1\right)  $-tuple $\left(  a_{0},a_{1}%
,...,a_{n}\right)  \in A^{n+1}$ such that $P\left(  X\right)  =\sum
\limits_{k=0}^{n}a_{k}X^{k}$. Thus, $a_{n}=1$ (since $P$ is monic, and $\deg
P=n$). Also, $\sum\limits_{k=0}^{n}a_{k}X^{k}=P\left(  X\right)  $ yields
$\sum\limits_{k=0}^{n}a_{k}u^{k}=P\left(  u\right)  =0$. Altogether, we now
know that $\left(  a_{0},a_{1},...,a_{n}\right)  \in A^{n+1}$ and%
\[
\sum\limits_{k=0}^{n}a_{k}u^{k}=0,\ \ \ \ \ \ \ \ \ \ a_{n}%
=1,\ \ \ \ \ \ \ \ \ \ \text{and}\ \ \ \ \ \ \ \ \ \ a_{i}\in A\text{ for
every }i\in\left\{  0,1,...,n\right\}  .
\]
Hence, by Definition 9 (applied to $\left(  A\right)  _{\rho\in\mathbb{N}}$
instead of $\left(  I_{\rho}\right)  _{\rho\in\mathbb{N}}$), the element $u$
is $n$-integral over $\left(  A,\left(  A\right)  _{\rho\in\mathbb{N}}\right)
$. This proves Lemma $\mathcal{M}$.

Combining Lemmata $\mathcal{L}$ and $\mathcal{M}$, we obtain that $u$ is
$n$-integral over $\left(  A,\left(  A\right)  _{\rho\in\mathbb{N}}\right)  $
if and only if $u$ is $n$-integral over $A$. This proves Theorem 12.

Finally, let us generalize Theorem 8 \textbf{(c)}:

\begin{quote}
\textbf{Theorem 13.} Let $A$ and $B$ be two rings such that $A\subseteq B$.
Let $\left(  I_{\rho}\right)  _{\rho\in\mathbb{N}}$ and $\left(  J_{\rho
}\right)  _{\rho\in\mathbb{N}}$ be two ideal semifiltrations of $A$.

Let $x\in B$ and $y\in B$. Let $m\in\mathbb{N}$ and $n\in\mathbb{N}$. Assume
that $x$ is $m$-integral over $\left(  A,\left(  I_{\rho}\right)  _{\rho
\in\mathbb{N}}\right)  ,$ and that $y$ is $n$-integral over $\left(  A,\left(
J_{\rho}\right)  _{\rho\in\mathbb{N}}\right)  $. Then, $xy$ is $nm$-integral
over $\left(  A,\left(  I_{\rho}J_{\rho}\right)  _{\rho\in\mathbb{N}}\right)
$.
\end{quote}

\textit{Proof of Theorem 13.} First, a trivial observation:

\textit{Lemma }$\mathcal{I}^{\prime}$\textit{:} Let $A$, $A^{\prime}$ and
$B^{\prime}$ be three rings such that $A\subseteq A^{\prime}\subseteq
B^{\prime}$. Let $\left(  I_{\rho}\right)  _{\rho\in\mathbb{N}}$ be an ideal
semifiltration of $A$. Let $v\in B^{\prime}$. Let $n\in\mathbb{N}$. If $v$ is
$n$-integral over $\left(  A,\left(  I_{\rho}\right)  _{\rho\in\mathbb{N}%
}\right)  $, then $v$ is $n$-integral over $\left(  A^{\prime},\left(
I_{\rho}A^{\prime}\right)  _{\rho\in\mathbb{N}}\right)  $. (Note that $\left(
I_{\rho}A^{\prime}\right)  _{\rho\in\mathbb{N}}$ is an ideal semifiltration of
$A^{\prime}$, according to Lemma $\mathcal{J}$.)

\textit{Proof of Lemma }$\mathcal{I}^{\prime}$\textit{:} Assume that $v$ is
$n$-integral over $\left(  A,\left(  I_{\rho}\right)  _{\rho\in\mathbb{N}%
}\right)  $. Then, by Definition 9 (applied to $B^{\prime}$ and $v$ instead of
$B$ and $u$), there exists some $\left(  a_{0},a_{1},...,a_{n}\right)  \in
A^{n+1}$ such that%
\[
\sum\limits_{k=0}^{n}a_{k}v^{k}=0,\ \ \ \ \ \ \ \ \ \ a_{n}%
=1,\ \ \ \ \ \ \ \ \ \ \text{and}\ \ \ \ \ \ \ \ \ \ a_{i}\in I_{n-i}\text{
for every }i\in\left\{  0,1,...,n\right\}  .
\]
But $\left(  a_{0},a_{1},...,a_{n}\right)  \in A^{n+1}$ yields $\left(
a_{0},a_{1},...,a_{n}\right)  \in\left(  A^{\prime}\right)  ^{n+1}$ (since
$A\subseteq A^{\prime}$), and $a_{i}\in I_{n-i}$ yields $a_{i}\in
I_{n-i}A^{\prime}$ (since $I_{n-i}\subseteq I_{n-i}A^{\prime}$) for every
$i\in\left\{  0,1,...,n\right\}  $. Thus, $\left(  a_{0},a_{1},...,a_{n}%
\right)  \in\left(  A^{\prime}\right)  ^{n+1}$ and%
\[
\sum\limits_{k=0}^{n}a_{k}v^{k}=0,\ \ \ \ \ \ \ \ \ \ a_{n}%
=1,\ \ \ \ \ \ \ \ \ \ \text{and}\ \ \ \ \ \ \ \ \ \ a_{i}\in I_{n-i}%
A^{\prime}\text{ for every }i\in\left\{  0,1,...,n\right\}  .
\]
Hence, by Definition 9 (applied to $B^{\prime}$, $A^{\prime}$, $\left(
I_{\rho}A^{\prime}\right)  _{\rho\in\mathbb{N}}$ and $v$ instead of $B$, $A$,
$\left(  I_{\rho}\right)  _{\rho\in\mathbb{N}}$ and $u$), the element $v$ is
$n$-integral over $\left(  A^{\prime},\left(  I_{\rho}A^{\prime}\right)
_{\rho\in\mathbb{N}}\right)  $. This proves Lemma $\mathcal{I}^{\prime}$.

Now let us prove Theorem 13.

We have $\left(  J_{\rho}\right)  _{\rho\in\mathbb{N}}=\left(  J_{\tau
}\right)  _{\tau\in\mathbb{N}}$. Hence, $y$ is $n$-integral over $\left(
A,\left(  J_{\tau}\right)  _{\tau\in\mathbb{N}}\right)  $ (since $y$ is
$n$-integral over $\left(  A,\left(  J_{\rho}\right)  _{\rho\in\mathbb{N}%
}\right)  $).

Consider the polynomial ring $A\left[  Y\right]  $ and its $A$-subalgebra
$A\left[  \left(  I_{\rho}\right)  _{\rho\in\mathbb{N}}\ast Y\right]  $. We
will abbreviate the ring $A\left[  \left(  I_{\rho}\right)  _{\rho
\in\mathbb{N}}\ast Y\right]  $ by $A_{\left[  I\right]  }$. We have
$A_{\left[  I\right]  }\subseteq B\left[  Y\right]  $, because $A_{\left[
I\right]  }=A\left[  \left(  I_{\rho}\right)  _{\rho\in\mathbb{N}}\ast
Y\right]  \subseteq A\left[  Y\right]  $ and we consider $A\left[  Y\right]  $
as a subring of $B\left[  Y\right]  $ as explained in Definition 7.

Theorem 7 (applied to $x$ and $m$ instead of $u$ and $n$) yields that $xY$ is
$m$-integral over $A\left[  \left(  I_{\rho}\right)  _{\rho\in\mathbb{N}}\ast
Y\right]  $ (since $x$ is $m$-integral over $\left(  A,\left(  I_{\rho
}\right)  _{\rho\in\mathbb{N}}\right)  $). In other words, $xY$ is
$m$-integral over $A_{\left[  I\right]  }$ (since $A\left[  \left(  I_{\rho
}\right)  _{\rho\in\mathbb{N}}\ast Y\right]  =A_{\left[  I\right]  }$).

On the other hand, Lemma $\mathcal{I}^{\prime}$ (applied to $A_{\left[
I\right]  }$, $B\left[  Y\right]  $, $\left(  J_{\tau}\right)  _{\tau
\in\mathbb{N}}$ and $y$ instead of $A^{\prime}$, $B^{\prime}$, $\left(
I_{\rho}\right)  _{\rho\in\mathbb{N}}$ and $v$) yields that $y$ is
$n$-integral over $\left(  A_{\left[  I\right]  },\left(  J_{\tau}A_{\left[
I\right]  }\right)  _{\tau\in\mathbb{N}}\right)  $ (since $y$ is $n$-integral
over $\left(  A,\left(  J_{\tau}\right)  _{\tau\in\mathbb{N}}\right)  $, and
$A\subseteq A_{\left[  I\right]  }\subseteq B\left[  Y\right]  $).

Hence, Theorem 8 \textbf{(c)} (applied to $A_{\left[  I\right]  },$ $B\left[
Y\right]  $, $\left(  J_{\tau}A_{\left[  I\right]  }\right)  _{\tau
\in\mathbb{N}}$, $y$, $xY$, $m$ and $n$ instead of $A,$ $B$, $\left(  I_{\rho
}\right)  _{\rho\in\mathbb{N}}$, $x$, $y$, $n$ and $m$ respectively) yields
that $y\cdot xY$ is $mn$-integral over $\left(  A_{\left[  I\right]  },\left(
J_{\tau}A_{\left[  I\right]  }\right)  _{\tau\in\mathbb{N}}\right)  $ (since
$y$ is $n$-integral over $\left(  A_{\left[  I\right]  },\left(  J_{\tau
}A_{\left[  I\right]  }\right)  _{\tau\in\mathbb{N}}\right)  $, and $xY$ is
$m$-integral over $A_{\left[  I\right]  }$). Since $y\cdot xY=xyY$ and
$mn=nm$, this means that $xyY$ is $nm$-integral over $\left(  A_{\left[
I\right]  },\left(  J_{\tau}A_{\left[  I\right]  }\right)  _{\tau\in
\mathbb{N}}\right)  $. Hence, Theorem 11 (applied to $xy$ and $nm$ instead of
$u$ and $n$) yields that $xy$ is $nm$-integral over $\left(  A,\left(
I_{\rho}J_{\rho}\right)  _{\rho\in\mathbb{N}}\right)  $. This proves Theorem 13.

\begin{center}
\color{blue} \textbf{4. Accelerating ideal semifiltrations} \color{black}
\end{center}

We start this section with an obvious observation:

\begin{quote}
\textbf{Theorem 14.} Let $A$ be a ring. Let $\left(  I_{\rho}\right)
_{\rho\in\mathbb{N}}$ be an ideal semifiltration of $A$. Let $\lambda
\in\mathbb{N}$. Then, $\left(  I_{\lambda\rho}\right)  _{\rho\in\mathbb{N}}$
is an ideal semifiltration of $A$.
\end{quote}

\textit{Proof of Theorem 14.} Since $\left(  I_{\rho}\right)  _{\rho
\in\mathbb{N}}$ is an ideal semifiltration of $A$, it is a sequence of ideals
of $A$, and it satisfies the two conditions%
\begin{align*}
I_{0}  &  =A;\\
I_{a}I_{b}  &  \subseteq I_{a+b}\ \ \ \ \ \ \ \ \ \ \text{for every }%
a\in\mathbb{N}\text{ and }b\in\mathbb{N}%
\end{align*}
(by Definition 6).

Now, $I_{\lambda\rho}$ is an ideal of $A$ for every $\rho\in\mathbb{N}$ (since
$\left(  I_{\rho}\right)  _{\rho\in\mathbb{N}}$ is a sequence of ideals of
$A$). Hence, $\left(  I_{\lambda\rho}\right)  _{\rho\in\mathbb{N}}$ is a
sequence of ideals of $A$. Thus, in order to prove that $\left(
I_{\lambda\rho}\right)  _{\rho\in\mathbb{N}}$ is an ideal semifiltration of
$A$, it is enough to verify that it satisfies the two conditions%
\begin{align*}
I_{\lambda\cdot0}  &  =A;\\
I_{\lambda a}I_{\lambda b}  &  \subseteq I_{\lambda\left(  a+b\right)
}\ \ \ \ \ \ \ \ \ \ \text{for every }a\in\mathbb{N}\text{ and }b\in
\mathbb{N}.
\end{align*}
But these two conditions are satisfied, since%
\begin{align*}
I_{\lambda\cdot0}  &  =I_{0}=A;\\
I_{\lambda a}I_{\lambda b}  &  \subseteq I_{\lambda a+\lambda b}%
\ \ \ \ \ \ \ \ \ \ \left(  \text{since }\left(  I_{\rho}\right)  _{\rho
\in\mathbb{N}}\text{ is an ideal semifiltration of }A\right) \\
&  =I_{\lambda\left(  a+b\right)  }\ \ \ \ \ \ \ \ \ \ \text{for every }%
a\in\mathbb{N}\text{ and }b\in\mathbb{N}.
\end{align*}
Hence, $\left(  I_{\lambda\rho}\right)  _{\rho\in\mathbb{N}}$ is an ideal
semifiltration of $A$ (by Definition 6, applied to $\left(  I_{\lambda\rho
}\right)  _{\rho\in\mathbb{N}}$ instead of $\left(  I_{\rho}\right)  _{\rho
\in\mathbb{N}}$). This proves Theorem 14.

I refer to $\left(  I_{\lambda\rho}\right)  _{\rho\in\mathbb{N}}$ as the
$\lambda$\textit{-acceleration} of the ideal semifiltration $\left(  I_{\rho
}\right)  _{\rho\in\mathbb{N}}$.

Now, Theorem 11, itself a generalization of Theorem 7, is going to be
generalized once more:

\begin{quote}
\textbf{Theorem 15.} Let $A$ and $B$ be two rings such that $A\subseteq B$.
Let $\left(  I_{\rho}\right)  _{\rho\in\mathbb{N}}$ and $\left(  J_{\rho
}\right)  _{\rho\in\mathbb{N}}$ be two ideal semifiltrations of $A$. Let
$n\in\mathbb{N}$. Let $u\in B$. Let $\lambda\in\mathbb{N}$.

We know that $\left(  I_{\lambda\rho}\right)  _{\rho\in\mathbb{N}}$ is an
ideal semifiltration of $A$ (according to Theorem 14).

Hence, $\left(  I_{\lambda\rho}J_{\rho}\right)  _{\rho\in\mathbb{N}}$ is an
ideal semifiltration of $A$ (according to Theorem 10 \textbf{(b)}, applied to
$\left(  I_{\lambda\rho}\right)  _{\rho\in\mathbb{N}}$ instead of $\left(
I_{\rho}\right)  _{\rho\in\mathbb{N}}$).

Consider the polynomial ring $A\left[  Y\right]  $ and its $A$-subalgebra
$A\left[  \left(  I_{\rho}\right)  _{\rho\in\mathbb{N}}\ast Y\right]  $.

We will abbreviate the ring $A\left[  \left(  I_{\rho}\right)  _{\rho
\in\mathbb{N}}\ast Y\right]  $ by $A_{\left[  I\right]  }$.

By Lemma $\mathcal{J}$ (applied to $A_{\left[  I\right]  }$ and $\left(
J_{\tau}\right)  _{\tau\in\mathbb{N}}$ instead of $A^{\prime}$ and $\left(
I_{\rho}\right)  _{\rho\in\mathbb{N}}$), the sequence $\left(  J_{\tau
}A_{\left[  I\right]  }\right)  _{\tau\in\mathbb{N}}$ is an ideal
semifiltration of $A_{\left[  I\right]  }$ (since $A\subseteq A_{\left[
I\right]  }$ and since $\left(  J_{\tau}\right)  _{\tau\in\mathbb{N}}=\left(
J_{\rho}\right)  _{\rho\in\mathbb{N}}$ is an ideal semifiltration of $A$).

Then, the element $u$ of $B$ is $n$-integral over $\left(  A,\left(
I_{\lambda\rho}J_{\rho}\right)  _{\rho\in\mathbb{N}}\right)  $ if and only if
the element $uY^{\lambda}$ of the polynomial ring $B\left[  Y\right]  $ is
$n$-integral over $\left(  A_{\left[  I\right]  },\left(  J_{\tau}A_{\left[
I\right]  }\right)  _{\tau\in\mathbb{N}}\right)  .$ (Here, $A_{\left[
I\right]  }\subseteq B\left[  Y\right]  $ because $A_{\left[  I\right]
}=A\left[  \left(  I_{\rho}\right)  _{\rho\in\mathbb{N}}\ast Y\right]
\subseteq A\left[  Y\right]  $ and we consider $A\left[  Y\right]  $ as a
subring of $B\left[  Y\right]  $ as explained in Definition 7.)
\end{quote}

\textit{Proof of Theorem 15.} First, note that%
\begin{align*}
\sum\limits_{\ell\in\mathbb{N}}I_{\ell}Y^{\ell}  &  =\sum\limits_{i\in
\mathbb{N}}I_{i}Y^{i}\ \ \ \ \ \ \ \ \ \ \left(  \text{here we renamed }%
\ell\text{ as }i\text{ in the sum}\right) \\
&  =A\left[  \left(  I_{\rho}\right)  _{\rho\in\mathbb{N}}\ast Y\right]
=A_{\left[  I\right]  }.
\end{align*}


In order to verify Theorem 15, we have to prove the following two lemmata:

\textit{Lemma }$\mathcal{E}^{\prime\prime}$\textit{:} If $u$ is $n$-integral
over $\left(  A,\left(  I_{\lambda\rho}J_{\rho}\right)  _{\rho\in\mathbb{N}%
}\right)  $, then $uY^{\lambda}$ is $n$-integral over $\left(  A_{\left[
I\right]  },\left(  J_{\tau}A_{\left[  I\right]  }\right)  _{\tau\in
\mathbb{N}}\right)  $.

\textit{Lemma} $\mathcal{F}^{\prime\prime}$\textit{:} If $uY^{\lambda}$ is
$n$-integral over $\left(  A_{\left[  I\right]  },\left(  J_{\tau}A_{\left[
I\right]  }\right)  _{\tau\in\mathbb{N}}\right)  $, then $u$ is $n$-integral
over $\left(  A,\left(  I_{\lambda\rho}J_{\rho}\right)  _{\rho\in\mathbb{N}%
}\right)  $.

\textit{Proof of Lemma }$\mathcal{E}^{\prime\prime}$\textit{:} Assume that $u$
is $n$-integral over $\left(  A,\left(  I_{\lambda\rho}J_{\rho}\right)
_{\rho\in\mathbb{N}}\right)  $. Then, by Definition 9 (applied to $\left(
I_{\lambda\rho}J_{\rho}\right)  _{\rho\in\mathbb{N}}$ instead of $\left(
I_{\rho}\right)  _{\rho\in\mathbb{N}}$), there exists some $\left(
a_{0},a_{1},...,a_{n}\right)  \in A^{n+1}$ such that%
\[
\sum\limits_{k=0}^{n}a_{k}u^{k}=0,\ \ \ \ \ \ \ \ \ \ a_{n}%
=1,\ \ \ \ \ \ \ \ \ \ \text{and}\ \ \ \ \ \ \ \ \ \ a_{i}\in I_{\lambda
\left(  n-i\right)  }J_{n-i}\text{ for every }i\in\left\{  0,1,...,n\right\}
.
\]


Note that $a_{k}Y^{\lambda\left(  n-k\right)  }\in A_{\left[  I\right]  }$ for
every $k\in\left\{  0,1,...,n\right\}  $ (because $a_{k}\in I_{\lambda\left(
n-k\right)  }J_{n-k}\subseteq I_{\lambda\left(  n-k\right)  }$ (since
$I_{\lambda\left(  n-k\right)  }$ is an ideal of $A$) and thus $a_{k}%
Y^{\lambda\left(  n-k\right)  }\in I_{\lambda\left(  n-k\right)  }%
Y^{\lambda\left(  n-k\right)  }\subseteq\sum\limits_{i\in\mathbb{N}}I_{i}%
Y^{i}=A_{\left[  I\right]  }$). Thus, we can define an $\left(  n+1\right)
$-tuple $\left(  b_{0},b_{1},...,b_{n}\right)  \in\left(  A_{\left[  I\right]
}\right)  ^{n+1}$ by%
\[
\left(  b_{k}=a_{k}Y^{\lambda\left(  n-k\right)  }\text{ for every }%
k\in\left\{  0,1,...,n\right\}  \right)  .
\]
Then,%
\begin{align*}
\sum\limits_{k=0}^{n}b_{k}\cdot\left(  uY^{\lambda}\right)  ^{k}  &
=\sum\limits_{k=0}^{n}a_{k}Y^{\lambda\left(  n-k\right)  }\cdot
\underbrace{\left(  uY^{\lambda}\right)  ^{k}}_{\substack{=u^{k}\left(
Y^{\lambda}\right)  ^{k}\\=u^{k}Y^{\lambda k}}}=\sum\limits_{k=0}^{n}%
a_{k}Y^{\lambda\left(  n-k\right)  }u^{k}Y^{\lambda k}=\sum\limits_{k=0}%
^{n}a_{k}u^{k}\underbrace{Y^{\lambda\left(  n-k\right)  }Y^{\lambda k}%
}_{\substack{=Y^{\lambda\left(  n-k\right)  +\lambda k}\\=Y^{\lambda n}%
}}=Y^{\lambda n}\cdot\underbrace{\sum\limits_{k=0}^{n}a_{k}u^{k}}_{=0}=0;\\
b_{n}  &  =\underbrace{a_{n}}_{=1}\underbrace{Y^{\lambda\left(  n-n\right)  }%
}_{=Y^{\lambda\cdot0}=Y^{0}=1}=1,
\end{align*}
and%
\[
b_{i}=\underbrace{a_{i}}_{\substack{\in I_{\lambda\left(  n-i\right)  }%
J_{n-i}\\=J_{n-i}I_{\lambda\left(  n-i\right)  }}}Y^{\lambda\left(
n-i\right)  }\in J_{n-i}\underbrace{I_{\lambda\left(  n-i\right)  }%
Y^{\lambda\left(  n-i\right)  }}_{\substack{\subseteq\sum\limits_{\ell
\in\mathbb{N}}I_{\ell}Y^{\ell}\\=A_{\left[  I\right]  }}}\subseteq
J_{n-i}A_{\left[  I\right]  }%
\]
for every $i\in\left\{  0,1,...,n\right\}  $.

Altogether, we now know that $\left(  b_{0},b_{1},...,b_{n}\right)  \in\left(
A_{\left[  I\right]  }\right)  ^{n+1}$ and%
\[
\sum\limits_{k=0}^{n}b_{k}\cdot\left(  uY^{\lambda}\right)  ^{k}%
=0,\ \ \ \ \ \ \ \ \ \ b_{n}=1,\ \ \ \ \ \ \ \ \ \ \text{and}%
\ \ \ \ \ \ \ \ \ \ b_{i}\in J_{n-i}A_{\left[  I\right]  }\text{ for every
}i\in\left\{  0,1,...,n\right\}  .
\]
Hence, by Definition 9 (applied to $A_{\left[  I\right]  },$ $B\left[
Y\right]  ,$ $\left(  J_{\tau}A_{\left[  I\right]  }\right)  _{\tau
\in\mathbb{N}},$ $uY^{\lambda}$ and $\left(  b_{0},b_{1},...,b_{n}\right)  $
instead of $A,$ $B,$ $\left(  I_{\rho}\right)  _{\rho\in\mathbb{N}},$ $u$ and
$\left(  a_{0},a_{1},...,a_{n}\right)  $), the element $uY^{\lambda}$ is
$n$-integral over $\left(  A_{\left[  I\right]  },\left(  J_{\tau}A_{\left[
I\right]  }\right)  _{\tau\in\mathbb{N}}\right)  $. This proves Lemma
$\mathcal{E}^{\prime\prime}$.

\textit{Proof of Lemma }$\mathcal{F}^{\prime\prime}$\textit{:} Assume that
$uY^{\lambda}$ is $n$-integral over $\left(  A_{\left[  I\right]  },\left(
J_{\tau}A_{\left[  I\right]  }\right)  _{\tau\in\mathbb{N}}\right)  $. Then,
by Definition 9 (applied to $A_{\left[  I\right]  },$ $B\left[  Y\right]  ,$
$\left(  J_{\tau}A_{\left[  I\right]  }\right)  _{\tau\in\mathbb{N}},$
$uY^{\lambda}$ and $\left(  p_{0},p_{1},...,p_{n}\right)  $ instead of $A,$
$B,$ $\left(  I_{\rho}\right)  _{\rho\in\mathbb{N}},$ $u$ and $\left(
a_{0},a_{1},...,a_{n}\right)  $), there exists some $\left(  p_{0}%
,p_{1},...,p_{n}\right)  \in\left(  A_{\left[  I\right]  }\right)  ^{n+1}$
such that%
\[
\sum\limits_{k=0}^{n}p_{k}\cdot\left(  uY^{\lambda}\right)  ^{k}%
=0,\ \ \ \ \ \ \ \ \ \ p_{n}=1,\ \ \ \ \ \ \ \ \ \ \text{and}%
\ \ \ \ \ \ \ \ \ \ p_{i}\in J_{n-i}A_{\left[  I\right]  }\text{ for every
}i\in\left\{  0,1,...,n\right\}  .
\]
For every $k\in\left\{  0,1,...,n\right\}  $, we have%
\begin{align*}
p_{k}  &  \in J_{n-k}A_{\left[  I\right]  }=J_{n-k}\sum\limits_{i\in
\mathbb{N}}I_{i}Y^{i}\ \ \ \ \ \ \ \ \ \ \left(  \text{since }A_{\left[
I\right]  }=\sum\limits_{i\in\mathbb{N}}I_{i}Y^{i}\right) \\
&  =\sum\limits_{i\in\mathbb{N}}J_{n-k}I_{i}Y^{i}=\sum\limits_{i\in\mathbb{N}%
}I_{i}J_{n-k}Y^{i},
\end{align*}
and thus, there exists a sequence $\left(  p_{k,i}\right)  _{i\in\mathbb{N}%
}\in A^{\mathbb{N}}$ such that $p_{k}=\sum\limits_{i\in\mathbb{N}}p_{k,i}%
Y^{i}$, such that $p_{k,i}\in I_{i}J_{n-k}$ for every $i\in\mathbb{N}$, and
such that only finitely many $i\in\mathbb{N}$ satisfy $p_{k,i}\neq0$. Thus,%
\begin{align*}
\sum\limits_{k=0}^{n}p_{k}\cdot\left(  uY^{\lambda}\right)  ^{k}  &
=\sum\limits_{k=0}^{n}\sum\limits_{i\in\mathbb{N}}p_{k,i}Y^{i}\cdot
\underbrace{\left(  uY^{\lambda}\right)  ^{k}}_{\substack{=u^{k}\left(
Y^{\lambda}\right)  ^{k}\\=u^{k}Y^{\lambda k}\\=Y^{\lambda k}u^{k}%
}}\ \ \ \ \ \ \ \ \ \ \left(  \text{since }p_{k}=\sum\limits_{i\in\mathbb{N}%
}p_{k,i}Y^{i}\right) \\
&  =\sum\limits_{k=0}^{n}\sum\limits_{i\in\mathbb{N}}p_{k,i}\underbrace{Y^{i}%
\cdot Y^{\lambda k}}_{=Y^{i+\lambda k}}u^{k}\\
&  =\sum\limits_{k=0}^{n}\sum\limits_{i\in\mathbb{N}}p_{k,i}Y^{i+\lambda
k}u^{k}=\sum\limits_{k\in\left\{  0,1,...,n\right\}  }\sum\limits_{i\in
\mathbb{N}}p_{k,i}Y^{i+\lambda k}u^{k}\\
&  =\sum\limits_{\left(  k,i\right)  \in\left\{  0,1,...,n\right\}
\times\mathbb{N}}p_{k,i}Y^{i+\lambda k}u^{k}=\sum_{\ell\in\mathbb{N}}%
\sum\limits_{\substack{\left(  k,i\right)  \in\left\{  0,1,...,n\right\}
\times\mathbb{N};\\i+\lambda k=\ell}}p_{k,i}\underbrace{Y^{i+\lambda k}%
}_{=Y^{\ell}}u^{k}\\
&  =\sum_{\ell\in\mathbb{N}}\sum\limits_{\substack{\left(  k,i\right)
\in\left\{  0,1,...,n\right\}  \times\mathbb{N};\\i+\lambda k=\ell}%
}p_{k,i}Y^{\ell}u^{k}=\sum_{\ell\in\mathbb{N}}\sum\limits_{\substack{\left(
k,i\right)  \in\left\{  0,1,...,n\right\}  \times\mathbb{N};\\i+\lambda
k=\ell}}p_{k,i}u^{k}Y^{\ell}.
\end{align*}
Hence, $\sum\limits_{k=0}^{n}p_{k}\cdot\left(  uY^{\lambda}\right)  ^{k}=0$
becomes $\sum\limits_{\ell\in\mathbb{N}}\sum\limits_{\substack{\left(
k,i\right)  \in\left\{  0,1,...,n\right\}  \times\mathbb{N};\\i+\lambda
k=\ell}}p_{k,i}u^{k}Y^{\ell}=0$. In other words, the polynomial $\sum
\limits_{\ell\in\mathbb{N}}\underbrace{\sum\limits_{\substack{\left(
k,i\right)  \in\left\{  0,1,...,n\right\}  \times\mathbb{N};\\i+\lambda
k=\ell}}p_{k,i}u^{k}}_{\in B}Y^{\ell}\in B\left[  Y\right]  $ equals $0$.
Hence, its coefficient before $Y^{\lambda n}$ equals $0$ as well. But its
coefficient before $Y^{\lambda n}$ is $\sum\limits_{\substack{\left(
k,i\right)  \in\left\{  0,1,...,n\right\}  \times\mathbb{N};\\i+\lambda
k=\lambda n}}p_{k,i}u^{k}$. Hence, $\sum\limits_{\substack{\left(  k,i\right)
\in\left\{  0,1,...,n\right\}  \times\mathbb{N};\\i+\lambda k=\lambda
n}}p_{k,i}u^{k}$ equals $0$.

Thus,%
\begin{align*}
0  &  =\sum\limits_{\substack{\left(  k,i\right)  \in\left\{
0,1,...,n\right\}  \times\mathbb{N};\\i+\lambda k=\lambda n}}p_{k,i}u^{k}%
=\sum\limits_{k\in\left\{  0,1,...,n\right\}  }\sum_{\substack{i\in
\mathbb{N};\\i+\lambda k=\lambda n}}p_{k,i}u^{k}=\sum\limits_{k\in\left\{
0,1,...,n\right\}  }p_{k,\lambda\left(  n-k\right)  }u^{k}\\
&  \ \ \ \ \ \ \ \ \ \ \left(
\begin{array}
[c]{c}%
\text{since }\left\{  i\in\mathbb{N}\text{\ }\mid\ i+\lambda k=\lambda
n\right\}  =\left\{  i\in\mathbb{N}\ \mid\ i=\lambda n-\lambda k\right\} \\
=\left\{  i\in\mathbb{N}\ \mid\ i=\lambda\left(  n-k\right)  \right\}
=\left\{  \lambda\left(  n-k\right)  \right\}  \text{ (because }\lambda\left(
n-k\right)  \in\mathbb{N}\text{,}\\
\text{since }k\in\left\{  0,1,...,n\right\}  \text{ yields }n-k\in
\mathbb{N}\text{ and we have }\lambda\in\mathbb{N}\text{)}\\
\text{yields }\sum\limits_{\substack{i\in\mathbb{N};\\i+\lambda k=\lambda
n}}p_{k,i}u^{k}=\sum\limits_{i\in\left\{  \lambda\left(  n-k\right)  \right\}
}p_{k,i}u^{k}=p_{k,\lambda\left(  n-k\right)  }u^{k}%
\end{array}
\right)  .
\end{align*}


Note that%
\begin{align*}
\sum\limits_{i\in\mathbb{N}}p_{n,i}Y^{i}  &  =p_{n}\ \ \ \ \ \ \ \ \ \ \left(
\text{since }\sum\limits_{i\in\mathbb{N}}p_{k,i}Y^{i}=p_{k}\text{ for every
}k\in\left\{  0,1,...,n\right\}  \right) \\
&  =1=1\cdot Y^{0}%
\end{align*}
in $A\left[  Y\right]  ,$ and thus the coefficient of the polynomial
$\sum\limits_{i\in\mathbb{N}}p_{n,i}Y^{i}\in A\left[  Y\right]  $ before
$Y^{0}$ is $1;$ but the coefficient of the polynomial $\sum\limits_{i\in
\mathbb{N}}p_{n,i}Y^{i}\in A\left[  Y\right]  $ before $Y^{0}$ is $p_{n,0};$
hence, $p_{n,0}=1$.

Define an $\left(  n+1\right)  $-tuple $\left(  a_{0},a_{1},...,a_{n}\right)
\in A^{n+1}$ by $\left(  a_{k}=p_{k,\lambda\left(  n-k\right)  }\text{ for
every }k\in\left\{  0,1,...,n\right\}  \right)  .$ Then, $a_{n}=p_{n,\lambda
\left(  n-n\right)  }=p_{n,\lambda\cdot0}=p_{n,0}=1$. Besides,%
\[
\sum\limits_{k=0}^{n}a_{k}u^{k}=\sum\limits_{k=0}^{n}p_{k,\lambda\left(
n-k\right)  }u^{k}=\sum\limits_{k\in\left\{  0,1,...,n\right\}  }%
p_{k,\lambda\left(  n-k\right)  }u^{k}=0.
\]
Finally, $a_{k}=p_{k,\lambda\left(  n-k\right)  }\in I_{\lambda\left(
n-k\right)  }J_{n-k}$ (since $p_{k,i}\in I_{i}J_{n-k}$ for every
$i\in\mathbb{N}$) for every $k\in\left\{  0,1,...,n\right\}  $. In other
words, $a_{i}\in I_{\lambda\left(  n-i\right)  }J_{n-i}$ for every
$i\in\left\{  0,1,...,n\right\}  $.

Altogether, we now know that%
\[
\sum\limits_{k=0}^{n}a_{k}u^{k}=0,\ \ \ \ \ \ \ \ \ \ a_{n}%
=1,\ \ \ \ \ \ \ \ \ \ \text{and}\ \ \ \ \ \ \ \ \ \ a_{i}\in I_{\lambda
\left(  n-i\right)  }J_{n-i}\text{ for every }i\in\left\{  0,1,...,n\right\}
.
\]
Thus, by Definition 9 (applied to $\left(  I_{\lambda\rho}J_{\rho}\right)
_{\rho\in\mathbb{N}}$ instead of $\left(  I_{\rho}\right)  _{\rho\in
\mathbb{N}}$), the element $u$ is $n$-integral over $\left(  A,\left(
I_{\lambda\rho}J_{\rho}\right)  _{\rho\in\mathbb{N}}\right)  $. This proves
Lemma $\mathcal{F}^{\prime\prime}$.

Combining Lemmata $\mathcal{E}^{\prime\prime}$ and $\mathcal{F}^{\prime\prime
}$, we obtain that $u$ is $n$-integral over $\left(  A,\left(  I_{\lambda\rho
}J_{\rho}\right)  _{\rho\in\mathbb{N}}\right)  $ if and only if $uY^{\lambda}$
is $n$-integral over $\left(  A_{\left[  I\right]  },\left(  J_{\tau
}A_{\left[  I\right]  }\right)  _{\tau\in\mathbb{N}}\right)  $. This proves
Theorem 15.

A particular case of Theorem 15:

\begin{quote}
\textbf{Theorem 16.} Let $A$ and $B$ be two rings such that $A\subseteq B$.
Let $\left(  I_{\rho}\right)  _{\rho\in\mathbb{N}}$ be an ideal semifiltration
of $A$. Let $n\in\mathbb{N}$. Let $u\in B$. Let $\lambda\in\mathbb{N}$.

We know that $\left(  I_{\lambda\rho}\right)  _{\rho\in\mathbb{N}}$ is an
ideal semifiltration of $A$ (according to Theorem 14).

Consider the polynomial ring $A\left[  Y\right]  $ and its $A$-subalgebra
$A\left[  \left(  I_{\rho}\right)  _{\rho\in\mathbb{N}}\ast Y\right]  $
defined in Definition 8.

Then, the element $u$ of $B$ is $n$-integral over $\left(  A,\left(
I_{\lambda\rho}\right)  _{\rho\in\mathbb{N}}\right)  $ if and only if the
element $uY^{\lambda}$ of the polynomial ring $B\left[  Y\right]  $ is
$n$-integral over the ring $A\left[  \left(  I_{\rho}\right)  _{\rho
\in\mathbb{N}}\ast Y\right]  .$ (Here, $A\left[  \left(  I_{\rho}\right)
_{\rho\in\mathbb{N}}\ast Y\right]  \subseteq B\left[  Y\right]  $ because
$A\left[  \left(  I_{\rho}\right)  _{\rho\in\mathbb{N}}\ast Y\right]
\subseteq A\left[  Y\right]  $ and we consider $A\left[  Y\right]  $ as a
subring of $B\left[  Y\right]  $ as explained in Definition 7).
\end{quote}

\textit{Proof of Theorem 16.} Theorem 10 \textbf{(a)} states that $\left(
A\right)  _{\rho\in\mathbb{N}}$ is an ideal semifiltration of $A$.

We will abbreviate the ring $A\left[  \left(  I_{\rho}\right)  _{\rho
\in\mathbb{N}}\ast Y\right]  $ by $A_{\left[  I\right]  }$.

We have the following five equivalences:

\begin{itemize}
\item The element $u$ of $B$ is $n$-integral over $\left(  A,\left(
I_{\lambda\rho}\right)  _{\rho\in\mathbb{N}}\right)  $ if and only if the
element $u$ of $B$ is $n$-integral over $\left(  A,\left(  I_{\lambda\rho
}A\right)  _{\rho\in\mathbb{N}}\right)  $ (since $I_{\lambda\rho}%
=I_{\lambda\rho}A$).

\item The element $u$ of $B$ is $n$-integral over $\left(  A,\left(
I_{\lambda\rho}A\right)  _{\rho\in\mathbb{N}}\right)  $ if and only if the
element $uY^{\lambda}$ of the polynomial ring $B\left[  Y\right]  $ is
$n$-integral over $\left(  A_{\left[  I\right]  },\left(  AA_{\left[
I\right]  }\right)  _{\tau\in\mathbb{N}}\right)  $ (according to Theorem 15,
applied to $\left(  A\right)  _{\rho\in\mathbb{N}}$ instead of $\left(
J_{\rho}\right)  _{\rho\in\mathbb{N}}$).

\item The element $uY^{\lambda}$ of the polynomial ring $B\left[  Y\right]  $
is $n$-integral over $\left(  A_{\left[  I\right]  },\left(  AA_{\left[
I\right]  }\right)  _{\tau\in\mathbb{N}}\right)  $ if and only if the element
$uY^{\lambda}$ of the polynomial ring $B\left[  Y\right]  $ is $n$-integral
over $\left(  A_{\left[  I\right]  },\left(  A_{\left[  I\right]  }\right)
_{\rho\in\mathbb{N}}\right)  $ (since $\left(  \underbrace{AA_{\left[
I\right]  }}_{=A_{\left[  I\right]  }}\right)  _{\tau\in\mathbb{N}}=\left(
A_{\left[  I\right]  }\right)  _{\tau\in\mathbb{N}}=\left(  A_{\left[
I\right]  }\right)  _{\rho\in\mathbb{N}}$).

\item The element $uY^{\lambda}$ of the polynomial ring $B\left[  Y\right]  $
is $n$-integral over $\left(  A_{\left[  I\right]  },\left(  A_{\left[
I\right]  }\right)  _{\rho\in\mathbb{N}}\right)  $ if and only if the element
$uY^{\lambda}$ of the polynomial ring $B\left[  Y\right]  $ is $n$-integral
over $A_{\left[  I\right]  }$ (by Theorem 12, applied to $A_{\left[  I\right]
}$, $B\left[  Y\right]  $ and $uY^{\lambda}$ instead of $A$, $B$ and $u$).

\item The element $uY^{\lambda}$ of the polynomial ring $B\left[  Y\right]  $
is $n$-integral over $A_{\left[  I\right]  }$ if and only if the element
$uY^{\lambda}$ of the polynomial ring $B\left[  Y\right]  $ is $n$-integral
over $A\left[  \left(  I_{\rho}\right)  _{\rho\in\mathbb{N}}\ast Y\right]  $
(since $A_{\left[  I\right]  }=A\left[  \left(  I_{\rho}\right)  _{\rho
\in\mathbb{N}}\ast Y\right]  $).
\end{itemize}

Combining these five equivalences, we obtain that the element $u$ of $B$ is
$n$-integral over $\left(  A,\left(  I_{\lambda\rho}\right)  _{\rho
\in\mathbb{N}}\right)  $ if and only if the element $uY^{\lambda}$ of the
polynomial ring $B\left[  Y\right]  $ is $n$-integral over $A\left[  \left(
I_{\rho}\right)  _{\rho\in\mathbb{N}}\ast Y\right]  .$ This proves Theorem 16.

Finally we can generalize even Theorem 2:

\begin{quote}
\textbf{Theorem 17.} Let $A$ and $B$ be two rings such that $A\subseteq B$.
Let $\left(  I_{\rho}\right)  _{\rho\in\mathbb{N}}$ be an ideal semifiltration
of $A$. Let $n\in\mathbb{N}$. Let $v\in B$. Let $a_{0},$ $a_{1},$ $...,$
$a_{n}$ be $n+1$ elements of $A$ such that $\sum\limits_{i=0}^{n}a_{i}v^{i}=0$
and $a_{i}\in I_{n-i}$ for every $i\in\left\{  0,1,...,n\right\}  $.

Let $k\in\left\{  0,1,...,n\right\}  $. We know that $\left(  I_{\left(
n-k\right)  \rho}\right)  _{\rho\in\mathbb{N}}$ is an ideal semifiltration of
$A$ (according to Theorem 14, applied to $\lambda=n-k$).

Then, $\sum\limits_{i=0}^{n-k}a_{i+k}v^{i}$ is $n$-integral over $\left(
A,\left(  I_{\left(  n-k\right)  \rho}\right)  _{\rho\in\mathbb{N}}\right)  $.
\end{quote}

\textit{Proof of Theorem 17.} Consider the polynomial ring $A\left[  Y\right]
$ and its $A$-subalgebra $A\left[  \left(  I_{\rho}\right)  _{\rho
\in\mathbb{N}}\ast Y\right]  $ defined in Definition 8. We have $A\left[
\left(  I_{\rho}\right)  _{\rho\in\mathbb{N}}\ast Y\right]  \subseteq B\left[
Y\right]  $, because $A\left[  \left(  I_{\rho}\right)  _{\rho\in\mathbb{N}%
}\ast Y\right]  \subseteq A\left[  Y\right]  $ and we consider $A\left[
Y\right]  $ as a subring of $B\left[  Y\right]  $ as explained in Definition 7.

As usual, note that%
\begin{align*}
\sum\limits_{\ell\in\mathbb{N}}I_{\ell}Y^{\ell}  &  =\sum\limits_{i\in
\mathbb{N}}I_{i}Y^{i}\ \ \ \ \ \ \ \ \ \ \left(  \text{here we renamed }%
\ell\text{ as }i\text{ in the sum}\right) \\
&  =A\left[  \left(  I_{\rho}\right)  _{\rho\in\mathbb{N}}\ast Y\right]  .
\end{align*}
In the ring $B\left[  Y\right]  $, we have%
\[
\sum_{i=0}^{n}a_{i}Y^{n-i}\underbrace{\left(  vY\right)  ^{i}}_{=v^{i}%
Y^{i}=Y^{i}v^{i}}=\sum_{i=0}^{n}a_{i}\underbrace{Y^{n-i}Y^{i}}_{=Y^{n}}%
v^{i}=Y^{n}\underbrace{\sum_{i=0}^{n}a_{i}v^{i}}_{=0}=0.
\]
Besides, $a_{i}Y^{n-i}\in A\left[  \left(  I_{\rho}\right)  _{\rho
\in\mathbb{N}}\ast Y\right]  $ for every $i\in\left\{  0,1,...,n\right\}  $
(since $\underbrace{a_{i}}_{\in I_{n-i}}Y^{n-i}\in I_{n-i}Y^{n-i}\subseteq
\sum\limits_{\ell\in\mathbb{N}}I_{\ell}Y^{\ell}=A\left[  \left(  I_{\rho
}\right)  _{\rho\in\mathbb{N}}\ast Y\right]  $). Hence, Theorem 2 (applied to
$A\left[  \left(  I_{\rho}\right)  _{\rho\in\mathbb{N}}\ast Y\right]  ,$
$B\left[  Y\right]  ,$ $vY$ and $a_{i}Y^{n-i}$ instead of $A,$ $B,$ $v$ and
$a_{i}$) yields that $\sum\limits_{i=0}^{n-k}a_{i+k}Y^{n-\left(  i+k\right)
}\left(  vY\right)  ^{i}$ is $n$-integral over $A\left[  \left(  I_{\rho
}\right)  _{\rho\in\mathbb{N}}\ast Y\right]  $. Since%
\[
\sum\limits_{i=0}^{n-k}a_{i+k}Y^{n-\left(  i+k\right)  }\underbrace{\left(
vY\right)  ^{i}}_{=v^{i}Y^{i}=Y^{i}v^{i}}=\sum\limits_{i=0}^{n-k}%
a_{i+k}\underbrace{Y^{n-\left(  i+k\right)  }Y^{i}}_{=Y^{\left(  n-\left(
i+k\right)  \right)  +i}=Y^{n-k}}v^{i}=\sum\limits_{i=0}^{n-k}a_{i+k}%
v^{i}\cdot Y^{n-k},
\]
this means that $\sum\limits_{i=0}^{n-k}a_{i+k}v^{i}\cdot Y^{n-k}$ is
$n$-integral over $A\left[  \left(  I_{\rho}\right)  _{\rho\in\mathbb{N}}\ast
Y\right]  $.

But Theorem 16 (applied to $u=$ $\sum\limits_{i=0}^{n-k}a_{i+k}v^{i}$ and
$\lambda=n-k$) yields that $\sum\limits_{i=0}^{n-k}a_{i+k}v^{i}$ is
$n$-integral over $\left(  A,\left(  I_{\left(  n-k\right)  \rho}\right)
_{\rho\in\mathbb{N}}\right)  $ if and only if $\sum\limits_{i=0}^{n-k}%
a_{i+k}v^{i}\cdot Y^{n-k}$ is $n$-integral over the ring $A\left[  \left(
I_{\rho}\right)  _{\rho\in\mathbb{N}}\ast Y\right]  $. Since we know that
$\sum\limits_{i=0}^{n-k}a_{i+k}v^{i}\cdot Y^{n-k}$ is $n$-integral over the
ring $A\left[  \left(  I_{\rho}\right)  _{\rho\in\mathbb{N}}\ast Y\right]  $,
this yields that $\sum\limits_{i=0}^{n-k}a_{i+k}v^{i}$ is $n$-integral over
$\left(  A,\left(  I_{\left(  n-k\right)  \rho}\right)  _{\rho\in\mathbb{N}%
}\right)  $. This proves Theorem 17.

\begin{center}
\color{blue} \textbf{5. Generalizing a lemma by Lombardi} \color{black}
\end{center}

Now, we are going to generalize Theorem 2 from [3] (which is the main result
of [3])\footnote{\textit{Caveat:} The notion "integral over $\left(
A,J\right)  $\ \ \ \ " defined in [3] has nothing to do with \textit{our}
notion "$n$-integral over $\left(  A,\left(  I_{n}\right)  _{n\in\mathbb{N}%
}\right)  $\ \ \ \ ".}. First, a very technical lemma:

\begin{quote}
\textbf{Lemma 18.} Let $A$ and $B$ be two rings such that $A\subseteq B$. Let
$x\in B$. Let $m\in\mathbb{N}$ and $n\in\mathbb{N}$. Let $u\in B$. Let $\mu
\in\mathbb{N}$ and $\nu\in\mathbb{N}$. Assume that%
\begin{equation}
u^{n}\in\left\langle u^{0},u^{1},...,u^{n-1}\right\rangle _{A}\cdot
\left\langle x^{0},x^{1},...,x^{\nu}\right\rangle _{A} \label{L18-1}%
\end{equation}
and that%
\begin{equation}
u^{m}x^{\mu}\in\left\langle u^{0},u^{1},...,u^{m-1}\right\rangle _{A}%
\cdot\left\langle x^{0},x^{1},...,x^{\mu}\right\rangle _{A}+\left\langle
u^{0},u^{1},...,u^{m}\right\rangle _{A}\cdot\left\langle x^{0},x^{1}%
,...,x^{\mu-1}\right\rangle _{A}. \label{L18-2}%
\end{equation}
Then, $u$ is $\left(  n\mu+m\nu\right)  $-integral over $A$.
\end{quote}

Before we prove this lemma, we recall a basic mathematical principle:

\begin{quote}
\textbf{Principle of strong induction (form \#1).} Let $\mathfrak{A}\left(
i\right)  $ be an assertion for every $i\in\mathbb{N}$. If%
\[
\text{every }I\in\mathbb{N}\text{ satisfying }\left(  \mathfrak{A}\left(
i\right)  \text{ for every }i\in\mathbb{N}\text{ such that }i<I\right)  \text{
satisfies }\mathfrak{A}\left(  I\right)  ,
\]
then%
\[
\text{every }i\in\mathbb{N}\text{ satisfies }\mathfrak{A}\left(  i\right)
\text{.}%
\]



\end{quote}

By renaming $i$, $I$ and $\mathfrak{A}$ as $j$, $J$ and $\mathfrak{B}$,
respectively, we can rewrite this principle as follows:

\begin{quote}
\textbf{Principle of strong induction (form \#2).} Let $\mathfrak{B}\left(
j\right)  $ be an assertion for every $j\in\mathbb{N}$. If%
\[
\text{every }J\in\mathbb{N}\text{ satisfying }\left(  \mathfrak{B}\left(
j\right)  \text{ for every }j\in\mathbb{N}\text{ such that }j<J\right)  \text{
satisfies }\mathfrak{B}\left(  J\right)  ,
\]
then%
\[
\text{every }j\in\mathbb{N}\text{ satisfies }\mathfrak{B}\left(  j\right)
\text{.}%
\]

\end{quote}

\textit{Proof of Lemma 18.} Let%
\[
S=\left(  \left\{  0,1,...,n-1\right\}  \times\left\{  0,1,...,\mu-1\right\}
\right)  \cup\left(  \left\{  0,1,...,m-1\right\}  \times\left\{  \mu
,\mu+1,...,\mu+\nu-1\right\}  \right)  .
\]
Then, $\left(  0,0\right)  \in S$\ \ \ \ \footnote{since
\begin{align*}
\left(  0,0\right)   &  \in\left\{  0,1,...,n-1\right\}  \times\left\{
0,1,...,\mu-1\right\} \\
&  \subseteq\left(  \left\{  0,1,...,n-1\right\}  \times\left\{
0,1,...,\mu-1\right\}  \right)  \cup\left(  \left\{  0,1,...,m-1\right\}
\times\left\{  \mu,\mu+1,...,\mu+\nu-1\right\}  \right)  =S
\end{align*}
}. Besides, $\left\vert S\right\vert =n\mu+m\nu\ \ \ \ $\footnote{since%
\begin{align*}
&  \left(  \left\{  0,1,...,n-1\right\}  \times\left\{  0,1,...,\mu-1\right\}
\right)  \cap\left(  \left\{  0,1,...,m-1\right\}  \times\left\{  \mu
,\mu+1,...,\mu+\nu-1\right\}  \right) \\
&  =\left(  \left\{  0,1,...,n-1\right\}  \cap\left\{  0,1,...,m-1\right\}
\right)  \times\underbrace{\left(  \left\{  0,1,...,\mu-1\right\}
\cap\left\{  \mu,\mu+1,...,\mu+\nu-1\right\}  \right)  }_{=\varnothing}\\
&  =\left(  \left\{  0,1,...,n-1\right\}  \cap\left\{  0,1,...,m-1\right\}
\right)  \times\varnothing=\varnothing
\end{align*}
yields%
\begin{align*}
&  \left\vert \left(  \left\{  0,1,...,n-1\right\}  \times\left\{
0,1,...,\mu-1\right\}  \right)  \cup\left(  \left\{  0,1,...,m-1\right\}
\times\left\{  \mu,\mu+1,...,\mu+\nu-1\right\}  \right)  \right\vert \\
&  =\underbrace{\left\vert \left\{  0,1,...,n-1\right\}  \times\left\{
0,1,...,\mu-1\right\}  \right\vert }_{=\left\vert \left\{
0,1,...,n-1\right\}  \right\vert \cdot\left\vert \left\{  0,1,...,\mu
-1\right\}  \right\vert }+\underbrace{\left\vert \left\{  0,1,...,m-1\right\}
\times\left\{  \mu,\mu+1,...,\mu+\nu-1\right\}  \right\vert }_{=\left\vert
\left\{  0,1,...,m-1\right\}  \right\vert \cdot\left\vert \left\{  \mu
,\mu+1,...,\mu+\nu-1\right\}  \right\vert }\\
&  =\underbrace{\left\vert \left\{  0,1,...,n-1\right\}  \right\vert }%
_{=n}\cdot\underbrace{\left\vert \left\{  0,1,...,\mu-1\right\}  \right\vert
}_{=\mu}+\underbrace{\left\vert \left\{  0,1,...,m-1\right\}  \right\vert
}_{=m}\cdot\underbrace{\left\vert \left\{  \mu,\mu+1,...,\mu+\nu-1\right\}
\right\vert }_{=\nu}=n\mu+m\nu,
\end{align*}
so that%
\begin{align*}
\left\vert S\right\vert  &  =\left\vert \left(  \left\{  0,1,...,n-1\right\}
\times\left\{  0,1,...,\mu-1\right\}  \right)  \cup\left(  \left\{
0,1,...,m-1\right\}  \times\left\{  \mu,\mu+1,...,\mu+\nu-1\right\}  \right)
\right\vert \\
&  =n\mu+m\nu
\end{align*}
}. Also,%
\begin{equation}
j<\mu+\nu\text{ for every }\left(  i,j\right)  \in S \label{L18-banal}%
\end{equation}
\footnote{In fact,%
\begin{align*}
S  &  =\left(  \left\{  0,1,...,n-1\right\}  \times\underbrace{\left\{
0,1,...,\mu-1\right\}  }_{\substack{\subseteq\left\{  0,1,...,\mu
+\nu-1\right\}  ,\\\text{since }\mu-1\leq\mu+\nu-1}}\right)  \cup\left(
\left\{  0,1,...,m-1\right\}  \times\underbrace{\left\{  \mu,\mu+1,...,\mu
+\nu-1\right\}  }_{\substack{\subseteq\left\{  0,1,...,\mu+\nu-1\right\}
,\\\text{since }\mu\geq0}}\right) \\
&  \subseteq\left(  \left\{  0,1,...,n-1\right\}  \times\left\{
0,1,...,\mu+\nu-1\right\}  \right)  \cup\left(  \left\{  0,1,...,m-1\right\}
\times\left\{  0,1,...,\mu+\nu-1\right\}  \right) \\
&  =\left(  \left\{  0,1,...,n-1\right\}  \cup\left\{  0,1,...,m-1\right\}
\right)  \times\left\{  0,1,...,\mu+\nu-1\right\}  .
\end{align*}
Hence, for every $\left(  i,j\right)  \in S$, we have $j\in\left\{
0,1,...,\mu+\nu-1\right\}  $ and thus $j<\mu+\nu$.}.

Let $U$ be the $A$-submodule $\left\langle u^{i}x^{j}\ \mid\ \left(
i,j\right)  \in S\right\rangle _{A}$ of $B$. Then, $U$ is an $\left(
n\mu+m\nu\right)  $-generated $A$-module (since $\left\vert S\right\vert
=n\mu+m\nu$). Besides, clearly,
\begin{equation}
u^{i}x^{j}\in U\text{ for every }\left(  i,j\right)  \in S \label{L18-U}%
\end{equation}
(since $U=\left\langle u^{i}x^{j}\ \mid\ \left(  i,j\right)  \in
S\right\rangle _{A}$). Thus, $u^{0}x^{0}\in U$ (by (\ref{L18-U}), applied to
$\left(  i,j\right)  =\left(  0,0\right)  $), since $\left(  0,0\right)  \in
S$. Since $\underbrace{u^{0}}_{=1}\underbrace{x^{0}}_{=1}=1,$ this becomes
$1\in U$.

Now, we will show that%
\begin{equation}
\text{every }i\in\mathbb{N}\text{ and }j\in\mathbb{N}\text{ satisfying }%
j<\mu+\nu\text{ satisfy }u^{i}x^{j}\in U. \label{L18-indU}%
\end{equation}


\textit{Proof of (\ref{L18-indU}).} For every $i\in\mathbb{N}$, define an
assertion $\mathfrak{A}\left(  i\right)  $ by%
\[
\mathfrak{A}\left(  i\right)  =\left(  \text{every }j\in\mathbb{N}\text{
satisfies }\left(  \text{if }j<\mu+\nu\text{, then }u^{i}x^{j}\in U\right)
\right)  .
\]


Let us now show that%
\begin{equation}
\text{every }I\in\mathbb{N}\text{ satisfying }\left(  \mathfrak{A}\left(
i\right)  \text{ for every }i\in\mathbb{N}\text{ such that }i<I\right)  \text{
satisfies }\mathfrak{A}\left(  I\right)  . \label{L18-indI}%
\end{equation}


\textit{Proof of (\ref{L18-indI}).} Let $I\in\mathbb{N}$ be such that%
\begin{equation}
\left(  \mathfrak{A}\left(  i\right)  \text{ for every }i\in\mathbb{N}\text{
such that }i<I\right)  . \label{L18-indIass}%
\end{equation}
We must prove that $\mathfrak{A}\left(  I\right)  $ holds.

For every $j\in\mathbb{N}$, define an assertion $\mathfrak{B}\left(  j\right)
$ by%
\[
\mathfrak{B}\left(  j\right)  =\left(  \text{if }j<\mu+\nu\text{, then }%
u^{I}x^{j}\in U\right)  .
\]


Let us now show that%
\begin{equation}
\text{every }J\in\mathbb{N}\text{ satisfying }\left(  \mathfrak{B}\left(
j\right)  \text{ for every }j\in\mathbb{N}\text{ such that }j<J\right)  \text{
satisfies }\mathfrak{B}\left(  J\right)  . \label{L18-indJ}%
\end{equation}


\textit{Proof of (\ref{L18-indJ}).} Let $J\in\mathbb{N}$ be such that%
\begin{equation}
\left(  \mathfrak{B}\left(  j\right)  \text{ for every }j\in\mathbb{N}\text{
such that }j<J\right)  . \label{L18-indJass}%
\end{equation}
We must prove that $\mathfrak{B}\left(  J\right)  $ holds.

Assume that $J<\mu+\nu$. Then,%
\begin{equation}
u^{I}x^{j}\in U\text{ for every }j\in\mathbb{N}\text{ such that }j<J
\label{L18-indL1}%
\end{equation}
(since for every $j\in\mathbb{N}$ such that $j<J$, the assertion
$\mathfrak{B}\left(  j\right)  $ holds (due to (\ref{L18-indJass})), i. e.,
the assertion $\left(  \text{if }j<\mu+\nu\text{, then }u^{I}x^{j}\in
U\right)  $ holds, which yields $u^{I}x^{j}\in U$ (since $j<J$ and $J<\mu+\nu$
yield $j<\mu+\nu$)). Now,%
\begin{align}
&  \left\langle u^{I}\right\rangle _{A}\cdot\underbrace{\left\langle
x^{0},x^{1},...,x^{J-1}\right\rangle _{A}}_{=\left\langle x^{j}\ \mid
\ j\in\left\{  0,1,...,J-1\right\}  \right\rangle _{A}}\nonumber\\
&  =\left\langle u^{I}\right\rangle _{A}\cdot\left\langle x^{j}\ \mid
\ j\in\left\{  0,1,...,J-1\right\}  \right\rangle _{A}=\left\langle u^{I}%
x^{j}\ \mid\ j\in\left\{  0,1,...,J-1\right\}  \right\rangle _{A}\nonumber\\
&  =\left\{  \sum\limits_{j\in\left\{  0,1,...,J-1\right\}  }a_{j}u^{I}%
x^{j}\ \mid\ \left(  a_{j}\right)  _{j\in\left\{  0,1,...,J-1\right\}  }\in
A^{\left\{  0,1,...,J-1\right\}  }\right\}  \subseteq U,
\label{L18-indL1final}%
\end{align}
since $\sum\limits_{j\in\left\{  0,1,...,J-1\right\}  }a_{j}u^{I}x^{j}\in U$
for every $\left(  a_{j}\right)  _{j\in\left\{  0,1,...,J-1\right\}  }\in
A^{\left\{  0,1,...,J-1\right\}  }$ (since $u^{I}x^{j}\in U$ for every
$j\in\left\{  0,1,...,J-1\right\}  $ (by (\ref{L18-indL1}), since $j<J$), and
since $U$ is an $A$-module). Also,%
\begin{equation}
u^{i}x^{j}\in U\text{ for every }i\in\mathbb{N}\text{ and }j\in\mathbb{N}%
\text{ such that }i<I\text{ and }j<\mu+\nu\label{L18-indL2}%
\end{equation}
(since for every $i\in\mathbb{N}$ and $j\in\mathbb{N}$ such that $i<I$ and
$j<\mu+\nu$, the assertion $\left(  \text{if }j<\mu+\nu\text{, then }%
u^{i}x^{j}\in U\right)  $ holds (because (\ref{L18-indIass}) and $i<I$ yield
$\mathfrak{A}\left(  i\right)  $), and thus $u^{i}x^{j}\in U$ (since
$j<\mu+\nu$)). Now,%
\begin{align}
&  \underbrace{\left\langle u^{0},u^{1},...,u^{I-1}\right\rangle _{A}%
}_{=\left\langle u^{i}\ \mid\ i\in\left\{  0,1,...,I-1\right\}  \right\rangle
_{A}}\cdot\underbrace{\left\langle x^{0},x^{1},...,x^{\mu+\nu-1}\right\rangle
_{A}}_{=\left\langle x^{j}\ \mid\ j\in\left\{  0,1,...,\mu+\nu-1\right\}
\right\rangle _{A}}\nonumber\\
&  =\left\langle u^{i}\ \mid\ i\in\left\{  0,1,...,I-1\right\}  \right\rangle
_{A}\cdot\left\langle x^{j}\ \mid\ j\in\left\{  0,1,...,\mu+\nu-1\right\}
\right\rangle _{A}\nonumber\\
&  =\left\langle u^{i}x^{j}\ \mid\ \left(  i,j\right)  \in\left\{
0,1,...,I-1\right\}  \times\left\{  0,1,...,\mu+\nu-1\right\}  \right\rangle
_{A}\nonumber\\
&  =\left\{  \sum\limits_{\left(  i,j\right)  \in\left\{  0,1,...,I-1\right\}
\times\left\{  0,1,...,\mu+\nu-1\right\}  }a_{i,j}u^{i}x^{j}\ \mid\ \left(
a_{i,j}\right)  _{\left(  i,j\right)  \in\left\{  0,1,...,I-1\right\}
\times\left\{  0,1,...,\mu+\nu-1\right\}  }\in A^{\left\{
0,1,...,I-1\right\}  \times\left\{  0,1,...,\mu+\nu-1\right\}  }\right\}
\nonumber\\
&  \subseteq U, \label{L18-indL2final}%
\end{align}
because $\sum\limits_{\left(  i,j\right)  \in\left\{  0,1,...,I-1\right\}
\times\left\{  0,1,...,\mu+\nu-1\right\}  }a_{i,j}u^{i}x^{j}\in U$ for every
$\left(  a_{i,j}\right)  _{\left(  i,j\right)  \in\left\{
0,1,...,I-1\right\}  \times\left\{  0,1,...,\mu+\nu-1\right\}  }\in
A^{\left\{  0,1,...,I-1\right\}  \times\left\{  0,1,...,\mu+\nu-1\right\}  }$
(since $u^{i}x^{j}\in U$ for every $\left(  i,j\right)  \in\left\{
0,1,...,I-1\right\}  \times\left\{  0,1,...,\mu+\nu-1\right\}  $ (by
(\ref{L18-indL2}), since $i<I$ (because $i\in\left\{  0,1,...,I-1\right\}  $)
and $j<\mu+\nu$ (because $j\in\left\{  0,1,...,\mu+\nu-1\right\}  $)), and
since $U$ is an $A$-module).

Note that $J<\mu+\nu$ yields $J\leq\mu+\nu-1$ (since $J$ and $\mu+\nu$ are integers).

Trivially,%
\[
\left(  I\geq m\ \wedge\ J\geq\mu\right)  \ \vee\ \left(  I<m\ \wedge
\ J\geq\mu\right)  \ \vee\ \left(  I\geq n\ \wedge\ J<\mu\right)
\ \vee\ \left(  I<n\ \wedge\ J<\mu\right)
\]
\footnote{since
\begin{align*}
&  \underbrace{\left(  I\geq m\ \wedge\ J\geq\mu\right)  \ \vee\ \left(
I<m\ \wedge\ J\geq\mu\right)  }_{\substack{=\ \left(  I\geq m\ \vee
\ I<m\right)  \ \wedge\ \left(  J\geq\mu\right)  \\=\ \left(  J\geq\mu\right)
\text{ (since }\left(  I\geq m\ \vee\ I<m\right)  \text{ is true)}}%
}\vee\ \underbrace{\left(  I\geq n\ \wedge\ J<\mu\right)  \ \vee\ \left(
I<n\ \wedge\ J<\mu\right)  }_{\substack{=\ \left(  I\geq n\ \vee\ I<n\right)
\ \wedge\ \left(  J<\mu\right)  \\=\ \ \left(  J<\mu\right)  \text{ (since
}\left(  I\geq n\ \vee\ I<n\right)  \text{ is true)}}}\\
&  =\ \left(  J\geq\mu\right)  \ \vee\ \left(  J<\mu\right)  \ =\ \text{true}%
\end{align*}
}. Hence, one of the following four cases must hold:

\textit{Case 1:} We have $I\geq m\ \wedge\ J\geq\mu$.

\textit{Case 2:} We have $I<m\ \wedge\ J\geq\mu$.

\textit{Case 3:} We have $I\geq n\ \wedge\ J<\mu$.

\textit{Case 4:} We have $I<n\ \wedge\ J<\mu$.

In Case 1, we have $I-m\geq0$ (since $I\geq m$) and $J-\mu\geq0$ (since
$J\geq\mu$), thus%
\begin{align*}
&  \underbrace{u^{I}}_{=u^{I-m}u^{m}}\underbrace{x^{J}}_{=x^{\mu}x^{J-\mu}}\\
&  =u^{I-m}\underbrace{u^{m}x^{\mu}}_{\substack{\in\left\langle u^{0}%
,u^{1},...,u^{m-1}\right\rangle _{A}\cdot\left\langle x^{0},x^{1},...,x^{\mu
}\right\rangle _{A}+\left\langle u^{0},u^{1},...,u^{m}\right\rangle _{A}%
\cdot\left\langle x^{0},x^{1},...,x^{\mu-1}\right\rangle _{A}\\\left(
\text{by (\ref{L18-2})}\right)  }}x^{J-\mu}\\
&  \in u^{I-m}\left(  \left\langle u^{0},u^{1},...,u^{m-1}\right\rangle
_{A}\cdot\left\langle x^{0},x^{1},...,x^{\mu}\right\rangle _{A}+\left\langle
u^{0},u^{1},...,u^{m}\right\rangle _{A}\cdot\left\langle x^{0},x^{1}%
,...,x^{\mu-1}\right\rangle _{A}\right)  x^{J-\mu}\\
&  =\underbrace{u^{I-m}\left\langle u^{0},u^{1},...,u^{m-1}\right\rangle _{A}%
}_{\substack{=\left\langle u^{I-m}u^{0},u^{I-m}u^{1},...,u^{I-m}%
u^{m-1}\right\rangle _{A}\\=\left\langle u^{\left(  I-m\right)  +0},u^{\left(
I-m\right)  +1},...,u^{\left(  I-m\right)  +\left(  m-1\right)  }\right\rangle
_{A}\\=\left\langle u^{I-m},u^{I-m+1},...,u^{I-1}\right\rangle _{A}%
\\\subseteq\left\langle u^{0},u^{1},...,u^{I-1}\right\rangle _{A}\text{
(since}\\\left\{  I-m,I-m+1,...,I-1\right\}  \subseteq\left\{
0,1,...,I-1\right\}  ,\\\text{since }I-m\geq0\text{)}}}\cdot
\underbrace{\left\langle x^{0},x^{1},...,x^{\mu}\right\rangle _{A}x^{J-\mu}%
}_{\substack{=\left\langle x^{0}x^{J-\mu},x^{1}x^{J-\mu},...,x^{\mu}x^{J-\mu
}\right\rangle _{A}\\=\left\langle x^{0+\left(  J-\mu\right)  },x^{1+\left(
J-\mu\right)  },...,x^{\mu+\left(  J-\mu\right)  }\right\rangle _{A}%
\\=\left\langle x^{J-\mu},x^{J-\mu+1},...,x^{J}\right\rangle _{A}%
\\\subseteq\left\langle x^{0},x^{1},...,x^{\mu+\nu-1}\right\rangle _{A}\text{
(since}\\\left\{  J-\mu,J-\mu+1,...,J\right\}  \subseteq\left\{
0,1,...,\mu+\nu-1\right\}  ,\\\text{since }J-\mu\geq0\text{ and }J\leq\mu
+\nu-1\text{)}}}\\
&  +\underbrace{u^{I-m}\left\langle u^{0},u^{1},...,u^{m}\right\rangle _{A}%
}_{\substack{=\left\langle u^{I-m}u^{0},u^{I-m}u^{1},...,u^{I-m}%
u^{m}\right\rangle _{A}\\=\left\langle u^{\left(  I-m\right)  +0},u^{\left(
I-m\right)  +1},...,u^{\left(  I-m\right)  +m}\right\rangle _{A}%
\\=\left\langle u^{I-m},u^{I-m+1},...,u^{I}\right\rangle _{A}\\\subseteq
\left\langle u^{0},u^{1},...,u^{I}\right\rangle _{A}\text{ (since}\\\left\{
I-m,I-m+1,...,I\right\}  \subseteq\left\{  0,1,...,I\right\}  ,\\\text{since
}I-m\geq0\text{)}}}\cdot\underbrace{\left\langle x^{0},x^{1},...,x^{\mu
-1}\right\rangle _{A}x^{J-\mu}}_{\substack{=\left\langle x^{0}x^{J-\mu}%
,x^{1}x^{J-\mu},...,x^{\mu-1}x^{J-\mu}\right\rangle _{A}\\=\left\langle
x^{0+\left(  J-\mu\right)  },x^{1+\left(  J-\mu\right)  },...,x^{\left(
\mu-1\right)  +\left(  J-\mu\right)  }\right\rangle _{A}\\=\left\langle
x^{J-\mu},x^{J-\mu+1},...,x^{J-1}\right\rangle _{A}\\\subseteq\left\langle
x^{0},x^{1},...,x^{J-1}\right\rangle _{A}\text{ (since}\\\left\{  J-\mu
,J-\mu+1,...,J-1\right\}  \subseteq\left\{  0,1,...,J-1\right\}
,\\\text{since }J-\mu\geq0\text{)}}}\\
&  \subseteq\underbrace{\left\langle u^{0},u^{1},...,u^{I-1}\right\rangle
_{A}\cdot\left\langle x^{0},x^{1},...,x^{\mu+\nu-1}\right\rangle _{A}%
}_{\subseteq U\text{ by (\ref{L18-indL2final})}}+\underbrace{\left\langle
u^{0},u^{1},...,u^{I}\right\rangle _{A}}_{=\left\langle u^{0},u^{1}%
,...,u^{I-1}\right\rangle _{A}+\left\langle u^{I}\right\rangle _{A}}%
\cdot\left\langle x^{0},x^{1},...,x^{J-1}\right\rangle _{A}\\
&  \subseteq U+\underbrace{\left(  \left\langle u^{0},u^{1},...,u^{I-1}%
\right\rangle _{A}+\left\langle u^{I}\right\rangle _{A}\right)  \cdot
\left\langle x^{0},x^{1},...,x^{J-1}\right\rangle _{A}}_{=\left\langle
u^{0},u^{1},...,u^{I-1}\right\rangle _{A}\cdot\left\langle x^{0}%
,x^{1},...,x^{J-1}\right\rangle _{A}+\left\langle u^{I}\right\rangle _{A}%
\cdot\left\langle x^{0},x^{1},...,x^{J-1}\right\rangle _{A}}\\
&  =U+\left\langle u^{0},u^{1},...,u^{I-1}\right\rangle _{A}\cdot
\underbrace{\left\langle x^{0},x^{1},...,x^{J-1}\right\rangle _{A}%
}_{\substack{\subseteq\left\langle x^{0},x^{1},...,x^{\mu+\nu-1}\right\rangle
_{A}\text{ (since}\\\left\{  0,1,...,J-1\right\}  \subseteq\left\{
0,1,...,\mu+\nu-1\right\}  \text{,}\\\text{since }J-1\leq J\leq\mu
+\nu-1\text{)}}}+\left\langle u^{I}\right\rangle _{A}\cdot\left\langle
x^{0},x^{1},...,x^{J-1}\right\rangle _{A}\\
&  \subseteq U+\underbrace{\left\langle u^{0},u^{1},...,u^{I-1}\right\rangle
_{A}\cdot\left\langle x^{0},x^{1},...,x^{\mu+\nu-1}\right\rangle _{A}%
}_{\subseteq U\text{ by (\ref{L18-indL2final})}}+\underbrace{\left\langle
u^{I}\right\rangle _{A}\cdot\left\langle x^{0},x^{1},...,x^{J-1}\right\rangle
_{A}}_{\subseteq U\text{ by (\ref{L18-indL1final})}}\\
&  \subseteq U+U+U\subseteq U\ \ \ \ \ \ \ \ \ \ \left(  \text{since }U\text{
is an }A\text{-module}\right)  .
\end{align*}
Thus, we have proved that $u^{I}x^{J}\in U$ holds in Case 1.

In Case 2, we have $I\in\left\{  0,1,...,m-1\right\}  $ (since $I<m$ and
$I\in\mathbb{N}$) and $J\in\left\{  \mu,\mu+1,...,\mu+\nu-1\right\}  $ (since
$J\geq\mu$ and $J<\mu+\nu$), thus%
\begin{align*}
\left(  I,J\right)   &  \in\left\{  0,1,...,m-1\right\}  \times\left\{
\mu,\mu+1,...,\mu+\nu-1\right\} \\
&  \subseteq\left(  \left\{  0,1,...,n-1\right\}  \times\left\{
0,1,...,\mu-1\right\}  \right)  \cup\left(  \left\{  0,1,...,m-1\right\}
\times\left\{  \mu,\mu+1,...,\mu+\nu-1\right\}  \right)  =S,
\end{align*}
so that $u^{I}x^{J}\in U$ (by (\ref{L18-U}), applied to $I$ and $J$ instead of
$i$ and $j$). Thus, we have proved that $u^{I}x^{J}\in U$ holds in Case 2.

In Case 3, we have $I-n\geq0$ (since $I\geq n$) and $J+\nu\leq\mu+\nu-1$
(since $J<\mu$ yields $J+\nu<\mu+\nu$, and since $J+\nu$ and $\mu+\nu$ are
integers), thus%
\begin{align*}
&  \underbrace{u^{I}}_{=u^{I-n}u^{n}}x^{J}\\
&  =u^{I-n}\underbrace{u^{n}}_{\substack{\in\left\langle u^{0},u^{1}%
,...,u^{n-1}\right\rangle _{A}\cdot\left\langle x^{0},x^{1},...,x^{\nu
}\right\rangle _{A}\\\left(  \text{by (\ref{L18-1})}\right)  }}x^{J}%
\in\underbrace{u^{I-n}\left\langle u^{0},u^{1},...,u^{n-1}\right\rangle _{A}%
}_{\substack{=\left\langle u^{I-n}u^{0},u^{I-n}u^{1},...,u^{I-n}%
u^{n-1}\right\rangle _{A}\\=\left\langle u^{\left(  I-n\right)  +0},u^{\left(
I-n\right)  +1},...,u^{\left(  I-n\right)  +\left(  n-1\right)  }\right\rangle
_{A}\\=\left\langle u^{I-n},u^{I-n+1},...,u^{I-1}\right\rangle _{A}%
\\\subseteq\left\langle u^{0},u^{1},...,u^{I-1}\right\rangle _{A}\text{
(since}\\\left\{  I-n,I-n+1,...,I-1\right\}  \subseteq\left\{
0,1,...,I-1\right\}  ,\\\text{since }I-n\geq0\text{)}}}\cdot
\underbrace{\left\langle x^{0},x^{1},...,x^{\nu}\right\rangle _{A}x^{J}%
}_{\substack{=\left\langle x^{0}x^{J},x^{1}x^{J},...,x^{\nu}x^{J}\right\rangle
_{A}\\=\left\langle x^{0+J},x^{1+J},...,x^{\nu+J}\right\rangle _{A}%
\\=\left\langle x^{J},x^{J+1},...,x^{J+\nu}\right\rangle _{A}\\\subseteq
\left\langle x^{0},x^{1},...,x^{\mu+\nu-1}\right\rangle _{A}\text{
(since}\\\left\{  J,J+1,...,J+\nu\right\}  \subseteq\left\{  0,1,...,\mu
+\nu-1\right\}  ,\\\text{since }J\geq0\text{ and }J+\nu\leq\mu+\nu-1\text{)}%
}}\\
&  \subseteq\left\langle u^{0},u^{1},...,u^{I-1}\right\rangle _{A}%
\cdot\left\langle x^{0},x^{1},...,x^{\mu+\nu-1}\right\rangle _{A}\subseteq
U\ \ \ \ \ \ \ \ \ \ \left(  \text{by (\ref{L18-indL2final})}\right)  .
\end{align*}
Thus, we have proved that $u^{I}x^{J}\in U$ holds in Case 3.

In Case 4, we have $I\in\left\{  0,1,...,n-1\right\}  $ (since $I<n$ and
$I\in\mathbb{N}$) and $J\in\left\{  0,1,...,\mu-1\right\}  $ (since $J<\mu$
and $J\in\mathbb{N}$), thus%
\begin{align*}
\left(  I,J\right)   &  \in\left\{  0,1,...,n-1\right\}  \times\left\{
0,1,...,\mu-1\right\} \\
&  \subseteq\left(  \left\{  0,1,...,n-1\right\}  \times\left\{
0,1,...,\mu-1\right\}  \right)  \cup\left(  \left\{  0,1,...,m-1\right\}
\times\left\{  \mu,\mu+1,...,\mu+\nu-1\right\}  \right)  =S,
\end{align*}
so that $u^{I}x^{J}\in U$ (by (\ref{L18-U}), applied to $I$ and $J$ instead of
$i$ and $j$). Thus, we have proved that $u^{I}x^{J}\in U$ holds in Case 4.

Therefore, we have proved that $u^{I}x^{J}\in U$ holds in each of the four
cases 1, 2, 3 and 4. Hence, $u^{I}x^{J}\in U$ always holds.

Hence, we have proved that if $J<\mu+\nu$, then $u^{I}x^{J}\in U$. In other
words, we have proved the assertion $\mathfrak{B}\left(  J\right)  $ (because
$\mathfrak{B}\left(  J\right)  =\left(  \text{if }J<\mu+\nu\text{, then }%
u^{I}x^{J}\in U\right)  $).

Thus, we have proved (\ref{L18-indJ}). Hence, the Principle of strong
induction (form \#2) yields that%
\[
\text{every }j\in\mathbb{N}\text{ satisfies }\mathfrak{B}\left(  j\right)
\text{.}%
\]
In other words,%
\[
\text{every }j\in\mathbb{N}\text{ satisfies }\left(  \text{if }j<\mu
+\nu\text{, then }u^{I}x^{j}\in U\right)  .
\]
Thus, the assertion $\mathfrak{A}\left(  I\right)  $ holds (because
$\mathfrak{A}\left(  I\right)  =\left(  \text{every }j\in\mathbb{N}\text{
satisfies }\left(  \text{if }j<\mu+\nu\text{, then }u^{I}x^{j}\in U\right)
\right)  $).

Thus, we have proved (\ref{L18-indI}). Hence, the Principle of strong
induction (form \#1) yields that%
\[
\text{every }i\in\mathbb{N}\text{ satisfies }\mathfrak{A}\left(  i\right)
\text{.}%
\]
In other words,%
\[
\text{every }i\in\mathbb{N}\text{ satisfies }\left(  \text{every }%
j\in\mathbb{N}\text{ satisfies }\left(  \text{if }j<\mu+\nu\text{, then }%
u^{i}x^{j}\in U\right)  \right)
\]
(since $\mathfrak{A}\left(  i\right)  =\left(  \text{every }j\in
\mathbb{N}\text{ satisfies }\left(  \text{if }j<\mu+\nu\text{, then }%
u^{i}x^{j}\in U\right)  \right)  $). This is equivalent to (\ref{L18-indU}).
Thus, (\ref{L18-indU}) is proven.

Now,%
\begin{equation}
u\cdot u^{i}x^{j}\in U\text{ for every }\left(  i,j\right)  \in S,
\label{L18-indUkonsequenz}%
\end{equation}
because $\underbrace{u\cdot u^{i}}_{=u^{i+1}}x^{j}=u^{i+1}x^{j}\in U$ (by
(\ref{L18-indU}) (applied to $i+1$ instead of $i$), since $j<\mu+\nu$ (by
(\ref{L18-banal}))).

Now,%
\begin{align*}
uU  &  =u\left\langle u^{i}x^{j}\ \mid\ \left(  i,j\right)  \in S\right\rangle
_{A}=\left\langle u\cdot u^{i}x^{j}\ \mid\ \left(  i,j\right)  \in
S\right\rangle _{A}\\
&  =\left\{  \sum\limits_{\left(  i,j\right)  \in S}a_{i,j}u\cdot u^{i}%
x^{j}\ \mid\ \left(  a_{i,j}\right)  _{\left(  i,j\right)  \in S}\in
A^{S}\right\}  \subseteq U,
\end{align*}
because $\sum\limits_{\left(  i,j\right)  \in S}a_{i,j}u\cdot u^{i}x^{j}\in U$
for every $\left(  a_{i,j}\right)  _{\left(  i,j\right)  \in S}\in A^{S}$
(since $\sum\limits_{\left(  i,j\right)  \in S}a_{i,j}\underbrace{u\cdot
u^{i}x^{j}}_{\in U\text{ by (\ref{L18-indUkonsequenz})}}\in U$, because $U$ is
an $A$-module).

Altogether, $U$ is an $\left(  n\mu+m\nu\right)  $-generated $A$-submodule of
$B$ such that $1\in U$ and $uU\subseteq U$. Thus, $u\in B$ satisfies Assertion
$\mathcal{C}$ of Theorem 1 with $n$ replaced by $n\mu+m\nu$. Hence, $u\in B$
satisfies the four equivalent assertions $\mathcal{A},$ $\mathcal{B},$
$\mathcal{C}$ and $\mathcal{D}$ of Theorem 1 with $n$ replaced by $n\mu+m\nu$.
Consequently, $u$ is $\left(  n\mu+m\nu\right)  $-integral over $A$. This
proves Lemma 18.

We record a weaker variant of Lemma 18:

\begin{quote}
\textbf{Lemma 19.} Let $A$ and $B$ be two rings such that $A\subseteq B$. Let
$x\in B$ and $y\in B$ be such that $xy\in A$. Let $m\in\mathbb{N}$ and
$n\in\mathbb{N}$. Let $u\in B$. Let $\mu\in\mathbb{N}$ and $\nu\in\mathbb{N}$.
Assume that%
\begin{equation}
u^{n}\in\left\langle u^{0},u^{1},...,u^{n-1}\right\rangle _{A}\cdot
\left\langle x^{0},x^{1},...,x^{\nu}\right\rangle _{A} \label{L19-1}%
\end{equation}
and that%
\begin{equation}
u^{m}\in\left\langle u^{0},u^{1},...,u^{m-1}\right\rangle _{A}\cdot
\left\langle y^{0},y^{1},...,y^{\mu}\right\rangle _{A}+\left\langle
u^{0},u^{1},...,u^{m}\right\rangle _{A}\cdot\left\langle y^{1},y^{2}%
,...,y^{\mu}\right\rangle _{A}. \label{L19-2}%
\end{equation}
Then, $u$ is $\left(  n\mu+m\nu\right)  $-integral over $A$.
\end{quote}

\textit{Proof of Lemma 19.} For every $i\in\left\{  0,1,...,\mu\right\}  $, we
have $\mu\geq i$ and thus $\mu-i\geq0$, so that%
\begin{align}
y^{i}\underbrace{x^{\mu}}_{=x^{\mu-i}x^{i}}  &  =y^{i}x^{\mu-i}x^{i}%
=\underbrace{x^{i}y^{i}}_{\substack{=\left(  xy\right)  ^{i}\in
A,\\\text{since }xy\in A}}x^{\mu-i}\in\left\langle x^{\mu-i}\right\rangle
_{A}\label{L19-P1}\\
&  \subseteq\left\langle x^{0},x^{1},...,x^{\mu}\right\rangle _{A}
\label{L19-P2}%
\end{align}
(since $\left\{  \mu-i\right\}  \subseteq\left\{  0,1,...,\mu\right\}  $,
because $\mu-i\in\left\{  0,1,...,\mu\right\}  $, since $i\in\left\{
0,1,...,\mu\right\}  $). Now,%
\begin{align}
&  \underbrace{\left\langle y^{0},y^{1},...,y^{\mu}\right\rangle _{A}%
}_{=\left\langle y^{i}\ \mid\ i\in\left\{  0,1,...,\mu\right\}  \right\rangle
_{A}}x^{\mu}=\left\langle y^{i}\ \mid\ i\in\left\{  0,1,...,\mu\right\}
\right\rangle _{A}x^{\mu}=\left\langle y^{i}x^{\mu}\ \mid\ i\in\left\{
0,1,...,\mu\right\}  \right\rangle _{A}\nonumber\\
&  =\left\{  \sum\limits_{i\in\left\{  0,1,...,\mu\right\}  }a_{i}y^{i}x^{\mu
}\ \mid\ \left(  a_{i}\right)  _{i\in\left\{  0,1,...,\mu\right\}  }\in
A^{\left\{  0,1,...,\mu\right\}  }\right\}  \subseteq\left\langle x^{0}%
,x^{1},...,x^{\mu}\right\rangle _{A}, \label{L19-Pa}%
\end{align}
since $\sum\limits_{i\in\left\{  0,1,...,\mu\right\}  }a_{i}y^{i}x^{\mu}%
\in\left\langle x^{0},x^{1},...,x^{\mu}\right\rangle _{A}$ for every $\left(
a_{i}\right)  _{i\in\left\{  0,1,...,\mu\right\}  }\in A^{\left\{
0,1,...,\mu\right\}  }$ (since $\sum\limits_{i\in\left\{  0,1,...,\mu\right\}
}a_{i}\underbrace{y^{i}x^{\mu}}_{\substack{\in\left\langle x^{0}%
,x^{1},...,x^{\mu}\right\rangle _{A}\\\text{by (\ref{L19-P2})}}}\in
\left\langle x^{0},x^{1},...,x^{\mu}\right\rangle _{A}$, because $\left\langle
x^{0},x^{1},...,x^{\mu}\right\rangle _{A}$ is an $A$-module).

Besides, for every $i\in\left\{  1,2,...,\mu\right\}  $, we have%
\begin{align}
y^{i}x^{\mu}  &  \in\left\langle x^{\mu-i}\right\rangle _{A}%
\ \ \ \ \ \ \ \ \ \ \left(  \text{by (\ref{L19-P1}), since }i\in\left\{
1,2,...,\mu\right\}  \text{ yields }i\in\left\{  0,1,...,\mu\right\}  \right)
\nonumber\\
&  \subseteq\left\langle x^{0},x^{1},...,x^{\mu-1}\right\rangle _{A}
\label{L19-P3}%
\end{align}
(since $\left\{  \mu-i\right\}  \subseteq\left\{  0,1,...,\mu-1\right\}  $,
because $\mu-i\in\left\{  0,1,...,\mu-1\right\}  $, since $i\in\left\{
1,2,...,\mu\right\}  $). Now,%
\begin{align}
&  \underbrace{\left\langle y^{1},y^{2},...,y^{\mu}\right\rangle _{A}%
}_{=\left\langle y^{i}\ \mid\ i\in\left\{  1,2,...,\mu\right\}  \right\rangle
_{A}}x^{\mu}=\left\langle y^{i}\ \mid\ i\in\left\{  1,2,...,\mu\right\}
\right\rangle _{A}x^{\mu}=\left\langle y^{i}x^{\mu}\ \mid\ i\in\left\{
1,2,...,\mu\right\}  \right\rangle _{A}\nonumber\\
&  =\left\{  \sum\limits_{i\in\left\{  1,2,...,\mu\right\}  }a_{i}y^{i}x^{\mu
}\ \mid\ \left(  a_{i}\right)  _{i\in\left\{  1,2,...,\mu\right\}  }\in
A^{\left\{  1,2,...,\mu\right\}  }\right\}  \subseteq\left\langle x^{0}%
,x^{1},...,x^{\mu-1}\right\rangle _{A}, \label{L19-Pb}%
\end{align}
since $\sum\limits_{i\in\left\{  1,2,...,\mu\right\}  }a_{i}y^{i}x^{\mu}%
\in\left\langle x^{0},x^{1},...,x^{\mu-1}\right\rangle _{A}$ for every
$\left(  a_{i}\right)  _{i\in\left\{  1,2,...,\mu\right\}  }\in A^{\left\{
1,2,...,\mu\right\}  }$ (since $\sum\limits_{i\in\left\{  1,2,...,\mu\right\}
}a_{i}\underbrace{y^{i}x^{\mu}}_{\substack{\in\left\langle x^{0}%
,x^{1},...,x^{\mu-1}\right\rangle _{A}\\\text{by (\ref{L19-P3})}}%
}\in\left\langle x^{0},x^{1},...,x^{\mu-1}\right\rangle _{A}$, because
$\left\langle x^{0},x^{1},...,x^{\mu-1}\right\rangle _{A}$ is an $A$-module).

Now, (\ref{L19-2}) yields%
\begin{align*}
u^{m}x^{\mu}  &  \in\left(  \left\langle u^{0},u^{1},...,u^{m-1}\right\rangle
_{A}\cdot\left\langle y^{0},y^{1},...,y^{\mu}\right\rangle _{A}+\left\langle
u^{0},u^{1},...,u^{m}\right\rangle _{A}\cdot\left\langle y^{1},y^{2}%
,...,y^{\mu}\right\rangle _{A}\right)  x^{\mu}\\
&  =\left\langle u^{0},u^{1},...,u^{m-1}\right\rangle _{A}\cdot
\underbrace{\left\langle y^{0},y^{1},...,y^{\mu}\right\rangle _{A}x^{\mu}%
}_{\substack{\subseteq\left\langle x^{0},x^{1},...,x^{\mu}\right\rangle
_{A}\\\left(  \text{by (\ref{L19-Pa})}\right)  }}+\left\langle u^{0}%
,u^{1},...,u^{m}\right\rangle _{A}\cdot\underbrace{\left\langle y^{1}%
,y^{2},...,y^{\mu}\right\rangle _{A}x^{\mu}}_{\substack{\subseteq\left\langle
x^{0},x^{1},...,x^{\mu-1}\right\rangle _{A}\\\left(  \text{by (\ref{L19-Pb}%
)}\right)  }}\\
&  \subseteq\left\langle u^{0},u^{1},...,u^{m-1}\right\rangle _{A}%
\cdot\left\langle x^{0},x^{1},...,x^{\mu}\right\rangle _{A}+\left\langle
u^{0},u^{1},...,u^{m}\right\rangle _{A}\cdot\left\langle x^{0},x^{1}%
,...,x^{\mu-1}\right\rangle _{A}.
\end{align*}
In other words, (\ref{L18-2}) holds. Also, (\ref{L18-1}) holds (because
(\ref{L19-1}) holds, and because (\ref{L18-1}) is the same as (\ref{L19-1})).
Thus, Lemma 18 yields that $u$ is $\left(  n\mu+m\nu\right)  $-integral over
$A$. This proves Lemma 19.

Something trivial now:

\begin{quote}
\textbf{Lemma 20.} Let $A$ and $B$ be two rings such that $A\subseteq B$. Let
$x\in B$. Let $n\in\mathbb{N}$. Let $u\in B$. Assume that $u$ is $n$-integral
over $A\left[  x\right]  $. Then, there exists some $\nu\in\mathbb{N}$ such
that%
\[
u^{n}\in\left\langle u^{0},u^{1},...,u^{n-1}\right\rangle _{A}\cdot
\left\langle x^{0},x^{1},...,x^{\nu}\right\rangle _{A}.
\]



\end{quote}

\textit{Proof of Lemma 20.} There exists a monic polynomial $P\in\left(
A\left[  x\right]  \right)  \left[  X\right]  $ with $\deg P=n$ and $P\left(
u\right)  =0$ (since $u$ is $n$-integral over $A\left[  x\right]  $). Since
$P\in\left(  A\left[  x\right]  \right)  \left[  X\right]  $ is a monic
polynomial with $\deg P=n$, there exist elements $\alpha_{0},$ $\alpha_{1},$
$...,$ $\alpha_{n-1}$ of $A\left[  x\right]  $ such that $P\left(  X\right)
=X^{n}+\sum\limits_{i=0}^{n-1}\alpha_{i}X^{i}$. Thus, $P\left(  u\right)
=u^{n}+\sum\limits_{i=0}^{n-1}\alpha_{i}u^{i}$, so that $P\left(  u\right)
=0$ becomes $u^{n}+\sum\limits_{i=0}^{n-1}\alpha_{i}u^{i}=0$. Hence,
$u^{n}=-\sum\limits_{i=0}^{n-1}\alpha_{i}u^{i}$.

For every $i\in\left\{  0,1,...,n-1\right\}  $, we have $\alpha_{i}\in
A\left[  x\right]  $, and thus there exist some $\nu_{i}\in\mathbb{N}$ and
some $\left(  \beta_{i,0},\beta_{i,1},...,\beta_{i,\nu_{i}}\right)  \in
A^{\nu_{i}+1}$ such that $\alpha_{i}=\sum\limits_{k=0}^{\nu_{i}}\beta
_{i,k}x^{k}$. Hence, $\alpha_{i}\in\left\langle x^{0},x^{1},...,x^{\nu_{i}%
}\right\rangle _{A}$ for every $i\in\left\{  0,1,...,n-1\right\}  $.

Let $\nu=\max\left\{  \nu_{0},\nu_{1},...,\nu_{n-1}\right\}  $. Then, for
every $i\in\left\{  0,1,...,n-1\right\}  $, we have $\nu_{i}\leq\nu$, hence
$\left\{  0,1,...,\nu_{i}\right\}  \subseteq\left\{  0,1,...,\nu\right\}  $,
thus $\left\langle x^{0},x^{1},...,x^{\nu_{i}}\right\rangle _{A}%
\subseteq\left\langle x^{0},x^{1},...,x^{\nu}\right\rangle _{A}$, and thus
$\alpha_{i}\in\left\langle x^{0},x^{1},...,x^{\nu}\right\rangle _{A}$ (since
$\alpha_{i}\in\left\langle x^{0},x^{1},...,x^{\nu_{i}}\right\rangle _{A}$).
Therefore,%
\begin{align*}
u^{n}  &  =-\sum\limits_{i=0}^{n-1}\alpha_{i}u^{i}=-\sum\limits_{i=0}%
^{n-1}\underbrace{u^{i}}_{\in\left\langle u^{0},u^{1},...,u^{n-1}\right\rangle
_{A}}\underbrace{\alpha_{i}}_{\in\left\langle x^{0},x^{1},...,x^{\nu
}\right\rangle _{A}}\\
&  \in-\sum\limits_{i=0}^{n-1}\left\langle u^{0},u^{1},...,u^{n-1}%
\right\rangle _{A}\cdot\left\langle x^{0},x^{1},...,x^{\nu}\right\rangle
_{A}\subseteq\left\langle u^{0},u^{1},...,u^{n-1}\right\rangle _{A}%
\cdot\left\langle x^{0},x^{1},...,x^{\nu}\right\rangle _{A}%
\end{align*}
(since $\left\langle u^{0},u^{1},...,u^{n-1}\right\rangle _{A}\cdot
\left\langle x^{0},x^{1},...,x^{\nu}\right\rangle _{A}$ is an $A$-module).
This proves Lemma 20.

A consequence of Lemmata 19 and 20 is the following theorem:

\begin{quote}
\textbf{Theorem 21.} Let $A$ and $B$ be two rings such that $A\subseteq B$.
Let $x\in B$ and $y\in B$ be such that $xy\in A$. Let $m\in\mathbb{N}$ and
$n\in\mathbb{N}$. Let $u\in B$. Assume that $u$ is $n$-integral over $A\left[
x\right]  $, and that $u$ is $m$-integral over $A\left[  y\right]  $. Then,
there exists some $\lambda\in\mathbb{N}$ such that $u$ is $\lambda$-integral
over $A$.
\end{quote}

\textit{Proof of Theorem 21.} Since $u$ is $n$-integral over $A\left[
x\right]  $, Lemma 20 yields that there exists some $\nu\in\mathbb{N}$ such
that%
\[
u^{n}\in\left\langle u^{0},u^{1},...,u^{n-1}\right\rangle _{A}\cdot
\left\langle x^{0},x^{1},...,x^{\nu}\right\rangle _{A}.
\]
In other words, (\ref{L19-1}) holds.

Since $u$ is $m$-integral over $A\left[  y\right]  $, Lemma 20 (with $x$, $n$
and $\nu$ replaced by $y$, $m$ and $\mu$) yields that there exists some
$\mu\in\mathbb{N}$ such that%
\[
u^{m}\in\left\langle u^{0},u^{1},...,u^{m-1}\right\rangle _{A}\cdot
\left\langle y^{0},y^{1},...,y^{\mu}\right\rangle _{A}.
\]
Hence,%
\[
u^{m}\in\left\langle u^{0},u^{1},...,u^{m-1}\right\rangle _{A}\cdot
\left\langle y^{0},y^{1},...,y^{\mu}\right\rangle _{A}+\left\langle
u^{0},u^{1},...,u^{m}\right\rangle _{A}\cdot\left\langle y^{1},y^{2}%
,...,y^{\mu}\right\rangle _{A}%
\]
(because%
\begin{align*}
&  \left\langle u^{0},u^{1},...,u^{m-1}\right\rangle _{A}\cdot\left\langle
y^{0},y^{1},...,y^{\mu}\right\rangle _{A}\\
&  \subseteq\left\langle u^{0},u^{1},...,u^{m-1}\right\rangle _{A}%
\cdot\left\langle y^{0},y^{1},...,y^{\mu}\right\rangle _{A}+\left\langle
u^{0},u^{1},...,u^{m}\right\rangle _{A}\cdot\left\langle y^{1},y^{2}%
,...,y^{\mu}\right\rangle _{A}%
\end{align*}
). In other words, (\ref{L19-2}) holds.

Since both (\ref{L19-1}) and (\ref{L19-2}) hold, Lemma 19 yields that $u$ is
$\left(  n\mu+m\nu\right)  $-integral over $A$. Thus, there exists some
$\lambda\in\mathbb{N}$ such that $u$ is $\lambda$-integral over $A$ (namely,
$\lambda=n\mu+m\nu$). This proves Theorem 21.

We record a generalization of Theorem 21 (which will turn out to be easily
seen equivalent to Theorem 21):

\begin{quote}
\textbf{Theorem 22.} Let $A$ and $B$ be two rings such that $A\subseteq B$.
Let $x\in B$ and $y\in B$. Let $m\in\mathbb{N}$ and $n\in\mathbb{N}$. Let
$u\in B$. Assume that $u$ is $n$-integral over $A\left[  x\right]  $, and that
$u$ is $m$-integral over $A\left[  y\right]  $. Then, there exists some
$\lambda\in\mathbb{N}$ such that $u$ is $\lambda$-integral over $A\left[
xy\right]  $.
\end{quote}

\textit{Proof of Theorem 22.} Obviously, $A\subseteq A\left[  xy\right]  $
yields $A\left[  x\right]  \subseteq\left(  A\left[  xy\right]  \right)
\left[  x\right]  $ and $A\left[  y\right]  \subseteq\left(  A\left[
xy\right]  \right)  \left[  y\right]  $.

Since $u$ is $n$-integral over $A\left[  x\right]  $, Lemma $\mathcal{I}$
(applied to $B$, $\left(  A\left[  xy\right]  \right)  \left[  x\right]  $,
$A\left[  x\right]  $ and $u$ instead of $B^{\prime}$, $A^{\prime}$, $A$ and
$v$) yields that $u$ is $n$-integral over $\left(  A\left[  xy\right]
\right)  \left[  x\right]  $.

Since $u$ is $m$-integral over $A\left[  y\right]  $, Lemma $\mathcal{I}$
(applied to $B$, $\left(  A\left[  xy\right]  \right)  \left[  y\right]  $,
$A\left[  y\right]  $, $m$ and $u$ instead of $B^{\prime}$, $A^{\prime}$, $A$,
$n$ and $v$) yields that $u$ is $m$-integral over $\left(  A\left[  xy\right]
\right)  \left[  y\right]  $.

Now, Theorem 21 (applied to $A\left[  xy\right]  $ instead of $A$) yields that
there exists some $\lambda\in\mathbb{N}$ such that $u$ is $\lambda$-integral
over $A\left[  xy\right]  $ (because $xy\in A\left[  xy\right]  $, because $u$
is $n$-integral over $\left(  A\left[  xy\right]  \right)  \left[  x\right]
$, and because $u$ is $m$-integral over $\left(  A\left[  xy\right]  \right)
\left[  y\right]  $). This proves Theorem 22.

Theorem 22 has a "relative version":

\begin{quote}
\textbf{Theorem 23.} Let $A$ and $B$ be two rings such that $A\subseteq B$.
Let $\left(  I_{\rho}\right)  _{\rho\in\mathbb{N}}$ be an ideal semifiltration
of $A$. Let $x\in B$ and $y\in B$.

\textbf{(a)} Then, $\left(  I_{\rho}A\left[  x\right]  \right)  _{\rho
\in\mathbb{N}}$ is an ideal semifiltration of $A\left[  x\right]  $. Besides,
$\left(  I_{\rho}A\left[  y\right]  \right)  _{\rho\in\mathbb{N}}$ is an ideal
semifiltration of $A\left[  y\right]  $. Besides, $\left(  I_{\rho}A\left[
xy\right]  \right)  _{\rho\in\mathbb{N}}$ is an ideal semifiltration of
$A\left[  xy\right]  $.

\textbf{(b)} Let $m\in\mathbb{N}$ and $n\in\mathbb{N}$. Let $u\in B$. Assume
that $u$ is $n$-integral over $\left(  A\left[  x\right]  ,\left(  I_{\rho
}A\left[  x\right]  \right)  _{\rho\in\mathbb{N}}\right)  $, and that $u$ is
$m$-integral over $\left(  A\left[  y\right]  ,\left(  I_{\rho}A\left[
y\right]  \right)  _{\rho\in\mathbb{N}}\right)  $. Then, there exists some
$\lambda\in\mathbb{N}$ such that $u$ is $\lambda$-integral over $\left(
A\left[  xy\right]  ,\left(  I_{\rho}A\left[  xy\right]  \right)  _{\rho
\in\mathbb{N}}\right)  $.
\end{quote}

\textit{Proof of Theorem 23.} \textbf{(a)} Since $\left(  I_{\rho}\right)
_{\rho\in\mathbb{N}}$ is an ideal semifiltration of $A$, Lemma $\mathcal{J}$
(applied to $A\left[  x\right]  $ instead of $A^{\prime}$) yields that
$\left(  I_{\rho}A\left[  x\right]  \right)  _{\rho\in\mathbb{N}}$ is an ideal
semifiltration of $A\left[  x\right]  $.

Since $\left(  I_{\rho}\right)  _{\rho\in\mathbb{N}}$ is an ideal
semifiltration of $A$, Lemma $\mathcal{J}$ (applied to $A\left[  y\right]  $
instead of $A^{\prime}$) yields that $\left(  I_{\rho}A\left[  y\right]
\right)  _{\rho\in\mathbb{N}}$ is an ideal semifiltration of $A\left[
y\right]  $.

Since $\left(  I_{\rho}\right)  _{\rho\in\mathbb{N}}$ is an ideal
semifiltration of $A$, Lemma $\mathcal{J}$ (applied to $A\left[  xy\right]  $
instead of $A^{\prime}$) yields that $\left(  I_{\rho}A\left[  xy\right]
\right)  _{\rho\in\mathbb{N}}$ is an ideal semifiltration of $A\left[
xy\right]  $.

Thus, Theorem 23 \textbf{(a)} is proven.

\textbf{(b)} We formulate a lemma:

\textit{Lemma} $\mathcal{N}$\textit{:} Let $A$, $A^{\prime}$ and $B$ be three
rings such that $A\subseteq A^{\prime}\subseteq B$. Let $v\in B$. Let $\left(
I_{\rho}\right)  _{\rho\in\mathbb{N}}$ be an ideal semifiltration of $A$.
Consider the polynomial ring $A\left[  Y\right]  $ and its $A$-subalgebra
$A\left[  \left(  I_{\rho}\right)  _{\rho\in\mathbb{N}}\ast Y\right]  $. We
have $A\left[  \left(  I_{\rho}\right)  _{\rho\in\mathbb{N}}\ast Y\right]
\subseteq A\left[  Y\right]  $, and (as explained in Definition 7) we can
identify the polynomial ring $A\left[  Y\right]  $ with a subring of $\left(
A\left[  v\right]  \right)  \left[  Y\right]  $ (since $A\subseteq A\left[
v\right]  $). Hence, $A\left[  \left(  I_{\rho}\right)  _{\rho\in\mathbb{N}%
}\ast Y\right]  \subseteq\left(  A\left[  v\right]  \right)  \left[  Y\right]
$. On the other hand, $\left(  A\left[  v\right]  \right)  \left[  \left(
I_{\rho}A\left[  v\right]  \right)  _{\rho\in\mathbb{N}}\ast Y\right]
\subseteq\left(  A\left[  v\right]  \right)  \left[  Y\right]  $.

\textbf{(a)} We have%
\[
\left(  A\left[  v\right]  \right)  \left[  \left(  I_{\rho}A\left[  v\right]
\right)  _{\rho\in\mathbb{N}}\ast Y\right]  =\left(  A\left[  \left(  I_{\rho
}\right)  _{\rho\in\mathbb{N}}\ast Y\right]  \right)  \left[  v\right]  .
\]


\textbf{(b)} Let $u\in B$. Let $n\in\mathbb{N}$. Then, the element $u$ of $B$
is $n$-integral over $\left(  A\left[  v\right]  ,\left(  I_{\rho}A\left[
v\right]  \right)  _{\rho\in\mathbb{N}}\right)  $ if and only if the element
$uY$ of the polynomial ring $B\left[  Y\right]  $ is $n$-integral over the
ring $\left(  A\left[  \left(  I_{\rho}\right)  _{\rho\in\mathbb{N}}\ast
Y\right]  \right)  \left[  v\right]  $.

\textit{Proof of Lemma }$\mathcal{N}$\textit{:} \textbf{(a)} We have proven
Lemma $\mathcal{N}$ \textbf{(a)} during the proof of Theorem 9 \textbf{(b)}.

\textbf{(b)} Theorem 7 (applied to $A\left[  v\right]  $ and $\left(  I_{\rho
}A\left[  v\right]  \right)  _{\rho\in\mathbb{N}}$ instead of $A$ and $\left(
I_{\rho}\right)  _{\rho\in\mathbb{N}}$) yields that the element $u$ of $B$ is
$n$-integral over $\left(  A\left[  v\right]  ,\left(  I_{\rho}A\left[
v\right]  \right)  _{\rho\in\mathbb{N}}\right)  $ if and only if the element
$uY$ of the polynomial ring $B\left[  Y\right]  $ is $n$-integral over the
ring $\left(  A\left[  v\right]  \right)  \left[  \left(  I_{\rho}A\left[
v\right]  \right)  _{\rho\in\mathbb{N}}\ast Y\right]  $. In other words, the
element $u$ of $B$ is $n$-integral over $\left(  A\left[  v\right]  ,\left(
I_{\rho}A\left[  v\right]  \right)  _{\rho\in\mathbb{N}}\right)  $ if and only
if the element $uY$ of the polynomial ring $B\left[  Y\right]  $ is
$n$-integral over the ring $\left(  A\left[  \left(  I_{\rho}\right)
_{\rho\in\mathbb{N}}\ast Y\right]  \right)  \left[  v\right]  $ (because Lemma
$\mathcal{N}$ \textbf{(a)} yields $\left(  A\left[  v\right]  \right)  \left[
\left(  I_{\rho}A\left[  v\right]  \right)  _{\rho\in\mathbb{N}}\ast Y\right]
=\left(  A\left[  \left(  I_{\rho}\right)  _{\rho\in\mathbb{N}}\ast Y\right]
\right)  \left[  v\right]  $). This proves Lemma $\mathcal{N}$ \textbf{(b)}.

Now, let us prove Theorem 23 \textbf{(b)}. In fact, for every $v\in B$, we can
consider the polynomial ring $\left(  A\left[  v\right]  \right)  \left[
Y\right]  $ and its $A\left[  v\right]  $-subalgebra $\left(  A\left[
v\right]  \right)  \left[  \left(  I_{\rho}A\left[  v\right]  \right)
_{\rho\in\mathbb{N}}\ast Y\right]  $. We have $\left(  A\left[  v\right]
\right)  \left[  \left(  I_{\rho}A\left[  v\right]  \right)  _{\rho
\in\mathbb{N}}\ast Y\right]  \subseteq\left(  A\left[  v\right]  \right)
\left[  Y\right]  $, and (as explained in Definition 7) we can identify the
polynomial ring $\left(  A\left[  v\right]  \right)  \left[  Y\right]  $ with
a subring of $B\left[  Y\right]  $ (since $A\left[  v\right]  \subseteq B$).
Hence, $\left(  A\left[  v\right]  \right)  \left[  \left(  I_{\rho}A\left[
v\right]  \right)  _{\rho\in\mathbb{N}}\ast Y\right]  \subseteq B\left[
Y\right]  $.

Lemma $\mathcal{N}$ \textbf{(b)} (applied to $x$ instead of $v$) yields that
the element $u$ of $B$ is $n$-integral over $\left(  A\left[  x\right]
,\left(  I_{\rho}A\left[  x\right]  \right)  _{\rho\in\mathbb{N}}\right)  $ if
and only if the element $uY$ of the polynomial ring $B\left[  Y\right]  $ is
$n$-integral over the ring $\left(  A\left[  \left(  I_{\rho}\right)
_{\rho\in\mathbb{N}}\ast Y\right]  \right)  \left[  x\right]  $. But since the
element $u$ of $B$ is $n$-integral over $\left(  A\left[  x\right]  ,\left(
I_{\rho}A\left[  x\right]  \right)  _{\rho\in\mathbb{N}}\right)  $, this
yields that the element $uY$ of the polynomial ring $B\left[  Y\right]  $ is
$n$-integral over the ring $\left(  A\left[  \left(  I_{\rho}\right)
_{\rho\in\mathbb{N}}\ast Y\right]  \right)  \left[  x\right]  $.

Lemma $\mathcal{N}$ \textbf{(b)} (applied to $y$ and $m$ instead of $v$ and
$n$) yields that the element $u$ of $B$ is $m$-integral over $\left(  A\left[
y\right]  ,\left(  I_{\rho}A\left[  y\right]  \right)  _{\rho\in\mathbb{N}%
}\right)  $ if and only if the element $uY$ of the polynomial ring $B\left[
Y\right]  $ is $m$-integral over the ring $\left(  A\left[  \left(  I_{\rho
}\right)  _{\rho\in\mathbb{N}}\ast Y\right]  \right)  \left[  y\right]  $. But
since the element $u$ of $B$ is $m$-integral over $\left(  A\left[  y\right]
,\left(  I_{\rho}A\left[  y\right]  \right)  _{\rho\in\mathbb{N}}\right)  $,
this yields that the element $uY$ of the polynomial ring $B\left[  Y\right]  $
is $m$-integral over the ring $\left(  A\left[  \left(  I_{\rho}\right)
_{\rho\in\mathbb{N}}\ast Y\right]  \right)  \left[  y\right]  $.

Since $uY$ is $n$-integral over the ring $\left(  A\left[  \left(  I_{\rho
}\right)  _{\rho\in\mathbb{N}}\ast Y\right]  \right)  \left[  x\right]  $, and
since $uY$ is $m$-integral over the ring $\left(  A\left[  \left(  I_{\rho
}\right)  _{\rho\in\mathbb{N}}\ast Y\right]  \right)  \left[  y\right]  $,
Theorem 22 (applied to $A\left[  \left(  I_{\rho}\right)  _{\rho\in\mathbb{N}%
}\ast Y\right]  $, $B\left[  Y\right]  $ and $uY$ instead of $A$, $B$ and $u$)
yields that there exists some $\lambda\in\mathbb{N}$ such that $uY$ is
$\lambda$-integral over $\left(  A\left[  \left(  I_{\rho}\right)  _{\rho
\in\mathbb{N}}\ast Y\right]  \right)  \left[  xy\right]  $.

Lemma $\mathcal{N}$ \textbf{(b)} (applied to $xy$ and $\lambda$ instead of $v$
and $n$) yields that the element $u$ of $B$ is $\lambda$-integral over
$\left(  A\left[  xy\right]  ,\left(  I_{\rho}A\left[  xy\right]  \right)
_{\rho\in\mathbb{N}}\right)  $ if and only if the element $uY$ of the
polynomial ring $B\left[  Y\right]  $ is $\lambda$-integral over the ring
$\left(  A\left[  \left(  I_{\rho}\right)  _{\rho\in\mathbb{N}}\ast Y\right]
\right)  \left[  xy\right]  $. But since the element $uY$ of the polynomial
ring $B\left[  Y\right]  $ is $\lambda$-integral over the ring $\left(
A\left[  \left(  I_{\rho}\right)  _{\rho\in\mathbb{N}}\ast Y\right]  \right)
\left[  xy\right]  $, this yields that the element $u$ of $B$ is $\lambda
$-integral over $\left(  A\left[  xy\right]  ,\left(  I_{\rho}A\left[
xy\right]  \right)  _{\rho\in\mathbb{N}}\right)  $. Thus, Theorem 23
\textbf{(b)} is proven.

We notice that Corollary 3 can be derived from Lemma 18:

\textit{Second proof of Corollary 3.} Let $n=1$. Let $m=1$. We have%
\[
u^{n}\in\left\langle u^{0},u^{1},...,u^{n-1}\right\rangle _{A}\cdot
\left\langle v^{0},v^{1},...,v^{\alpha}\right\rangle _{A}%
\]
\footnote{because%
\begin{align*}
u^{n}  &  =u^{1}=u=\sum\limits_{i=0}^{\alpha}\underbrace{s_{i}}_{\in A}%
v^{i}\in\left\langle v^{0},v^{1},...,v^{\alpha}\right\rangle _{A}%
=A\cdot\left\langle v^{0},v^{1},...,v^{\alpha}\right\rangle _{A}\\
&  =\left\langle u^{0},u^{1},...,u^{n-1}\right\rangle _{A}\cdot\left\langle
v^{0},v^{1},...,v^{\alpha}\right\rangle _{A}%
\end{align*}
(since $A=\left\langle 1\right\rangle _{A}=\left\langle u^{0}\right\rangle
_{A}=\left\langle u^{0},u^{1},...,u^{n-1}\right\rangle _{A}$, as $n=1$)} and%
\[
u^{m}v^{\beta}\in\left\langle u^{0},u^{1},...,u^{m-1}\right\rangle _{A}%
\cdot\left\langle v^{0},v^{1},...,v^{\beta}\right\rangle _{A}+\left\langle
u^{0},u^{1},...,u^{m}\right\rangle _{A}\cdot\left\langle v^{0},v^{1}%
,...,v^{\beta-1}\right\rangle _{A}%
\]
\footnote{because%
\begin{align*}
\underbrace{u^{m}}_{=u^{1}=u}v^{\beta}  &  =uv^{\beta}=\sum\limits_{i=0}%
^{\beta}t_{i}v^{\beta-i}=\sum\limits_{i=0}^{\beta}t_{\beta-i}v^{\beta-\left(
\beta-i\right)  }\ \ \ \ \ \ \ \ \ \ \left(  \text{here we substituted }%
\beta-i\text{ for }i\text{ in the sum}\right) \\
&  =\sum\limits_{i=0}^{\beta}\underbrace{t_{\beta-i}}_{\in A}v^{i}%
\in\left\langle v^{0},v^{1},...,v^{\beta}\right\rangle _{A}=A\cdot\left\langle
v^{0},v^{1},...,v^{\beta}\right\rangle _{A}\\
&  =\left\langle u^{0},u^{1},...,u^{m-1}\right\rangle _{A}\cdot\left\langle
v^{0},v^{1},...,v^{\beta}\right\rangle _{A}%
\end{align*}
(since $A=\left\langle 1\right\rangle _{A}=\left\langle u^{0}\right\rangle
_{A}=\left\langle u^{0},u^{1},...,u^{m-1}\right\rangle _{A}$, as $m=1$) and
\begin{align*}
&  \left\langle u^{0},u^{1},...,u^{m-1}\right\rangle _{A}\cdot\left\langle
v^{0},v^{1},...,v^{\beta}\right\rangle _{A}\\
&  \subseteq\left\langle u^{0},u^{1},...,u^{m-1}\right\rangle _{A}%
\cdot\left\langle v^{0},v^{1},...,v^{\beta}\right\rangle _{A}+\left\langle
u^{0},u^{1},...,u^{m}\right\rangle _{A}\cdot\left\langle v^{0},v^{1}%
,...,v^{\beta-1}\right\rangle _{A}%
\end{align*}
}. Thus, Lemma 18 (applied to $v$, $\beta$ and $\alpha$ instead of $x$, $\mu$
and $\nu$) yields that $u$ is $\left(  n\beta+m\alpha\right)  $-integral over
$A$. This means that $u$ is $\left(  \alpha+\beta\right)  $-integral over $A$
(because $n\beta+m\alpha=1\beta+1\alpha=\beta+\alpha=\alpha+\beta$). This
proves Corollary 3 once again.

In how far does this all generalize Theorem 2 from [3]? Actually, Theorem 2
from [3] can be easily reduced to the case when $J=0$ (by passing from the
ring $A$ to its localization $A_{1+J}$), and in this case it easily follows
from Lemma 18.

\begin{center}
\color{blue} \textbf{References} \color{black}
\end{center}

[1] J. S. Milne, \textit{Algebraic Number Theory}, version 3.02.\newline%
\texttt{http://www.jmilne.org/math/CourseNotes/ant.html}

[2] Craig Huneke and Irena Swanson, \textit{Integral Closure of Ideals, Rings,
and Modules}, London Mathematical Society Lecture Note Series, 336. Cambridge
University Press, Cambridge, 2006.\newline%
\texttt{http://people.reed.edu/\symbol{126}iswanson/book/index.html}

[3] Henri Lombardi, \textit{Hidden constructions in abstract algebra (1)
Integral dependance relations}, Journal of Pure and Applied Algebra 167
(2002), pp. 259-267.\newline%
\texttt{http://hlombardi.free.fr/publis/IntegralDependance.ps}

[4] Darij Grinberg, \textit{A few facts on integrality *BRIEF\ VERSION*}%
.\newline\texttt{http://www.cip.ifi.lmu.de/\symbol{126}%
grinberg/IntegralityBRIEF.pdf}


\end{document}