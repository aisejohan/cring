 \stepcounter{lecture}
 \setcounter{lecture}{39}
 \sektion{Lecture 39}

 Final Exam: Do three exercises. \\
 $R$ is a domain throughout. $C(R)=Inv(R)/Prin(R)$ is the ideal class group. The
 following are equivalent.
 \begin{itemize}
   \item $R$ is a Dedekind domain
   \item $R$ is noetherian, normal, and of dimension $\le 1$ (this is the definition)
   \item non-zero ideals are invertible
   \item for all $I\< R$, $I=\p_1^{r_1}\cdots \p_n^{r_n}$ uniquely (except for $I=0$),
   with $n\ge 0$ and $r_i\ge 1$.
 \end{itemize}
 This last condition is a very important one.

 Robert Lee Moore's method: he would write all the theorems and definitions in a subject and
 have his students figure out the proofs.

 \underline{A dozen things every Good Algebraist should know about Dedekind domains}. $R$
 is a Dedekind domain.
 \begin{enumerate}
   \item $R$ is local $\Longleftrightarrow$ $R$ is a field or a DVR.
   \item $R$ semi-local $\Longrightarrow$ it is a PID.
   \item $R$ is a PID $\Longleftrightarrow$ it is a UFD $\Longleftrightarrow$ $C(R)=\{1\}$
   \item $R$ is the full ring of integers of a number field $K$ $\Longrightarrow$ $|C(R)|<
   \infty$, and this number is the \emph{class number} of $K$.
   \item $C(R)$ can be any abelian group. This is Clayborn's Theorem.
   \item For any non-zero prime $\p\in \spec R$, $\p^n/\p^{n+1}\cong R/\p$ as an
   $R$-module.
   \item ``To contain is to divide'', i.e.~if $A,B\< R$, then $A\subseteq B$
   $\Longleftrightarrow$ $A=BC$ for some $C\< R$.
   \item (Generation of ideals) Every non-zero ideal $B\< R$ is generated by two elements.
   Moreover, one of the generators can be taken to be any non-zero element of $B$.
   \item (Factor rings) If $A\< R$ is non-zero, then $R/A$ is a PIR.
   \item (Steinitz Isomorphisms Theorem) If $A,B\< R$ are non-zero ideals, then $A\oplus
   B\cong {}_RR\oplus AB$ as $R$-modules.
   \item If ${}_RM$ is a finitely generated torsion-free $R$-module of rank $n$,\footnote{The
   rank is defined as $rk(M)=\dim_{Q(R)} M\otimes_R Q(R)$.} then it is of the
   form $M\cong R^{n-1}\oplus A$, where $A$ is a non-zero ideal, determined up to
   isomorphism.
   \item If ${}_RM$ is a finitely generated torsion $R$-module, then $M$ is uniquely of the
   form $M\cong R/A_1\oplus
   \cdots \oplus R/A_n$ with $A_1\subsetneq A_2\subsetneq \cdots \subsetneq
   A_n\subsetneq R$.
   \item[Bonus.] Any finitely generated ${}_RM$ can be written as $M\cong M_t \oplus
   M/M_t$, where $M_t$ is the torsion submodule.
 \end{enumerate}

 \begin{theorem}[Very Strong Krull-Akizuki Theorem]
   Let $R$ be a noetherian domain of Krull dimension $1$, with $K=Q(R)$. Let $L/K$ be a
   finite field extension, and let $S$ be a ring $R\subseteq S\subseteq L$. Then $S$ is
   noetherian of dimension $\le 1$.
 \end{theorem}
 \begin{corollary}
   If the $R$ is also normal (so it is Dedekind), then the integral closure of $R$ in $L$
   is a Dedekind domain.
 \end{corollary}
 We knew this result in the special case when $L/K$ is a finite separable extension.

 \underline{Pr\"ufer Domains} in some sense generalize Dedekind to non-noetherian and
 higher-dimensional cases.
 \begin{definition}
   A domain $R$ is \emph{Pr\"ufer} if every non-zero finitely generated ideal is
   invertible.
 \end{definition}
 Clearly Pr\"ufer and noetherian is Dedekind.
 \begin{example}
   $\{$Valuation rings$\}\subseteq \{$B\'ezout
     domains$\}\subseteq\{$Pr\"ufer$\}$.
 \end{example}
 There are about 20 different characterization of Pr\"ufer domains. Here are a few.
 \begin{theorem}
   Let $R$ be a domain. The following are equivalent.
   \begin{enumerate}
     \item $R$ is Pr\"ufer.
     \item For any $\m\in \Max R$, $R_\m$ is a valuation ring.
     \item $A\cap (B+C)=(A\cap B)+(A\cap C)$ for all ideals $A,B,C\< R$.
     \item All ideals in $R$ are flat modules.
   \end{enumerate}
 \end{theorem}

 \underline{Krull domains \& Divisors}.\\
 Mori's heartbreak story. Let $R$ be a ring with quotient field $K$. We'd like to form
 $R^*$, the integral closure of $R$. Mori discovered that if $R$ is noetherian, $R^*$
 need not be noetherian.

 But the implication $R$ noetherian implies $R^*$ noetherian holds in the following
 cases:
 \begin{enumerate}
   \item $\dim R=1$, by the Very Strong Krull-Akizuki Theorem, with $L=K$.
   \item (Mori, Nagata) $\dim R=2$.
   \item $R$ is an affine algebra.
 \end{enumerate}
 Grothendieck defined \emph{Japanese rings}, which have to do with this stuff.
 \begin{definition}
   $R\subseteq K$ as usual is a \emph{Krull domain} if there exists a family
   $\{R_i\}_{i\in I}\subseteq DVal(K)$ (DVRs of $K$) such that
   \begin{enumerate}
     \item $R = \bigcap R_i$
     \item for all non-zero $a\in R$, $a\in U(R_i)$ for almost all $i$.
   \end{enumerate}
 \end{definition}
 From the viewpoint of valuation theory, let $v_i:R_i\twoheadrightarrow \ZZ\cup \infty$,
 then $R=\{x\in K|v_i(x)\ge 0 \text{ for all }i\}$ and for all non-zero $x\in R$,
 $v_i(x)=0$ for almost all $i$.

 The set $\{R_i\}$ is called the \emph{defining family} of $R$.
