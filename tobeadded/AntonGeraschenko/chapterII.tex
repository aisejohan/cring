\documentclass{article}
\usepackage{amsmath,
            latexsym,
            amssymb}
\usepackage[all]{xy}

 \newcommand\<{\triangleleft}
 \renewcommand{\a}{\ensuremath{\mathfrak{a}}}
 \DeclareMathOperator{\ann}{ann}
 \DeclareMathOperator{\ass}{Ass}
 \DeclareMathOperator{\aut}{Aut}
 \newcommand\C{\mathcal{C}}
 \newcommand{\CC}{\ensuremath{\mathbb{C}}}
 \newcommand{\D}{\ensuremath{\mathcal{D}}}
 \DeclareMathOperator{\End}{End}
 \newcommand{\F}{\ensuremath{\mathcal{F}}}
 \newcommand{\FF}{\ensuremath{\mathbb{F}}}
 \newcommand{\HH}{\ensuremath{\mathbb{H}}}
 \let\hom\relax % kills the old hom
 \DeclareMathOperator{\hom}{Hom}
 \renewcommand{\labelitemi}{--}                    % changes the default bullet in itemize
 \newcommand{\id}{\mathrm{Id}}
 \DeclareMathOperator{\im}{im}
 \newcommand{\m}{\ensuremath{\mathfrak{m}}}
 \DeclareMathOperator{\Max}{Max}
 \DeclareMathOperator{\Min}{Min}
 \newcommand{\MM}{\ensuremath{\mathbb{M}}}
 \DeclareMathOperator{\nil}{Nil}
 \newcommand\p{\mathfrak{p}}
 \renewcommand\P{\mathcal{P}}
 \newcommand\q{\mathfrak{q}}
 \newcommand{\QQ}{\ensuremath{\mathbb{Q}}}
 \DeclareMathOperator{\rad}{rad}
 \newcommand{\RR}{\ensuremath{\mathbb{R}}}
 \newcommand{\smaltrix}[4]{\ensuremath{\left( %
            \begin{smallmatrix} #1 & #2 \\ #3 & #4 \end{smallmatrix} \right)}}
 \DeclareMathOperator{\spec}{Spec}
 \DeclareMathOperator{\supp}{Supp}
 \newcommand{\V}{\mathcal{V}}
 \newcommand\Z{\mathcal{Z}}
 \newcommand{\ZZ}{\ensuremath{\mathbb{Z}}}

 \openout0=\jobname solved.txt
 \openout1=lastupdated.html
 \newenvironment{exercise}[1]{\gdef\currentEx{#1}\begin{trivlist}\item[]%
                \textbf{Exercise #1.} \it}{\end{trivlist}}
 \makeatletter
 \newenvironment{solution}[1]{\def\x{#1}\begin{trivlist}\item[]\hspace*{-.5em}[\x]}
                {\hspace*{\fill} $\blacksquare$
                \protected@write0{}{\currentEx, \x}
                \end{trivlist}}
 \makeatother

\begin{document}

 \write1{\today}
 \closeout1

 {\large \noindent Solutions for Chapter II. Updated \today.}

 \begin{exercise}{II.1}
   Let $f\in R$ be a regular element. Show that $f\in U(R)$ if and only if the
   localization $R[f^{-1}]$ is module-finite over $R$.
 \end{exercise}
 \begin{solution}{anton@math}
   If $f\in U(R)$, then $R[f^{-1}]=R$ is clearly finitely generated as a module over $R$.
   On the other hand, if $R[f^{-1}]$ is finitely generated as a module over $R$, say by
   $\displaystyle\left\{\frac{r_1}{f^{n_1}},\dots, \frac{r_m}{f^{n_m}}\right\}$, then let
   $N=\max\{n_i\}$. Now we have that $R[f^{-1}]\subseteq \frac{1}{f^{N}}R$. In
   particular, $f^{-N-1}=f^{-N}r$ for some $r\in R$, so $1=fr$ (in $R[f^{-1}]$). Since
   $f$ is a regular element, $R$ injects into $R[f^{-1}]$, so the equality $1=fr$ holds
   in $R$, so $f\in U(R)$.
 \end{solution}

 \begin{exercise}{II.2}
  (The Rabinowitch Trick). For $J \< R$ and $g \in R$, show that $g
  \in \sqrt{J}$ iff $J$ and $1-tg$ generate the unit ideal in the
  polynomial ring $R[t]$. Use this to show that the Strong
  Nullstellensatz (4.17) can be deduced from the Weak Nullstellensatz
  (4.9) by ``adding a variable".
 \end{exercise}
 \begin{solution}{David Brown [brownda@math]}
  ``$\Rightarrow$" Suppose $g^{n} \in J$. Then as $1-tg$ divides
  $1-(tg)^{n}$, we have $1 = (tg)^{n} + (1-(tg)^{n})$ is in the ideal
  generated by $J$ and $1-tg$.

  ``$\Leftarrow$" Suppose that $J$ and $1-tg$ generate the unit ideal. Then
  modulo $J$,   $1 = (1-tg)f(t)$ for some $f(t) = \sum
  a_{i}t^{i} \in R[t]$, say of degree $n$. Then we get $n+1$ many equations mod $J$:
  \[
  \begin{array}{rcl}
  a_{0} & = &  1\\
  a_{1} & = & a_{0}g \\
   & \cdots   &  \\
  a_{i} & =  &  a_{i}g\\
  & \cdots   &  \\
  a_{n-1} & =  &  a_{n-2}g\\
  a_{n}g & = &  0
  \end{array}
  \]
  and see inductively that $g^n \in J$.

  ``\textbf{Weak Nullstellensatz} $\Rightarrow$ \textbf{Strong
  Nullstellensatz}'' Suppose $g \in \sqrt{J}$.
  Then for some $m$, $g^m \in J$, and so for any $a \in
  V_{\overline{k}}(J)$, $g^m(a) = g(a)^m = 0$. But $k$ is a field so
  $g(a) = 0$. Conversely, suppose that $I := <J,1-tg>$ is a proper
  ideal of $R[t]$. Then by the Weak Nullstellensatz, we can find $a =
  (a_1,\cdots,a_n,b) \in V_{\overline{k}}(I)$. Then $1-bg(a) = 0$, so
  $g(a_1,\cdots,a_n) \neq 0$ (else 0 = 1). Furthermore $f(a) =
  f(a_1,\cdots,a_n) = 0$, so $g \not \in V_{\overline{k}}(J)$.
 \end{solution}

 \begin{exercise}{II.3}
   For the ideal $J=\bigl( xy,yz,zx,(x-y)(x+1)\bigr)\< k[x,y,z]$, determine the
   irreducible $k$-components of the $k$-algebraic set $V_{\bar k}(J)$.
 \end{exercise}
 \begin{solution}{anton@math}
   I claim that the decomposition $V_{\bar k}(J) = V_{\bar k}(x,y)\cup V_{\bar
   k}(x+1,y,z)$ gives the irreducible components of $V_{\bar k}(J)$. Since $(x,y)$ and
   $(x+1,y,z)$ are prime (the quotients rings are clearly domains), the Nullstellensatz
   tells us that the sets $V_{\bar k}(x,y)$ and $V_{\bar k}(x+1,y,z)$ are irreducible,
   and there is no containment relation between them. Thus, it is enough to verify the
   equality. It is immediate that $J\subseteq (x,y)$ and $J\subseteq (x+1,y,z)$, so we
   get the containment $\supseteq$. For the other containment, assume $(a,b,c)\in \bar
   k^3$ is in $V_{\bar k}(J)$. Then $ab=bc=ac=0$, so two of $a,b,c$ are zero. Moreover,
   we have that either $a=b$ or $a+1=0$. If $a=b$, then they must both be zero, so
   $(a,b,c)\in V_{\bar k}(x,y)$. If $a+1=0$, then $b=c=0$, so $(a,b,c)\in V_{\bar
   k}(x+1,y,z)$.
 \end{solution}

 \begin{exercise}{II.4}
  Over a field $k=\overline{k}$, let $Y=\left\{  \left(  a^{p},a^{q}\right)
  :a\in k\right\}  $, where $p$, $q$ are fixed positive integers. Show that $Y$
  is a $k$-algebraic set, and determine the ideal $I\left(  Y\right)  \subseteq
  k\left[  x,y\right]  $.
 \end{exercise}
 \begin{solution}{Manuel Reyes; mreyes@math}
  I claim that it is enough to consider the case where $p$ and $q$ are
  relatively prime. Indeed, let $d=\gcd\left(  p,q\right)  $ and write
  $p=dp^{\prime}$, $q=dq^{\prime}$. Because $k$ is algebraically closed, every
  element has a $d$-th root so that $k^{d}=k$. Then
  \begin{align*}
  Y  &  =\left\{  \left(  a^{p},a^{q}\right)  :a\in k\right\} \\
  &  =\left\{  \left(  a^{dp^{\prime}},a^{dq^{\prime}}\right)  :a\in k\right\}
  \\
  &  =\left\{  \left(  b^{p^{\prime}},b^{q^{\prime}}\right)  :b\in k^{d}\right\}
  \\
  &  =\left\{  \left(  b^{p^{\prime}},b^{q^{\prime}}\right)  :b\in k\right\}
  \text{.}%
  \end{align*}
  So we may replace $p$ and $q$ by $p^{\prime}$ and $q^{\prime}$ if necessary so
  that $\left(  p,q\right)  =1$.

  Next we show that $Y=V_{k}\left(  x^{q}-y^{p}\right)  $. Clearly $Y\subseteq
  V_{k}\left(  x^{q}-y^{p}\right)  $; for the other inclusion, assume that
  $\left(  u,v\right)  \in V_{k}\left(  x^{q}-y^{p}\right)  $. If either $u$ or
  $v$ is zero then $\left(  u,v\right)  =\left(  0,0\right)  \in Y$, so we may
  assume that $u,v\neq0$. Because $k$ is algebraically closed, there exists
  $b\in k$ with $b^{p}=u$. Consider then that
  \[
  \left(  b^{q}\right)  ^{p}=b^{pq}=u^{q}=v^{p}\text{.}%
  \]
  It follows that $\zeta:=\frac{v}{b^{q}}$ is a $p$-th root of unity in $k$.
  Because $\left(  p,q\right)  =1$, $\overline{q}$ is a unit in $%
  %TCIMACRO{\U{2124} }%
  %BeginExpansion
  \mathbb{Z}
  %EndExpansion
  /p%
  %TCIMACRO{\U{2124} }%
  %BeginExpansion
  \mathbb{Z}
  %EndExpansion
  $. So there exists $m\in%
  %TCIMACRO{\U{2124} }%
  %BeginExpansion
  \mathbb{Z}
  %EndExpansion
  $ such that $mq\equiv1\left(  \operatorname{mod}p\right)  $. Then $\zeta^{m}$
  is a $p$-th root of unity with $\left(  \zeta^{m}\right)  ^{q}=\zeta$. It
  follows that for $a=\zeta^{m}b$ we have $a^{p}=b^{p}=u$~and$~a^{q}=\zeta
  b^{q}=v$. So $\left(  u,v\right)  =\left(  a^{p},a^{q}\right)  \in Y$. This
  means that $Y=V_{k}\left(  x^{q}-y^{p}\right)  $ is a $k$-algebraic set.

  Now because $k=\overline{k}$ Proposition 1.7 implies that $I\left(  Y\right)
  =I\left(  V_{k}\left(  x^{q}-y^{p}\right)  \right)  =\sqrt{\left(  x^{q}%
  -y^{p}\right)  }$. I\ claim that $\left(  x^{q}-y^{p}\right)  $ is a prime
  (hence radical) ideal; this will imply that $I\left(  Y\right)  =\left(
  x^{q}-y^{p}\right)  $. Consider the $k$-algebra homomorphism $k\left[
  x,y\right]  \rightarrow k\left[  z\right]  $ given by $x\mapsto z^{p}$ and
  $y\mapsto z^{q}$. Clearly $\left(  x^{q}-y^{p}\right)  $ is contained in the
  kernel, so this induces a map $\varphi:k\left[  x,y\right]  /\left(
  x^{q}-y^{p}\right)  \rightarrow k\left[  z\right]  $, which we will show is
  injective. First consider that any element $\overline{f}$ of $A:=k\left[
  x,y\right]  /\left(  x^{q}-y^{p}\right)  $ can be represented by a polynomial
  of the form $f=\sum_{i=0}^{p-1}f_{i}\left(  x\right)  y^{i}$, simply because
  we have the relation $\overline{y}^{p}=\overline{x}^{q}$. Then we have
  \[
  \varphi\left(  \overline{f}\right)  =\sum_{i=0}^{p-1}f_{i}\left(
  z^{p}\right)  z^{qi}\text{.}%
  \]
  Suppose that for positive integers $a$ and $c$ and integers $0\leq b,d<p$ we
  have $ap+bq=cp+dq$. It follows that $bq\equiv dq\left(  \operatorname{mod}%
  p\right)  $, and because $q$ is a unit modulo $p$ we have $b\equiv d\left(
  \operatorname{mod}p\right)  $. But because $\left\vert b-d\right\vert <p$ we
  must have $b=d$. It is then easy to see that $a=c$. It follows that for $0\leq
  i<j<p$ the monomials in the terms $f_{i}\left(  z^{p}\right)  z^{qi}$ are
  distinct from those in  $f_{j}\left(  z^{p}\right)  z^{qj}$. Then if
  $\overline{f}\neq0$, the representative $f_{i}$ are not all zero, and we must
  have $\varphi\left(  \overline{f}\right)  \neq0$. So $\varphi$ is injective,
  and $A$ is isomorphic to a subring of the domain $k\left[  z\right]  $. Hence
  $A$ is itself an integral domain. So $\left(  x^{q}-y^{p}\right)  $ is prime
  and we are done.
 \end{solution}

 \begin{exercise}{II.5}
   Let $k$ be a field that is not algebraically closed. Show that, for any $r$, there is
   a polynomial $g(x_1,\cdots,x_r) \in k[x_1,\cdots,x_r]$ such that $V_k(g) =
   \{(0,\cdots,0)\} \subset k^r$.
 \end{exercise}
 \begin{solution}{David Brown, brownda@math.berkeley.edu}
     Let $f(x) = (x-\alpha_{1})\ldots(x-\alpha_{n}) \in k[x]$ (with $\alpha_i \in
     \bar{k}$)     be irreducible. For $r = 1$, $g(x_1) = x_1$ works. For $r = 2$,
     $g(x_1,x_2) = (x_1-x_2\alpha_{1})\ldots(x_1-x_2\alpha_{n})$ works, because for any
     non-zero $a \in k$, $g(x,a)$ has the same splitting field as $f(x)$, and thus
     $g(x,a)$ is never zero if a is non zero (and if $a = 0$ then $x = 0$ is the only
     solution). For $r > 2$ suppose we have such a polynomial $h(x_{1},\ldots,x_{r-1})$
     for $k-1$. Then $g(x_{r}, h(x_{1},\ldots,x_{r-1}))$ works (by the same argument as
     the $r = 2$ case).
 \end{solution}

 \begin{exercise}{II.6}
   Let $k$ be a field that is not algebraically closed. Use Exercise 5 to show that any
   $k$-algebraic set $S$ in $k^{n}$ can be represented (set-theoretically) as $V_{k}(f)$
   for some $f \in k[x_{1},\ldots,x_{n}]$.
 \end{exercise}
 \begin{solution}{David Brown, brownda@math.berkeley.edu}
     Let $I(S)$ be generated by polynomials $f_{1},\ldots,f_{k}$ with each $f_i \in
     k[x_1,\cdots,x_n]$, and let $g \in k[x_{1},\ldots,x_{k}]$ be a polynomial such that
     $g(a_{1},\ldots,a_{k}) = 0$ iff each $a_{i} = 0$ (this exists from problem 5). Then
     the single polynomial $g(f_{1},\ldots,f_{r}) \in k[x_1,\cdots,x_n]$ also defines
     $S$.
 \end{solution}

  \begin{exercise}{II.7}
  (A generalization of the Weak Nullstellensatz) Let $J$  $\<$ $k[x_1,
  \cdots, x_n]$, where $k$ in any field. If $V_k(f) \neq \emptyset$
  for every $f \in J$, show that $V_k(J) \neq \emptyset$. Why is
  this a generalization of the Weak Nullstellensatz?
 \end{exercise}
 \begin{solution}{David Brown, brownda@math.berkeley.edu}
  If $k=\overline{k}$, then this is just the Weak Nullstellensatz, so
  assume $k \neq \overline{k}$.
  Let $f_1, \cdots, f_r$ generate $J$, and let $g(y_1, \cdots, y_r)$
  be the polynomial from exercise II.5.  Then $h=g(f_1, \cdots, f_r)
  \in J$, so by our hypothesis there is an $a=(a_1, \cdots, a_n) \in
  k^n$ such that $h(a)=0$.  By exercise II.5,
  $f_1(a)=\cdots=f_n(a)=0$.  This is a generalization of the Weak
  Nullstellensatz because it works when $k \neq \overline{k}$.
 \end{solution}

 \begin{exercise}{II.8}
   Let $K/k$ be a field extension, and let $Y\subseteq K^n$ be a $k$-algebraic set. For
   any $a\in Y$, let $\lambda_a:k[Y]\to K$ be ``evaluation at $a$''; that is,
   $\lambda_a(f)=f(a)$ for any $k$-polynomial function $f$ on $Y$. Show that $\lambda_a$
   is a $k$-algebra homomorphism, and that $a\mapsto \lambda_a$ defines a bijection from
   $Y$ to $\hom_{k\text{-alg}}(k[Y],K)$ (the set of $k$-algebra homomorphisms from $k[Y]$ to
   $K$).
 \end{exercise}
 \begin{solution}{anton@math}
   For the constant polynomial $c\in k[Y]$, we have $\lambda_a(c)=c(a)=c$, so $\lambda_a$
   maps $k$ to $k$. Moreover, we have $\lambda_a(f\cdot g)=f(a)\cdot
   g(a)=\lambda_a(f)\cdot \lambda_a(g)$ and $\lambda_a(f+g) =f(a)+g(a) =\lambda_a(f)
   +\lambda_a(g)$, so $\lambda_a$ is a $k$-algebra homomorphism.

   If $a$ and $a'$ are two points in $Y\subseteq K^n$, then they differ in some
   coordinate, say the $i$-th. Let $\bar x_i$ be the image of the polynomial $x_i$ in
   $k[Y]$. Then $\lambda_a(\bar x_i)\neq \lambda_{a'}(\bar x_i)$, so $\lambda_a\neq
   \lambda_{a'}$. We've shown that the map $\lambda_{-}:Y\to \hom_{k\text{-alg}}(k[Y],K)$
   is injective.

   If $f:k[Y]\to K$ is a $k$-algebra homomorphism, then I claim it is $\lambda_a$ for
   $a=\bigl(f(\bar x_1),\dots, f(\bar x_n)\bigr)$. To see this, note that $\lambda_a(\bar
   x_i)=f(\bar x_i)$ for all $i$. Since the $\bar x_i$ generate $k[Y]$ as a $k$-algebra,
   this implies that $\lambda_a=f$, proving that $\lambda_-$ is surjective. By the way,
   note that $a\in Y$ because $g\in I(Y)\Rightarrow \bar g=0\in k[Y]\Rightarrow f(g)=0$,
   so $g(a) = g\bigl(f(\bar x_1),\dots, f(\bar x_n)\bigr) = f(g)=0$. The last equality
   follows from the fact that $f$ is a $k$-algebra homomorphism, so $f(\bar x_i)\cdot
   f(\bar x_j) = f(\bar x_i \bar x_j)$, and so on.
 \end{solution}

 \begin{exercise}{II.9}
   True or False: for any field extension $K/k$, the only closed points in $K^n$ in the
   Zariski $k$-topology are the $k$-points
 \end{exercise}
 \begin{solution}{anton@math}
   False. Consider the extension $K=\QQ(\sqrt[3] 2)$ of $k=\QQ$. Then
   $V_K(x^3-2)=\{\sqrt[3] 2\}$ is a closed point which is not a $k$-point.
 \end{solution}
 \begin{solution}{lam@math}
   {\it Comment.}  It doesn't pay to add the assumption that $K/k$ is normal
   either.  If ${\rm char}(k)=p>0$ and $\,k\,$ is not perfect, we can take an element
   $\alpha\notin k$ with $\alpha^p=a\in k$.  For $K=k(\alpha)$, we have
   $V_K(x^p-a)=\{\alpha\}\subseteq K^1$, so $\,\{\alpha\}\,$ is a closed point in $K^1$
   (w.r.t.~the $k$-topology) that is not a $k$-point. How about assuming $\,K/k\,$ is
   (finite) Galois?
 \end{solution}

 \begin{exercise}{II.9'}
  (This exercise amplifies the former Ex. 9.) Let $K/k$ be fields such that $K \subset
  \overline{k}$. Show that (1) in the $k$-topology, any closed poin in $K^n$ is
  algebraic, and (2) if $K/k$ is a Galois extension, the only closed points in $K^n$
  are the $k$-points.
 \end{exercise}
 \begin{solution}{David Brown, brownda@math}
  (1) Let $X = \{(a_1,\cdots,a_n)\}  \subset K^n$ be algebraic.
  Let $X_{\text{alg}} = X \cap \overline{k}^n$. By exercise II.21,
  $\overline{X_{\text{alg}}} = X$. This is only possible if
  $V_{\text{alg}}$ is non-empty (else its closure would be empty),
  so $(a_1,\cdots,a_n)$ is in fact algebraic.\\

  (2) Suppose $P = (a_1,\cdots,a_n) \in K^n$, and assume $X =
  \{P\}$ is closed, i.e. there exists $J \< A = k[x_1,\cdots,x_n]$
  with $X := V_K(J)$. Let $J$ be generated by $f_1,\cdots,f_r$.
  Then $f_i(P) = 0$. Let $\sigma \in \text{Gal}(K/k)$. Then
  $P^{\sigma} = (a_1^{\sigma},\cdots,a_n^{\sigma})$, and
  $0 = 0^{\sigma} = (f_i(P))^{\sigma} = f_i(P^{\sigma})$ (since
  the coefficients of $f_i$ are in $k$, hence fixed by $\sigma$).
  Thus, $P^{\sigma} \in V_K(J) = \{P\}$, and we conclude
  that $P^{\sigma} = P$ for every $\sigma$, (thus $P \in k^n$).
 \end{solution}

 \begin{exercise}{II.10}
     For a nonempty space $X$, the following are equivalent:
     \begin{enumerate}
         \item $X$ is not the union of two proper closed sets.
         \item Any two nonempty open sets in $X$ intersect.
         \item Any nonempty open set in $X$ is dense.
     \end{enumerate}
 \end{exercise}
 \begin{solution}{Lars Kindler, lars\_k@berkeley.edu}
     \begin{itemize}
         \item[1. $\Rightarrow$ 2.] Let $U,V\subset X$ be open, nonempty and
             assume $U\cap V=\varnothing$. Passing to the complement shows
             $X\setminus U\cup X\setminus V = X$ which contradicts the
             irreducibility of $X$, so $U\cap V\neq \varnothing$.
         \item[2. $\Rightarrow$ 3.] Let $U\subset X$ be open and nonempty. Let
             $x\in X$ be an arbitrary point, then for every open neighborhood
             $U_x$ of $x$ we have $U_x\cap U\neq \varnothing$ by assumption, so $x\in
             \bar{U}$, which proves $\bar{U}=X$.
         \item[3. $\Rightarrow$ 2.] For open nonempty $U,V\subset X$ we have
             $\bar{U}=X=\bar{V}$, i.e. $\bar{U}\supset V$, so $V\cap U\neq
             \varnothing$.
         \item[2. $\Rightarrow$ 1.] Assume that $X=A\cup B$ for $A,B\subsetneqq X$
             closed, then we have $X\setminus A \cap X\setminus
             B=\varnothing$, which contradicts the assumption 2.
     \end{itemize}
 \end{solution}

 \begin{exercise}{II.11}
  (1) The irreducible closed subsets of $\spec\left( R\right) $ are
  precisely sets of the form $\mathcal{V}\left( \mathfrak{p}\right) $, where $%
  \mathfrak{p\in }\spec\left( R\right) $. (2) The irreducible
  components of $\spec\left( R\right) $ are precisely sets of the
  form $\mathcal{V}\left( \mathfrak{p}\right) $, where $\mathfrak{p\in }%
  \Min\left( R\right) $.
 \end{exercise}
 \begin{solution}{Manuel Reyes; mreyes@math}
  (1) It is straightforward to verify that a subset of a topological space is
  irreducible iff, whenever it is contained in the union of two closed
  subsets, it belongs to one of the two subsets. Let $\mathcal{V}\left(
  I\right) \neq \varnothing $ be an arbitrary closed subset of $\spec
  \left( R\right) $. We may assume that $I\neq R$ is radical. Then $\mathcal{V}%
  \left( I\right) $ is irreducible iff for any ideals $J_{1},J_{2}\subseteq R$
  such that $\mathcal{V}\left( I\right) \subseteq \mathcal{V}\left(
  J_{1}\right) \cup \mathcal{V}\left( J_{2}\right) =\mathcal{V}\left(
  J_{1}J_{2}\right) $ we have that $\mathcal{V}\left( I\right) \subseteq
  \mathcal{V}\left( J_{i}\right) $ for some $i$. Recalling that $I$ is
  radical, this holds iff $\sqrt{J_{1}J_{2}}\subseteq I$ implies that one of
  the $\sqrt{J_{i}}\subseteq I$, iff $J_{1}J_{2}\subseteq I$ implies that one
  of the $J_{i}\subseteq I$. This is true iff $I$ is prime (see the solution
  to Exercise II.15).

  (2) Consider that for two irreducible closed sets $\mathcal{V}\left(
  \mathfrak{p}\right) $, $\mathcal{V}\left( \mathfrak{q}\right) $, with $%
  \mathfrak{p},\mathfrak{q\in }\spec\left( R\right) $, we have $%
  \mathcal{V}\left( \mathfrak{p}\right) \subseteq \mathcal{V}\left( \mathfrak{q%
  }\right) $ iff $\mathfrak{q}\subseteq \mathfrak{p}$. \ It follows that
  \begin{eqnarray*}
  \left\{ F\subseteq \spec\left( R\right) :F\text{ is closed and
  irreducible}\right\} ^{\ast } &=&\left\{ \mathcal{V}\left( \mathfrak{p}%
  \right) :\mathfrak{p}\in \spec\left( R\right) \right\} ^{\ast } \\
  &=&\left\{ \mathcal{V}\left( \mathfrak{p}\right) :\mathfrak{p}\in \spec
  \left( R\right) _{\ast }\right\}  \\
  &=&\left\{ \mathcal{V}\left( \mathfrak{p}\right) :\mathfrak{p}\in \Min
  \left( R\right) \right\} \text{.}
  \end{eqnarray*}%
  Hence the irreducible components of $\spec\left( R\right) $ are the
  subsets of the form $\mathcal{V}\left( \mathfrak{p}\right) $ with $\mathfrak{%
  p}\in \Min\left( R\right) $.
 \end{solution}

 \begin{exercise}{II.12}
   (1) For any $I\<R$, let $\pi: R \rightarrow R/I$ be the
   projection map. Show that the map $\pi^*: \spec(R/I) \rightarrow \spec(R)$
   induces a homeomorphism from $\spec(R/I)$ onto $\V(I)$. (In particular,
   if $I \subseteq \nil(R)$, $\pi^*$ is a homeomorphism from $\spec(R/I)$
   onto $\spec(R)$. \newline
   (2) For any multiplicative set $S \subseteq R$, show that the localization
   map $f: R \rightarrow R_S$ induces a homeomorphism $f^*$ from $\spec(R_S)$
   onto the subspace $\{\p \in \spec(R): \p \cap S = \emptyset \}$ of
   $\spec(R)$.
 \end{exercise}
 \begin{solution}{los@math}
   (1) The map $K \mapsto \pi^{-1}(K)$ is a bijection from the set of ideals
   of $R/I$ to the set of ideals of $R$ containing $I$, and its inverse is
   given by $J \mapsto J/I$. For any $J \< R$ containing $I$, the rings
   $(R/I)/(J/I)$ and $R/J$ are isomorphic, hence $J$ is prime in $R$
   if and only if $J/I$ is prime in $R/I$. This establishes that $\pi^*$
   is a bijection. The map $\pi^*$ is also continuous. Therefore, to show it is a
   homeomorphism, it is enough to show it is closed. Any closed subset of
   $\spec(R/I)$ is of the form $\V(J/I)$ for some $J \< R$ containing $I$.
   We have $\pi^*(\V(J/I))=\V(J)$. The latter set is closed in $\spec(R)$,
   thus $\pi^*$ is a closed map. This proves that $\pi^*$ is a
   homeomorphism from $\spec(R/I)$ onto $\V(I)$. The last statement results
   from the fact that $\nil(R)$ is contained in every prime ideal of $R$.
   \newline
   (2) Let $W_S$ denote the subspace $\{\p \in \spec(R): \p \cap S =
   \emptyset \}$ of $\spec(R)$. If $I$ (resp. $J$) is an ideal in $R$
   (resp. $R_S$), write $I^e$ (resp. $J^c$) for the ideal $IR_S=I_S$
   of $R_S$ (resp. the ideal $f^{-1}(J)$ of $R$). (In particular,
   for $\q \in \spec(R_S)$, we have $f^*(\q) = \q^c$ by definition.)
   Because $f$ is a localization map, we have $J = J^{ce}$ for every
   ideal $J$ of $R_S$.
   First, we show that $f^*(\spec(R_S)) \subseteq W_S$. Let $\q \in
   \spec(R_S)$, and assume there exists $s \in \q^c \cap S$.  Then
   the invertible element $f(s)=s/1$ of $R_S$ belongs to $\q^{ce} = \q$.
   This is absurd, since $\q$, being prime, cannot be all of $R_S$.
   Therefore $\q^c \cap S = \emptyset$, which shows that $\q^c \in W_S$.
   Now we show that $f^*$ is a bijection from $\spec(R_S)$ onto
   $W_S$, with inverse given by $\p \mapsto \p^e$. For this it is enough
   to prove that $\p^e=\p_S$ is prime in $R_S$ and $\p^{ec}=\p$ whenever $\p \in W_S$.
   Let $\pi : R \rightarrow R/\p$ denote the projection. Then we have natural maps
   $R/\p \rightarrow R_S/\p^e = R_S/\p_S \stackrel{\sim}{\longrightarrow} (R/\p)_S
   \stackrel{\sim}{\longrightarrow} (R/\p)_{\pi(S)}$ the composite of which is the
   localization homomorphism $g$ relative to the multiplicative subset $\pi(S)$ of
   $R/\p$. By hypothesis, $\p \cap S = \emptyset$, hence $\pi(S) \subseteq
   (R/\p) \setminus \{0\}$. Furthermore, $R/\p$ is a domain. Thus $(R/\p)_{\pi(S)}$
   is a domain and the localization homomorpism $g$ is injective. In particular,
   $\p^e$ is prime and $\p^{ec}/\p = \ker(g) = 0$. This shows that $f^*$ is a bijection
   onto $W_S$. Finally, we show $f^*$ is a homeomorphism onto its image in $\spec(R)$.
   Since $f^*$ is continuous, it is enough to show that every closed subset $Z$ of
   $\spec(R_S)$ is $(f^*)^{-1}(Z')$ for some closed subset $Z'$ of $\spec(R)$.
   By definition of the topology on $\spec(R_S)$, we have $Z=\V(J)$ for some ideal
   $J$ of $R_S$. Now for $Z' = \V(J^c) \subseteq \spec(R)$ we have $(f^*)^{-1}(Z')
   = \V(J^{ce}) = \V(J) = Z$, which completes the proof.
 \end{solution}

 \begin{exercise}{II.13}
     For any ring $R$, show that the following are equivalent:
     \begin{enumerate}
         \item $\spec R$ is a discrete space
         \item $\spec R$ is finite and discrete
         \item $R$ is $0$-dimensional and semilocal
     \end{enumerate}
     Show that these conditions are not equivalent to $|\spec R|<\infty$. If $R$ is
     a noetherian ring, show that 1.-3. are equivalent to: 4. $R$ is artinian.
 \end{exercise}
 \begin{solution}{Lars Kindler, lars\_k@berkeley.edu}
     \begin{itemize}
         \item[1. $\Leftrightarrow$ 2.] $\spec R$ is discrete iff all points are
             open, iff $\spec R$ is finite and discrete, since $\spec R$ is
             compact.
         \item[2. $\Rightarrow$ 3.] It is clear that $R$ is semilocal. Assume
             there are $\p_1,\p_2\in \spec R$ such that $\p_1\subsetneqq\p_2$.
             Then every open neighborhood of $\p_2$ contains $\p_1$, since if
             $D(f)$ is a basic open set with $\p_2\in D(f)$, then $\p_1\in
             D(f)$. This contradicts the assumption that $\spec R$ is
             discrete.
         \item[3. $\Rightarrow$ 1.] Let $\spec R=\{\p_1,\ldots, \p_n\}$ and let
             $R$ be $0$-dimensional. Then for every $j\neq i$
             there is a $x_j\in \p_j\setminus \p_i$, so
             $x:=\prod_{j\neq i}x_j\in \p_k$ for
             all $k\neq i$, which means $D(x)=\{\p_i\}$, i.e. $\spec R$ is
             discrete.
     \end{itemize}
     If $R$ is noetherian, then $R$ is $0$-dimensional iff $R$ is artinian, i.e. 3.
     $\Leftrightarrow$ 4. To see that the above are not equivalent to $|\spec
     R|<\infty$, consider $R=\mathbb{F}_2[x,y]/(x^2,y^2)$. $R$ has only finitely many
     primes since $R$ is a finite ring, but $(x)\subsetneqq (x,y)$ is a chain of prime
     ideals, so $R$ is not $0$-dimensional, which proves $|\spec R|<\infty
     \not\Leftrightarrow$ 3.
 \end{solution}

 \begin{exercise}{II.14}
     For $\p_1\neq \p_2$ in $\spec R$, show that there is no prime $\p\subset \p_1\cap
     \p_2$ iff there exist disjoint open sets $X_1, X_2\subseteq \spec R$ such that
     $\p_i\in X_i$ for $i=1,2$.
 \end{exercise}
 \begin{solution}{Lars Kindler, lars\_k@berkeley.edu}
     First, let $X_1,X_2\subset \spec R$ be disjoint open sets with $\p_i\in X_i$.
     Then there are $f_1,f_2\in R$ such that $\p_i\in D(f_i)\subset X_i$, so if there
     is a prime  $\p\subset \p_1\cap \p_2$  this implies $f_i\not\in\p$, $i=1,2$, so $\p\in
     D(f_1)\cap D(f_2)=\varnothing$, which is absurd.\\
     Conversely, assume that there is no prime $\p\subset \p_1\cap \p_2$. We want to
     show that there are $f_1\in R\setminus\p_1, f_2\in R\setminus\p_2$ with
     $f_1f_2=0$, since that would give us $D(f_1)\cap D(f_2)=D(f_1f_2)=\varnothing$
     and $\p_i\in D(f_i)$, $i=1,2$. Assume that there are no such elements $f_1,f_2$.
     Let $S$ and $T$ denote the multiplicative sets $R\setminus \p_1$ and $R\setminus
     \p_2$, then  $ST$ is also a multplicative set because $0\not\in ST$ by
     assumption. Moreover note that $S\subset ST$ and $T\subset ST$, since $1\in S\cap
     T$. Now there is a prime ideal $\p$, maximal with respect to being disjoint from
     $ST$ (Kaplansky's Theorem 1), but this means $\p\subset R\setminus ST\subset
     R\setminus S\cap R\setminus T = \p_1\cap \p_2$ which is a contradiction.
 \end{solution}

 \begin{exercise}{II.15}
   Show that $\spec\left( R\right) $ is an irreducible space iff $R/%
   \nil\left( R\right) $ is an integral domain. (Conceptually, this is the same
   as Exercise 21(2) in Ch. I.)
 \end{exercise}
 \begin{solution}{mreyes@math}
   As a preliminary we prove the fact that an ideal $\mathfrak{p\subsetneq }R$ is prime iff,
   for ideals $I,J\subseteq R$, $IJ\subseteq \mathfrak{p}$
   implies that one of $I$ or $J$ is contained in $\mathfrak{p}$. Indeed, if $%
   \mathfrak{p}$ is an ideal with this property, then $ab\in \mathfrak{p}$ implies that
   $\left( a\right) \left( b\right) =\left( ab\right) \subseteq \mathfrak{p}$. Then, without
   loss of generality, we have $a\in \left( a\right) \subseteq \mathfrak{p}$ and
   $\mathfrak{p}$ is prime. Now suppose that there exist ideals $I,J\nsubseteq \mathfrak{p}$
   with $IJ\subseteq
   \mathfrak{p}$. Then there exist $a\in I\smallsetminus \mathfrak{p}$ and $%
   b\in J\smallsetminus \mathfrak{p}$, with $ab\in IJ\subseteq \mathfrak{p}$. Hence
   $\mathfrak{p}$ is not prime.

   This means that a prime lies above $IJ$ iff it lies above either $I$ or $J$,
   or $\mathcal{V}\left( IJ\right) =\mathcal{V}\left( I\right) \cup \mathcal{V}%
   \left( J\right) $. As another side note, consider that $\mathcal{V}\left(
   I\right) =\spec\left( R\right) $ iff $I\subseteq \nil%
   \left( R\right) $.

   We are now ready to solve the problem. The space $\spec\left(
   R\right) $ is irreducible iff, for any ideals $I,J\subseteq R$ with $%
   \spec\left( R\right) =\mathcal{V}\left( I\right) \cup \mathcal{V}%
   \left( J\right) =\mathcal{V}\left( IJ\right) $, we have that $\spec%
   \left( R\right) $ is equal to one of $\mathcal{V}\left( I\right)
   ,\mathcal{V}%
   \left( J\right) $. But this is true iff $IJ\subseteq \nil\left(
   R\right) $ implies that one of $I$ or $J$ is contained in $\nil%
   \left( R\right) $. This happens iff $\nil\left( R\right) $ is prime, iff
   $R/\nil\left( R\right) $ is an integral domain
 \end{solution}

 \begin{exercise}{II.16}
  Let $S \subseteq T$ where $S$ is Hilbert and $T$ is ring-finite over $S$.  Show that any
  maximal ideal in $T$ contracts to a maximal ideal in $S$.
 \end{exercise}
 \begin{solution}{Jonah (jblasiak@math)}
  Let $m$ be maximal in $T$ and let $\phi$ be the inclusion of $S$ into $T$.  The
  composition $S \hookrightarrow T \rightarrow T/m$ factors through $A := S/
  \phi^{-1}(m)$, yielding the injection $A \stackrel{\alpha}{\hookrightarrow} T/m$.  $A$
  is a domain and it follows easily from the definition that $A$ is Hilbert.   Now we'll
  assume that $A$ is not a field and arrive at a contradiction.

  Let $i$ be the inclusion of $A$ into its field of fractions $Q(A)$. There exists a
  unique map $\beta$ from $Q(A)$ to $T/m$ such that $\beta \circ i = \alpha$, since all
  elements of $A$ are sent by $\alpha$ to something invertible.  The map $\phi$ being
  ring-finite easily implies the same for $\alpha$ and $\beta$.  $Q(A)$ is a field so
  $\beta$ is injective and lemma 4.4 implies that $\beta$ is a finite field extension.
  We will show below that $i$ is ring-finite.  Assuming this, let $\frac{a_1}{b_1},
  \frac{a_2}{b_2}, \ldots, \frac{a_k}{b_k}$ be generators for $Q(A)$ as an $A$-algebra
  (where the $a_i$ and $b_i$ are in $A$).  Then $\frac{1}{b_1 b_2 \ldots b_k}$ generates
  $Q(A)$ as an $A$-algebra.  Since $A$ is a Hilbert domain, we can choose a maximal ideal
  $m' \subseteq A$ such that $b_1 b_2 \ldots b_k \notin m'$.  Then localizing at $m'$
  yields $Q(A)$ since $b_1 b_2 \ldots b_k$ is inverted, but contains the maximal ideal
  $m'$ as well as the non-maximal ideal $(0)$, contradiction.

  To see that $i$ is ring-finite, let $t_1, \ldots, t_r$ be generators for $T/m$ as an
  $A$-algebra.  Let $v_1, \ldots, v_s$ be a vector space basis for $T/m$ over $Q(A)$,
  where $v_1 = 1$.  Now we can write $t_i = \sum_j c_{ij} v_j$, with $c_{ij} \in Q(A)$,
  and $v_i v_j = \sum_{k} d_{ijk} v_k$, with $d_{ijk} \in Q(A)$.  Any $u \in Q(A)$ can be
  written as $\sum_{\mathbf{\nu}} a_{\mathbf{\nu}} \mathbf{t}^\mathbf{\nu}$ with the
  $a_\mathbf{\nu} \in A$, $\mathbf{\nu} \in \mathbb{N}^s$.  Since each
  $\mathbf{t}^\mathbf{\nu}$ can be multiplied out and expressed as a linear combination
  of the $v_i$ using the above relations, we have $u = \sum_i e_i v_i$, where $e_i$ is a
  polynomial in the $c_{ij}$ and $d_{ijk}$ with coefficients in $A$ .  Then we have $u =
  \sum_i e_i v_i = e_1 v_1 = e_1$ because $u \in Q(A)$ and the $v_i$ are a basis.
  Therefore the $c_{ij}$ and $d_{ijk}$ generate $Q(A)$ as an $A$-algebra.
 \end{solution}

 \begin{exercise}{II.17}
     Let $f:R\rightarrow S$ be a ring homomorphism. Then $\p\in \spec R$ is in the
     image of $f^*:\spec S\rightarrow \spec R$ iff $\p=\p^{ec}$.
 \end{exercise}
 \begin{solution}{Lars Kindler, lars\_k@berkeley.edu}
     First let $\p\in\operatorname{Im}(f^*)$, which means there is a $\mathfrak{q}\in \spec
     S$, such that $\mathfrak{q}^{c}=f^{-1}(\mathfrak{q})=\p$. Then we have
     $\p^e=(\mathfrak{q}^c)^e\subset \mathfrak{q}$ and thus $\p\subset
     \p^{ec}\subset \mathfrak{q}^c=\p$.\\
     Conversely, let $\p = \p^{ec}=f^{-1}(f(\p)S)$. Then we have
     $\p\subseteq f^{-1}(f(\p))\subseteq f^{-1}(f(\p)S)=\p^{ec}=\p$, and thus
     $f(\p)=f(\p)S=\p^e$, and since $\p$ is prime this implies $\p^e$ is prime, hence
     $\p\in\operatorname{Im}(f^*)$.
 \end{solution}

 \begin{exercise}{II.18}
  Show that a topological space $X$ is noetherian iff all subspaces
  (respectively, all open subspaces) of $X$ are compact.
 \end{exercise}
 \begin{solution}{Manuel Reyes; mreyes@math}
  If all subspaces of $X$ are compact, then all open subspaces of $X$ are
  certainly compact.

  Suppose that all open subspaces of $X$ are compact, and let $U_{1}\subseteq
  U_{2}\subseteq\cdots$ be an ascending chain of open subsets of $X$. Then $U=%
  %TCIMACRO{\tbigcup }%
  %BeginExpansion
  {\textstyle\bigcup}
  %EndExpansion
  U_{i}$ is an open subset and thus is compact. This means that the open cover
  $\left\{  U_{i}\right\}  $ has a finite subcover. Then choosing a maximal
  element $U_{n}$ of that subcover, it is clear that $U=U_{n}=U_{n+1}=\cdots$.
  So $X$ satisfies the ACC on open subsets and thus is noetherian.

  Finally suppose that $X$ is noetherian, and let $Y\subseteq X$ be any
  subspace. To show that $Y$ is compact, let $\left\{  U_{i}\right\}  $ be any
  nonempty open cover of $Y$. Denote by $\mathcal{F}$ the family of finite
  unions of elements of the cover $\left\{  U_{i}\right\}  $. Because
  $\mathcal{F\neq\varnothing}$ it has a maximal element, say $V:=U_{i_{1}}%
  \cup\cdots\cup U_{i_{n}}$. Assume for contradiction that $Y\subsetneq V$. Then
  choose some point $x\in Y\smallsetminus V$, and choose some $U_{j}$ containing
  $x$. Then $V\subsetneq V\cup U_{j}\in\mathcal{F}$, contradicting the
  maximality of $V$. So $U_{i_{1}},\ldots,U_{i_{n}}$ is a finite subcover of $Y$
  and thus $Y$ is compact.
 \end{solution}

 \begin{exercise}{II.19}
   Let $M$ be a f.g. $R$-module. What are the irreducible components of $\supp(M)$ (as a
   subspace of $\spec(R)$)? Show that the number of irreducible components is finite if
   $M$ is a noetherian module. ({\bf Hint.} Let $I = \ann(M)$, and use Exercise 12 in
   conjunction with (I.3.12) and (II.3.12)!)
 \end{exercise}
 \begin{solution}{los@math} Let $I= \ann(M)$. By (I.3.12), we have
   $\supp(M) = \V(I)$. The set $\V(I)$ is homeomorphic to $\spec(R/I)$ by Exercise 12,
   and for any ideal $J$ of $R$ containing $I$, the set $\V(J) \subseteq  \V(I)$
   corresponds to $\V(J/I) \subseteq \spec(R/I)$ under this identification.  By
   (II.3.12(2)) then, the irreducible components of $\V(I)$ are the the sets $V(\p)$ for
   each prime $\p$ of $R$ minimal over $I$. If $R/I$ is noetherian, then by (I.4.13)
   applied to the ideal $(0)$ of $R/I$, there will be only finitely many primes of $R$
   minimal over $I$, and therefore only finitely many irreducible components in
   $\supp(M)$. Thus in order to prove the last statement, it will be enough to show that
   if $M$ is a faithful noetherian module over a ring $A$ (which in our case will be
   $R/I$), then the ring $A$ is itself noetherian. Assume such an $A$ and $M$ given. Then
   $M$ is generated by finitely many elements, say $m_1,m_2,\ldots,m_n$. We define a
   linear map $\varphi : A \rightarrow M^n$ by $\varphi(a)=(am_1,am_2,\ldots,am_n)$.
   Because the elements $m_1,m_2,\ldots,m_n$ generate $M$ we have $\ker(\varphi)
   \subseteq \ann(M)$. But $\ann(M) = 0$ because $M$ is faithful. Therefore $A$ is
   isomorphic to an $A$-submodule of the noetherian $A$-module $M^n$, and hence is a
   noetherian $A$-module itself. By definition, $A$ is a noetherian ring. This completes
   the proof.
 \end{solution}

 \begin{exercise}{II.20}
   For any field $k$, show that any non-maximal prime ideal in $A=k[x,y]$ is principal.
   [If necessary, you may use the fact (to be proved in Chap. III) that prime chains in
   $A$ have length $\leq 2$.] Assuming $k$ is algebraically closed, give a 1-minute
   running commentary on the following picture of $\spec(A)$ in Mumford's ``Little Red
   Book'': (picture omitted)
 \end{exercise}
 \begin{solution}{los@math}
   We will show that if $R$ is any principal ideal domain, then any non-maximal prime
   ideal of $R[t]$ is principal. The result of the exercise will follow, taking $R=k[x]$.

   Let $\p$ be any non-maximal prime ideal of $R[t]$. $\p \cap R$ is a prime ideal $\p'$
   of $R$. Assume first $\p'$ is not zero. Then it is $(p)$ for some irreducible $p \in
   R$. Since $R$ is a principal ideal domain, $R/(p)$ is a field. Therefore, any nonzero
   prime ideal of $R[t]/pR[t] \cong (R/(p))[t]$ is maximal. $\p$ is a prime ideal of
   $R[t]$ which contains $pR[t]$ and is non-maximal, hence is equal to $pR[t]$. Assume
   therefore that $\p \cap R = \{0\}$. We may assume $\p \neq 0$. If $Q$ is any nonzero
   element of $\p$, then we may write $Q$ as a product of irreducible factors, one of
   which, say $P$, must belong to $\p$. We will show that $\p = (P)$, completing the
   proof. Otherwise, we have an ascending chain of three prime ideals $\{0\} \subset (P)
   \subset \p$, the ideal $(P)$ being prime because $P$ is irreducible and $R[t]$ is a
   unique factorization domain. The intersection of each of these prime ideals with the
   multiplicative subset $S = R \setminus \{0\}$ of $R[t]$ is empty. Let $K$ denote the
   fraction field of $R$. By the proof of Exercise 12, the localizations in $S^{-1}R[t]
   \cong K[t]$ of these ideals form a chain of prime ideals of length 2 in $K[t]$. But
   this is absurd, since $K[t]$ is a principal ideal domain.

   Thus the prime ideals of $A = k[x,y]$ are of three kinds: $\{0\}$, $(f)$ for some
   irreducible $f \in A$, and maximal ideals. The last two kinds are mutually exclusive
   since, by the Nullstellensatz, if $\m$ were a maximal ideal of $A$, then $A/\m$ would
   be an algebraic field extension of $k$. However, it is clear that for  an irreducible
   nonzero $f \in A$, the transcendence degree of $A/(f)$ over $k$ is 1. Now assume that
   $k$ is algebraically closed. The closed points in $\spec(A)$ correspond bijectively to
   the points of $k^2$. The ideal $\{0\}$ is the generic point of the plane $\spec(A)$.
   The remaining prime ideals of $A$ are of the form $(f)$ for some irreducible $f$, and
   are the generic points of the curves $f(x,y) = 0$. Here $f$ is determined up to a
   nonzero constant in $k$.
 \end{solution}

 \begin{exercise}{II.21}
  Let $K$ be a field containing $\overline{k}$. For any $k$-algebraic
  set $Y \subset K^n$, show that, in the $k$-topology, $Y_{\text{alg}}$
  is dense in $Y$.
 \end{exercise}
 \begin{solution}{David Brown, brownda@math}
    Note that $Y_{\text{alg}} = V_{\overline{k}}(J)$. By the Nullstellensatz,
    $I(V_{\overline{k}}(J)) = \sqrt{J}$. Thus,
    $\overline{Y_{\text{alg}}} = V_K(I(Y_{\text{alg}})) =
    V_K(\sqrt{J}) = V_K(J) = Y$.
 \end{solution}
 \begin{solution}{lam@math}
   {\it Comment.}  Those who dig characteristic $p$ would want to further amend this
   exercise into the following: ``$Y_{sep}$ is dense in $Y$'' (where $Y_{sep}$ denotes
   the set of $(y_1,\dots,y_n)\in Y$ with all $y_i$ separably algebraic over $k$). The
   proof takes another couple of lines.
 \end{solution}

 \begin{exercise}{II.22}
  It was pointed out (after (4.12)) that 0-dimensional rings are Hilbert. Coversely, show
  that \textit{semilocal} Hilbert rings are 0-dimensional.
 \end{exercise}
 \begin{solution}{mreyes@math}
  Let $R$ be a semilocal Hilbert ring. Then any prime $\mathfrak{p}$ of $R$ is an
  intersection of finitely many maximal ideals and thus contains the product of these
  finitely many maximal ideals. Because $\mathfrak{p}$ is prime it must then contain one of
  these maximal ideals (see the solution to problem II.15), so that $\mathfrak{p}$ itself
  is maximal. So every prime
  of $R$ is maximal, and $R$ is 0-dimensional.
 \end{solution}

 \begin{exercise}{II.23}
  Show that a 1-dimensional noetherian domain is Hilbert iff it is not semilocal.
 \end{exercise}
 \begin{solution}{Jonah (jblasiak@math)}
  Corollary 5.7 states that a ring $R$ is semilocal iff $R/\rad(R)$ is artinian.  A
  1-dimensional noetherian domain $R$ is Hilbert iff $(0)$ is an intersection of maximal
  ideals iff $\rad(R) = 0$ iff $R/\rad(R)$ is not artinian.  This last iff is because
  artinian is equivalent to 0-dimensional and neotherian, and $R$ is 1-dimensional.  The
  exercise then follows from the corollary.
 \end{solution}

 \begin{exercise}{II.24}
   Discuss the Ping Pong Paddle Picture in more detail in the case where $k=\QQ$,
   $K=\RR$, and $Y$ is the real parabola $V_K(y-x^2)$.
 \end{exercise}

 \end{document}
