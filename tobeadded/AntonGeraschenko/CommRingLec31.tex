 Exercise III.6: replace ``$S_\mathfrak{P}/R_\p$ with $S_\mathfrak{P}/\im (R_\p)$.\\
 III.21: $f(x) = \sum_{n=0}^\infty \frac{x^n}{2^{n^2}} \in \QQ((x))$. This exercise is a
 little ``provisional''.

 Notation: $D$-$Val(K)$ is the set of \emph{discrete} valuation rings of $K$ (this set
 may be empty).
 \begin{example}
   If $K$ is an algebraic extension of $\FF_p$, then $Val(K)=\{K\}$,
   $D$-$Val(K)=\varnothing$.
 \end{example}
 \begin{example}
   Take $K$ so that $K^\times = (K^\times)^n$ for some fixed $n\ge 2$. Then
   $D$-$Val(K)=\varnothing$. If $(R,(\pi))$ is a DVR of $K$, then $\pi=a^n$ for some $n$,
   so $a$ is integral over $R$, so it is in $R$ (because $R$ is normal). But then
   $(a)=(\pi)^m = (a)^{n+m}$. contradiction.
 \end{example}

 For any subring $R\subseteq K$, we defined $Val_R(K)$ to be the elements of $Val(K)$
 which contain $R$. If $R\in Val(K)$, then $Val_R(K)$ is just the set of rings between
 $R$ and $K$.

 \begin{theorem}[4.12]
   Describing all $R'$ so that $R\subseteq R'\subseteq K$, where $(R,\m)\in Val(K)$. A
   typical $R'$ is of the form $R_\p$, where $\p\subseteq \m$ is a prime in
   $R$. Furthermore, $\p R_\p \overset{!}{=} \p$ is the maximal ideal.
 \end{theorem}
 \begin{proof}
   omitted.\anton{}
 \end{proof}
 Consequently, the map $\p\mapsto R_\p$ defines an inclusion reversing bijection $\spec R
 \leftrightarrow Val_R(K)$. In particular, $Val_R(K)$ is a chain because $\spec R$ is a
 chain. The longest chain is the Krull dimension of $R$. In particular, $\dim R=1$ if and
 only if $R$ is a maximal subring of $K$. DVRs are 1-dimensional, but not all
 1-dimensional valuation rings are DVRs.

 \begin{definition}
   $Val^R(K) = \{R' \in Val(K)| R'\subseteq R\subseteq K\}$. This is only meaningful if
   $R\in Val(K)$ since any ring containing a valuation ring is a valuation ring (so we
   would have $Val^R(K)=\varnothing$ if $R\not\in Val(K)$).
 \end{definition}
 How do you tell the difference between $Val_R(K)$ and $Val^R(K)$? Well, $Val_R(K)$ has
 the $R$ below, and $Val^R(K)$ has the $R$ above.

 \begin{theorem}[4.13]
   For $(R,\m)\in Val(K)$, there is an inclusion preserving bijection
   $Val^R(K)\leftrightarrow Val(R/\m)$, with $R'\mapsto R'/\m =: \bbar {R'}$. Note that
   $\m\subseteq \m'\subseteq R'\subseteq R$, so this makes sense.
 \end{theorem}
 \begin{corollary}[4.14, Dimension-Summation formula]
   In the setting above, $\dim R' = \dim \bbar{R'} + \dim R$.
 \end{corollary}
 \begin{proof}
   easy chain composition argument.
 \end{proof}
 This allows us to come up with examples of valuation rings with dimension bigger than 1.

 \underline{Places}: intuitively, a place is a ``generalized field homomorphism'' that
 may send may elements to ``$\infty$''.
 \begin{definition}
   Let $K$ and $\W$ be fields. Then a \emph{place} is a map $\phi:K\to \W\sqcup
   \{\infty\} =: \W_\infty$ so that $\phi$ is a ``field homomorphism '' with the usual
   rules of addition and multiplication for $\infty$ ($\infty \pm \infty$, $0/0$,
   $\infty/\infty$, and $\infty\cdot 0$ are undefined).
 \end{definition}
 There is a triumvirate of ideas which are basically the same: valuation rings, places,
 and Krull valuations.

 Working with a place is equivalent to working with a valuation in the following way.
 Suppose $\phi$ is a place, then define $R=\phi^{-1}(\W)$. $R$ is a valuation ring of $K$
 \anton{}. Conversely, given a valuation ring $(R,\m)$ of $K$, there is a place $K\to
 R/\m \sqcup \{\infty\}$, sending $R$ to $R/\m$ in the usual way, and $K\smallsetminus R$
 to $\infty$.

 We say $K\to \W_\infty$ is the \emph{trivial place} if $\phi(K)\subseteq \W$ (i.e.~a
 good old field homomorphism)
