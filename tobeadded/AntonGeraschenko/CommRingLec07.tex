 \stepcounter{lecture}
 \setcounter{lecture}{7}
 \sektion{Lecture 7}

 \subsektion{Noetherian Descent}

 Certain theorems about commutative rings can be proven by a reduction to the noetherian
 case (the Hilbert basis theorem is secretly being used). If the statement $\F$ you want
 to prove only involves a finite number of elements of $R$, say $a_1,\dots, a_n$. Then
 look at the ring $R_0$ generated by $1, a_1,\dots, a_n$. It is the homomorphic image of
 the ring $\ZZ[x_1,\dots, x_n]$. By the Hilbert basis theorem, this ring is noetherian,
 so homomorphic images are also noetherian.

 Here is a typical application. A non-zero commutative ring satisfies the strong rank
 property ($R^m\hookrightarrow R^n$ implies $m\le n$). The strong rank property can be
 rephrased as $R^{n+1}\not\hookrightarrow R^n$ (module theoretically). This can be
 formulated as, ``a homogeneous system of $n$ equations and $n+1$ unknowns has a a
 nontrivial solution''. It suffices to solve this system in $R_0$, so apply descent. Now
 we just have to prove it in a noetherian ring. $R^{n+1}\hookrightarrow R^n$ can be
 easily contradicted in the noetherian case. Think of $R^{n+1}$ as $X_1=R^n\oplus R$;
 think of this $R$ as generated by $x$, so the image of $R^{n+1}$ is an image of $R^n$
 direct sum with another module. Now repeat by embedding $R^{n+1}$ into the copy of
 $R^n$. Then the module generated by $x,x_1,\dots, x_n$ gives an infinite ascending
 chain.

 \subsektion{\S 5 Artinian Rings}

 \begin{definition}
   The \emph{(Krull) dimension} of a commutative ring $R$ is defined as $\dim R =
   \sup\{$lengths of chains of primes$\}$.
 \end{definition}
 In particular, a zero dimensional ring is one in which every prime ideal is maximal:
 $\spec R=\Max R$.

 \begin{definition}[von Neumann]
   An element $a\in R$ is called \emph{von Neumann regular} if there is some $x\in R$
   such that $a=axa$.
 \end{definition}
 \begin{definition}[McCoy]
   A element $a\in R$ is \emph{$\pi$-regular} if some power of $a$ is von Neumann
   regular.
 \end{definition}
 \begin{definition}
   A element $a\in R$ is \emph{strongly $\pi$-regular} (in the commutative case)
   if the chain $aR\supseteq a^2R\supseteq a^3R\supseteq \cdots$ stabilizes.
 \end{definition}
 A ring $R$ is von Neumann regular (resp.\ (strongly) $\pi$-regular) if every element of
 $R$ is.

 \begin{theorem}[5.2]
   For a commutative ring $R$, the following are equivalent.
   \begin{enumerate}
     \item $\dim R=0$.
     \item $R$ is rad-nil (i.e. $\rad R = \nil R$) and $R/\rad R$ is von Neumann regular.
     \item $R$ is strongly $\pi$-regular.
     \item $R$ is $\pi$-regular.

     \item[] \hspace{-7ex} And any one of these implies
     \item $\C(R)=U(R)$; any non-zero-divisor is a unit.
   \end{enumerate}
 \end{theorem}
 \begin{proof}
   $1\Rightarrow 2\Rightarrow 3\Rightarrow 4 \Rightarrow 1$ and $4\Rightarrow 5$. We will
   not do $1\Rightarrow 2\Rightarrow 3$ here.

   ($3\Rightarrow 4$) Given $a\in R$, there is some $n$ such that $a^n R = a^{n+1}
   R=a^{2n}R$, which implies that $a^n = a^n x a^n$ for some $x$.

   ($4\Rightarrow 1$) Is $\p$ maximal? Let $a\not\in \p$. Since $a$ is $\pi$-regular, we
   have $a^n=a^{2n}x$, so $a^n(1-a^nx)=0$, so $1-a^nx\in \p$. It follows that $a$ has an
   inverse mod $\p$.

   ($4\Rightarrow 5$) Using $1-a^nx=0$, we get an inverse for $a$.
 \end{proof}
 \begin{example}
   Any local rad-nil ring is zero dimensional, since $2$ holds.
   In particular, for a ring $S$ and $\m\in \Max S$, $R=S/\m^n$ is zero dimensional
   because it is a rad-nil local ring.
 \end{example}
 \begin{example}[Split-Null Extension]
   For a ring $A$ and $A$-module $M$, let $R=A\oplus M$
   with the multiplication $(a,m)(a',m')=(aa',am'+a'm)$ (i.e.\ take the multiplication on
   $M$ to be zero). In $R$, $M$ is an ideal of square zero. ($A$ is called a
   \emph{retract} of $R$ because it sits in $R$ and can be recovered by quotienting by
   some complement.) If $A$ is a field, then $R$ is a rad-nil local ring, with maximal ideal $M$.
 \end{example}

 \[\def\x#1#2{\left\{\parbox{#1}{\hfuzz=1pt \centering #2}\right\}}
   \xymatrix @!0 @R=4pc @C=4pc{
     \x{3.5pc}{reduced rings} \ar@{-}[dr] & &
     \x{5pc}{zero-dimensional rings} \ar@{-}[dr]\ar@{-}[dl]& &
     \x{4.5pc}{noetherian rings} \ar@{-}[dl]\\
    & \x{6pc}{von Neumann regular rings} & & \x{3.3pc}{artinian rings}
 }\]
