 \stepcounter{lecture}
 \setcounter{lecture}{1}
 \sektion{Lecture 1}

 \subsektion{\S 1 Three theorems of McCoy}

  \begin{definition}
   An ideal $I\< R$ is called \emph{dense}\index{dense ideal} if $rI=0$ implies $r=0$.
   This is denoted $I\subseteq_d R$. This is the same as saying that ${}_RI$ is a
   faithful module over $R$.
 \end{definition}
 If $I$ is a principal ideal, say $Rb$, then $I$ is dense exactly when $b\in \C(R)$. The
 easiest case is when $R$ is a domain, in which case an ideal is dense exactly when it is
 non-zero.

 If $R$ is an integral domain, then by working over the quotient field, one can define
 the rank of a matrix with entries in $R$. But if $R$ is not a domain, rank becomes
 tricky. Let $\D_i(A)$ be the $i$-th \emph{determinantal ideal} in $R$, generated by all
 the determinants of $i\times i$ minors of $A$. We define $\D_0(A)=R$. If $i\ge
 \min\{n,m\}$, define $\D_i(A)=(0)$.

 Note that $\D_{i+1}(A)\supseteq \D_i(A)$ because you can expand by minors, so we have a
 chain
 \[
    R=\D_0(A)\supseteq \D_1(A)\supseteq \cdots \supseteq (0).
 \]
 \begin{definition}
   Over a non-zero ring $R$, the \emph{McCoy rank} (or just \emph{rank}) of $A$ to be
   the maximum $i$ such that $\D_i(A)$ is dense in $R$. The rank of $A$ is denoted
   $rk(A)$.
 \end{definition}
 If $R$ is an integral domain, then $rk(A)$ is just the usual rank. Note that over any
 ring, $rk(A)\le \min\{n,m\}$.

 If $rk(A)=0$, then $\D_1(A)$ fails to be dense, so there is some non-zero element $r$
 such that $rA=0$. That is, $r$ zero-divides all of the entries of $A$.

 If $A\in \MM_{n,n}(R)$, then $A$ has rank $n$ (full rank) if and only if $\det A$ is a
 regular element.

 \begin{exercise}
   Let $R=\ZZ/6\ZZ $, and let $A=diag(0,2,4)$, $diag(1,2,4)$, $diag(1,2,3)$, $diag(1,5,5)$
   ($3\times 3$ matrices). Compute the rank of $A$ in each case.
 \end{exercise}
 \begin{solution}\raisebox{-2\baselineskip}{
   $\begin{array}{c|cccc}
   A & \D_1(A) & \D_2(A) & \D_3(A) & \\ \hline
   diag(0,2,4) & (2) & (2) & (0) & 3\cdot (2)=0\text{, so }rk=0 \\
   diag(1,2,4) & R & (2) & (2) & 3\cdot (2)=0\text{, so }rk=1 \\
   diag(1,2,3) & R & R & (2) & 3\cdot (2)=0\text{, so }rk=2 \\
   diag(1,5,5) & R & R & R & \text{so }rk=3
  \end{array}$}
 \end{solution}
