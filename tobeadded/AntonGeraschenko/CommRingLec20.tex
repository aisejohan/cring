 \stepcounter{lecture}
 \setcounter{lecture}{20}
 \sektion{Lecture 20}

 Correction to cross-reference: p.~61, line~$-$12, the reference is to exercise I.70, not
 to I.55.

 We forgot to say that if $f:R\to S$ is a ring homomorphism, then $f^*:\spec S\to \spec
 R$ is defined as $f^*(\p)=\p^c=f^{-1}(\p)$. Then we have $(f^*)^{-1}\bigl(
 \V(J)\bigr)=\V\bigl(f(J)\bigr)$, so $f^*$ is continuous. We also get $(f\circ
 g)^*=f^*\circ g^*$, so $R\rightsquigarrow \spec R$ is a contravariant functor from
 the category of commutative unital rings to the category of compact $T_0$ spaces.

 \begin{remark}
   If $R$ and $S$ are not commutative, $f^{-1}(\p)$ is not prime in general!
 \end{remark}

 \begin{lemma}[Zariski]
   If $T$ is a ring-finite field extension of $k$, then $T$ is a module-finite extension
   of $k$.
 \end{lemma}
 \begin{proof}
   We have $T=k[t_1,\dots, t_n]$, and we need to prove that each $t_i$ is algebraic over
   $k$. Assume not. Then after relabeling, we may assume $t_1$,\dots, $t_r$ (with $r\ge
   1$) is a transcendence base for $T$ over $k$. Then $T$ is module-finite over
   $k(x_1,\dots, x_r)$ (by definition of transcendence base). By the Artin-Tate theorem,
   we get that $k(x_1,\dots, x_r)$ is ring-finite over $k$, contradicting Lemma
   \ref{lec19L:funcfieldnfin}.
 \end{proof}
 We can restate Zariski's lemma in the following way.
 \begin{theorem}
   Let $A$ be an affine $k$-algebra, and let $\m\in \Max(A)$, then $T=A/\m$ is a finite
   field extension of $k$.
 \end{theorem}
 \begin{proof}
   We have that $T$ is an affine $k$-algebra (i.e.\ it is ring-finite over $k$), and it
   is a field, so Zariski's lemma applies to give the desired result.
 \end{proof}
 \begin{corollary} \label{lec20C:1}
   If $f:B\to A$ is a $k$-algebra homomorphism of affine $k$-algebras. Then $f^*:\spec
   A\to \spec B$ takes closed points to closed points (i.e.\ takes $\Max A$ to $\Max B$).
 \end{corollary}
 \begin{proof}
   We need to prove that if $\m\in \Max A$, then $f^{-1}(\m)$ is maximal. We have
   injections $k\hookrightarrow B/f^{-1}(\m)\hookrightarrow A/\m$, and by the theorem,
   $A/\m$ is a finite extension of $k$. But we know that a ring inside an algebraic
   extension is a field (because the inverse of an element is a polynomial in that
   element by algebraicity).
 \end{proof}
 \begin{corollary}
   Let $\m\< A=k[x_1,\dots, x_n]$, then $\m$ is maximal if and only if there is an
   \emph{algebraic point} $(a_1,\dots, a_n)\in \bar k^n$ such that
   $\m=\I\bigl((a_1,\dots, a_n)\bigr)$. In particular, if $k=\bar k$, $\m$ is maximal if
   and only if it is of the form $(x-a_1,\dots, x-a_n)$.
 \end{corollary}
 \begin{proof}
   If $\m$ is maximal, then we have an algebraic extension $k\hookrightarrow
   A/\m\stackrel{\varphi}{\hookrightarrow} \bar k$, and we have $\varphi(x_i)=a_i$ for
   some $a_i$. For all $f\in A$, we have $\phi(f)=f(a_1,\dots, a_n)$. If $f\in \m$, then
   clearly $\phi(f)=0$, so $f(a_1,\dots, a_n)=0$.

   Conversely, if $\m=\I\bigl((a_1,\dots, a_n)\bigr)$, then we have the evaluation map at
   $(a_1,\dots, a_n)$ inducing $A/\m\hookrightarrow \bar k$. Then $A/\m$ is a ring in an
   algebraic extension of $k$, so it is a field, proving that $\m$ is maximal.
 \end{proof}
 \begin{corollary}[Weak NSS]
   If $J\< A=k[x_1,\dots, x_n]$ is proper, $V_{\bar k}(J)\neq \varnothing$.
 \end{corollary}
 \begin{proof}
   Enlarging $J$, we may reduce to the case $J=\m\in \Max A$, noting that $V_{\bar
   k}(\m)\subseteq V_{\bar k}(J)$. By the previous corollary, $(a_1,\dots, a_n)$ is an
   algebraic point on which all of $\m$ vanishes, as desired.
 \end{proof}
 \begin{corollary}
   Every maximal ideal $\m\subseteq A=k[x_1,\dots, x_n]$ can be generated by $n$
   irreducible polynomials of the form $f_1(x_1)$, $f_2(x_1,x_2)$, \dots, $f_n(x_1,\dots,
   x_n)$ ($f_i$ only uses the first $i$ variables).
 \end{corollary}
 \begin{proof}
   In the notes. \anton{}
 \end{proof}
 \begin{theorem}
   Every affine $k$-algebra $A$ is a \emph{Hilbert ring}, i.e.\ every prime ideal is an
   intersection (possibly infinite) of maximal ideals.
 \end{theorem}
 \begin{proof}
   Let $\p\in \spec A$, and let $s\not\in \p$; we wish to find a maximal ideal
   $\m\supseteq \p$ such that $s\not\in \m$. Consider the localization $A_s=A[s^{-1}]$,
   in which the extension $\p A_s$ of $\p$ is some proper ideal, so there is some maximal
   ideal $M\in \Max A_s$ such that $\p A_s\subseteq M$. Since $A$ and $A_s$ are both
   affine, $M^c\in \Max A$ by Corollary \ref{lec20C:1}, and $\p\subseteq M^c$, with
   $s\not\in M^c$, as desired.
 \end{proof}
 More generally, if $R$ is Hilbert, then any ring-finite extension $S$ of $R$ is Hilbert.
 The above theorem is the case where $R$ is a field.
