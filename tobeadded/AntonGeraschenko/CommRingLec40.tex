 \stepcounter{lecture}
 \setcounter{lecture}{40}
 \sektion{Lecture 40}

 Two corrections to the notes:\\
 p.~129, line 8: change ``Every'' to ``Up to associates, every''\\
 p.~131, line 20: Change ``the polynomial ring'' to ``a polynomial ring''.

 Note that a Krull domain is always completely normal (because DVRs are always completely
 normal).

 \begin{example}
   \begin{enumerate}
     \item Take $R$ a finite intersection of DVRs of $K$ (then the second condition is
     automatically satisfied.
     \item Noetherian normal domains are Krull. We can take $\{R_i\}=\{R_\p|\p\in \spec_1
     R\}$. Recall that an element is in a finite number of height 1 primes.
     \item UFDs are Krull. The defining family is the set of $R_{(\pi)}$, where $\pi$ is
     a prime element.
   \end{enumerate}
   \vspace*{-1.7\baselineskip}
 \end{example}
 \begin{theorem}
   $\{$Krull domains of dimension $\le 1\} = \{$Dedekind domains$\}$.
 \end{theorem}
 $\supseteq$ is clear. For the other direction, the only hard part is to show that $R$
 is noetherian. We won't do it here.

 Krull domains behave very well with respect to ``closure properties'':
 \begin{enumerate}
   \item $R$ Krull $\Longrightarrow$ any localization of $R$ is Krull.
   \item $R$ Krull $\Longrightarrow$ $R[\{x_i\}_{i\in I}]$ is Krull.
   \item $R$ Krull $\Longrightarrow$ $R[[x]]$ is Krull.
   \item Let $R$ be Krull with $Q(R)=K$, and let $L$ be a finite extension of $K$, with
   $S$ the integral closure of $R$ in $L$. If $R$ is Krull, then so is $S$.
   \item (Mori-Nagata Theorem) If $R$ is a noetherian domain then the integral closure
   $R^*$ is Krull.
 \end{enumerate}
 \begin{theorem}
   If $R$ is Krull, then for every $\p\in \spec_1 R$, $R_\p$ is a DVR. Moreover,
   $\{R_\p|\p\in \spec_1 R\}$ is a defining family for $R$.
 \end{theorem}
 \begin{definition}
   If $R$ is a Krull domain, the \emph{divisor class group} is $Cl(R) =
   \frac{D(R)}{div(K^\times)}$. $D(R)$ is the group of \emph{divisors}, the free abelian
   group on the set of height 1 primes. For each height 1 prime $\p$, we have a valuation
   $v_\p$. We define the set of \emph{principal divisors} to be $div(K^\times) = \{div(f)
   = \sum v_\p(f) \p| f\in K^\times\}$.
 \end{definition}
 \begin{theorem}
   If $R$ is a Krull domain, then $Cl(R)$ is trivial if and only if $R$ is a UFD.
 \end{theorem}
 $\Leftarrow$ is clear because each height 1 prime is principal. The other way is not
 hard either.

 Finally, the ideal class group $C(R)$ injects into $Cl(R)$ (with equality if $R$ is
 regular, whatever that means).

 \subsektion{Chapter IV: Dimension Theory}

 \begin{definition}
   Let $k$ be a field, and let $B$ be a $k$-algebra. We define $tr.d._k B = \sup
   \{tr.d._k (B/\p)|\p\in \Min(B)\}$.
 \end{definition}
 \begin{theorem}[Noether normalization]
   Let $k$ be a field, and $B$ an affine $k$-algebra. Then there exist algebraically
   independent (over $k$) $x_1,\dots, x_n\in B$ such that $B$ is integral over
   $A=k[x_1,\dots, x_n]$. In particular, since $B$ is finitely generated over $k$, it is
   module-finite over $A$.
 \end{theorem}
 \begin{example}
   Let $B=k[t^2,t^3]\subseteq k[t]$. Here $A=k[t^2]$, and it is clear that $t^3$ is
   integral over $A$.
 \end{example}
 \begin{example}
   Let $B=k[t,t^{-1}]$. Take $A=k[t+t^{-1}]$. Then note that $t$ and $t^{-1}$ satisfy
   $(x-t)(x-t^{-1}) = x^2-(t+t^{-1})x+1\in A[x]$.
 \end{example}
 \begin{proof}
   Write $B=k[y_1,\dots, y_m]$ and induct on $m$. The case $m=0$ is trivial. If
   $y_1,\dots, y_m$ are algebraically independent, we take $A=B$ and we're done. Thus, we
   may assume there is some dependence $f(y_1,\dots, y_m)=0$. Take $r$ larger than any
   exponent in $f$. Define $z_i:=y_i-y_1^{r^{i-1}}$ for $i\ge 2$. Then we get that
   $0=f(y_1,z_2+y_1^{r}, z_3+y_1^{r^2}, \dots, z_m + y_1^{r^m})$ has leading term
   $by_1^N$ for some huge $N$ and $b\neq 0$. This equation tells us that $y_1$ is
   integral over $B':=k[z_2,\dots, z_m]$. It is clear that all the other $y_i$ are also
   integral over $B'$, so $B$ is integral over $B'$. By induction, we can find a
   polynomial ring $A$ so that $B'$ is integral over $A$.
 \end{proof}
 It is clear that $tr.d._k B\le m$.
