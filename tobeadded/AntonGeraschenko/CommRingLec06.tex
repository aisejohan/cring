 \stepcounter{lecture}
 \setcounter{lecture}{6}
 \sektion{Lecture 6}

 \renewcommand\P{\mathcal{P}}

 \subsektion{\S 4 Noetherian Rings and Noetherian Induction}
 Recall the following facts about rings and modules with chain conditions.
 \begin{enumerate}
   \item A module is noetherian if and only if every submodule is finitely generated.
   \item A module is noetherian and artinian if and only if it has a (finite) composition
   series (quotients are simple). The Jordan-H\"older theorem tells us that the quotients
   are unique up to permutation.
   \item If $N\subseteq M$, $M$ is noetherian (resp.\ artinian) if and only if both
   $N$ and $M/N$ are.
   \item If $R$ is noetherian, then $M$ is noetherian if and only if it is finitely
   generated.
 \end{enumerate}

 \begin{theorem}[I.\ S.\ Cohen]\label{lec06T:cohen}
   A ring $R$ is noetherian if and only if all prime ideals are finitely generated.
 \end{theorem}
 For the proof, we need the following lemma and proposition.
 \begin{lemma}[Oka's Lemma]
   Let $I\< R$ and $b\in R$. If $I+(b)$ and $I:b$ are finitely generated, then so is $I$.
 \end{lemma}
 \begin{proof}
   Since $I+(b)$ is finitely generated, it can be written as $I_0+(b)$ for some finitely
   generated ideal $I_0\subseteq I$. Then $I = I_0+b(I:b)$ is finitely generated.
 \end{proof}
 \begin{proposition}\label{lec06P:maxisprime}
   If $I\< R$ is maximal with respect to \emph{not} being finitely generated (i.e.\ every
   $J\supsetneq I$ is f.g.), then $I$ is prime.
 \end{proposition}
 \begin{proof}
   Clearly $I\neq R$. Suppose $I$ is not prime, so there are $a,b\not\in I$, with $ab\in
   I$. Then $I+(b)$ and $I:b$ ($\ni a$) are finitely generated. By Oka's Lemma, $I$ is
   finitely generated, which is a contradiction.
 \end{proof}
 \begin{proof}[Proof of Theorem \ref{lec06T:cohen}]
   Assume all primes in $R$ are finitely generated, and that $\F=\{$non-f.g.\
   ideals$\}\neq \varnothing$. Since the union of non-finitely-generated ideals is not
   finitely generated, Zorn's Lemma gives us a maximal element, which is prime by
   Proposition \ref{lec06P:maxisprime}, so it is finitely generated by assumption.
   Contradiction.
 \end{proof}
 \[\xymatrix@!0 @R=3pc @C=10pc{
  \text{ \emph{all} ideals f.g.} \ar@{=>}@/_2.2ex/[r] \ar@{<=>}[d]
  &   \text{primes f.g.} \ar@{=>}[l]_{\text{Cohen}}
                         \ar@{==>}[r]^(.43){\text{Krull's PIT}}
                         \ar@{:>}[d]
  & \text{DCC on primes}\\
  \text{ACC on \emph{all} ideals} \ar@{=>}@/^2.2ex/[r]^{\text{Obvious}} & \text{ACC on primes}
 }\]

% Noetherian induction is this idea: let $R$ be noetherian, and we want to prove some
% statement for $R$. Assume it is false, and look at the collection of counterexamples,
% and get a maximal counterexample (by the maximum principle). Get a contradiction by hook
% or by crook.
 \begin{theorem}[Noetherian Induction Principle]
   Let $R$ be a noetherian ring, let $\P$ be a property, and let $\F$ be a family of
   ideals $R$. Suppose the inductive step: if all ideals in $\F$ strictly larger than
   $I\in \F$ satisfy $\P$, then $I$ satisfies $\P$. Then all ideals in
   $\F$ satisfy $\P$.
 \end{theorem}
 \begin{proof}
   Assume $\F_\text{crim} = \{J\in \F|J\text{ does not satisfy }\P\}\neq \varnothing$.
   Since $R$ is noetherian, $\F_\text{crim}$ has a maximal member $I$. By maximality, all
   ideals in $\F$ strictly containing $I$ satisfy $\P$, so $I$ also does by the inductive
   step.
 \end{proof}

 \begin{definition}
   An element $r$ is \emph{irreducible} if it cannot be written as a product of two
   non-units. A (proper) ideal $I$ is \emph{irreducible} if it cannot be written as the
   intersection of two strictly larger ideals. (irreducible ideals are primary!
   \anton{ref this result})
 \end{definition}

 \begin{example} For a noetherian ring $R$, we can prove the following results by
 checking the inductive step in each case. For the first two, take $\F$ to be the set of
 all ideals.
   \begin{enumerate}
     \item Every ideal is a finite intersection of irreducible ideals.

     Assume every ideal strictly containing $I$ is a finite intersection of irreducible
     ideals. If $I=R$, it is the empty intersection. If $I$ is irreducible, then we're
     done. Otherwise, $I = J_1\cap J_2$ for strictly larger ideals $J_1$ and $J_2$. By
     assumption, $J_i$ is a finite intersection of irreducible ideals, so $I$ is also.

     \item ($R\neq 0$) Every ideal in $R$ contains a finite product of primes.

     Assume any ideal larger than $I$ contains a finite product of primes. If $I$ is
     prime or $R$, we're done. Otherwise, there are ideals $J_1$ and $J_2$ which contain
     $I$ such that $J_1J_2\subseteq I$. Since each $J_i$ contains a finite product of
     primes, so does $I$.

     \item ($R$ a domain) Every $r\not\in (0)\cup U(R)$ is a finite product of
     irreducible elements.

     Here we let $\F$ be non-zero, non-$R$, principal ideals, and let $\P$ be the
     statement that the generator is a finite product of irreducible elements (note that
     this is independent of the choice of generator since $R$ is a domain). \anton{finish
     ... easy}

     \item If $J=\sqrt J$, then $J$ is a \emph{finite} intersection of primes.
     (Kaplansky's Theorems 87 and 88)

     Take $\F$ to be the set of radical ideals.\anton{finish}

     As a corollary, for any $I\< R$ in a noetherian ring, there are finitely many
     minimal primes over $I$.
   \end{enumerate}
   \vspace*{-1.7\baselineskip}
 \end{example}
 Note that the examples where $\F$ is not the set of all ideals illustrate that to
 apply noetherian induction, you only need the ideals \emph{in $\F$} to satisfy the
 ascending chain condition.
