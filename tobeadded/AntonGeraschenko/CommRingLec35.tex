 \stepcounter{lecture}
 \setcounter{lecture}{35}
 \sektion{Lecture 35}

 The notes have been revised to contain some more stuff. When is $(R,\m)$ or principal
 type? Here are some necessary and sufficient conditions (individually \dots any of them
 is enough)
 \begin{itemize}
   \item $\m\neq\m^2$
   \item $R$ surjects onto a DVR.
   \item $\Gamma^+\smallsetminus \{0\}$ has a least element.
 \end{itemize}

 \underline{Any OAG $\Gamma$ is a valuation group}. That is, there is a surjective
 valuation $v:K\twoheadrightarrow \Gamma_\infty$. For example, let $\Gamma=\QQ$ with the
 usual ordering. This produces a valuation ring $(R,\m)$ which is not of principal type
 ($\m=\m^2$).

 To prove the result, form the group algebra $A=k\Gamma$ over a fixed field $k$, with
 formal basis $\{t_\alpha|\alpha\in \Gamma\}$, with $t_\alpha t_\beta=t_{\alpha+\beta}$.
 Given $f\in A$, we have $f=\sum a_\alpha t_\alpha = a_{\alpha_0}t_{\alpha_0} + $``higher
 terms'' with $a_{\alpha_0}\neq0$ \dots we're taking $\alpha_0$ to be the least element
 that appears. Define $v(f)=\alpha_0$. It is immediate that this $v$ is a valuation, and
 that it surjects onto $\Gamma$. In particular, $A$ is a domain. Note that the residue
 field is $k$.

 \underline{Convex subgroups of an OAG}: A subgroup $G\subseteq \Gamma$ is called
 \emph{convex} (or \emph{isolated}) if whenever $0\le a\le b$ and $b\in G$, we also have
 $a\in G$.
 \begin{theorem}
   Convex subgroups are exactly the kernels of OAG morphisms.
 \end{theorem}
 \begin{proof}
   If $G=\ker (\phi:\Gamma\to \Gamma')$ and $0\le a\le b$ with $\phi(b)=0$, then we get
   $0\le \phi(a)\le 0$, so $\phi(a)=0$.

   Conversely, if $G$ is a convex subgroup of $\Gamma$, then we declare a non-identity
   coset positive if all elements are positive. After some checking, this makes
   $\Gamma/G$ into an OAG, and the natural map $\Gamma\to \Gamma/G$ is an OAG morphism.
   \anton{}
 \end{proof}
 \begin{lemma}
   Convex subgroups of $\Gamma$ form a chain (under inclusion).
 \end{lemma}
 \begin{proof}
   easy to check \anton{}
 \end{proof}
 \begin{definition}
   The \emph{order-rank} of an OAG is the order type of the chain of convex subgroups. If
   you're a baby, you can define $\ork(\Gamma)=n\in \ZZ$ if $\Gamma$ has exactly
   $n+1$ convex subgroups, and $\infty$ otherwise.
 \end{definition}
 There is an order reversing bijection between convex subgroups of $\Gamma$ and prime
 ideals in $R$ (any valuation ring with valuation group $\Gamma$), given by $G\mapsto
 \{a\in R|v(a)>G\}$.
 \[\xymatrix{
 0 & \m\\
 \Gamma & (0)
 }\]
 \begin{itemize}
   \item As a corollary, $\ork(\Gamma)=\dim R$.
   \item $\ork(\Gamma)\le 1$ corresponds to $\Gamma$ being archimedean.
   \item $\ork(\Gamma)\le rk(\Gamma) := \dim_\QQ(\QQ\otimes_\ZZ \Gamma)$.
 \end{itemize}

 \subsektion{\S 6. Characterizations of Normal Domains}
 \begin{lemma}[Chevalley]
   Let $K\supseteq V$ be rings, and let $u\in U(K)$. If $I\< V$ is a proper ideal, then
   either $I\cdot V[u]\neq V[u]$ or $I\cdot V[u^{-1}]\neq V[u^{-1}]$.
 \end{lemma}
 \begin{proof}
   Tricky calculation. \anton{}
 \end{proof}
 \begin{theorem}[Existence theorem for valuation rings]
   Let $S$ be a sub-ring of a field $K$, and let $\p\in\spec S$. Then there exists a
   valuation ring $(V,\m)\in Val(K)$ containing $S$ such that $\m\cap S=\p$.
 \end{theorem}
 \begin{proof}
   Localize at $\p$ first, so we may assume $S$ is local and $\p$ is the maximal ideal.

   Consider the family $\F=\{T| S\subseteq T\subseteq K, \p T\neq T\}$, ordered by
   inclusion. It is easy to see that Zorn's lemma applies to give a maximal member $V$.
   Then check that $V$ is a valuation ring of $K$: if $u,u^{-1}\in K\smallsetminus V$,
   then you could extend $V$ by Chevalley's lemma, contradicting maximality, so either
   $u\in V$ or $u^{-1}\in V$.
 \end{proof}
 \begin{definition}
   Let $(S,\p)$ and $(V,\m)$ be local rings, with $S\subseteq V$. We say that $V$
   \emph{deominates} $S$ (written $V\ge S$ or $S\le V$) if $\m\cap S=\p$.
 \end{definition}
 This can be understood in two alternative ways. It is sufficient for $\p\subseteq \m$,
 or to say that $\p V\subsetneq V$.

 In general, if $K/F$ is a field extension and $(V,\m)$ is a local ring in $K$, then
 $(S,\p)=(F\cap V,F\cap \m)$ a local ring dominated by $(V,\m)$. To see that $S$ is
 local, let $a\in S\smallsetminus \p\subseteq V\smallsetminus \m$, then $a$ is invertible
 in $V$ and the inverse must lie in $F$, so $a$ is invertible in $S$.

 Given a field $K$, define $Loc(K)$ to be the set of local rings in $K$.
 \begin{theorem}
   For every field $K$, $Loc(K)^*=Val(K)$, where $loc(K)$ is ordered by dominance.
 \end{theorem}
 \begin{proof}
   If $S\in Loc(K)^*$, then there is a $V\in Val(K)$ so that $V\ge S$ \anton{}. This
   implies $V=S$. Conversely, if $(V,\m)\in Val(K)$, then if $V'\ge V$ in $Loc(K)$,
   $\m'\subseteq \m$, so $V'$ cannot dominate $V$, so $V$ is maximal in $Loc(K)$.
 \end{proof}
