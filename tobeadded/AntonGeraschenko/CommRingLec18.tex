 \stepcounter{lecture}
 \setcounter{lecture}{18}
 \sektion{Lecture 18}

 noetherian $\Rightarrow$ subspaces are noetherian $\Rightarrow$ subspaces are compact.
 If $K\supseteq k$, then the $k$-topology on $K^n$ is noetherian (hence compact).

 \begin{proposition}
   If $X$ is a non-empty topological space, then the following are equivalent.
   \begin{enumerate}
     \item $X$ is not the union of two proper closed subsets.
     \item Any two non-empty open sets intersect.
     \item Any non-empty open set is dense in $X$.
   \end{enumerate}
 \end{proposition}
 \begin{proof}
   easy exercise.
 \end{proof}
 In this case, we call $X$ \emph{irreducible}. In particular, ``irreducible subspace'' is
 meaningful (a subspace which is irreducible in the subspace topology).
 \begin{corollary}
   $Y\subseteq X$ is an irreducible subspace if and only if $\bbar Y$ is irreducible.
 \end{corollary}
 \begin{proof}
   Follows from the fact that an open set intersects $Y$ is and only if it intersects
   $\bbar Y$.
 \end{proof}
 \begin{example}
   Singleton subspaces (and therefore their closures) are irreducible.
 \end{example}
 \begin{definition}
   A maximal irreducible subset $Y$ of $X$ is called an \emph{irreducible component} of $X$.
 \end{definition}
 Such a $Y$ is always closed because $Y\subseteq \bbar Y$, which is irreducible, so by
 maximality, $Y=\bbar Y$.
 \begin{proposition}
   \begin{enumerate}
     \item Any irreducible set in $X$ is contained in some irreducible component.
     \item $X$ is the union of its irreducible components.
   \end{enumerate}
 \end{proposition}
 \begin{proof}
   (1) Zorn's Lemma.
   (2) Every point is in some irreducible component by (1).
 \end{proof}
 Note that in a Hausdorff space, any two points are separated by open sets, so they
 cannot be in an irreducible component together, i.e.\ only irreducible sets are points.

 \begin{theorem}
   If $X$ is noetherian, then the number of irreducible components is finite. If
   $\{X_i\}$ are the irreducible components, all of them are needed to cover $X$.
 \end{theorem}
 \begin{proof}
   If $X_i$ can be omitted, it is contained in the remaining union, so it must be
   contained in one of the other components (since it is irreducible), contradicting the
   fact that it is a maximal irreducible set.

   \anton{}
 \end{proof}
 \begin{theorem}
   Given $k\subseteq K$, let $X\subseteq K^n$ be a $k$-algebraic set. The irreducible
   components of $X$ (in the $k$-topology) are given by $V_K(\p_i)$, where the $\p_i$
   are the minimal primes over the ideal $I(X)$. Moreover, $\p_i=I\bigl(V(\p_i)\bigr)$.
 \end{theorem}
 Note that it would be wrong to say that $I\bigl(V(\p)\bigr)=\p$ for all primes because
 $V(\p)$ may be empty. Note that we don't use the Nullstellensatz.
 \begin{proof}
   check the proof in the notes \anton{}.
 \end{proof}

 \underline{Generic Points}: For a subset $Y\subseteq X$, a point $y\in Y$ is called a
 generic point of $Y$ if $\bbar{\{y\}}=Y$. Note that to have a generic point, $Y$ must be
 closed and irreducible. In general, these conditions are not sufficient! In classical
 algebraic geometry, even $k$-varieties need not have generic points!
 \begin{example}
   Let $k=\bar k$, and $Y=V(y-x^2)$, the parabola. $Y$ is irreducible, but it has no
   generic point in the $k$-topology because points are closed (since $k=\bar k$).
 \end{example}
 Classically, we take $k\subseteq K$ to have infinite transcendence degree (e.g.\
 $\QQ\subseteq \CC$). In this case, $k$-varieties in $K^n$ will have generic points. In
 the example above, take $K=\text{Frac}\left(\frac{k[s,t]}{(t^2-s)}\right)$. Then $(\bar
 s, \bar t)$ is a generic point for the parabola.

 \subsektion{\S 3. Zariski Prime Spectrum}
 \begin{tabular}{c|c|c|}
    & algebraic & geometric\\ \hline
   Classical & $k[x_1,\dots, x_n]$ & $K^n$, $k$-algebraic sets, etc. \\ \hline
   Grothendieck & any (commutative) $R$ & $\spec R$
 \end{tabular}\\
 Define $\V(J)=\{\p\in \spec R| J\subseteq \p\}$.
 \begin{theorem}
  \begin{enumerate}\item[]
   \item Taking $\V(J)$ to be closed sets gives a topology on $\spec R$.
   \item The sets $\D(f) = \spec R\smallsetminus \V(f)$, $f\in R$, are a basis for
         the topology.
   \item $\spec R$ is a compact $T_0$ space.\footnote{Given two points, one of them
         (you don't know which) has an open neighborhood avoiding the other.}
  \end{enumerate}
 \end{theorem}
 \begin{proof}
  (1) This follows from $\bigcap \V(J_\alpha)=\V\bigl( \sum J_\alpha \bigr)$ and
  $\V(J_1)\cup \V(J_2)=\V(J_1\cap J_2)$.

  (2) An open set is of the form $\spec R\smallsetminus \V(J) = \bigcup_{f\in J}\D(f)$.

  (3) If $\p,\p'\in \spec R$ are distinct, then there is some $f\in \p\smallsetminus \p'$
  or $f\in \p'\smallsetminus \p$, say the former. Then $\D(f)$ contains $\p'$ but note
  $\p$. This proves $T_0$. If you have a cover, then refine it to a cover by open sets of
  the form $\D(f_\alpha)$. Then the $f_\alpha$ generate the unit ideal \anton{}, so a
  finite number of them give 1, so those $\D(f_\alpha)$ cover.
 \end{proof}
