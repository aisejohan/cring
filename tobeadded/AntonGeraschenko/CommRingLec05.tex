 \stepcounter{lecture}
 \setcounter{lecture}{5}
 \sektion{Lecture 5}

 \begin{example}
   Let's compute the primes associated to $I=\p_1\p_2$ under the assumptions that (1)
   $\p_1=(x)$, where $x$ is regular; (2) There is some $y\in \p_2\smallsetminus \p_1$.

   We claim that $I:y=\p_1$ and $I:x=\p_2$, which would show that $\p_1$ and $\p_2$ are
   associated prime. The inclusions $\supseteq$ are clear. For the other inclusion,
   assume $ty\in I=\p_1\p_2\subseteq \p_1$. Since $y\not\in \p_1$, we get $t\in \p_1$, as
   desired. If $tx\in I=\p_1\p_2 = x\p_2$. Since $x$ is regular, we can cancel it, so
   $t\in \p_2$.

   Now we'd like to show that there are no other associated primes. We have
   $I=\p_1\p_2\subseteq \p_1=(x)\subseteq R$, so any associated prime is an associated
   prime of $R/\p_1$ or of $\p_1/\p_1\p_2\cong R/\p_2$ (via multiplication by $x$). But
   we showed that the only associated prime of $R/\p$ is $\p$.
 \end{example}
 \begin{example}
   Now let's specialize the previous example to $R=k[x,y]$ where $k$ is a field, and $x$
   and $y$ as in the previous example: $\p_1=(x)$ and $\p_2$ is one of the following:
   \begin{itemize}
     \item $\p_2=(y)$, $I=(xy)$. In this case, we see that $\p_1$ and $\p_2$ are the
     minimal primes over $I$; if $\p\supseteq I=\p_1\p_2$, then $\p$ must be one or the
     other since it is minimal (it must contain one since $\p$ is prime).

     \item $\p_2=(x,y)$, $I=(x^2,xy)$. Here it is clear that $\sqrt I=(x)$. The
     remarkable thing is that $(x,y)$ is an ``embedded associated prime'' to $I$. We call
     an associated prime if it contains another embedded prime; if it doesn't contain an
     associated prime, it is called an ``isolated prime'' of $I$.
   \end{itemize}
   \vspace*{-1.5\baselineskip}
 \end{example}

 Recall that $\supp M = \{\p\in \spec R| M_\p\neq 0\}$. If $I\subseteq R$ is a subset, then we
 define $\V(I) = \{\p\in \spec R|I\subseteq \p \}$.
 \begin{proposition}
   For any module $M$ over $R$, let $I=\ann M$.
   \begin{enumerate}
     \item \label{lec05:1} ``Specialization'': If $\p'\subseteq\p$ and $\p'\in\supp M$, then $\p\in \supp
     M$.
     \item \label{lec05:2} $\ass M \subseteq \supp M$.
     \item \label{lec05:3} $\supp M \subseteq \V(I)$.
     \item \label{lec05:4} If $M$ is finitely generated, then $\supp M=\V(I)$.
   \end{enumerate}
 \end{proposition}
 \begin{proof}
   (\ref{lec05:1}) Let $\p'\subseteq \p$ and assume $\p\not\in \supp M$, so $M_\p=0$. Then
   $M_{\p'}=(M_\p)_{\p'} = 0$.

   (\ref{lec05:2}) If $\p\in \ass M$, then $R/\p\hookrightarrow M$.
   Localizing, we get $R_\p/\p_\p = (R/\p)_\p\hookrightarrow M_\p$, so $M_\p$ contains a
   field, so it is non-zero.

   (\ref{lec05:3}) Take $\p\not\in \V(I)$, so there is some $i\in
   I\smallsetminus \p$ so that $iM=0$. This implies that $M_\p=0$.

   (\ref{lec05:4}) Say $M = \sum_{i=1}^n Rm_i$. Take $\p\in\V(I)$ and assume it is not a
   supporting prime, so $M_\p=0$. Then for each $i$, there is some $r_i\not\in \p$ so
   that $r_im_i=0$. Then $r=r_1\cdots r_n$ kills all of $M$, so it is in $I$, but
   not in $\p$ (since $\p$ is prime). Contradiction.
 \end{proof}
 \begin{proposition}[Theorem 84 in Kaplansky]
   For any module $M$, any minimal prime $\p$ over $I=\ann M$ must lie in $\Z(M)$.
   \mpar[ ``Minimal primes consist of zero divisors'']{}
 \end{proposition}
 In particular, taking $M=R$, we get that minimal primes in $R$ lie in $\Z(R)$. Note that
 there are \emph{no chain conditions} on $M$ in this proposition.
 \begin{proof}
   Let $I\subseteq \p$ and we know $I\subseteq \Z(M)$. Then $R/\p$ and $R\smallsetminus
   \Z(M)$ are multiplicative sets away from $I$. Let $S$ be the multiplicative set
   generated by these two. We claim that $S$ is disjoint from $I$. To see this, assume
   $a\not\in \p$ and $b\not\in \Z(M)$ such that $ab\in I$, so $abM=0$. But $bM=M$ by
   assumption, so $aM=0$, so $a\in I\subseteq \p$, which is a contradiction.

   Thus, there is some prime $\p'\supseteq I$ which is disjoint from $S$. It follows that
   $\p'\subseteq \p$, and $\p$ is minimal over $I$, so $\p'=\p$. Similarly, $\p'\subseteq
   \Z(M)$. That is, $\p\subseteq \Z(M)$, as desired.
 \end{proof}
 \begin{theorem}\label{lec05radNOembedded}
   Let $I\< R$ be a radical ideal, and consider the cyclic module $R/I$.
   \begin{enumerate}
     \item \label{lec05cyc1} $\Z(R/I) = \bigcup_{\p\text{ min'l over } I} \p$.
     \item \label{lec05cyc2} Every $\p\in \ass (R/I)$ is a minimal prime over $I$.
     \mpar[ ``A radical ideal has no embedded points'']{}
   \end{enumerate}
 \end{theorem}
 \begin{proof}
   (\ref{lec05cyc1}) The inclusion $\supseteq$ is clear from the previous proposition.
   Given $x\in \Z(R/I)$, fix $y\not\in I$ such that $xy\in I$. Then $I\subsetneq I:x$
   (since $y\not\in I$). So there is a minimal prime $\p$ over $I$ such that
   $I:x\not\subseteq \p$ (since $I$ is the intersection of minimal primes over it). But
   $x\cdot (I:x)\subseteq \p$, which implies that $x\in \p$.

   (\ref{lec05cyc2}) Let $\p=I:m$ be an associated prime, where $m\not\in I$. Then
   $m\not\in \p$, lest $m\cdot m=0\in I$, which would imply $m\in I$ since $I$ is
   radical. Now assume that there  is some $\p'$ so that $\p\supsetneq \p'\supseteq I$.
   Fix $x\in \p\smallsetminus \p'$. Then $xm\in I\subseteq \p'$, which implies $m\in \p$.
   Contradiction.
 \end{proof}

 \begin{proposition}
   Let $S\subseteq R$ be a multiplicative set disjoint from some prime $\p$. For any
   \mpar[ ``$\ass M$ behaves well under localization'']{}
   module $M$, if $\p\in \ass M$, then $\p_S\in \ass (M_S)$. If $\p$ is finitely
   generated, then the converse holds.
 \end{proposition}
 Assuming all primes in $\ass M$ are finitely generated, this can be rewritten as the
 equality
 \[
    \ass (M_S) = \ass(M) \cap \spec (R_S).
 \]

 This is proven in the notes. \anton{}
