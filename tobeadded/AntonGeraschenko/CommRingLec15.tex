 \stepcounter{lecture}
 \setcounter{lecture}{15}
 \sektion{Lecture 15}

 In exercise I.67, ``$(x,y)^2$'' should be ``$(x_1,x_2)^2$''.

 \begin{example}
   Let $I=(x^2-yz, x(z-1))\< k[x,y,z]$. Then $\p_1=(x,z)=I:y(z-1)$, $\p_2=(x,y)=I:z(z-1)$,
   $\p_3=(x^2-y,z-1)=I:x$, and $I=\p_1\cap\p_2\cap\p_3$. In particular, $I$ is radical.
 \end{example}

 We want to understand what $IN=N$ means.

 Let $I\< R$ and ${}_R M$ finitely generated. Let
 $E=\End_R (M)$, which is not commutative in general. We may view $M$ as an $E$-module
 ${}_E M$. Since every element in $R$ commutes with all of $E$, $E$ is an $R$-algebra (i.e.\
 There is a homomorphism $R\to E$ sending $R$ into the center of $E$).
 \begin{lemma}[Determinant Trick]
  \begin{enumerate}\item[]
    \item Every $\phi\in E$ such that $\phi(M)\subseteq IM$ satisfies a monic equation
    of the form $\phi^n+a_1\phi^{n-1} +\cdots + a_n=0$, where each $a_i\in I$, i.e.\
    $\phi$ is ``integral over $I$''.

    \item $IM=M$ if and only if $(1-a)M=0$ for some $a\in I$.
  \end{enumerate}
 \end{lemma}
 \begin{proof}
   (1) Fix a finite set of generators, $M=Rm_1+\cdots + Rm_n$. Then we have
   $\phi(m_i)=\sum_j a_{ij} m_j$, with $a_{ij}\in I$ by assumption. Let $A=(a_{ij})$.
   Then these equations tell us that $(I\phi-A)\vec{m}=0$. Multiplying by the adjoint of
   the matrix $I\phi-A$, we get that $\det(I\phi-A)m_i=0$ for each $i$. It follows that
   $\det(I\phi-A)=0\in E$. But $\det(I\phi-A)=\phi^n+a_1\phi^{n-1}+\cdots +a_n$ for some
   $a_i\in I$.

   (2) The ``if'' part is clear. The ``only if'' part follows from (1), applied to
   $\phi=\id_M$.
 \end{proof}
 \begin{remark}
   Determinant trick (part 2) actually includes Nakayama's Lemma, because if $I$ is in
   $\rad R$, $(1-a)$ is a unit, so $M=(1-a)M=0$.
 \end{remark}
 \begin{corollary}
   For a finitely generated ideal $I\< R$, $I=I^2$ if and only if $I=eR$ for some
   $e=e^2$.
 \end{corollary}
 \begin{proof}
   ($\Leftarrow$) clear.

   ($\Rightarrow$) Apply determinant trick (part 2) to the case $M={}_R I$. We get
   $(1-e)I=0$ for some $e\in I$, so $(1-e)a=0$ for each $a\in I$, so $a=ea$, so $I$ is
   generated by $e$. Letting $a=e$, we see that $e$ is idempotent.
 \end{proof}
 \begin{corollary}[Vasconcelos-Strooker Theorem]
   For any finitely generated module $M$ over \emph{any} commutative $R$. If $\phi\in
   \End_R(M)$ is onto, then it is injective.
 \end{corollary}
 \begin{proof}
   We can view $M$ as a module over $R[t]$, where $t$ acts by $\phi$. Apply the
   determinant trick (part 2) to $I=t\cdot R[t]\subseteq R[t]$. We have that $IM=M$
   because $\phi$ is surjective, so $m =\phi(m_0)=t\cdot m_0\in IM$. It follows that
   there is some $th(t)$ such that $(1-th(t))M=0$. In particular, if $m\in  \ker \phi$,
   we have that $0=(1-h(t)t)m=1\cdot m=m$, so $\phi$ is injective.
 \end{proof}
 Now we can make Krull's intersection theorem more impressive:
 \begin{theorem}[Krull's Intersection Theorem]
   Let $M$ be finitely generated over a noetherian ring $R$, and let $I\< R$ be an ideal.
   Define $N = \bigcap_{n\ge 0} I^nM$. Then $N=\{m\in M|(1-a)m=0\text{ for some } a\in
   I\}$.
 \end{theorem}
 \begin{proof}
   The inclusion $\supseteq$ is trivial since $m=am=a^2m=\cdots \in I^nM$ for each $n$.
   The other inclusion follows from the preliminary version and the determinant trick
   (part 2), noting that $N$ remains finitely generated.
 \end{proof}
 Here are some special cases of the Krull intersection theorem.
 \begin{corollary}
   If $I\< R$ is a proper ideal in a noetherian domain, then $\bigcap_{n\ge 0} I^n=0$.
 \end{corollary}
 \begin{proof}
   The set $1-I$ consists of non-zero elements, and hence non-zero-divisors. Apply Krull
   to $M=R$.
 \end{proof}
 \begin{corollary}
   Let $M$ be finitely generated over a noetherian ring $R$. Let $J\subseteq \rad R$.
   Then $\bigcap_{n\ge 0} J^nM=0$.
 \end{corollary}
 \begin{proof}
   Again, $1-J$ consists of units.
 \end{proof}
