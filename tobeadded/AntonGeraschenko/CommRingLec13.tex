 \stepcounter{lecture}
 \setcounter{lecture}{13}
 \sektion{Lecture 13}

 \begin{example}
  \begin{enumerate}
   \item Prime ideals are primary.

   \item In a rad-nil local ring (0-dimensional local ring), any proper ideal is primary.
   As a consequence, if $\m\in \Max R$ for any ring $R$, and $\m^n\subseteq \q\subseteq
   \m$, then $\q$ is $\m$-primary.

   \item In a UFD, a principal ideal $(a)$ is primary if and only if $a$ is 0 or a power
   of an irreducible element; see Ex.\ 52.

   \item In $R=k[x,y]$, where $k$ is a field. Then $I=(x^2,xy)$ is \emph{not} primary. In
   the factor ring, $y$ is a zero-divisor because $xy\in I$, but $y$ is not nilpotent
   modulo $I$.

   \item Let $\q=(y^2,x+yz)\< k[x,y,z]$. We claim that $\q$ is $\p$-primary for
   $\p=(x,y)$. First check that $\p=\sqrt \q$. $R/\q = k[x,y,z]/(y^2,x+yz)\cong
   k[y,z]/(y^2)$, in which every zero-divisor is nilpotent because $(y^2)$ is primary in
   $k[y,z]$.
  \end{enumerate}
  \vspace*{-1.7\baselineskip}
 \end{example}
 \begin{theorem}[Lasker-Noether Primary Decomposition]
   Let $R$ be noetherian and $N\subsetneq M$, with $M/N$ finitely generated. Then $N$ has
   a minimal primary decomposition in $M$.
 \end{theorem}
 \begin{proof}
   First write $N=Q_1\cap \cdots \cap Q_n$, where the $Q_i$ are irreducible. By Noether's
   theorem, each $Q_i$ is $\p_i$-primary for some prime $\p_i$.

   If $\p_i=\p_j$, we may replace $Q_i$ and $Q_j$ by $Q_i\cap Q_j$, which is also
   $\p_i$-primary. Now we may assume the $\p_i$ are distinct.

   Now remove unneeded $Q_i$ until we have a minimal decomposition.
 \end{proof}
 What can we say about uniqueness? Recall some facts about localization of modules at a
 multiplicative set $S\subseteq R$.
 \begin{definition}
   The \emph{$S$-saturation of $Q$} is $\{m\in M|sm\in Q\text{ for some }s\in S\}$.
 \end{definition}
 \begin{exercise}
   The saturation of $Q$ is $Q^{ec}$, the contraction of the extension of $Q$, i.e.\
   ``$Q_S\cap M$''$=i^{-1}(Q_S)$, where $i:M\to M_S$.
 \end{exercise}
 \begin{lemma}
   $Q\subsetneq M$. Suppose $S\cap \Z(M/Q)=\varnothing$, then $Q^{ec}=Q$.
 \end{lemma}
 \begin{proof}
   if $m\in Q^{ec}$, then $sm\in Q$ for some $s\in S$, which implies $m\in Q$ since
   $s\not\in \Z(M/Q)$. The reverse inclusion is clear.
 \end{proof}
 We will apply this to the situation where $Q$ is a $\p$-primary submodule of $M$ and
 $S=R\smallsetminus \p$.

 \begin{theorem}[Main Uniqueness Theorem]
   Same hypotheses as in the Lasker-Noether Theorem. Let $N=Q_1\cap \cdots \cap Q_n$ be
   \emph{any} minimal primary decomposition, where $Q_i$ is $\p_i$-primary. Then
   \begin{enumerate}
     \item $\ass (M/N) = \{\p_1,\dots, \p_n\}$. In particular, all the $\p_i$ are
     uniquely determined (up to permutation).

     \item If $\p\in \ass (M/N)_*$ is an isolated prime, then $Q_i=N^{ec}$ (with respect
     to localization at $\p_i$). In particular, $Q_i$ is uniquely determined.
   \end{enumerate}
 \end{theorem}
 \begin{proof}
  \begin{enumerate}
   \item[]
   \item
   $M/N\hookrightarrow \bigoplus M/Q_i$, so
   $\ass(M/N)\subseteq \bigcup \ass (M/Q_i) = \{\p_1,\dots, \p_n\}$.
   Now let's show that $\p_1\in \ass (M/N)$. Fix $x\in
   (Q_2\cap\cdots \cap Q_n)\smallsetminus Q_1$ (such an $x$ exists by minimality of the
   decomposition). Since $\p_1$ is finitely generated ($R$ noetherian),
   $\p_1^{k+1}x\subseteq Q_1$ for some $k\gg 1$; choose the minimal such $k$. Then there
   is some $y\in \p_1^k x\smallsetminus Q_1$. Then we have $\bar y \in M/N$ is non-zero,
   and that $\p_1 y=0\in M/N$, so $\p_1\subseteq \ann (\bar y)$. For the reverse
   inclusion, suppose $r\bar y=0$, so $ry\in N\subseteq Q_1$. Thus, $r\in
   \Z(M/Q_1)=\p_1$.

   \item Let $\p=\p_i$ be an isolated prime. By assumption, $\p_j\not\subseteq \p$ for
   $j\neq i$. Choose $k$ large enough so that $\p_j^k M\subseteq Q_j$ (for $j\neq i$)
   ($M/N$ finitely generated). But $\p_j^k\not\subseteq \p$. Now consider localization at
   $\p$; we get $(\p_j)_{\p}^k M_\p = M_\p = (Q_j)_\p$ for each $j\neq i$. Localizing the
   equation $N=Q_1\cap \cdots \cap Q_n$, we get
   \[
    N_\p=(Q_1\cap \cdots \cap Q_n)_\p = \bigcap_j (Q_j)_\p = (Q_i)_\p.
   \]
   So we get $N^{ec}=Q_i^{ec} = Q_i$.\qedhere
  \end{enumerate}
 \end{proof}
 \begin{example}[A point from last time]
   The $\p_i$ being distinct doesn't guarantee minimality of the decomposition. If
   $N=Q_1\cap Q_2\cap Q_3$. Choose some $\p$ containing
 \end{example}
