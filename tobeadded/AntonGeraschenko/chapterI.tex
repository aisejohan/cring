\documentclass{article}
\usepackage{amsmath,
            latexsym,
            amssymb}
\usepackage[all]{xy}

 \newcommand\<{\triangleleft}
 \renewcommand{\a}{\ensuremath{\mathfrak{a}}}
 \DeclareMathOperator{\ann}{ann}
 \DeclareMathOperator{\ass}{Ass}
 \DeclareMathOperator{\aut}{Aut}
 \newcommand\C{\mathcal{C}}
 \newcommand{\CC}{\ensuremath{\mathbb{C}}}
 \newcommand{\D}{\ensuremath{\mathcal{D}}}
 \DeclareMathOperator{\End}{End}
 \newcommand{\F}{\ensuremath{\mathcal{F}}}
 \newcommand{\FF}{\ensuremath{\mathbb{F}}}
 \newcommand{\HH}{\ensuremath{\mathbb{H}}}
 \let\hom\relax % kills the old hom
 \DeclareMathOperator{\hom}{Hom}
 \renewcommand{\labelitemi}{--}                    % changes the default bullet in itemize
 \newcommand{\id}{\mathrm{Id}}
 \DeclareMathOperator{\im}{im}
 \newcommand{\m}{\ensuremath{\mathfrak{m}}}
 \DeclareMathOperator{\Max}{Max}
 \DeclareMathOperator{\Min}{Min}
 \newcommand{\MM}{\ensuremath{\mathbb{M}}}
 \DeclareMathOperator{\nil}{Nil}
 \newcommand\p{\mathfrak{p}}
 \renewcommand\P{\mathcal{P}}
 \newcommand\q{\mathfrak{q}}
 \newcommand{\QQ}{\ensuremath{\mathbb{Q}}}
 \DeclareMathOperator{\rad}{rad}
 \newcommand{\RR}{\ensuremath{\mathbb{R}}}
 \newcommand{\smaltrix}[4]{\ensuremath{\left( %
            \begin{smallmatrix} #1 & #2 \\ #3 & #4 \end{smallmatrix} \right)}}
 \DeclareMathOperator{\spec}{Spec}
 \DeclareMathOperator{\supp}{Supp}
 \newcommand{\V}{\mathcal{V}}
 \newcommand\Z{\mathcal{Z}}
 \newcommand{\ZZ}{\ensuremath{\mathbb{Z}}}

 \openout0=\jobname solved.txt
 \openout1=lastupdated.html
 \newenvironment{exercise}[1]{\gdef\currentEx{#1}\begin{trivlist}\item[]%
                \textbf{Exercise #1.} \it}{\end{trivlist}}
 \makeatletter
 \newenvironment{solution}[1]{\def\x{#1}\begin{trivlist}\item[]\hspace*{-.5em}[\x]}
                {\hspace*{\fill} $\blacksquare$
                \protected@write0{}{\currentEx, \x}
                \end{trivlist}}
 \makeatother

\begin{document}

 \write1{\today}
 \closeout1

 {\large \noindent Solutions for Chapter I. Updated \today.}
 \begin{exercise}{I.1}
   Give a proof for the fact (used in the proof of (1.1)) that any unipotent element in a
   commutative ring is a unit. Does this work over a noncommutative ring?

   Clarification: ``unipotent'' means ``unit plus a nilpotent''.
 \end{exercise}
 \begin{solution}{anton@math}
   Assume $x=u+r$ is unipotent, where $uv=vu=1$ and $r$ is nilpotent. So for some $n$,
   $0=(1-vx)^n=\sum_{i=0}^n \binom{n}{i}(-vx)^i$. Then $1 = x \cdot \sum_{i=1}^n \binom ni
   v^i(-x)^{i-1}$ so $x$ is invertible.

   The result is false over noncommutative rings. Let $R$ be a nonzero ring. Then
   $\smaltrix 0110$ is a unit in $\MM_2(R)$, $\smaltrix 0{-1}00$ is nilpotent, but
   $\smaltrix 0110 + \smaltrix 0{-1}00 = \smaltrix 0010$ is not a unit since its
   determinant is not invertible.
 \end{solution}

 \begin{exercise}{I.2}
   (Snapper) Show that $a_0 + a_1x + \cdots + a_n x^n$ is nilpotent in $R[x]$ if
   and only if all $a_i$ are nilpotent in $R$; that is, $\nil (R[x])=(\nil R)[x]$. Show
   that these are also equal to $\rad (R[x])$ by using Theorem 1.1.
 \end{exercise}
 \begin{solution}{anton@math}
   Observation: if $a$ and $b$ are commuting ring elements, with $a^n=b^m=0$, then
   $(a+b)^{n+m}=0$ (since every term in the expansion has a factor of either $a^n$ or
   $b^m$). So the set of nilpotent elements in a commutative ring is closed under
   addition.

   If all of the $a_i$ are nilpotent, then $a_i x^i$ is nilpotent for all $i$, so
   $a_0+a_1x+\cdots +a_nx^n$ is nilpotent by the observation. Conversely, if
   $(a_0+a_1x+\cdots a_nx^n)^m=0$, then the leading term, $(a_n x^n)^m$, must be zero, so
   $a_nx^n$ is nilpotent (and therefore $a_n$ is nilpotent). Then $a_0+a_1x+\cdots
   a_{n-1}x^{n-1}$ is nilpotent by the observation (we've added $-a_nx^n$, which is
   nilpotent). By induction on degree, the rest of the $a_i$ are nilpotent.

   What is Theorem 1.1? Anyway, it is clear that $\nil (R[x])\subseteq \rad (R[x])$, so
   we just need to show the reverse inclusion. If $f=a_0+a_1x+\cdots + a_nx^n$ is in
   $\rad(R[x])$, then $1+xf$ is invertible. For any prime ideal $\mathfrak p\subseteq R$,
   we get a homomorphism $R[x]\to (R/\mathfrak p) [x]$ by reducing the coefficients
   modulo $\mathfrak p$, and the image of $1+xf$ must be invertible. But the only
   invertible elements of $(R/\mathfrak p)[x]$ are constants because $R/\mathfrak p$ is a
   domain (when you multiply non-constant polynomials, the leading terms cannot
   cancel). Thus, the coefficients $a_i$ all lie in $\mathfrak p$. Since this is true for
   every prime ideal, each $a_i$ is in the intersection of all primes, so it is
   nilpotent. By the first part of the exercise, $f$ is nilpotent.
 \end{solution}

 \begin{exercise}{I.3} %% replace XXXX with the exercise number, like I.20
    Prove that if $f=a_0+a_1x+a_2x^2+...+a_kx^k \in R[x]$ is
    invertible, then $a_0$ is a unit and $a_1,...,a_k$ are nilpotent.
 \end{exercise}

 \begin{solution}{Soroosh}
    First, notice that if $f$ is invertible in $R[x]$, $\alpha$ is a nilpotent
    element, and $h$ is any polynomial in $R[x]$, then $f+\alpha g$ is also
    invertible. In fact, in any commutative ring $S$ with have that if $u$ is
    a unit, and $\nu$ is a nilpotent element, then $u+\nu$ is a unit as well,
    since \[u^{-1}(1-\nu u^{-1}+\nu^2 u^{-2}-\dots \]
    is a finite sum and the inverse of $u+\nu$. Therefore, without loss of
    generality we can assume that none of the $a_i$'s are zero divisor,
    and we'll try to show that $f$ is just a constant function.

    Now, consider $f$ as an element in $R[[x]]$. Elements of $R[[x]]$ are
    invertible if and only if their constant term $\alpha_0$ is
    invertible. Therefore $a_0$ must be a unit in $R$. Assume without
    loss of generality that $a_0=1$.
    From now on assume that $f(x)=1+a_1x+\dots+a_nx^n$ with $a_n$ not
    nilpotent. We will show that if $n>0$ then there is a contradiction.

    Assume that $n > 0$.
    Let \[g(x)=1/f(x)=\sum_{k=0}^\infty b_kx^k \] be the formal
    inverse of $f$. By our assumption $g$ is a polynomial, and
    hence there is an $K$ such that $b_k=0$ for all $k>K$. Assume
    without loss of generality that $K$ is the smallest such value,
    that is $b_K \neq 0$.
    By expanding $g(x)f(x)=1$ we get that  $b_i$'s satisfy
    the recurrence relation
    \[ a_n b_i+a_{n-1}b_{i+1}+...+a_0 b_{i+n}=0, \]
    for $i+n>0$.
    This gives us
    \[\begin{array}{ccccc
    }
        a_n b_{K}+a_{n-1}b_{K+1}+\dots+a_0 b_{n+K}&=& a_n b_K &=&  0 \\
        a_n b_{K-1}+a_{n-1}b_{K}+\dots+a_0 b_{n-1+K}&=& a_n b_{K-1}+a_{n-1}b_K &=&  0 \\
        \vdots &=& \vdots &=& \vdots \\
        a_n b_{0}+a_{n-1}b_{1}+\dots+a_0 b_{n}&=& a_n b_0+a_{n-1}b_1+\dots+a_0b_n &=&  0.
    \end{array}\]
    We can write this in terms of a (lower triangular) matrix $A b = 0$ with
    $b=(b_K, b_{K-1}, \dots, b_0)^T$, and $A=(\alpha_{ij})$ with
    $\alpha_{ij}=a_{n-j+i}$. Now, doing row reduction, without using the
    division we get that $a_n^{K+1}=0$, which implies $a_n$ is a nilpotent element,
    contradicting our assumption. Therefore if we subtract away all the nilpotent
    elements of $f$ we are left with a unit. That is
    $a_0$ is a unit, and $a_1,\dots,a_n$ are all nilpotent elements.
 \end{solution}
  \begin{solution}{Shenghao Sun; shenghao@math}
    Joel helped me out with the following argument. First, if $f\in U(R[x])$ then $a_{0}\in
    U(R)$ because if $(a_{0}+a_{1}x+\cdots)(b_{0}+b_{1}x+\cdots)=1$ then $a_{0}b_{0}=1.$
    Assume $a_{0}=1$ and let $g=1-f.$

    Claim:if $A$ is an integral domain, then $g=0.$

    Assume $(1+a_{1}x+\cdots+a_{n}x^{n})(1+b_{1}x+\cdots+b_{m}x^{m})=1,$ then $a_{n}b_{m}=0,$
    but neither of them is $0,$ a contradiction.

    Now let $A$ be a general commutative ring. For any prime ideal $\mathfrak{p}\subset A,$
    since a ring homomorphism sends a unit to a unit, the reduction-mod-$\mathfrak{p}$ map
    defines $U(A[x])\to U(\frac{A}{\mathfrak{p}}[x]),$ which sends $1-g$ to $1-\bar{g}.$
    Since $\frac{A}{\mathfrak{p}}$ is an integral domain, $\bar{g}=0,$ which means
    $a_{i}\in\mathfrak{p},$ for any prime $\mathfrak{p}.$ Therefore
    $a_{i}\in\bigcap\mathfrak{p}=Nil(A),$ for all $i.$
 \end{solution}

 \begin{exercise}{I.4}
   (1) Show that the Prime Avoidance Theorem does not apply if the $I_1$, \dots, $I_n$ in
   $(1.3)$ are replaced by an infinite family of prime ideals.\\
   (2) Show that the theorem applies without assuming any of the ideals $I_i$ to be
   prime, but assuming that $A\< R$ and $R$ contains an infinite field.
 \end{exercise}
 \begin{solution}{anton@math}
   (1) Let $R=\ZZ$ and $A=\{2,3,4,\dots\}$. Then $A$ is contained in the union of all
   prime ideals, but is not contained in any one prime ideal. Gregory told me this
   example.

   (2) Let $k\subseteq R$ be an infinite field. We induct on $n$. If $n$ is 1 or 2, we
   already have the theorem. For larger $n$, we may assume $A\not\subseteq I_1\cup \cdots
   \hat I_j\cdots \cup I_n$ for each $j$, so there is some $a_j\in A \smallsetminus
   (I_1\cup \cdots \hat I_j\cdots \cup I_n)$. Since $A$ is in $\bigcup I_j$, we must have
   that $a_j\in I_j$. Now consider $a_1+ca_2\in A$, where $c$ is a non-zero element of
   $k$. For each such $c$, we must have $a_1+ca_2\in I_{j(c)}$ for some $j(c)$. Note that
   $j(c)$ is never 2, for otherwise we get $a_1\in I_2$. Since $k$ has an infinite number
   of non-zero elements, there must be distinct elements $c,c'\in k$ such that $j(c)=j(c')$. That is,
   $a_1+ca_2,a_1+c'a_2\in I_{j(c)}$. But then $(c-c')a_2\in I_{j(c)}$, and $(c-c')\in k$
   is invertible, so $a_2\in I_{j(c)}$, which is a contradiction.
 \end{solution}
 \begin{exercise}{I.5}
   Assume $R\neq 0$. If ${}_R M$ can be generated by $n$ elements, show that any set of
   $n+1$ (or more) elements in $M$ is linearly dependent. (\textbf{Hint.} Reduceto the
   case where $M\cong R^n$.)
 \end{exercise}
 \begin{solution}{anton@math}
   First let's prove the result for $R^n$. Clearly if each set of size $n+1$ is
   dependent, all larger sets are also dependent. If $\alpha_1,\dots, \alpha_{n+1}\in
   R^n$, then we can make them the columns of a matrix $A$. Since $A$ is an $n\times
   (n+1)$ matrix, its rank is at most $n$, so by the third theorem of McCoy, the columns
   of $A$ are linearly dependent.

   Now if $M$ is any module generated by $n$ elements, we have a surjection
   $R^n\twoheadrightarrow M$. If $\alpha_1,\dots, \alpha_{n+1}\in M$, then take some preimages
   $\hat \alpha_1,\dots,\hat\alpha_{n+1}$. By the preceding paragraph, these preimages
   are dependent. A homomorphism preserves linear dependence, so the original
   $\alpha_1,\dots, \alpha_{n+1}$ are dependent.
 \end{solution}
 \begin{exercise}{I.6} For $A\in\mathbb{M}_{n}\left(  R\right)  $ (where $R\neq0$),
  show that $A$ is a left 0-divisor in $\mathbb{M}_{n}\left(  R\right)  $ iff $A$ is a
  right 0-divisor, iff $\det\left(  A\right)  \in\mathcal{Z}\left( R\right)$.
  \end{exercise}

  \begin{solution}{Manuel Reyes; mreyes@math}
  To begin, we show that $A$ is a left 0-divisor iff there
  exists a nontrivial vector solution to $Av=0$. \ First, if such a solution $v\neq0$
  exists, we may let $B\in\mathbb{M}_{n}\left(  R\right) \smallsetminus\left\{  0\right\}
  $ be the matrix all of whose columns are $v$. \ Then clearly $AB=0$. \ Conversely,
  suppose $AB=0$ with $B\in \mathbb{M}_{n}\left(  R\right)  \smallsetminus\left\{
  0\right\}  $. \ Then $B$ must have a nonzero column, say $v$, and it follows that $Av=0$.
  \ Thus $A$ is a left 0-divisor iff there exists a nontrivial solution to $Av=0$, iff
  $\det\left(  A\right)  \in\mathcal{Z}\left(  R\right)  $. \

  By a symmetric argument we have that $A$ is a right 0-divisor iff there exists
  a nontrivial solution to $wA=0$, iff $\det\left(  A\right)  \in\mathcal{Z}%
  \left(  R\right)  $.
  \end{solution}

 \begin{exercise}{I.7}
   For $I_{1},\cdots,I_{n}\lhd R,$ show that $R=\bigoplus_{i}I_{i}$ iff there
   exist idempotents $e_{1},\cdots,e_{n}$ with sum $1$ such that $e_{i}e_{j}=0$ for all
   $i\neq j$ and $I_{i}=e_{i}R$ for all $i.$ In this case, show that each $I_{i}$ is a ring
   with identity $e_{i},$ and $R\cong\prod_{i=1}^{n}I_{i}$ in the category of rings. Show
   that any isomorphism of $R$ with a direct product of $n$ rings arises in this way.
 \end{exercise}
 \begin{solution}{Shenghao Sun; shenghao@math}
   ($\Rightarrow$) Suppose
   $R=\bigoplus_{i}I_{i}.$ Then $1$ can be written uniquely as $1=e_{1}+\cdots+e_{n},$ where
   $e_{i}\in I_{i}.$ For any $i\neq j, \ e_{i}e_{j}\in I_{i}\cap I_{j}=0.$ Certainly
   $e_{i}R\subset I_{i}.$ For any $x\in I_{i},\
   x=x(e_{1}+\cdots+e_{n})=xe_{1}+\cdots+xe_{n},$ and $xe_{j}\in I_{j}$ for all $j.$ Since
   $R$ is the direct sum of those ideals, have $xe_{j}=0$ for all $j\neq i$ and $x=xe_{i}\in
   e_{i}R.$ Hence $I_{i}=e_{i}R.$ In particular $e_{i}^{2}=e_{i},$ so they are idempotents.

   ($\Leftarrow$) Suppose we have idempotents $e_{i}$ with those conditions. Then take
   $I_{i}=e_{i}R,$ and we will show $R=\bigoplus_{i}I_{i}.$ $I_{1}+\cdots +I_{n}=R$ because
   $e_{1}+\cdots+e_{n}=1.$ Next we will show if there're two representations
   $\sum_{i}x_{i}=\sum_{i}y_{i}$ with $x_{i},y_{i}\in I_{i},$ then $x_{i}=y_{i}$ for all
   $i.$ Subtracting we reduce the problem to showing if $\sum_{i}x_{i} =0,$ then all
   $x_{i}=0.$ Since $I_{i}=e_{i}R,$ say $x_{i}=e_{i}r_{i}$ for $r_{i}\in R.$
   Multiplying by $e_{i_{0}},$ using $e_{i_{0}}e_{j}=\begin{cases}0,\ \text{if $i_{0}\neq j$,} \\
   e_{i_{0}},\ \text{if $j=i_{0}$}\end{cases}$ we obtain $e_{i_{0}}r_{i_{0}}=0,$ so all
   $x_{i}=0,$ and hence $R=\bigoplus_{i}I_{i}.$

   In this case, $I_{i}$ is closed under addition and multiplication, and $xe_{i}=x$ for all
   $x\in I_{i},$ and (obviously) multiplication is distributive over addition, so $I_{i}$ is
   a ring with identity $e_{i}.$ In the category of rings, the product $\prod_{i}R_{i}$ of
   rings $R_{i}$ is the cartesian product of those sets, with addition and multiplication
   defined coordinate-wise, and the identity is $(1,\cdots,1).$ In our case addition is
   operated coordinate-wise, and $1=e_{1}+\cdots+e_{n},$ as it should be. So it remains to
   check multiplication. Each element $x$ in $R$ is given by its coordinates $r_{i}\in R$ in
   the way $x=\sum_{i}e_{i}r_{i}.$ Using $e_{i}e_{j}=\begin{cases}0,\ \text{if $i\neq j$,}
   \\ e_{i},\ \text{if $i=j.$}\end{cases}$ we have
   $$(\sum_{i}e_{i}r_{i})(\sum_{i}e_{i}s_{i})=\sum_{i}e_{i}r_{i}s_{i},$$
   as expected.

   Now assume $R=\prod_{i}R_{i}$ for $n$ rings $R_{i}.$ Define $I_{i}= 0\times\cdots\times
   R_{i}\times\cdots\times0=\{(0,\cdots,r_{i},\cdots,0)|r_{i} \in R_{i}\}\subset R.$ Then
   each $I_{i}$ is an ideal in $R,$ generated by $e_{i}=(0,\cdots,1, \cdots,0),$ and these
   $e_{i}$'s are orthogonal and idempotent.
 \end{solution}
 \begin{exercise}{I.8}
     If $R=\bigoplus_{i=1}^n I_i$ as in Exer. 7, show that any $A\lhd R$ has
     the form $\bigoplus_{i} A_i$ where $A_i \lhd I_i$ for all $i$. For
     such an ideal $A$ show that $A\in \text{Spec}(R)$ iff for some $i$,
     $A_i\in\text{Spec}(I_i)$ and $A_j = I_j$ for all $j\neq i$. Prove the same statement with
     ``$\text{Spec}$`` replaced by ''$\text{Min}$`` and by ''$\text{Max}$``.
 \end{exercise}
 \begin{solution}{Lars Kindler; lars\_k@berkeley.edu}
     For every $i$ let $e_i$ denote the element $e_i=(0,\ldots,0,1,0,\ldots,0)$ ($1$
     at the $i$-th position) and write $A_i$ for the projection of $e_i A$ onto the $i$-th
     component. Then we have $A=\bigoplus_{i=1}^n A_i$ since $e_i A\subset A$ for every $i$ and
     every element $x\in A$ can be written as $x=\sum_i e_i x$.\\
     Furthermore the $A_i$ are ideals in $I_i$: Since for $x,y\in A_i$ the corresponding
     elements $(0,\ldots,0,x,0,\ldots,0)$ and $(0,\ldots,0,y,0,\ldots,0)$ are in $A$,
     so is their sum. Projection shows that $x+y$ is in $A_i$. The same method shows, that $I_i
     A_i=A_i$. So the $A_i$ are indeed ideals, als claimed.\\
     If $A_i=I_i$ for all but exactly one $i$, say $i=1$, and $A_1$ is prime in $I_1$ then $A$ is
     prime in $R$, since it clearly is an ideal and for $x,y\in R$ the product $xy$ is in $A$ if and only if $x_1 y_1 \in
     A_1$, but this implies $x_1\in A_1$ or $y_1\in A_1$, so $x\in A$ or $y\in A$, since all the
     other factors are $I_i$.\\
     Conversely let $A$ be an arbitrary prime ideal of $R$, this implies $A\neq R$, so for some $i$ we have $A_i\neq
     I_i$; without loss of generality we may assume $I_1\neq A_1$. Then $A_1$ is prime, for the
     multiplication is componentwise and $A$ is prime. Now assume there is another $i$ with
     $A_i\neq I_i$, say $i=2$. Let $x_1\in I_1\setminus A_1$ and $x_2\in I_2\setminus A_2$. Then the
     elements $(x_1,0,\ldots)$ and $(0,x_2,0,\ldots)$ are both not in $A$, but their product is
     $0$ and thus in $A$. This is a contradiction to the primality of $A$, so we indeed have
     $A_i=I_i$ for all but one index $i$.\\
     If $A$ is a minimal prime then so is $A_1$, since a strictly smaller prime of $I_1$
     contained in $A_1$ would produce a prime ideal (as shown above) strictly contained in $A$.
     If on the other hand $A_1$ is a minimal prime then $A:=A_1 \oplus I_2 \ldots \oplus I_n$ is
     a minimal prime of $R$, because a strictly smaller prime would have a prime strictly smaller than
     $A_1$ as its first direct summand. For the maximal case the argument is exactly the same.
 \end{solution}

 \begin{exercise}{I.9}
     A ring $S\neq 0$ is called \emph{indecomposable} (or \emph{connected}) if it is not
     isomorphic to a direct product of two nonzero rings. [By Ex. 7, this is equivalent to $S$
     having only trivial idempotents ($e=e^2\in S \Rightarrow e\in \{0,1\}$).] If
     $R=\bigoplus_{i=1}^n I_i = \bigoplus_{j=1}^m J_j$ where each $I_i, J_j$ is indecomposable
     as a ring, show that $n=m$ and that, after a permutation of the indices, $I_i=J_i$ for all
     $i$.
 \end{exercise}
 \begin{solution}{Lars Kindler; lars\_k@berkeley.edu}
     We use Ex. 7: Let $E:=\left\{ e_1,\ldots,e_n \right\}$ and $F:=\left\{
     f_1,\ldots,f_m \right\}$ be the sets of idempotents so that $I_i=e_i R$ and $J_j =
     f_j R$ for all $i,j$. For every $i$ we can write $e_i=(x_1,\ldots,
     x_m)\in\bigoplus_{j=1}^m J_j$, with $x_j\in J_j$ and idempotent. Since the $J_j$ are
     all indecomposable, the $x_j$ are all trivial idempotents in $J_j$, so $e_i$ is the
     sum of (different) elements of $F$, say $e_i=\sum_{k} f_k$. But since $f_k f_k' = 0$
     for $k\neq k'$ we have $f_k\in \left(\sum_k f_k\right) R=e_i R = I_i$ for all $k$,
     which implies, that there is actually only one summand in the sum $\sum_k f_k$,
     since otherwise $I_i$ would not be indecomposable. Since all the $e_i$ different the
     proposition follows.
%    If e.g. $E$
%    is properly contained in $F$, then for every $i$ there is a $j$, such that $I_i=e_i R =
%    f_j R = J_j$, which implies $E=F$ (since otherwise some of the $J_j$ would be trivial
%    which is wrong by hypothesis), and thus $n=m$ and $I_i=J_i$ for all $i$ after some permutation of the
%    indices.\\
%    If on the other hand neither of theses sets is properly contained in the other, then there
%    is a nontrivial idempotent $e_i \in E\setminus F$, which can be written as
%    $e_i=\sum_{j=1}^m x_j$, with $x_j\in J_j$ idempotent. Because of the indecomposability of
%    the $J_j$ the $x_j$ are trivial idempotents in $J_j$, so $e_i$ is a sum of (different)
%    elements of $F$.
 \end{solution}

 \begin{exercise}{I.10}
   Show that not all rings are finite direct products of
   indecomposable rings, but noetherian rings are.
 \end{exercise}

 \begin{solution}{David Brown, brownda@math}
 Let $k$ be any field, and let $R = \prod _{i \in \mathbb{N}} k$.
 Then
 \[
 R \cong k \times R.
 \]
 Suppose that $R$ were the finite direct product of indecomposable
 rings
 \[
 R \cong \bigoplus_{i = 1}^{n} I_{i}.
 \]
 then we also would have
 \[
 R \cong k \times \bigoplus_{i = 1}^{n} I_{i},
 \]
 but this contradicts uniqueness of decomposition (Ex. I.9).\\

 Now suppose that $R$ is not the finite direct product of
 indecomposable rings. Then $R$ is necessairly decomposable, so
 by Ex. I.7 we have
 \[
 R \cong R_{1} \times R_{1}', 1 = e_{1} + e_{1}'
 \]
 for some non-zero orthogonal idempotents
 $e_{1}$ and $e_{1}'$. We can assume that $R_{1}$ decomposes
 into $R_{2} \times R_{2}'$, with $e_{1} = e_{2} + e_{2}'$
 and $R_{2}$ decomposable, and inductively set
 \[
 R_{i} \cong R_{i+1} \times R_{i+1}', e_{i} = e_{i+1} + e_{i+1}'.
 \]
 Now define the increasing chain of ideals
 \[
 I_{n} = \bigg<e_{i}' : i \leq n \bigg>.
 \]
 For all $n$, $I_{n+1} \neq I_{n}$. Indeed, we would then have
 \[
 e_{i+1}' = a_{1}e_{1}' + \cdots + a_{n}e_{n}',
 \]
 but multiplication by $e_{n+1}$ gives $e_{i+1}' = 0$, a contradiction.
 We conclude that $R$ is not noetherian.
 \end{solution}

 \begin{exercise}{I.11}
   Using the last exercise, show that a von Neumann regular ring is noetherian if and
   only if it is a finite direct product of fields.
 \end{exercise}
 \begin{solution}{anton@math}
   A finite direct product of fields is clearly noetherian and von Neumann regular.

   By exercise I.10, a noetherian ring is a finite direct product of indecomposable
   rings. If the ring is von Neumann regular, then each direct factor must be von Neumann
   regular (since the equation $a=axa$ holds in each factor). So we must show that an
   indecomposable von Neumann regular ring is a field. To see this, recall that in a von
   Neumann regular ring, every element $a$ has a pseudo-inverse $x$, satisfying $a=axa$.
   Then $xaxa=xa$, so $xa$ is idempotent. By indecomposability, $xa=0$ or 1. If $xa=0$,
   then $a=axa=0$; if $xa=1$, then $a$ is invertible. This means that all non-zero
   elements are invertible, so we are in a field.
 \end{solution}

 \begin{exercise}{I.12}
  \begin{enumerate}
   \item For an \emph{arbitrary} direct product $R=\prod_i R_i$,
       show that $\nil(R)\subset \prod_i
       \nil(R_i)$ and $\rad(R)=\prod_i\rad(R_i)$.
   \item If each $R_i$ is reduced (resp. J-semisimple, von Neumann regular), prove the
       same for $R$.
   \item Show that the inclusion in the relation for the nilradicals in (1) cannot be
       replaced by an equality relation in general.
  \end{enumerate}
 \end{exercise}
 \begin{solution}{Lars Kindler, lars\_k@berkeley.edu}
  \begin{enumerate}
   \item Let $R=\prod_i R_i$ and $(x_i)_i$ an element of $\nil(R)$. Then there is a
       $n\in \mathbb{N}$, such that $0=(x_i)_i^n=(x_i^n)$ so $(x_i)_i\in\prod_i
       \nil(R_i)$ and $\nil(R)\subset \prod_i \nil(R_i)$.\\
       We now observe that an element $(x_i)_i$ in $R=\prod_i R_i$ is
       invertible in $R$ if and only if $x_i$ is invertible in every $R_i$.
       $\rad(R)$ can be characterized as the set of elements $x=(x_i)_i\in R$ such that
       $1+Rx\subset U(R)$. But $1+Rx=\prod_i \left(1+R_i x_i\right)$, so
       $1+Rx\subset U(R)=\prod_i U(R_i)$ if and only if $1+R_i x_i \subset U(R_i)$
       for every $i$
       and thus  $x=(x_i)_i$ is in
       $\rad(R)$ if and only if $x_i\in \rad(R_i)$ for every $i$.
   \item $R_i$ is reduced for every $i$ if and only if $\nil(R_i)=(0)$, so by part 1 we
       have $(0)=\prod_i \nil(R_i)\supset \nil(R)$. Similarily, if $R_i$ is
       J-semisimple for every $i$ we have $\rad(R_i)=\left( 0 \right)$ for every
       $i$, so $(0)=\prod_i\rad(R_i)=\rad(R)$, so $R$ is J-semisimple.\\
       $R$ is von Neumann regular if for every $x\in R$ there is a $y\in R$ such
       that $x=xyx$. Let $x=(x_i)_i\in R=\prod_i R_i$. If each $R_i$ is von
       Neumann regular then for every $i$ we have a $y_i\in R_i$ with $x_i=x_i y_i
       x_i$, so in $R$ we have $(x_i)_i=(x_i)_i(y_i)_i(x_i)_i$ and thus $R$ is von
       Neumann regular.
   \item Let $R$ be the product $\prod_{i\geq 2} R_i = \prod_{i\geq 2}k[x_i]/\left( x_i^i \right)$ for some
       ring $k$. Then the element $(\overline{x_i})_i\in R$ is not nilpotent, since for every
       $n\in \mathbb{N}$ we have $\overline{x_{n+1}}^n\neq 0$, but each
       $\overline{x_i}$ is nilpotent in $R_i=k[x_i]/\left(x_i^i\right)$, so
       $\nil(R)\subsetneq \prod_i \nil(R_i)$.
  \end{enumerate}
 \end{solution}

 \begin{exercise}{I.13}
  Show that the direct product of any finite family of semilocal rings is semilocal.  Do
  the same with ``semilocal'' replaced by ``noetherian'', ``artinian'', ``0-dimensional'',
  or ``PIR''.  For each of these cases, does the same conclusion hold for \emph{arbitrary}
  direct products?
 \end{exercise}

 \begin{solution}{ecarter@math}
   By induction, it suffices to show that each claim holds for direct products of two
   rings.  First, we will show that any ideal $I$ of $R\times S$ has the form
   $\a\times\mathfrak{b}$ for some $\a\< R$ and $\mathfrak{b}\< S$, both not
   necessarily proper.  Given $I\< R\times S$, let
   \[
           \a=\{ a\in R\mid (a,s)\in I\;\text{for some}\; s\in S\}
   \]
   and
   \[
           \mathfrak{b} = \{b\in S\mid (r,b)\in I\;\text{for some}\; r\in R\}.
   \]
   Then if $a,a'\in\a$, there exist $s,s'\in S$ such that $(a,s),(a',s')\in I$,
   so $(a+a',s+s')\in I$.  Also, if $r\in R$, $(ra,s)=(r,1)\cdot (a,s)\in I$, so
   $\a\< R$.  Similarly, $\mathfrak{b}\< S$.  Now, let $a\in\a$ and
   $b\in\mathfrak{b}$.  Then there exist $r\in R$ and $s\in S$ such that
   $(a,s),(r,b)\in I$.  Then
   \[
           (a,b) = (a,s)\cdot (1,0) + (r,b)\cdot (0,1) \in I.
   \]
   Therefore $I=\a\times\mathfrak{b}$.

   Let $R$ and $S$ be semilocal.  Every maximal ideal of $R\times S$ has the form
   $\m\times S$ for some maximal $\m\< R$ or $R\times\mathfrak{n}$ for some maximal
   $\mathfrak{n}\< S$.  There are finitely many of each type, so $R\times S$ is
   semilocal.

   Let $R$ and $S$ be noetherian, and let
   \[
           \a_0\times\mathfrak{b}_0
                   \subseteq \a_1\times\mathfrak{b}_1
                   \subseteq\cdots
   \]
   be an increasing chain of ideals in $R\times S$.  The chains
   $\a_0\subseteq\a_1\subseteq\cdots$ and
   $\mathfrak{b}_0\subseteq\mathfrak{b}_1\subseteq\cdots$ both stabilize,
   so the chain of ideals in $R\times S$ stabilizes as well.  Therefore $R\times S$ is
   noetherian.

   Similarly, if $R$ and $S$ are both artinian, then $R\times S$ is also artinian.

   Let $R$ and $S$ be 0-dimensional rings.  Let $\a\times\mathfrak{b}\< R\times S$ be
   a prime ideal.  Then $(1,0)\cdot (0,1)=(0,0)\in\a\times\mathfrak{b}$, so either
   $1\in\a$ or $1\in\mathfrak{b}$.  Suppose $1\in\mathfrak{b}$.  If $(r,1)$ and
   $(r',1)$ are such that $(r,1)\cdot (r',1)=(rr',1)\in\a\times\mathfrak{b}$, then
   either $(r,1)$ or $(r',1)$ is in $\a\times\mathfrak{b}$.  Therefore $\a$ is a
   prime ideal, and hence maximal since $R$ is 0-dimensional.  Similarly, if
   $1\in\a$, then $\mathfrak{b}$ is a prime ideal, and hence maximal.  In both cases,
   $\a\times\mathfrak{b}$ is maximal, so $R\times S$ is 0-dimensional.

   Let $R$ and $S$ be PIRs.  Then any ideal $(a)\times (b)$ is generated by $(a,b)$,
   so $R\times S$ is also a PIR.

   An infinite product of semilocal rings can fail to be semilocal, as demonstrated
   by $R=k\times k\times\cdots$, which has a maximal ideal for each copy of $k$
   consisting of all elements of $R$ with the corresponding value being equal to 0.
   This example also shows that an infinite direct product of noetherian rings can
   fail to be noetherian, since we have the infinite ascending chain of ideals
   \[
           \a_i = \{ (b_1,b_2,\dots) \mid b_j=0\;\forall j\geq i\}.
   \]
   This ring also has an infinite descending chain of ideals
   \[
           \mathfrak{b}_i = \{ (c_1,c_2,\dots) \mid c_j=0\;\forall j\leq i\},
   \]
   so an infinite direct product of artinian rings can fail to be artinian. The ring
   $R$ contains an ideal $k\oplus k\oplus\cdots$ which is not principal, since any
   element of this ideal has only zeroes past a certain point, while there are other
   elements of the ideal which do not have only zeroes past that point. Therefore an
   infinite direct product of PIRs can fail to be a PIR.

   Any 0-dimensional ring is rad-nil.  Consider the example $R=\prod_{i\geq 2}
   R_i=\prod_{i\geq 2} k[x_i]/(x_i^i)$ from exercise 12. Each $R_i$ has a unique
   prime $(x_i)$, so $\rad(R_i)=\nil(R_i)$, and $\rad(R)=\prod_{i\geq 2}\rad(R_i)$ by
   exercise 12.  However,
   \[
       \nil(R)
               \neq \prod_{i\geq 2}\nil(R_i)
               = \prod_{i\geq 2}\rad(R_i)
               = \rad(R),
   \]
   so this is an example of an infinite direct product of 0-dimensional rings failing
   to be 0-dimensional.
 \end{solution}

 \begin{exercise}{I.14}
     Show that $0$ is the only idempotent contained in $\text{rad}(R)$.
     Deduce from this that any local ring is indecomposable.
 \end{exercise}
 \begin{solution}{Lars Kindler; lars\_k@berkeley.edu}
     Let $x\in \text{rad}(R)$ be an idempotent. Then we have $x(1-x)=0$, but $1-x$ is a unit, so
     $x=0$. If $R$ is a local ring with maximal ideal $\mathfrak{m}$ then the only idempotent
     which is not a unit is $0$, since every non-invertible element is contained in
     $\mathfrak{m}=\text{rad}(R)$. Exercise 7 now implies that $R$ is indecomposable.
 \end{solution}

  \begin{exercise}{I.15} Let $R$ be a power series ring $A[[x_{1},\cdots,x_{n}]],$
    where $A$ is a ring with Jacobson radical $J.$ Show that
    $rad(R)=J+\sum_{i=1}^{n}Rx_{i}.$ Deduce from this that, if $A$ is local, so is $R.$
  \end{exercise}
  \begin{solution}{Shenghao Sun; shenghao@math} $1+J\subset U(A)\subset U(R),$ so $J\subset
    rad(R).$ For any $r\in R, h=1+rx_{i}+(rx_{i})^{2}+\cdots$ is Cauchy because
    $(rx_{i})^{n}\subset(x_{1} ,\cdots,x_{n})^{n}R,$ and $R$ is complete, so $h$ is
    convergent in $R$ and $h=\frac{1}{1-rx_{i}}.$ This shows $1+Rx_{i}\subset U(R),$ so
    $Rx_{i}\subset rad(R).$ Hence $J+\sum_{i}Rx_{i}\subset rad(R).$ Since $a\in A,\ af=1$
    for some $f\in R\Rightarrow f\in A,$ we have $U(R)\cap A=U(A).$ For any $f\in R$ we
    have $f-f(0,\cdots,0)\in(x_{1},\cdots,x_{n}).$ Now let $f\in rad(R).$ For any $a\in A,\
    1+af\in U(R).$ Now $1+af= 1+af(\underline{0})+a(f-f(\underline{0}))$ is a unit, and
    $a(f-f(\underline{0}))\in (x_{1},\cdots,x_{n})\subset rad(R),$ so if we subtract we get
    a unit $1+af(\underline{0})$ in $R,$ which is a scalar, so it's a unit in $A.$ Then
    $1+f(\underline{0})A\subset U(A)\Rightarrow f(\underline{0})\in rad(A)=J,$ so
    $f=f(\underline{0})+(f-f(\underline{0}))\in J+(x_{1},\cdots,x_{n}).$ This completes the
    proof that $rad(R)=J+(x_{1},\cdots,x_{n}).$  A ring $A$ is local iff $rad(A)$ is
    maximal. Note that $\frac{R}{\sum_{i}Rx_{i}}= A.$ So if $A$ is local, then
    $\frac{R}{rad(R)}=\frac{R}{rad(A)+\sum_{i}Rx_{i}}= \frac{A}{rad(A)}$ is a field, so
    $rad(R)$ is maximal in $R$ and therefore $R$ is also local.
  \end{solution}

 \begin{exercise}{I.16}
  If $f:R\to S$ is a surjective ring homomorphism, show that $f(\rad R)\subseteq \rad S$.  Show
  that equality holds if $\ker(f)\subseteq\rad(R)$, or if $R$ is a semilocal ring.  Give an example where
  the equality fails.
 \end{exercise}

 \begin{solution}{ecarter@math}
   Since $f$ is surjective, $S\cong R/\a$, where $\a=\ker(f)$.  Then the maximal ideals of
   $R/\a$ are the maximal ideals of $R$ which contain $\a$.
   In this notation, $f(\rad R)=\rad R/(\rad R\cap\a)$.
   Since $\rad R$ is a subset of the intersection of all maximal ideals of $R$ which contain $\a$,
   $f(\rad R)\subseteq\rad S$.

   If $\a=\ker(f)\subseteq\rad(R)$, the the intersection of all maximal ideal of $R$ is the same as the
   intersection of all maximal ideals of $R$ containing $\a$, so
   $f(\rad R)=\rad R/(\rad R\cap\a)=\rad R/\a=\rad S$.

   If $R$ is semilocal, let $\m_1,\dots,\m_s$ be the maximal ideals of $R$.  If all of these maximal ideal
   contain $\a$, then $f(\rad R)=\rad S$ be the above argument.  By reordering if necessary, suppose
   $\m_1,\dots,\m_r$ do not contain $\a$, while $\m_{r+1},\dots,\m_s$ do.  For each $1\leq i\leq r$,
   $\m_i+\a=R$, so that $\m_i$ generated $R/\a$.  Therefore,
   \begin{align*}
    \rad R/(\rad R\cap\a)
        &= (\m_1\cdots\m_s)/(\m_1\cdots\m_s\cap\a) \\
        &= (\m_{r+1}\cdots\m_s)/(\m_{r+1}\cdots\m_s\cap\a)
   \end{align*}
   which is equal to $\rad S$, since $\m_{r+1},\dots,\m_s$ are exactly the ideals of $R$ containing $\a$.

   Let $R=k[x]$ and $S=k[x]/(x^2)$, with $f:k[x]\to k[x]/(x^2)$ being the obvious quotient map.  Then
   $\rad R=0$ so that $f(\rad R)=0$, but $\rad S=(x)/(x^2)$.
 \end{solution}

  \begin{exercise}{I.17} For any $I\lhd R$ lying in $rad(R)$ such that $R/I$
    is J-semisimple, show that $I=rad(R).$
  \end{exercise}
  \begin{solution}{Shenghao Sun; shenghao@math} By Ex.16 we know that
    $rad(R/I)=rad(R)/I$ since $I=\ker(R\twoheadrightarrow R/I) \subset rad(R).$ Then $R/I$
    being Jacobson-semisimple $\Rightarrow\ rad(R/I)=0 \Rightarrow I=rad(R).$
  \end{solution}

 \begin{exercise}{I.18}
   For any ring $R$, show that $\operatorname*{rad}(  R) \subseteq\{  r\in R:r+U( R)
   \subseteq U(  R) \}  $. Is this an equality?
 \end{exercise}
 \begin{solution}{Manuel Reyes, mreyes@math}
   Let $r\in\rad(  R)  $. Then for any $u\in U(
   R)  $, we have $r+u=u(  1+u^{-1}r)  $. Now $1+u^{-1}%
   r\in1+\rad(  R)  \subseteq U(  R)  $. This means that $r+u$ is a product of two units and
   thus is itself a unit. This gives the desired inclusion.

   The inclusion is not an equality. As a counterexample, let $A$ be a reduced ring with
   nonzero radical, and let $a\in\rad A\smallsetminus \{  0\}  $. (More specifically,
   according to Exercise I.15 we may let $A=k[  [  x] ] $, where $k$ is a field, and let
   $a=x$.) Let $R=A[  y]  $ be the polynomial ring in one variable over $A$.  Because $A$ is
   reduced, Exercise I.3 shows that $U( R) =U( A)  $. The fact that
   $a\in\operatorname*{rad}A$ implies that
   \[
   a+U(  R)  =a+U(  A)  \subseteq U(  A)
   =U(  R)  \text{,}%
   \]
   so $a\in\{  r\in R:r+U(  R)  \subseteq U(  R) \}  $. However $1+ay\notin U( A) =U( R) $,
   and $1+Ra\nsubseteq U( R)  $ implies that $a\notin \rad( R)  $. Hence the inclusion is
   strict in this case.
 \end{solution}

 \begin{exercise}{I.19}
   Show that any ring $R$ with $\left\vert U\left(  R\right)
   \right\vert <\infty$ is rad-nil. \ Deduce from this that any reduced ring $R$ with
   $\left\vert U\left(  R\right)  \right\vert <\infty$ is Jacobson semisimple.
 \end{exercise}

 \begin{solution}{Manuel Reyes; mreyes@math}
   Let $r\in\operatorname*{rad}\left(  R\right)$. Then for all positive integers $m$,
   $1+r^{m}\in1+\operatorname*{rad}\left( R\right) \subseteq U\left(  R\right)  $. \
   Because $U\left(  R\right)  $ is finite, there exist $n,k>0$ such that
   $1+r^{n}=1+r^{n+k}$. \ Taking the difference of each side of the equation gives
   $r^{n}\left(  1-r^{k}\right) =r^{n}-r^{n+k}=0$. \ But
   $1-r^{k}\in1+\operatorname*{rad}\left(  R\right) \subseteq U\left(  R\right)  $, so
   the previous equation implies that $r^{n}=0$. \ Thus $R$ is rad-nil.

   Now for any reduced ring $R$ with $\left\vert U\left(  R\right)  \right\vert <\infty$, we
   have $\operatorname*{rad}\left(  R\right)  \subseteq \operatorname*{Nil}\left(  R\right)
   =0$. \ So $R$ is $J$-semisimple.
 \end{solution}

 \begin{exercise}{I.20}
   Suppose $\mathcal{Z}\left(  R\right)  \subseteq
   \operatorname*{rad}\left(  R\right)  $, and $aR+bR=dR$, where $a,b,d\in R$ and
   $d\neq0$. \ If $a=dx$ and $b=dy$, show that $xR+yR=R$.
 \end{exercise}

 \begin{solution}{Manuel Reyes; mreyes@math.berkeley.edu}
   Because $aR+bR=dR$, there exist $m,n\in R$ with $am+bn=d$. \ Then%
   \begin{align*}
   d  & =am+bn\\
   & =dxm+dyn\text{.}%
   \end{align*}
   This implies that $d\left(  xm+yn-1\right)  =0$. \ Because $d\not =0$ we have
   $\alpha:=xm+yn-1\in\mathcal{Z}\left(  R\right)  \subseteq\operatorname*{rad}%
   \left(  R\right)  $. \ But then
   \[
   xm+yn=1+\alpha\in1+\operatorname*{rad}\left(  R\right)  \subseteq U\left(
   R\right)  \text{.}%
   \]
   It follows that $xR+yR=R$. \
 \end{solution}

 \begin{exercise}{I.21}
   Show that a reduced ring $R \neq 0$ that is not a domain has at
   least two minimal prime ideals.
 \end{exercise}

 \begin{solution}{David Brown, brownda@math}
   Suppose that $R$ is a reduced domain with only one minimal prime
   $\q$. Then
   \[
   (0) = \nil R =  \bigcap_{\p \in \spec R} \p = \bigcap_{\p
   \text{ minimal}} \p
   = \q.
   \]
   Thus $(0)$ is prime, i.e. $R$ is an integral domain. We conclude that
   $R$ has at least two minimal primes.
 \end{solution}

 \begin{exercise}{I.22}
   For $I,J\< R$, show that $\sqrt I\cap \sqrt J = \sqrt{I\cap J}$. How much would you
   bet on this equaling $\sqrt{IJ}$, or $\sqrt I \sqrt J$?
 \end{exercise}
 \begin{solution}{anton@math}
   The radical of an ideal is the intersection of primes containing it. Given a prime
   $\p$, I claim that the following are equivalent:
   \begin{enumerate}
     \item[1.] $IJ\subseteq \p$
     \item[2.] $I\subseteq \p$ or $J\subseteq \p$
     \item[3.] $I\cap J \subseteq \p$
   \end{enumerate}
   $1\Rightarrow 2$ is the definition of prime ideal. $2\Rightarrow 3$ is obvious.
   $3\Rightarrow 1$ follows from the fact that $IJ\subseteq I\cap J$. It follows that
   $\sqrt I \cap \sqrt J = \sqrt{I\cap J} = \sqrt{IJ}$.

   However, $\sqrt I \cap \sqrt J\neq \sqrt I \sqrt J$ in general. For example, let $I=J$
   be a radical ideal not equal to its square (like $(2)\subseteq \ZZ$). Then $\sqrt I
   \cap \sqrt J=I$, but $\sqrt I \sqrt J = I^2$.
 \end{solution}

 \begin{exercise}{I.23}
  Show that, with respect to the inclusion relation, the set $\mathcal{F}$ of
  radical ideals in a ring forms a complete lattice with $0$ and $1$. (This
  means that, in the poset $\mathcal{F}$, any subset has a "sup" and an "inf",
  and $\mathcal{F}$ has a largest element and a smallest element.) Show that
  every element in $\mathcal{F}$ is the "inf" of a set of prime ideals. Is $%
  \mathcal{F}$ closed with respect to ideal sums and ideal products?
 \end{exercise}

 \begin{solution}{Manuel Reyes, mreyes@math}
  Let $R$ be the ring in question. Let $\left\{ \mathfrak{a}_{i}:i\in
  I\right\} \subseteq \mathcal{F}$ be a collection of radical ideals, and let $%
  \mathfrak{b}\in \mathcal{F}$. Suppose that $\mathfrak{b}\subseteq
  \mathfrak{a}_{i}$ for all $i$. Then $\mathfrak{b}\subseteq \bigcap
  \mathfrak{a}_{i}$, and $\bigcap \mathfrak{a}_{i}\subseteq \mathfrak{a}_{j}$
  for all $j$. Because each $\mathfrak{a}_{i}$ is an intersection of prime
  ideals, so is $\bigcap \mathfrak{a}_{i}$. Hence $\bigcap \mathfrak{a}%
  _{i}\in \mathcal{F}$ and we may set the inf of the family $\left\{ \mathfrak{%
  a}_{i}\right\} $ to be
  \[
  \bigwedge_{i\in I}\mathfrak{a}_{i}:=\bigcap_{i\in I}\mathfrak{a}_{i}\text{.}
  \]
  Now suppose that $\mathfrak{a}_{i}\subseteq \mathfrak{b}$ for all $i$.
  Then $\sum \mathfrak{a}_{i}\subseteq \mathfrak{b}$, which implies that $%
  \sqrt{\sum \mathfrak{a}_{i}}\subseteq \sqrt{\mathfrak{b}}=\mathfrak{b}$.
  Clearly $a_{j}\subseteq \sqrt{\sum a_{i}}$ for all $j$. Because $\sqrt{%
  \sum \mathfrak{a}_{i}}\in \mathcal{F}$, we may set the sup of the family to
  be
  \[
  \bigvee_{i\in I}a_{i}:=\sqrt{\sum_{i\in I}\mathfrak{a}_{i}}\text{.}
  \]%
  Finally, $\mathcal{F}$ has $\nil\left( R\right) $ as a smallest
  element and has $R$ as a largest element. So $\mathcal{F}$ is a complete
  lattice. Now for any $\mathfrak{a}\in \mathcal{F}$, we have $\mathcal{V}%
  \left( \mathfrak{a}\right) =\left\{ \mathfrak{p}\in \spec R:%
  \mathfrak{p}\supseteq \mathfrak{a}\right\} \subseteq \mathcal{F}$. So
  \[
  \mathfrak{a}=\sqrt{\mathfrak{a}}=\bigcap_{\mathfrak{p}\in V\left( \mathfrak{a%
  }\right) }\mathfrak{p}=\bigwedge_{p\in V\left( a\right) }\mathfrak{p}
  \]%
  is the inf of a set of of prime ideals in $\mathcal{F}$ (note that, by
  convention, the empty intersection is equal to $R$).

  In general, the set $\mathcal{F}$ is not closed with respect to ideal sums
  and ideal products. Taking $R=k\left[ x,y\right] $ for some field $k$, the
  ideals $\left( y\right) $ and $\left( x^{2}-y\right) $ are both prime, hence
  radical. However, the sum $\left( y\right) +\left( x^{2}-y\right) =\left(
  x^{2},y\right) $ and the product $\left( x\right) ^{2}=\left( x^{2}\right) $
  are not radical ideals because they both contain $x^{2}$ while neither of
  them contains $x$.
 \end{solution}

 \begin{exercise}{I.24}
   For a ring $R$, show that the following are equivalent: (1) $R$ is local; (2) $R/\rad
   (R)$ is a field; (3) $R\smallsetminus U(R)$ is an ideal of $R$; (4) $R\smallsetminus
   U(R)$ is an additive group; (5) $R\neq 0$ and for any $n$, $a_1+\cdots + a_n\in U(R)$
   implies that some $a_i\in U(R)$; (6) $R\neq 0$ and $a+b\in U(R)\Rightarrow a\in U(R)$
   or $b\in U(R)$.
 \end{exercise}
 \begin{solution}{anton@math}\\
   ($1\Leftrightarrow 2$) $R$ is local $\Leftrightarrow$ $\rad R$ is a maximal ideal
   $\Leftrightarrow$ $R/\rad R$ is a field.\\
   ($1\Leftrightarrow 3$) $R\smallsetminus U(R)$ is the union of all maximal ideals in
   $R$, which is an ideal if and only if there is a single maximal ideal.\\
   ($1\Rightarrow 4$) $R\smallsetminus U(R)$ is the maximal ideal, so it is an
   additive group.\\
   ($4\Rightarrow 5$) If $R\smallsetminus U(R)$ is an additive group, a sum of non-units
   will always be a non-unit.\\
   ($5\Rightarrow 6$) Take $n=2$.\\
   ($6\Rightarrow 3$) By (6), the sum of non-units is a non-unit, and multiplying a
   non-unit by something cannot ever make it a unit, so $R\smallsetminus U(R)$ is an
   ideal.
 \end{solution}

 \begin{exercise}{I.25}
   For a ring $R$, show that the following are equivalent:(0) $|\spec (R)|=1$; (1) $R$ is
   local and rad-nil; (2) $R/\nil (R)$ is a field; (3) $R\smallsetminus U(R) = \nil (R)$.
   (Recall from (A), (B) in (5.3) that such rings $R$ need not be noetherian.)
 \end{exercise}
 \begin{solution}{anton@math}\\
   ($0\Rightarrow 1$) If $R$ has a unique prime ideal (which is then the only maximal
   ideal), then the intersection of primes ($\nil R$) is the same as the intersection of
   maximal ideals ($\rad R$).\\
   ($1\Rightarrow 2$) If $R$ is local, with maximal ideal $\m$, and the intersection of
   prime ideals ($\nil R$) is $\rad R$ ($=\m$), then $R/\nil R = R/\m$ is a field.\\
   ($2\Rightarrow 3$) If $R/\nil R$ is a field, $\nil R$ is a maximal ideal, and
   therefore the only prime ideal, so it is the complement of $U(R)$.\\
   ($3\Rightarrow 0$) $R\smallsetminus U(R)=\nil R$ is the union of maximal ideals, which
   is an ideal only if it is the unique maximal ideal. Since $\nil R$ is the intersection
   of primes, it is also the unique prime ideal of $R$.
 \end{solution}

 \begin{exercise}{I.26}
   For any integral domain $k$, show that $k[x_1, \dots, x_n]/(x_1\cdot \dots \cdot x_n)$ is
   a reduced ring.
 \end{exercise}
 \begin{solution}{derman@math}
   \paragraph{Remark:}  I think that $k$ reduced is sufficient.
   It suffices to show that $I=(x_1\cdot \dots \cdot x_n)$ is radical in $k[x_1, \dots,
   x_n]$.  Note that since $I$ is a monomial ideal, a polynomial $f\in I$ if and only each
   monomial which appears with a nonzero coefficient in $f$ is divisible by $x_1\cdot \dots
   \cdot x_n$. To show that $I=\sqrt{I}$ we take an arbitrary$ f\in \sqrt{I}$ and induct on
   the number of terms of $f$.  If $f$ is a single term $f=am, a\in k, m$ a monomial then
   $a^nm^n\in I$.  But since $a^n\ne 0$ as $k$ is reduced, we have $x_1\cdot \dots \cdot
   x_n$ divides $m^n$ and thus it also divides $m$.  So $am\in I$.

   For the induction step, place any term order on the monomials of $k[x_1, \dots, x_n]$.
   Let $a_\nu m_\nu$ be the leading term of $f$, where $a_\nu\in k$ and $m_\nu$ a monomial.
   Then the leading term of $f^n$ is $(a_\nu)^n(m_\nu)^n\ne 0$ because $k$ is reduced.  If
   $f^n\in I$ thent $(m_\nu)^n$ is divisible by $x_1\cdot \dots \cdot x_n$ and thus $m_\nu$
   is also divisible by $x_1\cdot \dots \cdot x_n$.  Then we have that $(f-a_\nu
   m_\nu)^n=\sum_{i=0}^n (-1)^if^{n-i}(a_\nu m_\nu)^i=f^n-\m_\nu\cdot (\text{other
   stuff})\in I$ as $f^n\in I$ and $m_\nu \in I$.  But $(f-a_\nu m_\nu)$ has one less term
   than $f$, and thus by induction hypothesis, $(f-a_\nu m_\nu)^n\in I \Rightarrow (f-a_\nu
   m_\nu)\in I \Rightarrow f\in I$.
 \end{solution}
 \begin{solution}{dustin@math}
   We will prove that the ideal $I = (x_1\cdots x_n)$ is reduced in the ring $R=k[x_1,
   \ldots x_n]$ by showing that it is the intersection of prime ideals.

   First I claim that the ideals $(x_i)$ are prime for each $i$. The quotient $R/(x_i)$ is
   isomorphic to $k[x_1,\ldots,\hat{x_i},\ldots, x_n]$ by the substitution $x_i \mapsto 0$.
   Thus $R/(x_i)$ is a domain, so $(x_i)$ is prime.

   Second I claim that $I$ is the intersection of the $(x_i)$ for $1\leq i \leq n$. If $r
   \in I$, then clearly $r \in (x_i)$ for all $i$. Conversely, suppose $r$ is in the
   intersection of all the $(x_i)$. We can write $r$ as the sum of monomials, and for each
   $i$, each monomial has a factor of $x_i$. But this means that each monomial must have a
   factor of $x_1 \cdots x_n$, so $r$ is divisible by $x_1\cdots x_n$, so $r \in I$. Thus,
   $I$ is the intersection of prime ideals, so it is reduced.
 \end{solution}

 \begin{exercise}{I.27}
   Show that, for any noetherian ring $R, Nil(R)$ is nilpotent.
 \end{exercise}
 \begin{solution}{derman@math}
    First note that $Nil(R)$ is an ideal of $R$.  (If $a^{n}=0$ and $b^m=0$ then
    $(a+b)^{n+m}=0$.  Also $(ra)^n=r^na^n=0$ for all $r\in R, a\in Nil(R)$.)

    Since $R$ is Noetherian, we can choose $f_1, \dots f_s$ which generate $Nil(R)$, and
    where $f_i^{e_i}=0$.  Then for any $g=\sum a_if_i \in Nil(R)$ it follows that $g^{\sum
    e_i}=0$.  This is because of the pigeonhole principle.  Namely, since $g^{\sum e_i}$ can
    be written as a sum consisting of terms that look like $(a_1f_1)^{b_1}\cdot \dots
    (a_nf_n)^{b_n}$ where $\sum b_i=\sum e_i$ it follows that at least one of the $b_i\geq
    e_i$ and thus this whole term equals zero. Thus $Nil(R)$ is nilpotent.
 \end{solution}
 \begin{exercise}{I.28}
 Show that an artinian ring $R$ is local iff $\mathcal Z(R) = \nil(R)$.
 \end{exercise}
 \begin{solution}{dustin@math}
 First, I claim that every element of any artinian ring $R$ is either a zero divisor or a
 unit. Let $r \in R$. Consider the descending chain of ideals $(r) \supset (r^2) \supset
 \cdots$. By the artinian condition, this chain stabilizes, so $(r^n) = (r^{n+1})$ for
 some $n$. This means that there exists $s \in R$ such that $r^{n+1}s = r^n$, or
 equivalently $r^n(rs - 1) = 0$. If $rs - 1 = 0$, then $r$ is a unit. If not, then $rs - 1
 \neq 0$. In this case, for some $k > 0$, $x = (rs-1)r^{k-1}$ is non-zero, but $rx = 0$,
 so $r$ is a zero divisor.

 Let $R$ be a local artinian ring. Let $\mathfrak p$ be any prime ideal in $R$. Because
 the ideals of $R/\mathfrak p$ lift to ideals in $R$, preserving inclusion and equality,
 $R/\mathfrak p$ is also artinian. Since $\mathfrak p$ is prime, $R/\mathfrak p$ is a
 domain, so $\mathcal Z(R/\mathfrak p) = \{0\}$. By the previous paragraph, $R/\mathfrak
 p$ must be a field, so $\mathfrak p$ is in fact a maximal ideal. Since $R$ is local,
 there is therefore only a single prime ideal and so $\nil(R)$ is exactly the unique
 maximal ideal in $R$. Furthermore, $\mathcal Z(R)$ is the set of non-units, which is also
 equivalent to the maximal ideal.

 Conversely, let $R$ be an artinian ring such that $\mathcal Z(R) = \nil(R)$. The
 nilradical is an ideal and I claim that it is in fact the unique maximal ideal. Any
 proper ideal in $R$ must be disjoint from the units, so it must be contained in $\mathcal
 Z(R)$ by the first paragraph. Thus $\mathcal Z(R) = \nil(R)$ is the unique maximal ideal
 in $R$.
 \end{solution}

 \begin{exercise}{I.29}
     For any artinian ring $R$, show that there exists an integer $n_0$
     such that, for all $n\geq n_0$ and all $\mathfrak{m}\in \Max(R)$,
     there is a ring isomorphism $R/\mathfrak{m}^n\cong R_{\mathfrak{m}}$.
 \end{exercise}
 \begin{solution}{Lars Kindler, lars\_k@berkeley.edu}
     We can write $R$ as a finite product $R=R_1\times\ldots\times R_k$ of local artinian rings
     $(R_i,\widehat{\mathfrak{m}}_i)$. Then the maximal ideals of $R$ are just
     $\mathfrak{m}_i=R_1\times\ldots\times R_{i-1}\times \widehat{\mathfrak{m}}_i\times
     R_{i+1}\times\ldots\times R_k$
     (see exercise 8) and for each $i\in\left\{
     1,\ldots,k \right\}$ we define the integer $n_i$ to be the unique number for which
     $\widehat{\mathfrak{m}}_i^{n_i-1}\neq (0)$ and
     $\widehat{\mathfrak{m}}_i^{n_i}=(0)$ (in a local artinian ring the maximal ideal is
     nilpotent).\\
     Now observe that for all $n\geq n_i$ we have $R/\mathfrak{m_i}^n \cong
     R_i/\widehat{\mathfrak{m}}_i^n\cong R_i \cong R_{\mathfrak{m}_i}$, where the third
     isomorphism is just the localization of the projection $(x_1,\ldots,x_k)\mapsto x_i$.
     Indeed we have $(R_i)_{\widehat{m}_i}=R_i$ and for $x/s\in R_{\mathfrak{m}_i}$ with
     $x_i/s_i=0$ in $R_i$, there is a $t_i\in R_i\setminus\widehat{\mathfrak{m}}_i$ such that $t_ix_i=0$.
     But this means we have $t=(0,\ldots,0,t_i,0,\ldots,0)\in R\setminus\mathfrak{m}_i$, $tx=0$
     and thus $x/s=0$ in $R_{\mathfrak{m_i}}$. So the surjective morphism
     $R_{\mathfrak{m_i}}\rightarrow R_i$ is injective and therefore an isomorphism.
     It follows that $n_0:=\max\left\{ n_1,\ldots,n_k \right\}$ is the integer we are looking for.
 \end{solution}

 \begin{exercise}{I.30}
  True or False: if $M$ is a f.g. module over a noetherian ring $R$ and $M= P +Q$, then
  $\ass(M) = \ass(P) \cup \ass(Q)$?
 \end{exercise}
 \begin{solution}{Jonah}
  This is false.  Take $R = \ZZ$, $M = \ZZ \oplus \ZZ/6\ZZ$, $P = ((1,2)) \cong \ZZ$, and
  $Q = ((1,3)) \cong \ZZ$.  $P +Q = M$ since $(1,3) - (1,2) = (0,1) \in P+Q$ and therefore
  $(1,0) = (1,3) - 3(0,1) \in P+Q$.  $\ass(P) = \ass(Q) = \{(0)\}$ whereas $\ass(M) =
  \{(0),(2),(3)\}$.
 \end{solution}

 \begin{exercise}{I.31}
  True or False: a module over a local ring $(R,\m)$ is f.g. if $M/\m M$ is f.g. over $R/\m$?
 \end{exercise}
 \begin{solution}{igusa@math}
   False: Let $k$ be a field, let $R=k[x]_{(x)}$, and let $M=k(x)$\\
  $M$ is clearly not finitely generated over $R$, but $\m M=xM=M$ so \\$M/\m M=M/M=0$ which
  is finitely generated over $R/\m=k$
 \end{solution}

 \begin{exercise}{I.32}
Let $f : M \to N$ be an $R$-module homomorphism, where $N$ is f.g. Show that $f$ is
surjective iff $\bar{f} : M/\m M \to N/\m N$ is surjective for every $\m \in \Max(R)$.
 \end{exercise}

 \begin{solution}{siveson@math}
   Suppose $f$ is surjective and let $\m  \in \Max(R)$. Then for any $x \in N/\m N$, $x$ is
   the image of $n$ under the map $N \to N/\m N$ for some $n \in N$. Since $f$ is
   surjective, $f(m) = n$ for some $m \in M$. Then $\bar{f}(\bar{m}) = \bar{n} = x$ so
   $\bar{f}$ is surjective for any $\m $.

   Now suppose that $\bar{f}: M/\m M \to N/\m N$ is surjective for every $\m  \in \Max(R)$.
   Let $C = \textup{coker}(f)$, so
   $$M \to N \to C \to 0$$ is exact. Now $f$ is surjective iff $C =
   0$ iff $C_{\m } = 0$ for all $\m  \in \Max(R)$. Since tensoring is right exact,
   $$M_{\m } \to N_{\m } \to C_{\m } \to
   0$$ is exact. Since $N$ is a f.g. $R$-module, $N_{\m }$ is finitely generated as an
   $R_{\m }$-module, and since $N_{\m } \twoheadrightarrow C_{\m }$, $C_{\m }$ is also
   finitely generated as an $R_{\m }$-module. Since $R_{\m }$ is a local ring with
   maximal ideal $\m _{\m }$, Nakayama's lemma then says that $\m _{\m }C_{\m } = C_{\m }
   \Rightarrow C_{\m } = 0$.  So to prove that $f$ is surjective, we need only show
   $\m_{\m }C_{\m } = C_{\m }$ for all $\m $. But $\bar{f}: M/\m M \to N/\m N$ is
   surjective for every $\m  \in \Max(R)$, so $C/\m C = \textup{coker}(\bar{f}) = 0$ for
   all $\m $. Then localizing at $\m $, we get $C_{\m }/\m _{\m }C_{\m } = 0$ for all $\m
   $. So $\m_{\m }C_{\m } = C_{\m }$ for all $\m $, and hence $f$ is surjective.
 \end{solution}

 \begin{exercise}{I.33}
   If $M$ is a f.g.\ module over a local ring $R$ show that any minimal (that is,
   ``unshrinkable'') set of generators for $M$ has constant cardinality, namely,
   $\mu_R(M)$. Does the same statement hold over a semilocal ring?
 \end{exercise}
 \begin{solution}{anton@math}
   Let $\m\subseteq R$ be the maximal ideal, and let $x_1,\dots, x_n$ be a set of
   generators for $M$. Then the images $\bar x_1,\dots, \bar x_n$ generate $M/\m M$ as an
   $R/\m$ vector space. If one of these, say $\bar x_n$, can be removed (so that the rest
   generate $M/\m M$), then by Nakayama's Lemma, $x_1,\dots, x_{n-1}$ generate $M$ (let
   $N$ be the submodule generated by $x_1,\dots, x_n$). Thus, any minimal set of
   generators has size $\dim_{R/\m} M/\m M$, which must therefore be equal to $\mu_R(M)$.

   The result is not true in a semilocal ring. Let $R$ be the ring $\ZZ$ semilocalized at
   the primes $(2)$ and $(3)$. Then $\mu_R(R)=1$ because the element $1$ generates. The
   two elements $2$ and $3$ also generate. However, since $2$ and $3$ are not units
   (i.e.\ neither can generate $R$ by itself), this generating set is minimal.
 \end{solution}

 \begin{exercise}{I.34}
   Find an example of a ring $R$ such that, for any positive integers $n$, $m$, there
   exists an $R$-module $M$ with $\mu_R(M)=n$ but $M$ has a minimal set $S$ of generators
   with $|S|=n+m$. (\textbf{Hint.} Use Euclid's theorem on the infinitude of primes.
   Author's solemn pledge: this hint works!)
 \end{exercise}
 \begin{solution}{anton@math}
   Let $R=\ZZ$. First we'll do the case $n=1$. Clearly $\mu_R(R)=1$. Let $p_1$,
   \dots, $p_{m+1}$ be the first $m+1$ primes. Then let $S=\{p_1\cdots \hat p_i\cdots
   p_{m+1}\}_{1\le i\le m+1}$. $S$ generates $R$ because it is a set of relatively prime
   integers, but if one removes $p_1\cdots\hat p_i\cdots p_{m+1}$, every element of the
   remaining set is divisible by $p_i$, so the set no longer generates.

   For larger $n$, choose $M=\ZZ ^n$. Then $\mu_R(M)=n$ by the ``rank property'' in the
   corollary to McCoy's third theorem (which tells us that $M$ cannot be generated by
   fewer than $n$ elements). Then let $e_i$ be the element of $M$ with a 1 in the $i$-th
   place and zeros elsewhere, and let $S=\{e_i\}_{i< n}\cup \{p_1\cdots \hat p_i\cdots
   p_{m+1}\cdot e_n\}_{1\le i\le m+1}$. By essentially the same argument as in the
   previous paragraph, this is a minimal generating set for $M$.
 \end{solution}

 \begin{exercise}{I.35}
  Let $I_1, \ldots, I_n \lhd R$ be such that $I_1 + \dots + I_n \neq R$. For $M=
  \bigoplus_i R/I_i$, show that $\mu_R(M) = n$. Does this formula still hold if $I_1 +
  \ldots + I_n = R$?
 \end{exercise}
 \begin{solution}{Jonah}
   Let $e_i \in M$ be the image of 1 in the $i$th summand $R/I_i$ and 0 in the
   other summands.  These clearly generate $M$ so $\mu_R(M) \leq n$.
   To see that no smaller generating set exists suppose $f_1, \ldots, f_k$ generate $M$.
   Choose some maximal ideal $m$ containing $I_1, \ldots, I_n$.
   $M$ surjects onto $M/mM$ so the images $\bar{f_i}$ of the $f_i$ generate
   $M/mM$.  This is equivalent to the $\bar{f_i}$ generating $M/mM$ as
   a module over $R/m$.  Now since $M/mM \equiv \bigoplus_i (R/I_i)/m(R/I_i)
   \equiv (R/m)^n$ linear algebra tells us $k \geq n$.

   This formula does not necessarily hold if $I_1 + \ldots + I_n = R$.  For
   example, take $R = \ZZ$, $I_1 = (2)$, and $I_2 = (3)$.  $(1,1)$
   generates the module $M = \ZZ/2\ZZ \oplus \ZZ/3\ZZ$.
 \end{solution}

 \begin{exercise}{I.36}
   For any $_R M$ and any multiplicative set $S\subseteq R$, show that
   $\text{ann}(M)_S\subseteq\text{ann}(M_S)$.  If $M$ is f.g., show that
   equality holds.
   Give an example to show that equality fails in general.
 \end{exercise}
 \begin{solution}{ecarter@math}
   Any element of $\text{ann}(M)_S$ has the form $r/s$ for some $r\in\text{ann}(M)$
   and some $s\in S$.  Let $m/s'\in M_S$.  Then
   \[
           (r/s)(m/s')=(rm)/(ss')=0/(ss')=0.
   \]
   Therefore $r/s\in\text{ann}(M_S)$.

   Now suppose $M$ is generated over $R$ by $m_1,m_2,\dots,m_n$, and let
   $r/s\in\text{ann}(M_S)$.  Then $(rm_i)/s=0$ in $M_S$ for each $i$.  This means
   that for each $i$, there exists $s_i\in S$ such that $s_i r m_i=0$ in $M$. If we
   let $s_0=\prod_{i=1}^n s_i$, then $s_0\in S$ and $s_0 r m_i=0$ for all $i$. Since
   the $m_i$'s generate $M$, $s_0 r\in\text{ann}(M)$, so that $r/s=(s_0 r)/(s_0
   s)\in\text{ann}(M)_S$.

   For a field $k$, let $R=k[x]$, $S=\{1,x,x^2,\dots\}$, and
   $M=\bigoplus_{n=1}^\infty k[x]/(x^n)$. Then $\text{ann}(M)=\bigcap_{n=1}^\infty
   (x^n)=(0)$ so that $\text{ann}(M)_S=(0)$. However, for all $n>0$,
   $(k[x]/(x^n))_S=0$, so $M_S=0$.  Therefore $\text{ann}(M_S)=R_S$.
 \end{solution}

 \begin{exercise}{I.37}
     Let $=k\times k \times \ldots$ where $k$ is a field, and let $M$ be
     the ideal $k\oplus k \oplus\ldots$. Show that
     $\operatorname{Supp}(M)=\spec R\setminus V(M)$. Does the equality
     $\operatorname{Supp}(M)=V(\operatorname{ann}(M))$ hold in this example? What is
     ``wrong''?
 \end{exercise}
 \begin{solution}{Lars Kindler, lars\_k@berkeley.edu}
     First, let $\mathfrak{p}\in V(M)$ and $m=(m_i)_i\in M$. Then infinitely many
     coordinates of $m$ are zero and only finitely many are nonzero. Pick a $s\in R$
     such that for all $i$ the coordinate $s_i$ is zero if $m_i$ is nonzero and nonzero if
     $m_i$ is zero. Then $s$ is not in $\mathfrak{p}$, since otherwise $s+m\in U(R)$
     would be in $\mathfrak{p}$. But we also have $sm=0$, so $M_{\mathfrak{p}}=0$.\\
     Now let $\mathfrak{p}\in \spec(R)\setminus V(M)$. Then an element $x\in
     M\setminus\mathfrak{p}$ is invertible in $R_\mathfrak{p}$, so
     $M_{\mathfrak{p}}=R_\mathfrak{p}\neq 0$, and thus we have
     $\operatorname{Supp}(M)=\spec(R)\setminus V(M)$ as claimed.\\
     We have $\operatorname{ann}(M)=(0)$, since an element in the annihilator would have
     to kill each element $e_i=(\delta_{ij})_j$, where $\delta$ is the Kronecker symbol.
     This means $V(\operatorname{ann}(M))=\spec(R)$, but by what we have shown in the
     first part we have a strict inclusion $\operatorname{Supp}(M)\subsetneqq \spec R$,
     since $V(M)\neq \emptyset$. Together this gives $\operatorname{Supp}(M)\subsetneqq
     V(\operatorname{ann}(M))$. The ``wrong'' thing is that $M$ is not finitely
     generated!
 \end{solution}

 \begin{exercise}{I.38}
     For any f.g. module ${}_R M$, show that $\operatorname{Supp}(M)=\spec(R)$ iff
     $\operatorname{ann}(M)\subset\nil(R)$.
 \end{exercise}
 \begin{solution}{Lars Kindler, lars\_k@berkeley.edu}
  ``$\Rightarrow$'': Suppose there is an $a\in \operatorname{ann}(M)\setminus\nil(R)$
  and hence a $\mathfrak{p}\in\spec(R)$ such that $a\not\in \mathfrak{p}$, since
  $\nil(R)$ is the intersection of all primes in $R$. But this means
  $M_\mathfrak{p}=0$, since for every $m/t\in M_{\mathfrak{p}}$, where $t\in
  R\setminus\mathfrak{p}$ and $m\in M$, we have
  $m/t=(am)/(at)=0$. This contradicts the hypothesis $\spec(R)=\operatorname{Supp}(M)$.\\
  ``$\Leftarrow$'': Now let $M=\sum_{i=1}^n R x_i$ be f.g. and
  $\operatorname{ann}(M)\subset \nil(R)$. Assume that there is a
  $\mathfrak{p}\in\spec(R)$ such that $M_\mathfrak{p}=0$. Then we have for every
  $i\in\left\{ 1,\ldots,n \right\}$ an $a_i\in R\setminus\mathfrak{p}$ with $a_i x_i = 0$
  and since $R\setminus\mathfrak{p}$ is a multiplicative set we have
  $a:=a_1\cdot\ldots\cdot a_n\in R\setminus\mathfrak{p}$. But we also have $aM=0$ which
  is a contradiction, since $ \operatorname{ann}(M) \subset\nil(R)$ implies
  $\operatorname{ann}(M)\subset\mathfrak{p}$.
 \end{solution}

 \begin{exercise}{I.39}
  Give an example of a f.g. module over a noetherian ring such that for every $n$, $M$
  has a prime filtration of length $n$.
 \end{exercise}
 \begin{solution}{derman@math}
  Set $R=\ZZ$ and $M=\ZZ$.  Then the prime filtration $\ZZ\supset 2\ZZ \supset 4\ZZ
  \supset \dots 2^{n-1}\ZZ \supset 0$ has length $n$.  The first $n$ factors are
  isomorphic to $\ZZ / 2\ZZ$ and the last one is isomorphic to $\ZZ/(0)$.
 \end{solution}

 \begin{exercise}{I.40}
 Give an example of f.g. modules $N\subset M$ over a noetherian ring with
 $Ass(M/N)\subsetneq Ass(M)$.
 \end{exercise}
 \begin{solution}{derman@math}
 Let $R=\mathbb Z[x]$, $M=\mathbb Z_2[x]$ and $N=(x)\mathbb Z_2[x]$.  Then
 $Ass(M/N)=\{(2,x)\}$ and $Ass(M)=\{(2)\}$.
 \end{solution}

 \begin{exercise}{I.41}
   Give a direct proof that $\dim(R)=0$ implies $C(R)=U(R)$ using the fact that every
   minimal prime consists of $0$-divisors.
 \end{exercise}
 \begin{solution}{derman@math}
    Since $\dim(R)=0$ it follows maximal ideals are the same as minimal primes.  Thus if
   $a\notin U(R)$ then $(a)$ belongs to a maximal ideal $\mathfrak m$.  But $\mathfrak m$
   is also a minimal prime, and thus $a$ must be a zero divisor.  So $C(R)\subseteq
   U(R)$.  The other inclusion always holds.
 \end{solution}

 \begin{exercise}{I.42}
     Show that any ring $R$ can be embedded in a Marot ring.
     (\textbf{Hint}. This exercise is somewhat related to Ex. 41).
 \end{exercise}
 \begin{solution}{Lars Kindler, lars\_k@berkeley.edu}
     Let $T=\mathcal{C}(R)^{-1}R$ be the total ring of quotients of $R$.
     Then the localization map $R\rightarrow T$ is injective and we claim
     that $T$ is a Marot ring.
     Let $x/c$ be a regular element in $T$ which means we have $x/c\cdot y/c'\neq
     0$ for all nonzero $y/c'\in T$. Since we have only
     inverted regular elements, this is equivalent to
     $xy\neq 0$ for all $y\neq 0$, i.e. to $x\in \mathcal{C}(R)$. We have
     shown that $\mathcal{C}(T)\subseteq U(T)$, but $U(T)\subseteq \mathcal{C}(T)$ is trivial, so
     we have $\mathcal{C}(T)=U(T)$ and thus $T$ is a Marot ring, since there are no
     regular ideals.
 \end{solution}

 \begin{exercise}{I.43}
   Let $\left(  R,\mathfrak{m}\right)  $ be a noetherian local ring such that
   $\mathfrak{m}\smallsetminus\mathfrak{m}^{2}\subseteq\mathcal{Z}\left( R\right)  $. \
   Show that $\mathfrak{m=}\mathcal{Z}\left(  R\right)  $, and
   $\mathfrak{m\in}\operatorname*{Ass}\left(  R\right)  $.
 \end{exercise}
 \begin{solution}{mreyes@math}
   First suppose that $\mathfrak{m}=\mathfrak{m}^{2}$. \ Because $R$ is
   noetherian, $\mathfrak{m}$ is a finitely generated $R$-module. \ Also because
   $R$ is local, $\mathfrak{m=}\operatorname*{rad}R$. \ So by Nakayama's Lemma,
   we must have $\mathfrak{m=}0$. \ So $R$ is a field and the claim clearly follows.

   So we may now assume that $\mathfrak{m\neq m}^{2}$. \ By Theorem 6.2, there
   exist primes $\mathfrak{p}_{1},\ldots,\mathfrak{p}_{n}\subseteq R$ and
   elements $r_{i}\in R$ such that $\mathcal{Z}\left(  R\right)  =\bigcup
   \mathfrak{p}_{i}$ and each $\mathfrak{p}_{i}=\operatorname*{ann}\left(
   r_{i}\right)  $. \ Now $\mathfrak{m\smallsetminus m}^{2}\subseteq
   \mathcal{Z}\left(  R\right)  $ implies that $\mathfrak{m\subseteq}%
   \mathcal{Z}\left(  R\right)  \cup\mathfrak{m}^{2}\mathfrak{=p}_{1}\cup
   \cdots\cup\mathfrak{p}_{n}\cup\mathfrak{m}^{2}$. \ Then since
   $\mathfrak{m\nsubseteq m}^{2}$, the Prime Avoidance Theorem implies that
   $\mathfrak{m\subseteq p}_{i}$ for some $i$. \ By maximality of $\mathfrak{m}$,
   this means that $\mathfrak{m=p}_{i}=\operatorname*{ann}\left(  r_{i}\right)
   $. \ At this point we have that $\mathfrak{m\in}\operatorname*{Ass}\left(
   R\right)  $. \ Also, $\mathfrak{m=p}_{i}\subseteq\mathcal{Z}\left(  R\right)
   $. \ Because $R$ is local, $R\smallsetminus\mathfrak{m=}U\left(  R\right)  $.
   \ This means that $\mathfrak{m=}\mathcal{Z}\left(  R\right)  $, completing the
   proof. \
 \end{solution}
 \begin{solution}{Lars Kindler, lars\_k@berkeley.edu}
    Assume $\m \neq 0$. If $\m^2\not\subset \mathcal{Z}(R)$, then $\m^2$ is a regular
    ideal. Furthermore, if $a,b\in \mathcal{C}(R)$ are elements such that $ab\in
    \m^2$, then either $a\in \m^2$ or $b\in \m^2$, because $\mathcal{C}(R)\cap
    (\m\setminus\m^2) = \varnothing$. So since $R$ in particular is a Marot ring we see
    that $\m^2$ is prime which implies $\m^2=\m$, so by Nakayama's Lemma we get
    $\m=0$, which is a contradiction.\\
    Now the maximality of $\m=\mathcal{Z}(R)$ implies $\m\in \ass(R)$ by Prime Avoidance, because
    $\mathcal{Z}(R)$ is a finite union of associated primes.
 \end{solution}
 \begin{solution}{lam@math}
   If you are not yet totally tired of this exercise, here's a third solution!

   \medskip
   As in Manny's solution, we may assume there exists an element $x\in {\mathfrak
   m}\setminus {\mathfrak m}^2$. By hypothesis, ${\mathfrak m}\setminus {\mathfrak
   m}^2\subseteq \bigcup_{i=1}^n {\mathfrak p}_i$, where ${\mathfrak p}_i$ ranges over ${\rm
   Ass}(R)$. Consider any $\,y\in {\mathfrak m}^2$. We have $x+y^p\notin {\mathfrak m}^2$
   for all $p$, so by the Pigeon-Hole Principle, there exist $p<q$ such that
   $x+y^p,\,x+y^q\in {\mathfrak p}_i$ for some $i$. Then, $y^p(1-y^{q-p})\in {\mathfrak
   p}_i\Rightarrow y\in {\mathfrak p}_i \Rightarrow y\in {\mathcal Z}(R)$. This shows that
   $\,{\mathfrak m}={\mathcal Z}(R)$.  By (6.2), ${\mathfrak m}\,a=0$ for some $\,a\neq 0$,
   so ${\mathfrak m} ={\rm ann}(a)\in {\rm Ass}(R)$.
 \end{solution}


 \begin{exercise}{I.44}
   For $\m\in \Max(R)$, show that the following are equivalent: (1) $\m\in \ass (R)$; (2)
   $\m$ is not dense in $R$; (3) $\ann(\ann (\m))=\m$.
 \end{exercise}
 \begin{solution}{anton@math}
   ($1\Rightarrow 2$) If $\m\in \ass R$, then $\m=\ann(r)$ for some $r\in R$, so
   $r\cdot \m=0$,  so $\m$ is not dense.

   ($2\Rightarrow 3$) If $S\subseteq R$ is a set, then $\ann(S)=R$ if and only if
   $S=\varnothing$ or $\{0\}$. If $\m$ is not dense, then $\ann(\m)$ is not one of these,
   so $\ann(\ann \m)\neq R$. But $\ann(\ann \m)$ is an ideal which clearly contains $\m$,
   so it must be $\m$ since $\m$ is maximal.

   ($3\Rightarrow 1$) If $\ann(\ann \m)=\m\neq R$, then $\ann(\m)$ contains a non-zero
   element, $r$. Then $\ann (r)$ is a proper ideal of $R$ which contains $\m$, so
   $\ann(r)=\m$ by maximality of $\m$. Thus, $\m\in \ass R$.
 \end{solution}

 \begin{exercise}{I.45}
  (Elaboration of (5.11).) Show that a f.g. module $M$ over a noetherian ring $R$ has
  finite length iff $\ass(M) \subseteq \Max(R)$, iff there exist (not necessarily distinct)
  $m_i \in \Max(R)$ such that $m_1 \dots m_n M = 0$.  Give examples to show that , for
  these equivalences, the assumptions that $M$ be f.g. and $R$ be noetherian are both
  essential.
 \end{exercise}
 \begin{solution}{Jonah, jblasiak@math}
  $M$ has finite length implies that any submodule has finite length so if $M$ contains a
  submodule isomorphic to $R/p$, then $R/p$ has finite length as an $R$ module.  This is
  equivalent to having finite length as an $R/p$ module so $R/p$ is an artinian ring and
  therefore of dimension 0.  Thus $p$ is maximal.  Since $p$ is an associated prime of $M$
  if and only if $R/p$ is isomorphic to a submodule of $M$ we have that $\ass(M) \subseteq
  \Max(R)$.

  Now assume that $\ass(M) \subseteq \Max(R)$.  All associated primes
  are then isolated primes and corollary 6.8 implies $\sqrt{\ann(M)} =
  \bigcap \{p:p \in \ass(M)\}$.  Let $p_1, p_2, \ldots, p_n$ be the
  associated primes of $M$.  We have that $p_1 p_2 \ldots p_n
  \subseteq \sqrt{\ann(M)}$ and therefore $p_1 p_2 \ldots p_n
  \subseteq \sqrt{0}$ in $R/\ann(M)$.  Since $R$ is noetherian
  (Exercise 27) $\sqrt{0}$ is nilpotent thus there exists an $r$ such
  that $p_1^r p_2^r \ldots p_n^r \subseteq \ann(M)$.  Then $p_1^r
  p_2^r \ldots p_n^r M = 0$, as desired.

  Now suppose there exist $m_i \in \Max(R)$ such that $m_1 \dots m_n M
  = 0$.
  $$0 = m_1 \dots m_n M \subseteq m_2 \dots m_n M \subseteq \ldots
  \subseteq m_n M \subseteq M$$ is a finite filtration of $M$.  The
  $i$th filtration factor is a vector space over $R/m_i$.  It is
  finite dimensional because $M$ is noetherian, and therefore it has a
  finite composition series.  Putting these composition series
  together for each vector space yields a finite composition series
  for $M$.

  If $M$ is not f.g. both equivalences fail.  For instance, take $R =
  k$ a field and $M = \bigoplus_{\ZZ} k$.  $\ass(M) = \{(0)\} =
  \Max(R)$, but $M$ doesn't have finite length.  Another example is
  with $R = \RR[x]$ and $M = \bigoplus_{i \in \ZZ} R/(x-i)$.  $\ass(M)
  = \{ (x-i) : i \in \ZZ\} \subseteq \Max(R)$ because all point
  annihilators are products of some $(x-i)$s.  However no finite
  product of maximal ideals can kill every element of $M$ : just
  choose an element that is 1 in the summand corresponding to
  $R/(x-j)$, where $j$ is such that $(x-j)$ does not appear in the
  list of maximals.

  Both equivalences also fail if $R$ is not noetherian.  Take $R =
  k[x_1, \ldots,x_n,\ldots]/(x_1^2,x_2^2, \ldots)$ and $M = R$.  $M$
  has no associated primes, but is not of finite length.  For
  instance, $0 \subset (x_1) \subset (x_1, x_2) \subset \ldots$ is an
  infinite chain of distinct submodules.  For the other equivalence
  take $R = k \times k \times \ldots$ and $M = R$.  $R$ is 0
  dimensional so $\ass(M) \subseteq \Max(R)$.  However if we suppose
  there exist $m_i \in \Max(R)$ such that $m_1 \dots m_n M = 0$, then
  $m_1 \dots m_n = 0$.  It is easy to see there are infinitely many
  maximal ideals, so take one, $m$ say, not equal to any of the $m_i$.
  Then $m_1 \dots m_n \subset m$ and $m$ prime implies $m_i \subseteq
  m$ for some $i$, contradiction.
 \end{solution}

 \begin{exercise}{I.46}
   For a module $M$ of finite length (over any ring $R$), let
   $R/\mathfrak{m}_{i}\ (\mathfrak{m}_{i}\in Max(R))$ be the distinct
   types of composition factors of $M,$ occurring with multiplicities
   $n_{i}>0.$ Show that
   $n_{i}=length_{R_{\mathfrak{m}_{i}}}M_{\mathfrak{m}_{i}},$ and that the
   localization map $M\to\bigoplus_{i}M_{\mathfrak{m}_{i}}$ is an
   isomorphism of $R$-modules.
 \end{exercise}
 \begin{solution}{Shenghao Sun, shenghao@math.}
  Let $0=M_{0}\subset M_{1}\subset\cdots\subset M_{n}=M$ be a composition series of $M.$
  Localization commutes with quotient, so
  $\frac{(M_{j+1})_{\m_{i}}}{(M_{j})_{\m_{i}}}=(M_{j+ 1}/M_{j})_{\m_{i}}.$ Say
  $M_{j+1}/M_{j}\cong R/\m_{i'}.$ If $\m_{i}=\m_{i'},$ then
  $(R/\m_{i})_{\m_{i}}=k(\m_{i})$ is the residue field of $\m_{i};$ if
  $\m_{i}\neq\m_{i'},$ then $\m_{i}+\m_{i'}=R$ and therefore
  $(R/\m_{i'})_{\m_{i}}=\frac{R}{\m_{i'}}\otimes_{R}\frac{R}{\m_{i}}=\frac{R}{\m_{i'}+
  \m_{i}}=0.$ Note that $k(\m_{i})$ is a simple $R_{\m_{i}}$-module. Localizing the above
  composition series at $\m_{i}$ we get a composition series for $M_{\m_{i}}$ over
  $R_{\m_{i}},$ some of $\subset$ becoming $=,$ and the number of proper inclusion left
  is $n_{i}.$ On the other hand, that number is also $length_{R_{\m_{i}}}M_{\m_{i}}.$

  An $R$-linear map $f:M\to N$ is an isomorphism if and only if $f_{\m}:M_{\m}\to N_{\m}$
  is an isomorphism, for any $\m\in Max(R).$ Let $\m,\mathfrak{n}\in Max(R).$ Then
  $R_{\m}\otimes_{R}R_{\mathfrak{n}}=\begin{cases}R_{\m},\ \text{if
  }\m=\mathfrak{n}; \\ 0,\ \text{otherwise.}\end{cases}.$ So the map
  $M\to\bigoplus_{i}M_{\m_{i}}$ becomes an isomorphism if we localize at any one of
  $\m_{i}.$ For any other maximals $\m,$ applying localization to the composition series
  for $M$ we find $M_{\m}=0,$ so again we get an isomorphism.
 \end{solution}

 \begin{exercise}{I.47}
     Let $I\lhd R$, where $R$ is a noetherian ring.
     \begin{enumerate}
         \item Show that $R/I$ is artinian iff $R/\sqrt{I}$ is artinian.
         \item If $R$ is semilocal, show that $R/I$ is artinian iff $I\supseteq \rad(R)^n$
             for some $n\geq 1$.
     \end{enumerate}
 \end{exercise}
 \begin{solution}{Lars Kindler, lars\_k@berkeley.edu}
     \begin{enumerate}
         \item Since $I\subseteq \sqrt{I}$ we have a homomorphism $R/I\rightarrow
             R/\sqrt{I}$, so $R/I$ being artinian implies $R/\sqrt{I}$ being artinian.\\
             Now assume $R/\sqrt{I}$ is artinian and let
             $\bar{\mathfrak{p}}_1\subseteq\bar{\mathfrak{p}}_2$ be primes of $R/I$. They
             correspond to primes $\mathfrak{p}_1\subseteq \mathfrak{p}_2$ of $R$
             containing $I$, which
             in turn correspond to primes $\tilde{\mathfrak{p}}_1\subseteq
             \tilde{\mathfrak{p}}_2$ of $R/\sqrt{I}$, since every prime containing
             $I$ also contains the radical of $I$. So since $R/\sqrt{I}$ is artinian
             we have $\tilde{\mathfrak{p}}_1=\tilde{\mathfrak{p}}_2$ and thus
             $\bar{\mathfrak{p}}_1=\bar{\mathfrak{p}}_2$, i.e $R/I$ is a $0$-dimensional
             noetherian ring and thus artinian.
         \item If $R/I$ is artinian then $\rad(R/I)$ is nilpotent, i.e. there is a $n\in
             \mathbb{N}$ such that
             $\rad(R/I)^{n}=\left( \bigcap \mathfrak{p}\right)^n+ I = 0 + I$, where
             $\mathfrak{p}$ ranges over all the primes of $R$ containing $I$. But this
             means $\rad(R)^n\subseteq \left( \bigcap \mathfrak{p} \right)^n \subset
             I$.\\
             Now assume $I\supseteq\rad(R)^n=( \bigcap_{i=1}^k
             \mathfrak{m}_i)^n=\mathfrak{m}_1^n\cdot\ldots\cdot\mathfrak{m}_k^n$, where $\max(R)=\left\{
             \mathfrak{m}_1,\ldots,\mathfrak{m}_k\right\}$.  Two primes
             $\bar{\mathfrak{p}}_1\subseteq \bar{\mathfrak{p}}_2$ of $R/I$ correspond to primes
             $\mathfrak{p}_1\subseteq \mathfrak{p}_2$ of $R$ containing $I$, so we
             have
             $\mathfrak{p_2}\supseteq\mathfrak{p}_1\supseteq I \supseteq \mathfrak{m}_1^n\cdot\ldots\cdot\mathfrak{m}_k^n$,
             which means $\mathfrak{p}_2=\mathfrak{p}_1=\mathfrak{m}_i$ for some $i\in
             \left\{ 1,\ldots,k \right\}$. It follows that
             $\bar{\mathfrak{p}}_1=\bar{\mathfrak{p}}_2$ in $R/I$, so $R/I$ again is a
             $0$-dimensional noetherian ring and therefore artinian.
     \end{enumerate}
 \end{solution}

 \begin{exercise}{I.48}
   (Nagata [${\it Na_1}$]) Show that $R$ is noetherian if it is locally noetherian
   (that is, $R_{\m}$ is noetherian for every $\m \in \Max(R)$) and $R/(a)$ is
   semilocal for every $a \in R \setminus \{0\}$. ({\bf Hint.} To show that
   $I \< R$ is f.g., pick $a_1,\ldots,a_n$ such that $I$ and $(a_1,
   \ldots,a_n)$ are contained in exactly the same maximal ideals ${\m}_j$.
   Then pick $b_{jk} \in I$ that generate $I_{{\m}_j}$, and show that
   $\{a_i, b_{jk}\}$ generates $I$.)
 \end{exercise}
 \begin{solution}{los@math}
   Let $I_k$ be an increasing sequence of ideals in $R$ for $k \geq 0$. We must show that
   the sequence $I_k$ is stationary. We may assume that $I_0
   \neq 0$. Let $a \in I_0 \setminus \{0\}$. It will be enough to show that the sequence
   of ideals $I_k/(a)$ in $R/(a)$ is stationary. Thus, replacing $R$ with $R/(a)$,
   which satisfies the same hypotheses as $R$ since a quotient of a locally noetherian
   (resp. semilocal) ring is again locally noetherian (resp. semilocal), we may assume
   that $R$ itself is semilocal. Let $\m_j$, $j=1,\ldots,n$, denote the maximal ideals of $R$.
   For each $j$, the sequence of ideals ${(I_k)}_{\m_j}$ of the noetherian ring $R_{\m_j}$
   is increasing, hence stationary. Since there are only finitely many $j$, there is therefore
   a $K \geq 0$ such that for all $j=1,\ldots,n$ and all $k \geq K$, we have ${(I_{k+1})}_{\m_j}=
   {(I_k)}_{\m_j}$. Let $k \geq K$. Then, for all $\m \in \Max(R)$, we have
   ${(I_{k+1}/I_k)}_{\m} \cong {(I_{k+1})}_{\m}/{(I_k)}_{\m}=0$. Hence $I_{k+1}/I_k=0$,
   and the sequence $I_k$ is stationary.
 \end{solution}

 \begin{exercise}{I.49}
    (Stable Range $1$ Property) If $Ra+Rb=R$ in a semilocal ring $R$,
    show that there exists $r \in R$ such that $a+rb \in U(R)$.
 \end{exercise}
 \begin{solution}{Soroosh}
    Let $\m_1, \m_2,...,\m_n$ be the set of all maximal ideal in $R$.
    Note that if $x \in R \setminus \bigcup_{i} \m_i$, then $x \in U(R)$.
    Otherwise, $xR$ is a proper ideal, which means that it must be
    contained in a maximal ideal.

    Assume without loss of generality that $a \in \m_i$ for
    $1 \leq i \leq k$ but $a \not \in \m_i$ for $i>k$. We first
    show that
    \[\Bigl( \bigcap_{i=k+1}^n \m_i \Bigr) \setminus \Bigl( \bigcup_{i=1}^k \m_i \Bigr)\]
    is not empty. Note that $\bigcap_{i=k+1}^n \m_i$ is an ideal
    (hence it is closed
    under addition and multiplication), and each $\m_i$ for $i \leq k$ is
    prime. If the above set is empty, then by prime avoidance theorem
    we get that $\bigcap_{i=k+1}^n \m_i$ is contained in $\m_j$ for
    some $j \leq k$. This certainly can't happen. Therefore the above set
    is not empty. Let $r$ be an element in the above set.
    Then $r \in \m_i$ if and only if $a \not \in \m_i$.

    Note that since $bR+aR=R$ we have that if $a \in \m_i$ then
    $b \not \in \m_i$. Therefore $a \in \m_i$ implies that
    $br \not \in \m_i$ since $\m_i$'s are all maximal, and hence prime.
    Furthermore if $a \not \in \m_i$ then $r \in \m_i$ and therefore
    $br \in \m_i$. Thus $br \in \m_i$ if and only $a \not \in \m_i$.

    Now consider $a+br$. By above construction we get that $a+br$
    avoids all the maximal ideals, since each maximal ideal contains exatly
    one of $a$ and $br$. Therefore $a+br \in R \setminus \bigcup \m_i$,
    which we've shown it is exactly the set of units.
 \end{solution}

 \begin{exercise}{I.50}
  (D. D. Anderson) Let $I\vartriangleleft R$. If all minimal prime ideals over
  $I$ are finitely generated modulo $I$, show that there are only finitely many
  minimal prime ideals over $I$. (This result generalizes the fact that
  $\left\vert \Min\left(  R\right)  \right\vert <\infty$ for a
  noetherian ring $R$.)
 \end{exercise}
 \begin{solution}{mreyes@math}
  We may assume that $R\neq0$. Note that by replacing $R$ with $R/I$ we reduce
  to the following problem: If all minimal primes of $R$ are finitely generated,
  then $R$ has only finitely many minimal primes. We will prove this latter
  statement. Let $\mathcal{F}$ denote the family of all finite products of (not
  necessarily distinct) minimal primes of $R$. We claim that $0\in\mathcal{F}$.
  If this is the case, say $0=\mathfrak{p}_{1}\cdots\mathfrak{p}_{n}$ (with the
  $\mathfrak{p}_{i}$ not necessarily distinct), then for any minimal prime
  $\mathfrak{p}$ of $R$ we have $\mathfrak{p\supseteq p}_{1}\cdots
  \mathfrak{p}_{n}$. This implies that some $\mathfrak{p}_{i}\subseteq
  \mathfrak{p}$, and thus $\mathfrak{p=p}_{i}$ by minimality of $\mathfrak{p}$.
  So $\Min\left(  R\right)  =\left\{  \mathfrak{p}_{1}%
  ,\ldots,\mathfrak{p}_{n}\right\}  $ is finite.

  So assume for contradiction that $0\notin\mathcal{F}$. Letting $\mathcal{C}$
  be the collection of ideals that do not contain any element of $\mathcal{F}$,
  we have $0\in\mathcal{C}$. We will use Zorn's Lemma to produce a maximal
  element of $\mathcal{C}$. Let $\left\{  J_{i}\right\}  $ be a chain of ideals
  in $\mathcal{C}$, and let $J^{\prime}=%
  %TCIMACRO{\tbigcup }%
  %BeginExpansion
  {\textstyle\bigcup}
  %EndExpansion
  J_{i}$. Assume for contradiction that there exists $F\in\mathcal{F}$ such that
  $F\subseteq J^{\prime}$. Because $F$ is a product of finitely generated
  ideals, $F=\left(  f_{1},\ldots,f_{r}\right)  $ is a finitely generated ideal.
  Choosing (withouth loss of generality) $J_{i_{1}}\subseteq\cdots\subseteq
  J_{i_{r}}$ such that $f_{j}\in J_{i_{j}}$, we have $F\subseteq J_{i_{r}}$ with
  $J_{i_{r}}\in\mathcal{C}$, a contradiction. So Zorn's Lemma applies and
  $\mathcal{C}$ indeed has some maximal element which we will call $J$.

  If $J$ is prime it contains a minimal prime of $R$ (Proposition 2.3), which is
  an element of $\mathcal{F}$. But this contradicts that $J\in\mathcal{C}$, so
  $J$ cannot be prime. Also, because $R\neq0$ and the ideal $R$ contains all
  elements of $\mathcal{F\neq\varnothing}$, we cannot have $J=R$. So there exist
  $a_{1},a_{2}\in R\smallsetminus J$ such that $a_{1}a_{2}\in J$. For
  $J_{i}=J+\left(  a_{i}\right)  $, we have each $J_{i}\supsetneq J$. By
  maximality of $J$, there must exist $F_{1},F_{2}\in\mathcal{F}$ with
  $F_{i}\subseteq J_{i}$. But then $F_{1}F_{2}\subseteq J_{1}J_{2}\subseteq J$
  and $F_{1}F_{2}\in\mathcal{F}$ (clearly $\mathcal{F}$ is closed under ideal
  products). This contradicts $J\in\mathcal{C}$, so it must be the case that
  $0\in\mathcal{F}$ as claimed.
 \end{solution}

 \begin{exercise}{I.51}
  Prove the analogue of (8.14) for primary ideals.
 \end{exercise}
 \begin{solution}{Jonah (jblasiak@math) and Anton}
  Let $I$ be a proper regular ideal in a Marot ring $R$.  Assume that, for all $a,b \in
  \C(R)$, $ab \in I$ implies either $a \in I$ or $b \in \sqrt{I}$. Then $I$ is primary.  We
  mimic the proof of 8.14.  Suppose there exist $x \notin I$ and $y \notin \sqrt{I}$ such
  that $xy \in I$.  Then $I + (x)$ is generated by its regular elements, so it must contain
  an element $a \in \C(R) \backslash I$.  In addition, $\sqrt{I} + (y)$ is generated by its
  regular elements, so it must contain an element $b \in \C(R) \backslash \sqrt{I}$.  We
  can write $b = s +ry$, where $s^k \in I$ and $r \in R$.  But then $a b^k = a (s+ry)^k \in
  (I + (x))(I + (y)) \subseteq I$, where the middle containment follows from the binomial
  theorem.  We have $a \notin I$, and $b \notin \sqrt{I}$ implies $b^k \notin \sqrt{I}$,
  but this contradicts our original assumption since $a$ and $b^k$ are regular.
 \end{solution}

 \begin{exercise}{I.52}
   Show that, in a PID, the primary ideals are $(0)$ and $(p^n)$, where $p$ is a nonzero
   prime element and $n \geq 1$.
 \end{exercise}
 \begin{solution}{siveson@math}
   Let $R$ be a PID, and $P \lhd R$ a primary ideal. If $P = (0)$ then we're done, so we can
   assume $P=(a)$ with $a \in R$ and $a \neq 0$. If $a$ is irreducible, then $a$ is prime,
   so we can assume $a$ is not irreducible.  Then $a = a_1a_2\cdots a_t$ with each $a_i$
   irreducible. Then $a_1a_2\cdots a_t \in P$ but $a_2\cdots a_t \notin P$ (this is because
   if $a_2\cdots a_t$ were in $P$, then $a_2\cdots a_t = ra_1a_2\cdots a_t$, so then $a_1$
   is a unit since $R$ is a PID and hence also a UFD). So since $P$ is primary, $a_1^n =
   r(a_1\cdots a_t) \in P$ for some $n$. But since $R$ is a UFD, and we can write things
   uniquely as a product of irreducibles (up to multiplication by a unit), $a_i = u_i
   a_1^{k_i}$ for some unit $u_i$ and some $k_i$ for $2 \leq i \leq t$. Then $a = ua_1^m$
   for some unit $u$ and some $m$, and hence $P = (a_1^m)$ where $a_1$ is irreducible (and
   prime).

   Now the ideal $P = (0)$ is primary since it is prime. So we just need to check now that
   the ideals $(p^n)$ are primary for all $n \geq 1$ and all nonzero primes $p$.  If $ab \in
   (p^n)$, then $ab = rp^n$ for $r \in R$. Writing $a$, $b$, and $r$ as products of
   irreducibles, $a_1 \cdots a_t b_1 \cdots b_s = r_1 \cdots r_m p^n$. But since $a_i \notin
   (p^n)$ for all $i$, and because we are in a UFD, $b_i = up$ for some $i$ and some unit $u
   \in R$. So $b_i^n \in (p^n) \Rightarrow b^n \in (p^n)$. Hence $(p^n)$ is also primary.
 \end{solution}

 \begin{exercise}{I.52'}
  Let $\pi$ be an irreducible element in $R$, where $R$ is a UFD. Show that\\
  (1) $(\pi^n)$ is a primary ideal for all $n > 0$, and \\
  (2) any nonzero primary ideal in $(\pi)$ has the form $(\pi^n)$.
 \end{exercise}
 \begin{solution}{siveson@math}
  (1) The argument used in I.52 will work to show this (replacing $p$ by $\pi$), as it only
  used the fact that $p$ was irreducible and the ring was a UFD.\\
  (2) Let $I \subset (\pi)$ be a nonzero primary ideal. Then $I$ is generated by elements
  of the form $a_i\pi$ ($a_i \in R$), so we can write $I = \pi J$ where $J$ is the ideal
  generated by the $a_i$.  Suppose $\pi^n \notin I$ for any $n$. Then because $I$ is
  primary and $a \pi \in I$ for all $a \in J$, each $a \in I$.  Then $J \subseteq I = \pi
  J \subseteq J$, and hence $J = \pi J$. Then inductively, $J = \pi^k J$ for any $k$.
  Since $\pi$ is irreducible and $J = I \neq 0$, this is not possible, because then there
  is a nonzero $j \in J = \pi^k J$, meaning that $j$ contains as many powers of $\pi$ as
  we want when we write it as a product as irreducibles.  Therefore $\pi^n \in I$ for
  some $n$, which we can assume to be of minimal degree (so $\pi^c \notin I$ for $c <n$),
  so $(\pi^n) \subseteq I$.  Supposed there is some $x \in I$ with $x \notin (\pi^n)$.
  Then $x = \pi^m a$, where $a$ does not contain $\pi$ as a factor, and $m < n$. Now
  $\pi^m \notin I$ (otherwise this would contradict the minimality of $n$), so $a^k \in I
  = \pi J$ for some $k$, but then writing $a^k$ as a product of irreducibles, since it is
  in $\pi J$, $a$ must contain $\pi$ as a factor, which is a contradiction. Hence $x \in
  (\pi^n)$ for all $x \in I \Rightarrow (\pi^n) = I$.\\
  Now given any nonzero principal ideal $I$ in a UFD, it is contained in some $(\pi)$, so
  by (1) and (2), it is primary iff it is of the form $(\pi^n)$.
 \end{solution}

 \begin{exercise}{I.53}
  Give an example of a dense ideal that is not a regular ideal.  Which result in the text
  implies that this is impossible in a noetherian ring?
 \end{exercise}
 \begin{solution}{Jonah}
   The ring $R = k[x_1,\ldots,x_n,\ldots]/(x_1^2,x_2^2, \ldots)$ and the maximal ideal $I =
  (x_1, x_2, \ldots)$ is an example.  No elements of I are regular
  because any polynomial is killed by the product of all the $x_i$'s
  that appear in that polynomial.  I is dense because any element of
  the ring only annihilates finitely many of the $x_i$.  This is
  impossible in a noetherian ring because if an ideal is not regular
  then it is contained in the set of zero divisors and therefore is
  contained in a point annihilator (Theorem 6.2).
 \end{solution}

 \begin{exercise}{I.54}
  Does Vasconcelos's Theorem 8.9 hold for non f.g. modules?
 \end{exercise}
 \begin{solution}{mreyes@math}
  Certainly not. \ Indeed, let $R\neq0$ be any ring, and consider the countably
  generated free module $_{R}M=\oplus_{n=1}^{\infty}e_{i}R$. \ Then for
  $f\in\operatorname*{End}\left(  _{R}M\right)  $ given by $e_{1}\mapsto0$ and
  $e_{i}\mapsto e_{i-1}$ for $i>1$, we see that $f$ is surjective but not
  injective. \
 \end{solution}

 \begin{exercise}{I.55}
  For a module $M$ over a noetherian ring $R$, show that $\p \in
  \text{Ass}(M)$ iff $\p$ is a minimal prime over a point annihilator
  of $M$.
 \end{exercise}
 \begin{solution}{David Brown, brownda@math.berkeley.edu}
  If $\p \in \text{Ass}$ then $\p$ is by definition a point
  annihilator, thus minimal over itself. Conversely, let $\p$ be
  minimal over a point annihilator. As $R$ is noetherian, $\p$ is
  finitely generated, so by proposition 3.16, $\p \in \text{Ass}(M)$
  iff $\p_{\p} \in \text{Ass}(M_{\p})$. Now $\p_{\p}$ is minimal over
  some point annihilator $I$. Let $I'$ be a maximal point annihilator
  containing $I$. As $R_{\p}$ is noetherian, $I'$ exists, and by
  Proposition 3.10 $I' \in \text{Ass}(M_{\p})$. But $R_{\p}$ is local
  with maximal ideal $\p_{\p} \supset I'$ which is minimal over $I$,
  so we must have $\p_{\p} = I'$. We conclude that $\p \in
  \text{Ass}(M)$.
 \end{solution}

 \begin{exercise}{I.56}
  For a module $M$ over an Artinian ring $R$, show that $\text{Ass}(M)
  = \text{Supp}(M)$.
 \end{exercise}
 \begin{solution}{David Brown, brownda@math.berkeley.edu}
  Artin $\Rightarrow$ noetherian and 0-dimensional, and that all
  primes are f.g. Thus by Proposition 3.16, a prime $0 \neq \p \in \text{Supp}(M)$ is in
  $\text{Ass}(M)$ iff $\p_{\p} \in \text{Ass}(M_{\p})$. But $A_{\p}$
  is Artinian and local, with unique prime ideal $p_{\p}$, and so if
  a maximal point annihilator exists then it must be $p_{\p}$, which
  by Proposition 3.10 would imply that $\p_{\p} \in
  \text{Ass}(M_{\p})$.

  Suppose that for every $m \in M$, $\text{Ann}(m) \not \subset \p$.
  Then we would have $M_{\p} = 0$, contradicting $\p \in
  \text{Supp}(M)$. Thus we can choose $m$ such that $\text{Ann}(m) \subset \p$.
  Then $\text{Ann}(m/1) \subset \p_{\p}$, and we have a point
  annihilator. We conclude that $\p \in \text{Ass}(M)$.
 \end{solution}

 \begin{exercise}{I.57}
   For modules over a local ring $R$, show that $\mu_R(M\oplus N) = \mu_R(M)+\mu_R(N)$.
   Does this formula hold over semilocal rings?
 \end{exercise}
 \begin{solution}{anton@math}
   If $\{m_i\}$ and $\{n_j\}$ generate $M$ and $N$, respectively, then $\{(m_i,0)\}\cup
   \{(0,n_i)\}$ generates $M\oplus N$, and if $\{x_k\}$ generate $M\oplus N$, then the
   projections onto $M$ (resp.\ $N$) generate $M$ (resp.\ $N$). Therefore,
   \[
   \max\{\mu_R(M),\mu_R(N)\}\le \mu_R(M\oplus N) \le \mu_R(M)+\mu_R(N).
   \]
   If either $\mu_R(M)$ or $\mu_R(N)$ is an infinite cardinal, then these two bounds are
   equal, so we get the result.

   So we may assume that $\mu_R(M)$ and $\mu_R(N)$ are finite. In this case, Nakayama's
   lemma tells us that $\mu_R(M) = \dim_k M/\m M$, where $k=R/\m$ is the residue field of
   $R$. So
   \begin{align*}
    \mu_R(M\oplus N) &= \dim_k (M\oplus N)/\m(M\oplus N) \\
    &= \dim_k M/\m M + \dim_k N/\m N =
    \mu_R(M) + \mu_R(N).
   \end{align*}

   The result doesn't hold over semilocal rings. Let $R$ be $\ZZ$, semilocalized at the
   primes $(2)$ and $(3)$. Then $\mu(\ZZ/3 \oplus \ZZ/2) = \mu(\ZZ/6) = 1$.
 \end{solution}

 \begin{exercise}{I.58}
 Let $R$ be a ring, and $M_i, i \in I$ be a collection of $R$-modules. Is it true
 that $\text{Ass} (\bigoplus_{i \in I} M_{i}) = \bigcup_{i \in I}\text{Ass} (M_{i})$?
 \end{exercise}
 \begin{solution}{David Brown, brownda@math.berkeley.edu}
  \emph{Yes!} For one inclusion, a prime $\p \in M_{i}$ is equivalent
  to the inclusion of $R$-modules
  \[
  R/\p \to M_{i} \to \bigoplus_{i \in I} M_{i},
  \]
  and so $\p \in \text{Ass} \left(\bigoplus_{i \in I} M_{i}\right)$

  Conversely, given a prime $\p \in \text{Ass} \left(\bigoplus_{i \in I}
  M_{i}\right)$, and hence an inclusion
  \[
  R/\p \xrightarrow{i}  \bigoplus_{i \in I} M_{i},
  \]
  consider the image $i(1) = (m_{i})_{i \in I}$. Let $I'$ be the set
  of all $i \in I$ such that $m_{i} \neq 0$. Then $I'$ is finite, and
  $i$ factors through the finite direct sum
  \[
  R/\p \to \bigoplus_{i \in I'} M_{i} \to  \bigoplus_{i \in I} M_{i}.
  \]
  We conclude by $3.8, (2)$ that $\p \in \text{Ass}\left(\bigoplus_{i \in I'}
  M_{i}\right) = \bigcup_{i \in I'}\text{Ass}(M_{i})$.
 \end{solution}

 \begin{exercise}{I.59}
  Show that a f.g. ideal $I = r_1R + \cdots + r_nR$ contained in a
  minimal prime $\p$ of $R$ is not dense.
 \end{exercise}
 \begin{solution}{David Brown, brownda@math.berkeley.edu}
  In the local ring $R_{\p}$, $\p$ is maximal and minimal, so is the
  unique prime. Hence $\nil(R_{\p}) = \p$. Choose $n_i$ such that
  $r_i^{n_i} = 0$ in $R_{\p}$, i.e. there exists an $a_i \not \in \p$
  such that $a_ir_i^{n_i} = 0$ in $R$. Then $a := a_1\ldots a_n \neq 0$,
  because otherwise $a_i \in \p$ for some $i$. Now choose a tuple
  $(m_1,\ldots,m_n)$ maximal w.r.t the property that $b := a \cdot r_1^{m_1}
  \ldots r_n^{m_n} \neq 0$. Then for each $i$, $b\cdot r_i = 0$.
  As $b$ is non-zero by construction we conclude that $I$ is not
  dense.
 \end{solution}

 \begin{exercise}{I.60}
   (For philosophical discussions.) In teaching mathematical induction in an introductory
   course in abstract algebra, it is incumbent upon an instructor to stress the need to
   ``begin the induction''; that is, to check the case $n=1$. (``Otherwise, you can prove
   $n=n+1$.'') In the Noetherian Induction Principle (4.9), what happened to the ``beginning
   of the induction''?
 \end{exercise}
 \begin{solution}{los@math}
   {\bf Noetherian Induction Principle 4.9.} {\em Let $R$ be a noetherian ring, and $\P$
   be a certain property that we want to test on a family $\F$ of ideals in $R$. Suppose
   it is known that an ideal $I \in \F$ satisfies $\P$ whenever all ideals in $\F$ properly
   containing $I$ satisfy $\P$. Then all ideals in $\F$ satisfy $\P$.}

   \noindent
   Let $\F'$ be the family of all ideals of $R$, and let $\P'$ denote the following
   property for ideals $I \in \F'$. An ideal $I$ satisfies $\P'$ if either: $I$ belongs to
   $\F$ and satisfies $\P$; or $I$ does not belong to $\F$. Then the statement above is
   equivalent to the same statement with $\F$ and $\P$ replaced by $\F'$ and $\P'$,
   respectively. Under the hypotheses stated above, the ideal $R \in \F'$ necessarily satisfies $\P'$,
   since there are no ideals properly containing it. The induction may be considered to
   ``begin'' with $R$ in this sense.

   \noindent
   {\it Note:} The fact that in the above statement $\F$ is called
   a ``family'' while $\P$ is called a ``property'' obscures the fact that the only relevant set
   is the subset of $\F'$ consisting of those elements of $\F$ which do not satisfy $\P$, i.e.,
   the set of elements of $\F'$ not satisfying $\P'$.
 \end{solution}

 \begin{exercise}{I.61}
  Let $Q \subsetneq M$ be $\p$-primary. Show that, for any $r \in R \setminus
  \text{ann}(M/Q)$, $Q:r = \{m \in M : rm \in Q \}$ is also $\p$-primary.
 \end{exercise}

 \begin{solution}{siveson@math}
  First note that $Q:r \subsetneq M$ since $r \in R \setminus
  \text{ann}(M/Q)$ so $rx \notin Q$ for some $x \in M$.\\
  Now we first show that $Q:r$ is primary. Let $a \in R$ and $x \in M \setminus (Q:r)$
  (so $rx \in M \setminus Q$), and suppose that $ax \in Q:r$, meaning $arx \in Q$.  Since
  $rx \in M \setminus Q$ and $arx \in Q$, since $Q$ is primary, $a^nM \subseteq Q
  \subseteq Q:r$ for some $n$. Hence $Q:r$ is primary.\\
  Now we just need to show that $\sqrt{\text{ann}(M/(Q:r))} = \sqrt{\text{ann}(M/Q)} =
  \p$. First of all, if $a \in R$ and $a^nM \subseteq Q$ (meaning $a \in
  \sqrt{\text{ann}(M/Q)}$), then since $Q \subseteq Q:r$, we have $a^nM \subseteq Q:r$,
  so we have containment in one direction: $\sqrt{\text{ann}(M/(Q:r))} \supseteq
  \sqrt{\text{ann}(M/Q)}$. Now if $a \in R$ and $a^nM \subseteq Q:r$ for some $n$
  (meaning $a \in \sqrt{\text{ann}(M/(Q:r))}$), then $a^nrm \in Q$ for all $m \in M$. But
  there is some $x \in M \setminus (Q:r)$, so $rx \notin Q$ but $a^nrx \in Q$. Then since
  $Q$ is primary, $(a^n)^mM \subseteq Q$ for some $m$, meaning $a \in
  \sqrt{\text{ann}(M/Q)}$. This gives us containment in the other direction, so we get
  equality $\sqrt{\text{ann}(M/(Q:r))} = \sqrt{\text{ann}(M/Q)} = \p$, so $Q:r$ is also
  $\p$-primary.
 \end{solution}

 \begin{exercise}{I.62}
  Given $\mathfrak{q}\subseteq\m\in\Max(R)$ in a noetherian ring $R$, show that the
  following are equivalent:

  (1) $\mathfrak{q}$ is $\m$-primary.

  (2) $\mathfrak{q}\supseteq\m^n$ for some $n$.

  (3) $R/\mathfrak{q}$ is local artinian.
 \end{exercise}

 \begin{solution}{ecarter@math}
  $(2)\implies (1)$: Let $a,b\in R$ such that $ab\in\mathfrak{q}$ and
  $b\notin\mathfrak{q}$. Then $b\notin\m^n$.  Since $\m$ is prime, $b\notin\m$.  Then
  $ab\in\m$ but $b\notin\m$, so $a\in\m$.  Therefore $a^n\in\mathfrak{q}$, so
  $\mathfrak{q}$ is primary.  Since $\m\subseteq\sqrt{\mathfrak{q}}$ and $\m$ is already
  maximal, $\m=\sqrt{\mathfrak{q}}$, so $\mathfrak{q}$ is $\m$-primary.

  $(1)\implies (3)$: Since $\sqrt{\mathfrak{q}}=\m$ is maximal,
  $R/\mathfrak{q}$ has a unique
  prime.  Since $R/\mathfrak{q}$ is also noetherian, it must be local
  artinian.

  $(3)\implies (2)$: Since $R/\mathfrak{q}$ is local artinian, it has a unique prime,
  $\m$.  In a noetherian ring, any ideal contains a finite product of prime ideals, so
  $\m^n$ must be equal to the zero ideal in $R/\mathfrak{q}$ for some $n$.  Therefore
  $\m^n\subseteq\mathfrak{q}$.
 \end{solution}

 \begin{exercise}{I.63}
     For an artinian ring $R$, let $(0)=\mathfrak{q}_1\cap \ldots\cap \mathfrak{q_n}$
     be a MPD. Show that this gives a natural ring isomorphism $R\cong \prod_{i=1}^n
     R/\mathfrak{q}_i$, where each $R/\mathfrak{q}_i$ is local artinian.
 \end{exercise}
 \begin{solution}{Lars Kindler, lars\_k@berkeley.edu}
     Let $\mathfrak{m}_i$ denote $\sqrt{\ann(R/\mathfrak{q}_i)}$ so that each of the
     $\mathfrak{q}_i$ is $\mathfrak{m}_i$-primary. Then the
     Lasker-Noether Theorem tells us that
     $\ass(R/(0))=\ass(R)=\{\mathfrak{m}_1,\ldots,\mathfrak{m}_n\}$ and
     $\dim R = 0$ implies that the $\mathfrak{m}_i$ are maximal ideals and thus
     isolated, so the  $\mathfrak{q}_i$ are uniquely determined.\\
     Now $R$ is noetherian and rad-nil, so $\rad(R)$ is nilpotent, i.e. there is a $t$
     such that we have the decomposition
     $0=(\mathfrak{m}_1\cap\ldots\cap\mathfrak{m}_k)^t=\mathfrak{m}^t_1\cdot\ldots\cdot\mathfrak{m}_k^t=\mathfrak{m}_1^t\cap
     \ldots\cap\mathfrak{m}_k^t$. Now $\mathfrak{m}_i^t$ is $\mathfrak{m}_i$-primary
     and since we have seen that $\ass(R)=\{\mathfrak{m}_1,\ldots,\mathfrak{m}_n\}$,
     this decomposition must be a MPD by the Lasker-Noether theorem, hence we have $\mathfrak{q}_i = \mathfrak{m}_i^t$ and by the
     Chinese Remainder Theorem $R\cong \prod_{i=1}^n R/\mathfrak{q}_i$, with
     $R/\mathfrak{q}_i=R/\mathfrak{m}_i^t$ local artinian.
 \end{solution}

 \begin{exercise}{I.64}
  Show that a power of a prime ideal need not be primary.  (Hint. Let
  $R=k[x,y,z]/(xy-z^2)$, and consider the prime ideal $(\overline{x}, \overline{z})$.)
 \end{exercise}
 \begin{solution}{ecarter@math}
  Since $R/(\overline{x},\overline{z})\cong k[x,y,z]/(xy-z^2,x,z)\cong k[y]$ is an
  integral domain, $(\overline{x},\overline{z})$ is prime.  However,
  \[
     R/(\overline{x}^2, \overline{x}\overline{z}, \overline{z}^2)
             \cong k[x,y,z]/(xy-z^2, x^2, xz, z^2) = k[x,y,z]/(xy,x^2, xz, z^2).
  \]
  In this ring, $y$ is a zero-divisor which is not nilpotent.  Therefore
  $(\overline{x}^2, \overline{x}\overline{z},
  \overline{z}^2)=(\overline{x},\overline{z})^2$ is not primary.
 \end{solution}

 \begin{exercise}{I.65}
   If an ideal $\q \< R$ is $\p$-primary, show that $\q[x]$ is $\p[x]$-primary in the polynomial
   ring $R[x]$.
 \end{exercise}
 \begin{solution}{los@math}
   First, note that $\p[x]$ is indeed a prime ideal, since it is the kernel of the projection
   morphism $R[x] \rightarrow (R/\p)[x]$, where $(R/\p)[x]$ is a domain because $R/\p$ is.
   We may assume $\q=0$, since $R[x]/\q[x] \cong (R/\q)[x]$ and $\p[x]/\q[x]$
   corresponds to $(\p/\q)[x]$ under this isomorphism. Thus we are reduced to showing that if a
   ring $R$ satisfies $\nil(R)=\p=\Z(R)$ for a prime ideal $\p$, then $\nil(R[x])=\p[x]=\Z(R[x])$.
   That $\nil(R[x])=\p[x]$ is clear from Exercise 1.2 and the equality $\nil(R)=\p$. This also shows
   $\p[x] = \nil(R[x]) \subseteq \Z(R[x])$. Conversely, let $f \in \Z(R[x])$. Theorem 1.2 shows
   that all the coefficients of $f$ are in $\Z(R)=\p$, so $f \in \p[x]$. Hence $\Z(R[x]) \subseteq
   \p[x]$ and we are done.
 \end{solution}

 \begin{exercise}{I.66} Use Exer. 65 to show that, for the prime ideal $\p = (x_1, \ldots ,x_m)
   \< k[x_1, \ldots ,x_n]$ (where $k$ is a field and $1 \leq m \leq n$), every power $\p^r$ $(r>0)$
   is $\p$-primary.
 \end{exercise}
 \begin{solution}{los@math}
   It follows easily by induction from Exercise 65 that,
   under the same hypotheses,
   for any $s \geq 0$, the ideal $\q[x_1,\ldots,x_s]$ is $\p[x_1,\ldots,x_s]$-primary
   in the ring $R[x_1,\ldots,x_s]$. In Exercise 66, let $R=k[x_1,\ldots,x_m]$, and let $\m$ be the
   maximal ideal of $R$ generated by $x_1,\ldots,x_m$. Then $\m^r$ is $\m$-primary by Remark (7.4)(2).
   Identifying $k[x_1,\ldots,x_n]$ with $R[x_{m+1},\ldots,x_n]$, we clearly have
   $\p=\m[x_{m+1},\ldots,x_n]$
   and $\p^r=\m^r[x_{m+1},\ldots,x_n]$. Applying the previous remark with $s=n-m$, it follows that
   $\p^r$ is $\p$-primary.
 \end{solution}

 \begin{exercise}{I.67}
   Show that, in the noetherian ring $k[x_1, \ldots , x_n]$ (where $k$ is a field and $n \geq 2$),
   the primary ideal $(x_1,x_2)^2$ is not irreducible.
 \end{exercise}
 \begin{solution}{los@math}
   Let $R=k[x_3,\ldots,x_n]$ and let $I$, $J$ and $K$ be the ideals $(x_1,x_2)^2$, $(x_1,x_2^2)$
   and $(x_1^2,x_2)$ of $R[x_1,x_2]$, respectively. We also have $I=(x_1^2,x_1x_2,x_2^2)$.
   Elements $f$ of the ideal $I$ (resp. $J$, resp. $K$) are characterized by the following property:
   for every monomial $a_{{i_1}{i_2}}x_1^{i_1}x_2^{i_2}$ appearing in $f$ (viewed as an element
   of $R[x_1,x_2]$), we have $i_1 \geq 2$; $i_1 \geq 1$ and $i_2 \geq 1$; or $i_2 \geq 2$ (resp.
   $i_1 \geq 1$ or $i_2 \geq 2$, resp. $i_1 \geq 2$ or $i_2 \geq 1$). From these considerations it
   is clear that $I \neq J$, $I \neq K$ and $I = J \cap K$, hence $I$ is not irreducible.
 \end{solution}

 \begin{exercise}{I.68}
   Let $\p = (x,y) \subseteq k[x,y,z]$, where $k$ is a field. Then $\q = (y^2, x+yz)$ and
   $\q' = (y^2, -x+yz)$ are $\p$-primary by (7.4)(5). Show that $\q+\q'$ is not primary.
 \end{exercise}
 \begin{solution}{los@math}
   {\it Note:} It must be assumed that ${\rm char}(k) \neq 2$ in order for the conclusion to hold.
   We will make this assumption.

   \noindent
   We have $\q + \q' = (y^2,x+yz,-x+yz) = (y^2,2yz,x-yz) = (y^2,yz,x-yz) = (y^2,yz,x)$, the
   third equality holding because 2 is invertible in $k$ by assumption. Thus
   $k[x,y,z]/(\q+\q') \cong k[y,z]/(y^2,yz)$. By Example (7.4)(4), we see that $\q+\q'$ is not
   primary.
 \end{solution}

 \begin{exercise}{I.69}
   Prove the following module-theoretic analogue of the first part of (7.4)(2): if $Q \subsetneq
   {}_RM$ is such that $\sqrt{\ann(M/Q)} = \m \in \Max(R)$, then $Q$ is $\m$-primary.
 \end{exercise}
 \begin{solution}{los@math}
   {\it Note:} The ideal $\m$ wasn't originally defined in the statement of the exercise. Also, the
   assumption $Q \neq M$ is superfluous.

   \noindent
   First we make a general remark about primary submodules. If
   $I$ is an ideal of $R$ and $N \subseteq M$ are $R/I$-modules,
   we can also view $M$ and $N$ as $R$-modules. It is obvious from the
   definitions that if $\p$ is any prime ideal of $R$ containing $I$,
   then $N$ is a $\p$-primary submodule of ${}_RM$ if and only if $N$
   is $\p/I$-primary as a submodule of ${}_{R/I}M$. Now in the situation
   of the exercise, we may assume $Q=0$, replacing $M$ with $M/Q$.
   Furthermore, by the foregoing argument, we may without loss of
   generality replace $R$ with $R/{\ann(M)}$ and hence assume $\ann(M)=0$.
   In particular, $M \neq 0$.
   We now have by assumption $\nil(R)=\m \in \Max(R)$. In other words,
   $R$ is a rad-nil local ring with maximal ideal $\m$. In order to
   show that $0$ is $\m$-primary in $M$, we need only show $\Z(M)=\m$.
   Because $M \neq 0$, we have $\nil(R) \subseteq \Z(M)$, which proves the
   inclusion $\m \subseteq \Z(M)$. Conversely, let $a \in \Z(M)$. The element
   $a$ cannot be invertible in $R$, for otherwise multiplication by $a$ would be an
   automorphism of $M$, contradicting $a \in \Z(M)$. But $R$ is a local ring, so in fact
   $a \in \m$. Thus $\Z(M) \subseteq \m$, which proves that $0$ is $\m$-primary in $M$.
 \end{solution}

 \begin{exercise}{I.70}
     Show that a prime ideal is irreducible, and an irreducible radical ideal is prime.
 \end{exercise}
 \begin{solution}{Lars Kindler, lars\_k@berkeley.edu}
     Let $R$ be a ring and $\p\lhd R$ a prime ideal. Assume
     $\p=I\cap J$, where $I,J\lhd R$ and $I\neq \p\neq J$. Pick an
     element $x\in J\setminus \p$, then for all $a\in I$ we have $ax\in
     I\cap J=\p$ and since $\p$ is prime this implies
     $I=\p$ which is a contradiction.\\
     An irreducible radical ideal $I$ is the intersection of all primes containing it,
     i.e. $I=\bigcap_{\alpha\in \mathfrak{A}} \mathfrak{p}_\alpha$, for some index set
     $\mathfrak{A}$. Now let $b\not\in I$ and define
     $\mathfrak{A}_1:=\{\alpha\in\mathfrak{A}|b\in \mathfrak{p}_\alpha\}\subsetneqq
     \mathfrak{A}$ and $\mathfrak{A}_2=\mathfrak{A}\setminus\mathfrak{A}_1\neq
     \varnothing$. Then we have $I=\bigcap_{\alpha\in
     \mathfrak{A}_1}\mathfrak{p}_\alpha \cap \bigcap_{\alpha\in
     \mathfrak{A}_2}\mathfrak{p}_\alpha$, so since $I$ is irreducible and $b\not\in I$
     this implies $I=\bigcap_{\alpha\in \mathfrak{A}_2}\mathfrak{p}_\alpha$. But then
     $ab\in I$ implies $a\in I$ and thus $I$ is prime.
 \end{solution}

 \begin{solution}{lam@math}
  \textit{Another Solution to 2nd Part:}

  \medskip\noindent
  Suppose there exist $a,b\notin I$ with $ab\in I$. We claim $I\subseteq
  (I+(a))\cap (I+(b))$ is an equality (which would show $I$ is not irreducible). Let
  $x=i+ar=j+bs$, where $i,j\in I$. Then $ai+a^2r=aj+abs\Rightarrow a^2r\in I\Rightarrow
  ar\in I \Rightarrow x\in I$.
 \end{solution}

 \begin{exercise}{I.71}
  Let $A$ be a UFD with prime ideal $\p$ containing an irreducible element $\pi$ such
  that $(\pi) \subsetneq \p$.  Let $M$ be the $A$-module $\bigoplus A/(p)$, where $(p)$
  ranges over principal ideas of $A$ generated by irreducible elements $p$ such that $(p)
  \neq (\pi)$. For the split-null extension $R := A \oplus M$ constructed in (5.3)(B), show
  that
  \smallskip
  \noindent (1) $\C(R) = \{(u \pi^k,m) : u\in \text{U}(A), \; k\geq0 ; \; m \in M\}$.\\
  (2) The ideal $\p \oplus M \< R$ is regular, but is \textit{not} generated by regular
  elements.\\
  (Thus, $R$ is not a Marot ring.)
 \end{exercise}
 \begin{solution}{Jonah (jblasiak@math) and Anton}
  Since $A$ is a UFD the only way the product of two elements can be zero is if one of the
  $A$ summands of one of the elements is 0.  Now simply compute the following product for
  an arbitrary nonzero element $a = u \pi_1^{r_1} \pi_2^{r_2} \ldots \pi_k^{r_k}$ of $A$
  (where $u \in \text{U}(A)$ and the $\pi_i$ are irreducible), and $m_1, m_2 \in M$: $(a,
  m_1) \cdot (0, m_2) = (0, a \; m_2)$.  So $(a, m_1)$ is a zero divisor iff $a$
  annihilates some nonzero element of $M$ iff $\pi_i \neq \pi$ for some $i$.  This proves
  (1).

  The ideal $I = \p \oplus M$ is regular since it contains $(\pi, 0)$.  It is easy to see from (1) that the regular elements of $I$ generate the subideal $(\pi) \oplus M$.  This is an ideal and is proper since the $A$ summand of a product of two elements is determined only by the $A$ summands of the elements: for any $x \in \p \backslash (\pi)$ $(x, 0) \in \p \oplus M$ and $(x, 0) \notin (\pi) \oplus M$ .
 \end{solution}


 \closeout0
\end{document}
