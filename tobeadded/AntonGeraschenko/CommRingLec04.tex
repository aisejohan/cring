 \stepcounter{lecture}
 \setcounter{lecture}{4}
 \sektion{Lecture 4}

 \subsektion{\S 3 Associated Primes of Modules}

 For any module ${}_R M$, you can consider
 \begin{enumerate}
   \item $\ann M\< R$, the annihilator of $M$.

   \item $\Z(M) = \{r\in R|rm=0\text{ for some nonzero }m\in M\}$ (defined for $M\neq
       0$). This is no longer an ideal of $R$, buy it contains $\ann M$.

   \item $\ass M = \{\p\in \spec R| \p = \ann(m)\text{ for some non-zero }m\in M\}$. We
   call $\ann (m)$ a \emph{point annihilator}. Clearly any such $\p$ contains $\ann M$.

   \item $\supp M = \{\p\in \spec R| M_\p\neq 0\}$.
 \end{enumerate}
 A very important case is when $M=R/I$ is cyclic (where $I\< R$). In this case, $\ann R/I
 = I$, $\Z(R/I) = \{r\in R|rm\in I\text{ for some }m\in R\smallsetminus I\}$.

 \begin{lemma}
   For $\p\in \spec R$, and suppose you are given ${}_R M$, then $\p\in \ass M$ if and
   only if $R/\p\hookrightarrow M$.\mpar[How to deal with $\ass M$]{}
 \end{lemma}
 \begin{proof}
   If $f:R/\p\hookrightarrow M$ then $\p = \ann (f(1))$. If $\p=\ann(x)$, then $f:R\to M$
   defined by $f(1)=x$ has kernel $\p$.
 \end{proof}
 \begin{lemma}\label{lec04L:assinclusion}
   If $N\subseteq M$, then $\ass N\subseteq \ass M$.
 \end{lemma}
 \begin{proof}
   $\p\in \ass N\Rightarrow \p=\ann(x)\Rightarrow \p\in \ass M$.
 \end{proof}
 \begin{lemma}[Herstein]
   Any maximal point annihilator of $M$ is an associated prime.
 \end{lemma}
 \begin{proof}
   Let $\ann(x)$ be a maximal point annihilator. To check that $\ann(x)$ is prime,
   assume $ab\in \ann(x)$ and $b\not\in \ann(x)$. Then $bx\neq 0$, so $\ann(x)=
   \ann(bx)$ by maximality. But $abx=0$, so $a\in \ann(bx)=\ann(x)$.
 \end{proof}
   Note that if $M$ is not noetherian, there may not be any maximal point annihilators, so
   this does not prove existence of an associated prime.
 \begin{example}
   Here an a non-zero module $M$ with $\ass M=\varnothing$. Take suitable $R$, and let
   $M = {}_R R$. Take $R = \QQ[x_1,x_2,\dots]/(x_i^2)_{i=1,2,\dots}$. In this ring,
   there is a unique prime $\p = (x_1,x_2,\dots)$. In particular, $R$ is a local ring.
   If $\ass R\neq \varnothing$, then $\p=\ann (m)$ for some nonzero $m$. But then $m$
   kills every $x_i$. However, if any polynomial annihilates $\p$, write it so that all
   terms are square-free, but then take $x_N$ with $N$ larger than any indices appearing
   in $m$, and $mx_N\neq 0$.
 \end{example}
 \begin{example}
    Let $R=\ZZ$. Then $\ass(\ZZ) = \{(0)\}$, $\ass(\QQ) = \{(0)\}$, $\ass(\ZZ\oplus
    \ZZ/60)=\{(0),(2),(3),(5)\}$, and $\ass(\QQ/\ZZ)=\Max \ZZ = \spec \ZZ\smallsetminus
    \{(0)\}$.
 \end{example}
 For every $M$, $\Z(M)$ is a union of prime ideals. In general, there is an easy
 characterization of sets $Z$ which are a union of primes: it is exactly when
 $R\smallsetminus Z$ is a \emph{saturated multiplicative set}. This is Kaplansky's
 Theorem 2.
 \begin{definition}
   A multiplicative set $S\neq \varnothing$ is a \emph{saturated multiplicative set} if
   for all $a,b\in R$, $a,b\in S$ if and only if $ab\in S$. (``multiplicative set'' just
   means the ``if'' part)
 \end{definition}
 To see that $\Z(M)$ is a union of primes, just verify that its complement is a saturated
 multiplicative set.

 Let $M = R/I$. Then $\ass (R/I)$ is often called ``the set of primes associated to
 $I$''. For any set $S$ and ideal $I\< R$, we define $I:S = \{r\in R|r\cdot S\subseteq
 I\}$. Of course, you can replace $S$ by the ideal generated by $S$. Then $\ass R/I$
 consists of prime ideals of the form $I:x$.

 \begin{example}
   Take $R=k[x,y,z]$, where $k$ is an integral domain, and let $I = (x^2-yz,x(z-1))$. Any
   prime associated to $I$ must contain $I$, so let's consider
   $\p=(x^2-yz,z-1)=(x^2-y,z-1)$, which is $I:x$. It is prime because $R/\p = k[x]$,
   which is a domain. To see that $I:x\subseteq \p$, assume $tx\in I\subseteq \p$; since
   $x\not\in \p$, $t\in p$, as desired.

   There are two more associated primes, but we will not find them here.
 \end{example}

 \begin{proposition}
   \begin{enumerate}
     \item[]
     \item \label{lec04:AssFact1} If $N\subseteq M$, then
        $\ass M \subseteq \ass N \cup \ass M/N$
     \item \label{lec04:AssFact2} If $M = \bigoplus_{i=1}^n M_i$, then
        $\ass M = \bigcup_{i=1}^n \ass M_i$.
   \end{enumerate}
 \end{proposition}
 \begin{proof}
   (\ref{lec04:AssFact1}) \underline{Observation}: If $N\subseteq R/\p$ is a nonzero
   submodule, then $\ass N=\{\p\}$. To see this, take $x\in R\smallsetminus \p$; then
   $\p:x=\p$ by primeness of $\p$.

   Now if $\p\in \ass M$, we have $R/\p\hookrightarrow M$; call the image $Y$. If $N\cap
   Y=0$, then $Y\cong R/\p\hookrightarrow M/N$, so $\p\in \ass (M/N)$. Otherwise, $N\cap
   Y\subseteq Y$ is a nonzero submodule, so $\{\p\} \in \ass (N\cap Y)\subseteq \ass N$.

   (\ref{lec04:AssFact2}) The containment $\subseteq$ follows from
   (\ref{lec04:AssFact1}), and the containment $\supseteq$ follows from Lemma
   \ref{lec04L:assinclusion}.
 \end{proof}
