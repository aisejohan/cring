 \stepcounter{lecture}
 \setcounter{lecture}{27}
 \sektion{Lecture 27}

 Do some (two) exercises (from chapter II)!

 Then we can form the ``complete integral closure of $R$ in $S$''.
 \begin{proposition}
   If $s_1,\dots, s_n\in S$ are almost integral over $R$, then $R[s_1,\dots, s_n]$ lies
   in some finitely generated $R$-module in $S$. In particular, the complete integral
   closure of $R$ in $S$, $C:=\{s\in S| s\text{ almost integral over }R\}$, is a subring
   of $S$. If $C=R$, we say $R$ is completely integrally closed in $S$.
 \end{proposition}
 \begin{definition}
   If $S=Q(R)$, then we denote the integral closure of $R$ in $S$ by $R^*$ and the
   complete integral closure of $R$ in $S$ by $R^\dag$. We call $R^\dag$ the
   \emph{complete integral closure} of $R$. If $R=R^\dag$, we say that $R$ is
   \emph{completely integrally closed}.
 \end{definition}
 \begin{remark}
   Note that if $R$ is completely integrally closed, then it is integrally closed since
   $R\subseteq R^*\subseteq R^\dag$. If $R$ is noetherian, $R^\dag=R^*$. Note that
   $R^\dag=\{s\in Q(R)|\text{there is some }d\in \C(R) \text{ such that }ds^i\in R\text{
   for all }i\}$.
 \end{remark}

 If $R$ is a completely integrally closed domain, it is called a \emph{completely normal
 domain}.
 \begin{proposition}
   If $R$ is a UFD, then it is a completely normal domain.
 \end{proposition}
 \begin{proof}
   Let $K=Q(R)$, and assume $a/b\in K\smallsetminus R$ is almost integral over $R$. We
   may assume that $a$ and $b$ have no common prime factors, and that there is some prime
   $\pi$ dividing $b$ (lest $a/b\in R$). Choose a non-zero $d\in R$ so that $da^i/b^i\in
   R$ for all $i\ge 0$, so $d a^i=b^i$. It follows that $\pi^i|d$ for all $i$,
   contradicting unique (finite) factorization of $d$.
 \end{proof}
 Now let's pursue the completely normal analogues of the results on $R[x]$ when $R$ is
 normal.
 \begin{proposition}
   Let $C$ be the complete integral closure of $R$ in $S$, then $C[x]$ is the complete
   integral closure of $R[x]$ in $S[x]$.
   \marginpar{\hspace*{-2ex}$\xymatrix@C=1pc{ S \ar@{-}[d] & S[x] \\ C \ar@{-}[d]
   & C[x] \\ R & R[x] }$}
 \end{proposition}
 \begin{proof}
   Elements of $C[x]$ are clearly almost integral over $R[x]$ since $C$ and $x$ are
   almost integral over $R[x]$ (and almost integral elements form a ring). If $f(x)=\sum
   s_j x^j\in S[x]$ is almost integral over $R[x]$, then we'd like to show that each
   $s_j$ is almost integral over $R$. Fix $g_1,\dots, g_m\in S[x]$ such that $f(x)^i\in
   \sum_{k=1}^m R[x]\cdot g_k$ for all $i$. The leading term of $f(x)^i$ is $s_n^i
   x^{n\cdot i}$. It follows that all powers of $s_n$ lie in the $R$-module generated by
   all the coefficients of the $g_k$, so it is almost integral over $R$, so they are in
   $C$. Then $f-s_nx^n$ is still almost integral over $R[x]$. Inducting on degree, we get
   that $f\in C[x]$
 \end{proof}
 \begin{corollary}
   A domain $R$ is completely normal if and only if $R[x]$ is.
 \end{corollary}
 The proof is the same as before. In fact, we get a second (less tricky) proof of this
 result with the word ``completely'' removed using noetherian descent.
 \[\xymatrix{
 & S & & S[x]\\
 & C & & C[x]\\
 & R & & R[x]\\
 R_0 \ar@{-}[ruuu]|{C_0} & & R_0[x] \ar@{-}[ruuu]|{C_0[x]}
 }\]
 Take $f\in S[x]$ integral over $R[x]$, then it satisfies some monic polynomial. Let
 $R_0$ be the ring generated over $\ZZ\cdot 1$ by all the coefficients involved
 everywhere \dots this ring is noetherian, so the word ``completely'' can be removed. We
 get that $f\in C_0[x]\subseteq C[x]$.

 \medskip
 Power series case: $R[x]\rightsquigarrow R[[x]]$. If $R$ is completely normal, then so
 is $R[[x]]$ (this fails for normality!).
 \begin{theorem}[Hilbert basis theorem for power series]
   If $R$ is noetherian, then so is $A:=R[[x]]$.
 \end{theorem}
 \begin{proof}
   By Cohen's theorem, it suffices to prove that any $\p\in \spec A$ is finitely
   generated. Let $I\< R$ be the ideal of all constant terms of elements in $\p$. Since
   $R$ is noetherian, $I$ is finitely generated, say $I=\sum_{i=1}^n a_i R$.

   \underline{Case 1}: If $x\in \p$, then $\p=I\cdot A + x\cdot A$, so $\p$ is
   generated by $n+1$ elements.

   \underline{Case 2}: If $x\not\in \p$, choose $f_i\in \p$ so that $f_1(0)=a_i$. We
   claim that $\p$ is generated by the $f_i$. Let $f\in \p$, then we can choose $b_i$ so
   that $f-\sum b_i f_i = x\cdot g$ for some $g\in A$. Since $x\not\in \p$, $g\in \p$.
   Repeating the argument for $g$ and inducting, we get $f=\sum
   (b_i+xc_i+x^2d_i+\cdots)f_i$.
 \end{proof}
 \begin{theorem}
   Let $R$ be a domain, with quotient field $K$ and $A=R[[x]]$. Then
   \begin{enumerate}
     \item If $A$ is normal, then $R$ is normal.
     \item $A$ is completely normal if and only if $R$ is completely normal
     \item $A$ is noetherian and normal if and only if $R$ is noetherian and normal.
   \end{enumerate}
 \end{theorem}
 \begin{proof}
   (3) follows from (2) and the Hilbert basis theorem for power series. The direction
   $\Leftarrow$ in (1) and (2) are done as before (in the case $A=R[x]$). It remains to
   prove that if $R$ is completely normal, then so is $A$.
   \[\xymatrix{
    & K((x)) \ar@{-}[dr] \ar@{-}[d]\\
    K\ar@{-}[d] & K[[x]]\ar@{-}[d] & Q(A) \ar@{-}[d]\\
    R & R[[x]] \ar@{}[r]|{=}& A
   }\]
   Assume $f\in Q(A)$ is almost integral over $A$. Then as an element of $K((x))$, $f$ is
   almost integral over $K[[x]]$, which is a PID (so a UFD), so it is completely normal.
   It follows that $f\in K[[x]]$. Say $f=a_0 + a_1x+ \cdots$. We want to show that each
   $a_i$ is in $R$. By almost integrality, there is some $h\in A\smallsetminus \{0\}$ so
   that $h\cdot f^i\in A$ for all $i\ge 0$. We will show by induction that $a_j\in R$.
   Suppose $a_0,\dots, a_j\in R$. Then $f'=a_0+\cdots a_{j-1}x^{j-1}\in A$, and $h\cdot
   (f-f')^i \in A$ for all $i\ge i$. Suppose $h=dx^m+ \cdots$, with $d\in R$ non-zero.
   Now we have
   \[
    h(f-f')^i = da_j^i x^{m+ij} + \cdots \in A
   \]
   so $da_j^i\in R$ for all $i$. It follows that $a_j$ is almost integral over $R$.
 \end{proof}
  \begin{corollary}
   If $k$ is a field, then $A_n=k[[x_1,\dots, x_n]]$ is noetherian and normal.
 \end{corollary}
 In fact, Weierstra\ss\ showed that $A_n$ is even a UFD.
