 \stepcounter{lecture}
 \setcounter{lecture}{19}
 \sektion{Lecture 19}

 $\spec R$ is connected if and only if $R$ has only trivial idempotents. All of the
 closed and open (\emph{clopen}) sets of $\spec R$ are $\V(e)$, where $e$ is an
 idempotent.

 \begin{proposition}
   For any ring $R$, the following are equivalent.
   \begin{enumerate}
     \item $\dim R=0$.
     \item $\spec R$ is $T_1$ (points are closed).
     \item $\spec R$ is $T_2$ (hausdorff).
     \item $\spec R$ is a Boolean space (compact, hausdorff, and totally disconnected).
     \item $\spec R$ is $T_4$ (normal).
   \end{enumerate}
 \end{proposition}
 \begin{example}
   $\spec \ZZ = \{(0),(2),(3),(5),(7),\dots\}$. For $n\neq 0$, $\V\bigl((n)\bigr) = \{p|
   p$ divides $n\}$, which is finite; it is clear that any finite set (not containing
   $(0)$) can be realized this way. So the non-empty open sets are the cofinite sets
   contining $(0)$.
 \end{example}
 \begin{example}
   Let $R$ be the semi-localization of $\ZZ$ at $\{p_1,\dots, p_r\}$. Now the non-empty
   open sets are all the subsets containing $(0)$.
 \end{example}
 \begin{example}
   If $R$ is a PID, then prime ideals are generated by irreducible elements. The
   non-empty open sets are still the cofinite sets containing $(0)$.
 \end{example}
 \begin{definition}
   For a subset $Y\subseteq \spec R$, we define $\I(Y):=\bigcap_{\p\in Y}\p$, which is a
   radical ideal in $R$.
 \end{definition}
 \begin{proposition}
   Here are some fairly easy results.
   \begin{enumerate}
     \item If $J\< R$, $\I\bigl(\V(J)\bigr) = \sqrt J$.
     \item For $Y\subseteq \spec R$, $\V\bigl(\I(Y)\bigr)=\bbar Y$. In particular,
     $\bbar{\{\p\}}=\V(\p)$.
     \item $\p$ is a closed point if and only if it is a maximal ideal.
   \end{enumerate}
 \end{proposition}
 That is, we have an inclusion-reversing bijection between radical ideals and closed
 sets. Note that $\spec R$ is noetherian if and only if radical ideals satisfy ACC. In
 this case, the set of minimal primes is finite. Observe that if $R$ is noetherian, then
 so is $\spec R$, but the converse is false.
 \[\xymatrix @R=1.5pc{
 \left\{\raisebox{4pt}{\txt{radical\\ ideals}}\right\}\ar@/^/[r]^\V \ar@{<-}@/_/[r]_\I
 & \left\{\raisebox{4pt}{\txt{closed\\ sets}}\right\}\\
 \left\{\raisebox{4pt}{\txt{prime\\ ideals}}\right\}\ar@/^/[r]^\V \ar@{<-}@/_/[r]_\I \ar@{}[u]|{\cup \rule{.3pt}{5pt}}
 & \left\{\raisebox{4pt}{\txt{irreducible\\ closed sets}}\right\} \ar@{}[u]|{\cup \rule{.3pt}{5pt}}\\
 \left\{\raisebox{3pt}{\txt{minimal\\ primes}}\right\}\ar@/^/[r]^\V \ar@{<-}@/_/[r]_\I \ar@{}[u]|{\cup \rule{.3pt}{5pt}}&
 \left\{\raisebox{3pt}{\txt{irreducible\\ components}}\right\} \ar@{}[u]|{\cup \rule{.3pt}{5pt}}
 }\]

 \subsektion{\S 4 Hilbert's Nullstellensatz}
 Fix a field $k$.
 \begin{definition}
   If $M$ is an $R$-module, we say that $M$ is \emph{module-finite} if $M$ is finitely
   generated as a module. If $S$ is an $R$-algebra, we say it is \emph{ring-finite} if it
   is finitely generated as an $R$-algebra.
 \end{definition}
 \begin{lemma} \label{lec19L:funcfieldnfin}
   Let $S=k(x_1,\dots, x_r)$, with $r\ge 1$. Then $S$ is not ring-finite over $k$.
 \end{lemma}
 \begin{proof}[Sketch of Proof]
   Assume not. Then $S=k[f_1/g,\dots, f_\ell/g]$ (we can choose a common denominator).
   But then every rational function can be written with denominator a power of $g$, which
   is clearly false.
 \end{proof}
 \begin{theorem}[Artin-Tate Theorem]
   Let $R\subseteq S\subseteq T$ be rings, with  $R$ noetherian, $T$ ring-finite
   over $R$ and module-finite over $S$. Then $S$ is ring-finite over $R$.
 \end{theorem}
 \begin{proof}[Sketch of Proof]
   We have $T=R[t_1,\dots, t_n]=\sum_{j=1}^m Sy_j$ (choose one of the $y_j$ to be 1), so we
   get $t_i=\sum s_{ij} y_j$ and $y_iy_j=\sum s_{ij\ell} y_\ell$. Consider
   $S_0:=R[s_{ij},s_{ij\ell}]\subseteq S$, which is noetherian by Hilbert's basis
   theorem, and it is ring-finite over $R$. Note that $T=\sum S_0 y_j$ by construction,
   so $T$ is module-finite over $S_0$. So $T$ is a noetherian module over $S_0$, so $S$
   is module finite over $S_0$ (as a submodule of a noetherian module-finite module). So
   $S$ is module-finite over a ring-finite guy over $R$, so it is ring-finite.
 \end{proof}
