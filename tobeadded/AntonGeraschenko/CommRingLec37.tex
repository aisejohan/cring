 \stepcounter{lecture}
 \setcounter{lecture}{37}
 \sektion{Lecture 37}

 Easy applications of the Existence theorem:
 \begin{enumerate}
   \item $K/F$ any field extension, then $Val(K)\twoheadrightarrow Val(F)$
   \item $K/k$ algebraic if and only if $Val_k(K)=\{K\}$.
   \item $Val(K)=\{K\}$ if and only if $K$ is algebraic over some $\FF_p$.
   \item charaterization of noetherian normal domains via height 1 primes. (Later, we'll
   do some Krull ring stuff, maybe)
 \end{enumerate}

 We'd like to characterize UFDs.
 \begin{theorem}[Characterization of UFDs]
   For a domain $R$, the following are equivalent.
   \begin{enumerate}
     \item $R$ is a UFD.
     \item Every non-zero prime ideal contains a prime element.
     \item Principal ideal satisfy ACC, and every irreducible element is prime.
   \end{enumerate}
 \end{theorem}
 \begin{proof}
   $3\Rightarrow 2$. Let $\p\in \spec R$ be non-zero. Fix a non-zero $a\in \p$, and write
   $a=p_1\cdots p_n$ with the $p_i$ irreducible (we can do this because we have ACC on
   principal ideals). Then some $p_i$ is in $\p$, and $p_i$ is prime by assumption.

   $2\Rightarrow 1$. Form the multiplicative set $S=\{up_1\cdots p_n|u\in U(R), n\ge 0,
   p_i \text{ prime}\}$. We claim that this multiplicative set is saturated,
   i.e.~whenever $ab\in S$, $a$ and $b$ are in $S$. This is because any factor of
   $up_1\cdots p_n$ is of the same form; suppose $xy=up_1\cdots p_n$, then $p_1$ divides
   either $x$ or $y$, so we cancel it and induct. Thus, $R\smallsetminus S=\cup \p_i$ is
   a union of primes (by earlier stuff). If some $\p_i$ is non-zero, then it contains a
   prime element $p$, which would be in $S$. Thus, each $\p_i$ is zero, so
   $S\cup\{0\}=R$. So every non-zero element has a prime factorization, from which
   uniqueness follows in the usual way (note that we needed a prime factorization, not
   just an irreducible factorization).

   $1\Rightarrow 3$. In a UFD, it is clear that irreducible elements are prime. If ACC
   fails for principal ideals, we have $a_1R\subsetneq a_2R \subsetneq \cdots$. Then we
   have $a_n=r_{n+1} a_{n+1}$, where $r_{n+1}$ is not a unit. Then
   $a_1=r_2a_2=r_2r_3\cdots r_{n+1}a_{n+1}$ for any $n$. Thus, $a_1$ is divisible by $n$
   primes (counting multiplicity) for any $n$, a contradiction.
 \end{proof}
 Let's assume the following theorem for the moment.
 \begin{theorem}[Krull's Principal Ideal Theorem]
   Let $R$ be noetherian. If $\p$ is a minimal prime over some principal ideal $aR$, then
   $ht(\p)\le 1$.
 \end{theorem}
 \begin{theorem}
   Let $R$ be a domain. If $R$ is a UFD, then every height 1 prime is principal. If $R$
   is noetherian, then the converse is true.
 \end{theorem}
 \begin{proof}
   $(\Rightarrow)$ Consider $\p\in \spec_1(R)$.\footnote{$\spec_n R$ denotes the set of
   height $n$ primes of $R$.} Consider a non-zero $a\in \p$, so $a=p_1\cdots p_n$, for
   some primes $p_i$. Then some $p_i\in \p$, so $0\subsetneq(p_i)\subseteq \p$. Since
   $\p$ is height 1, we get $\p=(p_i)$.

   $(\Leftarrow)$ Now we assume $R$ is noetherian and every height 1 prime is principal.
   We will verify property 2 in the characterization of UFDs. If $\p$ is a non-zero
   prime, it contains some non-zero principal ideal $aR$. By Zorn's Lemma, there is a
   minimal prime $\p'$ over $aR$ contained in $\p$. By the PIT, $\p'$ has height 1 (it
   cannot be zero because we are in a domain). By assumption, $\p'$ is principal,
   generated by some prime element (which is in $\p$).
 \end{proof}
 \noindent\underline{Easy fact}: If $R$ is a UFD and $S$ is a multiplicative set, then
 $R_S=S^{-1}R$ is a UFD.
 \begin{theorem}[Nagata]
   Assume $R$ is a domain, and $S$ is a multiplicative set generated by some family
   $\{p_i\}$ of prime elements. Then $R$ is a UFD if and only if $R$ has
   ACC$_\text{prin}$ and $R_S$ is UFD.
 \end{theorem}
 \begin{proof}
   $(\Rightarrow)$ Follows from the easy fact and the characterization of UFDs.

   $(\Leftarrow)$ Assuming the given conditions, we will check condition 2 of the
   characterization of UFDs. Fix a non-zero $\p\in \spec R$. If $\p\cap
   S\neq\varnothing$, then $\p$ contains a prime element because $S$ is generated by
   prime elements and $\p$ is prime. So assume $\p\cap S=\varnothing$. Upon localizing at
   $S$, we know that $\p_S$ contains some prime element because $R_S$ is a UFD. Take
   $\pi\in \p$ which maps to a prime element $\pi\in \p_S$ (we can scale by ``units''
   from $S$ if needed). If $\pi$ is divisible by some $p_i$, say $\pi=\pi_1 p_i$, then
   $\pi_1\in \p$ and $\pi_1$ maps to the same prime element (well, an associate) in
   $\p_S$. Repeating, we may assume $\pi$ has no factor $p_i$ (because we have ACC on
   principal ideals in $R$). Now we claim that $\pi$ is a prime in $\p$. To check this,
   assume $\pi|ab$. Locally, we have $\pi|a$ (or $b$), so $p_{i_1}\cdots p_{i_k}a = \pi
   r$ for some $r\in R$ and some $p_{i_j}$. Since no $p_i$ divides $\pi$, they must all
   divide $r$. So $a=\pi r'$ for some $r'\in R$, so $\pi|a$. Thus, $\pi$ is prime, as
   desired.
 \end{proof}
 We all know the following theorem.
 \begin{theorem}[Gauss]
   If $R$ is a UFD, then $R[x]$ is a UFD.
 \end{theorem}
 However, $R[[x]]$ may fail to be a UFD, even if $R$ is noetherian.
 \begin{example}
   Let $R=\FF_2[x,y,z]/(x^2+y^3+z^7)$. This is a noetherian UFD, but $R[[t]]$ is not a
   UFD.
 \end{example}
 \begin{theorem}
   If $R$ is a PID, then $A=R[[x]]$ is a UFD.
 \end{theorem}
 \begin{proof}
   Again, we'll check that second condition for $A$. Let $\P\in\spec A$ be non-zero. If
   $\P\cap R$ is generated by $n$ elements, then $\P$ is generated by at most $n+1$
   elements (we only need the extra generator if $x\in \P$) as in the proof of the
   Hilbert basis theorem. If $x\in \P$, we are done because $x$ is a prime element. If
   $x\not\in \P$, $\P$ is generated by $n$ elements. Since $R$ is a PID, $n=1$, so $\P$
   is principal, generated by some prime element.
 \end{proof}
