 \stepcounter{lecture}
 \setcounter{lecture}{30}
 \sektion{Lecture 30}

Exercise III.3: (reciprocal polynomial trick) Show that a unit $u\in S$ is integral over
a subring $R$ if and only if $u\in R[u^{-1}]$.

\begin{theorem}
  For a local ring $(R,\m)$, the following are equivalent.
  \begin{enumerate}
    \item $R$ is a PID but not a field.
    \item $R$ is a noetherian normal domain of dimension 1.
    \item $R$ is noetherian and $\m = (\pi)$ is principal, with $\pi\not\in \nil R$.
    \item $R$ is noetherian, $\m\not\in \nil R$, and $\dim_{R/\m} \m/\m^2=1$.
  \end{enumerate}
  If these hold, $R$ is called a \emph{discrete valuation ring (DVR)}.
\end{theorem}
\begin{proof}
  omitted, but not trivial! \anton{}
\end{proof}
\begin{remark}
  In a DVR, the ideals are all of the form $\m^i$ (with $i\ge 0$), and $\dim_{R/\m}
  \m^i/\m^{i+1}=1$. $K^\times = U(R)\times \langle \pi\rangle$ (as a group).
\end{remark}
\begin{definition}
  The element $\pi$ is called a \emph{uniformizer} of $R$.
\end{definition}
\begin{example}\\
  \begin{tabular}{c|c|c|c}
    DVR & $\pi$ & $Q(R)$ & $U(R)$\\ \hline
    $\ZZ_{(p)}$ & $p$ & $\QQ$ & $\{a/b| a,b$ prime to $p\}$\\
    $k[[x]]$ & $x$ & $k((x))$ & $\{a_0+a_1x+\cdots|a_0\neq 0\}$\\
    $R=\{f/g| f,g\in k[x], \deg f\le \deg g\}$ & $1/x$ & $k(x)$ & $\{f/g|\deg f=\deg g\}$
  \end{tabular}
\end{example}
General properties of valuation rings
\begin{theorem}
  Let $R\in Val(K)$, then
  \begin{enumerate}
    \item $R$ is normal.
    \item $R$ is a B\'ezout ring (every finitely generated ideal is principal).
    \item Every ring between $R$ and $K$ is also a valuation ring.
    \item For every $\p\in \spec R$, $R/\p$ is a valuation ring.
    \item Every proper radical ideal is prime.
    \item For any proper ideal $I\< R$, $I_\infty = \bigcap I^n$ is prime. Moreover,
    every prime $\p\not\supseteq I$ is contained in $I_\infty$.
  \end{enumerate}
\end{theorem}
\begin{proof}
  (1) Take $u\in K\smallsetminus R$. Then $u^{-1}\in R$, so $R=R[u^{-1}]$, so $u$ is not
  integral over $R$ by exercise III.3. (2) If $I$ is generated by $a_1, a_2, \dots, a_n$,
  then the ideals generated by the $a_i$ form a chain, so $I$ is principal. (3) follows
  from the first point in the definition of a valuation ring. (4) immediate
  as for (3). (5) Take $ab\in I$, and assume $a=rb$ for some $r\in R$. Then $(rb)^2=
  r\cdot ab\in I$, so $rb=a\in I$. (6) Let $a,b\not\in I_\infty$, so $a\not\in I^m$ and
  $b\not\in I^n$. Then  we have $I^m\subseteq aR$ and $I^n\subseteq bR$. Suppose $ab\in
  I^{n+m}$, then $ab\in I^m I^n\subseteq aI^n$, so $b\in I^n$. Contradiction. Thus,
  $ab\not\in I^{n+m}$, so $ab\not\in I_\infty$. Finally, assume $\p\not\supseteq I$, then
  $\p\not\supseteq I^n$ for each $n$. Thus, $\p\subseteq I^n$ for each $n$, so
  $\p\subseteq I_\infty$.
\end{proof}
\noindent
Question 1: What are the noetherian valuation rings?\\
Question 2: What are the completely normal valuation rings?\\
\begin{corollary}
  Let $(R,\m)$ be a valuation ring of $K$.
  \begin{enumerate}
    \item $R$ is noetherian if and only if $R$ is a DVR or a field.
    \item The following are equivalent.
    \begin{enumerate}
      \item $R$ is completely normal.
      \item $R$ is good (nothing is infinitely divisible by a non-unit).
      \item $\dim R\le 1$.
    \end{enumerate}
  \end{enumerate}
\end{corollary}
\begin{proof}
  (1) $\Leftarrow$ is obvious. Assume $R$ is noetherian, then by the B\'ezout property,
  $R$ is a PID. Hence, $R$ is a DVR or a field.

  (2) $(a)\Rightarrow(b)$ was done in the last section. $(b)\Rightarrow(c)$ We must show
  that there is no prime between $(0)$ and $\m$. Assume there is such a $\p$, then choose
  $x\in \m\smallsetminus \p$, so $(x)\not\subseteq \p$. Then $0=(x)_\infty\supseteq \p$.
  $(c)\Rightarrow(a)$ Assume $R$ is not completely normal, so there is some $x\in
  Q(R)\smallsetminus R$ and $d\in R$ nonzero so that $dx^{-i}\in R$ for all $i\ge 0$. Let
  $y=x^{-1}$, so $d\in y^i R$ for every $i\ge 0$, so $d\in (y)_\infty$. Thus,
  $(y)_\infty$ is a non-zero prime ideal. Moreover, if $y=ry^2$, then $y$ would be a unit
  (since we are in a domain), so $y\not\in (y)^2$, so $y\not\in (y)_\infty$, so
  $(y)_\infty$ is not equal to $\m$.
\end{proof}
Next time: if $R\subseteq K$ is a subring, then $Val_R(K)$ is the set of \emph{relative
valuation rings} $R'$ of $K$ which contain $R$. If $R\in Val(K)$, we'll describe this
family.

\begin{corollary}\marginpar{\raggedleft \small Welcome to the Krull world}
  If $(R,\m)$ is a valuation ring with $\dim R\ge 2$, then $\m_\infty = \bigcap \m^n \neq
  0$.
\end{corollary}
