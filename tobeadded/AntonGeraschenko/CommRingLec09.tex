 \stepcounter{lecture}
 \setcounter{lecture}{9}
 \sektion{Lecture 9}

 \begin{itemize}
   \item Every finitely generated module $M$ over an artinian ring $R$ has finite length.
   \item Every finite length module $M$ over any (commutative) ring $R$ ``arises in this
   way''
 \end{itemize}

 \begin{proof}[Proof of Corollary 8.9 (5.11)]
   ($3\Rightarrow 2$)
   \[
    lg_R(M)=lg_{R/\ann M}(M) < \infty
   \]
   by Akizuki.

   ($2\Rightarrow 1$) clear

   ($1\Rightarrow 3$) $R/\ann M$ has a finitely generated faithful artinian module,
   namely $M$. Now use Akizuki.
 \end{proof}
 \begin{theorem}[Akizuki-Cohen]
   Any artinian ring $R$ is a \emph{finite} direct product of local artinian rings
   $R_1$,\dots, $R_n$, whose isomorphism types (as rings) are uniquely determined.
 \end{theorem}
 \begin{proof}
   Let $J=\rad R = \m_1\cap\cdots\cap \m_n$ ($R$ is semi-local by \anton{}). We also know
   that $J$ is nilpotent, so $(\m_1\cdots\m_n)^t=0$ for large enough $t$. By the Chinese
   Remainder Theorem,
   \[
     R=\frac{R}{\m_1^t\cdots \m_n^t} \cong \prod R/\m_i^t.
   \]
   But $R/\m_i^t$ is a local ring (with maximal ideal $\m_i/\m_i^t$) and artinian (being
   a quotient of an artinian ring). Uniqueness follows from Exercise 9.
 \end{proof}
 \begin{definition}
   $R$ is a \emph{principal ideal ring} (or \emph{PIR}) if every ideal of $R$ is
   principal.
 \end{definition}
 \begin{theorem}[5.13]
   Let $(R,\m)$ be a local artinian ring. Then the following are equivalent.
   \begin{enumerate}
     \item $R$ is a PIR.
     \item $\m$ is principal.
     \item $\dim_{R/\m}(\m/\m^2)\le 1$.
     \item $R$ is a ``chain ring'' (for any ideals $I$ and $J$, either $I\subseteq J$ or
     $J\subseteq I$).
   \end{enumerate}
 \end{theorem}
 \begin{proof}
   ($1\Rightarrow 2\Rightarrow 3$) clear.

   ($4\Rightarrow 2$) Let $\m=Ra_1+\cdots + Ra_n$. Since $R$ is a chain ring, we may
   assume $Ra_1\supseteq Ra_i$. Then $\m = Ra_1$.

   ($3\Rightarrow 1,4$) Find $a\in \m$ such that $\bar a$ generates $\m/\m^2$ over
   $R/\m$. By Nakayama's lemma, $\m = Ra$. Let's show that any non-zero $I\< R$ is
   principal. We know that $\m$ is nilpotent, so there is a largest integer $r$ such that
   $I\subseteq \m^r$ (so $I\not\subseteq \m^{r+1}$. Let $y\in I\smallsetminus \m^{r+1}$.
   We can write $y=ta^r$ because $y\in \m^r$. Then $t$ must be a unit, lest $y\in
   \m^{r+1}$. So $Ra^r=Ry\subseteq I\subseteq Ra^r$. If follows that $I$ is principal,
   and that all the ideals are of the form $\m^r$, proving (4).
 \end{proof}
 \begin{definition}
   A 0-dimensional Gorenstein ring is a local artinian ring in which the zero ideal is
   irreducible.
 \end{definition}
 \begin{example}
   Finite rings are artinian. For example $\ZZ/60$. We have $\ZZ/60\cong \ZZ/4\times
   \ZZ/3\times \ZZ/5$ illustrating Akizuki-Cohen.
 \end{example}
 \begin{example}
   Let $A$ is any ring, and let $\m$ be a finitely generated maximal ideal in $A$. Let
   $I$ be any ideal containing some $\m^k$. Then $R=A/I$ is artinian. The only prime
   ideal is $\m/I$, so this is a local zero-dimensional ring ... it is noetherian because
   all primes are finitely generated, so it is artinian.
 \end{example}
 \begin{example}
   In a local artinian ring $(R,\m)$, we have the filtration
   \[
    R\supseteq \m\supseteq \cdots \supseteq \m^n=0
   \]
   where consecutive quotients are vector spaces over $k=R/\m$. You may form a generating
   function $f(t)$ out of these dimensions, which will be a polynomial. For example, use
   the construction from the previous example, with $A=k[x,y]$, $\m=(x,y)$ and
   $I=(x^3,y^4)$, so $R=A/I$. Then we get that $\m^6=0$, and
   $f(t)=1+2t+3t^2+3t^3+2t^4+1t^5$ by inspection (just look at the number of (surviving)
   generators in each line).
   \[\xymatrix @dr @R=.5pc @C=.5pc{
      1 & y & y^2 & y^3 & y^4 \ar@{-}[d] \ar@{-}[r] & y^5 \ar@{-}[r]& y^6\ar@{-}[r] & \\
      x & xy & xy^2 & xy^3 & xy^4 \ar@{-}[d] & xy^5\\
      x^2 & x^2y & x^2y^2 & x^2y^3 & x^2y^4 \ar@{-}[d] \\
      x^3 \ar@{-}[r] \ar@{-}[d] & x^3y \ar@{-}[r]& x^3y^2 \ar@{-}[r]& x^3y^3 \ar@{-}[r]&\\
      x^4 \ar@{-}[d]& x^4y & x^3y^2\\
      x^5 \ar@{-}[d]& x^5y\\
      x^6 \ar@{-}[d]\\
      &
   }\]

   Note that in the case of a PIR, $f(t)=1+t+t^2+\cdots+t^{n-1}$ (the coefficient of $t$
   is 1 if and only if you are in a PIR).
 \end{example}
 \begin{example}
   ``a ring where $x^5=0$ but $x^6\neq 0$'' Let $A=k[x^2,x^3]\subseteq k[x]$; note that
   $x\not\in A$. Take $I=x^5 A$ and let $R=A/I$, and find $f(t)$. Write a $k$-basis for
   everything: $A$ has basis $\{1,x^2,x^3,x^4,\dots\}$; $I$ has basis
   $\{x^5,x^7,x^8,x^9,\dots\}$. Then $f(t)=1+2t+t^2+t^3$.

   Note that $A$ is the coordinate ring of the cuspidal cubic.
 \end{example}
