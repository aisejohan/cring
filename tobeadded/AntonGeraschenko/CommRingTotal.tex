  \subsection{\S 2. The Nilradical and Jacobson radical}

 The three spectra of a commutative ring $R$ are denoted $\spec(R) = \{\p\subset R \text{
 prime}\}$, $\Max(R) = \{\m\subset R\text{ maximal}\}$, and $\Min(R) = \{\p\< R \text{ minimal
 prime}\}$. $\p$ will always be a prime ideal.

 The \emph{nilradical} of $R$ is $\nil{R}=\{r\in R| r^n=0 \text{ for some }n\gg 0\}$.
 This is an ideal in $R$ (since $R$ is commutative). This is a special case of the
 ``radical formation''. If $I\subset R$, then define $\sqrt{I}$ (sometimes denoted $\rad I$)
 to be the elements $r\in R$ so that $r^n\in I$ for some $n>0$.  In particular, $\nil R =
 \sqrt 0$.

 \begin{lemma}
   $\sqrt{I} = \bigcap_{\p \supseteq I}\p$
 \end{lemma}
 \begin{proof}
   The inclusion $\sqrt I \subseteq \bigcap_{\p \supseteq I}\p$ is clear. If $r\not\in
   \sqrt{I}$, then $\{1,r,r^2,\dots\}$ is disjoint from $\sqrt I$. Take an ideal $J$
   containing $I$ which is maximal with respect to not intersecting $\{1,r,r^2,\dots\}$.
   We wish to show that $J$ is prime, so assume $a,b\not\in J$ and $ab\in J$. Then there
   is some $r^n= j_1 + xa \in J+(a)$ and $r^m = j_2+yb\in J+(b)$ by maximality of $J$.
   But then
   \[
    r^{n+m} = j_1j_2 + xaj_2 + j_1yb + xyab \in J.
   \]
   Contradicting the construction of $J$.
 \end{proof}
 \begin{definition}
   $R$ is called \emph{reduced} if $\nil R=0$. An ideal $I$ is called \emph{reduced} (or
   \emph{radical}) if $I=\sqrt I$.
 \end{definition}

 \begin{definition}
   A prime $\p\supseteq I\subset R$ is a \emph{minimal prime over $I$} if there there is no
   prime $\p'$ such that $I\subseteq \p'\subset \p$.
 \end{definition}
 Every $\p\supseteq I$ contains a minimal prime over $I$ (by Zorn's Lemma). In
 particular, $\sqrt I = \bigcap_{\p\supseteq I}\p = \bigcap_{\p'\text{ min'l over } I}
 \p'$.

 \begin{definition}
   The \emph{Jacobson radical} $\rad R$ is the intersection of all maximal ideals.
 \end{definition}
 Note that the maximal ideals of $R$ are in bijection with isomorphism classes of simple
 $R$-modules.\footnote{A \emph{simple} module is a non-zero module with no proper
 submodules.} How does the correspondence work? $\m\mapsto R/\m$, which is a simple
 module. On the other hand, given a simple module $S$, the annihilator of $S$ is a
 maximal ideal.\footnote{For noncommutative rings, many maximal ideals can correspond to
 the same isomorphism class of simple module.}

 Given this correspondence, one can conclude that $\rad R = \{r\in R| rS=0 \text{ for all
 simple modules } S\}$. In noncommutative theory, this is one definition of the Jacobson
 radical.

 \begin{lemma}[Key Property of $\rad(R)$]\label{lec03L:KeyRad}
   An ideal $I\subset R$ is contained in $\rad (R)$ if and only if $1+I\subseteq U(R)$.
 \end{lemma}
 That is, $\rad R$ is the largest ideal $I$ such that $1+I\subseteq U(R)$.
 \begin{proof}
   Say $I\subseteq \rad R$, and $i\in I$. Then if $1+i$ is not a unit, it is in some
   maximal ideal $\m$. But $i\in \m$, so $1\in\m$. Contradiction.

   Conversely, assume $1+I\subseteq U(R)$ and that there is an $i\in I\smallsetminus\rad
   R$. Then there is some maximal ideal $\m$ that doesn't contain $i$. The ideal
   generated by $i$ and $\m$ is all of $R$, so we have $1=ri+m$ for $r\in R$ and $m\in
   \m$. Then $m=1-ri\in 1+I\subseteq U(R)$. Contradiction.
 \end{proof}

 \begin{lemma}
   $\nil R\subseteq \rad R$.
 \end{lemma}
 \begin{proof}[Proof \#1]
   Maximal ideals are prime, so $\bigcap \p\subseteq \bigcap \m$.
 \end{proof}
 \begin{proof}[Proof \#2]
   $1+\nil R\subseteq U(R)$ because $(1+r)^{-1}=1-r+r^2-\cdots \in R$ whenever $r$ is
   nilpotent. The result follows from Lemma \ref{lec03L:KeyRad}.
 \end{proof}
 \begin{proof}[Proof \#3]
   Any nilpotent element is in every maximal ideal.
 \end{proof}

 Next we discuss two important classes of rings.
 \begin{definition}
   $R$ is called \emph{Jacobson semisimple} if $\rad R=0$.
 \end{definition}
 \begin{definition}
   $R$ is called \emph{rad-nil} if $\rad R=\nil R$.
 \end{definition}
 Clearly J-semisimple implies rad-nil.
 \begin{example}
   Some J-semisimple rings: $\mathbb{Z}$, $k[x]$ when $k$ is a field, $\QQ[x,y]/(xy)$.
 \end{example}

 \begin{lemma}[Nakayama's Lemma 2.9]
   For $J\subset R$, the following are equivalent:
   \begin{enumerate}
     \item $J\subseteq \rad(R)$
     \item Any finitely generated $R$-module $M$ satisfying $M=JM$ is zero.
     \item For any $R$-modules $N\subseteq M$ with $M/N$ finitely generated, $M=N+JM$
     implies $M=N$.
   \end{enumerate}
 \end{lemma}
 \begin{proof}
   $(1\Rightarrow 2)$ If $M\neq 0$, let $x_1$,\dots, $x_n$ is a minimal generating set
   for $M$ (with $n>0$). Since $JM = M$, we have $x_n= j_1 x_1+ \cdots + j_n
   x_n$ for $j_i\in J\subseteq \rad(R)$. But then $(1-j)x_n = j_1x_1+\cdots
   j_{n-1}x_{n-1}$, and $1-j\in U(R)$, so $x_n$ may be removed from the generating set,
   contradicting minimality.

   $(2\Rightarrow 3)$ Apply $(2)$ to $M/N$.

   $(3\Rightarrow 1)$ If $y\in J\smallsetminus \rad(R)$, then there is a maximal ideal
   $\m$ not containing $y$. But then $R = \m + JR$, so $(3)$ implies $\m=R$.
 \end{proof}

 \begin{corollary}[2.10]
   Let $J$ and $M$ be as above. Elements $x_1,\dots,x_n\in M$ generate $M$ if and only if
   their images $\bar x_1,\dots, \bar x_n$ generate $M/JM$.
 \end{corollary}
 \begin{proof}
   Take $N=Rx_1+\cdots + Rx_n$.
 \end{proof}

 \begin{definition}
   $R$ is \emph{local} if $|\Max R|=1$.
 \end{definition}
 We write ``$(R,\m)$ is local'', where $\m$ is the unique maximal ideal. Note that if
 $R$ is local then $R\neq 0$. Also note that we do not require $R$ to be Noetherian.
 \noindent Two major sources of local rings:
 \begin{enumerate}
   \item Take any maximal ideal $\m\subset R$, and consider $R/\m^t$, where $t$ is a positive
   integer. The unique maximal ideal is $\m/\m^t$.

   \item If $\p\subset R$ is a prime, then you can form the localization $R_\p$, whose unique
   maximal ideal is $\p R_\p$.
 \end{enumerate}
 Note that in a local ring $(R,\m)$, $U(R)=R\smallsetminus \m$. Also note that $R/\m$ is
 a field, called the \emph{residue field} of $R$.

 \begin{definition}
   $R$ is \emph{semi-local} if $|\Max R|<\infty$.
 \end{definition}
 \underline{Semi-localization}: Let $\p_1$, \dots, $\p_n\subset R$ be primes. The complement of
 the union, $S=R\smallsetminus \bigcup \p_i$, is closed under multiplication, so we can
 localize. $R[S^{-1}] = R_S$ is called the \emph{semi-localization}
 \index{semi-localization} of $R$ at the $\p_i$.

 The result of semi-localization is always semi-local. To see this, recall that the ideals
 in $R_S$ are in bijection with ideals in $R$ contained in $\bigcup \p_i$. Assume $\p
 R_S$ is maximal in $R_S$, then $\p\subseteq \bigcup \p_i$. By prime avoidance (Theorem
 \ref{lec01T:Prime}), $\p$ must be in one of the $\p_i$, so the only maximal ideals are
 $\p_i R_S$.

 \begin{definition}
   For a finitely generated $R$-module $M$, define $\mu_R(M)$ to be the smallest number
   of elements that can generate $M$.
 \end{definition}
 This is not the same as the cardinality of a minimal set of generators. For example, 2
 and 3 are a minimal set of generators for $\mathbb{Z}$ over itself, but $\mu_\mathbb{Z} (\mathbb{Z}) =1$.

 \begin{theorem}
   Let $R$ be semi-local with maximal ideals $\m_1,\dots, \m_n$. Let $k_i = R/\m_i$. Then
   \[
     \mu_R(M) = \max \{\dim_{k_i} M/\m_i M\}
   \]
 \end{theorem}
 The proof is in the notes. \anton{find a short proof}
 \stepcounter{lecture}
 \setcounter{lecture}{4}
 \section{Lecture 4}

 \subsection{\S 3 Associated Primes of Modules}

 For any module ${}_R M$, you can consider
 \begin{enumerate}
   \item $\ann M\subset R$, the annihilator of $M$.

   \item $\Z(M) = \{r\in R|rm=0\text{ for some nonzero }m\in M\}$ (defined for $M\neq
       0$). This is no longer an ideal of $R$, buy it contains $\ann M$.

   \item $\ass M = \{\p\in \spec R| \p = \ann(m)\text{ for some non-zero }m\in M\}$. We
   call $\ann (m)$ a \emph{point annihilator}. Clearly any such $\p$ contains $\ann M$.

   \item $\supp M = \{\p\in \spec R| M_\p\neq 0\}$.
 \end{enumerate}
 A very important case is when $M=R/I$ is cyclic (where $I\subset R$). In this case, $\ann R/I
 = I$, $\Z(R/I) = \{r\in R|rm\in I\text{ for some }m\in R\smallsetminus I\}$.

 \begin{lemma}
   For $\p\in \spec R$, and suppose you are given ${}_R M$, then $\p\in \ass M$ if and
   only if $R/\p\hookrightarrow M$.\mpar[How to deal with $\ass M$]{}
 \end{lemma}
 \begin{proof}
   If $f:R/\p\hookrightarrow M$ then $\p = \ann (f(1))$. If $\p=\ann(x)$, then $f:R\to M$
   defined by $f(1)=x$ has kernel $\p$.
 \end{proof}
 \begin{lemma}\label{lec04L:assinclusion}
   If $N\subseteq M$, then $\ass N\subseteq \ass M$.
 \end{lemma}
 \begin{proof}
   $\p\in \ass N\Rightarrow \p=\ann(x)\Rightarrow \p\in \ass M$.
 \end{proof}
 \begin{lemma}[Herstein]
   Any maximal point annihilator of $M$ is an associated prime.
 \end{lemma}
 \begin{proof}
   Let $\ann(x)$ be a maximal point annihilator. To check that $\ann(x)$ is prime,
   assume $ab\in \ann(x)$ and $b\not\in \ann(x)$. Then $bx\neq 0$, so $\ann(x)=
   \ann(bx)$ by maximality. But $abx=0$, so $a\in \ann(bx)=\ann(x)$.
 \end{proof}
   Note that if $M$ is not noetherian, there may not be any maximal point annihilators, so
   this does not prove existence of an associated prime.
 \begin{example}
   Here an a non-zero module $M$ with $\ass M=\varnothing$. Take suitable $R$, and let
   $M = {}_R R$. Take $R = \QQ[x_1,x_2,\dots]/(x_i^2)_{i=1,2,\dots}$. In this ring,
   there is a unique prime $\p = (x_1,x_2,\dots)$. In particular, $R$ is a local ring.
   If $\ass R\neq \varnothing$, then $\p=\ann (m)$ for some nonzero $m$. But then $m$
   kills every $x_i$. However, if any polynomial annihilates $\p$, write it so that all
   terms are square-free, but then take $x_N$ with $N$ larger than any indices appearing
   in $m$, and $mx_N\neq 0$.
 \end{example}
 \begin{example}
    Let $R=\mathbb{Z}$. Then $\ass(\mathbb{Z}) = \{(0)\}$, $\ass(\QQ) = \{(0)\}$, $\ass(\mathbb{Z}\oplus
    \mathbb{Z}/60)=\{(0),(2),(3),(5)\}$, and $\ass(\QQ/\mathbb{Z})=\Max \mathbb{Z} = \spec \mathbb{Z}\smallsetminus
    \{(0)\}$.
 \end{example}
 For every $M$, $\Z(M)$ is a union of prime ideals. In general, there is an easy
 characterization of sets $Z$ which are a union of primes: it is exactly when
 $R\smallsetminus Z$ is a \emph{saturated multiplicative set}. This is Kaplansky's
 Theorem 2.
 \begin{definition}
   A multiplicative set $S\neq \varnothing$ is a \emph{saturated multiplicative set} if
   for all $a,b\in R$, $a,b\in S$ if and only if $ab\in S$. (``multiplicative set'' just
   means the ``if'' part)
 \end{definition}
 To see that $\Z(M)$ is a union of primes, just verify that its complement is a saturated
 multiplicative set.

 Let $M = R/I$. Then $\ass (R/I)$ is often called ``the set of primes associated to
 $I$''. For any set $S$ and ideal $I\subset R$, we define $I:S = \{r\in R|r\cdot S\subseteq
 I\}$. Of course, you can replace $S$ by the ideal generated by $S$. Then $\ass R/I$
 consists of prime ideals of the form $I:x$.

 \begin{example}
   Take $R=k[x,y,z]$, where $k$ is an integral domain, and let $I = (x^2-yz,x(z-1))$. Any
   prime associated to $I$ must contain $I$, so let's consider
   $\p=(x^2-yz,z-1)=(x^2-y,z-1)$, which is $I:x$. It is prime because $R/\p = k[x]$,
   which is a domain. To see that $I:x\subseteq \p$, assume $tx\in I\subseteq \p$; since
   $x\not\in \p$, $t\in p$, as desired.

   There are two more associated primes, but we will not find them here.
 \end{example}

 \begin{proposition}
   \begin{enumerate}
     \item[]
     \item \label{lec04:AssFact1} If $N\subseteq M$, then
        $\ass M \subseteq \ass N \cup \ass M/N$
     \item \label{lec04:AssFact2} If $M = \bigoplus_{i=1}^n M_i$, then
        $\ass M = \bigcup_{i=1}^n \ass M_i$.
   \end{enumerate}
 \end{proposition}
 \begin{proof}
   (\ref{lec04:AssFact1}) \underline{Observation}: If $N\subseteq R/\p$ is a nonzero
   submodule, then $\ass N=\{\p\}$. To see this, take $x\in R\smallsetminus \p$; then
   $\p:x=\p$ by primeness of $\p$.

   Now if $\p\in \ass M$, we have $R/\p\hookrightarrow M$; call the image $Y$. If $N\cap
   Y=0$, then $Y\cong R/\p\hookrightarrow M/N$, so $\p\in \ass (M/N)$. Otherwise, $N\cap
   Y\subseteq Y$ is a nonzero submodule, so $\{\p\} \in \ass (N\cap Y)\subseteq \ass N$.

   (\ref{lec04:AssFact2}) The containment $\subseteq$ follows from
   (\ref{lec04:AssFact1}), and the containment $\supseteq$ follows from Lemma
   \ref{lec04L:assinclusion}.
 \end{proof}
 \stepcounter{lecture}
 \setcounter{lecture}{5}
 \section{Lecture 5}

 \begin{example}
   Let's compute the primes associated to $I=\p_1\p_2$ under the assumptions that (1)
   $\p_1=(x)$, where $x$ is regular; (2) There is some $y\in \p_2\smallsetminus \p_1$.

   We claim that $I:y=\p_1$ and $I:x=\p_2$, which would show that $\p_1$ and $\p_2$ are
   associated prime. The inclusions $\supseteq$ are clear. For the other inclusion,
   assume $ty\in I=\p_1\p_2\subseteq \p_1$. Since $y\not\in \p_1$, we get $t\in \p_1$, as
   desired. If $tx\in I=\p_1\p_2 = x\p_2$. Since $x$ is regular, we can cancel it, so
   $t\in \p_2$.

   Now we'd like to show that there are no other associated primes. We have
   $I=\p_1\p_2\subseteq \p_1=(x)\subseteq R$, so any associated prime is an associated
   prime of $R/\p_1$ or of $\p_1/\p_1\p_2\cong R/\p_2$ (via multiplication by $x$). But
   we showed that the only associated prime of $R/\p$ is $\p$.
 \end{example}
 \begin{example}
   Now let's specialize the previous example to $R=k[x,y]$ where $k$ is a field, and $x$
   and $y$ as in the previous example: $\p_1=(x)$ and $\p_2$ is one of the following:
   \begin{itemize}
     \item $\p_2=(y)$, $I=(xy)$. In this case, we see that $\p_1$ and $\p_2$ are the
     minimal primes over $I$; if $\p\supseteq I=\p_1\p_2$, then $\p$ must be one or the
     other since it is minimal (it must contain one since $\p$ is prime).

     \item $\p_2=(x,y)$, $I=(x^2,xy)$. Here it is clear that $\sqrt I=(x)$. The
     remarkable thing is that $(x,y)$ is an ``embedded associated prime'' to $I$. We call
     an associated prime if it contains another embedded prime; if it doesn't contain an
     associated prime, it is called an ``isolated prime'' of $I$.
   \end{itemize}
   \vspace*{-1.5\baselineskip}
 \end{example}

 Recall that $\supp M = \{\p\in \spec R| M_\p\neq 0\}$. If $I\subseteq R$ is a subset, then we
 define $\V(I) = \{\p\in \spec R|I\subseteq \p \}$.
 \begin{proposition}
   For any module $M$ over $R$, let $I=\ann M$.
   \begin{enumerate}
     \item \label{lec05:1} ``Specialization'': If $\p'\subseteq\p$ and $\p'\in\supp M$, then $\p\in \supp
     M$.
     \item \label{lec05:2} $\ass M \subseteq \supp M$.
     \item \label{lec05:3} $\supp M \subseteq \V(I)$.
     \item \label{lec05:4} If $M$ is finitely generated, then $\supp M=\V(I)$.
   \end{enumerate}
 \end{proposition}
 \begin{proof}
   (\ref{lec05:1}) Let $\p'\subseteq \p$ and assume $\p\not\in \supp M$, so $M_\p=0$. Then
   $M_{\p'}=(M_\p)_{\p'} = 0$.

   (\ref{lec05:2}) If $\p\in \ass M$, then $R/\p\hookrightarrow M$.
   Localizing, we get $R_\p/\p_\p = (R/\p)_\p\hookrightarrow M_\p$, so $M_\p$ contains a
   field, so it is non-zero.

   (\ref{lec05:3}) Take $\p\not\in \V(I)$, so there is some $i\in
   I\smallsetminus \p$ so that $iM=0$. This implies that $M_\p=0$.

   (\ref{lec05:4}) Say $M = \sum_{i=1}^n Rm_i$. Take $\p\in\V(I)$ and assume it is not a
   supporting prime, so $M_\p=0$. Then for each $i$, there is some $r_i\not\in \p$ so
   that $r_im_i=0$. Then $r=r_1\cdots r_n$ kills all of $M$, so it is in $I$, but
   not in $\p$ (since $\p$ is prime). Contradiction.
 \end{proof}
 \begin{proposition}[Theorem 84 in Kaplansky]
   For any module $M$, any minimal prime $\p$ over $I=\ann M$ must lie in $\Z(M)$.
   \mpar[ ``Minimal primes consist of zero divisors'']{}
 \end{proposition}
 In particular, taking $M=R$, we get that minimal primes in $R$ lie in $\Z(R)$. Note that
 there are \emph{no chain conditions} on $M$ in this proposition.
 \begin{proof}
   Let $I\subseteq \p$ and we know $I\subseteq \Z(M)$. Then $R/\p$ and $R\smallsetminus
   \Z(M)$ are multiplicative sets away from $I$. Let $S$ be the multiplicative set
   generated by these two. We claim that $S$ is disjoint from $I$. To see this, assume
   $a\not\in \p$ and $b\not\in \Z(M)$ such that $ab\in I$, so $abM=0$. But $bM=M$ by
   assumption, so $aM=0$, so $a\in I\subseteq \p$, which is a contradiction.

   Thus, there is some prime $\p'\supseteq I$ which is disjoint from $S$. It follows that
   $\p'\subseteq \p$, and $\p$ is minimal over $I$, so $\p'=\p$. Similarly, $\p'\subseteq
   \Z(M)$. That is, $\p\subseteq \Z(M)$, as desired.
 \end{proof}
 \begin{theorem}\label{lec05radNOembedded}
   Let $I\subset R$ be a radical ideal, and consider the cyclic module $R/I$.
   \begin{enumerate}
     \item \label{lec05cyc1} $\Z(R/I) = \bigcup_{\p\text{ min'l over } I} \p$.
     \item \label{lec05cyc2} Every $\p\in \ass (R/I)$ is a minimal prime over $I$.
     \mpar[ ``A radical ideal has no embedded points'']{}
   \end{enumerate}
 \end{theorem}
 \begin{proof}
   (\ref{lec05cyc1}) The inclusion $\supseteq$ is clear from the previous proposition.
   Given $x\in \Z(R/I)$, fix $y\not\in I$ such that $xy\in I$. Then $I\subsetneq I:x$
   (since $y\not\in I$). So there is a minimal prime $\p$ over $I$ such that
   $I:x\not\subseteq \p$ (since $I$ is the intersection of minimal primes over it). But
   $x\cdot (I:x)\subseteq \p$, which implies that $x\in \p$.

   (\ref{lec05cyc2}) Let $\p=I:m$ be an associated prime, where $m\not\in I$. Then
   $m\not\in \p$, lest $m\cdot m=0\in I$, which would imply $m\in I$ since $I$ is
   radical. Now assume that there  is some $\p'$ so that $\p\supsetneq \p'\supseteq I$.
   Fix $x\in \p\smallsetminus \p'$. Then $xm\in I\subseteq \p'$, which implies $m\in \p$.
   Contradiction.
 \end{proof}

 \begin{proposition}
   Let $S\subseteq R$ be a multiplicative set disjoint from some prime $\p$. For any
   \mpar[ ``$\ass M$ behaves well under localization'']{}
   module $M$, if $\p\in \ass M$, then $\p_S\in \ass (M_S)$. If $\p$ is finitely
   generated, then the converse holds.
 \end{proposition}
 Assuming all primes in $\ass M$ are finitely generated, this can be rewritten as the
 equality
 \[
    \ass (M_S) = \ass(M) \cap \spec (R_S).
 \]

 This is proven in the notes. \anton{}
 \stepcounter{lecture}
 \setcounter{lecture}{6}
 \section{Lecture 6}

 \renewcommand\P{\mathcal{P}}

 \subsection{\S 4 Noetherian Rings and Noetherian Induction}
 Recall the following facts about rings and modules with chain conditions.
 \begin{enumerate}
   \item A module is noetherian if and only if every submodule is finitely generated.
   \item A module is noetherian and artinian if and only if it has a (finite) composition
   series (quotients are simple). The Jordan-H\"older theorem tells us that the quotients
   are unique up to permutation.
   \item If $N\subseteq M$, $M$ is noetherian (resp.\ artinian) if and only if both
   $N$ and $M/N$ are.
   \item If $R$ is noetherian, then $M$ is noetherian if and only if it is finitely
   generated.
 \end{enumerate}

 \begin{theorem}[I.\ S.\ Cohen]\label{lec06T:cohen}
   A ring $R$ is noetherian if and only if all prime ideals are finitely generated.
 \end{theorem}
 For the proof, we need the following lemma and proposition.
 \begin{lemma}[Oka's Lemma]
   Let $I\subset R$ and $b\in R$. If $I+(b)$ and $I:b$ are finitely generated, then so is $I$.
 \end{lemma}
 \begin{proof}
   Since $I+(b)$ is finitely generated, it can be written as $I_0+(b)$ for some finitely
   generated ideal $I_0\subseteq I$. Then $I = I_0+b(I:b)$ is finitely generated.
 \end{proof}
 \begin{proposition}\label{lec06P:maxisprime}
   If $I\subset R$ is maximal with respect to \emph{not} being finitely generated (i.e.\ every
   $J\supsetneq I$ is f.g.), then $I$ is prime.
 \end{proposition}
 \begin{proof}
   Clearly $I\neq R$. Suppose $I$ is not prime, so there are $a,b\not\in I$, with $ab\in
   I$. Then $I+(b)$ and $I:b$ ($\ni a$) are finitely generated. By Oka's Lemma, $I$ is
   finitely generated, which is a contradiction.
 \end{proof}
 \begin{proof}[Proof of Theorem \ref{lec06T:cohen}]
   Assume all primes in $R$ are finitely generated, and that $\F=\{$non-f.g.\
   ideals$\}\neq \varnothing$. Since the union of non-finitely-generated ideals is not
   finitely generated, Zorn's Lemma gives us a maximal element, which is prime by
   Proposition \ref{lec06P:maxisprime}, so it is finitely generated by assumption.
   Contradiction.
 \end{proof}
 \[\xymatrix@!0 @R=3pc @C=10pc{
  \text{ \emph{all} ideals f.g.} \ar@{=>}@/_2.2ex/[r] \ar@{<=>}[d]
  &   \text{primes f.g.} \ar@{=>}[l]_{\text{Cohen}}
                         \ar@{==>}[r]^(.43){\text{Krull's PIT}}
                         \ar@{:>}[d]
  & \text{DCC on primes}\\
  \text{ACC on \emph{all} ideals} \ar@{=>}@/^2.2ex/[r]^{\text{Obvious}} & \text{ACC on primes}
 }\]

% Noetherian induction is this idea: let $R$ be noetherian, and we want to prove some
% statement for $R$. Assume it is false, and look at the collection of counterexamples,
% and get a maximal counterexample (by the maximum principle). Get a contradiction by hook
% or by crook.
 \begin{theorem}[Noetherian Induction Principle]
   Let $R$ be a noetherian ring, let $\P$ be a property, and let $\F$ be a family of
   ideals $R$. Suppose the inductive step: if all ideals in $\F$ strictly larger than
   $I\in \F$ satisfy $\P$, then $I$ satisfies $\P$. Then all ideals in
   $\F$ satisfy $\P$.
 \end{theorem}
 \begin{proof}
   Assume $\F_\text{crim} = \{J\in \F|J\text{ does not satisfy }\P\}\neq \varnothing$.
   Since $R$ is noetherian, $\F_\text{crim}$ has a maximal member $I$. By maximality, all
   ideals in $\F$ strictly containing $I$ satisfy $\P$, so $I$ also does by the inductive
   step.
 \end{proof}

 \begin{definition}
   An element $r$ is \emph{irreducible} if it cannot be written as a product of two
   non-units. A (proper) ideal $I$ is \emph{irreducible} if it cannot be written as the
   intersection of two strictly larger ideals. (irreducible ideals are primary!
   \anton{ref this result})
 \end{definition}

 \begin{example} For a noetherian ring $R$, we can prove the following results by
 checking the inductive step in each case. For the first two, take $\F$ to be the set of
 all ideals.
   \begin{enumerate}
     \item Every ideal is a finite intersection of irreducible ideals.

     Assume every ideal strictly containing $I$ is a finite intersection of irreducible
     ideals. If $I=R$, it is the empty intersection. If $I$ is irreducible, then we're
     done. Otherwise, $I = J_1\cap J_2$ for strictly larger ideals $J_1$ and $J_2$. By
     assumption, $J_i$ is a finite intersection of irreducible ideals, so $I$ is also.

     \item ($R\neq 0$) Every ideal in $R$ contains a finite product of primes.

     Assume any ideal larger than $I$ contains a finite product of primes. If $I$ is
     prime or $R$, we're done. Otherwise, there are ideals $J_1$ and $J_2$ which contain
     $I$ such that $J_1J_2\subseteq I$. Since each $J_i$ contains a finite product of
     primes, so does $I$.

     \item ($R$ a domain) Every $r\not\in (0)\cup U(R)$ is a finite product of
     irreducible elements.

     Here we let $\F$ be non-zero, non-$R$, principal ideals, and let $\P$ be the
     statement that the generator is a finite product of irreducible elements (note that
     this is independent of the choice of generator since $R$ is a domain). \anton{finish
     ... easy}

     \item If $J=\sqrt J$, then $J$ is a \emph{finite} intersection of primes.
     (Kaplansky's Theorems 87 and 88)

     Take $\F$ to be the set of radical ideals.\anton{finish}

     As a corollary, for any $I\subset R$ in a noetherian ring, there are finitely many
     minimal primes over $I$.
   \end{enumerate}
   \vspace*{-1.7\baselineskip}
 \end{example}
 Note that the examples where $\F$ is not the set of all ideals illustrate that to
 apply noetherian induction, you only need the ideals \emph{in $\F$} to satisfy the
 ascending chain condition.
 \stepcounter{lecture}
 \setcounter{lecture}{7}
 \section{Lecture 7}

 \subsection{Noetherian Descent}

 Certain theorems about commutative rings can be proven by a reduction to the noetherian
 case (the Hilbert basis theorem is secretly being used). If the statement $\F$ you want
 to prove only involves a finite number of elements of $R$, say $a_1,\dots, a_n$. Then
 look at the ring $R_0$ generated by $1, a_1,\dots, a_n$. It is the homomorphic image of
 the ring $\mathbb{Z}[x_1,\dots, x_n]$. By the Hilbert basis theorem, this ring is noetherian,
 so homomorphic images are also noetherian.

 Here is a typical application. A non-zero commutative ring satisfies the strong rank
 property ($R^m\hookrightarrow R^n$ implies $m\le n$). The strong rank property can be
 rephrased as $R^{n+1}\not\hookrightarrow R^n$ (module theoretically). This can be
 formulated as, ``a homogeneous system of $n$ equations and $n+1$ unknowns has a a
 nontrivial solution''. It suffices to solve this system in $R_0$, so apply descent. Now
 we just have to prove it in a noetherian ring. $R^{n+1}\hookrightarrow R^n$ can be
 easily contradicted in the noetherian case. Think of $R^{n+1}$ as $X_1=R^n\oplus R$;
 think of this $R$ as generated by $x$, so the image of $R^{n+1}$ is an image of $R^n$
 direct sum with another module. Now repeat by embedding $R^{n+1}$ into the copy of
 $R^n$. Then the module generated by $x,x_1,\dots, x_n$ gives an infinite ascending
 chain.

 \subsection{\S 5 Artinian Rings}

 \begin{definition}
   The \emph{(Krull) dimension} of a commutative ring $R$ is defined as $\dim R =
   \sup\{$lengths of chains of primes$\}$.
 \end{definition}
 In particular, a zero dimensional ring is one in which every prime ideal is maximal:
 $\spec R=\Max R$.

 \begin{definition}[von Neumann]
   An element $a\in R$ is called \emph{von Neumann regular} if there is some $x\in R$
   such that $a=axa$.
 \end{definition}
 \begin{definition}[McCoy]
   A element $a\in R$ is \emph{$\pi$-regular} if some power of $a$ is von Neumann
   regular.
 \end{definition}
 \begin{definition}
   A element $a\in R$ is \emph{strongly $\pi$-regular} (in the commutative case)
   if the chain $aR\supseteq a^2R\supseteq a^3R\supseteq \cdots$ stabilizes.
 \end{definition}
 A ring $R$ is von Neumann regular (resp.\ (strongly) $\pi$-regular) if every element of
 $R$ is.

 \begin{theorem}[5.2]
   For a commutative ring $R$, the following are equivalent.
   \begin{enumerate}
     \item $\dim R=0$.
     \item $R$ is rad-nil (i.e. $\rad R = \nil R$) and $R/\rad R$ is von Neumann regular.
     \item $R$ is strongly $\pi$-regular.
     \item $R$ is $\pi$-regular.

     \item[] \hspace{-7ex} And any one of these implies
     \item $\C(R)=U(R)$; any non-zero-divisor is a unit.
   \end{enumerate}
 \end{theorem}
 \begin{proof}
   $1\Rightarrow 2\Rightarrow 3\Rightarrow 4 \Rightarrow 1$ and $4\Rightarrow 5$. We will
   not do $1\Rightarrow 2\Rightarrow 3$ here.

   ($3\Rightarrow 4$) Given $a\in R$, there is some $n$ such that $a^n R = a^{n+1}
   R=a^{2n}R$, which implies that $a^n = a^n x a^n$ for some $x$.

   ($4\Rightarrow 1$) Is $\p$ maximal? Let $a\not\in \p$. Since $a$ is $\pi$-regular, we
   have $a^n=a^{2n}x$, so $a^n(1-a^nx)=0$, so $1-a^nx\in \p$. It follows that $a$ has an
   inverse mod $\p$.

   ($4\Rightarrow 5$) Using $1-a^nx=0$, we get an inverse for $a$.
 \end{proof}
 \begin{example}
   Any local rad-nil ring is zero dimensional, since $2$ holds.
   In particular, for a ring $S$ and $\m\in \Max S$, $R=S/\m^n$ is zero dimensional
   because it is a rad-nil local ring.
 \end{example}
 \begin{example}[Split-Null Extension]
   For a ring $A$ and $A$-module $M$, let $R=A\oplus M$
   with the multiplication $(a,m)(a',m')=(aa',am'+a'm)$ (i.e.\ take the multiplication on
   $M$ to be zero). In $R$, $M$ is an ideal of square zero. ($A$ is called a
   \emph{retract} of $R$ because it sits in $R$ and can be recovered by quotienting by
   some complement.) If $A$ is a field, then $R$ is a rad-nil local ring, with maximal ideal $M$.
 \end{example}

 \[\def\x#1#2{\left\{\parbox{#1}{\hfuzz=1pt \centering #2}\right\}}
   \xymatrix @!0 @R=4pc @C=4pc{
     \x{3.5pc}{reduced rings} \ar@{-}[dr] & &
     \x{5pc}{zero-dimensional rings} \ar@{-}[dr]\ar@{-}[dl]& &
     \x{4.5pc}{noetherian rings} \ar@{-}[dl]\\
    & \x{6pc}{von Neumann regular rings} & & \x{3.3pc}{artinian rings}
 }\]
 \stepcounter{lecture}
 \setcounter{lecture}{8}
 \section{Lecture 8}

 A topological point: $\spec R$ is always $T_0$ (each point has a neighborhood that
 misses \emph{some} other point). $\spec R$ is $T_1$ (points are closed) if and only if
 it is $T_2$ (Hausdorff) if and only if $\dim R=0$.\anton{check this}
 \begin{remark}
   The separation axioms are called $T_i$ because they are ``Trennung's'' axioms.
 \end{remark}
 \begin{corollary}[to the last theorem]
   The following are equivalent.
   \begin{itemize}
     \item $R$ is von Neumann regular
     \item $R$ is reduced and $\dim R=0$
     \item $R_\m$ is a field for all $\m\in \Max(R)$ (In particular, $R$ is
     ``locally noetherian'').
   \end{itemize}
 \end{corollary}

 \begin{proposition}
   If $R$ is artinian, then
   \begin{enumerate}
     \item $\dim R=0$,
     \item $J:=\rad R$ is nilpotent, and
     \item $R$ is semi-local.
   \end{enumerate}
 \end{proposition}
 \begin{proof}
   (2) see notes

   (1) For each $a\in R$, $aR\supseteq a^2R\supseteq \cdots$ stabilizes (i.e.\ $R$ is
   strongly $\pi$-regular), so $\dim R=0$ by Theorem from last time.

   (3) Among all finite products of maximal ideals, choose one that is minimal, say
   $I=\m_1\cdots \m_n$. Then for any maximal ideal $\m$, $\m I = I$ by minimality, so
   $I\subseteq \m$, so some $\m_i\subseteq \m$ for some $i$ since $\m$ is prime. Then
   $\m=\m_i$ by maximality.
 \end{proof}
 This gives a characterization of semi-local rings $R$ via chain conditions.
 \begin{corollary}
   A ring $R$ is semi-local if and only if $R/\rad R$ is artinian.
 \end{corollary}
 \begin{proof}
   ($\Leftarrow$) Assume $R/J$ is artinian ($J:=\rad R$), so $R/J$ is semi-local. By any
   maximal ideal in $R$ contains $J$, so they are in bijection with maximal ideals of
   $R/J$. So $R$ is semi-local.

   ($\Rightarrow$) If $R$ is semi-local, with maximal ideals $\m_1$,\dots, $\m_n$. In
   \S 2, we showed that since $J=\m_1\cap \cdots \cap \m_n$, $R/J \cong \prod_{i=1}^n
   R/\m_i$. So $R/J$ is a finite product of fields, so it is artinian.
 \end{proof}
 \begin{lemma}[Key Lemma for Akizuki]
   Let $R$ be a ring with (not necessarily distinct) maximal ideals $\m_1$,\dots $\m_n$
   such that $\m_1\cdots \m_n=0$. Then $R$ is noetherian if and only if $R$ is artinian.
 \end{lemma}
 \begin{proof}
   Look at the filtration
   \[
    R\supseteq \m_1 \supseteq \m_1\m_2 \supseteq \cdots \supseteq \m_1\cdots\m_n =0.
   \]
   The filtration factor $\m_1\cdots \m_i/\m_1\cdots \m_{i+1}$ is a $R/\m_{i+1}$ vector
   space. In a vector space, noetherian and artinian are both equivalent to finite
   dimensional. $R$ is noetherian if and only if each filtration factor is noetherian,
   which occurs if and only if each factor is artinian, which occurs if and only if $R$
   is artinian!
 \end{proof}
 \begin{theorem}[Akizuki]
   For any ring $R$, the following are equivalent.
   \begin{enumerate}
     \item $R$ is artinian.
     \item $R$ is noetherian and $\dim R=0$.
     \item ${}_R R$ has finite length.
     \item \emph{All} finitely generated modules ${}_R M$ have finite length.
     \item There is a faithful module ${}_R M$ of finite length.
     \item There is a faithful finitely generated artinian module ${}_R M$.
   \end{enumerate}
 \end{theorem}
 \begin{proof}
   $3\Rightarrow 4\Rightarrow 1\Rightarrow 2 \Rightarrow 5 \Rightarrow 3$ and
   $5\Rightarrow 6\Rightarrow 1$

   ($3\Rightarrow 4$) We have ${}_R R^n\twoheadrightarrow M$, and $lg(R^n)=n\cdot lg(R)
   \ge lg(M)$.

   ($4\Rightarrow 1$) Apply $(4)$ to ${}_R R$.

   ($1\Rightarrow 2$) We already have $\dim R=0$ from the Proposition. We also know that
   $R$ is semi-local, with maximal ideals $\m_1$,\dots, $\m_n$. Then $\rad R = \m_1\cap
   \cdots\cap \m_n\supseteq \m_1\cdots \m_n$ is nilpotent. So $(\m_1\cdots\m_n)^t=0$ for
   some big $t$. Then by the Key lemma, $R$ is noetherian.

   ($2\Rightarrow 5$) From the second example of noetherian induction, $(0)=\p_1\cdots
   \p_n$ (every ideal contains a finite product of primes). Since $\dim R=0$, these
   $\p_i$ are maximal. By the Key lemma, $R$ is artinian, so ${}_RR$ is a faithful module
   of finite length.

   ($5\Rightarrow 3$) If ${}_R M$ is finite length, it is finitely generated, say $M =
   Rm_1+\cdots Rm_k$. We have a map $R\to M^k=M\oplus \cdots \oplus M$ given by $r\mapsto
   (rm_1,\cdots ,rm_k)$. This is an $R$-module homomorphism, and it is injective because
   $M$ is faithful (no $r$ can kill all the generators). Now $lg(R)\le k\cdot lg(M)<
   \infty$.

   ($5\Rightarrow 6$)  ${}_RR$ is already a faithful module of finite length.

   ($6\Rightarrow 1$) Same argument as $5\Rightarrow 3$, but with ``finite length''
   replaced by ``artinian''.
 \end{proof}
 \begin{remark}
   ($6$) is not equivalent to ($6'$) There is a faithful artinian module ${}_R M$. For
   example, take $R=\mathbb{Z}$ and $M = \varinjlim \mathbb{Z}/p^n\mathbb{Z}$, the Pr\"ufer $p$-group. Then
   $M$ is artinian (all the submodules are $\mathbb{Z}/p^k\mathbb{Z}$), and faithful, but not finitely
   generated.
 \end{remark}
 \begin{remark}
   Note that artinian implies noetherian! This statement is true for rings (even
   non-commutative rings), but not for modules. Take the same example $M = \varinjlim
   \mathbb{Z}/p^n\mathbb{Z}$ over $\mathbb{Z}$. However, there is a module-theoretic statement which is
   related.
 \end{remark}
 \begin{corollary}
   For a finitely generated module $M$ over any commutative ring $R$, the following are
   equivalent.
   \begin{enumerate}
     \item $M$ is an artinian module.
     \item $M$ has finite length (i.e.\ is noetherian and artinian).
     \item $R/\ann M$ is an artinian ring.
   \end{enumerate}
 \end{corollary}
 Note that we don't assume that $R$ is noetherian (as in Eisenbud).
 \stepcounter{lecture}
 \setcounter{lecture}{9}
 \section{Lecture 9}

 \begin{itemize}
   \item Every finitely generated module $M$ over an artinian ring $R$ has finite length.
   \item Every finite length module $M$ over any (commutative) ring $R$ ``arises in this
   way''
 \end{itemize}

 \begin{proof}[Proof of Corollary 8.9 (5.11)]
   ($3\Rightarrow 2$)
   \[
    lg_R(M)=lg_{R/\ann M}(M) < \infty
   \]
   by Akizuki.

   ($2\Rightarrow 1$) clear

   ($1\Rightarrow 3$) $R/\ann M$ has a finitely generated faithful artinian module,
   namely $M$. Now use Akizuki.
 \end{proof}
 \begin{theorem}[Akizuki-Cohen]
   Any artinian ring $R$ is a \emph{finite} direct product of local artinian rings
   $R_1$,\dots, $R_n$, whose isomorphism types (as rings) are uniquely determined.
 \end{theorem}
 \begin{proof}
   Let $J=\rad R = \m_1\cap\cdots\cap \m_n$ ($R$ is semi-local by \anton{}). We also know
   that $J$ is nilpotent, so $(\m_1\cdots\m_n)^t=0$ for large enough $t$. By the Chinese
   Remainder Theorem,
   \[
     R=\frac{R}{\m_1^t\cdots \m_n^t} \cong \prod R/\m_i^t.
   \]
   But $R/\m_i^t$ is a local ring (with maximal ideal $\m_i/\m_i^t$) and artinian (being
   a quotient of an artinian ring). Uniqueness follows from Exercise 9.
 \end{proof}
 \begin{definition}
   $R$ is a \emph{principal ideal ring} (or \emph{PIR}) if every ideal of $R$ is
   principal.
 \end{definition}
 \begin{theorem}[5.13]
   Let $(R,\m)$ be a local artinian ring. Then the following are equivalent.
   \begin{enumerate}
     \item $R$ is a PIR.
     \item $\m$ is principal.
     \item $\dim_{R/\m}(\m/\m^2)\le 1$.
     \item $R$ is a ``chain ring'' (for any ideals $I$ and $J$, either $I\subseteq J$ or
     $J\subseteq I$).
   \end{enumerate}
 \end{theorem}
 \begin{proof}
   ($1\Rightarrow 2\Rightarrow 3$) clear.

   ($4\Rightarrow 2$) Let $\m=Ra_1+\cdots + Ra_n$. Since $R$ is a chain ring, we may
   assume $Ra_1\supseteq Ra_i$. Then $\m = Ra_1$.

   ($3\Rightarrow 1,4$) Find $a\in \m$ such that $\bar a$ generates $\m/\m^2$ over
   $R/\m$. By Nakayama's lemma, $\m = Ra$. Let's show that any non-zero $I\subset R$ is
   principal. We know that $\m$ is nilpotent, so there is a largest integer $r$ such that
   $I\subseteq \m^r$ (so $I\not\subseteq \m^{r+1}$. Let $y\in I\smallsetminus \m^{r+1}$.
   We can write $y=ta^r$ because $y\in \m^r$. Then $t$ must be a unit, lest $y\in
   \m^{r+1}$. So $Ra^r=Ry\subseteq I\subseteq Ra^r$. If follows that $I$ is principal,
   and that all the ideals are of the form $\m^r$, proving (4).
 \end{proof}
 \begin{definition}
   A 0-dimensional Gorenstein ring is a local artinian ring in which the zero ideal is
   irreducible.
 \end{definition}
 \begin{example}
   Finite rings are artinian. For example $\mathbb{Z}/60$. We have $\mathbb{Z}/60\cong \mathbb{Z}/4\times
   \mathbb{Z}/3\times \mathbb{Z}/5$ illustrating Akizuki-Cohen.
 \end{example}
 \begin{example}
   Let $A$ is any ring, and let $\m$ be a finitely generated maximal ideal in $A$. Let
   $I$ be any ideal containing some $\m^k$. Then $R=A/I$ is artinian. The only prime
   ideal is $\m/I$, so this is a local zero-dimensional ring ... it is noetherian because
   all primes are finitely generated, so it is artinian.
 \end{example}
 \begin{example}
   In a local artinian ring $(R,\m)$, we have the filtration
   \[
    R\supseteq \m\supseteq \cdots \supseteq \m^n=0
   \]
   where consecutive quotients are vector spaces over $k=R/\m$. You may form a generating
   function $f(t)$ out of these dimensions, which will be a polynomial. For example, use
   the construction from the previous example, with $A=k[x,y]$, $\m=(x,y)$ and
   $I=(x^3,y^4)$, so $R=A/I$. Then we get that $\m^6=0$, and
   $f(t)=1+2t+3t^2+3t^3+2t^4+1t^5$ by inspection (just look at the number of (surviving)
   generators in each line).
   \[\xymatrix @dr @R=.5pc @C=.5pc{
      1 & y & y^2 & y^3 & y^4 \ar@{-}[d] \ar@{-}[r] & y^5 \ar@{-}[r]& y^6\ar@{-}[r] & \\
      x & xy & xy^2 & xy^3 & xy^4 \ar@{-}[d] & xy^5\\
      x^2 & x^2y & x^2y^2 & x^2y^3 & x^2y^4 \ar@{-}[d] \\
      x^3 \ar@{-}[r] \ar@{-}[d] & x^3y \ar@{-}[r]& x^3y^2 \ar@{-}[r]& x^3y^3 \ar@{-}[r]&\\
      x^4 \ar@{-}[d]& x^4y & x^3y^2\\
      x^5 \ar@{-}[d]& x^5y\\
      x^6 \ar@{-}[d]\\
      &
   }\]

   Note that in the case of a PIR, $f(t)=1+t+t^2+\cdots+t^{n-1}$ (the coefficient of $t$
   is 1 if and only if you are in a PIR).
 \end{example}
 \begin{example}
   ``a ring where $x^5=0$ but $x^6\neq 0$'' Let $A=k[x^2,x^3]\subseteq k[x]$; note that
   $x\not\in A$. Take $I=x^5 A$ and let $R=A/I$, and find $f(t)$. Write a $k$-basis for
   everything: $A$ has basis $\{1,x^2,x^3,x^4,\dots\}$; $I$ has basis
   $\{x^5,x^7,x^8,x^9,\dots\}$. Then $f(t)=1+2t+t^2+t^3$.

   Note that $A$ is the coordinate ring of the cuspidal cubic.
 \end{example}
 \stepcounter{lecture}
 \setcounter{lecture}{10}
 \section{Lecture 10}

 \subsection{\S 6 Associated Primes over Noetherian Rings}

 \begin{proposition}
   \begin{enumerate}
     \item[]
     \item If $R$ is noetherian and ${}_R M\neq 0$, then $\ass M \neq \varnothing$.
     \item If ${}_R M$ is noetherian, then $|\ass M| < \infty$.
   \end{enumerate}
 \end{proposition}
 \begin{example}
   The following usually give good counter-examples. Look at $M=\QQ/\mathbb{Z}$ or look at
   $R=k\times k\times \cdots$, where $k$ is a field. Notice that $M$ is faithful over
   $\mathbb{Z}$.
 \end{example}
 \begin{proof}
   (1) If $M\neq 0$, then the set of point annihilators is non-empty, so there is a maximal
   element by the noetherian hypothesis. By Herstein's lemma, we've found an associated
   prime.

   (2) Let $\p_1,\p_2,\cdots \in \ass M$ be an infinite number of distinct associated
   primes of $M$. Then you can find some $R/\p_1\hookrightarrow M$. Since $\ass
   (R/\p_1)=\{\p_1\}$, the other primes must be associated primes of $M/(R/\p_1)$. This
   gives you an ascending chain of submodules of $M$, contradicting the noetherian
   hypothesis.
 \end{proof}
 \begin{theorem}[6.2]
   Let $M\neq 0$ be a module over a noetherian ring $R$. Then $\Z(M)=\bigcup \p_i$, where
   the $\p_i$ are maximal point annihilators. If $M$ is finitely generated, then this is
   a \emph{finite} union, and any ideal $A\subseteq \Z(M)$ lies in some $\p_i$. In
   particular, $Am=0$ for some non-zero element $m\in M$. (This last part is Theorem 82
   of Kaplansky\footnote{Kaplansky says it is one of the most useful facts about
   commutative rings.})
 \end{theorem}
 \begin{proof}
   It is enough to show $\Z(M)\subseteq \bigcup \p_i$. Note that $\Z(M)$ is the union of
   all point annihilators. Since $R$ is noetherian, every point annihilator is in some
   maximal point annihilator.

   If $M$ is finitely generated, then $\ass M$ is finite by the proposition, so the union
   is finite. If $A\subseteq \Z(M)$ is closed under addition and multiplication, then by
   prime avoidance, $A$ is in some $\p_i=\ann m_i$.
 \end{proof}
 In general, for a given $M$, the set $\ass M$ is the important set of primes, as far as
 the behavior of $M$ is concerned.
 \begin{proposition}
  Let $R$ be a noetherian ring.
  \begin{enumerate}
   \item Let $M'\subseteq M$ over $R$, and let $m\in M$. Then $m\in M'$
   if and only if $m/1\in M'_\p$ for every $\p\in \ass (M/M')$.

   \item $f:M\to Q$ is injective if and only if $f:M_\p\to Q_\p$ is injective for
   every $\p\in \ass M$.

   \item $g:N\to M$ is surjective if and only if $g_\p:N_\p\to M_\p$ is surjective for
   every $\p\in \ass (M/g(N)) = \ass (\coker g)$.
 \end{enumerate}\end{proposition}
 \begin{proof}
  First we do the ``if'' parts:
  \begin{enumerate}
   \item We may replace $M$ by $M/M'$ and assume $M'=0$. Suppose $m\neq 0$, then $\ann
   m\subseteq \p\in \ass M$ (some maximal annihilator) by the noetherian hypothesis on
   $R$. Then if $m/1=0$ in $M_\p$, there is some $x\in R\smallsetminus \p$ so that
   $xm=0$, contradicting $\ann m\subseteq \p$.

   \item[2,3.] If $f(m)=0$, then $m/1\in M_\p$ must be zero because $f_\p(m/1)=0$ and
   $f_\p$ are all injective. By part (1), $m=0$. For (3), use part (1), with $M'=g(N)$.

   \[\xymatrix{
     M_\p \ar@{^(->}[r]^{f_\p} & Q_\p\\
     M \ar[u] \ar[r]^f & Q\ar[u]
   }\qquad\qquad
   \xymatrix{
     N_\p \ar@{->>}[r]^{g_\p} & M_\p\\
     N \ar[u] \ar[r]^g & M\ar[u]
   }\]
  \end{enumerate}
  the ``only if'' parts are elsewhere
 \end{proof}
 From now on, assume $R$ is noetherian and $M$ is finitely generated. We will write
 $I=\ann M$. Let $(B,\le)$ be a poset. The set of maximal elements of $B$ is denoted
 $B^*$. Similarly, $B_*$ is the set of minimal elements.
 \begin{example}
   $\spec (R)^*=\Max(R)$ and $\spec (R)_* =\Min(R)$. Finally, $\V(I)_*$ is the set of
   minimal primes over $R$.
 \end{example}

 Given $R$ and $M$ as above, we want to study $\ass(M)^*$ and $\ass(M)_*$.
 \begin{lemma}[Star Principle]
   Let $A\subseteq (B,\le)$.
   \begin{enumerate}
     \item If for every $b\in B$, there is an $a\in A$ so that $b\le a$, then $A^*=B^*$.
     \item If for every $b\in B$, there is an $a\in A$ so that $a\le b$, then $A_*=B_*$.
   \end{enumerate}
 \end{lemma}
 \begin{proof}
   Totally obvious.
 \end{proof}
 \begin{theorem}
   For the given $M$, $\ass (M)^*=\{$point annihilators$\}^*=\{$ideals $\subseteq
   \Z(M)\}^*$.
 \end{theorem}
 \begin{proof}
   Clearly $\ass (M)\subseteq \{$point annihilators$\}\subseteq \{$ideals $\subseteq
   \Z(M)\}$. To get the desired conclusion, it suffices to check that every ideal
   $A\subseteq \Z(M)$ is contained in some associated prime. This is Theorem 10.3.
 \end{proof}
 How about $\ass(M)_*$? The following proposition is key.
 \begin{proposition} \label{lec10P:minlAss}
  \begin{enumerate}
   \item[]
   \item Any minimal prime $\p$ over $I=\ann (M)$ is in $\ass M$.
   \item $\ass (R/I)\subseteq \ass (M)$.
  \end{enumerate}
 \end{proposition}
 \begin{proof}
   (1) By an earlier result, $\p_\p$ is a minimal prime over $\ann(M_\p)\supseteq I_\p$.
   Thus, $\p_\p\subseteq \Z(M_\p)$ because minimal primes always consist of
   zero-divisors. Since $M_\p$ is finitely generated over the noetherian ring $R_\p$,
   $\p_\p$ must annihilate some nonzero $m/1$. But $\p_\p$ is maximal in $R_\p$, so
   $\p_\p=\ann (m/1)$, so $\p_\p\in \ass (M_\p)$. By 5.5 (3.16), $\p\in \ass M$.

   (2) Let $\p_0\in \ass (R/I)$. Write $\p_0 = I:b$ for some $b\not\in I$. Look at
   $N=bM\neq 0$. Then $\ann (N)=\{r\in R|rbM=0\} = \{r\in R|rb\in I\}=I:b = \p_0$. Thus,
   $\p_0$ is a minimal prime over $\ann(N)=\p_0$, so it is an associated prime of $N$. It
   follows that $\p_0\in \ass M$.
 \end{proof}

 \begin{theorem}
   $\ass (M)_* = \supp (M)_* = \V(I)_* = \ass (R/I)_*$.
 \end{theorem}
 \begin{proof}
   Let $\V(I)_*= \{\p_1,\dots, \p_n\}$. $\V(I)=\ass(R/I)_*$ by applying the first
   equality to $M=R/I$. $\supp M=\V(I)$, so applying $-_*$ we get the second equality.

   It happens that $\ass (M)_* = \supp (M)_*$ even if $M$ is not finitely generated.
   First note that $\ass M \subseteq \supp M$. Now we need to show that any supporting
   prime $\p$ contains an associated prime. We have that $M_\p\neq 0$, so $M_\p$ has an
   associated prime $(\p_0)_\p$ with $\p_0\subseteq \p$ (since $R_\p$ is noetherian).
   Since $R$ is noetherian, $\p_0$ is finitely generated, so $\p_0\in \ass M$.
 \end{proof}
 \stepcounter{lecture}
 \setcounter{lecture}{11}
 \section{Lecture 11}

 \begin{definition}
   $\p\in \left\{\begin{array}{c}\ass (M)_* \\ \ass(M)^*\\ \ass (M)\smallsetminus
   \ass(M)_*\end{array} \right\}$ is called $\left\{\begin{array}{c}\emph{isolated}
   \\
   \emph{maximal}\\ \emph{embedded}\end{array} \right\}$.
 \end{definition}
 \begin{example}
   Let $R=\mathbb{Z}$ and $M=\mathbb{Z}\oplus \mathbb{Z}/60$. Then $\ass M = \{(0),(2),(3),(5)\}$, so $(0)$ is
   an isolated prime, and $(2)$, $(3)$, and $(5)$ are embedded maximal primes.
 \end{example}
 \begin{example}
   $k$ a field, $R=k[x,y]$, $M=R/I$ for $I=x\cdot (x,y)$. Then $\ass
   (R/I)=\{(x),(x,y)\}$, so $(x)$ is an isolated prime and $(x,y)$ is a maximal embedded
   prime.
 \end{example}
 \begin{example}
   $R=k[x,y]$, $M=R/I$ with $I=(xy)$. Then $\ass (R/I)=\{(x),(y)\}$, so both primes are
   isolated and maximal, and there are no embedded prime.
 \end{example}

 If we look at $\ass (R/I)$, then things are much simpler when $I$ is a radical ideal. By
 the stuff in lecture 5, all $\p\in \ass(R/I)$ are minimal primes over $I$. Then there
 cannot be any containments, so $\ass(R/I)$ has no comparisons as a poset. Therefore, all
 primes ``associated to $I$'' are isolated (there are no embedded primes).

 \begin{definition}
   The \emph{total ring of quotients} of a commutative ring $R$, denoted $Q(R)$, is the
   localization of $R$ at $\C(R)$.
 \end{definition}
 Note that $R\hookrightarrow Q(R)$, and if you invert anything else, you kill stuff.

 \begin{theorem}
   A noetherian ring $R$ is reduced if and only if $Q(R)$ is a finite direct product of
   fields.
 \end{theorem}
 \begin{proof}\def\P{\mathfrak{P}}
   ($\Leftarrow$) is clear because $R\subseteq Q(R)$, and $Q(R)$ is reduced.

   ($\Rightarrow$) $\Z(R) = \p_1\cup \cdots \p_n$ where $\p_i$ are the minimal primes, so
   $Q(R)$ is the semi-localization of $R$ at these primes. Since $R$ is reduced, $\bigcap
   \p_i=\sqrt{(0)}=(0)$. So $Q(R)$ is semi-local, with maximal ideals $\P_1$, \dots,
   $\P_n$. Since $\bigcap \p_i=(0)$, we also get $\bigcap \P_i=(0)$. To see this, assume
   $x\in \bigcap \P_i$, then $x=\frac{p_i}{s_i}$ for each $i$. Then $s_1\cdots s_n x\in
   \p_i$ for each $i$, so it is zero. Since each $s_i$ is regular, we get $x=0$.
   Then we get
   \[
    Q(R)/(0) \cong \prod Q(R)/\P_i
   \]
   so $Q(R)$ is a finite direct product of fields.
 \end{proof}
 \begin{example}
   If $R$ is not noetherian, the result fails. Let $R=k\times k\times \cdots$. Then $R$
   is reduced and von Neumann regular, so it is zero-dimensional, so $C(R)=U(R)$. In
   particular $Q(R)=R$, which is not a finite product of fields.
 \end{example}

 \begin{definition}
   A prime filtration of a module $M$ is a finite filtration where each consecutive
   quotient is $R/\p$.
 \end{definition}
 \begin{theorem}
   A finitely generated module $M$ over a noetherian ring $R$ has a prime filtration.
   Furthermore, $\ass(M)_* = \{\p_1,\dots, \p_k\}_*$, where $\p_1,\dots, \p_k$ are the
   primes occurring in the filtration.
 \end{theorem}
 \begin{proof}
   Go from the bottom and use the fact that (over a noetherian ring) every non-zero
   module has an associated prime. $R/\p_1\hookrightarrow M$, then $R/\p_2\hookrightarrow
   M/(R/\p_1)$, etc.

   Finally, apply the Star principle to the containment $\ass (M)\subseteq \{\p_1,\dots,
   \p_k\} \subseteq \supp (M)$. We already saw that any supporting prime contains an
   associated prime.
 \end{proof}
 \stepcounter{lecture}
 \setcounter{lecture}{12}
 \section{Lecture 12}

 picture of history

 Useful mnemonics (6.9):
 \[
  \bigcap \{\p\in \ass (M)_*\} = \sqrt{\ann (M)} \subseteq   \Z(M) =
  \bigcup \{\p\in \ass (M)^*\}
 \]
 where the equalities require all noetherian hypotheses (though the containment is true
 in general).
 \begin{definition}[Lasker, 1905]
   An ideal $\q\subsetneq R$ is \emph{primary} if $ab\in \q$ and $b\not\in \q$ implies
   that $a^n\in \q$ for some $n$.
 \end{definition}
 We'd like to do primary decomposition for modules because it doesn't cost any more work,
 so we need to generalize this definition.
 \begin{definition}
   If $M$ is an $R$-module. We call a submodule $Q\subsetneq M$ \emph{primary} (in $M$) if
   $\sqrt{\ann (M/Q)}= \Z(M/Q)$ (you get ``$\subseteq$'' for free). i.e.\ for every $a\in
   R$ and $x\in M\smallsetminus Q$ with $ax\in Q$, we have $a^n M\subseteq Q$ for some
   $n$.
 \end{definition}
 \begin{warning}
   This definition is not completely standard in the literature, and the fact that $M/Q$
   may not be finitely generated often mucks things up.
 \end{warning}

 \begin{proposition}
   Suppose $Q\subsetneq M$ is primary. Then $\q:= \ann (M/Q)$ is a primary ideal, and
   $\p:=\sqrt \q$ is prime.
 \end{proposition}
 \begin{proof}
   Clearly $\q\subsetneq R$ since $1\in R\smallsetminus \q$. Let $a,b\in R$, with $ab\in
   \q$ and $b\not\in \q$. Then $abM\subseteq Q$ but $bM\not\subseteq Q$. So $a\in
   \Z(M/Q)=\sqrt{\ann (M/Q)}$ (since $Q\subsetneq M$ is primary), so $a^n\in \q$ for some
   $n$.

   Finally, we check that $\p=\sqrt \q$ is prime. Let $ab\in \p$ with $b\not\in \p$. Then
   we have $a^nb^n\in \q$ and $b^n\not\in \q$. Since $\q$ is primary, $(a^n)^N\in \q$ for
   some $N$, so $a\in \p$, as desired.
 \end{proof}
 \begin{definition}
   Henceforth we will say that $Q$ is \emph{$\p$-primary}.
 \end{definition}
 \begin{proposition}
   If $Q,Q'\subsetneq M$ are both $\p$-primary, then $Q\cap Q'$ is $\p$-primary.
 \end{proposition}
 \begin{proof}
   Let $\q=\ann (M/Q)$, $\q'=\ann (M/Q')$, and assume $a\in \Z\bigl(M/(Q\cap Q')\bigr)$,
   with $ax\in Q\cap Q'$, where $x\not\in Q\cap Q'$. We may assume $x\not\in Q$. Then
   $a\in \Z(M/Q)$, so  $a\in \sqrt \q=\p=\sqrt{\q'}$ since $Q$ is primary. Then
   $a^nM\subseteq Q\cap Q'$ for some $n$, as desired.
 \end{proof}
 \begin{proposition}
   Let $R$ be a noetherian ring, and let $Q\subsetneq M$ with $M/Q$ finitely generated.
   Then $Q\subsetneq M$ is primary if and only if $|\ass(M/Q)|=1$. In fact, if $Q$ is
   $\p$-primary, then $\ass (M/Q)=\{\p\}$.
 \end{proposition}
 \begin{proof}
   ($\Rightarrow$) For this direction, we don't use that $M/Q$ is finitely generated. We
   know that $\ass(M/Q)\neq \varnothing$ since $R$ is noetherian. Let $P\in \ass (M/Q)$,
   then $P\supseteq \ann (M/Q)=:\q$, so $P\supseteq \sqrt \q=:\p$. On the other hand, if
   $a\in P$, then $a=\ann(m)$, so $a\in \Z(M/Q)=\sqrt \q=\p$. So $P=\p$.

   \def\P{\mathfrak{P}}
   ($\Leftarrow$) Say $\ass (M/Q)=\{\P\}$. Then we get
   \begin{align*}
     \Z(M/Q) &= \bigcup \{P\in \ass (M/Q)^*\} \\ &= \P \\ &= \bigcap \{P\in
     \ass(M/Q)_*\} \\ &=\sqrt{\ann(M/Q)}. \qedhere
   \end{align*}
 \end{proof}
 \begin{definition}
   A module $Q\subsetneq M$ is called \emph{irreducible} if it cannot be written as the
   intersection of two strictly larger submodules.
 \end{definition}
 \begin{lemma}[Noether's Lemma]
   Suppose $M$ is a noetherian module over a ring $R$. Then any $Q\subseteq M$ is a
   finite intersection of irreducible submodules.
 \end{lemma}
 \begin{proof}
   Noetherian induction. If all submodules containing $Q$ are finite intersections of
   irreducibles, then so is $Q$.
 \end{proof}
 \begin{theorem}[Noether's Theorem]
   Say $R$ is noetherian and $M/Q$ is finitely generated. If $Q$ is irreducible, then $Q$
   is primary.
 \end{theorem}
 \begin{proof}
   Assume $\p_1,\p_2\in \ass (M/Q)$. Say $R/\p_i \cong K_i/Q \subseteq M/Q$ for $i=1,2$.
   If $Q\subsetneq K_1\cap K_2$, then there is some $x\in K_1\cap K_2\smallsetminus Q$,
   which gives a non-zero element of $K_i/Q\subseteq M/Q$. But then the annihilator of
   $x$ in $M/Q$ must be $\p_i$ (since $K_i/Q\cong R/\p_i$), so $\p_1=\p_2$, which would
   imply that $Q$ is primary.

   Thus, we may assume $Q=K_1\cap K_2$. \anton{finish}
 \end{proof}
 \begin{definition}
   Suppose $N\subsetneq M$. Then an equality of the form $N=Q_1\cap \cdots \cap Q_n$ is
   called a \emph{primary decomposition of $N$} if $Q_i$ is $\p_i$-primary, with all of
   the $\p_i$ distinct. This decomposition is called \emph{minimal} (or
   \emph{irreducible}) if no $Q_i$ can be omitted.
 \end{definition}
 Note that the decomposition if minimal exactly when $\bigcap_{i\neq j} Q_j\not\subseteq
 Q_i$ for each $i$.
 \stepcounter{lecture}
 \setcounter{lecture}{13}
 \section{Lecture 13}

 \begin{example}
  \begin{enumerate}
   \item Prime ideals are primary.

   \item In a rad-nil local ring (0-dimensional local ring), any proper ideal is primary.
   As a consequence, if $\m\in \Max R$ for any ring $R$, and $\m^n\subseteq \q\subseteq
   \m$, then $\q$ is $\m$-primary.

   \item In a UFD, a principal ideal $(a)$ is primary if and only if $a$ is 0 or a power
   of an irreducible element; see Ex.\ 52.

   \item In $R=k[x,y]$, where $k$ is a field. Then $I=(x^2,xy)$ is \emph{not} primary. In
   the factor ring, $y$ is a zero-divisor because $xy\in I$, but $y$ is not nilpotent
   modulo $I$.

   \item Let $\q=(y^2,x+yz)\subset k[x,y,z]$. We claim that $\q$ is $\p$-primary for
   $\p=(x,y)$. First check that $\p=\sqrt \q$. $R/\q = k[x,y,z]/(y^2,x+yz)\cong
   k[y,z]/(y^2)$, in which every zero-divisor is nilpotent because $(y^2)$ is primary in
   $k[y,z]$.
  \end{enumerate}
  \vspace*{-1.7\baselineskip}
 \end{example}
 \begin{theorem}[Lasker-Noether Primary Decomposition]
   Let $R$ be noetherian and $N\subsetneq M$, with $M/N$ finitely generated. Then $N$ has
   a minimal primary decomposition in $M$.
 \end{theorem}
 \begin{proof}
   First write $N=Q_1\cap \cdots \cap Q_n$, where the $Q_i$ are irreducible. By Noether's
   theorem, each $Q_i$ is $\p_i$-primary for some prime $\p_i$.

   If $\p_i=\p_j$, we may replace $Q_i$ and $Q_j$ by $Q_i\cap Q_j$, which is also
   $\p_i$-primary. Now we may assume the $\p_i$ are distinct.

   Now remove unneeded $Q_i$ until we have a minimal decomposition.
 \end{proof}
 What can we say about uniqueness? Recall some facts about localization of modules at a
 multiplicative set $S\subseteq R$.
 \begin{definition}
   The \emph{$S$-saturation of $Q$} is $\{m\in M|sm\in Q\text{ for some }s\in S\}$.
 \end{definition}
 \begin{exercise}
   The saturation of $Q$ is $Q^{ec}$, the contraction of the extension of $Q$, i.e.\
   ``$Q_S\cap M$''$=i^{-1}(Q_S)$, where $i:M\to M_S$.
 \end{exercise}
 \begin{lemma}
   $Q\subsetneq M$. Suppose $S\cap \Z(M/Q)=\varnothing$, then $Q^{ec}=Q$.
 \end{lemma}
 \begin{proof}
   if $m\in Q^{ec}$, then $sm\in Q$ for some $s\in S$, which implies $m\in Q$ since
   $s\not\in \Z(M/Q)$. The reverse inclusion is clear.
 \end{proof}
 We will apply this to the situation where $Q$ is a $\p$-primary submodule of $M$ and
 $S=R\smallsetminus \p$.

 \begin{theorem}[Main Uniqueness Theorem]
   Same hypotheses as in the Lasker-Noether Theorem. Let $N=Q_1\cap \cdots \cap Q_n$ be
   \emph{any} minimal primary decomposition, where $Q_i$ is $\p_i$-primary. Then
   \begin{enumerate}
     \item $\ass (M/N) = \{\p_1,\dots, \p_n\}$. In particular, all the $\p_i$ are
     uniquely determined (up to permutation).

     \item If $\p\in \ass (M/N)_*$ is an isolated prime, then $Q_i=N^{ec}$ (with respect
     to localization at $\p_i$). In particular, $Q_i$ is uniquely determined.
   \end{enumerate}
 \end{theorem}
 \begin{proof}
  \begin{enumerate}
   \item[]
   \item
   $M/N\hookrightarrow \bigoplus M/Q_i$, so
   $\ass(M/N)\subseteq \bigcup \ass (M/Q_i) = \{\p_1,\dots, \p_n\}$.
   Now let's show that $\p_1\in \ass (M/N)$. Fix $x\in
   (Q_2\cap\cdots \cap Q_n)\smallsetminus Q_1$ (such an $x$ exists by minimality of the
   decomposition). Since $\p_1$ is finitely generated ($R$ noetherian),
   $\p_1^{k+1}x\subseteq Q_1$ for some $k\gg 1$; choose the minimal such $k$. Then there
   is some $y\in \p_1^k x\smallsetminus Q_1$. Then we have $\bar y \in M/N$ is non-zero,
   and that $\p_1 y=0\in M/N$, so $\p_1\subseteq \ann (\bar y)$. For the reverse
   inclusion, suppose $r\bar y=0$, so $ry\in N\subseteq Q_1$. Thus, $r\in
   \Z(M/Q_1)=\p_1$.

   \item Let $\p=\p_i$ be an isolated prime. By assumption, $\p_j\not\subseteq \p$ for
   $j\neq i$. Choose $k$ large enough so that $\p_j^k M\subseteq Q_j$ (for $j\neq i$)
   ($M/N$ finitely generated). But $\p_j^k\not\subseteq \p$. Now consider localization at
   $\p$; we get $(\p_j)_{\p}^k M_\p = M_\p = (Q_j)_\p$ for each $j\neq i$. Localizing the
   equation $N=Q_1\cap \cdots \cap Q_n$, we get
   \[
    N_\p=(Q_1\cap \cdots \cap Q_n)_\p = \bigcap_j (Q_j)_\p = (Q_i)_\p.
   \]
   So we get $N^{ec}=Q_i^{ec} = Q_i$.\qedhere
  \end{enumerate}
 \end{proof}
 \begin{example}[A point from last time]
   The $\p_i$ being distinct doesn't guarantee minimality of the decomposition. If
   $N=Q_1\cap Q_2\cap Q_3$. Choose some $\p$ containing
 \end{example}
 \stepcounter{lecture}
 \setcounter{lecture}{14}
 \section{Lecture 14}

 Combining Theorem \ref{lec05radNOembedded} and Proposition \ref{lec10P:minlAss}, we have
 that $\ass (R/I) = \spec (R/I)_*$, i.e.\ $R/I$ has no embedded primes.

 \begin{example}[Nonuniqueness of minimal primary decomposition (Noether)]
  We saw last time that primary components at isolated primes are unique, but at embedded
  primes, we don't have uniqueness. Take $R=k[x,y]$ where $k$ is a field. Let $I=x\cdot
  (x,y)$, then $\ass (R/I) = \{\p_1=(x),\p_2=(x,y)\}$. $\p_1$ is isolated and $\p_2$ is
  embedded.

  Let's find some primary decompositions $I=\q_1\cap \q_2$. The $\q_1$ should be
  unique since $\p_1$ is isolated: $\q_1$ is the saturation of $I$ with respect to
  localization at $\p_1$. Since $xy\in I$, but $y\not\in \p_1$, so $x\in I^{ec}$, so
  $I^{ec}=(x)=\q_1$.

  Take $\q_{2,a}:=(x^2,y+ax)\supseteq (x,y)^2=\p_2^2$, so it is $\p_2$-primary. Clearly
  $I\subseteq \q_1\cap \q_{2,a}$. To see that this is an equality, assume
  $g_0x=g_1x^2+g_2(y+ax)$, then $x|g_2$ since we are in a UFD; but then $f\in I$, a
  contradiction.

  Finally, to contradict uniqueness, we must show that $\q_{2,a}\neq \q_{2,b}$ for $a\neq
  b$. If $\q_{2,a}=\q_{2,b}$, then $\q_{2,a}$ contains $(b-a)x$, so it contains $x$. But
  $R/\q_{2,a} \cong k[x]/(x^2)$, so $x\not\in \q_{2,a}$.

  We could have also chosen $\q_2 = (x^2,xy,y^\mu)$ for $\mu\ge 2$. This way, we get
  infinite non-uniqueness even if the ground field is not infinite.
 \end{example}
 \begin{example}
  Here are some example types:
  \begin{enumerate}
    \item $I$ is primary, so $I=I$ is a MPD.

    \item $R$ a UFD, and $(a)\neq (0)$. Then factor $a$ into irreducibles
    $u\pi_1^{r_1}\cdots \pi_n^{r_n}$, then $(a) = \bigcap (\pi_i^{r_i})$ by CRT. Each
    $(\pi_i^{r_i})$ is $(\pi_i)$-primary, and $(\pi_i)$ is an isolated prime.

    \item If $R$ is a Dedekind domain and $\a\subseteq R$ is an ideal. Then we get
    $\a=\p_1^{r_1}\cdots \p_n^{r_n} = \bigcap \p_i^{r_i}$ by CRT.
  \end{enumerate}
 \end{example}
 \begin{example}
  To appreciate machine computation, try doing the following by hand: $R=\QQ[x,y]$ and
  $I=\bigl(x^2-(y+1)^3,(y^2-1)^2\bigr)$. Then $\p_1=(x^2-8,y-1)$ and $\p_2(x,y+1)$, which
  are maximal (so they are both isolated since they are not comparable). The primary
  components are $\q_1=\bigl(x^2-12y+4,(y-1)^2\bigr)$ and $\q_2=\bigl( x^2 ,
  (y+1)^2\bigr)$.
 \end{example}

 \begin{theorem}
   If $R$ is noetherian, and $I=\q_1\cap \cdots\cap \q_n$ is any MPD. Then $I=\sqrt I$ if
   and only if $\q_1=\p_i$ for all $i$. In this case, $\p_1$,\dots, $\p_n$ are exactly
   the minimal primes over $I$; in particular, the MPD is unique (up to permutation).
 \end{theorem}
 \begin{proof}
   ($\Leftarrow$) Clear.

   ($\Rightarrow$) Assume $I=\sqrt I$. By (6.12 Lam), MPD is unique (there are no
   embedded primes). But we know that $\ass(R/I)=\{\p_1,\dots, \p_n\}$, the set of
   minimal primes over $I$. However, $I=\p_1\cap\cdots \cap \p_n$ since $I$ is radical.
   By uniqueness of the decomposition, $\q_i=\p_i$
 \end{proof}
 \begin{remark}[Non-existence of MPD]
   Let $R$ be von Neumann regular. Observe that every ideal is radical since $a^2\in
   I\Rightarrow a=a^2x\in I$. It follows that every primary ideal is prime. Take, for
   example, $R=k\times k\times \cdots$. Then the zero ideal has no primary decomposition:
   otherwise we would have $(0)=\p_1\cap \cdots \p_n$, which would imply that $\ass
   R\subseteq \{\p_1,\cdots, \p_n\}$, which is false because $\ass R$ is infinite.
 \end{remark}

 \subsection{\S 8 More Theorems on Noetherian Rings}

 \begin{theorem}[Krull's Intersection Theorem, preliminary version]
   Let $I$ is an ideal in a noetherian ring $R$, and $M$ is a finitely generated module.
   Define $N:= \bigcap_{n=0}^\infty I^n M$. Then $I\cdot N=N$.
 \end{theorem}
 There will be a very slick proof in the notes. Here is another one.
 \begin{proof}
   Assume $IN\subsetneq N$. Take a MPD for $IN$ as a submodule of $M$, say $IN=Q_1\cap
   \cdots \cap Q_n$, where $Q_i$ is $\p_i$-primary. Then $N\not\subseteq Q_i$ for some
   $i$. Then $\ass\bigl( (N+Q_i)/Q_i\bigr)\subseteq \ass(M/Q_i)=\{\p_i\}$, so we must
   have equality because over a noetherian ring, associated primes exist. Choose $k$ such
   that $\p_i^k M\subseteq Q_i$ (since $\p_i$ is finitely generated). Then $I\cdot
   \left(\frac{N+Q_i}{Q_i}\right)=0$, so $I\subseteq \ann
   \left(\frac{N+Q_i}{Q_i}\right)\subseteq \p_i$. But $N\subseteq I^k M$ by definition of
   $N$, and $I^k M\subseteq \p_i^k M \subseteq Q_i$. Contradiction.
 \end{proof}
 \stepcounter{lecture}
 \setcounter{lecture}{15}
 \section{Lecture 15}

 In exercise I.67, ``$(x,y)^2$'' should be ``$(x_1,x_2)^2$''.

 \begin{example}
   Let $I=(x^2-yz, x(z-1))\subset k[x,y,z]$. Then $\p_1=(x,z)=I:y(z-1)$, $\p_2=(x,y)=I:z(z-1)$,
   $\p_3=(x^2-y,z-1)=I:x$, and $I=\p_1\cap\p_2\cap\p_3$. In particular, $I$ is radical.
 \end{example}

 We want to understand what $IN=N$ means.

 Let $I\subset R$ and ${}_R M$ finitely generated. Let
 $E=\End_R (M)$, which is not commutative in general. We may view $M$ as an $E$-module
 ${}_E M$. Since every element in $R$ commutes with all of $E$, $E$ is an $R$-algebra (i.e.\
 There is a homomorphism $R\to E$ sending $R$ into the center of $E$).
 \begin{lemma}[Determinant Trick]
  \begin{enumerate}\item[]
    \item Every $\phi\in E$ such that $\phi(M)\subseteq IM$ satisfies a monic equation
    of the form $\phi^n+a_1\phi^{n-1} +\cdots + a_n=0$, where each $a_i\in I$, i.e.\
    $\phi$ is ``integral over $I$''.

    \item $IM=M$ if and only if $(1-a)M=0$ for some $a\in I$.
  \end{enumerate}
 \end{lemma}
 \begin{proof}
   (1) Fix a finite set of generators, $M=Rm_1+\cdots + Rm_n$. Then we have
   $\phi(m_i)=\sum_j a_{ij} m_j$, with $a_{ij}\in I$ by assumption. Let $A=(a_{ij})$.
   Then these equations tell us that $(I\phi-A)\vec{m}=0$. Multiplying by the adjoint of
   the matrix $I\phi-A$, we get that $\det(I\phi-A)m_i=0$ for each $i$. It follows that
   $\det(I\phi-A)=0\in E$. But $\det(I\phi-A)=\phi^n+a_1\phi^{n-1}+\cdots +a_n$ for some
   $a_i\in I$.

   (2) The ``if'' part is clear. The ``only if'' part follows from (1), applied to
   $\phi=\id_M$.
 \end{proof}
 \begin{remark}
   Determinant trick (part 2) actually includes Nakayama's Lemma, because if $I$ is in
   $\rad R$, $(1-a)$ is a unit, so $M=(1-a)M=0$.
 \end{remark}
 \begin{corollary}
   For a finitely generated ideal $I\subset R$, $I=I^2$ if and only if $I=eR$ for some
   $e=e^2$.
 \end{corollary}
 \begin{proof}
   ($\Leftarrow$) clear.

   ($\Rightarrow$) Apply determinant trick (part 2) to the case $M={}_R I$. We get
   $(1-e)I=0$ for some $e\in I$, so $(1-e)a=0$ for each $a\in I$, so $a=ea$, so $I$ is
   generated by $e$. Letting $a=e$, we see that $e$ is idempotent.
 \end{proof}
 \begin{corollary}[Vasconcelos-Strooker Theorem]
   For any finitely generated module $M$ over \emph{any} commutative $R$. If $\phi\in
   \End_R(M)$ is onto, then it is injective.
 \end{corollary}
 \begin{proof}
   We can view $M$ as a module over $R[t]$, where $t$ acts by $\phi$. Apply the
   determinant trick (part 2) to $I=t\cdot R[t]\subseteq R[t]$. We have that $IM=M$
   because $\phi$ is surjective, so $m =\phi(m_0)=t\cdot m_0\in IM$. It follows that
   there is some $th(t)$ such that $(1-th(t))M=0$. In particular, if $m\in  \ker \phi$,
   we have that $0=(1-h(t)t)m=1\cdot m=m$, so $\phi$ is injective.
 \end{proof}
 Now we can make Krull's intersection theorem more impressive:
 \begin{theorem}[Krull's Intersection Theorem]
   Let $M$ be finitely generated over a noetherian ring $R$, and let $I\subset R$ be an ideal.
   Define $N = \bigcap_{n\ge 0} I^nM$. Then $N=\{m\in M|(1-a)m=0\text{ for some } a\in
   I\}$.
 \end{theorem}
 \begin{proof}
   The inclusion $\supseteq$ is trivial since $m=am=a^2m=\cdots \in I^nM$ for each $n$.
   The other inclusion follows from the preliminary version and the determinant trick
   (part 2), noting that $N$ remains finitely generated.
 \end{proof}
 Here are some special cases of the Krull intersection theorem.
 \begin{corollary}
   If $I\subset R$ is a proper ideal in a noetherian domain, then $\bigcap_{n\ge 0} I^n=0$.
 \end{corollary}
 \begin{proof}
   The set $1-I$ consists of non-zero elements, and hence non-zero-divisors. Apply Krull
   to $M=R$.
 \end{proof}
 \begin{corollary}
   Let $M$ be finitely generated over a noetherian ring $R$. Let $J\subseteq \rad R$.
   Then $\bigcap_{n\ge 0} J^nM=0$.
 \end{corollary}
 \begin{proof}
   Again, $1-J$ consists of units.
 \end{proof}
 \stepcounter{lecture}
 \setcounter{lecture}{16}
 \section{Lecture 16}

 \noindent
 Exercise 68: You should assume that the characteristic of $k$ is not 2.\\
 Exercise 71: A non-Marot ring example.

 The easy form of Krull's intersection theorem is: If $(R,\m)$ is local noetherian and
 $M$ is finitely generated, then $\bigcap_{n\ge 0} \m^n M=0$.

 \begin{example}
   Counterexample if $R$ is not noetherian. Let $R=\QQ[x_1,x_2,\dots]/(x_{i+1}^2=x_i,x_1=0)$,
   with $\m=(x_1,x_2,x_3,\dots)$. Then $\m=\m^2=\m^3=\cdots$, so $\bigcap \m^n=\m\neq 0$.
 \end{example}
 \begin{example}
   Counterexample if $M$ is not finitely generated. Let $R=\mathbb{Z}_{(p)}$, which is local
   with maximal ideal $p\mathbb{Z}_{(p)}$. Take $M={}_{R}\QQ$, which is not finitely generated.
   Then we get that $\m M= pR\cdot M = p\QQ=\QQ=M$, so $\bigcap \m^n M=M\neq 0$.
 \end{example}

 Next we study a class of rings studied by Jean Marot (french guy). Recall that $\C(R)$
 is the set of non-zero-divisors. These elements are called regular.
 \begin{definition}
   An ideal $I$ is \emph{regular} if $I$ contains a regular element.
 \end{definition}
 \begin{definition}
   A ring $R$ is \emph{Marot}\index{Marot ring|idxbf} if every regular ideal can be
   generated by regular elements. (i.e.\ a regular ideal $I$ is generated by $I\cap
   \C(R)$.)
 \end{definition}
 Note that this class of rings includes integral domains (regular is the same as
 non-zero). Also, rings $R$ such that $\C(R)= U(R)$ are Marot (the only regular ideal is
 all of $R$). For example, zero-dimensional rings have this property. In particular,
 artinian rings are zero-dimensional.
 \begin{definition}
   $R$ has \emph{few zero divisors} if $\Z(R)$ is a \underline{finite} union of primes.
 \end{definition}
 \begin{lemma}
   Let $R$ be a ring with few zero divisors. Then for any $a\in R$ and any $b\in \C(R)$,
   there is some $r\in R$ such that $a+br\in \C(R)$. (i.e.\ every coset of $bR$ intersects
   $\C(R)$, provided $b$ is regular.)
 \end{lemma}
 \begin{proof}
   Write $\Z(R)=\bigcup_{i=1}^n \p_i$. We may assume there are no inclusions among the
   $\p_i$. After relabelling, we may assume that $a\in \p_i$ for $i\le k$ and $a\not\in
   \p_i$ for $i>k$. We may assume $k\neq 0$, lest $a$ be regular, in which case $r=0$
   works.

   If $\bigcap_{i=1}^k\p_i\smallsetminus \bigcup_{j=k+1}^n \p_j=\varnothing$, then by
   prime avoidance, there is some $j$ such that $\bigcap_i\p_i\subseteq \p_j$. Then since
   $\p_j$ is prime, we get $\p_i\subseteq \p_j$ for some $i$, contradicting the fact that
   there are no inclusions among the $\p$'s.

   Thus, we may choose $r\in \bigcap_{i=1}^k\p_i\smallsetminus \bigcup_{j=k+1}^n \p_j$,
   so $r\in \p_i$ exactly when $a\not\in \p_i$. Then we get $a+rb\in \C(R)$.
 \end{proof}
 \begin{theorem}
   Noetherian $\Rightarrow$ few zero divisors $\Rightarrow$ Marot.
 \end{theorem}
 \begin{proof}
   If $R$ is noetherian, then $\Z(R)$ is the union of the finitely many ``maximal
   primes'', $\bigcup_{\p\in \ass(R)^*} \p$.

   If $b\in I\cap \C(R)$, and $a\in I$, then by the Lemma, we get some $r_a$ so that
   $c_a=a+r_ab\in \C(R)\cap I$. Then $I$ is clearly generated by $b$, together with all
   the $c_a$.
 \end{proof}
 \[\xymatrix{
  & \text{Marot} \ar@{-}[d] \ar@{-}@/_/[ddl] \ar@{-}[dr]\\
  & \text{few 0-divs}\ar@{-}[d] \ar@{-}[dl] & \text{0-dim'l}\ar@{-}[dd]\\
  \text{domain} & \text{noetherian}\ar@{-}[dr]\\
  & & \text{artinian}
 }\]

 None of these implications is reversible. It doesn't take much thought to produce
 examples to demonstrate this.

 \begin{proposition}
   Suppose $R$ is Marot, and $I\subset R$ is a proper regular ideal. Assume
   \[
     \text{for every }a,b\in \C(R),\qquad ab\in I\Rightarrow a\in I \text{ or } b\in I
     \tag{$\ast$}
   \]
   Then $I$ is prime.
 \end{proposition}
 \begin{proof}
   Assume $I$ is not prime, with $x,y\not\in I$, but $xy\in I$. Then $I+(x)\supsetneq I$
   is regular as well, so it is generated by regular elements. Those generators cannot
   all lie in $I$, so there is some generator $a\not\in I$. Similarly, there is some
   regular generator $b\in (I+(y))\smallsetminus I$. Then $ab\in (I+(x))(I+(y))\subseteq
   I$, contradicting ($\ast$).
 \end{proof}
 \begin{theorem}[E.~D.~Davis]
   A commutative ring $R$ has few zero divisors if and only if the total ring of
   quotients $Q(R)$ is a semi-local ring. In particular, if $R$ is noetherian, $Q(R)$ is
   semi-local.
 \end{theorem}
 \begin{proof}
   ($\Rightarrow$) Write $\Z(R)=\bigcup_{i=1}^n \p_i$, with no inclusions among the
   $\p_i$. Then $Q(R)$ is the localization at the complement of this set, which is the
   semi-localization of $R$ at this finite set of primes, so $Q(R)$ is semi-local.

   ($\Leftarrow$) Assume $Q(R)$ is semi-local, with $\Max(Q(R))=\{\m_1,\dots, \m_n\}$.
   Form the contractions $\p_i=\m_i\cap R$ (recall that $R\hookrightarrow Q(R)$). We
   claim that each $\p_i$ consists of zero-divisors; otherwise, $\p_i$ would contain a
   regular element, which would become a unit upon localization. Now we will show that
   $\p_1\cup \cdots\cup \p_n\subseteq \Z(R)$ is an equality. If $r\in \Z(R)$, then
   $r\cdot a=0$ for some $a\neq 0$, so $rQ(R)\neq Q(R)$, so $r\in \m_i$ for some $i$.
   Then $r\in \p_i$, as desired.
 \end{proof}
 \stepcounter{lecture}
 \setcounter{lecture}{17}
 \section{Lecture 17}

 \subsection{Chapter II. Affine Varieties and the Nullstellensatz (``NSS'')}
 \S 1. Affine algebraic sets\\
 \S 2. General Topology\\
 \S 3. Zariski Prime Spectrum\\
 \S 4. Hilbert's Nullstellensatz

 \smallskip

 \begin{proposition}[``Going Up'']
   Let $R\subseteq S$ be commutative rings with unit, and let $S$ be finitely generated
   as an $R$-module. Then if $R$ is noetherian, so is $S$.
 \end{proposition}
 \begin{proof}
   ${}_R S$ is a noetherian, so ${}_S S$ is noetherian, i.e.\ $S$ is noetherian as a
   ring.
 \end{proof}
 \begin{theorem}[Eakin-Nagata-(Formanek)\footnote{Formanek did something that works
    for non-commutative rings.}, ``Going Down'']
   The converse of the above proposition is true.
 \end{theorem}

 \subsection{\S 1. Affine algebraic sets}

 We fix some fields $k\subseteq K$, fix $A=k[x_1,\dots,x_n]$, and let $K^n$ be $n$-space
 over $K$.

 Let $S\subseteq A$ be a subset. Define $V_K(S)=\{a\in K^n|f(a)=0 \text{ for all }f\in
 S\}$; we call this an algebraic $k$-set. Clearly, we may replace $S$ by the ideal it
 generates, so we will always take $S$ to be an ideal. Since $A$ is noetherian, $S$ is
 always finitely generated.

 On the other hand, if $Y\subseteq K^n$ is a subset, then we can define $I(Y)$, the
 ideal in $A$ of functions vanishing on $Y$. Note that $I(Y)$ is always a radical ideal.

 Both directions are inclusion-reversing. We always have $Y\subseteq
 V_K\bigl(I(Y)\bigr)$, trivially. Equality holds if and only if $Y=V_k($something). It is
 also clear that $J\subseteq I\bigl(V_K(J)\bigr)$. Equality holds if and only if
 $J=I$(something).

 \begin{definition}
   The \emph{Zariski $k$-topology} on $K^n$ has closed sets of the form $V_K(J)$.
 \end{definition}
 In this topology, the $k$-points are closed because $(a_1,\dots,a_n)$ is the vanishing
 set of $(x_1-a_1,\dots, x_n-a_n)\subset A$. You get a topology because $\bigcap_i
 V_K(J_i)=V_K\bigl(\sum_i J_i\bigr)$ (this is an arbitrary intersection!) and
 $V_K(J_1)\cup V_K(J_2)=V(J_1\cap J_2)$.

 \begin{enumerate}
   \item If $Y\subseteq K^n$ is a subset, then the closure of $Y$ is
   $V_K\bigl(I(Y)\bigr)$.

   \item If $J\subset A$, then $\sqrt J\subseteq I\bigl(V_K(J)\bigr)$. In general, this is not
   an equality.
 \end{enumerate}
 \begin{theorem}[Hilbert's Nullstellensatz]
   If $\bar k\subseteq K$, then $\sqrt J= I\bigl(V_K(J)\bigr)$.
 \end{theorem}
 We will prove this theorem in \S 4.

 The problem with general $k\subseteq K$ is as follows.
 \begin{example}
   Let $k=K=\RR$ and $J=(x^2+y^2)$. Then $J$ is a prime ideal, so $J=\sqrt J$. However,
   $V_K(J)=\{(0,0)\}$, so $I\bigl(V_K(J)\bigr)=(x,y)$, which is strictly larger than
   $\sqrt J$.

   Even worse, if $J=(x^2+y^2+1)$, then $J$ is still radical, but $V_K(J)=\varnothing$,
   so $I\bigl(V_K(J)\bigr)=A$.
 \end{example}
 \begin{definition}
   An \emph{affine $k$-algebra} is a finitely generated (as an algebra) commutative
   $k$-algebra. (i.e.\ these are homomorphic images of $k[x_1,\dots, x_n]$)
 \end{definition}
 By the Hilbert basis theorem, affine $k$-algebras are always noetherian.
 \begin{definition}
   If $Y$ is a $k$-algebraic set in $K^n$, then the \emph{$k$-coordinate ring $k[Y]$ of
   $Y$} is $A/I(Y)$.
 \end{definition}
 Since $I(Y)$ is always radical, $k[Y]$ is always reduced.
 \begin{definition}
   An algebraic $k$-set $Y$ is \emph{$k$-irreducible} if it is non-empty and cannot be
   written as the union of two proper closed subsets. We also call $Y$ a \emph{variety}.
 \end{definition}
 \begin{proposition}
   A $k$-algebraic set $Y\subseteq K^n$ is irreducible if and only if $I(Y)$ is prime if
   and only if $k[Y]$ is a domain.
 \end{proposition}
 In this case, we define $k(Y)=Q(k[Y])$ to be the function field of $Y$.
 \begin{warning}
   If you start with $J\in \spec A$, $V_K(J)$ need not be a variety!
 \begin{example}
   Let $k=K=\FF_2$ and let $J=(x+y)$, which is prime. Then $V_K(J)=\{(0,0),(1,1)\}$,
   which is not irreducible since $K^2$ is discrete! In particular,
   $I\bigl(V_K(J)\bigr)=(x,y)\cap (x+1,y+1)\supsetneq J$.
 \end{example}
 \end{warning}

 \subsection{\S 2. General Topology}
 \begin{proposition}
   For a topological space $X$, the following are equivalent.
   \begin{enumerate}
     \item Open sets in $X$ satisfy ACC.
     \item Every non-empty family of open sets has a maximal member.
     \item Every non-empty family of closed sets has a minimal member.
     \item Closed sets in $X$ satisfy DCC.
   \end{enumerate}
 \end{proposition}
 \begin{proof}
   Easy.
 \end{proof}
 \begin{definition}
   If any of the above hold, we call $X$ \emph{noetherian}.
 \end{definition}
 \begin{proposition}
   Noetherian spaces are compact.
 \end{proposition}
 \begin{proof}
   Given a cover of a noetherian space $X$, consider the family of finite unions. There
   is a maximal member, which must cover all of $X$ by maximality.
 \end{proof}
 \begin{corollary}
   For $k\subseteq K$, every $k$-algebraic set, with the Zariski topology, is noetherian
   and hence compact.
 \end{corollary}
 \begin{proof}
   Since $A$ is noetherian, $K^n$ is noetherian as a topological space. Finally, closed
   subsets of noetherian  spaces are noetherian.
 \end{proof}
 \stepcounter{lecture}
 \setcounter{lecture}{18}
 \section{Lecture 18}

 noetherian $\Rightarrow$ subspaces are noetherian $\Rightarrow$ subspaces are compact.
 If $K\supseteq k$, then the $k$-topology on $K^n$ is noetherian (hence compact).

 \begin{proposition}
   If $X$ is a non-empty topological space, then the following are equivalent.
   \begin{enumerate}
     \item $X$ is not the union of two proper closed subsets.
     \item Any two non-empty open sets intersect.
     \item Any non-empty open set is dense in $X$.
   \end{enumerate}
 \end{proposition}
 \begin{proof}
   easy exercise.
 \end{proof}
 In this case, we call $X$ \emph{irreducible}. In particular, ``irreducible subspace'' is
 meaningful (a subspace which is irreducible in the subspace topology).
 \begin{corollary}
   $Y\subseteq X$ is an irreducible subspace if and only if $\bbar Y$ is irreducible.
 \end{corollary}
 \begin{proof}
   Follows from the fact that an open set intersects $Y$ is and only if it intersects
   $\bbar Y$.
 \end{proof}
 \begin{example}
   Singleton subspaces (and therefore their closures) are irreducible.
 \end{example}
 \begin{definition}
   A maximal irreducible subset $Y$ of $X$ is called an \emph{irreducible component} of $X$.
 \end{definition}
 Such a $Y$ is always closed because $Y\subseteq \bbar Y$, which is irreducible, so by
 maximality, $Y=\bbar Y$.
 \begin{proposition}
   \begin{enumerate}
     \item Any irreducible set in $X$ is contained in some irreducible component.
     \item $X$ is the union of its irreducible components.
   \end{enumerate}
 \end{proposition}
 \begin{proof}
   (1) Zorn's Lemma.
   (2) Every point is in some irreducible component by (1).
 \end{proof}
 Note that in a Hausdorff space, any two points are separated by open sets, so they
 cannot be in an irreducible component together, i.e.\ only irreducible sets are points.

 \begin{theorem}
   If $X$ is noetherian, then the number of irreducible components is finite. If
   $\{X_i\}$ are the irreducible components, all of them are needed to cover $X$.
 \end{theorem}
 \begin{proof}
   If $X_i$ can be omitted, it is contained in the remaining union, so it must be
   contained in one of the other components (since it is irreducible), contradicting the
   fact that it is a maximal irreducible set.

   \anton{}
 \end{proof}
 \begin{theorem}
   Given $k\subseteq K$, let $X\subseteq K^n$ be a $k$-algebraic set. The irreducible
   components of $X$ (in the $k$-topology) are given by $V_K(\p_i)$, where the $\p_i$
   are the minimal primes over the ideal $I(X)$. Moreover, $\p_i=I\bigl(V(\p_i)\bigr)$.
 \end{theorem}
 Note that it would be wrong to say that $I\bigl(V(\p)\bigr)=\p$ for all primes because
 $V(\p)$ may be empty. Note that we don't use the Nullstellensatz.
 \begin{proof}
   check the proof in the notes \anton{}.
 \end{proof}

 \underline{Generic Points}: For a subset $Y\subseteq X$, a point $y\in Y$ is called a
 generic point of $Y$ if $\bbar{\{y\}}=Y$. Note that to have a generic point, $Y$ must be
 closed and irreducible. In general, these conditions are not sufficient! In classical
 algebraic geometry, even $k$-varieties need not have generic points!
 \begin{example}
   Let $k=\bar k$, and $Y=V(y-x^2)$, the parabola. $Y$ is irreducible, but it has no
   generic point in the $k$-topology because points are closed (since $k=\bar k$).
 \end{example}
 Classically, we take $k\subseteq K$ to have infinite transcendence degree (e.g.\
 $\QQ\subseteq \CC$). In this case, $k$-varieties in $K^n$ will have generic points. In
 the example above, take $K=\text{Frac}\left(\frac{k[s,t]}{(t^2-s)}\right)$. Then $(\bar
 s, \bar t)$ is a generic point for the parabola.

 \subsection{\S 3. Zariski Prime Spectrum}
 \begin{tabular}{c|c|c|}
    & algebraic & geometric\\ \hline
   Classical & $k[x_1,\dots, x_n]$ & $K^n$, $k$-algebraic sets, etc. \\ \hline
   Grothendieck & any (commutative) $R$ & $\spec R$
 \end{tabular}\\
 Define $\V(J)=\{\p\in \spec R| J\subseteq \p\}$.
 \begin{theorem}
  \begin{enumerate}\item[]
   \item Taking $\V(J)$ to be closed sets gives a topology on $\spec R$.
   \item The sets $\D(f) = \spec R\smallsetminus \V(f)$, $f\in R$, are a basis for
         the topology.
   \item $\spec R$ is a compact $T_0$ space.\footnote{Given two points, one of them
         (you don't know which) has an open neighborhood avoiding the other.}
  \end{enumerate}
 \end{theorem}
 \begin{proof}
  (1) This follows from $\bigcap \V(J_\alpha)=\V\bigl( \sum J_\alpha \bigr)$ and
  $\V(J_1)\cup \V(J_2)=\V(J_1\cap J_2)$.

  (2) An open set is of the form $\spec R\smallsetminus \V(J) = \bigcup_{f\in J}\D(f)$.

  (3) If $\p,\p'\in \spec R$ are distinct, then there is some $f\in \p\smallsetminus \p'$
  or $f\in \p'\smallsetminus \p$, say the former. Then $\D(f)$ contains $\p'$ but note
  $\p$. This proves $T_0$. If you have a cover, then refine it to a cover by open sets of
  the form $\D(f_\alpha)$. Then the $f_\alpha$ generate the unit ideal \anton{}, so a
  finite number of them give 1, so those $\D(f_\alpha)$ cover.
 \end{proof}
 \stepcounter{lecture}
 \setcounter{lecture}{19}
 \section{Lecture 19}

 $\spec R$ is connected if and only if $R$ has only trivial idempotents. All of the
 closed and open (\emph{clopen}) sets of $\spec R$ are $\V(e)$, where $e$ is an
 idempotent.

 \begin{proposition}
   For any ring $R$, the following are equivalent.
   \begin{enumerate}
     \item $\dim R=0$.
     \item $\spec R$ is $T_1$ (points are closed).
     \item $\spec R$ is $T_2$ (hausdorff).
     \item $\spec R$ is a Boolean space (compact, hausdorff, and totally disconnected).
     \item $\spec R$ is $T_4$ (normal).
   \end{enumerate}
 \end{proposition}
 \begin{example}
   $\spec \mathbb{Z} = \{(0),(2),(3),(5),(7),\dots\}$. For $n\neq 0$, $\V\bigl((n)\bigr) = \{p|
   p$ divides $n\}$, which is finite; it is clear that any finite set (not containing
   $(0)$) can be realized this way. So the non-empty open sets are the cofinite sets
   contining $(0)$.
 \end{example}
 \begin{example}
   Let $R$ be the semi-localization of $\mathbb{Z}$ at $\{p_1,\dots, p_r\}$. Now the non-empty
   open sets are all the subsets containing $(0)$.
 \end{example}
 \begin{example}
   If $R$ is a PID, then prime ideals are generated by irreducible elements. The
   non-empty open sets are still the cofinite sets containing $(0)$.
 \end{example}
 \begin{definition}
   For a subset $Y\subseteq \spec R$, we define $\I(Y):=\bigcap_{\p\in Y}\p$, which is a
   radical ideal in $R$.
 \end{definition}
 \begin{proposition}
   Here are some fairly easy results.
   \begin{enumerate}
     \item If $J\subset R$, $\I\bigl(\V(J)\bigr) = \sqrt J$.
     \item For $Y\subseteq \spec R$, $\V\bigl(\I(Y)\bigr)=\bbar Y$. In particular,
     $\bbar{\{\p\}}=\V(\p)$.
     \item $\p$ is a closed point if and only if it is a maximal ideal.
   \end{enumerate}
 \end{proposition}
 That is, we have an inclusion-reversing bijection between radical ideals and closed
 sets. Note that $\spec R$ is noetherian if and only if radical ideals satisfy ACC. In
 this case, the set of minimal primes is finite. Observe that if $R$ is noetherian, then
 so is $\spec R$, but the converse is false.
 \[\xymatrix @R=1.5pc{
 \left\{\raisebox{4pt}{\txt{radical\\ ideals}}\right\}\ar@/^/[r]^\V \ar@{<-}@/_/[r]_\I
 & \left\{\raisebox{4pt}{\txt{closed\\ sets}}\right\}\\
 \left\{\raisebox{4pt}{\txt{prime\\ ideals}}\right\}\ar@/^/[r]^\V \ar@{<-}@/_/[r]_\I \ar@{}[u]|{\cup \rule{.3pt}{5pt}}
 & \left\{\raisebox{4pt}{\txt{irreducible\\ closed sets}}\right\} \ar@{}[u]|{\cup \rule{.3pt}{5pt}}\\
 \left\{\raisebox{3pt}{\txt{minimal\\ primes}}\right\}\ar@/^/[r]^\V \ar@{<-}@/_/[r]_\I \ar@{}[u]|{\cup \rule{.3pt}{5pt}}&
 \left\{\raisebox{3pt}{\txt{irreducible\\ components}}\right\} \ar@{}[u]|{\cup \rule{.3pt}{5pt}}
 }\]

 \subsection{\S 4 Hilbert's Nullstellensatz}
 Fix a field $k$.
 \begin{definition}
   If $M$ is an $R$-module, we say that $M$ is \emph{module-finite} if $M$ is finitely
   generated as a module. If $S$ is an $R$-algebra, we say it is \emph{ring-finite} if it
   is finitely generated as an $R$-algebra.
 \end{definition}
 \begin{lemma} \label{lec19L:funcfieldnfin}
   Let $S=k(x_1,\dots, x_r)$, with $r\ge 1$. Then $S$ is not ring-finite over $k$.
 \end{lemma}
 \begin{proof}[Sketch of Proof]
   Assume not. Then $S=k[f_1/g,\dots, f_\ell/g]$ (we can choose a common denominator).
   But then every rational function can be written with denominator a power of $g$, which
   is clearly false.
 \end{proof}
 \begin{theorem}[Artin-Tate Theorem]
   Let $R\subseteq S\subseteq T$ be rings, with  $R$ noetherian, $T$ ring-finite
   over $R$ and module-finite over $S$. Then $S$ is ring-finite over $R$.
 \end{theorem}
 \begin{proof}[Sketch of Proof]
   We have $T=R[t_1,\dots, t_n]=\sum_{j=1}^m Sy_j$ (choose one of the $y_j$ to be 1), so we
   get $t_i=\sum s_{ij} y_j$ and $y_iy_j=\sum s_{ij\ell} y_\ell$. Consider
   $S_0:=R[s_{ij},s_{ij\ell}]\subseteq S$, which is noetherian by Hilbert's basis
   theorem, and it is ring-finite over $R$. Note that $T=\sum S_0 y_j$ by construction,
   so $T$ is module-finite over $S_0$. So $T$ is a noetherian module over $S_0$, so $S$
   is module finite over $S_0$ (as a submodule of a noetherian module-finite module). So
   $S$ is module-finite over a ring-finite guy over $R$, so it is ring-finite.
 \end{proof}
 \stepcounter{lecture}
 \setcounter{lecture}{20}
 \section{Lecture 20}

 Correction to cross-reference: p.~61, line~$-$12, the reference is to exercise I.70, not
 to I.55.

 We forgot to say that if $f:R\to S$ is a ring homomorphism, then $f^*:\spec S\to \spec
 R$ is defined as $f^*(\p)=\p^c=f^{-1}(\p)$. Then we have $(f^*)^{-1}\bigl(
 \V(J)\bigr)=\V\bigl(f(J)\bigr)$, so $f^*$ is continuous. We also get $(f\circ
 g)^*=f^*\circ g^*$, so $R\rightsquigarrow \spec R$ is a contravariant functor from
 the category of commutative unital rings to the category of compact $T_0$ spaces.

 \begin{remark}
   If $R$ and $S$ are not commutative, $f^{-1}(\p)$ is not prime in general!
 \end{remark}

 \begin{lemma}[Zariski]
   If $T$ is a ring-finite field extension of $k$, then $T$ is a module-finite extension
   of $k$.
 \end{lemma}
 \begin{proof}
   We have $T=k[t_1,\dots, t_n]$, and we need to prove that each $t_i$ is algebraic over
   $k$. Assume not. Then after relabeling, we may assume $t_1$,\dots, $t_r$ (with $r\ge
   1$) is a transcendence base for $T$ over $k$. Then $T$ is module-finite over
   $k(x_1,\dots, x_r)$ (by definition of transcendence base). By the Artin-Tate theorem,
   we get that $k(x_1,\dots, x_r)$ is ring-finite over $k$, contradicting Lemma
   \ref{lec19L:funcfieldnfin}.
 \end{proof}
 We can restate Zariski's lemma in the following way.
 \begin{theorem}
   Let $A$ be an affine $k$-algebra, and let $\m\in \Max(A)$, then $T=A/\m$ is a finite
   field extension of $k$.
 \end{theorem}
 \begin{proof}
   We have that $T$ is an affine $k$-algebra (i.e.\ it is ring-finite over $k$), and it
   is a field, so Zariski's lemma applies to give the desired result.
 \end{proof}
 \begin{corollary} \label{lec20C:1}
   If $f:B\to A$ is a $k$-algebra homomorphism of affine $k$-algebras. Then $f^*:\spec
   A\to \spec B$ takes closed points to closed points (i.e.\ takes $\Max A$ to $\Max B$).
 \end{corollary}
 \begin{proof}
   We need to prove that if $\m\in \Max A$, then $f^{-1}(\m)$ is maximal. We have
   injections $k\hookrightarrow B/f^{-1}(\m)\hookrightarrow A/\m$, and by the theorem,
   $A/\m$ is a finite extension of $k$. But we know that a ring inside an algebraic
   extension is a field (because the inverse of an element is a polynomial in that
   element by algebraicity).
 \end{proof}
 \begin{corollary}
   Let $\m\subset A=k[x_1,\dots, x_n]$, then $\m$ is maximal if and only if there is an
   \emph{algebraic point} $(a_1,\dots, a_n)\in \bar k^n$ such that
   $\m=\I\bigl((a_1,\dots, a_n)\bigr)$. In particular, if $k=\bar k$, $\m$ is maximal if
   and only if it is of the form $(x-a_1,\dots, x-a_n)$.
 \end{corollary}
 \begin{proof}
   If $\m$ is maximal, then we have an algebraic extension $k\hookrightarrow
   A/\m\stackrel{\varphi}{\hookrightarrow} \bar k$, and we have $\varphi(x_i)=a_i$ for
   some $a_i$. For all $f\in A$, we have $\phi(f)=f(a_1,\dots, a_n)$. If $f\in \m$, then
   clearly $\phi(f)=0$, so $f(a_1,\dots, a_n)=0$.

   Conversely, if $\m=\I\bigl((a_1,\dots, a_n)\bigr)$, then we have the evaluation map at
   $(a_1,\dots, a_n)$ inducing $A/\m\hookrightarrow \bar k$. Then $A/\m$ is a ring in an
   algebraic extension of $k$, so it is a field, proving that $\m$ is maximal.
 \end{proof}
 \begin{corollary}[Weak NSS]
   If $J\subset A=k[x_1,\dots, x_n]$ is proper, $V_{\bar k}(J)\neq \varnothing$.
 \end{corollary}
 \begin{proof}
   Enlarging $J$, we may reduce to the case $J=\m\in \Max A$, noting that $V_{\bar
   k}(\m)\subseteq V_{\bar k}(J)$. By the previous corollary, $(a_1,\dots, a_n)$ is an
   algebraic point on which all of $\m$ vanishes, as desired.
 \end{proof}
 \begin{corollary}
   Every maximal ideal $\m\subseteq A=k[x_1,\dots, x_n]$ can be generated by $n$
   irreducible polynomials of the form $f_1(x_1)$, $f_2(x_1,x_2)$, \dots, $f_n(x_1,\dots,
   x_n)$ ($f_i$ only uses the first $i$ variables).
 \end{corollary}
 \begin{proof}
   In the notes. \anton{}
 \end{proof}
 \begin{theorem}
   Every affine $k$-algebra $A$ is a \emph{Hilbert ring}, i.e.\ every prime ideal is an
   intersection (possibly infinite) of maximal ideals.
 \end{theorem}
 \begin{proof}
   Let $\p\in \spec A$, and let $s\not\in \p$; we wish to find a maximal ideal
   $\m\supseteq \p$ such that $s\not\in \m$. Consider the localization $A_s=A[s^{-1}]$,
   in which the extension $\p A_s$ of $\p$ is some proper ideal, so there is some maximal
   ideal $M\in \Max A_s$ such that $\p A_s\subseteq M$. Since $A$ and $A_s$ are both
   affine, $M^c\in \Max A$ by Corollary \ref{lec20C:1}, and $\p\subseteq M^c$, with
   $s\not\in M^c$, as desired.
 \end{proof}
 More generally, if $R$ is Hilbert, then any ring-finite extension $S$ of $R$ is Hilbert.
 The above theorem is the case where $R$ is a field.
 \stepcounter{lecture}
 \setcounter{lecture}{21}
 \section{Lecture 21}

 \begin{theorem}[Strong NSS]
   For any field $k$, let $J\subset A=k[x_1,\dots, x_n]$. Then $g\in A$ vanishes on $V_{\bar
   k}(J)$ if and only if $g\in \sqrt J$. That is, $I\bigl(V_{\bar k}(J)\bigr)=\sqrt J$.
 \end{theorem}
 \begin{proof}
   We need only to prove $\Rightarrow$, as the other direction is trivial. This is often
   done with the Rabinowich trick, but we will take a ring-theoretic approach. First note
   that $\sqrt J = \bigcap_{J\subseteq \p} \p =\bigcap_{J\subseteq \m}\m$ since $A$ is a
   Hilbert ring. So it suffices to check that $g\in \m$ for each $\m\supseteq J$. We saw
   last time that a maximal ideal is exactly the vanishing ideal of an algebraic
   point $a=(\alpha_1,\dots, \alpha_n)\in \bar k^n$, $\m=\I(\{a\})$. Since $J$ vanishes on
   $\{a\}$, $a\in V_{\bar k}(J)$. By assumption, $g(a)=0$, so $g\in \m$, as desired.
 \end{proof}
 \begin{remark}
   We used the Weak NSS in the proof that maximal ideals are vanishing ideals of
   algebraic points.
 \end{remark}
 \begin{corollary}
   Let $f,g\in A[x_1,\dots, x_n]$, and assume $f$ is square-free and non-zero. Then $f|g$
   if and only if $V_{\bar k}(f)\subseteq V_{\bar k}(g)$.
 \end{corollary}
 \begin{proof}
   $\Rightarrow$ is trivial. By the strong NSS applied to $J=(f)$. Since $g$ vanishes on
   $V_{\bar k}(J)$, $g^r=f\cdot h$ for some $h$. Each prime dividing $f$ must then divide
   $g$. Since $f$ is square-free, $f|g$.
 \end{proof}
 \begin{theorem}
   Let $J\subset A$, where $A$ is an affine $k$-algebra. Then $A/J$ is artinian if and only if
   $\dim_k A/J< \infty$. If $A$ is a polynomial ring, then this occurs if and only if
   $|V_{\bar k}(J)|$ is finite. Furthermore, $|V_{\bar k}(J)|\le \dim_k A/J$.
 \end{theorem}
 \begin{warning}
   In the literature, such a $J$ is called a ``zero-dimensional'' ideal. This is confusing
   because $A/J$ is artinian if it is zero-dimensional \emph{as a ring}, not as a
   $k$-vector space.
 \end{warning}
 \begin{proof}
   If $A/J$ is finite-dimensional over $k$, then it is clearly artinian. For the other
   direction, apply Akizuki-Cohen \anton{} to reduce to: An affine \emph{local} artinian
   $k$-algebra $(R,\m)$ is finite dimensional over $k$. If $R$ is a field, then we are
   done by Zariski's lemma. In particular, $\dim_k R/\m < \infty$. We also know that $\m$
   is nilpotent (the jacobson radical in an artinian ring is always nilpotent) and that
   $\dim_k \m^i/\m^{i+1} <\infty$ because $\m^i/\m^{i+1}$ is finitely generated over
   $R/\m$. It follows that $\dim_k R<\infty$.

   Now let's consider the case where $A=k[x_1,\dots, x_n]$. We want to show that $\dim
   A/J< \infty$ (i.e.~$A/J$ is artinian) if and only if $V_{\bar k}(J)$ is finite and
   that $|V_{\bar k}(J)|\le \dim_k A/J$. First reduce to the case $k=\bar k$; the
   conditions do not change when you tensor up to $\bar k$. We may also assume that $J$
   is radical; again, the conditions don't change when we replace $J$ by $\sqrt J$ (uses
   exercise I.47), noting that $|V_{\bar k}(J)|$ stays the same, and $\dim_k A/J$ gets
   smaller.

   Recall that $J=\sqrt J=\bigcap_{\m\supseteq J} \m$, and each $\m$ is of the form
   $(x_1-a_1,\dots, x_n-a_n)$, where $(a_1,\dots, a_n)\in V_k(J)$. For the implication
   $\Rightarrow$, we note that if $A/J$ is artinian, then it is semi-local, so there are
   only finitely many $\m$ containing $J$, so $V_k(J)$ is finite. For the implication
   $\Leftarrow$, we note that $|V_k(J)|<\infty$ is the number of maximal ideals
   containing $J$, so $A/J$ is semi-local with Jacobson radical equal to zero (since $J$
   is radical). By an earlier fact \anton{} $A/J=\frac{A/J}{\rad (A/J)}$ is the product of
   all its residue fields (all of which are $k$), so $\dim_k A/J = |V_{k}(J)|$.
 \end{proof}

 Connection between $k$-algebraic sets $Y\subseteq K^n$ and $\spec k[Y]$. There is a
 canonical map $\varphi:Y\mapsto \spec k[Y]$, but it is neither one-to-one nor onto in
 general.
 \stepcounter{lecture}
 \setcounter{lecture}{22}
 \section{Lecture 22}

 Recall that a typical maximal ideal in $k[x_1,\dots, x_n]$ is of the form $I(y)$, where
 $y\in \bar k^n$ is an algebraic point.

 Let $K/k$ be a fixed extension. As usual, we have the ideal $I(\{y\})$ for every $y\in
 K^n$, and we have $k[y]$, the coordinate ring $k[x_1,\dots, x_n]/I(\{y\}) = k[y_1,\dots,
 y_n]$, which is an affine $k$-algebra.
 \begin{definition}
   If $y,z\in K^n$, then we say that $y$ \emph{specializes to} $z$ (written
   $y\rightsquigarrow z$) if there is a $k$-algebra homomorphism $k[y]\to k[z]$ taking
   $y_i$ to $z_i$ (in particular, it is surjective).
 \end{definition}
 Clearly $y\rightsquigarrow z$ if and only if $I(y)\subseteq I(z)$. Furthermore, this
 occurs if and only if $z\in \bbar {\{y\}}=V_K\bigl(I(y)\bigr)$. Therefore, we can thing
 of $\bbar {\{y\}}$ as $\{z|y\rightsquigarrow z\}$.
 \begin{definition}
   If $y,z\in K^n$, $y\rightsquigarrow z$, and $z\rightsquigarrow y$, then we say that
   $y$ and $z$ are \emph{$k$-conjugate} (written $ y\leftrightsquigarrow z$).
 \end{definition}
 It follows immediately that $y\leftrightsquigarrow z$ if and only if $\bbar {\{z\}} =
 \bbar {\{y\}}$. In particular, $k$-conjugacy is an equivalence relation; we write $[y]$
 for the equivalence class of $y$. Note that $[y]\subseteq \bbar {\{y\}}$.
 \begin{definition}
   We say that $y=(y_1,\dots, y_n)\in K^n$ is \emph{$k$-algebraic} if each $y_i$ is
   algebraic over $k$.
 \end{definition}
 Note $y$ is algebraic if and only if $k[y]$ is a field ($\Leftarrow$ follows from
 Zariski's Lemma), which occurs if and only if $I(y)$ is maximal. Moreover, if $y$ is
 algebraic, then $y \rightsquigarrow z$ implies that $y\leftrightsquigarrow z$. i.e.\
 ``algebraic points cannot be further specialized''.\footnote{``Being algebraic is very
 special, and you cannot make it more special.''} So in the case when $y$ is algebraic,
 $[y]=\bbar {\{y\}}$.
 \begin{remark}
   Any $k$-point (\emph{$k$-rational point}) is always algebraic.
 \end{remark}
 \begin{example}
   $k=\QQ$ and $K=\QQ[\sqrt[3]{2}]$, with $y=\sqrt[3]{2}$. Then $[y]=\bbar
   {\{y\}}=V_K(x^3-2)=\{y\}$. So $y$ is a closed point, but not a $k$-rational point.

   If we take $K$ to be the normal hull (Galois hull) of $k(y)$, then $[y]=V_K(x^3-2) =
   \{y,\omega y,\omega^2 y\}$, where $\omega^3=1$.
 \end{example}
 \begin{example}
   Choose $k$ a field which is not perfect, with characteristic $p>0$. Then there is some
   $a\in k\smallsetminus k^p$. Take $y\in \bar k$ such that $y^p=a$. Then consider any
   $K$ which contains $y$. Then $[y]=V_K(x^p-a)=\{y\}$, so $y$ is closed!
 \end{example}

 Given a $k$-algebraic set $Y\subseteq K^n$, we can construct a map $\varphi:Y\to \spec
 k[Y]$, given by $y\mapsto I(y)$ (which is always prime). Since $I(Y)\subseteq I(y)$, we
 get that $I(y)\in \spec k[Y]$. Let $\varphi(y)=\p_y= I(y)/I(Y)$. The map $\varphi$ is
 always continuous. To see this, consider a closed set $\V(\bar J)\subseteq \spec k[Y]$,
 where $J\supseteq I(Y)$. Then $\varphi^{-1}\bigl(\V(\bar J)\bigr) = Y\cap V(J)$, which
 is closed.

 In general, $\varphi$ is neither 1-to-1 nor onto. Let's describe the image and fibers of
 $\varphi$.

 \begin{theorem}
   Let $y,z\in  Y$. Then
   \begin{enumerate}
     \item $\varphi(y)=\varphi(z)$ if and only if $y\leftrightsquigarrow z$. In
     particular, the fibers of $\varphi$ are the $k$-conjugacy classes in
     $Y$.\footnote{Since $Y$ is closed, a conjugacy class that intersects $Y$ must be
     contained in $Y$.}

     \item $y\in Y_{alg}$ (i.e.~$y$ is algebraic) if and only if $\varphi(y)\in \Max
     k[Y]$. In particular, $\varphi(Y_{alg})=\Max k[Y]$ and the fiber containing $y$ is
     $\bbar {\{y\}}$.

     \item A prime $\p/I(Y)\in \spec k[Y]$ is in the image of $\varphi$ if and only if
     $\p=I\bigl(V_K(\p)\bigr)$ and $V_K(\p)$ has a generic point.
   \end{enumerate}
 \end{theorem}
 \begin{proof}
   Items 1 and 2 are clear. Let's do $\Rightarrow$ for 3. Suppose $\p/I(Y)$ is in the
   image of $\varphi$, so $\p=I(y)$, then $\p=I\bigl(V_K(\p)\bigr)$ by \anton{}. Since
   $V_K(\p) = V_K\bigl( I(y)\bigr)$, $y$ is a generic point for $V_K(\p)$. The converse
   is dry formal work \anton{}.
 \end{proof}
 Special cases of 3 above:
 \begin{enumerate}
   \item if $\bar k \subseteq K$, then we can delete the hypothesis
   $\p=I\bigl(V_K(\p)\bigr)$ because of the NSS.

   \item If $K$ is a ``universal domain'' ($K=\bar K$, and $K/k$ has infinite
   transcendence degree), then $\varphi$ is onto because we can remove the condition that
   $V_K(\p)$ has a generic point.

   \item If $K=\bar k$, $\im \varphi = \Max (k[Y])$. Moreover, there is a 1-to-1
   correspondence $\{k$-subvarieties of $Y\}\leftrightarrow \spec k[Y]$, with
   $V\leftrightarrow I(V)/I(Y)$.

   \item If $K=\bar k=k$, then $\varphi(a_1,\dots, a_n)=(x_1-a_1,\dots, x_n-a_n)/I(Y)$.
 \end{enumerate}
 \stepcounter{lecture}
 \setcounter{lecture}{23}
 \section{Lecture 23}

 \subsection{Chapter III. Integral Extensions and Normal Domains.}% Dimension Theory.}
 \S 1. Going up Theorem. (Cohen-Siedenberg theorems)\\

 \subsection{\S 1. Going Up Theorem}
 Let $R\subseteq S$ be a ring extension, and let $I\subset R$.
 \begin{definition}
   We say $s\in S$ is \emph{integral over $I$} if there is a monic $p\in I[x]\subseteq
   R[x]$ so that $p(s)=0$. If every $s\in S$ is integral over $R$, then we say that $S/R$
   is an \emph{integral extension}.
 \end{definition}
 \begin{example}
   If $S$ and $R$ are fields, then $S/R$ integral is the same as $S/R$ being algebraic.
 \end{example}
 \begin{example}
   If every $s\in S$ satisfies $s^{n(s)}=s$ (with each $n(s)\ge 2$), then $S$ is integral
   over any sub-ring. The same is true if $S$ is any ring of algebraic integers (as soon
   as all the coefficients are integers, we're on the gravy train).
 \end{example}
 \begin{example}
   If $d\in \mathbb{Z}$, then $\sqrt d$ is always integral over $\mathbb{Z}$. More interestingly, if
   $d=1+4b$, then $\alpha=\frac{1+\sqrt d}{2}$ is integral over $\mathbb{Z}$ because
   $\alpha^2-\alpha-b=0$.
 \end{example}
 \begin{example}
   Let $M\subseteq R$ be a multiplicatively closed set, and let $S/R$ be an integral
   extension, then $M^{-1}S/M^{-1}R$ is integral.
 \end{example}
 \begin{proposition}
   If $S/R$ is a ring extension, and $s\in S$, then the following are equivalent.
   \begin{enumerate}
     \item $s$ is integral over $R$.
     \item $R[s]$ is module-finite over $R$.
     \item $s\in T$ for some \underline{ring} $T$ between $R$ and $S$ that is
     module-finite over $R$.
     \item There is a faithful $R[s]$-module $M$ that is module-finite over $R$.
   \end{enumerate}
   In particular, if $S/R$ is module-finite, then it is integral.
 \end{proposition}
 \begin{proof}
   $1\Rightarrow 2\Rightarrow 3\Rightarrow 4$ are clear. $4\Rightarrow 1$ follows from
   the determinant trick: we get $R[s]\hookrightarrow \End_R(M)$ \anton{finish}.
 \end{proof}
 \begin{corollary}
   If $s_1,\dots, s_n$ are integral over $R$, then $R[s_1,\dots, s_n]$ is module-finite
   over $R$. In particular, the elements of $S$ that are integral over $R$ form a
   sub-ring of $S$, called the \emph{integral closure} of $R$ inside $S$.
 \end{corollary}
 \begin{corollary}[Transitivity of integrality]
   If $T/S$ and $S/R$ are integral, then $T/R$ is integral.
 \end{corollary}
 \begin{proposition}
   Let $I\subset R\subseteq S$, and let $C$ be the integral closure of $R$ in $S$. Then the
   elements of $S$ that are integral over $I$ are exactly $\sqrt{I\cdot C}$ (the radical
   in $C$, though it doesn't matter). In particular, it is an ideal in $C$.
 \end{proposition}
 \begin{remark}
   Note that the elements integral over $I$ are the same as the elements integral over
   $\sqrt I$. If $S=R$, then the result says that elements of $R$ integral over $I$ are
   exactly $\sqrt I$.
 \end{remark}
 \begin{proof}
   Let $c\in C$ be integral over $R$, so $c^n+a_1c^{n-1}+\cdots +a_n=0$ with $a_j\in I$.
   It follows immediately that $c\in \sqrt {I\cdot C}$. For the other containment, let
   $c\in \sqrt {I\cdot C}$. Let $c^n=b_1c_1+\cdots + b_mc_m$, where $b_j\in I$ and
   $c_j\in C$. Then $M=R[c,c_1,\dots, c_m]$ is module-finite over $R$. Now $c^nM\subseteq
   \sum b_iM\subseteq I\cdot M$. By the determinant trick, $c^n$ is integral over $I$.
   Hence $c$ is integral over $I$.\anton{why not just use the proposition above}
 \end{proof}
 \begin{lemma}
   If $I\subset R$ is a proper ideal and $S/R$ is integral, then $I\cdot S\neq S$.
 \end{lemma}
 \begin{proof}
   Assume not, so $1=a_1s_1+\cdots+a_ns_n$, with $a_j\in I$ and $s_j\in S$. Then
   $M=R[s_1,\dots, s_n]$ is module-finite over $R$. Now we have $I\cdot M=M$. By the
   determinant trick, there is some $a\in I$ so that $(1-a)M=0$, so $1=a\in I$,
   contradicting that $I$ is proper.
 \end{proof}
 \stepcounter{lecture}
 \setcounter{lecture}{24}
 \section{Lecture 24}

 Typos: p 73 (proof 4.1) ``$g^m$ prime to $g$'' should be ``$g^m$ prime to $1+g$''\\
 p 76, ref to $\ast$ in rmk 4.15 should ref to 4.16.

 \begin{example}
   $k\subsetneq R\subseteq S=k[x]$, then $S/R$ is integral because $x$ is integral over
   $R$, since $R$ must contain some monic polynomial in $x$.
 \end{example}
 \begin{theorem}[Going Up]
   Let $S/R$ be integral.
   \begin{enumerate}
     \item (Lying over) For every $\p\in \spec R$, there is some $\mathfrak{P}\in \spec
     S$ so that $\mathfrak{P}\cap R=\p$. This means that $\spec S\to \spec R$ is
     surjective.

     \item (Going up) If $I\subseteq \p\subseteq R$ and $J\subseteq S$ lying over $I$,
     then there is some prime $\mathfrak{P}\supseteq J$ lying over $\p$.

     \item (Incomparability) If $\mathfrak{P}\subsetneqq \mathfrak{P}'$, then $\mathfrak{P}\cap R\subsetneqq
     \mathfrak{P}'\cap R$.
   \end{enumerate}
 \end{theorem}
 \begin{proof}
   (1) First assume $(R,\p)$ is local, then $\p S\neq S$, so $\p S$ is in some maximal
   ideal $\m\subseteq S$, then $\m\cap R$ contains $\p$ and is prime (in particular, not
   all of $R$), so it is equal to $\p$. In the general case, just localize at $\p$ first.

   (2) Pass to the integral extension $R/I\hookrightarrow S/J$ and apply Lying over.

   (3) Pass to $R/\p\hookrightarrow S/\mathfrak{P}$ to assume $\p=0$ and $\mathfrak{P}=0$.
   Then we need to prove that if $\mathfrak{P}'\neq 0$, then $\p:=R\cap \mathfrak{P}'\neq
   0$. Choose $x\in \mathfrak{P}'\smallsetminus \{0\}$; then we have
   $x^n+a_1x^{n-1}+\cdots + a_n=0$, with $n$ minimal. Then
   $a_n=-x(x^{n-1}+a_1x^{n-2}+\cdots )\in \mathfrak{P}'\cap R$. Finally, since $n$ is
   minimal and $S$ ($S/\mathfrak{P}$) is a domain, $a_n\neq 0$.
 \end{proof}
 \begin{corollary}
   Let $S/R$ be integral.
   \begin{enumerate}
     \item If $\mathfrak{P}\in \spec S$, then $\mathfrak{P}$ is maximal if and only if
     $\mathfrak{P}\cap R\in \Max R$.

     \item $\rad R = R\cap \rad S $.

     \item If $S$ is a domain, then $R$ is a field if and only if $S$ is a field.

     \item If $S$ is (semi-)local, then so is $R$.
   \end{enumerate}
 \end{corollary}
 We also get the following analog of the Eakin-Nagata Theorem
 \begin{corollary}
   Let $R\subseteq S$ be rings, such that $S$ is module-finite (in particular integral)
   over $R$. Then $R$ is artinian if and only if $S$ is artinian.
 \end{corollary}
 \begin{proof}
   $\Rightarrow$ A finitely generated module over an artinian ring is artinian, so ${}_R
   S$ is artinian, so ${}_S S$ is artinian.

   $\Leftarrow$ If $S$ is artinian, then it is noetherian, then by Eakin-Nagata, $R$ is
   also noetherian. Since $S$ is artinian, every prime is maximal, so $R$ is also
   0-dimensional. By Akizuki, $R$ is artinian.
 \end{proof}

 \subsection{\S 2. Going Down Theorem and Krull Dimension}
 \begin{definition}
   Let $Q(R) = \bigl(\C(R)\bigr)^{-1} R$ be the total ring of fractions of $R$. We say
   that $R$ is \emph{integrally closed} if it is equal to its integral closure in $Q(R)$.
   If $R$ is an integrally closed domain, then we say it is a \emph{normal domain}.
 \end{definition}
 \begin{remark}
   Any localization of a normal domain is again a normal domain.
 \end{remark}
 \begin{proposition}
   Any UFD is a normal domain.
 \end{proposition}
 \begin{proof}
   Let $a/b$ be integral over $R$. We may assume $a/b$ is in lowest terms ($a$ and $b$
   have no common prime divisors), with $b$ not a unit. Then we have
   $(a/b)^n+c_1(a/b)^{n-1}+\cdots+ c_n=0$, so $a^n+c_1 a^{n-1}b+\cdots + c_nb^n=0$. It
   follows that $b|a^n$, so some prime dividing $b$ divides $a$, a contradiction.
 \end{proof}
 \begin{lemma}
   Let $R$ be a normal domain with $Q(R)=K$, let $L$ be a field extension of $K$, and let
   $s\in L$ be algebraic over $K$ with minimal polynomial $f(x)=x^n+c_1x^{n-1}+\cdots
   + c_n\in K[x]$. Given $I\subset R$, $s$ is integral over $I$ if and only if $c_i\in \sqrt
   I$ for all $i$. In particular, $s$ is integral over $R$ if and only if $f(x)\in R[x]$.
 \end{lemma}
 \begin{proof}
   ($\Leftarrow$) Suppose $c_i \sqrt I$ for all $i$. Then $s$ is integral over $\sqrt I$,
   and so integral over $I$.

   ($\Rightarrow$) Assume $s$ is integral over $I$. Let $(x-s_1)\cdots (x-s_n)$ be a
   complete factorization of $f$ is $\bbar K[x]$, say with $s_1=s$. Then $K(s_j)\cong
   K(s)$ for each $j$, so each $s_j$ is integral over $I$ \anton{you can see this easier
   because the minimal poly must divide the monic}. Each $c_j$ is an elementary symmetric
   function in the $s_i$, so $c_j$ is integral over $I$. Since $R$ is integrally closed,
   $c_j\in R$. Finally, the only elements of $R$ that are integral over $I$ are in $\sqrt
   I$.
 \end{proof}
 \stepcounter{lecture}
 \setcounter{lecture}{25}
 \section{Lecture 25}

 \begin{lemma}[Contracted prime criterion, II.3.15]
   If $f:R\to C$, then $\p\in \spec R$ is a contracted prime (is $f^{-1}$ of a prime) if
   and only if $\p=\p^{ec}$.
 \end{lemma}
 \begin{lemma}
   If $S/R$ is integral, $I\subset R$, and $s\in S$. Then $s$ is integral over $I$ if and only
   if $s\in \sqrt{I\cdot S}$ if and only if the non-leading coefficients of the
   irreducible polynomial of $s$ are in $\sqrt I$ (this last part assumes that $R$ is
   normal and $S$ is a domain).
 \end{lemma}

 \begin{theorem}[Going Down]
   If $S/R$ is integral, with $S$ a domain and $R$ a normal domain, then for every pair
   of primes $\p\subseteq \p'\subseteq R$ and prime $\mathfrak{P}'$ lying over $\p'$,
   there is a prime $\mathfrak{P}\in \spec S$ contained in $\mathfrak{P}'$ which lies
   over $\p$.
 \end{theorem}
 \begin{proof}
   It suffices to show that $\p$ is a contracted prime under $f:R\to
   C=S_{\mathfrak{P}'}$; that is, to show that $\p=\p^{ec}=R\cap \p C$. If not, there is
   an element $r\in R\cap \p C$ but $r\not\in \p$. We can write $r= s/t$ where $s\in \p
   S$ and $t\in S\smallsetminus \mathfrak{P}'$.

   picture

   Let the minimal polynomial of $s$ over $K=Q(R)$ be $s^n+c_1s^{n-1}+\cdots + c_n=0$.
   Then a minimal equation for $t$ over $K$ is obtained by dividing by $r^n$, so
   $t^n+\frac{c_1}{r} t^{n-1} + \cdots + \frac{c_n}{r^n}=0$; let $c_i/r^i=d_i$. Since $t$
   is integral over $R$, the $d_i$ are in $R$. We have that $s\in \p S$, so $s$ is
   integral over $\p$, so $c_i\in \sqrt \p=\p$. Then $c_i=d_i r^i$, and $r\not\in \p$, so
   $d_i\in \p$, so $t$ is integral over $\p$. Thus, $t\in \sqrt {\p S}\subseteq
   \mathfrak{P}'$, contradicting that $t\not\in \mathfrak{P}'$.
 \end{proof}

 \underline{Krull dimension and height}.
 \begin{definition}
   If $\p\in \spec R$, then the \emph{height} of $\p$ is the supremum of lengths of
   chains of primes contained in $\p$; $\p\supsetneq \p_1\supsetneq \cdots\supsetneq
   \p_n$ has length $n$. Some people use the words \emph{rank} or \emph{altitude} instead
   of height.
 \end{definition}
 In particular, $ht(\p)=0$ means that $\p$ is a minimal prime. If $R$ is a domain, then
 note that $(0)$ is a prime, so the primes that you want to call ``minimal'' are actually
 height 1.
 \begin{definition}
   The \emph{(Krull) dimension} of a ring $R$ is the supremum of heights of primes in
   $R$.
 \end{definition}

 \begin{theorem}
   Let $S/R$ be integral, then
   \begin{enumerate}
     \item $\dim R = \dim S$, and
     \item for $\mathfrak{P}\in \spec S$, $\p=\mathfrak{P}\cap R$, then
     $ht(\mathfrak{P})\le ht(\p)$, with equality if $S$ is a domain and $R$ is normal.
   \end{enumerate}
 \end{theorem}
 \begin{proof}
   (1) By incomparability, prime chains in $S$ contract to prime chains in $R$, so $\dim
   R\ge \dim S$. By going up, prime chains in $R$ lift to prime chains in $S$, so $\dim
   R\le \dim S$.

   (2) Incomparability also shows that $ht(\mathfrak{P})\le ht(\p)$. If $S$ is a domain
   and $R$ is normal, then going down applies, so a prime chain from $\p$ lifts to a
   prime chain from $\mathfrak{P}$ (without going down, you wouldn't get a chain with
   upper bound $\mathfrak{P}$).
 \end{proof}
 In particular, if $\mathfrak{P}$ contracts to a minimal prime, then it is minimal. The
 converse is not true in general.

 \subsection{\S 3. Normal and Completely Normal Domains}
 \begin{lemma}
   Let $f\in S[x]$ be monic, then $f=\prod (x-a_i)\in S'[x]$ for a suitable ring
   extension $S'/S$.
 \end{lemma}
 \begin{proof}
   Induct on $\deg f=n$, simultaneously over all rings. If $n=1$, we're done. If $n>1$,
   then consider the embedding $S\hookrightarrow S[t]/\bigl(f(t)\bigr)$. Then we use the
   division algorithm for monic polynomials, we get $f(x)=(x-t)g(x)$, and $g(x)$ has
   smaller degree, so we can split it in some extension.
 \end{proof}
 \begin{lemma}[Monicity lemma]
   Let $R$ be integrally closed in $S$, and let $f\in S[x]$ and $h\in R[x]$ be monic.
   Then if $f|h$ in $S[x]$, then $f\in R[x]$.
 \end{lemma}
 \begin{proof}
   By the previous lemma, we have $f=\prod (x-a_i)$ in some $S'$. Then each $a_i$
   satisfies the monic $h$, so $a_i$ are integral over $R$. Let $b_i$ be the coefficients
   of $f$, then the $b_i$ are symmetric functions in the $a_j$, so the $b_i$ are integral
   over $R$. Since $R$ is integrally closed, $f\in R[x]$
 \end{proof}
 \stepcounter{lecture}
 \setcounter{lecture}{26}
 \section{Lecture 26}

 \begin{example}
   Counterexample for Going down when $R$ is normal, but $S$ is not a domain. Take
   $S=\mathbb{Z}\times \mathbb{Z}_2$ and let $R=\mathbb{Z}\cdot 1_S$. The minimal primes in $S$ are
   $\P''=(0)\times \mathbb{Z}_2$ and $\P'=\mathbb{Z}\times (0)$. $\P''$ contracts to
   the minimal prime $(0)$, but $\P'$ contracts to $\p'=2R$, which is not minimal; that
   is, we have $ht(\P')\lneq ht(\p')$.
 \end{example}

 \begin{proposition}
   If $S/R$ is a ring extension, and $C\subseteq S$ is the integral closure of $R$, then
   $C[x]$ is the integral closure of $R[x]$ in $S[x]$.
 \end{proposition}
 \begin{proof}
   Clearly $C[x]$ is integral over $R[x]$ because $C$ and $x$ are integral over $R[x]$.
   Now it suffices to show that $C[x]$ is integrally closed; i.e.~we've reduced to the
   case where $R$ is integrally closed in $S$, and we'd like to show that $R[x]$ is
   integrally closed in $S[x]$. Let $f\in S[x]$ be integral over $R[x]$, with
   $G(t)=t^n+g_{n-1}(x)t^{n-1}+\cdots+g_0(x)\in R[x][t]$ monic so that $G(f)=0$. Choose
   $r> \max\{\deg f, \deg g_i\}$, then $G(x^r)\in R[x]$ is monic! Now we adjust $f$ in
   the following way: define $f_0=x^r-f\in S[x]$, which is monic. Define
   $H(t):=G(x^r-t)\in R[x][t]$, so $H(f_0)=G(f)=0$. Say $H(t)=(-1)^nt^n +
   h_{n-1}(x)t^{n-1}+\cdots+h_0(x)$, with $h_i\in R[x]$. We have $H(0)=h_0(x)=G(x^r)$ is
   monic in $R[x]$, and since $H(f_0)=0$, we get that $f_0|h_0$. By the Monicity lemma
   from last time, $f_0\in R[x]$. Then $f=x^r-f_0\in R[x]$.
 \end{proof}
 \begin{theorem}
   A domain $R$ is normal if and only if $R[x]$ is normal.
 \end{theorem}
 \begin{proof}
   $(\Leftarrow)$ If $\alpha\in K:=Q(R)$ is integral over $R$ and $R[x]$ is integrally
   closed in $K(x)=Q(R[x])$, then $\alpha\in R[x]$ and $\alpha\in K$. But $R[x]\cap K=R$,
   so $R$ is normal.

   $(\Rightarrow)$ If $R$ is integrally closed in $K$, then by the proposition, $R[x]$ is
   integrally closed in $K[x]$. Since $K[x]$ is a UFD, it is integrally closed in
   $Q(K[x])=K(x)$. It follows that $R[x]$ is integrally closed in $K(x)=Q(R[x])$.
 \end{proof}
 \begin{definition}
   If $S/R$ is a ring extension. An element $s\in S$ is \emph{almost integral} over $R$
   if $R[s]\subseteq T\subseteq S$, where $T$ is some module-finite module over
   $R$.\footnote{Krull made this definition in 1928. I didn't make it up last night.}
 \end{definition}
 \begin{remark}
   The definition depends on $S$ (not just on $s$), because $R$ is to be found in $S$.
   Note that the definition is the same as $\{s^i|i\ge 0\}\subseteq T$. Integral implies
   almost integral (we can take $T=R[s]$). If $R$ is noetherian, then if $s$ is almost
   integral over $R$, it is integral over $R$ (since $T$ is module-finite, so is $R[s]$).
   If $s$ is almost integral over $R$, then it is also almost integral over any
   $R'\supseteq R$.
 \end{remark}
 \begin{example}[$D+(x)$ construction] Almost integral is strictly weaker than integral.
  Let $D$ be a normal domain with $K=Q(D)$, and let $R=\{f(x)\in K[x]|f(0)\in D\} =
  D+x\cdot K[x]$. Let $S=K(x)$. Any element of $s\in K\subseteq K(x)$ is almost integral
  over $R$ because $s^i x\in R$, so $s^i\in \frac{1}{x}R = T\subseteq S$. But if $s\in
  K\smallsetminus D$, then $s$ is not integral over $R$; otherwise there would be some
  monic $s^n+f_1(x)s^{n-1}+\cdots f_n(x)=0$. Taking $x=0$, we get that $s$ is integral
  over $D$, so it is in $D$, a contradiction. Note that this implies that $R$ is not
  noetherian.
 \end{example}
 \stepcounter{lecture}
 \setcounter{lecture}{27}
 \section{Lecture 27}

 Do some (two) exercises (from chapter II)!

 Then we can form the ``complete integral closure of $R$ in $S$''.
 \begin{proposition}
   If $s_1,\dots, s_n\in S$ are almost integral over $R$, then $R[s_1,\dots, s_n]$ lies
   in some finitely generated $R$-module in $S$. In particular, the complete integral
   closure of $R$ in $S$, $C:=\{s\in S| s\text{ almost integral over }R\}$, is a subring
   of $S$. If $C=R$, we say $R$ is completely integrally closed in $S$.
 \end{proposition}
 \begin{definition}
   If $S=Q(R)$, then we denote the integral closure of $R$ in $S$ by $R^*$ and the
   complete integral closure of $R$ in $S$ by $R^\dag$. We call $R^\dag$ the
   \emph{complete integral closure} of $R$. If $R=R^\dag$, we say that $R$ is
   \emph{completely integrally closed}.
 \end{definition}
 \begin{remark}
   Note that if $R$ is completely integrally closed, then it is integrally closed since
   $R\subseteq R^*\subseteq R^\dag$. If $R$ is noetherian, $R^\dag=R^*$. Note that
   $R^\dag=\{s\in Q(R)|\text{there is some }d\in \C(R) \text{ such that }ds^i\in R\text{
   for all }i\}$.
 \end{remark}

 If $R$ is a completely integrally closed domain, it is called a \emph{completely normal
 domain}.
 \begin{proposition}
   If $R$ is a UFD, then it is a completely normal domain.
 \end{proposition}
 \begin{proof}
   Let $K=Q(R)$, and assume $a/b\in K\smallsetminus R$ is almost integral over $R$. We
   may assume that $a$ and $b$ have no common prime factors, and that there is some prime
   $\pi$ dividing $b$ (lest $a/b\in R$). Choose a non-zero $d\in R$ so that $da^i/b^i\in
   R$ for all $i\ge 0$, so $d a^i=b^i$. It follows that $\pi^i|d$ for all $i$,
   contradicting unique (finite) factorization of $d$.
 \end{proof}
 Now let's pursue the completely normal analogues of the results on $R[x]$ when $R$ is
 normal.
 \begin{proposition}
   Let $C$ be the complete integral closure of $R$ in $S$, then $C[x]$ is the complete
   integral closure of $R[x]$ in $S[x]$.
   \marginpar{\hspace*{-2ex}$\xymatrix@C=1pc{ S \ar@{-}[d] & S[x] \\ C \ar@{-}[d]
   & C[x] \\ R & R[x] }$}
 \end{proposition}
 \begin{proof}
   Elements of $C[x]$ are clearly almost integral over $R[x]$ since $C$ and $x$ are
   almost integral over $R[x]$ (and almost integral elements form a ring). If $f(x)=\sum
   s_j x^j\in S[x]$ is almost integral over $R[x]$, then we'd like to show that each
   $s_j$ is almost integral over $R$. Fix $g_1,\dots, g_m\in S[x]$ such that $f(x)^i\in
   \sum_{k=1}^m R[x]\cdot g_k$ for all $i$. The leading term of $f(x)^i$ is $s_n^i
   x^{n\cdot i}$. It follows that all powers of $s_n$ lie in the $R$-module generated by
   all the coefficients of the $g_k$, so it is almost integral over $R$, so they are in
   $C$. Then $f-s_nx^n$ is still almost integral over $R[x]$. Inducting on degree, we get
   that $f\in C[x]$
 \end{proof}
 \begin{corollary}
   A domain $R$ is completely normal if and only if $R[x]$ is.
 \end{corollary}
 The proof is the same as before. In fact, we get a second (less tricky) proof of this
 result with the word ``completely'' removed using noetherian descent.
 \[\xymatrix{
 & S & & S[x]\\
 & C & & C[x]\\
 & R & & R[x]\\
 R_0 \ar@{-}[ruuu]|{C_0} & & R_0[x] \ar@{-}[ruuu]|{C_0[x]}
 }\]
 Take $f\in S[x]$ integral over $R[x]$, then it satisfies some monic polynomial. Let
 $R_0$ be the ring generated over $\mathbb{Z}\cdot 1$ by all the coefficients involved
 everywhere \dots this ring is noetherian, so the word ``completely'' can be removed. We
 get that $f\in C_0[x]\subseteq C[x]$.

 \medskip
 Power series case: $R[x]\rightsquigarrow R[[x]]$. If $R$ is completely normal, then so
 is $R[[x]]$ (this fails for normality!).
 \begin{theorem}[Hilbert basis theorem for power series]
   If $R$ is noetherian, then so is $A:=R[[x]]$.
 \end{theorem}
 \begin{proof}
   By Cohen's theorem, it suffices to prove that any $\p\in \spec A$ is finitely
   generated. Let $I\subset R$ be the ideal of all constant terms of elements in $\p$. Since
   $R$ is noetherian, $I$ is finitely generated, say $I=\sum_{i=1}^n a_i R$.

   \underline{Case 1}: If $x\in \p$, then $\p=I\cdot A + x\cdot A$, so $\p$ is
   generated by $n+1$ elements.

   \underline{Case 2}: If $x\not\in \p$, choose $f_i\in \p$ so that $f_1(0)=a_i$. We
   claim that $\p$ is generated by the $f_i$. Let $f\in \p$, then we can choose $b_i$ so
   that $f-\sum b_i f_i = x\cdot g$ for some $g\in A$. Since $x\not\in \p$, $g\in \p$.
   Repeating the argument for $g$ and inducting, we get $f=\sum
   (b_i+xc_i+x^2d_i+\cdots)f_i$.
 \end{proof}
 \begin{theorem}
   Let $R$ be a domain, with quotient field $K$ and $A=R[[x]]$. Then
   \begin{enumerate}
     \item If $A$ is normal, then $R$ is normal.
     \item $A$ is completely normal if and only if $R$ is completely normal
     \item $A$ is noetherian and normal if and only if $R$ is noetherian and normal.
   \end{enumerate}
 \end{theorem}
 \begin{proof}
   (3) follows from (2) and the Hilbert basis theorem for power series. The direction
   $\Leftarrow$ in (1) and (2) are done as before (in the case $A=R[x]$). It remains to
   prove that if $R$ is completely normal, then so is $A$.
   \[\xymatrix{
    & K((x)) \ar@{-}[dr] \ar@{-}[d]\\
    K\ar@{-}[d] & K[[x]]\ar@{-}[d] & Q(A) \ar@{-}[d]\\
    R & R[[x]] \ar@{}[r]|{=}& A
   }\]
   Assume $f\in Q(A)$ is almost integral over $A$. Then as an element of $K((x))$, $f$ is
   almost integral over $K[[x]]$, which is a PID (so a UFD), so it is completely normal.
   It follows that $f\in K[[x]]$. Say $f=a_0 + a_1x+ \cdots$. We want to show that each
   $a_i$ is in $R$. By almost integrality, there is some $h\in A\smallsetminus \{0\}$ so
   that $h\cdot f^i\in A$ for all $i\ge 0$. We will show by induction that $a_j\in R$.
   Suppose $a_0,\dots, a_j\in R$. Then $f'=a_0+\cdots a_{j-1}x^{j-1}\in A$, and $h\cdot
   (f-f')^i \in A$ for all $i\ge i$. Suppose $h=dx^m+ \cdots$, with $d\in R$ non-zero.
   Now we have
   \[
    h(f-f')^i = da_j^i x^{m+ij} + \cdots \in A
   \]
   so $da_j^i\in R$ for all $i$. It follows that $a_j$ is almost integral over $R$.
 \end{proof}
  \begin{corollary}
   If $k$ is a field, then $A_n=k[[x_1,\dots, x_n]]$ is noetherian and normal.
 \end{corollary}
 In fact, Weierstra\ss\ showed that $A_n$ is even a UFD.
 \stepcounter{lecture}
 \setcounter{lecture}{28}
 \section{Lecture 28}

 \begin{definition}
   A ring $R$ is called \emph{good} (called \emph{property $\ast$} in the notes) if for
   every non-unit $a\in R$, $\bigcap (a^i)=0$.
 \end{definition}
 ``good'' is a necessary condition for certain normality properties.
 \begin{proposition}
   $R$ is good if any of the following hold.
   \begin{enumerate}
     \item $R$ is a UFD.
     \item $R$ is a noetherian domain.
     \item $R$ is completely normal.
     \item $R[[x]]$ is completely normal.
   \end{enumerate}
 \end{proposition}
 \begin{proof}
   1 and 2 we've done already \anton{}. Assume $R$ is a bad domain, so $0\neq d=\bigcap
   (a^i)$ for some $a\in R$. Then $d(a^{-1})^i\in R$ for all $i$. Then $a^{-1}\not\in R$
   is almost integral over $R$, contradicting complete normality.

   Believe for the moment that there is an $f=x+b_2x^2+\cdots \in \mathbb{Z}[[x]]$ such that
   $f^2-f+x=0$. Now replace $x$ by $\frac{x}{a^2}$. Then we have
   $f(x/a^2)^2-f(x/a)+x/a^2=0$, so if we define $g:= a\cdot f(x/a^2)$, we have
   $g^2-ag+x=0$, so $g$ is integral over $A=R[[x]]$ (even over $R[x]$). Note that $dg\in
   A$ since $d$ ``clears'' the denominators, so $g\in Q{A}$. But $g\not\in A$, which we
   can see by expanding it (we get a $1/a$ coefficient).

   Finally, the $f$ above can be constructed by looking at the constraints on the $b_i$,
   or solve with the quadratic equation ... you get catalan numbers as coefficients.
 \end{proof}

 picture
 \[\xymatrix{
  &R\ UFD \ar[d]\\
  R\ cn\ar[r] \ar@/^/[rr] & R\ g & R\ n \ar@{.>}[l]|{\times}\\
  R[[x]]\ cn \ar@{<->}[u] \ar[rr] &  & R[[x]]\ n \ar[lu] \ar[u]\\
   & R\ ndom \ar[uu]
 }\]

 \begin{example}[a normal $R$ with $A=R[[x]]$ not normal]
   Take $R_0=\QQ[x,y]\subseteq \QQ[x,y/x]\subseteq \QQ[x,y/x^2]\subseteq \cdots$, so
   $R_i=\QQ[x,y/x^i]$. Consider $R= \bigcup R_i$. Each $R_i\cong \QQ[x,t]$ is normal,
   from which it is easy to prove that $R$ is normal, but $x^i|y$ for all $i$, and $x$ is
   not a unit. So $R$ is bad.
 \end{example}
 For another example, refine the $D+(x)$ idea. Take $D$ a normal domain, with
 $D\subsetneq K=Q(D)$. Define $R= D+(x)\subseteq K[x]$.
 \begin{proposition}
   $R$ is normal and has complete integral closure $R^\dag = K[x]$ and is bad.
 \end{proposition}
 \begin{proof}
   Suppose $g\in K(x)=Q(R)$ is integral over $R$, so $g^n+f_1g^{n-1}+\cdots + f_n=0$,
   with $f_i\in R$. Evaluating at 0, we see that $g(0)$ is integral over $D$, so $g(0)\in
   D$, so $g\in R$. Thus, $R$ is normal.

   Take a non-unit $a\in D$, then $x/a^i\in R$ for all $i$, so $x\in \bigcap (a^i)$, so
   $R$ is bad.
 \end{proof}
 \stepcounter{lecture}
 \setcounter{lecture}{29}
 \section{Lecture 29}

 If $L/K$ is a finite field extension, then there is a trace function $T_{L/K}:L\to K$.
 Any $\ell\in L$ is a $K$-linear operator $L\to L$, so it has a trace valued in $K$. Note
 that if $L'/L$ is another finite extension, $T_{L'/K}= T_{L/K} \circ T_{L'/L}$.
 \begin{theorem}
   Let $R$ be a normal domain, with $K=Q(R)$, $L/K$ be a finite separable field
   extension, and let $S$ be the integral closure of $R$ in $L$. Then $L=Q(S)$ and there
   is a $K$-basis $\{u_i\}$ of $L$ so that $S\subseteq \bigoplus Ru_i$.
 \end{theorem}
 \begin{proof}
   We need two facts:
   \begin{enumerate}
     \item $T(S)\subseteq R$. To see this, let $s\in S$, then we have $T(s)=T_{K(s)/K}
     \bigl( T_{L/K(s)}(s)\bigr) = [L:K(s)] \cdot T_{K(s)/K}(s)$, which is in $R$ because
     the minimal polynomial of $s$ has coefficients in $R$ (since $s$ is integral over
     $R$).

     \item The $K$-bilinear pairing $(x,y) = T(xy)$ is non-degenerate. This follows from
     the separability of $L$ over $K$.
   \end{enumerate}
   Let $\alpha\in L$. Then there is some nonzero $r\in R$ so that $r\alpha$ is
   integral over $R$. To see this, clear denominators in the minimal polynomial of
   $\alpha$ to get $r\alpha^n + \cdots =0$; then multiply by $r^{n-1}$ to get
   $(r\alpha)^n+\cdots = 0$. Therefore, there is a $K$-basis $\{v_i\}$ of $L$ so that
   $\{v_i\}\subseteq S$. It follows that $Q(S)=L$. Let $\{u_i\}$ be the ``dual
   $K$-basis'' of $\{v_i\}$ with respect to the pairing above. For any $s\in S$, we
   have $s = \sum a_i u_i$ with $a_i\in K$. Then $(s,v_j)=a_j = T(s\cdot v_j)$ is in the
   image of $S$ under $T$, which is in $R$.
 \end{proof}
 \begin{corollary}
   Assume in the situation above that $R$ is noetherian. Then $S$ is module-finite over
   $R$ and is therefore a noetherian normal domain. If $R$ is a PID, then $S$ is free
   over $R$ of rank $[L:K]$.
 \end{corollary}
 In the classical case, $R=\mathbb{Z}$ and $K=\QQ$. Then $L$ is a number field, and $S$ is the
 ring of algebraic integers in $L$. The corollary above says that $S$ is a noetherian
 normal domain of dimension $1$ (a.k.a.~a \emph{Dedekind domain}), and is free over
 $\mathbb{Z}$. In fact, if $R$ is Dedekind, then we can show that $S$ is Dedekind without the
 separability assumption.

 \subsection{\S 4. Valuation Domains}
 1932: Krull wrote a paper, Allgemeine Bewerungstheorie.\\
 1935: Krull wrote a book, Idealtheorie.

 \emph{Valuation rings} (or \emph{valuation domains}) are
 \begin{itemize}
   \item local
   \item always normal
   \item noetherian if and only if they are DVRs (or fields)
   \item possibly infinite dimensional
 \end{itemize}
 \begin{proposition}
   For any domain $R$ with quotient field $K$, the following are equivalent.
   \begin{enumerate}
     \item For every non-zero $x\in K$, either $x\in R$ or $x^{-1}\in R$.
     \item For $a,b\in R$, either $a|b$ or $b|a$ in $R$.
     \item The ideals in $R$ form a chain under inclusion. In particular, $R$ is local.
   \end{enumerate}
 Such an $R$ is called a \emph{valuation ring of $K$}.
 \end{proposition}
 \begin{proof}
   $3\Rightarrow 2 \Rightarrow 1$ are immediate. Let's do $1\Rightarrow 3$. Assume $3$
   does not hold, so there are ideals $I,J\subset R$ with no inclusion relation. Let $a\in
   I\smallsetminus J$ and $b\in J\smallsetminus I$. Then consider $x=a/b\in K$;
   $x\not\in R$ and $x^{-1}\not\in R$, lest $a\in J$ or $b\in I$.
 \end{proof}
 \begin{definition}
   $Val(K)$ is the set of valuation rings of $K$ (including $K$ in particular).
 \end{definition}
 Let $\m\in \Max R$, with $R$ a valuation ring. Then $K^\times = U(R)\sqcup
 \m\smallsetminus \{0\}\sqcup \bigl\{x^{-1}| x\in (\m\smallsetminus \{0\})\smallsetminus
 \{0\}\bigr\}$.

 \begin{proposition}
   Let $L/K$ be a field extension, $(V,\p)\in Val(L)$, and let $R=V\cap K$ and
   $\m=\p\cap K$. Then $(R,\m)\in Val(K)$. That is, $Val($-$)$ is a contravariant functor.
 \end{proposition}
 \begin{proof}
   For $x\in K\smallsetminus R$, $x^{-1}\in V\cap K=R$, so $R$ is a valuation ring of
   $K$. Now we will show that $R\smallsetminus \m = U(R)$ to prove that $\m$ is the
   unique maximal ideal of $R$. Take $x\in R\smallsetminus \m$, then $x^{-1}\in V\cap
   K=R$.
 \end{proof}
 \begin{example}
   Let $K$ be an algebraic extension of $\FF_p$. What is $Val(K)$? Take any $R\in
   Val(K)$. Then $\FF_p\subseteq R \subseteq K$, so $R$ is a field. Since $Q(R)=K$, we
   have $R=K$
 \end{example}
 \begin{example}
   Let $R=\mathbb{Z}$ and $K=\QQ$. Then $Val(K)=\{\QQ, \mathbb{Z}_{(p)}\}$. To see this, let $(R,\m)\in
   Val(K)$, with $R\neq K$. Then $\m\neq 0$, so $\mathbb{Z}\cap \m=(p)$ for some prime $p$. It
   follows that $\mathbb{Z}\smallsetminus (p)\subseteq U(R)$, so $\mathbb{Z}_{(p)}\subseteq R$. But
   $\mathbb{Z}_{(p)}$ is a maximal subring of $\QQ$.
 \end{example}
 
 \stepcounter{lecture}
 \setcounter{lecture}{30}
 \section{Lecture 30}

Exercise III.3: (reciprocal polynomial trick) Show that a unit $u\in S$ is integral over
a subring $R$ if and only if $u\in R[u^{-1}]$.

\begin{theorem}
  For a local ring $(R,\m)$, the following are equivalent.
  \begin{enumerate}
    \item $R$ is a PID but not a field.
    \item $R$ is a noetherian normal domain of dimension 1.
    \item $R$ is noetherian and $\m = (\pi)$ is principal, with $\pi\not\in \nil R$.
    \item $R$ is noetherian, $\m\not\in \nil R$, and $\dim_{R/\m} \m/\m^2=1$.
  \end{enumerate}
  If these hold, $R$ is called a \emph{discrete valuation ring (DVR)}.
\end{theorem}
\begin{proof}
  omitted, but not trivial! \anton{}
\end{proof}
\begin{remark}
  In a DVR, the ideals are all of the form $\m^i$ (with $i\ge 0$), and $\dim_{R/\m}
  \m^i/\m^{i+1}=1$. $K^\times = U(R)\times \langle \pi\rangle$ (as a group).
\end{remark}
\begin{definition}
  The element $\pi$ is called a \emph{uniformizer} of $R$.
\end{definition}
\begin{example}\\
  \begin{tabular}{c|c|c|c}
    DVR & $\pi$ & $Q(R)$ & $U(R)$\\ \hline
    $\mathbb{Z}_{(p)}$ & $p$ & $\QQ$ & $\{a/b| a,b$ prime to $p\}$\\
    $k[[x]]$ & $x$ & $k((x))$ & $\{a_0+a_1x+\cdots|a_0\neq 0\}$\\
    $R=\{f/g| f,g\in k[x], \deg f\le \deg g\}$ & $1/x$ & $k(x)$ & $\{f/g|\deg f=\deg g\}$
  \end{tabular}
\end{example}
General properties of valuation rings
\begin{theorem}
  Let $R\in Val(K)$, then
  \begin{enumerate}
    \item $R$ is normal.
    \item $R$ is a B\'ezout ring (every finitely generated ideal is principal).
    \item Every ring between $R$ and $K$ is also a valuation ring.
    \item For every $\p\in \spec R$, $R/\p$ is a valuation ring.
    \item Every proper radical ideal is prime.
    \item For any proper ideal $I\subset R$, $I_\infty = \bigcap I^n$ is prime. Moreover,
    every prime $\p\not\supseteq I$ is contained in $I_\infty$.
  \end{enumerate}
\end{theorem}
\begin{proof}
  (1) Take $u\in K\smallsetminus R$. Then $u^{-1}\in R$, so $R=R[u^{-1}]$, so $u$ is not
  integral over $R$ by exercise III.3. (2) If $I$ is generated by $a_1, a_2, \dots, a_n$,
  then the ideals generated by the $a_i$ form a chain, so $I$ is principal. (3) follows
  from the first point in the definition of a valuation ring. (4) immediate
  as for (3). (5) Take $ab\in I$, and assume $a=rb$ for some $r\in R$. Then $(rb)^2=
  r\cdot ab\in I$, so $rb=a\in I$. (6) Let $a,b\not\in I_\infty$, so $a\not\in I^m$ and
  $b\not\in I^n$. Then  we have $I^m\subseteq aR$ and $I^n\subseteq bR$. Suppose $ab\in
  I^{n+m}$, then $ab\in I^m I^n\subseteq aI^n$, so $b\in I^n$. Contradiction. Thus,
  $ab\not\in I^{n+m}$, so $ab\not\in I_\infty$. Finally, assume $\p\not\supseteq I$, then
  $\p\not\supseteq I^n$ for each $n$. Thus, $\p\subseteq I^n$ for each $n$, so
  $\p\subseteq I_\infty$.
\end{proof}
\noindent
Question 1: What are the noetherian valuation rings?\\
Question 2: What are the completely normal valuation rings?\\
\begin{corollary}
  Let $(R,\m)$ be a valuation ring of $K$.
  \begin{enumerate}
    \item $R$ is noetherian if and only if $R$ is a DVR or a field.
    \item The following are equivalent.
    \begin{enumerate}
      \item $R$ is completely normal.
      \item $R$ is good (nothing is infinitely divisible by a non-unit).
      \item $\dim R\le 1$.
    \end{enumerate}
  \end{enumerate}
\end{corollary}
\begin{proof}
  (1) $\Leftarrow$ is obvious. Assume $R$ is noetherian, then by the B\'ezout property,
  $R$ is a PID. Hence, $R$ is a DVR or a field.

  (2) $(a)\Rightarrow(b)$ was done in the last section. $(b)\Rightarrow(c)$ We must show
  that there is no prime between $(0)$ and $\m$. Assume there is such a $\p$, then choose
  $x\in \m\smallsetminus \p$, so $(x)\not\subseteq \p$. Then $0=(x)_\infty\supseteq \p$.
  $(c)\Rightarrow(a)$ Assume $R$ is not completely normal, so there is some $x\in
  Q(R)\smallsetminus R$ and $d\in R$ nonzero so that $dx^{-i}\in R$ for all $i\ge 0$. Let
  $y=x^{-1}$, so $d\in y^i R$ for every $i\ge 0$, so $d\in (y)_\infty$. Thus,
  $(y)_\infty$ is a non-zero prime ideal. Moreover, if $y=ry^2$, then $y$ would be a unit
  (since we are in a domain), so $y\not\in (y)^2$, so $y\not\in (y)_\infty$, so
  $(y)_\infty$ is not equal to $\m$.
\end{proof}
Next time: if $R\subseteq K$ is a subring, then $Val_R(K)$ is the set of \emph{relative
valuation rings} $R'$ of $K$ which contain $R$. If $R\in Val(K)$, we'll describe this
family.

\begin{corollary}\marginpar{\raggedleft \small Welcome to the Krull world}
  If $(R,\m)$ is a valuation ring with $\dim R\ge 2$, then $\m_\infty = \bigcap \m^n \neq
  0$.
\end{corollary}
 Exercise III.6: replace ``$S_\mathfrak{P}/R_\p$ with $S_\mathfrak{P}/\im (R_\p)$.\\
 III.21: $f(x) = \sum_{n=0}^\infty \frac{x^n}{2^{n^2}} \in \QQ((x))$. This exercise is a
 little ``provisional''.

 Notation: $D$-$Val(K)$ is the set of \emph{discrete} valuation rings of $K$ (this set
 may be empty).
 \begin{example}
   If $K$ is an algebraic extension of $\FF_p$, then $Val(K)=\{K\}$,
   $D$-$Val(K)=\varnothing$.
 \end{example}
 \begin{example}
   Take $K$ so that $K^\times = (K^\times)^n$ for some fixed $n\ge 2$. Then
   $D$-$Val(K)=\varnothing$. If $(R,(\pi))$ is a DVR of $K$, then $\pi=a^n$ for some $n$,
   so $a$ is integral over $R$, so it is in $R$ (because $R$ is normal). But then
   $(a)=(\pi)^m = (a)^{n+m}$. contradiction.
 \end{example}

 For any subring $R\subseteq K$, we defined $Val_R(K)$ to be the elements of $Val(K)$
 which contain $R$. If $R\in Val(K)$, then $Val_R(K)$ is just the set of rings between
 $R$ and $K$.

 \begin{theorem}[4.12]
   Describing all $R'$ so that $R\subseteq R'\subseteq K$, where $(R,\m)\in Val(K)$. A
   typical $R'$ is of the form $R_\p$, where $\p\subseteq \m$ is a prime in
   $R$. Furthermore, $\p R_\p \overset{!}{=} \p$ is the maximal ideal.
 \end{theorem}
 \begin{proof}
   omitted.\anton{}
 \end{proof}
 Consequently, the map $\p\mapsto R_\p$ defines an inclusion reversing bijection $\spec R
 \leftrightarrow Val_R(K)$. In particular, $Val_R(K)$ is a chain because $\spec R$ is a
 chain. The longest chain is the Krull dimension of $R$. In particular, $\dim R=1$ if and
 only if $R$ is a maximal subring of $K$. DVRs are 1-dimensional, but not all
 1-dimensional valuation rings are DVRs.

 \begin{definition}
   $Val^R(K) = \{R' \in Val(K)| R'\subseteq R\subseteq K\}$. This is only meaningful if
   $R\in Val(K)$ since any ring containing a valuation ring is a valuation ring (so we
   would have $Val^R(K)=\varnothing$ if $R\not\in Val(K)$).
 \end{definition}
 How do you tell the difference between $Val_R(K)$ and $Val^R(K)$? Well, $Val_R(K)$ has
 the $R$ below, and $Val^R(K)$ has the $R$ above.

 \begin{theorem}[4.13]
   For $(R,\m)\in Val(K)$, there is an inclusion preserving bijection
   $Val^R(K)\leftrightarrow Val(R/\m)$, with $R'\mapsto R'/\m =: \bbar {R'}$. Note that
   $\m\subseteq \m'\subseteq R'\subseteq R$, so this makes sense.
 \end{theorem}
 \begin{corollary}[4.14, Dimension-Summation formula]
   In the setting above, $\dim R' = \dim \bbar{R'} + \dim R$.
 \end{corollary}
 \begin{proof}
   easy chain composition argument.
 \end{proof}
 This allows us to come up with examples of valuation rings with dimension bigger than 1.

 \underline{Places}: intuitively, a place is a ``generalized field homomorphism'' that
 may send may elements to ``$\infty$''.
 \begin{definition}
   Let $K$ and $\W$ be fields. Then a \emph{place} is a map $\phi:K\to \W\sqcup
   \{\infty\} =: \W_\infty$ so that $\phi$ is a ``field homomorphism '' with the usual
   rules of addition and multiplication for $\infty$ ($\infty \pm \infty$, $0/0$,
   $\infty/\infty$, and $\infty\cdot 0$ are undefined).
 \end{definition}
 There is a triumvirate of ideas which are basically the same: valuation rings, places,
 and Krull valuations.

 Working with a place is equivalent to working with a valuation in the following way.
 Suppose $\phi$ is a place, then define $R=\phi^{-1}(\W)$. $R$ is a valuation ring of $K$
 \anton{}. Conversely, given a valuation ring $(R,\m)$ of $K$, there is a place $K\to
 R/\m \sqcup \{\infty\}$, sending $R$ to $R/\m$ in the usual way, and $K\smallsetminus R$
 to $\infty$.

 We say $K\to \W_\infty$ is the \emph{trivial place} if $\phi(K)\subseteq \W$ (i.e.~a
 good old field homomorphism)
 \stepcounter{lecture}
 \setcounter{lecture}{32}
 \section{Lecture 32}

 Supplemental points:
 \begin{itemize}
   \item A valuation ring is the same thing as a local B\'ezout domain.
   \item A 1-dimensional valuation ring $(R,\m)$ with $\m$ principal is the same thing as a DVR.
   \item If $R$ is a valuation ring and $n\le \infty$, then $\dim R=n$ if and only if
   $|\spec R|=n+1$.
   \item If $k\subseteq K$ is a subfield, then $Val_k(K)$ is the \emph{Zariski space} (or
   \emph{Zariski Riemann Surface}) for $K/k$.
 \end{itemize}

 Composition of places: Let $(R,\m),(R',\m')\in Val(K)$, with $R'\subseteq R$ (so that
 $\m\subseteq \m'\subseteq R'\subseteq R$). Then we showed that $R'\m\in Val(R/\m)$, so
 it comes with a place $\sigma$ \anton{}, and we get a commutative diagram of places.
 \[\xymatrix{
  K \ar[r]^<>(.5)\phi_R \ar[dr]_{\phi'}^{R'} & (R,\m)_\infty \ar[d]^\sigma_{\frac{R'}{\m}}\\
  & (R',\m')_\infty
 }\]
 That is, $\phi'=\sigma\circ \phi$. We shoed earlier that $\dim R'=\dim (R'/\m)+\dim R$.
 We can phrase this as $\dim (\sigma\circ \phi)=\dim \sigma + \dim \phi$.

 Valuation rings on $K=k(x)$. Let's construct examples of $R\in Val_k(K)$ (points in
 the Zariski Riemann surface).
 \begin{example}
   Fix a monic irreducible $\pi(x)\in k[x]$, then define $R=k[x]_{(\pi)}$. $R$ is a DVR;
   it is called the ``$\pi$-adic valuation ring''. $\W =
   k[x]/(\pi) = k[\theta]$ (note that this is already a field), where $\theta = \bar x$.
   The $\pi$-adic place is $\phi:K\to \W_\infty$, with $\phi(f/g) = f(\theta)/g(\theta)$.
   Since $f/g$ can be assumed to be in lowest terms, we know when to send $f/g$ to
   infinity.

   In the special case where $\pi = x-a$, then $\theta = \bar x = a$, so the place is
   $f/g\mapsto f(a)/g(a)$.
 \end{example}
 \begin{example}
   Set $y = 1/x$, then $K=k(x)=k(y)$. Consider the $(y)$-adic place (or $1/x$ place) is
   the one with valuation ring $k[y]_(y)$. Let's figure out what this ring is in terms of
   $x$. If $r(x)$ is a non-zero rational function $\frac{a_0 x^n+\cdots+ a_n}{b_0x^m +\cdots
   +b_m} = \frac{x^n(\cdots)}{x^m(\cdots)} = y^{m-n} \frac{a_0+\cdots+ a_n y^n}{b_0 +\cdots
   +b_m y^m}$. Thus, $\phi(r(x)) =${\scriptsize $\begin{cases}
     0 & m> n\\
     \infty & m<n\\
     a_0/b_0 & m=n
   \end{cases}$}. So the valuation ring is $S=\{f/g|f,g\in k[x], g\neq 0, \deg f\le \deg
   g\}$, with uniformizer $y=1/x$.
 \end{example}
 \begin{theorem}
   $Val_k(K)$ is the exactly the set of rings described in the two examples above.
 \end{theorem}
 \begin{proof}
   Same as in the computation of all valuation rings of $\QQ$.
 \end{proof}
 If $k=\bar k$, then there are only linear irreducibles, so we get a valuation ring in
 $Val_k(K)$ for every point in $\mathbb{P}^1_k$. Thus the terminology ``Zariski Riemann
 surface''.

 \begin{definition}
   A valuation ring $(R,\m)$ has \emph{principal type} if $\m = (\pi)\neq 0$. We still
   call $\pi$ a uniformizer, but there is no noetherian hypothesis. Equivalently, we can
   say that $\m$ is non-zero and finitely generated (because $R$ is B\'ezout).
 \end{definition}
 These rings emulate DVRs. In dimension 1, these are exactly DVRs. Note that if $R$ is
 noetherian, Krull's principal ideal theorem doesn't apply, so $\m$ can have large height
 even through it is principal. We say that a place has principal type if the
 corresponding valuation ring does.

 \begin{claim}
   In the composition of places picture, if $\sigma$ (i.e.~$R'/\m$) has principal type,
   then so does $\phi'$ (i.e.~$R'$ has principal type).
 \end{claim}
 \begin{proof}
   Write $\m' = \pi R'+\m$, with $\pi\not\in \m$. Since $(\pi)\not\subseteq \m$, we
   must have $\m\subseteq (\pi)$. This implies that $\m'=(\pi)$, so $R'$ has principal
   type.
 \end{proof}
 \begin{example}
   \[\xymatrix{
    K = \QQ(x) \ar[r]^<>(.5)\phi_<>(.5){\QQ[x]_{(x)}} \ar[dr]_{\phi'} & \QQ_\infty \ar[d]^{\sigma_p}_{\mathbb{Z}_{(p)}}\\
    & (\FF_p)_\infty
   }\]
   We get that $\phi'$ has principal type because the maximal ideal of $\mathbb{Z}_{(p)}$ is
   principal. Let $R'$ be the valuation ring associated to $\phi'$. Then $\dim R'=2$ and
   $R'$ has principal type. $R' = \{f(x)/g(x)\big| x\nmid g(x), f(0)/g(0)\in
   \mathbb{Z}_{(p)}\}$. The uniformizer is $p$.
 \end{example}
 \stepcounter{lecture}
 \setcounter{lecture}{33}
 \section{Lecture 33}

 If $k$ is a field, and $F = k(x_1,\dots, x_{n-1})$, $K= k(x_1,\dots, x_n) = F(x_n)$.
 Then there is an $R\in Val_k(K)$ of principal type with residue field $k$ and $\dim
 R=\ell$ for any $0\le \ell\le n$.

 ``Compose and induct'':$\xymatrix@C+1pc{ K=F(x_n) \ar[r]^<>(.5){(x_n)\text{-adic}} \ar@{-->}[dr] &
 F_\infty \ar[d]\\
 & k_\infty}$

 \subsection{\S 5 Krull Valuations}
 In this section KV means ``Krull valuation'' and OAG means ``ordered abelian group''.

 \begin{definition}
   An \emph{OAG} is an (additive) abelian group $(\Gamma, \le)$, where $\le$ is a total
   ordering of $\Gamma$ that respects the addition (i.e.~$a\le b\Rightarrow a+c\le b+c$).
 \end{definition}
 Given such a $\Gamma$, we define the \emph{positive cone} of $\Gamma$ by
 $\Gamma^+:=\{a\in \Gamma| a\ge 0\}$. $\Gamma^+$ is a sub-monoid of $\Gamma$ that
 satisfies the properties $\Gamma^+\cap -\Gamma^+ = \{0\}$ and $\Gamma^+\cup
 -\Gamma^+=\Gamma$. Conversely, if $P\subseteq \Gamma$ is a sub-monoid satisfying these
 properties, and $\Gamma$ does not have an order, then we may define an order by $a\le b
 \Leftrightarrow b-a\in P$.

 \begin{remark}
 \begin{trivlist}
   \item
   \item -- An OAG is always torsion-free.
   \item -- Any subgroup of an OAG is also an OAG.
   \item -- We may reverse the order and get another OAG. Note that this is not true for
   ordered fields ($1=1\cdot 1$ implies that $1$ is in the positive cone).
   \item -- Morphisms of OAGs are order-preserving.
   \item -- Given OAGs $(\Gamma_i,\le)$, we may define an order on $\Gamma = \Gamma_1\times
   \cdots \Gamma_n$ lexicographically. In the case $n=2$, the positive cone looks like
   $(\Gamma_1^+\smallsetminus 0)\times \Gamma_2 \cup 0\times \Gamma_2^+$.
 \end{trivlist}
 \end{remark}
 \begin{example}
   The zero group, $\mathbb{Z}$, $\QQ$, \dots (irrational stuff), $\RR$ are OAGs with their
   usual orderings. For instance, if $\alpha_1,\dots, \alpha_n\in \RR$ are $\QQ$-linearly
   independent, then we may take $\Gamma$ to be $\sum \alpha_i \mathbb{Z}$ or $\sum \alpha_i
   \QQ$. For example, $\mathbb{Z}[\sqrt 2]$ and $\mathbb{Z}[\sqrt[3] 2]$. These are isomorphic to
   $\mathbb{Z}^2$ and $\mathbb{Z}^3$. Note that you can also put lexicographic orderings on these
   groups.
 \end{example}

 \begin{definition}
   Given a domain $A$, a \emph{KV} is a map to an OAG $v:A\smallsetminus \{0\}\to \Gamma$
   such that
   \begin{enumerate}
     \item $v(ab)=v(a)+v(b)$ for $a,b$ non-zero.
     \item $v(a+b)\ge \min\{v(a), v(b)\}$ if $a,b$, and $a+b$ are non-zero.
   \end{enumerate}
 \end{definition}
 It is useful to introduce $\Gamma_\infty := \Gamma \sqcup \{\infty\}$, and define
 $v(0)=\infty$ and $\Gamma< \infty$. We also define $a+\infty = \infty$ for all $a\in A$,
 $\infty+\infty=\infty$ (this is different from what we did with places)

 \begin{proposition}
   \begin{enumerate}
     \item $v(\pm 1)=0$.
     \item $v(-a)=v(a)$ for all $a\in A$.
     \item (Predictable value property) If $v(a_1), \dots, v(a_n)$ are all distinct, then
     $v(a_1+\cdots +a_n)=\min \{v(a_i)\}$.
   \end{enumerate}
 \end{proposition}
 \begin{proposition}
   $v:A\smallsetminus \{0\} \to \Gamma$ as above can be uniquely extended to a KV
   $v:Q(A)\smallsetminus \{0\} \to \Gamma$.
 \end{proposition}
 \begin{proof}
   Uniqueness is easy because we must have $v(a/b)=v(a)-v(b)$ for $a,b$ non-zero. For
   existence, use this formula as a definition and verify the conditions in the
   definition.\footnote{``I don't think I've ever checked this.''}
 \end{proof}
 \begin{definition}
   If $v:K^\times \to \Gamma$ is a KV on a field $K$, then the image $v(K^\times)$ is called
   the \emph{value group} of $v$. It is an OAG (unlike in the case of a domain, where you
   only get a monoid).
 \end{definition}
 \begin{definition}
   $|\cdot|:A\smallsetminus \{0\}\to \Gamma$, where $\Gamma$ is a (multiplicative) OAG,
   is called a \emph{KV} if $|ab|=|a|\cdot |b|$ and $|a+b|\ge \max \{|a|,|b|\}$ for
   $a,b$, and $a+b$ non-zero.
 \end{definition}
 This is the same definition, but with $\Gamma$ written multiplicatively and the order
 reversed. These are also called \emph{non-archimedean absolute values} when $\Gamma =
 \RR_{> 0}$ and $A$ is a field. Instead of introducing ``$\infty$'', we introduce
 ``$0$''.

 \begin{theorem}
   A valuation on a field $v:K\to \Gamma_\infty$ determines a valuation ring $R_v:=
   \{a\in K|v(a)\ge 0\}\in Val(K)$. Conversely, a valuation ring $R\in Val(K)$ determines
   a KV $v_R:K^\times \to \Gamma_R:=K^\times /U(R)$, suitably ordered.
 \end{theorem}
 \stepcounter{lecture}
 \setcounter{lecture}{34}
 \section{Lecture 34}

 There are non-archimedean absolute values ($|a+b|\le \max\{|a|,|b|\}$) and Krull
 valuations. The archimedean absolute values are exactly the real Krull valuations.

 \[\xymatrix{
 K\ar[r]^v \ar[dr]_{\txt{non-arch\\ absolute \\ value}} & \Gamma_\infty \ar[d]^{e^{-x}\txt{ (order\\ reversing)}}\\
  & **[r] \RR_{\ge 0}
 }\]

 An OAG $(\Gamma,\le)$ is called \emph{archimedean} if for all $x,y>0$ in
 $\Gamma$ if there is some $n\in \mathbb{N}$ such that $nx>y$.

 \begin{theorem}[H\"older, 1901]
   $(\Gamma,\le)$ is archimedean if and only if it can be order embedded into
   $(\RR,\le)$.
 \end{theorem}
 Thus, non-archimedean absolute values are the Krull valuations with archimedean ordered
 group. How confusing!

 Relating KVs to valuation rings. Suppose $v:K\to \Gamma_\infty$ is a KV, then we may
 define $R_v:=\{a\in K| v(a)\ge 0\}$. This is a valuation ring with $\m_v=\{a\in
 K|v(a)>0\}$. Conversely, if $R\in Val(K)$, define a KV $v:K^\times \to \text{Prin}(R)$,
 where $\text{Prin}(R)$ is the group of principal fractional ideals $aR\subseteq {}_R K$,
 where $a\in K^\times$. When $R$ is a valuation ring, $\text{Prin}(R)$ is an OAG, ordered
 by REVERSE inclusion. Then define $v(a)=aR$ and check that $v(ab)=abR=aR\cdot
 bR=v(a)+v(b)$ and that $v(a+b)=(a+b)R \supseteq v(a), v(b)$.

 There is another useful view of the value group. Look at $K^\times /U(R)$, ordered by
 $aU(R)\le bU(R)$ when $b/a\in R$. Then we have an order isomorphism $K^\times/U(R)\to
 \text{Prin}(R)$, given by $aU(R)\mapsto aR$.

 \begin{corollary}
   There is a one to one correspondence between $Val(K)$ and equivalence classes of KVs
   on $K$, where $v:K^\times\twoheadrightarrow \Gamma$ and $v':K^\times\twoheadrightarrow
   \Gamma'$ are equivalent if there is an order isomorphism between $\Gamma$ and
   $\Gamma'$.
 \end{corollary}
 \begin{definition}
   A KV $K\twoheadrightarrow \Gamma\cong \mathbb{Z}$ is called a \emph{discrete valuation}.
   Rings corresponding to discrete valuations are DVRs.
 \end{definition}
 \begin{example}
   If $A$ is a UFD, with quotient field $K$ and some irreducible element $\pi$ (so $\pi$
   generates a prime ideal). Define $v=v_\pi:A\smallsetminus\{0\}\to \mathbb{Z}$ by
   $v(a)=\max\{n|\pi^n\text{ divides } a \text{ in } A\}$; this is called the $\pi$-adic
   KV. It is easy to check that this is a KV. This valuation corresponds to the DVR
   $A_{(\pi)}$. Here are some examples of such things.
   \begin{itemize}
     \item $A=\mathbb{Z}$ and $\pi=p$, then you get the usual $p$-adic valuation on $\QQ$.
     \item $A=k[[x]]$ and $\pi=x$, then $v(f)$ is the order of vanishing (or of a pole) at
     0.
     \item $A=k[x]$, and $\pi$ a monic irreducible, then you get the $\pi(x)$-adic
     valuation.
     \item $A=k[x]$, and $v:k[x]\smallsetminus\{0\}\to \mathbb{Z}$ given by $f\mapsto -\deg(f)$.
     This gives a discrete valuation on $k(x)$. This is the $1/x$-adic valuation!
   \end{itemize}
 \end{example}
 \begin{example}
   What is the transcendence degree of $k(\!(t)\!)$ over $k$? It must be $\infty$,
   because it cannot be any integer like 17.

   Then we can embed $K=k(x_1,\dots, x_n)\hookrightarrow k(\!(t)\!)$ and use the $t$-adic
   valuation, giving a discrete valuation on $K$! We need to check that this valuation is
   non-trivial.
   \[\xymatrix{
    K & k(\!(t)\!)\\
    R & k[[t]]=V\\
    \m & tV
   }\]
   We have $k\subseteq R/\m\hookrightarrow V/tV=k$, so $R/\m=k$. Thus, the valuation ring
   $R$ is non-trivial (it is not all of $K$).
 \end{example}

 We haven't proven that $tr.d._k k(\!(t)\!)=\infty$, and I [Lam] think it is hard. Let's
 show that $tr.d._\QQ \QQ(\!(t)\!)\ge 2$. We claim that $t$ and $e^t=\sum t^n/n!$ are
 algebraically independent. Assume that $f_0(t)(e^t)^n + f_1(t)(e^t)^{n-1}+\cdots =0$. We
 may assume that not all of the $f_i(t)$ are divisible by $t-1$. Then substitute $t=1$ to
 get that $e$ is algebraic over $\QQ$, a contradiction!

 For general $k$, we can also show that $t$ and $1+t^{1!}+t^{2!}+t^{3!}+\cdots$ are
 algebraically independent.
 \stepcounter{lecture}
 \setcounter{lecture}{35}
 \section{Lecture 35}

 The notes have been revised to contain some more stuff. When is $(R,\m)$ or principal
 type? Here are some necessary and sufficient conditions (individually \dots any of them
 is enough)
 \begin{itemize}
   \item $\m\neq\m^2$
   \item $R$ surjects onto a DVR.
   \item $\Gamma^+\smallsetminus \{0\}$ has a least element.
 \end{itemize}

 \underline{Any OAG $\Gamma$ is a valuation group}. That is, there is a surjective
 valuation $v:K\twoheadrightarrow \Gamma_\infty$. For example, let $\Gamma=\QQ$ with the
 usual ordering. This produces a valuation ring $(R,\m)$ which is not of principal type
 ($\m=\m^2$).

 To prove the result, form the group algebra $A=k\Gamma$ over a fixed field $k$, with
 formal basis $\{t_\alpha|\alpha\in \Gamma\}$, with $t_\alpha t_\beta=t_{\alpha+\beta}$.
 Given $f\in A$, we have $f=\sum a_\alpha t_\alpha = a_{\alpha_0}t_{\alpha_0} + $``higher
 terms'' with $a_{\alpha_0}\neq0$ \dots we're taking $\alpha_0$ to be the least element
 that appears. Define $v(f)=\alpha_0$. It is immediate that this $v$ is a valuation, and
 that it surjects onto $\Gamma$. In particular, $A$ is a domain. Note that the residue
 field is $k$.

 \underline{Convex subgroups of an OAG}: A subgroup $G\subseteq \Gamma$ is called
 \emph{convex} (or \emph{isolated}) if whenever $0\le a\le b$ and $b\in G$, we also have
 $a\in G$.
 \begin{theorem}
   Convex subgroups are exactly the kernels of OAG morphisms.
 \end{theorem}
 \begin{proof}
   If $G=\ker (\phi:\Gamma\to \Gamma')$ and $0\le a\le b$ with $\phi(b)=0$, then we get
   $0\le \phi(a)\le 0$, so $\phi(a)=0$.

   Conversely, if $G$ is a convex subgroup of $\Gamma$, then we declare a non-identity
   coset positive if all elements are positive. After some checking, this makes
   $\Gamma/G$ into an OAG, and the natural map $\Gamma\to \Gamma/G$ is an OAG morphism.
   \anton{}
 \end{proof}
 \begin{lemma}
   Convex subgroups of $\Gamma$ form a chain (under inclusion).
 \end{lemma}
 \begin{proof}
   easy to check \anton{}
 \end{proof}
 \begin{definition}
   The \emph{order-rank} of an OAG is the order type of the chain of convex subgroups. If
   you're a baby, you can define $\ork(\Gamma)=n\in \mathbb{Z}$ if $\Gamma$ has exactly
   $n+1$ convex subgroups, and $\infty$ otherwise.
 \end{definition}
 There is an order reversing bijection between convex subgroups of $\Gamma$ and prime
 ideals in $R$ (any valuation ring with valuation group $\Gamma$), given by $G\mapsto
 \{a\in R|v(a)>G\}$.
 \[\xymatrix{
 0 & \m\\
 \Gamma & (0)
 }\]
 \begin{itemize}
   \item As a corollary, $\ork(\Gamma)=\dim R$.
   \item $\ork(\Gamma)\le 1$ corresponds to $\Gamma$ being archimedean.
   \item $\ork(\Gamma)\le rk(\Gamma) := \dim_\QQ(\QQ\otimes_\mathbb{Z} \Gamma)$.
 \end{itemize}

 \subsection{\S 6. Characterizations of Normal Domains}
 \begin{lemma}[Chevalley]
   Let $K\supseteq V$ be rings, and let $u\in U(K)$. If $I\subset V$ is a proper ideal, then
   either $I\cdot V[u]\neq V[u]$ or $I\cdot V[u^{-1}]\neq V[u^{-1}]$.
 \end{lemma}
 \begin{proof}
   Tricky calculation. \anton{}
 \end{proof}
 \begin{theorem}[Existence theorem for valuation rings]
   Let $S$ be a sub-ring of a field $K$, and let $\p\in\spec S$. Then there exists a
   valuation ring $(V,\m)\in Val(K)$ containing $S$ such that $\m\cap S=\p$.
 \end{theorem}
 \begin{proof}
   Localize at $\p$ first, so we may assume $S$ is local and $\p$ is the maximal ideal.

   Consider the family $\F=\{T| S\subseteq T\subseteq K, \p T\neq T\}$, ordered by
   inclusion. It is easy to see that Zorn's lemma applies to give a maximal member $V$.
   Then check that $V$ is a valuation ring of $K$: if $u,u^{-1}\in K\smallsetminus V$,
   then you could extend $V$ by Chevalley's lemma, contradicting maximality, so either
   $u\in V$ or $u^{-1}\in V$.
 \end{proof}
 \begin{definition}
   Let $(S,\p)$ and $(V,\m)$ be local rings, with $S\subseteq V$. We say that $V$
   \emph{deominates} $S$ (written $V\ge S$ or $S\le V$) if $\m\cap S=\p$.
 \end{definition}
 This can be understood in two alternative ways. It is sufficient for $\p\subseteq \m$,
 or to say that $\p V\subsetneq V$.

 In general, if $K/F$ is a field extension and $(V,\m)$ is a local ring in $K$, then
 $(S,\p)=(F\cap V,F\cap \m)$ a local ring dominated by $(V,\m)$. To see that $S$ is
 local, let $a\in S\smallsetminus \p\subseteq V\smallsetminus \m$, then $a$ is invertible
 in $V$ and the inverse must lie in $F$, so $a$ is invertible in $S$.

 Given a field $K$, define $Loc(K)$ to be the set of local rings in $K$.
 \begin{theorem}
   For every field $K$, $Loc(K)^*=Val(K)$, where $loc(K)$ is ordered by dominance.
 \end{theorem}
 \begin{proof}
   If $S\in Loc(K)^*$, then there is a $V\in Val(K)$ so that $V\ge S$ \anton{}. This
   implies $V=S$. Conversely, if $(V,\m)\in Val(K)$, then if $V'\ge V$ in $Loc(K)$,
   $\m'\subseteq \m$, so $V'$ cannot dominate $V$, so $V$ is maximal in $Loc(K)$.
 \end{proof}
 \stepcounter{lecture}
 \setcounter{lecture}{36}
 \section{Lecture 36}

 Results presented in the notes:
 \begin{enumerate}
   \item $K/k$ field extension, with $v:K^\times\twoheadrightarrow \Gamma$ trivial on
   $k$, then $\ork \Gamma \le rk(\Gamma) \le tr.d._k K$.

   \item $tr.d._k K=1$, $R\in Val_k(K)$, then $R$ is a DVR. See Hartshorne \S I.6.
 \end{enumerate}

 \begin{theorem}
   Let $R\subseteq K$ is a sub-ring of a field, with integral closure $C$. Define
   $T=\bigcap \{V\in Val_R(K)\}$. Then $C=T$. We could also define the intersection $T$
   by restricting to those $V$ whose maximal ideals contract to a maximal ideal in $R$.
   In particular, if $R$ is local, then we only intersect those $V$ which dominate $R$.
 \end{theorem}
 \begin{corollary}
   A domain is normal if and only if it is an intersection of some family of valuation
   rings of its quotient field.
 \end{corollary}
 \begin{proof}[Proof of Theorem]
   $C\subseteq T$: each $V$ is normal and contains $R$, so integral elements over
   $R$ are contained in each $V$.

   Conversely, assume $x\not\in C$. By the reciprocal polynomial trick (Ex.~III.3)
   $x\not\in R[x^{-1}]=S$. Then $x^{-1}\notin U(S)$ (let $x\in S$). Choose some $\p\in
   \Max(S)$ containing $x^{-1}$. By the existence theorem, there is some $(V,\m)\in
   Val(K)$ so that $S\subseteq V$ and $\m\cap S=\p$. We have that $x^{-1}\in \p\subseteq
   \m$, so $x\not\in V$.

   Claim: $\m\cap R$ is a maximal ideal in $R$.\\
   $\m\cap R = \m\cap S\cap R = \p\cap R$. Consider the map $R\to S\to S/\p$; since
   $x^{-1}$ is killed by the second map, the composition is onto, with kernel $\p\cap R$.
   Since we chose $\p\in \Max (S)$, $S/\p$ is a field, so $\p\cap R$ is maximal.
 \end{proof}

 \begin{theorem}
   Let $R$ be a noetherian domain with quotient field $K$. Then $R$ is normal if and only
   if
   \begin{trivlist}
     \item[\rm (Nor1)] for any height 1 prime $\p\in R$, $R_\p$ is a DVR, and
     \item[\rm (Nor2)] for any $0\neq a\in R$, all primes in the set $\ass \bigl(R/(a)\bigr)$
     have height 1.
   \end{trivlist}
   In this case, $R = \bigcap_{ht(\p)=1} R_\p$.
 \end{theorem}
 \begin{proof}
   Suppose Nor1 and Nor2. First we check $R=\bigcap_{ht(\p)=1} R_\p$. If this holds, then
   since each $R_\p$ is normal, $R$ is normal. Let $a, b\in R$ with $a\neq 0$, an assume
   $b/a\in R_\p$ for each $\p$ of height 1. Consider a minimal primary decomposition
   $aR=\q_1\cap \cdots \cap \q_n$, with $\p_i=\sqrt{\q_i}$. Each $\p_i$ has height 1
   \anton{}, so they are all isolated primes, so each $\q_i$ is determined: $\q_i =
   aR_{\p_i}\cap R$. By assumption, $b\in aR_{\p_i}\cap R=\q_i$, so $b\in aR$, so
   $b/a\in R$.

   Now suppose $R$ is normal. To verify Nor1, take $\p$ of height 1. Then $R_\p$ is
   clearly normal, local, noetherian, and dimension 1, so by an earlier characterization,
   $R_\p$ is a DVR. To verify Nor2, consider $\p\in \ass (R/aR)$. Localize at $\p$, so we
   may assume $(R,\p)$ is local. We want to show that $ht(\p)=1$. We can write $\p=aR:b$,
   with $b\neq 0$ in $aR$, so $a^{-1}b\p\subseteq R$.

   Case 1: If $a^{-1}b\p =R$, $\p=ab^{-1}R$, so $\p$ is principal with generator $a/b$.
   This implies $R_\p$ is a DVR (being a noetherian local domain with principal maximal
   ideal), so it is dimension 1, so $ht(\p)=1$.

   Case 2: If $a^{-1}b\p \subsetneq R$, then $a^{-1}b\p\subseteq \p$, so by the
   determinant trick, $a^{-1}b$ is integral over $R$. Since $R$ is normal, $a^{-1}b\in
   R$, so $b\in aR$, a contradiction.
 \end{proof}

 Some other results follow.
 \begin{theorem}
   A noetherian domain $R$ is a UFD if and only if every height 1 prime is principal.
 \end{theorem}
 The proof depends on the following result.
 \begin{theorem}
   A domain $R$ is a UFD if and only if
   \begin{enumerate}
     \item every non-zero non-unit is a finite product of irreducible elements, and
     \item every irreducible element generates a prime ideal.
   \end{enumerate}
 \end{theorem}
 \stepcounter{lecture}
 \setcounter{lecture}{37}
 \section{Lecture 37}

 Easy applications of the Existence theorem:
 \begin{enumerate}
   \item $K/F$ any field extension, then $Val(K)\twoheadrightarrow Val(F)$
   \item $K/k$ algebraic if and only if $Val_k(K)=\{K\}$.
   \item $Val(K)=\{K\}$ if and only if $K$ is algebraic over some $\FF_p$.
   \item charaterization of noetherian normal domains via height 1 primes. (Later, we'll
   do some Krull ring stuff, maybe)
 \end{enumerate}

 We'd like to characterize UFDs.
 \begin{theorem}[Characterization of UFDs]
   For a domain $R$, the following are equivalent.
   \begin{enumerate}
     \item $R$ is a UFD.
     \item Every non-zero prime ideal contains a prime element.
     \item Principal ideal satisfy ACC, and every irreducible element is prime.
   \end{enumerate}
 \end{theorem}
 \begin{proof}
   $3\Rightarrow 2$. Let $\p\in \spec R$ be non-zero. Fix a non-zero $a\in \p$, and write
   $a=p_1\cdots p_n$ with the $p_i$ irreducible (we can do this because we have ACC on
   principal ideals). Then some $p_i$ is in $\p$, and $p_i$ is prime by assumption.

   $2\Rightarrow 1$. Form the multiplicative set $S=\{up_1\cdots p_n|u\in U(R), n\ge 0,
   p_i \text{ prime}\}$. We claim that this multiplicative set is saturated,
   i.e.~whenever $ab\in S$, $a$ and $b$ are in $S$. This is because any factor of
   $up_1\cdots p_n$ is of the same form; suppose $xy=up_1\cdots p_n$, then $p_1$ divides
   either $x$ or $y$, so we cancel it and induct. Thus, $R\smallsetminus S=\cup \p_i$ is
   a union of primes (by earlier stuff). If some $\p_i$ is non-zero, then it contains a
   prime element $p$, which would be in $S$. Thus, each $\p_i$ is zero, so
   $S\cup\{0\}=R$. So every non-zero element has a prime factorization, from which
   uniqueness follows in the usual way (note that we needed a prime factorization, not
   just an irreducible factorization).

   $1\Rightarrow 3$. In a UFD, it is clear that irreducible elements are prime. If ACC
   fails for principal ideals, we have $a_1R\subsetneq a_2R \subsetneq \cdots$. Then we
   have $a_n=r_{n+1} a_{n+1}$, where $r_{n+1}$ is not a unit. Then
   $a_1=r_2a_2=r_2r_3\cdots r_{n+1}a_{n+1}$ for any $n$. Thus, $a_1$ is divisible by $n$
   primes (counting multiplicity) for any $n$, a contradiction.
 \end{proof}
 Let's assume the following theorem for the moment.
 \begin{theorem}[Krull's Principal Ideal Theorem]
   Let $R$ be noetherian. If $\p$ is a minimal prime over some principal ideal $aR$, then
   $ht(\p)\le 1$.
 \end{theorem}
 \begin{theorem}
   Let $R$ be a domain. If $R$ is a UFD, then every height 1 prime is principal. If $R$
   is noetherian, then the converse is true.
 \end{theorem}
 \begin{proof}
   $(\Rightarrow)$ Consider $\p\in \spec_1(R)$.\footnote{$\spec_n R$ denotes the set of
   height $n$ primes of $R$.} Consider a non-zero $a\in \p$, so $a=p_1\cdots p_n$, for
   some primes $p_i$. Then some $p_i\in \p$, so $0\subsetneq(p_i)\subseteq \p$. Since
   $\p$ is height 1, we get $\p=(p_i)$.

   $(\Leftarrow)$ Now we assume $R$ is noetherian and every height 1 prime is principal.
   We will verify property 2 in the characterization of UFDs. If $\p$ is a non-zero
   prime, it contains some non-zero principal ideal $aR$. By Zorn's Lemma, there is a
   minimal prime $\p'$ over $aR$ contained in $\p$. By the PIT, $\p'$ has height 1 (it
   cannot be zero because we are in a domain). By assumption, $\p'$ is principal,
   generated by some prime element (which is in $\p$).
 \end{proof}
 \noindent\underline{Easy fact}: If $R$ is a UFD and $S$ is a multiplicative set, then
 $R_S=S^{-1}R$ is a UFD.
 \begin{theorem}[Nagata]
   Assume $R$ is a domain, and $S$ is a multiplicative set generated by some family
   $\{p_i\}$ of prime elements. Then $R$ is a UFD if and only if $R$ has
   ACC$_\text{prin}$ and $R_S$ is UFD.
 \end{theorem}
 \begin{proof}
   $(\Rightarrow)$ Follows from the easy fact and the characterization of UFDs.

   $(\Leftarrow)$ Assuming the given conditions, we will check condition 2 of the
   characterization of UFDs. Fix a non-zero $\p\in \spec R$. If $\p\cap
   S\neq\varnothing$, then $\p$ contains a prime element because $S$ is generated by
   prime elements and $\p$ is prime. So assume $\p\cap S=\varnothing$. Upon localizing at
   $S$, we know that $\p_S$ contains some prime element because $R_S$ is a UFD. Take
   $\pi\in \p$ which maps to a prime element $\pi\in \p_S$ (we can scale by ``units''
   from $S$ if needed). If $\pi$ is divisible by some $p_i$, say $\pi=\pi_1 p_i$, then
   $\pi_1\in \p$ and $\pi_1$ maps to the same prime element (well, an associate) in
   $\p_S$. Repeating, we may assume $\pi$ has no factor $p_i$ (because we have ACC on
   principal ideals in $R$). Now we claim that $\pi$ is a prime in $\p$. To check this,
   assume $\pi|ab$. Locally, we have $\pi|a$ (or $b$), so $p_{i_1}\cdots p_{i_k}a = \pi
   r$ for some $r\in R$ and some $p_{i_j}$. Since no $p_i$ divides $\pi$, they must all
   divide $r$. So $a=\pi r'$ for some $r'\in R$, so $\pi|a$. Thus, $\pi$ is prime, as
   desired.
 \end{proof}
 We all know the following theorem.
 \begin{theorem}[Gauss]
   If $R$ is a UFD, then $R[x]$ is a UFD.
 \end{theorem}
 However, $R[[x]]$ may fail to be a UFD, even if $R$ is noetherian.
 \begin{example}
   Let $R=\FF_2[x,y,z]/(x^2+y^3+z^7)$. This is a noetherian UFD, but $R[[t]]$ is not a
   UFD.
 \end{example}
 \begin{theorem}
   If $R$ is a PID, then $A=R[[x]]$ is a UFD.
 \end{theorem}
 \begin{proof}
   Again, we'll check that second condition for $A$. Let $\P\in\spec A$ be non-zero. If
   $\P\cap R$ is generated by $n$ elements, then $\P$ is generated by at most $n+1$
   elements (we only need the extra generator if $x\in \P$) as in the proof of the
   Hilbert basis theorem. If $x\in \P$, we are done because $x$ is a prime element. If
   $x\not\in \P$, $\P$ is generated by $n$ elements. Since $R$ is a PID, $n=1$, so $\P$
   is principal, generated by some prime element.
 \end{proof}
 \stepcounter{lecture}
 \setcounter{lecture}{38}
 \section{Lecture 38}

 Normality is a local property. If you localize a UFD, it is still a UFD, but if all the
 localizations at maximal ideals are UFDs, the ring need not be a UFD, so being a UFD is
 not a local property. For example, $R=\mathbb{Z}[\sqrt{-5}]$, then $2\cdot
 3=(1+\sqrt{-5})(1-\sqrt{-5})$ are two essensially different factorizations (you can
 check that all the everything is irreducible using the norm). Since $R$ is the full ring
 of algebraic integers of $\QQ[\sqrt{-5}]$, it is a Dedekind domain, so all its
 localizations are DVRs, so they are UFDs.

 How do you use the Nagata theorem to check if a domain $R$ is a UFD? Find a prime
 element $p\in R$, and look at $R[1/p]$. We know by the theorem that if the later is a
 UFD, then so is $R$. Perhaps we can find some UFD which localizes to $R[1/p]$.

 For example, consider $R_n=\RR[x_0,\cdots ,x_n]/(x_0^2+\cdots +x_n^2-1)$, the coordinate
 ring of the $n$-sphere, and let $A_n=R_n\otimes_\RR \CC$.
 \begin{theorem}
   (1) $R_n$ is a UFD if $n\ge 2$. (2) $A_n$ is a UFD if $n\ge 3$ (or if $n=1$).
 \end{theorem}
 \begin{proof}
   (1) First we show that $1-x_0\in R$ is prime. To see this, note that
   $R/(1-x_0)=\RR[x_1,\dots, x_n]/(x_1^2+\cdots +x_n^2=0)$. Since $n\ge 2$, the sum of
   squares is irreducible, so it is prime. Thus, $R/(1-x_0)$ is a domain.

   Let $t:=(1-x_0)^{-1}$. The localization is $\RR[x_0,\cdots ,x_n,t] = \RR[tx_1,\dots,
   tx_n,t^{-1}]$ (all these adjunctions are done in the quotient ring of $R$). To see
   this, note that $tx_i$ is in the left hand side, and $t^{-1}=1-x_0$ is also in the
   left hand side. To see the reverse inclusion, note that $x_0=1-t^{-1}$ and
   $x_i=t^{-1}\cdot tx_i$ for $i\ge 1$. Finally, $(tx_1)^2+\cdots
   (tx_n)^2=t^2-t^2x_0^2=t^2-(t-1)^2=2t-1$, so $t$ is in the right hand side. But
   $\RR[tx_1,\cdots, tx_n]$ is a polynomial ring (which contains $t$ by the computation
   above), and the right hand side is the localization at $t$.

   (2) For $n=1$, we have $\CC[x_0,x_1]/(x_0^2+x_1^2)$. We change variables to
   $z=x_0+ix_1$ and $\bar z=x_0-ix_1$. Then the relation is $z\bar z=1$, so $\bar
   z=z^{-1}$. Thus, we have the ring $\CC[z,z^{-1}]$, the Laurent polynomial ring, which
   is a UFD.

   Now we do $n\ge 3$.\\
   \underline{Case 1}: $n=2k$ with $k\ge 2$. Do a change of variables to get
   $A_n=\CC[z_0,\cdots, z_{2k}]/(z_0^2+z_1z_2+\cdots +z_{2k-1}z_{2k}=1)$. Now we check
   that $z_1$ is a prime: $A/(z_1)=\CC[z_0, z_2,\cdots, z_{2k}]/(z_0^2+z_3z_4+\cdots
   +z_{2k-1}z_{2k}=1) = A_{n-2}[z_2]$ which is a domain. Is $A[z_1^{-1}]$ a domain? Well,
   $A[z_1^{-1}]=\CC[z_0,z_1,\rlap{\rule[3pt]{6pt}{.4pt}}z_2,z_3\cdots, z_{2k},
   z_1^{-1}]/(z_0^2+z_1z_2+\cdots +z_{2k-1}z_{2k}=1) = \CC[z_0,z_1,z_3\cdots,
   z_{2k}][z_1^{-1}]$ is a localization of a UFD, so it is a UFD.

   \noindent \underline{Case 2}: $n=2k+1$ with $k\ge 1$. As in case 1, we change
   variables to get $A_n=\CC[z_0,\cdots, z_{2k+1}]/(z_0z_1+\cdots +z_{2k-1}z_{2k}=1)$.
   Then $z_0$ is a prime: $A_n/(z_0)\cong A_{n-2}[z]$ is a domain (this is why $n=2$
   doesn't work, because $A_{n-2}=A_0$ is not a domain). Now check that
   $A[z_0^{-1}]=\CC[z_0,\cdots, z_{2k+1},z_0^{-1}]/(z_0z_1+\cdots +z_{2k-1}z_{2k}=1) =
   \CC[z_0,z_2,\cdots, z_{2k+1}][z_0^{-1}]$ is a localization of a UFD.
 \end{proof}
 \begin{theorem}
   $R_1$ and $A_2$ are not UFDs.
 \end{theorem}
 Intuitively, $R_1=\RR[x,y]/(x^2+y^2=1)$, so we get $x^2=(1+y)(1-y)$, and we can believe
 that these are two different factorizations. $A_1=\CC[x,y,z]/(x_2+y^2+z^2=1)$, so we get
 $(x+iy)(x-iy)=(1-z)(1+z)$.

 \subsection{Chapter IV. Dedekind domains and Krull domains}
 We will only give an overview.

 Let $R$ be a domain, with $K=Q(R)$. A \emph{fractional ideal} is an $R$-submodule
 $A\subseteq {}_R K$ so that there exists a non-zero $r\in R$ so that $rA\subseteq R$.
 \begin{example}
   \begin{itemize}
   \item Any ideal $I\subset R\subseteq K$ is a fractional ideal; actual ideals are
   sometimes called \emph{integral} ideals.

   \item If $A\subseteq {}_R K$ is finitely generated, then it is a fractional ideal.

   \item $s\in K$ is almost integral (all powers have a common denominator) if and only
   if $R[s]$ is a fractional ideal.
   \end{itemize}
   \vspace*{-1.7\baselineskip}
 \end{example}
 Given fractional ideals $A$ and $B$, $A\cdot B=\{\sum a_ib_i\}$ is a fractional ideal.
 Thus, the set of fractional ideals $Id(R)$ forms a monoid (with identity $R$).
 \begin{definition}
   An $R$-submodule $A\subseteq {}_RK$ is called \emph{invertible} if there is some
   $R$-submodule $B\subseteq {}_RK$ so that $A\cdot B=R$.
 \end{definition}
 Such an $A$ is always finitely generated as an $R$-module (express 1 as $\sum a_ib_i$
 and show that the $a_i$ generate), so all invertible ideals are fractional ideals. The
 invertible fractional ideals are exactly the invertible elements of the monoid $Id(R)$.
 Let $Inv(R)$ be the group of invertible fractional ideals. We have that $Prin(R)$, the
 set of principal fractional ideals, forms a subgroup. The factor group
 $C(R)=Inv(R)/Prin(R)$ is called the \emph{ideal class group} of $R$.

 \begin{definition}
   A domain $R$ is \emph{Dedekind} if (1) $R$ is noetherian, (2) $R$ is normal, and (3)
   $\dim R\le 1$.\footnote{Some people like to say $\dim R=1$ to exclude fields from
   being Dedekind. We allow fields to be Dedekind so that PIDs $\Rightarrow$ Dedekind.
   However, we like to think of Dedekind domains as locally DVRs, which fails for fields.
   Whatever, you can never make everybody happy.}
 \end{definition}
 The following hold for Dedekind domains.
 \begin{enumerate}
   \item $R$ is a field or $R$ is noetherian with $R_\m$ a DVR for all $\m\in \Max R$.
   \item All non-zero fractional ideals are invertible ($Id(R)=Inv(R)$).
   \item Every ideal is a finite product of primes (note that we do not assume
   noetherian)
 \end{enumerate}
 \stepcounter{lecture}
 \setcounter{lecture}{39}
 \section{Lecture 39}

 Final Exam: Do three exercises. \\
 $R$ is a domain throughout. $C(R)=Inv(R)/Prin(R)$ is the ideal class group. The
 following are equivalent.
 \begin{itemize}
   \item $R$ is a Dedekind domain
   \item $R$ is noetherian, normal, and of dimension $\le 1$ (this is the definition)
   \item non-zero ideals are invertible
   \item for all $I\subset R$, $I=\p_1^{r_1}\cdots \p_n^{r_n}$ uniquely (except for $I=0$),
   with $n\ge 0$ and $r_i\ge 1$.
 \end{itemize}
 This last condition is a very important one.

 Robert Lee Moore's method: he would write all the theorems and definitions in a subject and
 have his students figure out the proofs.

 \underline{A dozen things every Good Algebraist should know about Dedekind domains}. $R$
 is a Dedekind domain.
 \begin{enumerate}
   \item $R$ is local $\Longleftrightarrow$ $R$ is a field or a DVR.
   \item $R$ semi-local $\Longrightarrow$ it is a PID.
   \item $R$ is a PID $\Longleftrightarrow$ it is a UFD $\Longleftrightarrow$ $C(R)=\{1\}$
   \item $R$ is the full ring of integers of a number field $K$ $\Longrightarrow$ $|C(R)|<
   \infty$, and this number is the \emph{class number} of $K$.
   \item $C(R)$ can be any abelian group. This is Clayborn's Theorem.
   \item For any non-zero prime $\p\in \spec R$, $\p^n/\p^{n+1}\cong R/\p$ as an
   $R$-module.
   \item ``To contain is to divide'', i.e.~if $A,B\subset R$, then $A\subseteq B$
   $\Longleftrightarrow$ $A=BC$ for some $C\subset R$.
   \item (Generation of ideals) Every non-zero ideal $B\subset R$ is generated by two elements.
   Moreover, one of the generators can be taken to be any non-zero element of $B$.
   \item (Factor rings) If $A\subset R$ is non-zero, then $R/A$ is a PIR.
   \item (Steinitz Isomorphisms Theorem) If $A,B\subset R$ are non-zero ideals, then $A\oplus
   B\cong {}_RR\oplus AB$ as $R$-modules.
   \item If ${}_RM$ is a finitely generated torsion-free $R$-module of rank $n$,\footnote{The
   rank is defined as $rk(M)=\dim_{Q(R)} M\otimes_R Q(R)$.} then it is of the
   form $M\cong R^{n-1}\oplus A$, where $A$ is a non-zero ideal, determined up to
   isomorphism.
   \item If ${}_RM$ is a finitely generated torsion $R$-module, then $M$ is uniquely of the
   form $M\cong R/A_1\oplus
   \cdots \oplus R/A_n$ with $A_1\subsetneq A_2\subsetneq \cdots \subsetneq
   A_n\subsetneq R$.
   \item[Bonus.] Any finitely generated ${}_RM$ can be written as $M\cong M_t \oplus
   M/M_t$, where $M_t$ is the torsion submodule.
 \end{enumerate}

 \begin{theorem}[Very Strong Krull-Akizuki Theorem]
   Let $R$ be a noetherian domain of Krull dimension $1$, with $K=Q(R)$. Let $L/K$ be a
   finite field extension, and let $S$ be a ring $R\subseteq S\subseteq L$. Then $S$ is
   noetherian of dimension $\le 1$.
 \end{theorem}
 \begin{corollary}
   If the $R$ is also normal (so it is Dedekind), then the integral closure of $R$ in $L$
   is a Dedekind domain.
 \end{corollary}
 We knew this result in the special case when $L/K$ is a finite separable extension.

 \underline{Pr\"ufer Domains} in some sense generalize Dedekind to non-noetherian and
 higher-dimensional cases.
 \begin{definition}
   A domain $R$ is \emph{Pr\"ufer} if every non-zero finitely generated ideal is
   invertible.
 \end{definition}
 Clearly Pr\"ufer and noetherian is Dedekind.
 \begin{example}
   $\{$Valuation rings$\}\subseteq \{$B\'ezout
     domains$\}\subseteq\{$Pr\"ufer$\}$.
 \end{example}
 There are about 20 different characterization of Pr\"ufer domains. Here are a few.
 \begin{theorem}
   Let $R$ be a domain. The following are equivalent.
   \begin{enumerate}
     \item $R$ is Pr\"ufer.
     \item For any $\m\in \Max R$, $R_\m$ is a valuation ring.
     \item $A\cap (B+C)=(A\cap B)+(A\cap C)$ for all ideals $A,B,C\subset R$.
     \item All ideals in $R$ are flat modules.
   \end{enumerate}
 \end{theorem}

 \underline{Krull domains \& Divisors}.\\
 Mori's heartbreak story. Let $R$ be a ring with quotient field $K$. We'd like to form
 $R^*$, the integral closure of $R$. Mori discovered that if $R$ is noetherian, $R^*$
 need not be noetherian.

 But the implication $R$ noetherian implies $R^*$ noetherian holds in the following
 cases:
 \begin{enumerate}
   \item $\dim R=1$, by the Very Strong Krull-Akizuki Theorem, with $L=K$.
   \item (Mori, Nagata) $\dim R=2$.
   \item $R$ is an affine algebra.
 \end{enumerate}
 Grothendieck defined \emph{Japanese rings}, which have to do with this stuff.
 \begin{definition}
   $R\subseteq K$ as usual is a \emph{Krull domain} if there exists a family
   $\{R_i\}_{i\in I}\subseteq DVal(K)$ (DVRs of $K$) such that
   \begin{enumerate}
     \item $R = \bigcap R_i$
     \item for all non-zero $a\in R$, $a\in U(R_i)$ for almost all $i$.
   \end{enumerate}
 \end{definition}
 From the viewpoint of valuation theory, let $v_i:R_i\twoheadrightarrow \mathbb{Z}\cup \infty$,
 then $R=\{x\in K|v_i(x)\ge 0 \text{ for all }i\}$ and for all non-zero $x\in R$,
 $v_i(x)=0$ for almost all $i$.

 The set $\{R_i\}$ is called the \emph{defining family} of $R$.
 \stepcounter{lecture}
 \setcounter{lecture}{40}
 \section{Lecture 40}

 Two corrections to the notes:\\
 p.~129, line 8: change ``Every'' to ``Up to associates, every''\\
 p.~131, line 20: Change ``the polynomial ring'' to ``a polynomial ring''.

 Note that a Krull domain is always completely normal (because DVRs are always completely
 normal).

 \begin{example}
   \begin{enumerate}
     \item Take $R$ a finite intersection of DVRs of $K$ (then the second condition is
     automatically satisfied.
     \item Noetherian normal domains are Krull. We can take $\{R_i\}=\{R_\p|\p\in \spec_1
     R\}$. Recall that an element is in a finite number of height 1 primes.
     \item UFDs are Krull. The defining family is the set of $R_{(\pi)}$, where $\pi$ is
     a prime element.
   \end{enumerate}
   \vspace*{-1.7\baselineskip}
 \end{example}
 \begin{theorem}
   $\{$Krull domains of dimension $\le 1\} = \{$Dedekind domains$\}$.
 \end{theorem}
 $\supseteq$ is clear. For the other direction, the only hard part is to show that $R$
 is noetherian. We won't do it here.

 Krull domains behave very well with respect to ``closure properties'':
 \begin{enumerate}
   \item $R$ Krull $\Longrightarrow$ any localization of $R$ is Krull.
   \item $R$ Krull $\Longrightarrow$ $R[\{x_i\}_{i\in I}]$ is Krull.
   \item $R$ Krull $\Longrightarrow$ $R[[x]]$ is Krull.
   \item Let $R$ be Krull with $Q(R)=K$, and let $L$ be a finite extension of $K$, with
   $S$ the integral closure of $R$ in $L$. If $R$ is Krull, then so is $S$.
   \item (Mori-Nagata Theorem) If $R$ is a noetherian domain then the integral closure
   $R^*$ is Krull.
 \end{enumerate}
 \begin{theorem}
   If $R$ is Krull, then for every $\p\in \spec_1 R$, $R_\p$ is a DVR. Moreover,
   $\{R_\p|\p\in \spec_1 R\}$ is a defining family for $R$.
 \end{theorem}
 \begin{definition}
   If $R$ is a Krull domain, the \emph{divisor class group} is $Cl(R) =
   \frac{D(R)}{div(K^\times)}$. $D(R)$ is the group of \emph{divisors}, the free abelian
   group on the set of height 1 primes. For each height 1 prime $\p$, we have a valuation
   $v_\p$. We define the set of \emph{principal divisors} to be $div(K^\times) = \{div(f)
   = \sum v_\p(f) \p| f\in K^\times\}$.
 \end{definition}
 \begin{theorem}
   If $R$ is a Krull domain, then $Cl(R)$ is trivial if and only if $R$ is a UFD.
 \end{theorem}
 $\Leftarrow$ is clear because each height 1 prime is principal. The other way is not
 hard either.

 Finally, the ideal class group $C(R)$ injects into $Cl(R)$ (with equality if $R$ is
 regular, whatever that means).

 \subsection{Chapter IV: Dimension Theory}

 \begin{definition}
   Let $k$ be a field, and let $B$ be a $k$-algebra. We define $tr.d._k B = \sup
   \{tr.d._k (B/\p)|\p\in \Min(B)\}$.
 \end{definition}
 \begin{theorem}[Noether normalization]
   Let $k$ be a field, and $B$ an affine $k$-algebra. Then there exist algebraically
   independent (over $k$) $x_1,\dots, x_n\in B$ such that $B$ is integral over
   $A=k[x_1,\dots, x_n]$. In particular, since $B$ is finitely generated over $k$, it is
   module-finite over $A$.
 \end{theorem}
 \begin{example}
   Let $B=k[t^2,t^3]\subseteq k[t]$. Here $A=k[t^2]$, and it is clear that $t^3$ is
   integral over $A$.
 \end{example}
 \begin{example}
   Let $B=k[t,t^{-1}]$. Take $A=k[t+t^{-1}]$. Then note that $t$ and $t^{-1}$ satisfy
   $(x-t)(x-t^{-1}) = x^2-(t+t^{-1})x+1\in A[x]$.
 \end{example}
 \begin{proof}
   Write $B=k[y_1,\dots, y_m]$ and induct on $m$. The case $m=0$ is trivial. If
   $y_1,\dots, y_m$ are algebraically independent, we take $A=B$ and we're done. Thus, we
   may assume there is some dependence $f(y_1,\dots, y_m)=0$. Take $r$ larger than any
   exponent in $f$. Define $z_i:=y_i-y_1^{r^{i-1}}$ for $i\ge 2$. Then we get that
   $0=f(y_1,z_2+y_1^{r}, z_3+y_1^{r^2}, \dots, z_m + y_1^{r^m})$ has leading term
   $by_1^N$ for some huge $N$ and $b\neq 0$. This equation tells us that $y_1$ is
   integral over $B':=k[z_2,\dots, z_m]$. It is clear that all the other $y_i$ are also
   integral over $B'$, so $B$ is integral over $B'$. By induction, we can find a
   polynomial ring $A$ so that $B'$ is integral over $A$.
 \end{proof}
 It is clear that $tr.d._k B\le m$.
 \stepcounter{lecture}
 \setcounter{lecture}{41}
 \section{Lecture 41}

 In the notes, p.~129, line 7 (statement of (6.19)), the ``if'' part needs a correction.

 We replace $B$ by $S$ and $A$ by $R$.

 Basic trick:\\
 $(\ast)$ If $B/A$ is an integral extension of $k$-domains, then $tr.d._k B=tr.d._k A$.
 This is because the field extension $Q(A)\subseteq Q(B)$ is algebraic.

 \begin{theorem}
   Keep all notation from before.
   \begin{enumerate}
     \item $n=tr.d._k S$
     \item for $I\subset S$, $tr.d._k S/I\le tr.d._k S$
     \item $n$ is the largest integer $d$ such that $S$ has $d$ algebraically independent elements
     \item for all $k$-subalgebras $T\subseteq S$, $tr.d._k T\le tr.d._k S$.
   \end{enumerate}
 \end{theorem}
 \begin{proof}
   (1) Let $\P\in \spec S$, $\p=\P\cap R$. Then $R/\p\subseteq S\P$ is integral. By
   $(\ast)$, $tr.d._k S/\P = tr.d._k R/\p\le n$. By Going Up, there is a $\P_0\in \spec
   S$ (which we may assume is minimal) so that $\P_0\cap R=(0)$. Now we have $R\subseteq
   S/\P_0$. By $(\ast)$, $n=tr.d._k R= tr.d._k S/\P_0\le tr.d._k S$.

   (2) Consider $I\subset S$. We may assume $I\in \spec S$. By what we did in part (1), we get
   $tr.d._k (S/I)\le n=tr.d._k S$.

   (3) Let $d$ be as defined. We already know $n\le d$. Say that $R_0=k[t_1,\dots, t_d]$
   is a polynomial algebra in $S$. Say $\Min S=\{\P_1,\dots, \P_r)$, and let $\p_i =
   \P_i\cap R_0$. We have that
   \[
   \p_1\cdots \p_r \subseteq \P_1\cdots \P_r \cap R_0 \subseteq \nil (S)\cap R_0 = \nil R_0 =
   0.
   \]
   It follows that some $\p_i$ is zero, say $\p_1=0$. By $(\ast)$ applied to
   $R_0=R_0/\p_1\subseteq S/\P_1$ to get $d=tr.d._k R_0 = tr.d._k S/\P_1 \le tr.d._k
   S=n$.

   (4) follows from (3) immediately.
 \end{proof}
 In $k[x_1,\dots, x_n]$, there is an obvious chain of primes
 \[
  (0)\subsetneq (x_1)\subsetneq \cdots \subsetneq (x_1,\dots, x_n)
 \]
 which implies that $\dim k[x_1,\dots, x_n]\ge n$.
 \begin{theorem} \label{lec41T:dim=trd}
   For every affine $k$-algebra $S$, $\dim S=tr.d._k S$.
 \end{theorem}
 \begin{remark}
   This theorem includes Zariski's Lemma, which says that an affine algebra over $k$
   which is a field must be a finite algebraic extension. To see this from the theorem,
   let $S$ be a field, then $\dim S=0$. It follows that $tr.d._k S=0$, so $S$ is
   algebraic over $k$. It is finite because $S$ is finitely generated.

   This provides an alternative approach to Hilbert's Nullstellensatz.
 \end{remark}
 Before we prove the theorem, we need a lemma.
 \begin{lemma}
   Let $S$ be a $k$-affine domain with $tr.d._k S=n$, and let $\p\in \spec_1 S$. Then
   $tr.d._k (S/\p)=n-1$.
 \end{lemma}
 \begin{proof}
   \underline{Case 1}: assume $S=k[x_1,\dots, x_n]$ is a polynomial algebra. In this
   case, height 1 primes are principal, so $\p=(f)$ for some $f$. Say $f$ has positive
   degree with respect to $x_1$, so $f = g_r(x_2,\dots, x_n)x_1^r + \cdots$. We have
   that $k[x_2,\dots, x_n]\cap (f)=(0)$ (just look at degree with respect to $x_1$). It
   follows that $k[x_2,\dots, x_n]\hookrightarrow S/(f)$, so $\bar x_2,\dots, \bar x_n$
   are algebraically independent in $S/\p$. By $\bar x_1$ is algebraic over $Q(k[\bar
   x_2,\dots, \bar x_n])$ as witnessed by $f$. This, $tr.d._k S/\p=n-1$.

   \underline{Case 2}: reduction to case 1. Let $R=k[x_1,\dots, x_n]$ be a Noether
   normalization for $S$, and let $\p_0=\p\cap R$. Observe that Going Down applies
   (because $S$ is a domain and $R$ is normal). It follows that $ht_R(\p_0)=ht_S(\p)=1$.
   By case 1, we get that $tr.d. (R/\p_0)=n-1$. By $(\ast)$, we get that $tr.d.
   R/\p_0=tr.d. (S/\p)$.
 \end{proof}
 \begin{proof}[Proof of Theorem \ref{lec41T:dim=trd}]
   Let $n=tr.d._k S$. We induct on $n$. If $n=0$, then $S$ is algebraic over $k$. Then
   for any minimal prime $\P\subseteq S$, $S/\P$ is a field. Thus, minimal primes are
   maximal, so $S$ has dimension 0.

   Now assume the equation is true up to $n-1$. After replacing $S$ by a Noether
   normalization (without affecting $\dim S$ or $tr.d._k S$), we may assume
   $S=k[x_1,\dots, x_n]$ is a polynomial algebra. Consider any prime chain of length $r$
   in this polynomial algebra. Let $\p$ be the smallest non-zero prime in the chain, and
   let $f\in \p$ be a non-zero irreducible, then $(f)$ is prime and contained in $\p$. By
   case 1 of the Lemma, $tr.d._k S/(f)=n-1$. By the inductive hypothesis, $\dim
   S/(f)=n-1$. But $S/(f)$ has a prime chain of length $r-1$. Thus, $r-1\le n-1$, so
   $r\le n$.
 \end{proof}
 \stepcounter{lecture}
 \setcounter{lecture}{42}
 \section{Lecture 42}

 What we would've done next:
 \begin{enumerate}
   \item Generalized princepal ideal theorem: $R$ noetherian, $\p$ minimal prime over
   $(a_1,\dots, a_n$, then $ht(\p)\le n$.\\
   Corollary 1: $\spec R$ satisfies DCC (the length of a chain from $\p$ is bounded by
   $ht(\p)< \infty$ ($R$ noetherian, so $\p$ is finitely generated)).\\
   Corollary 2: $R$ local $\Rightarrow$ $\dim R< \infty$\\
   (Nagata): There exist ``bad noetherian rings'' which are infinite dimensional noetherian
   domains.

   \item $R$ noetherian of dimension $n$ $\Longrightarrow$ $\dim R[x] = n+1$.\\
   In general, $n+1\le \dim R[x] \le 2n+1$ (``Seidenberg bounds''), and these bounds are
   tight.
 \end{enumerate}

 \begin{definition}
   A ring $R$ is \emph{catenary} if given any two primes $\p\subsetneq \p'$, any two
   maximal prime chains from $\p$ to $\p'$ have the same length.
 \end{definition}
 Nagata showed that there are noetherian domains which are not catenary.

 \begin{definition}
   If $\p\in \spec R$, then $\dim \p:= \dim R/\p$.
 \end{definition}
 \begin{theorem}
   Any $k$-affine algebra $S$ is catenary (even if $S$ is not a domain). In fact, any
   saturated prime chain from $\p$ to $\p'$ has length $\dim \p - \dim \p'$. If $S$ is a
   domain, then all maximal ideals have the same height.
 \end{theorem}
 \begin{proof}
   Consider any chain $\p\subsetneq \p_1\subsetneq \cdots \subsetneq \p_r = \p'$. Then we
   get the chain
   \[
    S/\p \twoheadrightarrow S/\p_1 \twoheadrightarrow \cdots \twoheadrightarrow S/\p_r
    = S/\p'
   \]
   Here $\p_i/\p_{i-1}$ is height 1 in $S/\p_{i-1}$, so each arrow decreases the
   transcendence degree by exactly 1. Therefore, $tr.d._k S/\p' = tr.d._k S/\p -r$.
   \[
    r = tr.d._k S/\p - tr.d._k S/\p' = \dim S/\p - \dim S/\p' = \dim \p-\dim \p'.
   \]
   To get the last statement, take $\p=0$ and $\p'=\m$. Then we get that $ht(\m)=\dim S$.
 \end{proof}
 Note that the last statement fails in general.
 \begin{example}
   Take $S=k\times k[x_1,\dots, x_n]$. Then $ht(0\times k[x_1,\dots, x_n])=0$, but
   $ht\bigl(k\times (x_1,\dots, x_n)\bigr) = n$.
 \end{example}
 But that example is not connected.
 \begin{example}
   $S = k[x,y,z]/(xy,xz)$.
 \end{example}
 But this example is not a domain. In general, for any prime $\p$ in any ring $S$, we
 have
 \[
    ht(\p) + \dim \p \le \dim S.
 \]
 \begin{theorem}
   Let $S$ be an affine algebra, with $\Min S = \{\p_1,\dots, \p_r\}$. Then the following
   are equivalent.
   \begin{enumerate}
     \item $\dim \p_i$ are all equal.
     \item $ht(\p)+\dim \p =\dim S$ for all primes $\p\in \spec S$. In particular, if $S$
     is a domain, we get this condition.
   \end{enumerate}
 \end{theorem}
 \begin{proof}
   $(1\Rightarrow 2)$ $ht(\p)$ is the length of some saturated prime chain from $\p$ to
   some minimal prime $\p_i$. This length is $\dim \p_i - \dim \p = \dim S - \dim \p$ (by
   condition 1). Thus, we get $(2)$.

   $(2\Rightarrow 1)$ Apply (2) to the minimal prime $\p_i$ to get $\dim \p_i=\dim S$ for
   all $i$.
 \end{proof}
 We finish with a (non-affine) noetherian domain $S$ with maximal ideals of different
 heights. We need the following fact.\\
 \underline{Fact}: If $R$ is a ring with $a\in R$, then there is a canonical $R$-algebra
 isomorphism $R[x]/(ax-1) \cong R[a^{-1}]$, $x\leftrightarrow a^{-1}$.
 \begin{example}
   Let $\bigl(R,(\pi)\bigr)$ be a DVR with quotient field $K$. Let $S=R[x]$, and assume
   for now that we know that $\dim S=2$. Look at $\m_2=(\pi,x)$ and $\m_1=(\pi x-1)$.
   Note that $\m_1$ is maximal because $S/\m_1 = K$. It is easy to show that
   $ht(\m_1)=1$. However, $\m_2\supsetneq (x)\supsetneq (0)$, so $ht(\m_2)=2$.
 \end{example}
 Now let's come back to result I.1.1. The result we've just proven says that $ax-1\in
 U(R[x])$ if and only if $a\in \nil R$.
