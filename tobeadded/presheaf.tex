\documentclass[11pt]{article}
\usepackage[hmargin=26mm,vmargin=26mm]{geometry}
\usepackage{amsmath, amssymb, amsthm, verbatim, graphicx, mathabx, fancyhdr}
\usepackage[all,cmtip]{xy}

\newtheorem{proposition}{Proposition}

\begin{document}

\begin{proposition}
Let $ A$ be a ring and let $ X = \mathrm{Spec}(A)$. Then the assignment of the ring $A_f$ to the basic open set $X_f$ defines a presheaf of rings on $X$.
\end{proposition}

\begin{proof} \mbox{}

\emph{Part (i)}. If $ X_g \subseteq X_f$ are basic open sets, then there exist $ n \geq 1$ and $ u \in A$ such that $ g^n = uf$.

\emph{Proof of part (i)}. Let $ S = \{g^n : n \geq 0\}$ and suppose $ S \cap (f) = \emptyset$. Then the extension $ (f)^e$ into $ S^{-1}A$ is a proper ideal, so there exists a maximal ideal $ S^{-1}\mathfrak{p}$ of $ S^{-1}A$, where $ \mathfrak{p} \cap S = \emptyset$. Since $ (f)^e \in S^{-1}\mathfrak{p}$, we see that $ f/1 \in S^{-1}\mathfrak{p}$, so $ f \in \mathfrak{p}$. But $ S \cap \mathfrak{p} = \emptyset$ implies that $ g \notin \mathfrak{p}$. This is a contradiction, since then $ \mathfrak{p} \in X_g \setminus X_f$.

\emph{Part (ii)}. If $ X_g \subseteq X_f$, then there exists a unique map $ \rho : A_f \to A_g$, called the restriction map, which makes the following diagram commute.
\[ \xymatrix{ & A \ar[dl] \ar[dr] & \\ A_f \ar[rr] & & A_g } \]

\emph{Proof of part (ii)}. 
Let $ n \geq 1$ and $ u \in A$ be such that $ g^n = uf$ by part (i). Note that in $ A_g$,
\[ (f/1)(u/g^n) = (fu/g^n) = 1/1 = 1 \]
which means that $ f$ maps to a unit in $ A_g$. Hence every $ f^m$ maps to a unit in $ A_g$, so the universal property of $ A_f$ yields the desired unique map $ \rho : A_f \to A_g$.

\emph{Part (iii)}. 
If $ X_g = X_f$, then the corresponding restriction $ \rho : A_f \to A_g$ is an isomorphism.

\emph{Proof of part (iii)}. 
The reverse inclusion yields a $ \rho' : A_g \to A_f$ such that the diagram
\[ \xymatrix{
& A \ar[dr] \ar[dl] & \\
A_f \ar@/^/[rr]^{\rho} & & A_g \ar@/^/[ll]^{\rho'}
} \]
commutes. But since the localization map is epic, this implies that $ \rho \rho' = \rho' \rho = \mathbf{1}$.

\emph{Part (iv)}.
If $ X_h \subseteq X_g \subseteq X_f$, then the diagram
\[ \xymatrix{
A_f \ar[rr] \ar[dr] & & A_h \\
& A_g \ar[ur] &
} \]
of restriction maps commutes.

\emph{Proof of part (iv)}.
Consider the following tetrahedron.
\[ \xymatrix{
& A \ar[dl] \ar[dr] \ar[dd] & \\
A_f \ar@{.>}[rr] \ar[dr] & & A_h \\
& A_g \ar[ur] &
} \]
Except for the base, the commutativity of each face of the tetrahedron follows from the universal property of part (ii). But its easy to see that commutativity of the those faces implies commutativity of the base, which is what we want to show.

\emph{Part (v)}.
If $ X_{\tilde{g}} = X_g \subseteq X_f = X_{\tilde{f}}$, then the diagram
\[ \xymatrix{
A_f \ar[r] \ar[d] & A_g \ar[d] \\
A_{\tilde{f}} \ar[r] & A_{\tilde{g}}
} \]
of restriction maps commutes. (Note that the vertical maps here are isomorphisms.)

\emph{Proof of part (v)}.
By part (iv), the two triangles of
\[ \xymatrix{
A_f \ar[r] \ar[d] \ar[dr] & A_g \ar[d] \\
A_{\tilde{f}} \ar[r] & A_{\tilde{g}}
} \]
commute. Therefore the square commutes.

\emph{Part (vi)}.
Fix a prime ideal $ \mathfrak{p}$ in $ A$. Consider the direct system consisting of rings $ A_f$ for every $ f \notin \mathfrak{p}$ and restriction maps $ \rho_{fg} : A_f \to A_g$ whenever $ X_g \subseteq X_f$. Then $ \varinjlim A_f \cong A_{\mathfrak{p}}$.

\emph{proof of part (vi)}.
First, note that since $ f \notin \mathfrak{p}$ and $ \mathfrak{p}$ is prime, we know that $ f^m \notin \mathfrak{p}$ for all $ m \geq 0$. Therefore the image of $ f^m$ under the localization $ A \to A_\mathfrak{p}$ is a unit, which means the universal property of $ A_f$ yields a unique map $ \alpha_f : A_f \to A_\mathfrak{p}$ such that the following diagram commutes.
\[ \xymatrix{
& A \ar[dr] \ar[dl] & \\
A_f \ar[rr]^{\alpha_f} & & A_{\mathfrak{p}}
} \]
Then consider the following tetrahedron.
\[ \xymatrix{
& A \ar[dl] \ar[dr] \ar[dd] & \\
A_f \ar@{.>}[rr] \ar[dr] & & A_h \\
& A_\mathfrak{p} \ar[ur] &
} \]
All faces except the bottom commute by construction, so the bottom face commutes as well. This implies that the $ \alpha_f$ commute with the restriction maps, as necessary. Now, to see that $ \varinjlim A_f \cong A_\mathfrak{p}$, we show that $ A_\mathfrak{p}$ satisfies the universal property of $ \varinjlim A_f$.

Suppose $ B$ is a ring and there exist maps $ \beta_f : A_f \to B$ which commute with the restrictions. Define $ \beta : A \to B$ as the composition $ A \to A_f \to B$. The fact that $ \beta$ is independent of choice of $ f$ follows from the commutativity of the following diagram.
\[ \xymatrix{
& A \ar[dr] \ar[dl] & \\
A_f \ar[rr]^{\rho_{fg}} \ar[dr]^{\beta_f} & & A_g \ar[dl]_{\beta_g} \\
& B
} \]
Now, for every $ f \notin \mathfrak{p}$, we know that $ \beta(f)$ must be a unit since $ \beta(f) = \beta_f(f/1)$ and $ f/1$ is a unit in $ A_f$. Therefore the universal property of $ A_\mathfrak{p}$ yields a unique map $ A_{\mathfrak{p}} \to B$, which clearly commutes with all the arrows necessary to make $ \varinjlim A_f \cong A_\mathfrak{p}$.
\end{proof}

\end{document}
