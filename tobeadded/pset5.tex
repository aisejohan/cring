\documentclass[12pt, reqno]{amsart}

\usepackage{amsthm}
\usepackage{amsmath}
\usepackage{url}
\usepackage{fancyhdr}
\renewcommand{\thesection}{\arabic{section}}
\newtheorem{theorem}{Theorem}[section]

\usepackage{hyperref}

\newtheorem{lemma}[theorem]{Lemma}
\newtheorem{sublemma}[theorem]{Sublemma}
\newtheorem{corollary}[theorem]{Corollary}
\newtheorem{proposition}[theorem]{Proposition}
\theoremstyle{definition}
\newtheorem{definition}[theorem]{Definition}
\newtheorem*{remark}{Remark}
\newtheorem{example}[theorem]{Example}
\newtheorem{exercise}{ \sc Exercise}[chapter]
\newtheorem*{solution}{Solution}
\newtheorem*{question}{Question}
\newtheorem*{problem}{Problem}
\newtheorem*{dbend}{Dangerous bend}



\usepackage[top=1.3in, bottom=1.3in, left=1.5in, right=1.5in]{geometry}
\usepackage{amssymb}
\usepackage{amsfonts}
\usepackage{stackrel}
\usepackage{mathrsfs}
\usepackage{xy}
\usepackage{verbatim}


\newcommand{\whattosay}{}



\input xy
\xyoption{all}


\newcommand{\lecture}[1]{}
\newcommand{\rad}{\mathrm{rad}}
\newcommand{\im}{\mathrm{Im}}
\newcommand{\proj}{\mathrm{Proj}}
\renewcommand{\hom}{\mathrm{Hom}}
\newcommand{\id}{\mathrm{id}}
\providecommand{\cal}[1]{\mathcal{#1}}
\renewcommand{\cal}[1]{\mathcal{#1}}


%\swapnumbers

\renewcommand{\qedsymbol}{$\blacktriangle$}


\begin{document}

\title{Math 210A Problem Set 5}
\input{other/amsdata.tex}

\subsection*{1(a)}

Since $1$ spans $\ZZ / (10)$ and $1$ spans $\ZZ / (12)$, 
we see that $1 \otimes 1$ spans $\ZZ / (10) \otimes \ZZ / (12)$ and this tensor
product is a cyclic group. 

Note that 
$1 \otimes 0 = 1 \otimes (10 \cdot 0) = 10 \otimes 0 = 0 \otimes 0 = 0$
and 
$0 \otimes 1 = (12 \cdot 0) \otimes 1 = 0 \otimes 12 = 0 \otimes 0 = 0$.
Now,
$10 (1 \otimes 1) = 10 \otimes 1 = 0 \otimes 1 = 0$
and 
$12 (1 \otimes 1) = 1 \otimes 12 = 1 \otimes 0 = 0$,
so the cyclic group $\ZZ / (10) \otimes \ZZ / (12)$ has order dividing both
$10$ and $12$. This means that the cyclic group has order dividing 
$\gcd(10, 12) = 2$.

To show that the order of $\ZZ / (10) \otimes \ZZ / (12)$, define a bilinear map
$g: \ZZ / (10) \times \ZZ / (12) \to \ZZ / (2)$ via
$g : (x, y) \mapsto xy$. The universal property of tensor products then
says that there is a unique linear map 
$f: \ZZ / (10) \otimes \ZZ / (12) \to \ZZ / (2)$ making the diagram
\[ 
\xymatrix{
\ZZ / (10) \times \ZZ / (12) \ar[r]^\otimes \ar[rd]_g
	& \ZZ / (10) \otimes \ZZ / (12) \ar[d]^f \\
& \ZZ / (2).
}
\] 
commute. In particular, this means that $f (x \otimes y) = g(x, y) = xy$.
Hence, $f(1 \otimes 1) = 1$, so $f$ is surjective, and therefore, 
$\ZZ / (10) \otimes \ZZ / (12)$ has size at least two. This allows us to
conclude that $\ZZ / (10) \otimes \ZZ / (12) = \ZZ / (2)$.


\subsection*{1(b)}

\def\mod{\operatorname{mod}}

Let $\frac ab \in \QQ$ and $\frac cd \in \QQ/\ZZ$, where $a, b, c, d \in \ZZ$.
Then simple tensors of $\QQ
\otimes_\ZZ (\QQ / \ZZ)$ are of the form
\[ 
\frac ab \otimes_\ZZ \left( \frac cd \mod \ZZ \right) 
= d \frac a{bd} \otimes_\ZZ \left( \frac cd \mod \ZZ \right) 
= \frac a{bd} \otimes_\ZZ \left( d \frac cd \mod \ZZ \right) 
= \frac a{bd} \otimes_\ZZ \left( c \mod \ZZ \right) 
= \frac a{bd} \otimes_\ZZ 0.
\] 
Now, note that $q \otimes 0 = 0$ for all $q \in \QQ$ because
$q \otimes 0 = q \otimes (0+0) = q \otimes 0 + q \otimes 0$.
Therefore, all simple tensors satisfy $\frac ab \otimes \frac cd = 0$, 
and hence $\QQ \otimes_\ZZ (\QQ / \ZZ) = 0$.

\subsection*{2}

In order to show that $\cdot \otimes_R N$ is a functor from the category of
$R$-modules to itself, we need to show that it preserves the identity morphism
and composition of morphisms. 

Let this functor map each $R$-module $M$ to $M \otimes N$ and every $R$-module
homomorphism from $\f : M \to P$ into 
$\f \otimes id: M \otimes N \to P \otimes N$ defined by 
$(\f \otimes id) (m \otimes n) = \f(m) \otimes n$.

Now, 
$(\f \otimes id) (\psi \otimes id) (m \otimes n)
	= (\f \otimes id) (\psi(m) \otimes n) 
		= ((\f \psi (m)) \otimes n)
			= (\f \psi \otimes id)(m \otimes n)$, 
so hence $\cdot \otimes N$ is a covariant functor.

Suppose that 
\[ 
\xymatrix{
M' \ar[r]^{f'} & M \ar[r]^f & M'' \ar[r] & 0 
}
\] 
is an exact sequence of $R$-modules. This means that $f: M \to M''$ is
surjective, and $M'$ surjects onto the kernel of $f$. 

Consider the induced sequence
\[ 
\xymatrix{
M' \otimes_R N \ar[r]^{f' \otimes id} 
	& M \otimes_R N \ar[r]^{f \otimes id} 
		& M'' \otimes_R N \ar[r] 
			& 0 
}
\] 
%We need to show that $f \otimes id$ is surjective and that $f' \otimes id$
%surjects onto the kernel of $f \otimes id$.

\def\im{\operatorname{im}}

Since $f$ is surjective, $f \otimes id$ is clearly also surjective.
Additionally, since $f \circ f' = 0$, we see that 
$(f \otimes 1) \circ (f' \otimes 1) = f \circ f' \otimes 1 = 0$ as well. We
therefore only need to check that 
$\ker (f \otimes 1) = \im (f' \otimes 1)$.
To do this, it is sufficient to show that 
$(M \otimes N) / \im(f' \otimes 1) \to M'' \otimes N$ 
is an isomorphism. 
In one direction, $g: (M \otimes N) / \im(f' \otimes 1) \to M'' \otimes N$
factors naturally from the map $f \otimes id : M \otimes N \to M'' \otimes N$.
This means that we need to construct an inverse map. 

Define $\psi : M '' \times N \to (M \otimes N) / \im(f' \otimes 1)$ via
$(m'', n) \to m \otimes n + \im (f' \otimes 1)$ where $m \in f^{-1} (m'')$.
This map is well-defined because $M' \to M \to M'' \to 0$ is an exact sequence,
so hence $\ker f = \im f'$ and $\im f' \otimes N = \im (f' \otimes 1)$.
In addition, $\psi$ is bilinear, so it extends via the universal property
\[ 
\xymatrix{
M'' \times N \ar[r]^\otimes \ar[rd]_\psi
	& M'' \otimes N \ar[d]^\f \\
& (M \otimes N) / \im (f' \otimes 1).
}
\] 
to a map $\f : M'' \otimes N \to (M \otimes N) / \im(f' \otimes 1)$.
Now, since $\f \circ g = id$, we see that the map $g$ is 1-1, which is what we
needed for the exactness at $M \otimes N$.


\subsection*{3}

Suppose that we are given a map $f \in \Hom_R (M \otimes_R N, P)$, so 
$f : M \otimes N \to P$. 
Using this, define a map $g \in \Hom_R (M, \Hom_R(N, P))$ as a map 
$g: M \to \Hom(N, P)$ defined via $g(m)(n) = f(m \otimes n)$.

Conversely, given a map $g : M \to \Hom(N, P)$, define a map 
$f : M \otimes N \to P$ via $f(m \otimes n) = g(m)(n)$.

Composing these two processes yields the identity, so this gives the desired
bijection
$\Hom(M \otimes N, P) \leftrightarrow \Hom(M, \Hom(N, P))$.


\subsection*{4(a)}

We want to show that $B \otimes_A M$ naturally has the structure of a
$B$-module. 

%Recall that we have bilinear properties of the tensor:
%\begin{align*} 
%(b_1 + b_2) \otimes m &= b_1 \otimes m + b_2 \otimes m \\
%b \otimes (m_1 + m_2) &= b \otimes m_1 + b \otimes m_2 \\
%ab \otimes m &= b \otimes am.
%\end{align*} 
Now, for any element $\sum b_i \otimes m_i \in B \otimes M$, we define the
action
$b \left( \sum b_i \otimes m_i \right) = \sum (b b_i) \otimes m_i$.
We need to show that this action is well-defined, and we do this by showing
that it is independent by the particular representation of an element of $B
\otimes M$ by elementary tensors.

The tensor product can be defined as factoring by expressions of the form
\begin{align*} 
(b_1 + b_2) \otimes m &- b_1 \otimes m - b_2 \otimes m \\
b \otimes (m_1 + m_2) &- b \otimes m_1 - b \otimes m_2 \\
ab \otimes m &- b \otimes am.
\end{align*} 
Let these elements generate $H$.
Note that for $b' \in B$, this means that
\begin{align*} 
b'(b_1 + b_2) \otimes m &- b'b_1 \otimes m - b'b_2 \otimes m \\
b'b \otimes (m_1 + m_2) &- b'b \otimes m_1 - b'b \otimes m_2 \\
ab'b \otimes m &- b'b \otimes am.
\end{align*} 
are also elements of $H$.

Suppose that 
\[ 
\sum_i b_i \otimes m_i = \sum_j b'_j \otimes m'_j
\] 
are two different representations for the same element of $B \otimes M$.
Bilinearity implies that $\sum b_i \otimes m_i - \sum b'_j \otimes m'_j$ 
is also in $H$, and hence
\[ 
\sum_i bb_i \otimes m_i = \sum bb'_j \otimes m'_j
\] 
is also in $H$, which means that our $B$-action on $B \otimes M$ is indeed
well-defined.

It is now easy to check that this action makes $B \otimes M$ into a $B$-module.
Indeed, 
\[ 
(b + b')(b_i \otimes m_i) 
= ((b + b')b_i) \otimes m_i
= (b b_i + b' b_i) \otimes m_i
= b (b_i \otimes m_i) + b' (b_i \otimes m_i).
\] 
Analogously, we have the other module properties
$b (b_i \otimes m_i + b_j \otimes m_j) 
= b (b_i \otimes m_i) + b (b_j \otimes m_j)$ and 
$(b_1 b_2) (b_i \otimes m_i) = b_1 (b_2 b_i \otimes m_i)$, 
so $B \otimes M$ does indeed have the structure of a $B$-module.

We therefore have a map from $A$-modules $M$ to $B$-modules $B \otimes_A M$, 
and we want to show that this defines a functor from the category
of $A$-modules to the category of $B$-modules.

Suppose that $g$ and $f$ are morphisms of the category of $A$-modules. Then 
under this map, they map to $id \otimes g$ and $id \otimes f$, so that
their composition $gf$ maps to 
$id \otimes gf = (id \otimes g) (id \otimes f)$, which shows that this map is a
functor.


\subsection*{4(b)}

In this case, as we showed above, $B \otimes C$ is a $B$-module, so that
$$
(b_1 \otimes c_1)(b_2 \otimes c_2) 
= b_1 b_2 (id \otimes c_1) (id \otimes c_2)
= b_1 b_2 (id \otimes (c_1 c_2)) = (b_1 b_2 \otimes c_1 c_2),
$$
where the second equality is true because part (a) produced a functor taking 
$C$ to $B \otimes C$. Such a functor takes $c_1$ and $c_2$ to $id \otimes c_1$
and $id \otimes c_2$ respectively, 
and it takes $c_1 c_2$ to $id \otimes (c_1 c_2)$.

Therefore, $B \otimes C$ has a multiplicative operation, and the
additive operation simply follows from bilinearity. Hence, it has the 
structure of a ring.


\subsection*{5}

Define $\f: S^{-1} R \times M \to S^{-1}M$ via the mapping 
$(r/s, m) \mapsto rm/s$. This clearly a bilinear mapping, so we can apply the
universal property 
\[ 
\xymatrix{
S^{-1}R \times M \ar[r]^\otimes \ar[rd]_\f
	& S^{-1}R \otimes M \ar[d]^f \\
& S^{-1}M.
}
\] 
Therefore, the map $\f$ induces a map $f: S^{-1}R \otimes M \to S^{-1}M$.
In addition, the inverse map that sends $m/s \mapsto 1/s \otimes m$
gives  a well defined inverse homomorphism.
This is because if $m_1/s_1 = m_2/s_2$ then $s (s_2 m_1 - s_1 m_2) = 0$, which
means that 
$1 / s_1 \otimes m_1 
= 1 / (s s_2 s_1) \otimes s s_2 m_1  
= 1 / (s s_2 s_1) \otimes s s_1 m_2 = 1/s \otimes m_2$.

Hence, $\f$ and its inverse are both well-defined homomorphisms, and therefore
we have the desired isomorphism $S^{-1}R \otimes M \cong S^{-1}M$.

\subsection*{6}

We want to show that $\CC \otimes_\RR \CC \cong \CC \times \CC$.

Since each copy of $\CC$ has basis $1$ and $i$, we see that by 
problem 7, 
the tensor product is free of rank 4 with basis $1 \otimes 1$, $1 \otimes i$,
$i \otimes 1$, and $i \otimes i$. 
Since $\CC \times \CC$ 
has the same structure, we see that $\CC \otimes_R \CC \cong \CC \times \CC$.


\subsection*{7(a)}
We want to describe an isomorphism 
$M \otimes_R (N \oplus N') \cong (M \otimes_R N) \oplus (M \otimes_R N')$.
Define a map 
$\f : M \times (N \oplus N') \to (M \otimes N) \oplus (M \otimes N')$ via
$m \otimes (n, n') \mapsto (m \otimes n, m \otimes n')$.
This map is clearly bilinear, so the universal property 
\[ 
\xymatrix{
M \times (N \oplus N') \ar[r]^\otimes \ar[rd]_\f
	& M \otimes (N \oplus N') \ar[d]^f \\
& (M \otimes N) \oplus (M \otimes N').
}
\] 
induces a map
$f : M \otimes (N \oplus N') \to (M \otimes N) \oplus (M \otimes N')$.

We can define the inverse map
$(M \otimes N) \oplus (M \otimes N') \to M \otimes (N \oplus N')$ via 
the mapping
$(m \otimes n, m \otimes n') \mapsto m \otimes (n, n')$. We see that
both $f$ and its inverse are well-defined homomorphisms, so 
we therefore have the isomorphism
$M \otimes (N \oplus N') \cong (M \otimes N) \oplus (M \otimes N')$.

\subsection*{7(b)}
First, note that $R \otimes_R R \cong R$.
This is because every element of $R \otimes_R R$ can be written as
$a \otimes b = ab \otimes 1$ by bilinearity.
Hence, the mapping $R \otimes_R R \to R$ via $ab \otimes 1 \mapsto ab$ is
well-defined and invertible, so it is an isomorphism $R \otimes_R R \cong R$. 

Now, consider $R^m \otimes_R R^n$. Each of $R^m$ and $R^n$ are a direct sum of
copies of $R$, so we can apply part (a) to decompose $R^m \otimes_R R^n$ into
the direct sum of $mn$ copies of $R \otimes_R R$, which is isomorphic to the
direct sum of $mn$ copies of $R$:
\[ 
R^n \otimes_R R^n \cong (R \otimes_R R)^{mn} \cong  R^{mn}, 
\] 
as desired.

\end{document}
the mapping $R \otimes_R R \to R$ via $ab \otimes 1 \mapsto a