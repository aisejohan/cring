\documentclass{article}
\usepackage{amsmath}
\usepackage{amssymb}
\usepackage{amsthm}
\usepackage{array}
\usepackage{amsfonts}
\newtheoremstyle{norm}
{3pt}
{3pt}
{}
{}
{\bf}
{:}
{.5em}
{}

\theoremstyle{norm}
\newtheorem{theorem}{Theorem}[section]
\newtheorem{lemma}[theorem]{Lemma}
\newtheorem{definition}[theorem]{Definition}
\newtheorem{corollary}[theorem]{Corollary}

\begin{document}
\section{Euclidean domains, PIDs, and UFDs}
\begin{definition}
An integral domain $R$ is an \textbf{Euclidean domain} if there is a function $|\cdot |:R\to \mathbb N_0$ (called the norm) such that the following hold.
\begin{enumerate}
\item $|a|=0$ iff $a=0$.
\item For any nonzero $a,b\in R$ there exist $q,r\in R$ such that $b=aq+r$ and $|r|<|a|$.
\end{enumerate}
In other words, the norm is compatible with division with remainder.
\end{definition}
\begin{theorem}
An Euclidean domain is a principal ideal domain.
\end{theorem}
\begin{proof}
Let $R$ be an Euclidean domain, $I\subseteq R$ and ideal, and $b$ be the nonzero element of smallest norm in $I$.
Suppose $ a\in I$. Then we can write $ a = qb + r$ with $ 0\leq r < |b|$, but since $ b$ has minimal nonzero absolute value, $ r = 0$ and $ b|a$. Thus $ I=(b)$ is principal.
\end{proof}

\begin{theorem}
A PID is a UFD.
\end{theorem}
\begin{proof}
Let $R$ be a PID. Since every ideal in $R$ is principal, for every $a,b$ (not both 0) we have $(a)+(b)=(d)$
for some $d\in R$. This says there exist $ s,t\in R$ such that
\[as + bt = d.\]
From this, any divisor of $a,b$ must divide $ d$. Furthermore, $d$ must divide $a,b$ since $a,b\in (d)$. In other words, $d$ is the greatest common divisor of $ a,b$. The existence of $s, t$ as above  is known as B\'ezout's Theorem.

Now suppose $ p$ is irreducible; we show $ p$ is prime. Suppose $ p|ab$ but $ p$ does not divide $ a$. Then using Bezout's Theorem and the fact that $ a$ and $p$ are relatively prime, we get $ as + pt =1$ for some $ s,t$. Multiply by $ b$ to get
\[ abs + ptb = b.\]
Since $ p|ab|abs, p|ptb$, we have $ p|b$. This shows that irreducible elements are prime in $ \mathbb{Z}$.

It remains to show factoring terminates. Otherwise, there would be an infinite sequence of nonassociated elements $a_1,a_2,\ldots \in R$ such that $a_{i+1}|a_i$. Then $(a_1)\subset (a_2)\subset\cdots$. However, $\bigcup_{i\geq 1} (a_i)$ is an ideal, so it is principal, say generated by $b$. Then $b\in (a_i)$ for some $i$; this implies that $(b)=(a_i)$. Hence $(a_i)=(a_{i+1})=\cdots$, a contradiction.

Since irreducible elements are prime and every nonzero element of $R$ factors into irreducibles, $R$ is a UFD.
\end{proof}

\begin{corollary}
$\mathbb Z$ is a UFD.
\end{corollary}


\section{Gauss's Lemma}
Let $R$ be a UFD and let $K$ be the field of fractions of $R$. A nonzero polynomial $f\in R[x]$ is said to be \textbf{primitive} if all its coefficients have a common proper divisor; equivalently, there does not exist a prime $p\in R$ such that $p|f$.

\begin{lemma}\label{polyintdom}
If $\mathcal{R}$ is an integral domain, then so is $\mathcal{R}[x]$.
\end{lemma}
\begin{proof}
Take any $p,q\in \mathcal{R}[x]$ not equal to 0. We can write
\begin{eqnarray*}
p=\sum_{i=0}^m a_ix^i,\,a_m\neq 0\\
q=\sum_{i=0}^n b_ix^i,\, b_n\neq 0\\
\end{eqnarray*}
Then the leading coefficient of $pq$ is $a_mb_nx^{m+n}$. It is nonzero because since $\mathcal{R}$ is an integral domain, $a_m,b_n\neq 0$ imply that $a_mb_n\neq 0$. Hence $pq\neq 0$. This shows that $\mathcal{R}[x]$ is an integral domain.
\end{proof}

\begin{lemma}[Gauss's lemma]\label{gausslemma}
(A) An element of $R$ is prime in $R[x]$ iff it is a prime in $R$. Hence if a prime $p$ of $R$ divides a product $fg$ of polynomials in $R[x]$, then $p|f$ or $p|g$.\\
(B) The product of primitive polynomials in $R[x]$ is primitive.
\end{lemma}
\begin{proof}
If $p\in R$ is nonzero any factors of $p$ in $R[x]$ must be in $R$, so any proper factorization of $p$ in $R[x]$ is a proper factorization in $R$. Hence a prime in $R$ is prime in $R[x]$.

Let $p$ be any prime element in $R$. 
Then $R/(p)$ is an integral domain so by lemma~\ref{polyintdom}, $R/(p)[x]$ is an integral domain.

Suppose $p\in R$ is prime in $R[x]$, and suppose $p|fg$ for $f,g$. 
Then in $R/(p)[x]$, $\overline{fg}=\overline{f}\overline{g}=0$. 
Since $R/(p)[x]$ is an integral domain, either $\overline{f}=0$ or $\overline{g}=0$. 
In other words, either $p|f$ or $p|g$ in $R[x]$. Thus $p$ is a prime in $R[x]$.

If $f,g$ are primitive, then $p\not|  f$ and $p\not| g$ for all primes $p\in R$. Since $p$ is also prime in $R[x]$, $p\not| fg$. Hence $fg$ is not divisible by any prime in $R$, and it is primitive.
\end{proof}
\begin{lemma}\label{consttimesprimitive}
Every nonconstant polynomial $f\in K[x]$ can be written uniquely (up to multiplication by units) in the form $f=cf_0$, where $c\in K$ and $f_0$ is a primitive polynomial in $R[x]$.
\end{lemma}
\begin{proof}
Each coefficient $a_i$ of $f$ is in the form $\frac{p_i}{q_i}$, where $p_i,q_i\in R$. We can find a nonzero $t\in R$ such that $t$ is divisible by each denominator (for, example, take $t$ to be the product of the denominators). Then we can write
\[tf=f_1,\]
where $f_1\in R[x]$. Let $s\in R$ be a greatest common divisor of the coefficients of $f_1$. Then we have
\[f=\frac{s}{t}f_0\]
in $K[x]$ where $f_0\in R[x]$ and the coefficients of $f_0$ have no common divisor. This gives the desired representation.

Next we check uniqueness. Suppose
\[f=cf_0=c'f'_0,\]
where $c,c'\in K$ and $f_0,f'_0\in R[x]$ are primitive. Multiply by an element of $R$ to ``clear denominators," to reduce to the case where $c,c'\in R$. Now take any prime $p|c$. Since $p$ is prime in $R[x]$, $p|c'$ or $p|f'_0$. The second is impossible since $f'_0$ is primitive. Hence $p|c'$, and we can cancel $p$. Continuing in this way, we get that $c$ and $c'$ share the same prime factors with the same multiplicities. Hence $c,c'$ are associates.
\end{proof}
\begin{lemma}
%(A) 
Let $f_0$ be a primitive polynomial and let $g\in R[x]$. If $f_0|g$ in $K[x]$ then $f_0|g$ in $R[x]$.\\
%\noindent(B) If $f,g\in K[x]$ have a common factor in $K$, then they have a common factor in $R$.
%(B) isn't necessary.
\end{lemma}
\begin{proof}
If $f_0|g$ in $K[x]$, then we can write $g=f_0h$ where $h\in K[x]$. 
We need to show $h\in R[x]$. By lemma~\ref{consttimesprimitive}, we can write $h=ch_0$, where $c\in K$ and $h_0$ is primitive.  
Then $g=cf_0h_0$. By lemma~\ref{gausslemma}, the product $f_0h_0$ of primitive polynomials is primitive. 
We can write $c=\frac{s}{t}$, where $s,t\in R$ have no common factors. 
If a prime $p$ in $R$ divides the denominator $t$ then 
$p\not|  s$ so $p|f_0h_0$, contradicting the fact that $f_0h_0$ is primitive. 
Hence $t$ is a unit, and $c\in R$. Then $h=ch_0\in R[x]$, so $f_0|g$ in $R[x]$.\\
%\noindent(B) Suppose $h$ is a common factor of $f,g$ in $K[x]$. 
%Write $h=ch_0$, where $h_0$ is primitive. Then $h_0$ divides both $f$ and $g$ in $K[x]$, so by part (A), $h_0$ divides both $f$ and $g$ in $R[x]$.
\end{proof}
\begin{lemma}
Let $f$ be a nonzero element of $R[x]$. Then $f$ is an irreducible element of $R[x]$ iff it is an irreducible element of $R$ or a primitive irreducible polynomial in $K[x]$.
\end{lemma}
\begin{proof}
We showed in lemma~\ref{gausslemma} that if $f\in R$, then $f$ is prime (irreducible) in $R$ iff it is prime in $R[x]$. 
An element $f$ in $R$ is prime in $R[x]$ iff it is irreducible in $R[x]$: just note that the only factors $f$ has in $R[x]$ are those in $R$, so we just use the fact that $R$ is a UFD. This proves the lemma for $f\in R$. Now suppose $f\not\in R$. 

If $f$ is a primitive irreducible polynomial in $K[x]$, then it is irreducible in $R[x]$.

Note any irreducible polynomial in $R[x]$ must be primitive. Suppose $f$ is a primitive polynomial, and $f=gh$ is a proper factorization of $f$ in $K[x]$. We can write $g=cg_0,h=c'h_0$ where $c,c'\in K$ and $g_0,h_0$ are primitive. Since $g_0$ and $h_0$ are both primitive, so is $g_0h_0$. But $f=cc'(g_0h_0)$ is primitive, so by uniqueness in lemma~\ref{consttimesprimitive}, $cc'$ must be a unit. Thus $f=(cc')g_0h_0$ is a proper factorization of $f$ in $R[x]$ as well. This shows that a primitive polynomial in $R[x]$ that is reducible in $K[x]$ is reducible in $R[x]$, i.e. an irreducible polynomial in $R[x]$ is irreducible in $K[x]$.
\end{proof}
\begin{theorem}
The ring $R[x]$ is a unique factorization domain.
\end{theorem}
\begin{proof}
It suffices to show that every irreducible element of $R[x]$ is a prime element, and that factoring terminates. By lemma~\ref{gausslemma}, every irreducible element of $R[x]$ in $R$ is prime in $R$ and hence prime in $R[x]$.

Suppose $f\not\in R$ is irreducible in $R[x]$ and that $f|gh$ where $g,h\in R[x]$. Since $K$ is a field, $K[x]$ is a UFD. By lemma 1.5, $f$ is irreducible in $K[x]$. Thus $f$ is a prime in $K[x]$. Hence $f|g$ or $f|h$ in $K[x]$. By lemma 1.4, $f|g$ or $f|h$ in $R[x]$. This shows $f$ is prime.

A polynomial $f\in R[x]$ can only be the product of at most $\deg(f)$ many polynomials $p_i$ of positive degree in $R[x]$ because the sum of the degrees of the $p_i$ must equal $\deg(f)$. Factor terminates for the factors of $f$ in $R$ because factoring terminates in the UFD $R$, and the primes in $R$ dividing $f$ are the primes dividing every coefficient of $f$.

Hence $R[x]$ is a UFD.
\end{proof}
\end{document}