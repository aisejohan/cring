\documentclass[11pt]{article}
\usepackage{amsmath}
\usepackage{amssymb}
\usepackage{amsfonts}
\usepackage{amsthm}
\usepackage{array}
\usepackage[pdftex]{graphicx}
\input xy
\xyoption{all}
%% This command inserts \noindent and makes the input bold.
\newcommand{\num}[1]{\noindent \textbf{#1}}
%% This custom type produces a column of the specified width whose contents
are centered.
\newcolumntype{C}[1]{>{\centering\hspace{0pt}}p{#1}}
\newtheorem{theorem}{Theorem}
\newtheorem{lemma}{Lemma}[theorem]
\theoremstyle{definition}
\newtheorem{axiom}{Axiom}[section]
\newtheorem{definition}{Definition}
\newtheorem*{remark}{Remark}
\textwidth 6.9in
\textheight 9.4in
\oddsidemargin -0.2in
\topmargin -0.7in
\pagestyle{empty}
\begin{document}
\noindent Zev Chonoles \hfill \today\\[-0.4in]
\begin{center}
\noindent \underline{MATH 2520 - Assignment 1}
\end{center}

\num{Assigned in class.} For all $a\in I$, we have $a^1\in I$, so that
$ I\subseteq r(I)$. Thus $r(I)\subseteq r(r(I))$. If $x\in r(r(I))$,
there is an $n>0$ with $x^n\in r(I)$, so that there is an $m>0$ with
$(x^n)^m=x^{nm}\in I$. Thus $x\in r(I)$, so that $r(r(I))\subseteq r(I)$,
so that $r(r(I))=r(I)$.\\

We clearly have $r(IJ)\subseteq r(I\cap J)$ because $IJ\subseteq I\cap
J$. If $x\in r(I\cap J)$, there is an $n>0$ with $x^n\in I\cap J$, so that
$x^n\in I$ and $x^n\in J$, so that $x\in r(I)$ and $x\in r(J)$, so that
$x\in r(I)\cap r(J)$. Thus, $r(I\cap J)\subseteq r(I)\cap r(J)$. If $x\in
r(I)\cap r(J)$, there are $n,m>0$ such that $x^n\in I$ and $x^m\in J$, so that
$x^{m+n}=x^mx^n\in IJ$, so that $x\in r(IJ)$. Thus $r(I)\cap r(J)\subseteq
r(IJ)$, and combining inclusions we have $r(I\cap J)=r(IJ)=r(I)\cap r(J)$.\\

Because $I\subseteq r(I)\subseteq R = (1)$, we have that $I=(1)$ implies
$r(I)=(1)$. Conversely, if $r(I)=(1)$, then $1\in r(I)$, so there is an
$n>0$ with $1^n\in I$, but $1^n=1$ for all $n$, so that $1\in I$ and hence
$I=(1)$. Thus $r(I)=(1)$ iff $I=(1)$.\\

If $x\in r(P^n)$, there is an $m>0$ with $x^m\in P^n\subseteq P$, so that
$x\in P$ because $P$ is prime. Conversely, if $x\in P$, then $x^n\in P^n$ for
all $n$, so that $x\in r(P^n)$ for all $n$, so that $r(P^n)=P$ for all $n$.\\

\num{Problem 2.} Let $f=\sum_{i=0}^m a_ix^i$ and $g=\sum_{i=0}^n b_ix^i$ be
elements of $R[x]$. WLOG, we have $n\leq m$, and define $b_{n+1},\ldots,b_m=0$,
so that $g=\sum_{i=0}^m b_ix^i$, and thus $f+g=\sum_{i=0}^m
(a_i+b_i)x^i$. Then $\tilde{\phi}(f+g)=\tilde{\phi}(\sum_{i=0}^m
(a_i+b_i)x^i)=\sum_{i=0}^m\phi(a_i+b_i)x^i=\sum_{i=0}^m(\phi(a_i)+\phi(b_i))x^i=\sum_{i=0}^m\phi(a_i)x^i+\sum_{i=0}^m\phi(b_i)x^i=\tilde{\phi}(f)+\tilde{\phi}(g)$.
As for multiplication, we have $\tilde{\phi}(ax^m\cdot
bx^n)=\tilde{\phi}(abx^{m+n})=\phi(ab)x^{m+n}=\phi(a)\phi(b)x^{m+n}=\tilde{\phi}(ax^m)\tilde{\phi}(bx^n)$,
and checking that $\tilde{\phi}$ works for the multiplication of monomials
suffices to prove it for all elements of $R[x]$ by distributivity and the
above proof of $\tilde{\phi}$ being homomorphism over addition. \\

If $a_0u=1$ and $a_1,\ldots,a_n$ are nilpotent, then $-ua_1x,\ldots,-ua_nx^n$
are also nilpotent, and so $g=-u(a_1x+\cdots+a_nx^n)$ is nilpotent because
the sum of nilpotent elements is nilpotent. Thus, by Problem 1, $1-g$ is
a unit, and $\frac{u}{1-g}=\frac{1}{f}$, so $f$ is a unit. Unfortunately,
I couldn't figure out the other direction.    \\

Also, $f\in R[[x]]$ is a unit iff $a_0$ is a unit; one direction is obvious,
the other direction is proved by observing that $f^{-1}$ has coefficients
$b_0=a_0^{-1}$ and $b_n=-a_0^{-1}\sum_{i=1}^n a_ib_{n-i}$.    \\

\pagebreak

\num{Problem 5.} If $P\subset R$ is a prime ideal, then $R/P$
is an integral domain. For each $a\in R$, there is an $n>1$ such
that $a^n=a$. Thus, for each $\overline{a}\in R/P$, there is an
$n>1$ such that $\overline{a}^n=\overline{a^n}=\overline{a}$, so that
$\overline{a}(\overline{a}^{n-1}-1)=0$ in $R/P$. Because $P$ is prime, $R/P$
is an integral domain, so either $\overline{a}=0$ or $\overline{a}^{n-1}=1$,
so that either $\overline{a}=0$ or $\overline{a}$ is a unit. Thus $R/P$
is a field, and thus $P$ is maximal. \\

\num{Problem 6.} Clearly $(0)\in\Sigma$, so $\Sigma$ is non-empty. Order
$\Sigma$ by inclusion. If $\{C_\alpha\}\subset\Sigma$ is a chain, then
$\bigcup C_\alpha$ is an ideal (by the standard argument), which is comprised
entirely of 0 and zero-divisors (because each $C_\alpha$ is), and contains
each $C_\alpha$ (obviously). Thus $\bigcup C_\alpha$ is an upper bound for
$\{C_\alpha\}$ in $\Sigma$, and thus Zorn's Lemma applies to $\Sigma$, so
that $\Sigma$ has maximal elements. If $I\in\Sigma$ is a maximal element, and
$r,s\notin I$, then $I\subsetneq I+(r), I+(s)$, and because $I$ is maximal
in $\Sigma$, the ideals $I+(r)$ and $I+(s)$ cannot be in $\Sigma$ - that
is, they each must have at least one non-zero-divisor. Suppose $a\in I+(r)$
and $b\in I+(s)$ are not zero-divisors. Then $ab\in (I+(r))(I+(s))\subseteq
I+(rs)$ is not a zero-divisor, so that $I+(rs)\notin\Sigma$, so that $rs\notin
I$. Thus any maximal element $I\in\Sigma$ is prime, and because any element of
$\Sigma$ is contained in some maximal element of $\Sigma$, we must have that
the set consisting of 0 and all zero-divisors is a union of prime ideals.   \\

\num{Problem 7a.}  We claim that $Z:I(\mathcal{X})\rightarrow Z(\mathcal{J})$
is a bijection, with inverse $I:Z(\mathcal{J})\rightarrow I(\mathcal{X})$
(these are restrictions of the original maps to the specified subsets, but
no confusion will arise). Any $i\in I(\mathcal{X})$ has $i=I(x)$ for some
$x\in\mathcal{X}$. We have that $I(Z(i))\geq i$ because $IZ$ is increasing, and
$Z(I(x))\geq x$ because $ZI$ is increasing. But then $I(Z(i))= I(Z(I(x)))\leq
I(x) = i$ because $I$ is order reversing, so that $i\leq I(Z(i))\leq i$,
so that $I(Z(i))=i$ by the axioms of a partially ordered set. Thus $I\circ
Z = \text{id}_{I(\mathcal{X})}$. By the same argument, but with the roles
of $\mathcal{X}$ and $\mathcal{J}$, and $Z$ and $I$, reversed, we find that
$Z\circ I = \text{id}_{Z(\mathcal{J})}$, and thus $Z$ is a bijection with $I$
as its inverse.\\

\num{Problem 7b.} Order both $\mathcal{X}$, the power set of $k^n$, and
$\mathcal{J}$, the power set of $k[x_1,\ldots,x_n]$, by inclusion. For
$J\in\mathcal{J}$, we define $Z(J)$ to be the set of points in $k^n$ at which
every element of $J$ vanishes (which by definition is an algebraic set),
and for $X\in\mathcal{X}$, we define $I(X)$ to be the set of all polynomials
in $k[x_1,\ldots,x_n]$ which vanish on $X$ (which is clearly an ideal, and
by definition is a formally radical ideal). Thus, $Z(\mathcal{J})$ are the
algebraic sets, and $I(\mathcal{X})$ are the formally radical ideals. \\

Let $X_1,X_2\in\mathcal{X}$ and $J_1,J_2\in\mathcal{J}$. If $X_1\leq X_2$
(i.e., $X_1\subseteq X_2$), then any polynomial in $k[x_1,\ldots,x_n]$ which
vanishes on $X_2$ also vanishes on $X_1$, so that $I(X_2)\leq I(X_1)$ (i.e.,
$I(X_2)\subseteq I(X_1)$). If $J_1\leq J_2$ (i.e., $J_1\subseteq J_2$), then
for any point at which all of $J_2$ vanishes, all of $J_1$ vanishes at it as
well, so that $Z(J_2)\leq Z(J_1)$ (i.e. $Z(J_2)\subseteq Z(J_1)$). Thus $Z$
and $I$ are order reversing. Finally, note that $Z(I(X_1))$ is the set of
points which at which the polynomials that vanish on $X_1$, vanish - obviously,
$Z(I(X_1))\geq X_1$ (i.e. $X_1\subseteq Z(I(X_1))$. Similarly, $I(Z(J_1))$
is the set of polynomials which vanish at the points at which $J_1$ vanishes
- obviously, $I(Z(J_1))\geq J_1$ (i.e. $J_1\subseteq I(Z(J_1))$. Thus, $ZI$
and $IZ$ are increasing. \\

By Problem 7a, this means that $Z(\mathcal{J})$ and $I(\mathcal{X})$ are in
bijection.\\

\num{Problem 8.} Let $X$ be an algebraic set, say $X=Z(J)$. If $I(X)=J$ is
not prime then there are $f,g\notin J$ with $fg\in J$. Thus $J\subsetneq
J+(f), J+(g)$, so that $X=Z(J)\supsetneq Z(J+(f)),Z(J+(g))$, by the fact
that $Z$ is order reversing. But because $(J+(f))(J+(g))\subseteq J$, we
have $Z((J+(f))(J+(g)))=Z(J+(f))\cup Z(J+(g))\supseteq Z(J)=X$; thus in fact
$Z(J+(f))\cup Z(J+(g)) =Z(J)=X$ (because they are both subsets of $X$). This
also suffices to show that $Z(J+(f)),Z(J+(g))\neq\emptyset$. Thus if $I(X)$
is not prime, then $X$ is reducible.\\

Conversely, if $X$ is reducible, we have $X=A\cup B$ for algebraic sets
$A,B\neq\emptyset$. In particular, $A\not\subseteq B$ and $B\not\subseteq
A$, and thus $I(A)\not\supseteq I(B)$ and $I(B)\not\supseteq I(A)$, so that
there is an $f\in I(A)$ with $f\notin I(B)$ and a $g\in I(B)$ with $g\notin
I(A)$. But $fg\in I(A)\cap I(B) = I(A\cup B)=I(X)$, so $I(X)$ is not prime.









\end{document}
o show that $Z(J+(f)),Z(J+(g))\neq\emptyset$. Thus if $I(X)$ is not prime,
then $X$ is reducible.\\

Conversely, if $X$ is reducible, we have $X=A\cup B$ for algebraic sets
$A,B\neq\emptyset$. In particular, $A\not\subseteq B$ and $B\not\subseteq A$,
and thus $I(A)\not\supseteq I(B)$ and $I(B)\not\supseteq I(A)$, so that ther
