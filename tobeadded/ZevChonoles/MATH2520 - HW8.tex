\documentclass[11pt]{article}
\input{C:/Users/Public/stuff.tex}
\textwidth 7in
\textheight 9.5in
\oddsidemargin -0.25in
\topmargin -0.75in
\setlength\parindent{0pt}
\pagestyle{empty}
\begin{document}
Zev Chonoles \hfill 
\underline{MATH 2520 - Assignment 8} \hfill \today\\

\num{1.} (a $\Rightarrow$ b) By definition, if $P$ is projective, then for any epimorphism $\alpha\in\Hom(M,N)$, it is the case that for every $\beta\in\Hom(P,N)$, there is some $\gamma\in\Hom(P,M)$ such that $\beta=\alpha\circ\gamma$. Thus the induced map $\alpha_*:\Hom(P,M)\rightarrow\Hom(P,N)$ defined by $\alpha_*(\gamma)=\alpha\circ\gamma$ is surjective. We know that a homomorphism of modules is an epimorphism iff it is surjective, so that the induced map $\alpha_*$ is an epimorphism.\\

(b $\Rightarrow$ a) Given an epimorphism $\alpha\in\Hom(M,N)$, suppose the induced map $\alpha_*:\Hom(P,M)\rightarrow\Hom(P,N)$ defined by $\alpha_*(\gamma)=\alpha\circ\gamma$ is an epimorphism. We know that a homomorphism of modules is an epimorphism iff it is surjective, so that the induced map $\alpha_*$ is surjective, i.e. every $\beta\in\Hom(P,N)$ has $\beta=\alpha\circ\gamma$ for some $\gamma\in\Hom(P,M)$. Thus, by definition, $P$ is projective.\\

(b $\Rightarrow$ c) If, for every module $M$ and epimorphism $\alpha\in\Hom(M,N)$, the induced map $\alpha_*:\Hom(P,M)\rightarrow\Hom(P,N)$ defined by $\alpha_*(\gamma)=\alpha\circ\gamma$ is an epimorphism, then in particular, this is true when we fix $M=F$, for some free module $F$, and fix some epimorphism $\alpha\in\Hom(F,P)$. Thus, for some free module $F$ and epimorphism $\alpha\in\Hom(F,P)$, the induced map $\alpha_*:\Hom(P,F)\rightarrow\Hom(P,P)$ is an epimorphism.\\

(c $\Rightarrow$ d) Suppose that, for some free module $F$ and epimorphism $\alpha\in\Hom(F,P)$, the induced map $\alpha_*:\Hom(P,F)\rightarrow\Hom(P,P)$ is an epimorphism. We know that a homomorphism of modules is an epimorphism iff it is surjective, so that every $\beta\in\Hom(P,P)$ has $\beta=\alpha\circ\gamma$ for some $\gamma\in\Hom(P,F)$. Thus, in particular, the identity map $\id_P\in\Hom(P,P)$ has $\id_P=\alpha\circ\gamma$ for some $\gamma\in\Hom(P,F)$. Because $\alpha$ splits, we have $F=\ker(\alpha)\oplus\im(\gamma)$ and hence $F\simeq\ker(\alpha)\oplus M$ (we proved this on a previous homework, but I have included the proof below).\\

Proposition (that splits $\Rightarrow$ direct sum): Suppose $\alpha:M\rightarrow P$ is split by $\gamma:P\rightarrow M$, so that $\id_P=\alpha\circ\gamma$. Note that because $\id_P$ is a monomorphism, $\gamma$ must be a monomorphism, and hence an isomorphism onto its image, so that $P\simeq\im(\gamma)$. For any $m\in M$, we have $m=(m-\gamma(\alpha(m)))+\gamma(\alpha(m))$. Obviously, $\gamma(\alpha(m))\in\im(\gamma)$, and because $\alpha(m-\gamma(\alpha(m)))=\alpha(m)-\alpha(\gamma(\alpha(m)))=\alpha(m)-\id_P(\alpha(m))=\alpha(m)-\alpha(m)=0$, we have $m-\gamma(\alpha(m))\in\ker(\alpha)$. Finally, note that $\ker(\alpha)\cap\im(\gamma)=\{0\}$, because if $m=\gamma(p)$ for some $p\in P$ and $\alpha(m)=0$, then $\alpha(m)=\alpha(\gamma(p))=\id_P(p)=p=0$, so $m=\gamma(0)=0$. Thus, $M=\ker(\alpha)\oplus\im(\gamma)$, and thus $M\simeq\ker(\alpha)\oplus P$. \\

(d $\Rightarrow$ b) Suppose that $P\oplus Q\simeq F$ for some free module $F$. We know that free modules are projective, so that for any epimorphism $\alpha\in\Hom(M,N)$, the induced map $\widehat{\alpha_*}:\Hom(F,M)\rightarrow\Hom(F,N)$ is an epimorphism. We want to show that the induced map $\alpha_*:\Hom(P,M)\rightarrow\Hom(P,N)$ is an epimorphism. Because the direct sum is the coproduct in the category of $R$-modules, we have $\Hom(P\oplus Q,T)\simeq\Hom(P,T)\oplus\Hom(Q,T)$ for any $R$-module $T$ (as is mentioned in the problem), with $g\in\Hom(P\oplus Q,T)$ corresponding to $(g\circ i_1,g\circ i_2)\in\Hom(P,T)\oplus\Hom(Q,T)$ where $i_1:P\rightarrow P\oplus Q$ and $i_2:Q\rightarrow P\oplus Q$ are the associated maps for the coproduct. Furthermore, we have the projection maps $\pi_M:\Hom(P,M)\oplus\Hom(Q,M)\rightarrow\Hom(P,M)$ and $\pi_N:\Hom(P,N)\oplus\Hom(Q,N)\rightarrow\Hom(P,N)$, which are epimorphisms because they are surjective. Because $(\alpha_*\circ\pi_M)(g\circ i_1,g\circ i_2)=\alpha_*(g\circ i_1)=\alpha\circ g\circ i_1$ and $(\pi_N\circ\widehat{\alpha_*})(g\circ i_1,g\circ i_2)=\pi_N(\alpha\circ g\circ i_1,\alpha\circ g\circ i_1)=\alpha_\circ g\circ i_1$, we have that the diagram \\[-0.35in]
\begin{center}
$\xymatrix{
\Hom(P\oplus Q,M) \ar[r]^{\widehat{\alpha_*}} \ar[d]_{\pi_M} & \Hom(P\oplus Q, N) \ar[d]^{\pi_N}\\
\Hom(P,M) \ar[r]_{\alpha_*} & \Hom(P,N)}$
\end{center}
commutes (we have identified $\Hom(P\oplus Q,M)$ and $\Hom(P\oplus Q,N)$ with $\Hom(P,M)\oplus\Hom(Q,M)$ and $\Hom(P,N)\oplus\Hom(Q,N)$, respectively). Because $\pi_N$ and $\widehat{\alpha_*}$ are epimorphisms, $\pi_N\circ\widehat{\alpha_*}=\alpha_*\circ\pi_M$ is an epimorphism, so that $\alpha_*$ must be an epimorphism.\\

(a $\Rightarrow$ e) If $P$ is projective, then for any epimorphism $\alpha\in\Hom(M,N)$ and any map $\beta\in\Hom(P,N)$, there is a $\gamma\in\Hom(P,M)$ such that $\beta=\alpha\circ\gamma$. In particular, if $N=P$ and $\beta=\id_P\in\Hom(P,P)$, there is a $\gamma\in\Hom(P,M)$ such that $\id_P=\alpha\circ\gamma$. Thus every epimorphism $\alpha\in\Hom(M,P)$ splits.\\

(e $\Rightarrow$ d) Any module $P$ is the quotient of some free module - for example, the free module $F$ on the elements of $P$, with the epimorphism $\alpha:F\rightarrow P$ defined by $f(x_p)= p$. Thus, any module $P$ has some epimorphism $\alpha$ to it from a free module $F$. By the assumption that every epimorphism to $P$ splits, $P$ is a direct summand of the free module $F$ by the above Proposition.\\

\num{4.11a.} Suppose $\{m_1,\ldots,m_n\}$ is a generating set for the projective module $M$ with the minimal number of elements (we can do this because $M$ is finitely generated), and define $\alpha:R^n\rightarrow M$ by $\alpha(x_i)=m_i$. Then $\alpha$ is an epimorphism (because all the generators of $M$ are in the image), so that by Problem 1, $\alpha$ splits, and we have $R^n\simeq M\oplus\ker(\alpha)$ by the above Proposition. Tensoring with $R/P$, where $P$ is the maximal ideal of $R$, we have $(R/P)^n\simeq M/PM\oplus \ker(\alpha)/P\ker(\alpha)$ (because for any module $N$, we have $N\otimes_R R/I\simeq N/IN$, and tensor products distribute over direct sums). These are now $R/P$-modules, i.e. vector spaces. Thus the dimension of $M/PM$ as an $R/P$-vector space is less than or equal to $n$. Suppose $a_1,\ldots,a_k$ is a basis for $M/PM$ (so that $k\leq n$), and choose any $b_1\in\phi^{-1}(a_1),\ldots,b_k\in\phi^{-1}(a_k)$, where $\phi:M\rightarrow M/PM$ is the quotient map. Let $N=(b_1,\ldots,b_k)\subseteq M$. Then the composition $N\hookrightarrow M\rightarrow M/PM$ is onto, so that $N+PM=M$ and thus $M/N=P(M/N)$. But by Nakayama's Lemma, $M/N=0$, so $M=N$, so that the $b_1,\ldots,b_k$ generate $M$. But because the $m_1,\ldots,m_n$ are a generating set for $M$ with the least number of elements, we must have $k\geq n$, so that $k=n$. Thus the dimension of $M/PM$ as an $R/P$-vector space is $n$, so that because $(R/P)^n\simeq M/PM\oplus \ker(\alpha)/P\ker(\alpha)$, we must have $\ker(\alpha)/P\ker(\alpha)=0$, so that $\ker(\alpha)=P\ker(\alpha)$. By Nakayama's lemma again, $\ker(\alpha)=0$, so that by $R^n\simeq M\oplus\ker(\alpha)$, we have $R^n\simeq M$, and thus $M$ is free. \\

\num{4.11b.} This was a previous homework problem.    \\

\num{4.12a.} Let $P$ be any prime ideal of $R$, and suppose there is an isomorphism $f:M_P\rightarrow N_P$. Under the natural isomorphism between $\Hom(M_P,N_P)$ and $\Hom(M,N)_P$ (this works because $M$ and $N$ are finitely presented), we have that $\phi:M_P\rightarrow N_P$ corresponds to $\frac{\phi'}{f_1}$ for some $f_1\notin P$ and  $\phi':M\rightarrow N$ defined by $\phi(\frac{a}{b})=\frac{\phi'(a)}{f_1b}$. Because $\phi:M_P\rightarrow N_P$ is an isomorphism, every $\frac{c}{d}\in N_P$ has some $\frac{a}{b}\in M_P$ for which $\phi(\frac{a}{b})=\frac{c}{d}$, but then $\phi'_P(\frac{a}{f_1b})=\frac{\phi'(a)}{f_1b}=\phi(\frac{a}{b})=\frac{c}{d}$, so that $\phi'_P$ is surjective. 
\[M\stackrel{\phi'}{\rightarrow}N\rightarrow\coker(\phi')\rightarrow0\]
\[M_P\stackrel{\phi'_P}{\rightarrow}N_P\rightarrow\coker(\phi')_P\rightarrow0\]
Thus, we had the former exact sequence, then localized at $P$ to get the latter exact sequence, and because $\phi'_P$ is surjective, we have that $\coker(\phi')_P=0$. Because $N$ is finitely generated, $\coker(\phi')$, which is a quotient of $N$, is finitely generated, say by $x_1,\ldots,x_k$. Because $\coker(\phi')_P=0$, there must be $r_i\in R-P$ such that $r_ix_i=0$. Thus $f_2=r_1\cdots r_k\notin P$ annihilates $\coker(\phi')$. Because $\phi$ is an isomorphism and everything is symmetric in $M$ and $N$, we can run this argument for $\psi=\phi^{-1}$. Thus $\psi=\frac{\psi'}{f_3}$ for some $f_3\notin P$ and $f_4\notin P$ annihilates $\coker(\psi')$. Let $f=f_1f_2f_3f_4$.     \\

We have that $\phi'_f:M[f^{-1}]\rightarrow N[f^{-1}]$ is surjective because $\coker(\phi')[f^{-1}]=0$ (because $f_2$ annihilates $\coker(\phi')$ and $\coker(\phi')[f^{-1}]=\frac{f_2\coker(\phi')[f^{-1}]}{f_2}=0$). Similarly, $\psi'_f:N[f^{-1}]\rightarrow M[f^{-1}]$ is surjective. Thus both $\phi'_f\psi'_f$ and $\psi'_f\phi'_f$ are surjective, and thus both $\phi'_f\psi'_f$ and $\psi'_f\phi'_f$ are isomorphisms due to Corollary 4.4a (because $M$ and $N$ are finitely presented and hence in particular finitely generated). Thus, individually, $\phi'_f$ and $\psi'_f$ are isomorphisms, i.e. $M[f^{-1}]\simeq N[f^{-1}]$.  \\

Now suppose $M$ is projective. By Problem 4.11b, $M_P\simeq (R_P)^n\simeq (R^n)_P$ for some $n$, for each prime $P$. Above, we constructed, for each prime $P$, an $f_P\notin P$ such that $M[f^{-1}]\simeq N[f^{-1}]$, on the assumption that $M_P\simeq N_P$. Since we have $M_P\simeq (R^n)_P$, there is some $f_P\notin P$ such that $M[f^{-1}]\simeq (R^n)[f^{-1}]\simeq R[f^{-1}]^n$. Because each $f_P\notin P$, the set $\{f_P: P\text{ prime}\}$ must generate all of $R$ (otherwise, the ideal it generates is a proper ideal, and hence contained in some maximal ideal $P$, which in particular is prime; but $f_P\notin P$). Thus, $1\in R$ is in the ideal generated by the $f_P$, and thus is some finite combination of the $f_P$, say $f_{P_1},\ldots,f_{P_n}$. Then $(f_{P_1},\ldots,f_{P_n})=R$, and $M[f_{P_i}^{-1}]$ is a free $R[f_{P_i}^{-1}]$-module for each $f_{P_i}$.\\

Now we prove the converse - suppose that there is some $f_1,\ldots,f_n\in R$ with $(f_1,\ldots,f_n)=R$ and $M[f_i^{-1}]$ is a free $R[f_i^{-1}]$-module for each $f_i$. Let $P$ be any maximal ideal. Then there is some $f_i\notin P$ (otherwise $P=R$). Because $\{1,f_i,f_i^2,\ldots\}\subset R-P$, we have that $R[f^{-1}]_P\simeq R_P$ via $\frac{m/f_i^n}{s}\mapsto \frac{m}{f_i^ns}$ (this is obviously surjective, and a homomorphism via simple manipulation arguments - it is injective because $\frac{m}{f_i^ns}=\frac{0}{1}$ iff $\exists u\notin P$ with $um=0$, so that $\frac{m/f_i^n}{s}=\frac{u}{u}\frac{m/f_i^n}{s}=\frac{um/f_i^n}{us}=\frac{0/f_i^n}{us}=\frac{0/1}{1}$). Thus, because $M[f_i^{-1}]\simeq R[f_i^{-1}]^n$ for some $n$, we have $M[f_i^{-1}]_P\simeq (R[f_i^{-1}]^n)_P$ and hence $M_P\simeq (R^n)_P$, i.e. $M_P$ is a free $R_P$-module for all maximal ideals $P$. This implies that $M$ is projective, by Problem 4.11b.   \\

\num{4.12b.} It is clear that $K(R)$ is a flat $R$-module because $K(R)=R_0$, i.e. $R$ localized at the set of all non-zero elements, and it was shown in class that all localizations are flat. Suppose that $f:K(R)\rightarrow R$ is an $R$-module homomorphism, and that $f(\frac{1}{1})=a\in R$. If $a=0$, then $sf(\frac{r}{s})=rf(\frac{s}{s})=rf(\frac{1}{1})=r0=0$ for all $\frac{r}{s}\in K(R)$, and because $s\neq 0$ and $R$ is an integral domain, this implies $f(\frac{r}{s})=0$ for all $\frac{r}{s}\in K(R)$. If $a\neq 0$, then $af(\frac{1}{a})=f(\frac{a}{a})=f(\frac{1}{1})=a$, and because $R$ is an integral domain we can cancel $a$ from both sides to find that $f(\frac{1}{a})=1$. But because we assumed $R$ was not a field, there is some $b\in R$ with $b\neq0$, $b$ not a unit, and $bf(\frac{1}{ab})=f(\frac{b}{ab})=f(\frac{1}{a})=1$, so that $f(\frac{1}{ab})\in R$ is an inverse to $b$, contradiction. Thus all $R$-module homomorphisms from $K(R)$ to $R$ must be the zero map.\\

Suppose $K(R)$ were projective. Then it would be the direct summand of a free module $R^\alpha$ for some $\alpha$ (not necessarily finite), and in particular there is an injection $j:K(R)\rightarrow R^\alpha$. But $R^\alpha$ has projections $f_i:R^\alpha\rightarrow R$ on each of its coordinates, and each $f_i\circ j:K(R)\rightarrow R$ must be the zero map, so that $j$ must be the zero map, contradiction. Thus $K(R)$ is not projective.    \\

\num{4.} First, we show that every finitely generated free module has a coordinate family: given $R^n$, take $x_i=(0,\ldots,1,\ldots,0)$ where the 1 is in the $i$th coordinate, and $f_i:R^n\rightarrow R$ is projection onto the $i$th coordinate. Then $x=a_1x_1+\cdots+a_nx_n=\sum_{i=1}^n f_i(x)x_i$.\\

Suppose $M$ is projective. Then it is the direct summand of a free module $F$, which we can assume is finitely generated (because if $M$ is generated by $m_1,\ldots,m_n$, then the free module on $x_1,\ldots,x_n$ has the obvious epimorphism to $M$, namely $x_i\mapsto m_i$, which splits, say with $\gamma:M\rightarrow F$). There is a coordinate family $x_i,f_i$ on $F$, and $\gamma^{-1}(x_i),\gamma\circ f_i$ obviously satisfies the conditions of being a coordinate family on $M$.      \\

Suppose $M$ has a coordinate family $m_1,\ldots,m_n,f_1,\ldots,f_n$. The free module $F$ on $x_1,\ldots,x_n$ has the obvious epimorphism $\alpha:F\rightarrow M$ defined by $\alpha(x_i)=m_i$. To show that this splits, define $\gamma:M\rightarrow F$ by $\gamma(m)=\sum_{i=1}^n f_i(m)x_i$. Then the fact that $\id_M=\alpha\circ\gamma$ follows from $\alpha(\gamma(m))=\alpha(\sum_{i=1}^n f_i(m)x_i)=\sum_{i=1}^n f_i(m)m_i$, which is the defining relationship of a coordinate family. Thus, by Problem 1, $M$ is projective.\\

\num{5.} To check that $\lambda$ is well-defined on $M\otimes_R M^*$, we simply check that the map $\tilde{\lambda}:M\times M^*\rightarrow\Hom_R(M,M)$ defined by $\tilde{\lambda}(y,f)(x)=f(x)y$ is bilinear. This is the case because \[\tilde{\lambda}(ay_1+by_2,f)=f(x)(ay_1+by_2)=af(x)y_1+bf(x)y_2=a\tilde{\lambda}(f,y_1)+b\tilde{\lambda}(f,y_2)\] \[\tilde{\lambda}(y,af_1+bf_2)=(af_1+bf_2)(x)y=(af_1(x)+bf_2(x))y=af_1(x)y+bf_2(x)y=a\tilde{\lambda}(f_1,y)+b\tilde{\lambda}(f_2,y)\]
Thus the map $\lambda:M\otimes_R M^*\rightarrow\Hom_R(M,M)$ is a well-defined homomorphism because it is induced by $\tilde{\lambda}$. Now suppose that $M$ is projective. By Problem 4, it has a coordinate family $m_1,\ldots,m_n,f_1,\ldots,f_n$. For any $\phi\in\Hom(M,M)$ and $m\in M$, we have $\phi(m)=\sum_{i=1}^n f_i(\phi(m))m_i$. Thus, $\lambda(\sum_{i=1}^n m_i\otimes (f_i\circ\phi))(m)=\sum_{i=1}^n f_i(\phi(m))m_i=\phi(m)$, so that $\sum_{i=1}^n m_i\otimes(f_i\circ\phi)$ maps to $\phi$. Thus $\lambda$ is surjective.\\

If $\lambda$ is surjective, then some $\sum_{i=1}^n m_i\otimes f_i$ is mapped to $\id_M$ by $\lambda$. That is, $\lambda(\sum_{i=1}^n m_i\otimes f_i)=\sum_{i=1}^n f_i(m)m_i=m$. Thus the $m_i$ and $f_i$ form a coordinate family.\\

Suppose $M$ is projective, so that $M$ has a coordinate family $m_1,\ldots,m_n,f_1,\ldots,f_n$. Then the function $\psi:\Hom(M,M)\rightarrow M\otimes M^*$ defined by $\psi(g)=\sum_{i=1}^n m_i\otimes (f_i\circ g)$ is inverse to $\lambda$ (proving this is a mess).

%because $\lambda(\sum n_i\otimes g_i)(m)=\sum g_i(m)n_i$, so that $\psi(\lambda(\sum n_i\otimes g_i))=\sum_{i=1}^n m_i\otimes (f_i\circ \sum g_i(-)n_i)$\\

%$\psi(\lambda(\sum n_i\otimes g_i))=\sum\psi(\lambda(n_i\otimes g_i))=\sum\psi(g_i(-)n_i)=\sum(\sum m_i\otimes (f_i\circ g)(-)n_i)$








\end{document}
f_1,\ldots,f_n$. Then the function $\psi:\Hom(M,M)\rightarrow M\otimes M^*$ defined by $\psi(g)=\sum_{i=1}^n m_i\otimes (f_i\circ g)$ is inverse to $\lambda$ (proving this is a mess).

%because $\lambda(\sum n_i\otimes g_i)(m)=\sum g_i(m)n_i$, so that $\psi(\lam