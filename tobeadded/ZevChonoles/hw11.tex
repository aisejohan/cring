\documentclass[11pt]{article}
\usepackage{amsmath}
\usepackage{amssymb}
\usepackage{amsfonts}
\usepackage{amsthm}
\usepackage{array}
\usepackage{bm}
\usepackage{wrapfig}
\usepackage[pdftex]{color}
\usepackage[pdftex]{graphicx}

%\usepackage{pdftricks}
%\begin{psinputs}
%\usepackage[dvips,ps,all]{xy}
%\end{psinputs}

\input xy
\xyoption{all}

%% This package sets the date to the more logical YYYY - MM - DD format.
\usepackage{datetime}
\renewcommand{\dateseparator}{-}
\yyyymmdddate

%% This command inserts \noindent and makes the input bold.
\newcommand{\num}[1]{\noindent \textbf{#1}}

%% Some math commands.
\renewcommand{\ker}{\operatorname{ker}}
\newcommand{\im}{\operatorname{im}}
\newcommand{\coker}{\operatorname{coker}}
\newcommand{\disc}{\operatorname{disc}}
\newcommand{\id}{\operatorname{id}}
\newcommand{\rad}{\operatorname{rad}}
\newcommand{\Gal}{\operatorname{Gal}}
\newcommand{\Aut}{\operatorname{Aut}}
\newcommand{\Irr}{\operatorname{Irr}}
\newcommand{\Char}{\operatorname{char}}
\newcommand{\Hom}{\operatorname{Hom}}
\newcommand{\End}{\operatorname{End}}
\newcommand{\Ann}{\operatorname{Ann}}
\newcommand{\Ass}{\operatorname{Ass}}
\newcommand{\Supp}{\operatorname{Supp}}
\newcommand{\Frac}{\operatorname{Frac}}
\newcommand{\Spec}{\operatorname{Spec}}
\newcommand{\mSpec}{\operatorname{mSpec}}
\renewcommand{\dim}{\operatorname{dim}}
\newcommand{\codim}{\operatorname{codim}}
\newcommand{\height}{\operatorname{ht}}
\newcommand{\length}{\operatorname{length}}
\newcommand{\depth}{\operatorname{depth}}
\newcommand{\Ext}{\operatorname{Ext}}
\newcommand{\Tor}{\operatorname{Tor}}

%% Some categories.
\newcommand{\Set}{\ensuremath{\text{\sf Set}}}
\newcommand{\Cat}{\ensuremath{\text{\sf Cat}}}
\newcommand{\Top}{\ensuremath{\text{\sf Top}}}
\newcommand{\Grp}{\ensuremath{\text{\sf Grp}}}
\newcommand{\Ring}{\ensuremath{\text{\sf Ring}}}
\newcommand{\CRing}{\ensuremath{\text{\sf CRing}}}
\newcommand{\Ab}{\ensuremath{\text{\sf Ab}}}
\newcommand{\Mod}{\ensuremath{\text{\sf Mod}}}
\newcommand{\Fun}{\ensuremath{\text{\sf Fun}}}
\newcommand{\Nat}{\ensuremath{\text{\sf Nat}}}
\newcommand{\Vect}{\ensuremath{\text{\sf Vect}}}

%% Shortening of some standard commands.
\newcommand{\bb}[1]{\ensuremath{\mathbb{#1}}}
\renewcommand{\cal}[1]{\ensuremath{\mathcal{#1}}}
\newcommand{\san}[1]{\ensuremath{\text{\sf #1}}}
\renewcommand{\frak}[1]{\ensuremath{\mathfrak{#1}}}

%% Standard double-struck letters.
%% Integers.
\newcommand{\Z}{\ensuremath{\mathbb{Z}}}
%% Rational numbers.
\newcommand{\Q}{\ensuremath{\mathbb{Q}}}
%% Real numbers.
\newcommand{\R}{\ensuremath{\mathbb{R}}}
%% Complex numbers.
\newcommand{\C}{\ensuremath{\mathbb{C}}}
%% Natural numbers.
\newcommand{\N}{\ensuremath{\mathbb{N}}}
%% The sphere.
\let\SS\S\renewcommand{\S}{\ensuremath{\mathbb{S}}}
%% A field.
\newcommand{\F}{\ensuremath{\mathbb{F}}}
%% The quaternions.
\renewcommand{\H}{\ensuremath{\mathbb{H}}}
%% The octonions.
\renewcommand{\O}{\ensuremath{\mathbb{O}}}
%% Projective space, or probability.
\renewcommand{\P}{\ensuremath{\mathbb{P}}}
%% The torus.
\newcommand{\T}{\ensuremath{\mathbb{T}}}
%% Affine space.
\newcommand{\A}{\ensuremath{\mathbb{A}}}
%% The ball.
\newcommand{\B}{\ensuremath{\mathbb{B}}}
%% The disk.
\newcommand{\D}{\ensuremath{\mathbb{D}}}

%% Non-standard double-struck letters.
%\newcommand{\E}{\ensuremath{\mathbb{E}}}
%\newcommand{\G}{\ensuremath{\mathbb{G}}}
%\newcommand{\I}{\ensuremath{\mathbb{I}}}
%\newcommand{\J}{\ensuremath{\mathbb{J}}}
%\newcommand{\K}{\ensuremath{\mathbb{K}}}
%\renewcommand{\L}{\ensuremath{\mathbb{L}}}
%\newcommand{\M}{\ensuremath{\mathbb{M}}}
%\newcommand{\N}{\ensuremath{\mathbb{N}}}
%\newcommand{\U}{\ensuremath{\mathbb{U}}}
%\newcommand{\V}{\ensuremath{\mathbb{V}}}
%\newcommand{\W}{\ensuremath{\mathbb{W}}}
%\newcommand{\X}{\ensuremath{\mathbb{X}}}
%\newcommand{\Y}{\ensuremath{\mathbb{Y}}}


%% This custom type produces a column of the specified width whose contents are centered.
\newcolumntype{C}[1]{>{\centering\hspace{0pt}}p{#1}}

%% amsthm styles.
\newtheorem{theorem}{Theorem}
\newtheorem{thm}{Theorem}
\newtheorem{cor}{Corollary}
\newtheorem{lemma}{Lemma}
\theoremstyle{definition}
\newtheorem{axiom}{Axiom}[section]
\newtheorem{definition}{Definition}
\newtheorem*{remark}{Remark}
\textwidth 7in
\textheight 9.5in
\oddsidemargin -0.25in
\topmargin -0.75in
\setlength\parindent{0pt}
\pagestyle{empty}
\begin{document}
Zev Chonoles \hfill 
\underline{MATH 2520 - Assignment 11} \hfill \today\\

\num{5.10i.} This is was on a previous homework assignment.\\%(a) $\Rightarrow$
(b): Suppose $f^*$ is a closed mapping. Let $p_1\subset p_2\subset f(A)$
be prime ideals, and let $q_1\in B$ be a prime ideal such that $q_1\cap
f(A)=p_1$. By the definition of the Zariski topology, $V(q_1)=\{\ell\supseteq
q_1:\ell\text{ prime in }B\}\subset\text{Spec}(B)$ is a closed set, and hence
$f^*(V(q_1))=\{f^{-1}(\ell)\supseteq f^{-1}(q_1)=f^{-1}(p_1):\ell\text{
prime in }B\}$ is a closed subset of Spec($A$). Preimages of prime
ideals are prime, so $f^{-1}(p_1)\in\text{Spec}(A)$, and clearly
$f^{-1}(p_1)\in f^*(V(q_1))$, and by the definition of the Zariski
topology, the closure of $\{f^{-1}(p_1)\}\subset\text{Spec}(A)$ is
$V(f^{-1}(p_1))=\{d\supseteq f^{-1}(p_1):d\text{ prime in }A\}$, and
thus we must have $V(f^{-1}(p_1))\subseteq f^*(V(q_1))$. If $p_2$ is any
prime ideal of $f(A)$ with $p_2\supseteq p_1$, then $f^{-1}(p_2)\supseteq
f^{-1}(p_1)$, so $f^{-1}(p_2)\in  V(f^{-1}(p_1))$, and hence $f^{-1}(p_2)\in
f^*(V(q_1))=\{f^{-1}(\ell)\supseteq f^{-1}(q_1)=f^{-1}(p_1):\ell\text{
prime in }B\}$. Thus there is some prime $q_2\subset B$ which has
$f^{-1}(q_2)=f^{-1}(p_2)$. We know that $f(f^{-1}(p_2))=p_2$, so that $q_2\cap
f(A)=p_2$. Thus $f:A\rightarrow B$ has the going-up property.\\

%(b) $\Rightarrow$ (c): Suppose $f:A\rightarrow B$ has the going-up
property. Suppose that $q_1\subset B$ is a prime and $p_1=f^{-1}(q_1)\subset
A$. Then $p_1$ is a prime, and contains the kernel $f^{-1}(0)$ of $f$. Thus
$f(p_1)\subset f(A)\simeq A/f^{-1}(0)$ is prime. We also have the induced
map $\tilde{f}:A/p_1\rightarrow B/q_1$, and consequently also the map
$\tilde{f}^*:\text{Spec}(B/p_1)\rightarrow\text{Spec}(A/p_1)$, sending
$\ell/q_1$ to $\tilde{f}^*(\ell/q_1)=\tilde{f}^{-1}(\ell/q_1)=f^{-1}(\ell)/p_1$
for primes $\ell\in B$ containing $q_1$. For any prime $p_2\subset
A$ with $p_2\supset p_1$, we know that $f(p_2)\supset f(p_1)$ and
that $f(p_2)\subset f(A)$ is a prime. By the assumption that $f$ has
the going-up property, there exists some $q_2\subset B$ with $q_2\cap
f(A)=f(p_2)\supset f(p_1)$, i.e. $f^*(q_2)=f^{-1}(q_2)=p_2\supset p_1$. Thus
$\tilde{f}^*(q_2/q_1)=f^{-1}(q_2)/p_1=p_2/p_1$. Because any prime ideal of
$A/p_1$ is of this form, we have that $\tilde{f}^*$ is surjective.\\    

%(c) $\Rightarrow$ (b): Let $p_1\subset p_2\subset f(A)$ be prime ideals,
and let $q_1$ be a prime ideal of $B$ such that $q_1\cap f(A)=p_1$. We

\num{5.11.} We proved this in class. Also, by Problem 3.18, if $f:A\rightarrow
B$ is a flat homomorphism of rings, then $f^*:\Spec(B_q)\rightarrow\Spec(A_p)$
is surjective, and by Problem 5.10ii, this implies that $f$ has the going-down
property.    \\

\num{5.12.} For any $\rho\in G$, let $\tilde{\rho}$ denote its canonical
extension to an automorphism of $A[t]$ (via acting on coefficients). Then
$g\in A[t]$ lies in $A^G[t]$ iff $\tilde{\rho}(g)=g$ for all $\rho\in G$,
because $\tilde{\rho}(g)=\rho(g_0)+\cdots+\rho(g_n)t^n=g_0+\cdots+g_nt^n=g$
for all $\rho\in G$ iff $\rho(g_i)=g_i$ for all $\rho\in G$ for all $g_i$,
which is the case iff $g_i\in A^G$ for all $i$. \\

Obviously, any $x\in A$ is a root of the polynomial $f(t)=\prod_{\sigma\in
G}(t-\sigma(x))\in A[t]$, because $x$ is a root of $t-\id(x)$. We have $f\in
A^G[t]$ because $\tilde{\rho}(f)=\tilde{\rho}\left(\prod_{\sigma\in
G}(t-\sigma(x))\right)=\prod_{\sigma\in
G}\tilde{\rho}(t-\sigma(x))=\prod_{\sigma\in
G}(t-\rho(\sigma(x)))=\prod_{\rho^{-1}\sigma\in G}(t-\sigma(x))=f$ for all
$\rho\in G$. Thus, any $x\in A$ is a root of a monic polynomial in $A^G$,
so that $A$ is integral over $A^G$.   \\

Define the action of $G$ on $S^{-1}A$ by
$\sigma[\frac{a}{s}]=\frac{\sigma(a)}{\sigma(s)}$. This agrees with the action
on $A$, i.e. if $i:A\rightarrow S^{-1}A$ is the canonical inclusion map then
$\sigma[i(a)]=\sigma[\frac{a}{1}]=\frac{\sigma(a)}{\sigma(1)}=\frac{\sigma(a)}{1}=i(\sigma(a))$.
Define $f:(S^G)^{-1}A^G\rightarrow (S^{-1}A)^G$ by
$f(\frac{a}{s})=\frac{a}{s}$, which is obviously well-defined
(if $\frac{a}{s}=\frac{b}{t}\in (S^G)^{-1}A^G$, then there
is a $u\in S^G\subset S$ such that $u(at-bs)=0$, so that
$f(\frac{a}{s})=\frac{a}{s}=\frac{b}{t}=f(\frac{b}{t})\in S^{-1}A$; and
for any $\frac{a}{s}\in (S^G)^{-1}A^G$, we have $a\in A^G$, $s\in S^G$ so that
$\sigma[f(\frac{a}{s})]=\sigma[\frac{a}{s}]=\frac{\sigma(a)}{\sigma(s)}=\frac{a}{s}=f(\frac{a}{s})$
for all $\sigma\in G$, hence $f(\frac{a}{s})\in (S^{-1}A)^G$)
and a homomorphism. The map $f$ is injective, because if
$f(\frac{a}{s})=\frac{a}{s}=\frac{0}{1}\in (S^{-1}A)^G$, then there is
some $u\in S$ such that $u(a1-0s)=ua=0$, so that $v(a1-0s)=va=0$ where
$v=\prod_{\sigma\in G}\sigma(u)\in S^G$ ($v\in S$ because $\sigma(S)\subseteq
S$ for all $\sigma\in G$, so that $\sigma(u)\in S$ for all $\sigma\in G$,
and $v$ is fixed by $G$ because any $\rho\in G$ will only permute the factors
of the product), so that $\frac{a}{s}=\frac{0}{1}\in (S^G)^{-1}A^G$. \\

Finally, to show that $f$ is surjective, we will show that for any
$\frac{a}{s}\in (S^{-1}A)^G$, we have $\frac{a}{s}=\frac{b}{t}\in
(S^{-1}A)^G$ for some $b\in A^G$, $t\in S^G$, so that
$f(\frac{b}{t})=\frac{b}{t}=\frac{a}{s}$. First, note that for any
$\frac{a}{s}\in (S^{-1}A)^G$, we have $\frac{a}{s}=\frac{ax}{sx}\in
(S^{-1}A)^G$ where $x=\prod_{\sigma\neq\id}\sigma(s)$ (note that
$sx=\prod_{\sigma\in G}\sigma(s)\in S$ because $\sigma(S)\subseteq S$
for all $\sigma\in G$, so that $\sigma(s)\in S$ for all $\sigma\in
G$). Furthermore, $sx$ is fixed by $G$ by the standard argument. Thus,
it suffices to show that for all $\frac{a}{s}\in (S^{-1}A)^G$ with
$s\in S^G$, we have $\frac{a}{s}=\frac{b}{t}$ for some $b\in A^G$,
$t\in S^G$ (because we have shown that all elements of $(S^{-1}A)^G$
can be expressed as an $\frac{a}{s}$ with this property). By
definition, any $\frac{a}{s}\in (S^{-1}A)^G$ with $s\in S^G$ has
$\sigma[\frac{a}{s}]=\frac{\sigma(a)}{\sigma(s)}=\frac{\sigma(a)}{s}=\frac{a}{s}$
for all $\sigma\in G$, so that $\frac{\sigma(a)}{1}=\frac{a}{1}$
for all $\sigma\in G$, so that there is some $u_\sigma\in S$
for each $\sigma\in G$ such that $u_\sigma(\sigma(a)-a)=0$,
i.e. $u_\sigma\sigma(a)=u_\sigma a$. Define $y=\prod_{\sigma\in
G}\prod_{\rho\in G}\rho(u_\sigma)$. We have $y\in S$ because
each $\rho(u_\sigma)\in S$, and $y$ is fixed by $G$ by the standard
argument. Then $\frac{a}{s}=\frac{ay}{sy}$, and $sy\in S^G$ because $S^G=S\cap
A^G$ is a multiplicative set and $s,y\in S^G$, and $ay\in A^G$ because
\[\hskip-0.15in\tau(ay)=\tau(a)\tau(y)=\tau(a)y=\tau(a)u_\tau\left(\prod_{\sigma\neq\tau}\prod_{\rho\in
G}\rho(u_\sigma)\right)\left(\prod_{\rho\neq\id}\rho(u_\tau)\right)=au_\tau\left(\prod_{\sigma\neq\tau}\prod_{\rho\in
G}\rho(u_\sigma)\right)\left(\prod_{\rho\neq\id}\rho(u_\tau)\right)=ay\]
for all $\tau\in G$. Thus, for every $\frac{a}{s}\in (S^{-1}A)^G$, there
is some $b\in A^G$ and $t\in S^G$ such that $\frac{a}{s}=\frac{b}{t}$
as elements of $(S^{-1}A)^G$, and thus every $\frac{a}{s}\in (S^{-1}A)^G$
is mapped to by some $\frac{b}{t}\in(S^G)^{-1}A^G$.\\  

\pagebreak

\num{5.13.} Consider any two prime ideals $p_1,p_2\subset A$ whose contraction
to $A^G$ is $p$. We want to show that $p_1=\sigma(p_2)$ for some $\sigma\in
G$, i.e. $G$ acts transitively on $P=\{\text{primes of $A$ whose contraction
to $A^G$ is $p$}\}$. For any $x\in p_1$, we have that $\prod_{\sigma\in
G}\sigma(x)\in p_1$ because $p_1$ is an ideal, and is fixed by $G$ by the
standard argument. Thus $\prod_{\sigma\in G}\sigma(x)\in p_1\cap A^G=p\subseteq
p_2$, and because $p_2$ is prime, we have that $\sigma(x)\in p_2$ for
some $\sigma\in G$. Thus, every $x\in p_1$ has $x\in \sigma^{-1}(p_2)$
for some $\sigma\in G$, so that $p_1\subseteq\bigcup_{\sigma\in
G}\sigma^{-1}(p_2)=\bigcup_{\sigma\in G}\sigma(p_2)$. Note that,
for any $\sigma\in G$, we have that $\sigma(p_2)$ is a prime ideal (if
$ab\in\sigma(p_2)$, then $\sigma^{-1}(a)\sigma^{-1}(b)\in p_2$, hence WLOG
$\sigma^{-1}(a)\in p_2$, hence $a\in \sigma(p_2)$), and $\sigma(p_2)\cap
A^G=\sigma(p_2)\cap\sigma(A^G)=\sigma(p_2\cap A^G)=\sigma(p)=p$ (we have used
that $\sigma$ is bijective). Because each $\sigma(p_2)$ is a prime ideal,
by Proposition 1.11i $p_1\subseteq\bigcup_{\sigma\in G}\sigma(p_2)$ implies
that $p_1\subseteq\sigma(p_2)$ for some $\sigma\in G$. But the contraction
of both $p_1$ and $\sigma(p_2)$ is $p$, so that by Proposition 5.9 we have
that $p_1=\sigma(p_2)$.\\

By the orbit-stabilizer theorem, this implies that $P$ is isomorphic to a
quotient of $G$, and hence finite.\\

\num{5.14.} If $r\in B$, then $r$ satisfies an
integral relation $a_0+\cdots+a_nr^n=0$ over $A$. But then
$\sigma(a_0+\cdots+a_nr^n)=\sigma(a_0)+\cdots+\sigma(a_n)\sigma(r)^n=a_0+\cdots+a_n\sigma(r)^n=\sigma(0)=0$,
because $A\subset K$ and all $\sigma\in G=\Gal(L/K)$ fix $K$, so that
$\sigma(r)$ is integral over $A$. Thus, $\sigma(B)\subseteq B$ for
all $\sigma\in G$, or equivalently $\sigma^{-1}(B)\subseteq B$ for all
$\sigma\in G$; but then $B\subseteq \sigma(B)$ for all $\sigma\in G$,
so that $B=\sigma(B)$ for all $\sigma\in G$. If $x\in B^G$, then $x\in
K=L^G$ by basic Galois theory, but $x\in B^G\subset B$ means that $x$
is integral over $A$, and because $A$ is integrally closed (i.e. in $K$),
this implies $x\in A$. Thus $B^G\subseteq A$. Conversely, it is clear that
because $A\subset K=L^G$ is fixed by all $\sigma\in G$ and $A\subset B$,
we have $A\subseteq B^G$, so that $A=B^G$.     \\

\num{5.15.} Because $L/K$ is finite, by a basic result of field theory we
have that $K^{sep}/K$ is finite and separable, and $L/K^{sep}$ is finite
and purely inseparable, where $K^{sep}$ is the separable closue of $K$ in
$L$. Let $C$ be the integral closure of $A\subset K$ in $K^{sep}$ and $B$
be the integral closure of $A\subset K$ in $L$. Thus, if the result is true
for finite separable and finite purely inseparable extensions, then for any
prime $q
\subset C$ whose contraction is the prime $p\subset A$, there are finitely
many primes $\ell\subset B$ whose contraction is $q$, and there are only
finitely many $q$, so that there are finitely many primes of $B$ whose
contraction to $A$ is $p$, thus proving the result for any finite extension.\\

If $L/K$ is finite and separable, then $L$ embeds into a finite normal
separable extension $F/K$, say where $F$ is the normal closure of $L$. Let
$D$ be the integral closure of $A$ in $F$, and let $P$ be the set of
primes $q\subset B$ for which $q\cap A=p$. Then for each $q\in P$, there
is at least one prime $d\subset D$ for which $d\cap B=q$ by Theorem 5.10
(and consequently, each $d$ has $d\cap A= (d\cap B)\cap A=q\cap A=p$). But
by Problems 5.13 and 5.14, there are only finitely many primes $d\subset D$
whose contraction to $A$ is $p$, so there must also be finitely many primes
of $B$ whose contraction to $A$ is $p$.\\

If $L/K$ is finite and purely inseparable (with
$\text{char}(L)=\text{char}(K)=\ell>0$), and $p\subset A$ is a prime with
$q\subset B$ a prime for which $q\cap A=p$, then because every $x\in q$
has $x^{\ell^n}\in K$ for some $n$ (a basic result of field theory) and
$x^{\ell^n}\in q$, we have $x^{\ell^n}\in q\cap K=q\cap A=p$ for some
$n$. Conversely, if $x\in B$ has $x^{\ell^n}\in p\subset q$ for some $n$,
then because $q$ is prime we must have $x\in q$. Finally, note that Theorem
5.10 guarantees that at least one such $q$ exists, and by our result above
it must be $q=\{x:x^{\ell^n}\in p\text{ for some $n$}\}$. Thus, each prime
$p\subset A$ has a unique prime $q\subset B$ whose contraction is $p$,
so that $\Spec(B)\rightarrow\Spec(A)$ is bijective (and thus certainly has
finite fibers).









\end{document}
 A=p$ for some $n$. Conversely, if $x\in B$ has $x^{\ell^n}\in p\subset q$
 for some $n$, then because $q$ is prime we must have $x\in q$. Finally,
 note that Theorem 5.10 guarantees that at least one such $q$ exists, and
 by our result above it must be $q=\{x:x^{\ell^n}\in p\text{ for some $n$}\}$.
