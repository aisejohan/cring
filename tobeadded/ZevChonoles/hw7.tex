\documentclass[11pt]{article}
\input{C:/Users/Public/stuff.tex}
\textwidth 7in
\textheight 9.5in
\oddsidemargin -0.25in
\topmargin -0.75in
\setlength\parindent{0pt}
\pagestyle{empty}
\begin{document}
Zev Chonoles \hfill 
\underline{MATH 2520 - Assignment 7} \hfill \today\\

\num{1.} If $S$ is finite over $R$, then $S=Rs_1+\cdots+Rs_a$
for some $s_i\in S$. If $M$ is finitely generated as an
$S$-module, then $M=Sm_1+\cdots+Sm_b$ for some $m_j\in M$. Thus
$M=(Rs_1+\cdots+Rs_a)m_1+\cdots+(Rs_1+\cdots+Rs_a)m_b=Rs_1m_1+\cdots+Rs_am_b$,
i.e. $M$ is finitely generated as an $R$-module by the $ab$ elements
$s_im_j\in M$.      \\

\num{2.} If $r_0s^n+\cdots+r_{n-1}s+r_n=0$ and $r_n$ is a unit, then
$s(r_0s^{n-1}+\cdots+r_{n-1})=-r_n$, and in general, if $ab=u$ for a unit
$u$, then $a$ and $b$ must both be units. Thus $s$ is a unit. If $S$ is an
integral extension of $R$, then every $s\in S$ satisfies an integral equation
over $R$ of minimal degree. Suppose $s^n+r_0s^{n-1}+\cdots+r_{n-2}=0$ is an
integral equation over $R$ of minimal degree. Then $r_n$ can't be 0, because
then we would have $s(s^{n-1}+r_0s^{n-2}+\cdots+r_{n-1})=0$, and $s\neq 0$,
so that $s$ satisfies an integral equation of lower degree. But $r_n\in R$,
$r_n\neq 0$ implies $r_n$ is a unit since $R$ is a field, and we showed that
this implies that $s$ is a unit. Thus all non-zero elements of $S$ are units,
i.e. $S$ is a field.       \\

\num{3.}       \\

\num{4.}       \\

\num{5.}       \\

\num{6.}       \\

\num{7.}       \\

\num{8.} (a) $\Rightarrow$ (b): Suppose $f^*$ is a closed mapping. Let
$p_1\subset p_2\subset f(A)$ be prime ideals, and let $q_1\in B$
be a prime ideal such that $q_1\cap f(A)=p_1$. By the definition
of the Zariski topology, $V(q_1)=\{\ell\supseteq q_1:\ell\text{
prime in }B\}\subset\text{Spec}(B)$ is a closed set, and hence
$f^*(V(q_1))=\{f^{-1}(\ell)\supseteq f^{-1}(q_1)=f^{-1}(p_1):\ell\text{
prime in }B\}$ is a closed subset of Spec($A$). Preimages of prime
ideals are prime, so $f^{-1}(p_1)\in\text{Spec}(A)$, and clearly
$f^{-1}(p_1)\in f^*(V(q_1))$, and by the definition of the Zariski
topology, the closure of $\{f^{-1}(p_1)\}\subset\text{Spec}(A)$ is
$V(f^{-1}(p_1))=\{d\supseteq f^{-1}(p_1):d\text{ prime in }A\}$, and
thus we must have $V(f^{-1}(p_1))\subseteq f^*(V(q_1))$. If $p_2$ is any
prime ideal of $f(A)$ with $p_2\supseteq p_1$, then $f^{-1}(p_2)\supseteq
f^{-1}(p_1)$, so $f^{-1}(p_2)\in  V(f^{-1}(p_1))$, and hence $f^{-1}(p_2)\in
f^*(V(q_1))=\{f^{-1}(\ell)\supseteq f^{-1}(q_1)=f^{-1}(p_1):\ell\text{
prime in }B\}$. Thus there is some prime $q_2\subset B$ which has
$f^{-1}(q_2)=f^{-1}(p_2)$. We know that $f(f^{-1}(p_2))=p_2$, so that $q_2\cap
f(A)=p_2$. Thus $f:A\rightarrow B$ has the going-up property.\\

(b) $\Rightarrow$ (c): Suppose $f:A\rightarrow B$ has the going-up
property. Suppose that $q_1\subset B$ is a prime and $p_1=f^{-1}(q_1)\subset
A$. Then $p_1$ is a prime, and contains the kernel $f^{-1}(0)$ of $f$. Thus
$f(p_1)\subset f(A)\simeq A/f^{-1}(0)$ is prime. We also have the induced
map $\tilde{f}:A/p_1\rightarrow B/q_1$, and consequently also the map
$\tilde{f}^*:\text{Spec}(B/p_1)\rightarrow\text{Spec}(A/p_1)$, sending
$\ell/q_1$ to $\tilde{f}^*(\ell/q_1)=\tilde{f}^{-1}(\ell/q_1)=f^{-1}(\ell)/p_1$
for primes $\ell\in B$ containing $q_1$. For any prime $p_2\subset
A$ with $p_2\supset p_1$, we know that $f(p_2)\supset f(p_1)$ and
that $f(p_2)\subset f(A)$ is a prime. By the assumption that $f$ has
the going-up property, there exists some $q_2\subset B$ with $q_2\cap
f(A)=f(p_2)\supset f(p_1)$, i.e. $f^*(q_2)=f^{-1}(q_2)=p_2\supset p_1$. Thus
$\tilde{f}^*(q_2/q_1)=f^{-1}(q_2)/p_1=p_2/p_1$. Because any prime ideal of
$A/p_1$ is of this form, we have that $\tilde{f}^*$ is surjective.\\

(c) $\Rightarrow$ (b):      \\

% $\tilde{f}^{-1}(q_2/q_1)=p_2/p_1$.
%
%But the prime ideals $p_2\subset A$ with $p_2\supset p_1$ are in bijection
with the prime ideals $p_2/p_1$ of $A/p_1$, and the prime ideals $q_2\subset
B$ with $q_2\supset q_1$ are in bijection with the prime ideals $q_2/q_1$
of $B/q_1$, so that the map that





     









\end{document}
f^{-1}(q_2)/p_1=p_2/p_1$. Because any prime ideal of $A/p_1$ is of this form,
we have that $\tilde{f}^*$ is surjective.\\

(c) $\Rightarrow$ (b):      \\

% $\tilde{f}^{-1}(q_2/q_1)=p_2/p_1$.
%
%But the prime ideals $p_2\subset A$ with $p_2\supset p_1$ are in bijection
with the prime ideals $p_2/p_1$ of $A/p_1$, and the prime ideals $q_2\subset
B$ with $q_2\supset q_1$ are in bijection with the prime ideals $q_2/q_1$ of $
