\documentclass[11pt]{article}
\usepackage{amsmath}
\usepackage{amssymb}
\usepackage{amsfonts}
\usepackage{amsthm}
\usepackage{array}
\usepackage{bm}
\usepackage{wrapfig}
\usepackage[pdftex]{color}
\usepackage[pdftex]{graphicx}

%\usepackage{pdftricks}
%\begin{psinputs}
%\usepackage[dvips,ps,all]{xy}
%\end{psinputs}

\input xy
\xyoption{all}

%% This package sets the date to the more logical YYYY - MM - DD format.
\usepackage{datetime}
\renewcommand{\dateseparator}{-}
\yyyymmdddate

%% This command inserts \noindent and makes the input bold.
\newcommand{\num}[1]{\noindent \textbf{#1}}

%% Some math commands.
\renewcommand{\ker}{\operatorname{ker}}
\newcommand{\im}{\operatorname{im}}
\newcommand{\coker}{\operatorname{coker}}
\newcommand{\disc}{\operatorname{disc}}
\newcommand{\id}{\operatorname{id}}
\newcommand{\rad}{\operatorname{rad}}
\newcommand{\Gal}{\operatorname{Gal}}
\newcommand{\Aut}{\operatorname{Aut}}
\newcommand{\Irr}{\operatorname{Irr}}
\newcommand{\Char}{\operatorname{char}}
\newcommand{\Hom}{\operatorname{Hom}}
\newcommand{\End}{\operatorname{End}}
\newcommand{\Ann}{\operatorname{Ann}}
\newcommand{\Ass}{\operatorname{Ass}}
\newcommand{\Supp}{\operatorname{Supp}}
\newcommand{\Frac}{\operatorname{Frac}}
\newcommand{\Spec}{\operatorname{Spec}}
\newcommand{\mSpec}{\operatorname{mSpec}}
\renewcommand{\dim}{\operatorname{dim}}
\newcommand{\codim}{\operatorname{codim}}
\newcommand{\height}{\operatorname{ht}}
\newcommand{\length}{\operatorname{length}}
\newcommand{\depth}{\operatorname{depth}}
\newcommand{\Ext}{\operatorname{Ext}}
\newcommand{\Tor}{\operatorname{Tor}}

%% Some categories.
\newcommand{\Set}{\ensuremath{\text{\sf Set}}}
\newcommand{\Cat}{\ensuremath{\text{\sf Cat}}}
\newcommand{\Top}{\ensuremath{\text{\sf Top}}}
\newcommand{\Grp}{\ensuremath{\text{\sf Grp}}}
\newcommand{\Ring}{\ensuremath{\text{\sf Ring}}}
\newcommand{\CRing}{\ensuremath{\text{\sf CRing}}}
\newcommand{\Ab}{\ensuremath{\text{\sf Ab}}}
\newcommand{\Mod}{\ensuremath{\text{\sf Mod}}}
\newcommand{\Fun}{\ensuremath{\text{\sf Fun}}}
\newcommand{\Nat}{\ensuremath{\text{\sf Nat}}}
\newcommand{\Vect}{\ensuremath{\text{\sf Vect}}}

%% Shortening of some standard commands.
\newcommand{\bb}[1]{\ensuremath{\mathbb{#1}}}
\renewcommand{\cal}[1]{\ensuremath{\mathcal{#1}}}
\newcommand{\san}[1]{\ensuremath{\text{\sf #1}}}
\renewcommand{\frak}[1]{\ensuremath{\mathfrak{#1}}}

%% Standard double-struck letters.
%% Integers.
\newcommand{\Z}{\ensuremath{\mathbb{Z}}}
%% Rational numbers.
\newcommand{\Q}{\ensuremath{\mathbb{Q}}}
%% Real numbers.
\newcommand{\R}{\ensuremath{\mathbb{R}}}
%% Complex numbers.
\newcommand{\C}{\ensuremath{\mathbb{C}}}
%% Natural numbers.
\newcommand{\N}{\ensuremath{\mathbb{N}}}
%% The sphere.
\let\SS\S\renewcommand{\S}{\ensuremath{\mathbb{S}}}
%% A field.
\newcommand{\F}{\ensuremath{\mathbb{F}}}
%% The quaternions.
\renewcommand{\H}{\ensuremath{\mathbb{H}}}
%% The octonions.
\renewcommand{\O}{\ensuremath{\mathbb{O}}}
%% Projective space, or probability.
\renewcommand{\P}{\ensuremath{\mathbb{P}}}
%% The torus.
\newcommand{\T}{\ensuremath{\mathbb{T}}}
%% Affine space.
\newcommand{\A}{\ensuremath{\mathbb{A}}}
%% The ball.
\newcommand{\B}{\ensuremath{\mathbb{B}}}
%% The disk.
\newcommand{\D}{\ensuremath{\mathbb{D}}}

%% Non-standard double-struck letters.
%\newcommand{\E}{\ensuremath{\mathbb{E}}}
%\newcommand{\G}{\ensuremath{\mathbb{G}}}
%\newcommand{\I}{\ensuremath{\mathbb{I}}}
%\newcommand{\J}{\ensuremath{\mathbb{J}}}
%\newcommand{\K}{\ensuremath{\mathbb{K}}}
%\renewcommand{\L}{\ensuremath{\mathbb{L}}}
%\newcommand{\M}{\ensuremath{\mathbb{M}}}
%\newcommand{\N}{\ensuremath{\mathbb{N}}}
%\newcommand{\U}{\ensuremath{\mathbb{U}}}
%\newcommand{\V}{\ensuremath{\mathbb{V}}}
%\newcommand{\W}{\ensuremath{\mathbb{W}}}
%\newcommand{\X}{\ensuremath{\mathbb{X}}}
%\newcommand{\Y}{\ensuremath{\mathbb{Y}}}


%% This custom type produces a column of the specified width whose contents are centered.
\newcolumntype{C}[1]{>{\centering\hspace{0pt}}p{#1}}

%% amsthm styles.
\newtheorem{theorem}{Theorem}
\newtheorem{thm}{Theorem}
\newtheorem{cor}{Corollary}
\newtheorem{lemma}{Lemma}
\theoremstyle{definition}
\newtheorem{axiom}{Axiom}[section]
\newtheorem{definition}{Definition}
\newtheorem*{remark}{Remark}
\textwidth 7in
\textheight 9.5in
\oddsidemargin -0.25in
\topmargin -0.75in
\setlength\parindent{0pt}
\pagestyle{empty}
\begin{document}
Zev Chonoles \hfill 
\underline{MATH 2520 - Final Exam} \hfill \today\\

I've chosen Problem 7.\\

{\sf\textbf{1.}} We know that the prime ideals of $S=R_Q/P_Q$ are in
order-preserving bijection with the prime ideals $T\subset R$ with $P\subseteq
T\subseteq Q$, that $M=Q_Q/P_Q$ is the unique maximal ideal in $S$, and that
$0=P_Q/P_Q$ is prime in $S$ (note: this implies $S$ has no zero-divisors
other than 0). Thus, because there is at least one prime $T\subset R$
with $P\subset T\subset Q$, there is at least one prime $T_Q/P_Q\subset S$
with $0\subset T_Q/P_Q\subset M$. This shows that $\dim(S)=\codim(M)\geq 2$
(since 0 is prime in $S$). Take any $a_1\in M$, $a_1\neq 0$ (i.e., $a_1$
is not a zero-divisor or a unit), and consider any prime $U_1\subset S$
minimal over $(a_1)$. We cannot have $U_1=M$, because Krull's Principal
Ideal theorem would imply that $\codim(M)\leq 1$, contradiction. We certainly
cannot have $U_1=0$, since $a_1\neq 0$. Thus $0\subset U_1\subset M$. \\

Now suppose we have shown there are primes $U_1,\ldots,U_n\subset S$ with
$0\subset U_i\subset M$ for all $i$. We cannot have $\bigcup_{i=1}^n U_i=M$,
because by Proposition 1.11 in Atiyah-Macdonald, we would have $M\subseteq
U_i$ and hence $M=U_i$ for some $i$ (by the maximality of $M$). Thus, there
is some $a_{n+1}\in M$ with $a_{n+1}\notin U_i$ for all $1\leq i\leq n$ (so
that $a_{n+1}\neq0$, and $a_{n+1}$ not a unit because $a_{n+1}\in M$). Let
$U_{n+1}\subset S$ be any prime minimal over $(a_{n+1})$. We cannot have
$U_{n+1}=M$, again because this would imply $\codim(M)\leq 1$, we cannot
have $U_{n+1}=0$ because $a_{n+1}\neq0$, and obviously, $U_{n+1}\neq U_i$
for $1\leq i\leq n$, because $a_{n+1}\in U_{n+1}$ and $a_{n+1}\notin U_i$
for $1\leq i\leq n$. Thus, $U_{n+1}$ is a distinct prime ideal with $0\subset
U_{n+1}\subset M$. By induction, we can continue this process, so that there
must be infinitely many primes $U_i\subset S$ lying strictly between 0 and $M$,
and hence there must be infinitely many primes $T\subset R$ lying strictly
between $P$ and $Q$. \\

\num{2.} Suppose that every $a\in P_1$ had $a\in P_i$ for some $2\leq
i\leq t$. Then $P_1\subseteq\bigcup_{i=2}^t P_i$, and by Proposition 1.11
in Atiyah-Macdonald, $P_1\subseteq P_i$ for some $2\leq i\leq t$. But all
the $P_i$ are minimal over $I$, so that $P_1=P_i$, contradiction. Thus,
there is some $a\in P_1$ for which $a\notin P_i$ for $2\leq i\leq t$, and
clearly $a\notin I$ because $I\subseteq P_i$ for all $i$. By Proposition
3.11 in Atiyah-Macdonald, we know that the prime ideals of $R[a^{-1}]$
are in order-preserving bijection with the prime ideals of $R$ disjoint
from $\{1,a,a^2,\ldots\}$, that $I^e\neq R[a^{-1}]$ because $a\notin I$,
and that $P_1^e= R[a^{-1}]$ because $a\in P_1$. Because $I\subseteq P_i$ for
all $2\leq i\leq t$, we have $I^e\subseteq P_i^e$ for all $2\leq i\leq t$,
so the $P_i^e$ are primes of $R[a^{-1}]$ containing $I^e$. If there were a
prime $Q\subset R[a^{-1}]$ with $I^e\subseteq Q\subset P_i^e$, we would have
that the prime ideal $Q^c\subset R$ satisfies $I\subseteq I^{ec}\subseteq
Q^c\subset P_i^{ec}=P_i$, contradicting the minimality of $P_i$. Thus, the
$P_i^e$ for $2\leq i\leq t$ are minimal over $I^e$, and $P_1^e=R[a^{-1}]$.  \\

\num{3.}     \\

\num{4.} Consider the case $t=2$, so that $M\simeq I_1\oplus I_2$. Becaus
$I_1$ and $I_2$ are relatively prime, we have that $I_1+I_2=R$, so that
the map $g:I_1\oplus I_2\rightarrow R$ defined by $g(a_1,a_2)=a_1+a_2$
is surjective. The kernel is the set of $(a,-a)$ where $a\in I_1\cap I_2$,
which because $I_1$ and $I_2$ are relatively prime, is equal to $I_1I_2$
(see bottom p.6 of Atiyah-Macdonald). Thus we have an exact sequence
\[0\rightarrow I_1I_2\stackrel{f}{\rightarrow} I_1\oplus
I_2\stackrel{g}{\rightarrow} R\rightarrow 0\]
where $f$ is the map $f(a)=(a,-a)$. The map $h$ from $I_1\oplus I_2$ to
$I_1I_2$ defined by $h(a_1,a_2)=a_1x-a_2y$ where $x\in I_1$, $y\in I_2$
satisfy $x+y=1$ ($x$ and $y$ exist because $I_1+I_2=R$), is a retract of $f$
- that is, $f(h(a))=f(a,-a)=a(x+y)=a$ - and thus the sequence splits (see
p.16 of Eisenbud, as well as top of p.17). Thus, $I_1\oplus I_2\simeq R\oplus
I_1I_2$, and by induction (or repeated application, either will suffice),
we are done.\\

%where the unique prime factorizations of $I_1$ and $I_2$ are
$I_1=P_1^{a_1}\cdots P_n^{a_n}=P_1^{a_1}\cap\cdots\cap P_n^{a_n}$ and
$I_2=Q_1^{b_1}\cdots Q_m^{b_m} = Q_1^{b_1}\cap\cdots\cap Q_m^{b_m}$, where
the $P_i$ and $Q_j$ are prime ideals of $R$ (products and intersections
of distinct prime ideals are the same, as all non-zero primes in a Dedekind
domain are maximal and hence pairwise coprime). We know that $R$ is a Dedekind
domain, and hence locally a DVR. At a prime ideal $T\subset R$, with $T\neq
P_i, Q_j$ for all $i$ and $j$, we have $I_1,I_2\subset R-T$ and hence
$M_T\simeq(I_1\oplus I_2)\otimes R_T\simeq (I_1)_T\oplus (I_2)_T\simeq($    \\

\num{5.} Let $R=k[x_1,\ldots,x_n]$ for any field $k$. We proved in class
that $\dim(R)=n$. Let $A=R/(f)\simeq k[\bar{x}_1,\ldots,\bar{x}_n]$ for an
irreducible $f\in R$ (throughout, bars will denote reduction mod $f$, although
an element of a ring and its image in a localization will be identified without
comment). By Corollary 11.18 in Atiyah-Macdonald, $\dim(A)=n-1$, because $f$
irreducible implies $f$ is not 0 or a unit, and there are no zero-divisors
in $R$ other than 0. WLOG, let $P=(0,\ldots,0)$ be a point on the variety
defined by $f=0$ (one can move an arbitrary point to the origin by a linear
change of variables, which is an isomorphism). Let $M=(x_1,\ldots,x_n)\subset
R$, and let $M_M=(x_1,\ldots,x_n)\subset R_M$ be the maximal ideal of
the local ring $R_M$. Let $m=M/(f)=(\bar{x}_1,\ldots,\bar{x}_n)\subset
A$ be the maximal ideal of $A$ associated to the point $P$. Finally, let
$m_m=(\bar{x}_1,\ldots,\bar{x}_n)\subset A_m$ be the maximal ideal of the
local ring $A_m$, and note that $m_m\simeq M_M/(f)$ by Corollary 3.4 in
Atiyah-Macdonald. By Corollary 11.27 in Atiyah-Macdonald, we have that
$\dim(A_m)=n-1$. By Theorem 11.22 in Atiyah-Macdonald, $A_m$ is a regular
local ring iff $\dim_k(m_m/m_m^2)=n-1$. We have that $m_m/m_m^2\simeq
M_M/(M_M^2+(f))$. Because $P=(0,\ldots,0)$ is a point on the variety $f=0$,
we have that $f$ has no constant term, i.e. $f\in M$ (and also $f\in M_M$). \\

%(\bar{x}_1,\ldots,\bar{x}_n)/(\bar{x}_1,\ldots,\bar{x}_n)^2
(x_1,\ldots,x_n)/(((x_1,\ldots,x_n)^2+(f))=

If all partial derivatives of $f$ vanish at $P$, then all partial derivatives
of $f$ have no constant term, so that $f$ has no terms of the form $ax_i$
for $a\in k$, which combined with the fact that $f$ has no constant term
implies $f\in M^2$ (and also $f\in M_M^2$). The converse, that $f\in M^2$
(or equivalently $f\in M_M^2$) implies all partial derivatives of $f$ vanish
at $P$, is obvious by the product rule. Thus, if all partial derivatives
of $f$ vanish at $P$, we have that $f\in M_M^2$, and thus $m_m/m_m^2\simeq
M_M/(M_M^2+(f))=M_M/M_M^2=(x_1,\ldots,x_n)/(x_1,\ldots,x_n)^2\simeq k^n$
is $n$-dimensional, so that $A_m$ is not regular.\\

Now suppose not all partial derviatives of $f$ vanish at $P$, i.e. some
$\frac{\partial f}{\partial x_i}(P)=\frac{\partial f}{\partial
x_i}(0,\ldots,0)\neq0$. Because $f\in M=(x_1,\ldots,x_n)$, we
have that $f=\sum_{i=1}^n g_ix_i$ for some $g_i\in R$, so that
$\frac{\partial f}{\partial x_i}=\sum_{i=1}^n \frac{\partial
g_i}{\partial x_i}x_i + g_i$, so that $\frac{\partial f}{\partial
x_i}(0,\ldots,0)=\sum_{i=1}^n g_i(0,\ldots,0)\neq0$, so at least one
$g_j\notin M$, hence $g_j\notin M_M$, But $M_M$ is the maximal ideal
of $R_M$, so that $g_j$ is a unit in $R_M$. We certainly have that
$M_M=(x_1,\ldots,x_n)\supseteq (x_1,\ldots,x_{j-1},f,x_{j+1},\ldots,x_n)$,
but because $g_j$ is a unit in $R_M$, we also have
$(x_1,\ldots,x_{j-1},f,x_{j+1},\ldots,x_n)=(x_1,\ldots,x_{j-1},\sum_{i=1}^n
g_ix_i,x_{j+1},\ldots,x_n)=(x_1,\ldots,x_{j-1},g_ix_i,\ldots,x_n)=(x_1,\ldots,x_n)$.
Thus, if not all partial derivatives of $f$ vanish at $P$,
we have $M_M=(x_1,\ldots,x_{j-1},f,x_{j+1},\ldots,x_n)$, and thus
$m_m=M_M/(f)=(\bar{x}_1,\ldots,\bar{x}_{j-1},\bar{x}_{j+1},\ldots,\bar{x}_n)$
is generated by exactly $n-1$ elements. Thus $\dim_k(m_m/m_m^2)=n-1$, and
$A_m$ is regular. \\

(I'm sure this all could have been done more easily, and with less
notation... oh well)\\

\num{7.} We have that $R$ is a Noetherian ring, that $S\supseteq R$ is an
$R$-algebra which is finitely generated as an $R$-module, and $M\subset R$
is a maximal ideal. In particular, we have that $S$ is integral over $R$
(by Proposition 5.1 in Atiyah-Macdonald, because obviously every $x\in S$
satisfies $R[x]\subseteq S$ and $S$ is already known to be finitely generated
as an $R$-module), and $S$ is Noetherian (as a ring) by Proposition 7.2 of
Atiyah-Macdonald. \\

Let $f:R\hookrightarrow S$ be the inclusion of $R$ into $S$, and let
$\pi:\Spec(S)\rightarrow\Spec(R)$ be the induced map. Let $M^e=MS\subset
S$ denote the extension of $M$ to an ideal of $S$, and note that any ideal
$I\subset S$ has $I\supseteq M^e$ iff $I\supseteq M$. We proved in class that
in a Noetherian ring, there are only finitely many prime ideals minimal
over a given ideal, so that there are only finitely many prime ideals
$N_1,\ldots,N_r\subset S$ minimal over $M^e$. Note that $N_i\cap R=M$,
because $N_i\cap R\supseteq M$ and $M$ is maximal in $R$, so either $N_i\cap
R=M$ or $N_i\cap R=R$, but $N_i\cap R=R$ iff $1\in N_i$, which is not the
case. By Corollary 5.8 in Atiyah-Macdonald, each $N_i$ must be maximal as
well. Any prime $P\subset S$ with $P\cap R=M$ must have $P\supseteq M$,
hence $P\supseteq M^e$, and thus $P\supseteq N_i$ for some $i$ because the
$N_i$ are minimal over $M^e$, but the $N_i$ are maximal ideals of $S$, so
that  $P=N_i$. Thus, the only prime ideals of $S$ containing $M$ are the
$N_1,\ldots,N_r$, which we have shown to be maximal.\\

I can't seem to prove that the size of the fiber is $\leq t$, though. However,
once this was accomplished, I'm sure the proof that all fibers $\pi^{-1}(P)$
have size bounded by $t$ would proceed by localizing at any $P\in\Spec(R)$,
which is equivalent to restricting $\pi$ to a map on an open subset of
$\Spec(S)$ which would still contain all the primes in the fiber $\pi^{-1}(P)$,
only now $P$ is maximal (the number of generators of $S$ as an $R$-module
cannot increase upon localizing $S$ and $R$, so we don't have to worry about
ending up with a bigger $t$).

%I'm sorry I didn't get more done, but I've been busy preparing for another
final coming up. Thanks again for an excellent semester, and see you in 253
this fall! I'm sure the SMALL program will be great preparation.





\end{document}
 of $\Spec(S)$ which would still contain all the primes in the fiber
 $\pi^{-1}(P)$, only now $P$ is maximal (the number of generators of $S$
 as an $R$-module cannot increase upon localizing $S$ and $R$, so we don't
 have to worry about ending up with a bigger $t$).

%I'm sorry I
