\documentclass[11pt]{article}
\input{C:/Users/Public/stuff.tex}
\textwidth 7in
\textheight 9.5in
\oddsidemargin -0.25in
\topmargin -0.75in
\setlength\parindent{0pt}
\pagestyle{empty}
\begin{document}
Zev Chonoles \hfill 
\underline{MATH 2520 - Assignment 6} \hfill \today\\

\num{1.} Let $M$ be a finitely generated abelian group. By the relevant classification theorem,
\[M\simeq (\Z/p_1^{a_1}\Z)\oplus\cdots\oplus(\Z/p_n^{a_n}\Z)\oplus\Z^r\]
for some collection of not-necessarily-distinct primes $p_i$, some $a_i>0$, and some $r\geq 0$. Let $e_i=(0,\ldots,1,\ldots,0)\in M$, where the 1 is in the $i$th coordinate. If $r>0$, then $e_{n+r}$ has $\Ann(e_{n+r})=(0)$, because for $k\in\Z$, $k\cdot e_{n+r}=(0,\ldots,k\cdot1)=(0,\ldots,0)\in M$ iff $k=0$ because $\Z$ has no zero-divisors, and $(0)$ is indeed prime in $\Z$, so that $(0)$ is associated to $M$. However, if $r=0$, then every element of $M$ is torsion, hence every element has a non-zero annihilator, and hence $(0)$ will not be an associated prime of $M$. \\

Now, for each $1\leq i\leq n$, consider $\Ann(p_i^{a_i-1}e_i)$. For $k\in\Z$, $k\cdot p_i^{a_i-1}e_i=(0,\ldots,kp_i^{a_i-1}+p_i^{a_i}\Z,\ldots,0)=(0,\ldots,0)\in M$ iff $kp_i^{a_i-1}+p_i^{a_i}\Z=0+p_i^{a_i}\Z$, which is the case iff $p_i\mid k$ by unique factorization in $\Z$. Thus $\Ann(p_i^{a_i-1}e_i)=(p_i)$ for each $i$ (recall that the $p_i$ are not necessarily distinct, but this doesn't matter as long as we get all of them), and $(p_i)$ is prime, and hence $(p_i)$ is associated to $M$.\\

Finally, suppose $(q)\in\Z$ is a prime ideal associated to $M$ and $q\neq p_i$ for all $i$. Then there is an $m=(m_1+p_1^{a_1}\Z,\ldots,m_n+p_n^{a_n}\Z,m_{n+1},\ldots,m_{n+r})\in M$ such that $(q)=\Ann(m)$. Then $qm_i+p_i^{a_i}\Z=0+p_i^{a_i}\Z$ for each $1\leq i\leq n$, and $qm_i=0$ for each $n+1\leq i\leq n+r$. Because $\Z$ has no zero-divisors, $m_i=0$ for each $n+1\leq i\leq n+r$, and $qm_i+p_i^{a_i}\Z=0+p_i^{a_i}\Z$ iff $p_i^{a_i}\mid qm_i$ - but $q\neq p_i$ for all $i$, so that $p_i^{a_i}\mid m_i$ for all $i$, so that $m_i+p_i^{a_i}\Z=0+p_i^{a_i}\Z$ for all $i$, so that in fact $m=(0,\ldots,0)$. But $\Ann(0,\ldots,0)=\Z\neq(q)$, contradiction.\\

Thus, $\Ass(M)=\{(p_1),\ldots,(p_n)\}$ if $r=0$, and $\Ass(M)=\{(0),(p_1),\ldots,(p_n)\}$ if $r>0$.\\

\num{2.} The kernel of the map $f:M\rightarrow (M/M_1)\oplus(M/M_2)$ is obviously $M'=M_1\cap M_2$, so that we have the induced monomorphism $\tilde{f}:M/M'\rightarrow(M/M_1)\oplus(M/M_2)$. We can complete this to an exact sequence 
\[0\rightarrow M/M'\rightarrow (M/M_1)\oplus(M/M_2)\rightarrow C\rightarrow 0\]
where $C\simeq\coker(\tilde{f})$, and by Lemma 3.6b of Eisenbud, $\Ass(M/M')\subseteq\Ass((M/M_1)\oplus(M/M_2))$. Finally, by Lemma 3.6a we have that $\Ass((M/M_1)\otimes(M/M_2))=\Ass(M/M_1)\cup\Ass(M/M_2)$, so we are done.      \\

\num{3.} We know that $(M\otimes_R N)_P\simeq M_P\otimes_{R_P}N_P$. Because $M$ and $N$ are finitely generated over $R$, we know that $M_P$ and $N_P$ are finitely generated over $R_P$, so from a previous homework assignment we can conclude that $M_P\otimes_{R_P}N_P\simeq 0$ iff $M_P\simeq 0$, $N_P\simeq 0$, or both. Thus $(M\otimes_R N)_P\simeq 0$ iff $M_P\simeq 0$, $N_P\simeq 0$, or both, so that $P\notin\Supp(M\otimes_R N)$ iff $P\notin\Supp(M)\cup\Supp(N)$, so that $\Supp(M\otimes_R N)=\Supp(M)\cap\Supp(N)$.     \\

\num{4.} Because $R$ is Noetherian and $M$ and $N$ are finitely generated, we proved in class that $M$ and $N$ are finitely presented. Thus, by Proposition 2.10 (p.69 in Eisenbud), we have that $\Hom(M,N)_P\simeq\Hom(M_P,N_P)$ for any prime ideal $P$. We know that $\Ass(M[U^{-1}])=\{P[U^{-1}]:P\in\Ass(M),P\cap U=\emptyset\}$, so that 
\[\Ass(\Hom(M_P,N_P))=\Ass(\Hom(M,N)_P)=\{Q_P:Q\in\Ass(\Hom(M,N)),Q\subseteq P\}\]
Thus, if $P\in\Ass(\Hom(M,N))$, then $P_P\in\Ass(\Hom(M_P,N_P))=\Ass(\Hom(M,N)_P)$ (because $P\subseteq P$). If we had $M_P=0$, then any map from $M_P$ to $N_P$ would be the zero map, and hence $\Ass(\Hom(M_P,N_P))=\Ass(0)=\emptyset$ - but $P_P\in\Ass(\Hom(M_P,N_P))$, so we cannot have $M_P=0$. Thus $P\in\Supp(M)$. Furthermore, because $P_P\in\Ass(\Hom(M_P,N_P))$, there is a non-zero map $f:M_P\rightarrow N_P$ (that is, $\im(f)\neq\{\frac{0}{1}\}$) such that $P_P\cdot f=0$, i.e. the zero map. Thus, choosing any $\frac{n}{u}\in\im(f)\subseteq N_P, \frac{n}{u}\neq\frac{0}{1}$, we have that $P_P\cdot\frac{n}{u}=0$ and thus $P_P\subseteq\Ann(\frac{n}{u})$. But $\frac{n}{u}\neq\frac{0}{1}$, so that $\Ann(\frac{n}{u})\neq R_P$, and because $P_P$ is maximal in $R_P$, we must have $P_P=\Ann(\frac{n}{u})$. Thus $P_P\in\Ass(N_P)$, and we know that $\Ass(N_P)=\{Q_P:Q\in\Ass(N),Q\subseteq P\}$. Because we know that the bijection between primes of $R$ in contained in $P$ and primes of $R_P$ is given by $Q\mapsto Q_P$, we have that $P\in\Ass(N)$. Thus if $P\in\Ass(\Hom(M,N))$, then $P\in\Supp(M)\cap\Ass(N)$.      \\

Conversely, $\ldots$ I got stuck.\\ 

\num{5.} The proof is by induction. First, note that $P^{(1)}=\{r\in R: sr\in P\text{ for some }s\notin P\}$. But $sr\in P$, $s\notin P$ implies $r\in P$, so that $P^{(1)}=\{r\in P\}=P$, a prime ideal. If $ab\in P$, then because $P$ is prime either $a\in P$ or $b^1\in P$, so that $P^{(1)}$ is primary. Now suppose that $P^{(n-1)}$ is primary, and note that $P^{(n)}\subseteq P^{(n-1)}$ (because any $r\in P^{(n)}$ has $sr\in P^n\subset P^{n-1}$ for $s\notin P$). If $ab\in P^{(n)}$, then $ab\in P^{(n-1)}$ and thus either $a\in P^{(n-1)}$ or $b^m\in P^{(n-1)}$ for some $r$. That is, either there is some $s\notin P$ with $sa\in P^{n-1}$, or there is some $t\notin P$ with $tb^m\in P^{n-1}$. But if $sa\in P^{n-1}\subset P$ and $s\notin P$, then $a\in P$; and if $tb^m\in P^{n-1}\subset P$ and $t\notin P$ with $b\notin P$, then $tb^m\notin P\supset P^{n-1}$, contradiction, so that we must have $b\in P$. Thus we either have $sa^2\in P^n$ with $s\notin P$, or $tb^{m+1}\in P^n$ with $t\notin P$, so that $a^2\in P^{(n)}$ or $b^{m+1}\in P^{(n)}$, which suffices to show that $P^{(n)}$ is primary because the definition of primary is symmetric.       \\

Furthermore, suppose $P$ is prime (this includes the case of $P$ maximal). If $ab\in P^n\subset P$, then either $a\in P$ or $b\in P$, so that if $a\notin P$ then $b\in P$ and hence $b^n\in P^n$. Thus $P^n$ is primary.\\

\num{6.}        \\

\num{7.}        \\

\num{Bonus.}         









\end{document}
1}\in P^n$ with $t\notin P$, so that $a^2\in P^{(n)}$ or $b^{m+1}\in P^{(n)}$, which suffices to show that $P^{(n)}$ is primary because the definition of primary is symmetric.       \\

Furthermore, sup