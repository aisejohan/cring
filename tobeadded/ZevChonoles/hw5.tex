\documentclass[11pt]{article}
\usepackage{amsmath}
\usepackage{amssymb}
\usepackage{amsfonts}
\usepackage{amsthm}
\usepackage{array}
\usepackage[pdftex]{graphicx}
\input xy
\xyoption{all}
%% This command inserts \noindent and makes the input bold.
\newcommand{\num}[1]{\noindent \textbf{#1}}
\newcommand{\im}{\operatorname{im}}
\newcommand{\id}{\operatorname{id}}
\newcommand{\Hom}{\operatorname{Hom}}
\newcommand{\Z}{\mathbb{Z}}
\newcommand{\Q}{\mathbb{Q}}
\newcommand{\R}{\mathbb{R}}
\newcommand{\C}{\mathbb{C}}
%% This custom type produces a column of the specified width whose contents
are centered.
\newcolumntype{C}[1]{>{\centering\hspace{0pt}}p{#1}}
\newtheorem{theorem}{Theorem}
\newtheorem{lemma}{Lemma}[theorem]
\theoremstyle{definition}
\newtheorem{axiom}{Axiom}[section]
\newtheorem{definition}{Definition}
\newtheorem*{remark}{Remark}
\textwidth 6.9in
\textheight 9.4in
\oddsidemargin -0.2in
\topmargin -0.7in
\pagestyle{empty}
\begin{document}
\noindent Zev Chonoles \hfill \today\\[-0.4in]
\begin{center}
\noindent \underline{MATH 2520 - Assignment 5}
\end{center}

\num{1.} As we showed in class, because $0\rightarrow P\rightarrow
R\rightarrow R/P\rightarrow0$ is exact and localizations are flat, we have that
$0\rightarrow P\otimes R[U^{-1}]\rightarrow R\otimes R[U^{-1}]\rightarrow
(R/P)\otimes R[U^{-1}]\rightarrow0$ is exact, and because $M\otimes
R[U^{-1}]\simeq M[U^{-1}]$, we have that $0\rightarrow P[U^{-1}]\rightarrow
R[U^{-1}]\rightarrow (R/P)[U^{-1}]\rightarrow0$ is exact. This implies that
$R[U^{-1}]/P[U^{-1}]\simeq (R/P)[U^{-1}]$. In particular, if $U=R-P$, then
we have that $R_P/P_P\simeq (R/P)_P$.\\

\noindent Let $U=R-P$. Define the map $f:(R/P)_P\rightarrow\text{Frac}(R/P)$
by $f(\frac{a+P}{u})=\frac{a+P}{u+P}$ (note that $u\in U$). This
is well-defined because if $\frac{a+P}{u}=\frac{b+P}{v}$, then there
is a $w\in U$ such that $w\cdot(v\cdot(a+P)-u\cdot(b+P))=0$, which
is equivalent to $(w+P)((v+P)(a+P)-(u+P)(b+P))=0$. Because $R/P$ is an
integral domain and $w\in U$, we have that $(v+P)(a+P)-(u+P)(b+P)=0$, so
that $f(\frac{a+P}{u})=\frac{a+P}{u+P}=\frac{b+P}{v+P}=f(\frac{b+P}{v})$. The
map $f$ is a homomorphism of rings because 
\small
\[\hskip-0.2in
f\left(\frac{a+P}{u}\frac{b+P}{v}\right)=f\left(\frac{(a+P)(b+P)}{uv}\right)=\frac{(a+P)(b+P)}{uv+P}=\frac{(a+P)(b+P)}{(u+P)(v+P)}=\frac{a+P}{u+P}\frac{b+P}{v+P}=f\left(\frac{a+P}{u}\right)f\left(\frac{b+P}{v}\right)\]

\normalsize
and $f$ similarly respects addition. The map $f$ is injective
because if $f(\frac{a+P}{u})=\frac{a+P}{u+P}=\frac{0+P}{1+P}$,
then $(a+P)(1+P)-(u+P)(0+P)=a+P=0+P$, so $a\in P$,
i.e. $\frac{a+P}{u}=\frac{0+P}{1}$. Finally, $f$ is surjective because
any $\frac{a+P}{u+P}\in\text{Frac}(R/P)$ is mapped to by $\frac{a+P}{u}\in
(R/P)_P$. Thus $R_P/P_P\simeq (R/P)_P\simeq\text{Frac}(R/P)$. One consequence
of this is that $R_P/P_P$ is isomorphic to $R/P$ itself iff $P$ is maximal. \\

%Recall that the $R$-algebra structure of $R/P$ is given by
$r\cdot(a+P)=ra+P=(r+P)(a+P)$. 

%Let $\phi:R\rightarrow R/P$ be the natural map, and note that the
$R$-module structure of $R/P$ is given by $r\cdot(a+P)=ra+P=(r+P)(a+P)$. We
claim that for any multiplicatively closed set $U\subset R$ which
is disjoint from $P$, $(R/P)[U^{-1}]\simeq (R/P)[\phi(U)^{-1}]$,
that is, $R/P$ as an $R$-module, localized at $U\subset R$, is
isomorphic to $R/P$ as an $R/P$-module, localized at $\phi(U)\subset
R/P$. The map $f:(R/P)[U^{-1}]\rightarrow(R/P)[\phi(U)^{-1}]$
defined by $f(\frac{a+P}{u})=\frac{a+P}{u+P}$ is well-defined,
because if $\frac{a+P}{u}=\frac{b+P}{v}$, then there is a $w\in U$
such that $w\cdot(v\cdot(a+P)-u\cdot(b+P))=0$, and by $R$-module
structure of $R/P$, this is $(w+P)((v+P)(a+P)-(u+P)(b+P))=0$, so that
$f(\frac{a+P}{u})=\frac{a+P}{u+P}=\frac{b+P}{v+P}=f(\frac{b+P}{v})$.
Furthermore, $f$ is injective because if
$f(\frac{a+P}{u})=\frac{a+P}{u+P}=\frac{0+P}{1+P}$, then there is a
$(w+P)\in\phi(U)$ such that $(w+P)((a+P)(1+P)-(u+P)(0+P))=wa+P=0+P$,
but $w\in U$, so that $w\notin P$ and thus $a\in P$,
i.e. $\frac{a+P}{w}=\frac{0+P}{1}$. Finally, $f$ is surjective because
$\frac{a+P}{u+P}\in(R/P)[\phi(U)^{-1}]$ is mapped to by $\frac{a+P}{u}$. \\

%\noindent Thus, for $U\subset R$ disjoint from $P$,
we have $(R/P)[U^{-1}]\simeq(R/P)[\phi(U)^{-1}]$
in the sense described above. In particular,
$(R/P)_P=(R/P)[(R-P)^{-1}]\simeq(R/P)[\phi(R-P)^{-1}]=(R/P)[(R/P-\{0\})^{-1}]=(R/P)_{(0)}=\text{Frac}(R/P)$,
the fraction field of the integral domain $R/P$ (recall that $0$ is prime
in an integral domain). Thus, we can say for example that $(R/P)_P$ is
isomorphic to $R/P$ itself iff $P$ is maximal.     \\

\num{2.} Suppose $\frac{r}{u}\in N(R[U^{-1}])$, that is, there is some $n$ for
which $\left(\frac{r}{u}\right)^n=\frac{0}{1}$. Then there is a $v\in U$ such
that $v(r^n\cdot1-0\cdot u^n)=0$, so that $vr^n=0$. Thus $(vr)^n=0$, so that
$vr\in N(R)$. Thus $\frac{r}{u}=vr\cdot\frac{1}{uv}\in N(R)^e$. Conversely,
if $\frac{r}{u}\in N(R)^e$, then by the definition of the extension of an
ideal, $\frac{r}{u}$ is an $R[U^{-1}]$-linear combination of finitely many
nilpotent elements and hence is nilpotent itself.     \\

\num{3.} Of course, $R$ contains no nilpotent elements iff $N(R)=0$. We know
that for any $R$-module $M$, $M=0$ iff $M_P=0$ for all prime ideals $P\subset
R$, so that $N(R)=0$ iff $N(R)_P=0$ for all prime ideals $P$. But we proved in
Problem 2 that $N(R)_P=N(R_P)$ (because $N(R)^e=N(R)_P$ for this particular
problem), and $R_P$ has no nilpotent elements iff $N(R_P)=0$, so that $R$
has no nilpotent elements iff $R_P$ has no nilpotent elements for all prime
ideals $P$.     \\

\num{4.} We know that there is an inclusion-preserving bijection between
the prime ideals of $R_P$ and the prime ideals of $R$ disjoint from $R-P$,
i.e. contained in $P$. But $P$ is a minimal prime, so the only prime
contained in $P$ is $P$ itself, so the only prime of $R_P$ is $P_P$. Thus
$N(R_P)=\bigcap_{\text{prime } Q\subset R_P} Q = P_P$, so $P_P$ consists of
the nilpotent elements of $R_P$. But because we assumed $R$ has no nilpotent
elements, we have that $R_P$ has no nilpotent elements by Problem 3. Thus
$P_P$ is in fact the 0 ideal, and being $R_P$'s only prime ideal, it is
$R_P$'s only maximal ideal, so that the 0 ideal is maximal and hence $R_P$
is a field.     \\

\num{5.} If $R$ is an integral domain, then for any prime $P$, in $R_P$
we have that $\frac{r}{u}\cdot\frac{s}{v}=\frac{rs}{uv}=\frac{0}{1}$ when
there is a $w\in R-P$ such that $w(rs\cdot1-uv\cdot0)=wrs=0$. Because
$0\in P$ for all primes $P$, we must have that $w\neq0$, and because $R$
is an integral domain, we must have that $rs=0$, so that $r=0$ or $s=0$,
so that $\frac{r}{u}=0$ or $\frac{s}{v}=0$, so that $R_P$ is an integral
domain for all primes $P$. However, the converse is false - a counterexample
is $R=\mathbb{Z}/6\mathbb{Z}$, which has no nilpotent elements and whose only
prime ideals are $P=2\mathbb{Z}/6\mathbb{Z}$ and $Q=3\mathbb{Z}/6\mathbb{Z}$,
which are minimal primes by inspection, so that by Problem 4, $R_P$ and $R_Q$
are both fields and hence integral domains, while $R$ itself is not.    \\

\num{6.} We have the inclusion $B\hookrightarrow A$ and the quotient
$A\rightarrow A/C$. The kernel of their composition is obviously $B\cap C$. Let
$D$ be $B$'s image in $A/C$ under this composition, namely $(B+C)/C$. Then
$B/(B\cap C)\simeq D$ by the first isomorphism theorem. It is clear that
localization respects quotients, intersections, and sums of submodules
(we have shown all of this in class). Now, because $B_P\subseteq C_P$ for
all maximal ideals $P$, we have that $D_P=(B_P+C_P)/C_P=C_P/C_P\simeq0$
for all maximal ideals $P$. Thus $D\simeq 0$, by a problem on a previous
homework. But this implies that $B/(B\cap C)\simeq 0$, so that $B\cap C=
B$, so that $B\subseteq C$.   \\

\num{7.} Let $F$ be the free $R$-module on $\{e_1,\ldots,e_n\}$ (so that
$F\simeq R^n$), and suppose $\{y_1,\ldots,y_n\}$ generates $F$. Define
$\phi:F\rightarrow F$ by $\phi(e_i)=y_i$ (this specifies $\phi$ uniquely
by the universal property of free modules). By assumption, the $y_i$
generate $F$, so $\phi$ is surjective. It is clear that there is a linear
dependence among the $y_i$ iff $\ker(\phi)\neq0$; thus, we want to show that
$\ker(\phi)=0$, which is equivalent to showing $\phi$ is injective. We know
that $\phi$ is injective iff $\phi_P$ is injective for all maximal ideals
$P$, so consider the exact sequence $0\rightarrow \ker(\phi)_P\rightarrow
F_P\rightarrow F_P\rightarrow 0$ obtained by localizing the exact sequence
$0\rightarrow\ker(\phi)\rightarrow F\rightarrow F\rightarrow0$. We have that
$F_P\simeq R^n\otimes R_P\simeq(R_P)^n$, and hence we have an exact sequence
$0\rightarrow\ker(\phi)_P\rightarrow (R_P)^n\rightarrow (R_P)^n\rightarrow
0$ of $R_P$-modules for each maximal ideal $P$. It's clear that the next
step is to reduce this to the case of vector spaces (perhaps by tensoring
with the field $k=R_P/P_P$?) because $k^n\rightarrow k^n$ surjective implies
injective and hence $\ker(\phi)_P$ would be 0 for all $P$, and thus 0 itself -
but I couldn't determine why $k$ should be flat, and thus I couldn't justify
tensoring with it. \\

\num{8.} If $f\in\bigcap_{a\in k}(x-a)$, then $f(a)=0$ for all $a\in k$. But
any non-zero polynomial has only finitely many roots, so $f=0$. Thus
$(\bigcap_{a\in k}(x-a))[U^{-1}]=0$. On the other hand, $x-a\in U$ so
$\frac{1}{1}=\frac{x-a}{x-a}\in(x-a)[U^{-1}]$ and thus $(x-a)[U^{-1}]=k[x]$,
so $\bigcap_{a\in k}(x-a)[U^{-1}]=k[x]$. Clearly $0\neq k[x]$.     \\

\num{9a.} If the sequence $0\rightarrow
A\stackrel{\alpha}{\rightarrow}B\stackrel{\beta}{\rightarrow}C\rightarrow0$
is split, there is a map $\gamma:C\rightarrow B$ such that
$\beta\circ\gamma=\id_C$. Then for any $f\in\Hom(C,C)$, we have $f=\id_C\circ
f=(\beta\circ\gamma)\circ f=\beta\circ(\gamma\circ f)$, so that $f$ is mapped
to by $\gamma\circ f\in\Hom(C,B)$. Conversely, suppose each $f\in\Hom(C,C)$
is mapped to by some $g\in\Hom(C,B)$. Then in particular, $\id_C$ is mapped
to by some $\gamma\in\Hom(C,B)$, that is, $\beta\circ \gamma=\id_C$. Thus
the sequence splits.    \\

\num{9b.} Suppose $C$ is finitely presented and the sequence $0\rightarrow
A_P\stackrel{\alpha_P}{\rightarrow}B_P\stackrel{\beta_P}{\rightarrow}C_P\rightarrow0$
splits for all maximal ideals $P$. By Problem 9a, this is the case iff
$\Hom(C_P,B_P)\rightarrow\Hom(C_P,C_P)$ induced by $\beta_P$ is an epimorphism
for all $P$. By Proposition 2.10, $\Hom(C_P,B_P)\simeq R_P\otimes_R\Hom(C,B)$
and $\Hom(C_P,C_P)\simeq R_P\otimes_R\Hom(C,C)$ because $C$ is finitely
presented and $R_P$ is flat (since all localizations are flat). Thus,
$\beta_P$ also induces an epimorphism $R_P\otimes_R\Hom(C,B)\rightarrow
R_P\otimes_R\Hom(C,C)$ for all maximal ideals $P$. But we proved in class
that a map is an epimorphism iff each of its localizations is an epimorphism -
thus, the map $\beta$ is itself an epimorphism because each $\beta_P$ is. By
Problem 9a again, this is equivalent to the original sequence being split.










\end{document}
C)$ because $C$ is finitely presented and $R_P$ is flat (since all
localizations are flat).
