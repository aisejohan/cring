\documentclass[11pt]{article}
\usepackage{amsmath}
\usepackage{amssymb}
\usepackage{amsfonts}
\usepackage{amsthm}
\usepackage{array}
\usepackage[pdftex]{graphicx}
\input xy
\xyoption{all}
%% This command inserts \noindent and makes the input bold.
\newcommand{\num}[1]{\noindent \textbf{#1}}
\newcommand{\im}{\operatorname{im}}
\newcommand{\id}{\operatorname{id}}
\newcommand{\Hom}{\operatorname{Hom}}
\newcommand{\Z}{\mathbb{Z}}
\newcommand{\Q}{\mathbb{Q}}
\newcommand{\R}{\mathbb{R}}
\newcommand{\C}{\mathbb{C}}
%% This custom type produces a column of the specified width whose contents are centered.
\newcolumntype{C}[1]{>{\centering\hspace{0pt}}p{#1}}
\newtheorem{theorem}{Theorem}
\newtheorem{lemma}{Lemma}[theorem]
\theoremstyle{definition}
\newtheorem{axiom}{Axiom}[section]
\newtheorem{definition}{Definition}
\newtheorem*{remark}{Remark}
\textwidth 6.9in
\textheight 9.4in
\oddsidemargin -0.2in
\topmargin -0.7in
\pagestyle{empty}
\begin{document}
\noindent Zev Chonoles \hfill \today\\[-0.4in]
\begin{center}
\noindent \underline{MATH 2520 - Assignment 4}
\end{center}

\num{Problem 1.} Let $d=\gcd(m,n)$. We will show that $(\Z/m\Z)\otimes(\Z/n\Z)\simeq(\Z/d\Z)$, and thus in particular if $m$ and $n$ are relatively prime, then $(\Z/m\Z)\otimes(\Z/n\Z)\simeq(0)$. First, note that any $a\otimes b\in(\Z/m\Z)\otimes(\Z/n\Z)$ can be written as $ab(1\otimes 1)$, so that $(\Z/m\Z)\otimes(\Z/n\Z)$ is generated by $1\otimes 1$ and hence is a cyclic group. We know from elementary number theory that $d=xm+yn$ for some $x,y\in\Z$. We have $m(1\otimes 1)=m\otimes 1=0\otimes 1=0$ and $n(1\otimes 1)=1\otimes n=1\otimes0=0$. Thus $d(1\otimes 1)=(xm+yn)(1\otimes 1)=0$, so that $1\otimes1$ has order dividing $d$.\\

\noindent Conversely, consider the map $f:(\Z/m\Z)\times(\Z/n\Z)\rightarrow(\Z/d\Z)$ defined by $f(a+m\Z,b+n\Z)=ab+d\Z$. This is well-defined, since if $a'+m\Z=a+m\Z$ and $b'+n\Z=b+n\Z$ then $a'=a+mr$ and $b'=b+ns$ for some $r,s$ and thus $a'b'+d\Z=ab+(mrb+nsa+mnrs)+d\Z=ab+d\Z$ (since $d=\gcd(m,n)$ divides $m$ and $n$). This is obviously bilinear, and hence induces a map $\tilde{f}:(\Z/m\Z)\otimes(\Z/n\Z)\rightarrow(\mathbb{Z}/d\mathbb{Z})$, which has $\tilde{f}(1\otimes1)=1+d\mathbb{Z}$. But the order of $1+d\mathbb{Z}$ in $\mathbb{Z}/d\mathbb{Z}$ is $d$, so that the order of $1\otimes1$ in $(\Z/m\Z)\otimes(\Z/n\Z)$ must be at least $d$. Thus $1\otimes1$ is in fact of order $d$, and the map $\tilde{f}$ is an isomorphism between cyclic groups of order $d$. \\

\num{Problem 2.} Any finitely generated abelian group $A$ has $A\simeq (\mathbb{Z}/p_1^{a_1}\mathbb{Z})\oplus\cdots\oplus(\mathbb{Z}/p_n^{a_n}\mathbb{Z})\oplus\mathbb{Z}^r$, for not- necessarily-distinct primes $p_1,\ldots,p_n$ and $a_1,\ldots,a_n,r\geq0$. In class, we proved $\otimes$ distributes over $\oplus$, so
\[A\otimes A\simeq(A\otimes(\Z/p_1^{a_1}\Z))\oplus\cdots\oplus(A\otimes(\Z/p_n^{a_n}\Z))\oplus(A\otimes\Z)^r\]

\noindent where we have used that $\mathbb{Z}^r=\prod_{i=1}^r\mathbb{Z}$ is isomorphic to $\bigoplus_{i=1}^r\mathbb{Z}$. Again by distributivity,
\[A\otimes(\Z/p_i^{a_i}\Z)\simeq ((\mathbb{Z}/p_1^{a_1}\mathbb{Z})\otimes(\mathbb{Z}/p_i^{a_i}\mathbb{Z}))\oplus\cdots\oplus((\mathbb{Z}/p_n^{a_n}\mathbb{Z})\otimes(\mathbb{Z}/p_i^{a_i}\mathbb{Z}))\oplus(\mathbb{Z}\otimes(\mathbb{Z}/p_i^{a_i}\mathbb{Z}))^r\]
which by Problem 1 and the fact that $M\otimes_RR\simeq R\otimes_RM\simeq M$ for any $R$-module $M$, becomes \[A\otimes(\Z/p_i^{a_i}\Z)\simeq(\mathbb{Z}/p_i^{a_i}\mathbb{Z})^{r+1}\]
Furthermore, $A\otimes\Z\simeq A\simeq(\mathbb{Z}/p_1^{a_1}\mathbb{Z})\oplus\cdots\oplus(\mathbb{Z}/p_n^{a_n}\mathbb{Z})\oplus\mathbb{Z}^r$. Thus, 
\[A\otimes A\simeq (\Z/p_1^{a_1}\Z)^{r+1}\oplus\cdots\oplus(\Z/p_n^{a_n}\Z)^{r+1}\oplus A^r\simeq(\Z/p_1^{a_1}\Z)^{2r+1}\oplus\cdots\oplus(\Z/p_n^{a_n}\Z)^{2r+1}\oplus \Z^{r^2}\]

\num{Problem 3.} By the associativity of the tensor product, we have $(M\otimes_R N)\otimes_R(R/I)\simeq M\otimes_R(N\otimes_R (R/I))$. We have the isomorphism $N\otimes_R(R/I)\simeq N/IN$ (as mentioned in the problem), so $M\otimes_R(N\otimes_R (R/I))\simeq M\otimes_R(N/IN)$. But as an $R/I$-module, we of course have $(R/I)\otimes_{R/I}(N/IN)\simeq N/IN$, so that 
\[M\otimes_R(N/IN)\simeq M\otimes_R((R/I)\otimes_{R/I}(N/IN))\] 
Because $R/I$ is also an $R$-module, by the \textit{general} associativity of the tensor product we have
\[M\otimes_R((R/I)\otimes_{R/I}(N/IN))\simeq (M\otimes_R (R/I))\otimes_{R/I}(N/IN)\]
which, after putting together all of our isomorphisms, becomes
\[(M\otimes_R N)\otimes_R(R/I)\simeq (M/IM)\otimes_{R/I}(N/IN)\]

\num{Problem 4.} Let $m$ be the maximal ideal of the local ring $R$. Obviously, $R/m$ is a field and the Jacobson radical $J(R)=m$. By Problem 3, 
\[(M\otimes_RN)\otimes_R(R/m)\simeq (M/mM)\otimes_{R/m}(N/mN)\]
If $M\otimes_RN\simeq0$, then $(M\otimes_RN)\otimes_R(R/m)\simeq0\otimes_R(R/m)\simeq 0$ and thus $(M/mM)\otimes_{R/m}(N/mN)\simeq0$. But because $M$ and $N$ are finitely generated $R$-modules, $M/mM$ and $N/mN$ are finitely generated $R/m$-modules - i.e., finite dimensional $R/m$-vector spaces. By standard results on vector spaces, this implies $M/mM\simeq (R/m)^a\simeq\bigoplus_{i=1}^a(R/m)$ and $N/mN\simeq (R/m)^b\simeq\bigoplus_{i=1}^b(R/m)$ for some $a,b\geq0$. Thus 
\[(M/mM)\otimes_{R/m}(N/mN)\simeq \bigoplus_{i=1}^a(R/m)\otimes_{R/m}\bigoplus_{i=1}^b(R/m)\]
and because $\otimes$ distributes over $\oplus$, we have $(M/mM)\otimes_{R/m}(N/mN)\simeq (R/m)^{ab}$. Clearly, $(R/m)^{ab}\simeq0$ iff $a=0$, $b=0$, or both. Thus, $M/mM\simeq0$, $N/mN\simeq0$, or both. Thus $mM=J(R)M=M$, $mN=J(R)N=N$, or both. By Nakayama's lemma, this implies that $M=0$, $N=0$, or both.   \\

\num{Problem 5.} We first prove the statement for finitely generated free modules $E\simeq R^k\simeq\bigoplus_{i=1}^k R$. We have that $M\otimes E\simeq M^k$ and $N\otimes E\simeq N^k$ by the distributivity of $\otimes$ over $\oplus$ - the explicit isomorphisms are $f:M\otimes E\rightarrow M^k$ with $f(m\otimes(r_1a_1+\cdots+r_ka_k))=(r_1m,\ldots,r_km)$ and $g:N\otimes E\rightarrow N^k$ with $g(n\otimes(r_1a_1+\cdots+r_ka_k))=(r_1n,\ldots,r_1n)$. Let $\phi:M\rightarrow N$ be injective. We have the induced map $(\phi\otimes\id):M\otimes E\rightarrow N\otimes E$ with $(\phi\otimes\id)(m\otimes(r_1a_1+\cdots+r_ka_k))=(n\otimes(r_1a_1+\cdots+r_ka_k))$, which via the isomorphisms $f$ and $g$ further induces $g\circ(\phi\otimes\id)\circ f^{-1}:M^k\rightarrow N^k$, which has $(m_1,\ldots,m_k)\mapsto(\phi(m_1),\ldots,\phi(m_k))$. Because $\phi$ is injective, we must have that $g\circ(\phi\otimes\id)\circ f^{-1}$ is injective, and because $f$ and $g$ are isomorphisms, we must have that $(\phi\otimes\id)$ is injective.\\

\noindent Now let $E$ be any free module, say on $\{a_i\}$. Because any element $x\in M\otimes E$ is in fact a finite sum $x=\sum_{i=1}^k r_i(m_i\otimes a_i)$, $x$ is contained in the submodule $M\otimes B\subset M\otimes E$ where $B$ is free on $\{a_1,\ldots,a_k\}$. Thus, if $(\phi\otimes\id_E):M\otimes E\rightarrow N\otimes E$ has $(\phi\otimes\id_E)(x)=0$, then the restriction $(\phi\otimes\id_E)\vert_{M\otimes B}=(\phi\otimes\id_B):M\otimes B\rightarrow N\otimes B$ also has $(\phi\otimes\id_B)(x)=0$, but $B$ is finitely generated and free, so by our previous result that $(\phi\otimes\id_B)$ injective, we must have that $x=0\in M\otimes B$. But of course, then $x=0\in M\otimes E$, so that $(\phi\otimes\id_E)$ is injective.  \\

\num{Problem 6.} Clearly, $n\mathbb{Z}\stackrel{f}{\hookrightarrow}\mathbb{Z}$ is a monomorphism, but $(n\mathbb{Z})\otimes(\mathbb{Z}/n\mathbb{Z})\stackrel{f\otimes\text{id}}{\rightarrow}(\mathbb{Z})\otimes(\mathbb{Z}/n\mathbb{Z})$ is actually the zero map because $(f\otimes\id)(an\otimes (b+n\mathbb{Z}))=f(an)\otimes\id(b+n\mathbb{Z})=an\otimes(b+n\mathbb{Z})=a\otimes (bn+n\mathbb{Z})=a\otimes 0=0$, and the $an\otimes(b+n\mathbb{Z})$ generate $n\mathbb{Z}\otimes(\mathbb{Z}/n\mathbb{Z})$ so that $f\otimes\id$ must be 0 on all elements of $(n\mathbb{Z})\otimes(\mathbb{Z}/n\mathbb{Z})$. The zero map is not a monomorphism, so $\mathbb{Z}/n\mathbb{Z}$ is not flat.   \\

\num{Problem 7.} Define $\rho':M^*\times N\rightarrow\Hom_R(M,N)$ by $\rho'(f,n)(m)=f(m)n$ (note that $f(m)\in R$, and the multiplication $f(m)n$ is that between an element of $R$ and an element of $N$). This is bilinear,
\[\rho'(af+bg,n)(m)=(af+bg)(m)n=(af(m)+bg(m))n=af(m)n+bg(m)n=a\rho'(f,n)(m)+b\rho'(g,n)(m)\]        
\[\rho'(f,an_1+bn_2)(m)=f(m)(an_1+bn_2)=af(m)n_1+bf(m)n_2=a\rho'(f,n_1)(m)+b\rho'(f,n_2)(m)\]
so it induces a map $\rho:M^*\otimes N \rightarrow \Hom(M,N)$ with $\rho(f\otimes n)(m)=f(m)n$. This homomorphism is unique since the $f\otimes n$ generate $M^*\otimes N$. \\

\noindent Suppose $M$ is free on the set $\{a_1,\ldots,a_k\}$. Then $M^*=\Hom(M,R)$ is free on the set $\{f_i:M\rightarrow R,$ $ f_i(r_1a_1+\cdots+r_ka_k)=r_i\}$, because there are clearly no relations among the $f_i$ and because any $f:M\rightarrow R$ has $f=f(a_1)f_1+\cdots+f(a_n)f_n$. Also note that any element $\sum h_j\otimes p_j \in M^*\otimes N$ can be written in the form $\sum_{i=1}^k f_i\otimes n_i$, by setting $n_i=\sum h_j(a_i)p_j$, and \textit{that this is unique} because the $f_i$ are a basis for $M^*$.\\

\noindent We claim that the map $\psi:\Hom_R(M,N)\rightarrow M^*\otimes N$ defined by $\psi(g)=\sum_{i=1}^k f_i\otimes g(a_i)$ is inverse to $\rho$. Given any $\sum_{i=1}^k f_i\otimes n_i\in M^*\otimes N$, we have 
\[\rho(\sum_{i=1}^k f_i\otimes n_i)(a_j)=\sum_{i=1}^k\rho(f_i\otimes n_i)(a_j)=\sum_{i=1}^kf_i(a_j)n_i=n_j\]
 Thus, $\rho(\sum_{i=1}^k f_i\otimes n_i)(a_i)=n_i$, and thus $\psi(\rho(\sum_{i=1}^k f_i\otimes n_i))=\sum_{i=1}^k f_i\otimes n_i$. Thus, $\psi\circ\rho=\id_{M^*\otimes N}$.\\
 
 \noindent Conversely, recall that for $g:M\rightarrow N\in\Hom_R(M,N)$, we defined $\psi(g)=\sum_{i=1}^k f_i\otimes g(a_i)$. Thus, 
\[\rho(\psi(g))(a_j)=\rho(\sum_{i=1}^k f_i\otimes g(a_i))(a_j)=\sum_{i=1}^k\rho(f_i\otimes g(a_i))(a_j)=\sum_{i=1}^k f_i(a_j)g(a_i)=g(a_j)\]
and because $\rho(\psi(g))$ agrees with $g$ on the $a_i$, it is the same element of $\Hom_R(M,N)$ because the $a_i$ generate $M$. Thus, $\rho\circ\psi=\id_{\Hom_R(M,N)}$.\\

\noindent Thus, $\rho$ is an isomorphism.



\end{document}
ndent Conversely, recall that for $g:M\rightarrow N\in\Hom_R(M,N)$, we defined $\psi(g)=\sum_{i=1}^k f_i\otimes g(a_i)$. Thus, 
\[\rho(\psi(g))(a_j)=\rho(\sum_{i=1}^k f_i\otimes g(a_i))(a_j)=\sum_{i=1}^k\rho(f_i\otimes g(a_i))(a_j)=\sum_{i=1}^k f_i(a_j)g(a_i)=g(a_j)\]
and because $\rho(\psi(g))$ agrees with $g$ on the $a_i$, it is the same element of $\Hom_R(M,N)$ because the $a_i$ generate $M$. Thu