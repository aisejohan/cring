\documentclass[11pt]{article}
\usepackage{amsmath}
\usepackage{amssymb}
\usepackage{amsfonts}
\usepackage{amsthm}
\usepackage{array}
\usepackage[pdftex]{graphicx}
\input xy
\xyoption{all}
%% This command inserts \noindent and makes the input bold.
\newcommand{\num}[1]{\noindent \textbf{#1}}
%% This custom type produces a column of the specified width whose contents
are centered.
\newcolumntype{C}[1]{>{\centering\hspace{0pt}}p{#1}}
\newtheorem{theorem}{Theorem}
\newtheorem{lemma}{Lemma}[theorem]
\theoremstyle{definition}
\newtheorem{axiom}{Axiom}[section]
\newtheorem{definition}{Definition}
\newtheorem*{remark}{Remark}
\textwidth 6.9in
\textheight 9.4in
\oddsidemargin -0.2in
\topmargin -0.7in
\pagestyle{empty}
\begin{document}
\noindent Zev Chonoles \hfill \today\\[-0.4in]
\begin{center}
\noindent \underline{MATH 2520 - Assignment 3}
\end{center}

\num{Problem 1.} Suppose $a\in J(R)$. If $1+ra$ is not a unit for some $r\in
R$, then $(1+ra)\subsetneq R$, so that $(1+ra)\subseteq M$ for some maximal
ideal $M$, so that $1+ra\in M$. But by hypothesis, $a\in J(R)\subseteq M$,
so that $1\in M\subsetneq R$; contradiction. Now suppose $a\notin J(R)$,
i.e. suppose there is a maximal ideal $M$ with $a\notin M$. Then clearly
$(a)+M=R$, so that there is an $m\in M$ and $r\in R$ with $ra+m=1$. Thus
$1+(-r)a=m\in M$, and hence cannot be a unit.       \\

\num{Problem 2.} Because $IM=M$, we have $m_t=a_1m_1+\cdots+a_tm_t$ for some
$a_i\in I\subseteq J(R)$. By Problem 1, we have that $1-a_t$ is a unit,
say with inverse $b$. Then $(1-a_t)m_t=a_1m_1+\cdots+a_{t-1}m_{t-1}$ and
so $m_t=(a_1b)m_1+\cdots+(a_{t-1}b)m_{t-1}$, but then $M$ is generated by
$m_1,\ldots,m_{t-1}$, contradicting the minimality of $\{m_1,\ldots,m_t\}$
as a generating set.       \\
\num{Problem 5.}\\

\num{Problem 6.} Let $F$ be the free $R$-module on $\{a_1,\ldots,a_n\}$. If
$\phi:M\rightarrow F$ is an epimorphism, we have that $\phi^{-1}(a_i)$
is non-empty for each $i$. Choose some $m_i\in\phi^{-1}(a_i)$ for each
$i$, and define $\mu:F\rightarrow M$ by $\mu(a_i)=m_i$ (this defines
it on all of $F$ by the universal property of free modules). Then
$(\phi\circ\mu)(a_i)=\phi(m_i)=a_i=\text{id}(a_i)$, and any two maps which
agree on generators are the same map (again, by the universal property of
free modules). Thus $\phi\circ\mu=\text{id}$. Now, suppose that $m\in M$
is mapped to $\phi(m)=r_1a_1+\cdots+r_na_n\in F$. Then $\phi(m)=\phi(h)$
for $h=r_1\mu(a_1)+\cdots+r_n\mu(a_n)\in \text{im}(\mu)$, and this $h$
is determined uniquely by $m$. Because $\phi$ is a homomorphism we have
$\phi(m-h)=0$ and thus $m-h=k$ for some $k\in \text{ker}(\phi)$. Thus any
$m\in M$ has $m=h+k$ for a uniquely determined $h\in\text{im}(\mu)$ and
$k\in\text{ker}(\phi)$, and thus $M=\text{im}(\mu)\oplus\text{ker}(\phi)$.\\

\noindent Note that the map $f:M\rightarrow\text{ker}(\phi)$
with $f(h+k)=k$ is well-defined, and
$f((h_1+k_1)+(h_2+k_2))=f((h_1+h_2)+(k_1+k_2))=k_1+k_2=f(h_1+k_1)+f(h_2+k_2)$,
and  $f(r(h+k))=f(rh+rk)=rk=rf(h+k)$, and $f$ clearly has kernel
$\text{im}(\mu)$, and $f$ is surjective - all due to the uniqueness of the
representation of an element of $M$ as a sum of elements from im$(\mu)$
and ker$(\phi)$. Thus $M/\text{im}(\mu)\simeq \text{ker}(\phi)$, and thus,
if $M$ is finitely generated, then so is $\text{ker}(\phi)$ because it is
(isomorphic to) a quotient of a finitely generated module.\\

\num{Problem 7.} Let $F$ be the free $R$-module on $\{a_1,\ldots,a_n\}$,
let $M$ be any $R$-module, and suppose $F\otimes_R M=(0)$. If
$M\neq(0)$, then the map $\phi:F\times M\rightarrow M^n$ with
$\phi(r_1a_1+\cdots+r_na_n,m)=(r_1m,\ldots,r_nm)$ is well-defined
and bilinear, because of the uniqueness of the representation of
an element of $F$ as a linear combination of the $a_i$. Note that,
im$(\phi)\neq\{(0,\ldots,0)\}$ - for example, for any $m\in M$,
$m\neq 0$ (there is such an $m$ because we assumed $M\neq(0)$) we have
$\phi(a_1+\cdots+a_n,m)=(m,\ldots,m)\neq(0,\ldots,0)$. But there is no
homomorphism from $F\otimes_RM=(0)$ to $M^n$ which is not the zero map (by
the properties of homomorphisms), contradicting the universal property of
tensor products - that is, there can be no $\tilde{\phi}$ which makes the
diagram below commute for the specified $\phi$.\\

\begin{center}
$\xymatrix{
F\otimes_RM \ar@{.>}[dr]^{\tilde{\phi}} & {}\\
F\times M \ar[u]^{\lambda} \ar[r]^{\phi} & M^n}$\\
\end{center}   

\noindent Thus, we must have $M=(0)$. I also believe this argument holds
even when $F$ is free on an infinite set.










\end{document}
ro map (by the pr
