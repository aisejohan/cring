\documentclass[12pt, reqno]{amsart}

\usepackage{amsthm}
\usepackage{amsmath}
\usepackage{url}
\usepackage{fancyhdr}
\renewcommand{\thesection}{\arabic{section}}
\newtheorem{theorem}{Theorem}[section]

\usepackage{hyperref}

\newtheorem{lemma}[theorem]{Lemma}
\newtheorem{sublemma}[theorem]{Sublemma}
\newtheorem{corollary}[theorem]{Corollary}
\newtheorem{proposition}[theorem]{Proposition}
\theoremstyle{definition}
\newtheorem{definition}[theorem]{Definition}
\newtheorem*{remark}{Remark}
\newtheorem{example}[theorem]{Example}
\newtheorem{exercise}{ \sc Exercise}[chapter]
\newtheorem*{solution}{Solution}
\newtheorem*{question}{Question}
\newtheorem*{problem}{Problem}
\newtheorem*{dbend}{Dangerous bend}



\usepackage[top=1.3in, bottom=1.3in, left=1.5in, right=1.5in]{geometry}
\usepackage{amssymb}
\usepackage{amsfonts}
\usepackage{stackrel}
\usepackage{mathrsfs}
\usepackage{xy}
\usepackage{verbatim}


\newcommand{\whattosay}{}



\input xy
\xyoption{all}


\newcommand{\lecture}[1]{}
\newcommand{\rad}{\mathrm{rad}}
\newcommand{\im}{\mathrm{Im}}
\newcommand{\proj}{\mathrm{Proj}}
\renewcommand{\hom}{\mathrm{Hom}}
\newcommand{\id}{\mathrm{id}}
\providecommand{\cal}[1]{\mathcal{#1}}
\renewcommand{\cal}[1]{\mathcal{#1}}


%\swapnumbers

\renewcommand{\qedsymbol}{$\blacktriangle$}


\usepackage{geometry}
\geometry{lmargin = 1.0in, bmargin = 0.9in, tmargin = 1.0in, rmargin=1.0in}

\begin{document}

\title{Math 210A Problem Set 8}
\input{other/amsdata.tex}

\subsection*{1} (DF 10.5.14)
\subsection*{(a)}
\subsection*{(b)}


%This is very similar; the main difference is that we replace $\Hom_R (D,
%\cdot)$ by $\Hom_R (\cdot, D)$.

Suppose that 
$\xymatrix@1{0 \ar[r] & L \ar[r]^\psi & M \ar[r]^\f & N \ar[r] & 0}$ 
is a split short exact sequence.
We want to show that
$$\xymatrix@1{0 \ar[r] 
	& \Hom_R(N, D) \ar[r]^{\f'} 
	& \Hom_R(M, D) \ar[r]^{\psi'} 
	& \Hom_R(L, D) \ar[r] & 0}$$
is exact. 
%
Since $\Hom_R (\cdot, D)$ is a right-exact functor, we see that $\f'$ is
injective and $\psi'\circ \f' = 0$; we just need to show that $\psi'$ is
surjective.
%
Since we start with a split short exact sequence $M \cong L \oplus N$, we see
that $\Hom_R (L \oplus N, D) \cong \Hom_R(L, D) \oplus \Hom_R (N, D)$. 
Therefore, applying $\Hom_R (\cdot, D)$ does yield a split short exact sequence
$$\xymatrix@1{0 \ar[r] 
	& \Hom_R(N, D) \ar[r]^{\f'} 
	& \Hom_R(M, D) \ar[r]^{\psi'} 
	& \Hom_R(L, D) \ar[r] & 0}.$$

Now, assume instead that 
$$\xymatrix@1{0 \ar[r] 
	& \Hom_R(N, D) \ar[r]^{\f'} 
	& \Hom_R(M, D) \ar[r]^{\psi'} 
	& \Hom_R(L, D) \ar[r] & 0}$$
is a short exact sequence of abelian groups for all $R$-modules $D$.
%Then
%$$\xymatrix{0 \ar[r] & L \ar[r]^\psi & M \ar[r]^\f & N \ar[r] & 0}$$ 
%is a short exact sequence.  %%% why?
Set $D = M$, so we have
$$\xymatrix@1{0 \ar[r] 
	& \Hom_R(N, M) \ar[r]^{\f'} 
	& \Hom_R(M, M) \ar[r]^{\psi'} 
	& \Hom_R(L, M) \ar[r] & 0}.$$
Then $\Hom_R(M, M) = \Hom_R (N, M) \oplus \Hom_R (L, M)$. Suppose that 
$id_M = (f, g)$. Then $\f'(g) = g \circ \phi = id_M$.
This means that $g$ is a splitting homomorphism for the sequence 
$\xymatrix@1{0 \ar[r] & L \ar[r]^\psi & M \ar[r]^\f & N \ar[r] & 0}$, 
and therefore the sequence is a split short exact sequence.

%Here, $\f'$ is injective, so the identity map of $\Hom_R (M, M)$ lifts to a
%map $g \in \Hom_R (N, M)$ so that $\f \circ g = \f'(g) = id$.
%This means that $g$ is a splitting homomorphism for the sequence 
%$\xymatrix{0 \ar[r] & L \ar[r]^\psi & M \ar[r]^\f & N \ar[r] & 0}$, 
%and therefore the sequence is a split short exact sequence.

% not very happy with this.


\subsection*{2} (DF 10.5.16)
\subsection*{(a)}

Every $R$-module $M$ 
has an abelian group structure along with the action of $R$,
and the abelian group structure makes $M$ a $\ZZ$-module. Therefore, we can
apply the fact proved in class that every $\ZZ$-module is a submodule of an
injective $\ZZ$-module.
Therefore, $M$ is contained in an injective $\ZZ$-module $Q$. 

\subsection*{(b)}

A map $R \to M$ is in the set $\Hom_{\ZZ} (R, M)$ if it respects the abelian
group structure and the operation of addition. In order for it to be in the set
$\Hom_R (R, M)$, it must respect addition and in addition respect the group
action. This gives us the first inclusion 
$\Hom_R (R, M) \subseteq \Hom_{\ZZ} (R, M)$.

The second inclusion $\Hom_\ZZ (R, M) \subseteq \Hom_\ZZ (R, Q)$ is a corollary
of the fact that $M$ is contained in $Q$, so that every map $R \to M$ is also a
map $R \to Q$.

\subsection*{(c)}

%see 15(c). Do I need to prove this?
It is given to us that if $Q$ is an injective $\ZZ$-module then $\Hom_{\ZZ} (R,
Q)$ is an injective $R$-module. In addition, $\Hom_R (R, M) \cong M$.
Therefore, the result of part (b) yields that 
$M \subseteq \Hom_{\ZZ} (R, Q)$, which means that the $R$-module $M$ is
contained inside the injective $R$-module $\Hom_{\ZZ} (R, Q)$, which is what we
wanted to prove.

\subsection*{3} (DF 10.5.21)

Let $0 \to A \to B \to C \to 0$ be a short exact sequence.

Since $N$ is flat, we see that 
$0 \to N \otimes A \to N \otimes B \to N \otimes C \to 0$ 
is a short exact sequence.

Since $M$ is flat, this gives a short exact sequence
$$0 \to M \otimes (N \otimes A) \to M \otimes (N \otimes B) \to M \otimes (N
\otimes C) \to 0.$$ 
Since the tensor product is associative, this means that 
$$0 \to (M \otimes N) \otimes A \to (M \otimes N) \otimes B \to (M \otimes N)
\otimes C \to 0$$ 
is a short exact sequence, so that $M \otimes N$ is a flat
module.


\subsection*{4} (DF 17.1.31)
\subsection*{(a)}
Let $\f : A \to B$ be an element of $\Hom_R(A, B)$. 
%The standard embeddings 
%$A \to D^{-1}A$ and $B \to D^{-1}B$ via $a \mapsto a/1$ and $b \mapsto b/1$
This induces a map $\f' : D^{-1}A \to D^{-1}B$ defined via 
$\f'(a/d) = \f(a) / d$. Here, $\f' \in \Hom_{D^{-1}R} (D^{-1}A, D^{-1}B)$.

The map $\f \to \f'$ therefore represents a map
$D^{-1} \Hom_R(A, B) \to \Hom_{D^{-1}R}(D^{-1}A, D^{-1}B)$.

%%% fix me!

\subsection*{(b)}

For any $R$-module $N$, consider $\Hom_R (R^m,N)$. Every map $\f$ 
of $\Hom_R(R^m, N)$ satisfies
$\f(a_1, a_2, \dots, a_m) 
= a_1 \f(1, 0, \dots, 0) + a_2 \f(0, 1, \dots, 0) + \cdots + a_m \f(0, 0,
\dots, 1)$.
Therefore, the $m$ 
values of $\f(0, \dots, 0, 1, 0, \dots, 0)$ (where there is a 1
in the $m$-th position and 0's elsewhere) uniquely determine $\f$. They are
also completely independent, and each can take on any value of $N$, so 
therefore $\Hom_R (R^m, N) \cong N^m$. 

Taking localizations, we therefore have
$D^{-1}\Hom_R (R^m, N) \cong D^{-1} N^m = (D^{-1}N)^m$.


\subsection*{(c)}

Consider the diagram
\[ 
\xymatrix{
0 \ar[r] 
	& D^{-1} \Hom_R(M, N) \ar[r]^f \ar[d]^{h} 
	& D^{-1} \Hom_R(R^t, N) \ar[r]^g \ar[d]^{h'}
	& D^{-1} \Hom_R(R^s, N) \ar[d]^{h''} \\ 
0 \ar[r] 
	& \Hom_{D^{-1}R}(D^{-1}M, D^{-1}N) \ar[r]^{f'}
	& \Hom_{D^{-1}R}((D^{-1}R)^t, D^{-1}N) \ar[r]^{g'} 
	& \Hom_{D^{-1}R}((D^{-1}R)^s, D^{-1}N).
}
\] 

Since $M$ is finitely presented, $R^s \to R^t \to M \to 0$ is exact. Taking
$\Hom_R (\cdot, N)$ yields an exact sequence, and we proved earlier that
localization is exact, so therefore the first row is exact.
Tensoring the presentation with the flat module $D^{-1}R$ yields an exact
sequence. Taking $\Hom_{D^{-1}R}(\cdot, D^{-1}N)$ preserves exactness, 
so the second row is also exact.

%why does this commute?


\subsection*{(d)}

By the results of part (b), we obtain the isomorphisms
$$
\Hom_{D^{-1}R} ((D^{-1}R)^t, D^{-1}N) \cong (D^{-1}N)^t \qquad
\Hom_{D^{-1}R} ((D^{-1}R)^s, D^{-1}N) \cong (D^{-1}N)^s
$$
$$
D^{-1} \Hom_R(R^t, N) \cong (D^{-1}N)^t \qquad 
D^{-1} \Hom_R(R^s, N) \cong (D^{-1}N)^s, 
$$
so the preceding diagram has isomorphisms as the second two vertical maps:
\[ 
\xymatrix{
0 \ar[r] 
	& D^{-1} \Hom_R(M, N) \ar[r]^-f \ar[d]^{h} 
	& (D^{-1}N)^t \ar[r]^g \ar[d]^{\cong}
	& (D^{-1}N)^s \ar[d]^{\cong} \\ 
0 \ar[r] 
	& \Hom_{D^{-1}R}(D^{-1}M, D^{-1}N) \ar[r]^-{f'}
	& (D^{-1}N)^t \ar[r]^{g'} 
	& (D^{-1}N)^s.
}
\]

We now need to show that the first vertical map is an isomorphism. Consider any
$\f \in D^{-1}\Hom_R (M, N)$ satisfying $h(\f)= 0$. Then $f(\f) = f'(h(\f)) =
f'(0) = 0$, so therefore $\f = 0$ because $f$ is injective, and hence $h$ is
injective. 
%Now, consider any $\psi \in \Hom_{D^{-1}R} (D^{-1}M, D^{-1}N)$.
%
%Then $f(\f) \in \im \f = \ker g = \ker g' = \im
%f'$, so therefore there exists $\psi \in \Hom_{D^{-1}R} (D^{-1}M, D^{-1}N)$
%such that $f'(\psi) = f(\f) = f'(h(\f))$, where $f = f' \circ h$ because the
%diagram commutes. Since $f'$ is injective, this means that $\psi = h(\f)$. In
%the other direction, 
Suppose that $\psi \in \Hom_{D^{-1}R}(D^{-1}M, D^{-1}N)$.
Then $f'(\psi) \in \im f' = \ker g' = \ker g = \im f$, so there is some $\f \in
D^{-1} \Hom_R(M, N)$ such that $f'(h(\f)) = f(\f) = f'(\psi)$. Since $f'$ is
injective, this means that $h(\f) = \psi$, so therefore $\h$ is surjective.
Therefore, $h$ is an isomorphism, and hence
\[ 
D^{-1}\Hom_R(M, N) \cong \Hom_{D^{-1}R}(D^{-1}M, D^{-1}N).
\] 

\subsection*{5} (DF 17.2.1)

\subsection*{(a)}

Every element of 
$F_n = \ZZ G \otimes_\ZZ \ZZ G \otimes_\ZZ \cdots \otimes_\ZZ \ZZ G$ can be
written as a $\ZZ G$-linear combination of the basis elements
$1 \otimes g_1 \otimes g_2 \otimes \cdots \otimes g_n$, where $g_i \in G$. This
is because we can multiply any basis element by an integer to get
$1 \otimes a_1 g_1 \otimes a_2 g_2 \otimes \cdots \otimes a_n g_n$, and adding
basis elements of this form yields $1 \otimes \sum a_g g \otimes \sum a_g g
\otimes \cdots
\otimes \sum a_g g$, where each $\sum a_g g$ can be any element of $\ZZ G$; the tensor product is
multilinear. The group action then allows us to set the first element to be any
element of $\ZZ G$ that we want. Therefore, $F_n$ is a free $\ZZ G$ module with
basis elements $1 \otimes g_1 \otimes g_2 \otimes \cdots \otimes g_n$. There
are $|G|^n$ elements of this form, so $F_n$ has rank $|G|^n$.

\subsection*{(b)}

\def\e{\epsilon}

$\e$ is the augmentation map $\e : F_0 = \ZZ G \to \ZZ$ given by 
$\e (\sum a_g g) = \sum a_g$. 
$s_{-1}$ is defined as the contracting homomorphism $s_{-1} : \ZZ \to F_0$ such
that $s_{-1}(1) = 1$ so that $s_{-1}(n) = n \cdot s_{-1}(1) = n \cdot 1$. Then, 
$\e s_{-1} (n) = \e(n) = n$, so therefore we indeed have $\e s_{-1} = 1$. 

Now, 
\begin{align*}
(d_1 s_0 + s_{-1} \e)\left(\sum_{g \in G} a_g g\right) 
&= (d_1 s_0) \left(\sum_{g \in G} a_g g\right) 
	+ (s_{-1} \e) \left(\sum_{g \in G} a_g g\right) \\
&= d_1 \left(\sum_{g \in G} a_g (1 \otimes g)\right) 
	+ s_{-1} \left(\sum_{g \in G} a_g \right)  \\
&= \sum_{g \in G} a_g (g - 1) + \sum_{g \in G} a_g = \sum_{g \in G} a_g g.
\end{align*}

Finally, 
\begin{align*} 
&(d_{n+1} s_n + s_{n-1} d_n) \left( g_1, g_2, \cdots, g_n \right) 
= (d_{n+1} s_n) \left( g_1, \cdots, g_n \right) 
	+(s_{n-1} d_n) \left(g_1, \cdots, g_n \right) \\
& =  d_{n+1} (1, g_1, \cdots, g_n) \\
	&+ s_{n-1} \left( 
				g_1 \cdot (g_2, \cdots, g_n)
				+ \sum_{i=1}^{n-1} (-1)^i (g_1, \cdots, g_{i-1}, g_i g_{i+1},
				g_{i+2}, \cdots, g_n) 
				+ (-1)^n (g_1, \cdots, g_{n-1})
				\right) \\
&= (g_1, \cdots, g_n) + \sum_{i=1}^{n} (-1)^i (1, g_1, \cdots, g_{i-2}, g_{i-1}
g_{i}, g_{i+1}, \cdots, g_n) + (-1)^{n+1} (1, g_1, \cdots, g_{n-1}) \\
	&+  
				g_1\cdot(1, g_2, \cdots, g_n)
				+ \sum_{i=1}^{n-1} (-1)^i (1, g_1, \cdots, g_{i-1}, g_i g_{i+1},
				g_{i+2}, \cdots, g_{n}) 
				+ (-1)^n (1, g_1, \cdots, g_{n-1}) \\
&= (g_1, \cdots, g_n) + g_1 \cdot (g_2, \cdots, g_n) + (-1)^1 (g_2, g_2,
\cdots, g_n) = (g_1, \cdots, g_n),
\end{align*} 
which is what we wanted to show.


\subsection*{(c)}

This is true by definition of chain homotopy.

\[ 
\xymatrix{
\cdots \ar[r] & F_n \ar[r]^{d_{n}} & F_{n-1} \ar[r]^{d_{n-1}} 
\ar[dl]_{s_{n-1}} & F_{n-2} \ar[r]^{d_{n-2}} \ar[dl]_{s_{n-2}}
& \cdots \ar[r]^{d_2} & F_1 \ar[r]^{d_1}
& F_0 \ar[r]^{\e} \ar[dl]_{s_0} & \ZZ \ar[r] \ar[dl]_{s_{-1}} & 0 \\
%
\cdots \ar[r] & F_n \ar[r]^{d_{n}} & F_{n-1} \ar[r]^{d_{n-1}} &
F_{n-1} \ar[r]^{d_{n-2}}
& \cdots \ar[r]^{f_2} & F_1
\ar[r]^{d_1} & F_0 \ar[r]^{\e} & \ZZ \ar[r] & 0
}
\] 

Here, the difference between the identity map and the zero map is 1, so the
definition of chain homotopy is precisely what we checked in part (b):
$\e s_{-1} = 1$, $d_1 s_0 + s_{-1}\e = 0$, and 
$d_n s_{n-1} + s_{n-2} d_{n} = 1$ for all $n \ge 2$.
Therefore, the maps $s_n$ are indeed a chain homotopy between the identity
(chain) map and the zero (chain) map.

\subsection*{(d)}

Let the chain $\cdots \to F_n \to F_{n-1} \to \cdots \to F_0 \to \ZZ \to 0$ be
called $A$, so that the homology groups are $H^n (A)$.

We know that homotopic maps induce the same maps on homology.
From part (c), we therefore see that the identity chain map and the zero chain
map give equal induced group homomorphisms from $H^n (A)$ to $H^n(A)$. This is
only possible if the homology groups are all zero, since that's the only case
where the identity map and the zero map are the same. 

This means that $A$ is an exact sequence of $\ZZ$-modules. Since each $F_n$ is
free, each is also projective, so therefore $A$ represents a projective
$G$-module resolution of $\ZZ$.

\end{document}
me maps on homology.
From part (c), we therefore see that the identity chain map and the zero chain
map give equal induced group homomorphisms from $H^n (A)$ to $H^n(A)$. This is
only possible if the homology groups are all zero, since that's the only case
where the identity map and the zero map are the same. 

This means that $A$ is an exact sequence of $\ZZ$-modules. Since each $F_n$ is
free, each is also projective, so therefore $A$ represents a projective
$G$-module r
