\documentclass[12pt, reqno]{amsart}

\usepackage{amsthm}
\usepackage{amsmath}
\usepackage{url}
\usepackage{fancyhdr}
\renewcommand{\thesection}{\arabic{section}}
\newtheorem{theorem}{Theorem}[section]

\usepackage{hyperref}

\newtheorem{lemma}[theorem]{Lemma}
\newtheorem{sublemma}[theorem]{Sublemma}
\newtheorem{corollary}[theorem]{Corollary}
\newtheorem{proposition}[theorem]{Proposition}
\theoremstyle{definition}
\newtheorem{definition}[theorem]{Definition}
\newtheorem*{remark}{Remark}
\newtheorem{example}[theorem]{Example}
\newtheorem{exercise}{ \sc Exercise}[chapter]
\newtheorem*{solution}{Solution}
\newtheorem*{question}{Question}
\newtheorem*{problem}{Problem}
\newtheorem*{dbend}{Dangerous bend}



\usepackage[top=1.3in, bottom=1.3in, left=1.5in, right=1.5in]{geometry}
\usepackage{amssymb}
\usepackage{amsfonts}
\usepackage{stackrel}
\usepackage{mathrsfs}
\usepackage{xy}
\usepackage{verbatim}


\newcommand{\whattosay}{}



\input xy
\xyoption{all}


\newcommand{\lecture}[1]{}
\newcommand{\rad}{\mathrm{rad}}
\newcommand{\im}{\mathrm{Im}}
\newcommand{\proj}{\mathrm{Proj}}
\renewcommand{\hom}{\mathrm{Hom}}
\newcommand{\id}{\mathrm{id}}
\providecommand{\cal}[1]{\mathcal{#1}}
\renewcommand{\cal}[1]{\mathcal{#1}}


%\swapnumbers

\renewcommand{\qedsymbol}{$\blacktriangle$}


\begin{document}

\title{Math 210A Problem Set 6}
\input{other/amsdata.tex}


\subsection*{2}

Suppose $A$ is an $n \times n$ matrix with real entries such that the diagonal
entries are all positive, the off-diagonal entries are all negative, and the
row sums are all positive. Write $A = (A_{ij})$.

Suppose to the contrary that $\det A = 0$, and consider the corresponding
system of equations $AX = 0$. This means that $AX = 0$ has a nontrivial
solution $(x_1, \dots, x_n)$. Therefore, we have
\[ 
\sum_{j=1}^n a_{kj} x_j = 0
\] 
for each $1 \le k \le n$.

Suppose that $x_i$ has the largest absolute value in $(x_1, x_2, \dots, x_n)$.
Then the $i$-th equation states that 
$a_{i1} x_1 + a_{i2} x_2 + \dots + a_{ii}x_i + \dots + a_{in} x_n = 0$.
For $j \ne i$, $a_{ij} < 0$. 

In the case $x_i > 0$, we see that 
$a_{ij} x_j \ge a_{ij} x_i$. 
Therefore, we see that 
$$
0 = a_{i1} x_1 + a_{i2} x_2 + \dots + a_{ii}x_i + \dots + a_{in} x_n
\ge (a_{i1} + a_{i2} + \dots + a_{in}) x_i > 0
$$
because the row sums of the matrix $A$ are all positive,
which is impossible.

In the case $x_i < 0$, we can do an analogous calculation to see that
$a_{ij} x_j \le a_{ij} x_i$, so that 
$$
0 = a_{i1} x_1 + a_{i2} x_2 + \dots + a_{ii}x_i + \dots + a_{in} x_n
\le (a_{i1} + a_{i2} + \dots + a_{in}) x_i < 0, 
$$
which is also impossible.

Therefore, we can conclude that the $x_i = 0$. However, $x_i$ was the maximal
entry of $(x_1, x_2, \dots, x_n)$, so therefore our nontrivial solution to $AX
= 0$ is actually $(0, 0, \dots, 0)$, which is trivial. Hence, 
we must conclude that $\det A \ne 0$.



\subsection*{4}

Consider
\[ 
\xymatrix{
\dots \ar[r] & 0 \ar[r] \ar[d] & 0 \ar[r] \ar[d] & 0 \ar[r] \ar[d] & \dots \\
\dots \ar[r] & A^{i-1} \ar[r]^{f^{i-1}} \ar[d]_{\a_{i-1}}
	& A^i \ar[r]^{f^i} \ar[d]^{\a_i} 
	& A^{i+1} \ar[r]^{f^{i+1}} \ar[d]^{\a_{i+1}} & \dots \\
\dots \ar[r] & B^{i-1} \ar[r]^{g^{i-1}} \ar[d]_{\b_{i-1}}
	& B^i \ar[r]^{g^i} \ar[d]^{\b_i} 
	& B^{i+1} \ar[r]^{g^{i+1}} \ar[d]^{\b_{i+1}} & \dots \\
\dots \ar[r] & C^{i-1} \ar[r]^{h^{i-1}} \ar[d]
	& C^i \ar[r]^{h^i} \ar[d]
	& C^{i+1} \ar[r]^{h^{i+1}} \ar[d] & \dots \\
\dots \ar[r] & 0 \ar[r] & 0 \ar[r] & 0 \ar[r] & \dots
}
\] 

Consider any $c \in C^{i-1}$ that represents the class 
$x \in H^{i-1}(C) = \ker(h^{i-1}) / \im (h^{i-2})$. 
Then, since $\b_{i-1}$ is surjective, there exists $b \in B^{i-1}$ such that 
$\b_{i-1}(b) = c$. Now, since $c \in \ker(h^{i-1})$, we have
$h^{i-1} (c) = 0$, so that $h^{i-1} (\b_{i-1} (b)) = 0$.
By commutativity of squares, $\b_i (g^{i-1}(b)) = h^{i-1} (\b_{i-1} (b)) = 0$. 
Therefore $g^{i-1} (b) \in \ker \b_i = \im \a_i$ by the exactness at $B^i$.
Since $\a_i$ is injective, there exists a unique $a \in A^i$ such that
$\a_i(a) = g^{i-1}(b)$.

Since $B$ is a complex, $g^i(\a_i(a)) = g^i (g^{i-1}(b)) = 0$. By the
commutativity of the diagram, $\a_{i+1} (f^i(a)) = g^i (\a_i(a)) = 0$. Since
$\a_{i+1}$ is injective, we have $f^i(a) = 0$ and hence 
$a \in \ker f^i$. $a$ therefore defines a class $\ol a$ in the quotient group
$H^i(A) = \ker(f^i) / \im(f^{i-1})$. 

We want to show that $\ol a$ is independent of the choice of $b$. Suppose that
$b'$ is another choice of $b$ satisfying $\b_{i-1}(b) = \b_{i-1}(b') = c$.
This means that $\b_{i-1}(b-b') = 0$, so 
$b - b' \in \ker \b_{i-1} = \im  \a_{i-1}$ by the exactness at $B^{i-1}$.
Since $\a_{i-1}$ is injective, there exists a unique 
$\ul a \in A^{i-1}$ such that $\a_{i-1}(\ul a) = b-b'$. Therefore, by the
commutativity of the  diagram, 
$\a_i(f^{i-1}(\ul a)) = g^{i-1}(\a_{i-1}(\ul a)) = g^{i-1}(b-b')$.

Now, our choice of $b'$ leads by the previous reasoning to a new value of $a'$.
Here, $a$ and $a'$ differ by an amount $a - a'$ satisfying
$\a_i(a - a') = \a_i(a) - \a_i(a') = g^{i-1}(b) - g^{i-1}(b') 
= g^{i-1}(b-b') = \a_i (f^{i-1}(\ul a))$.
Since $\a_i$ is injective, this means that $a - a' = f^{i-1}(\ul a)$, so hence
$a - a' \in \im f^{i-1}$, and therefore $a$ and $a'$ are in the same class of
$H^i (A)$.

%We also want to show that $\ol a$ is independent of the choice of $c$
%representing the class $x$. Consider any $\tilde c$ 
%representing the same class $x$. 
%There then exists $\ul b \in B^{i-1}$ such that 
%$\b_{i-1}(\ul b) = c - \tilde c$. Since $c$ and $\tilde c$ are in the same
%class, $c - \tilde c \in \im h^{i-2} \subset \ker h^{i-1}$ because $C$ is a
%complex. This means that 
%$\b_i (g^{i-1}(\ul b)) = h^{i-1} (\b_{i-1} (\ul b)) = 0$ by the commutativity
%of the diagram. %this isn't enoguh!!!

%There then exists $\tilde b \in B^{i-1}$ such
%that $\b_{i-1}(\tilde c) = \tilde c$; as shown above, this is independent of
%the choice of $\tilde b$. Proceeding as before, we see that 
%$\b_i (g^{i-1}(b)) = \b_i (g^{i-1}(\tilde b)) = 0$.
%Therefore, $g^{i-1}(b - \tilde b) \in \ker \b_i = \im \a_i$ because of the
%exactness at $B^i$.

%At this point: Forget about this and move on!




\newpage

\subsection*{5}

Suppose that 
$ 
\xymatrix{
M' \ar[r]^f & M \ar[r]^g & M'' 
}
$
is exact. This means that $\im(f) = \ker(g)$, which 
is equivalent to saying that 
$0 \to \im(f) \to M \to M/\ker(g) \to 0$
is a short exact sequence.

Since the functor $F$ preserves short exact sequences, this means that
$$0 \to F(\im(f)) \to F(M) \to F(M / \ker(g)) \to 0$$ is a short exact sequence.

Now, $F(\im(f)) = \im(F(f))$ and $F ( M / \ker(g)) = F(M) / \ker(F(g))$, so that
$$0 \to \im(F(f)) \to F(M) \to F(M) / \ker(F(g)) \to 0$$ is a short exact
sequence and hence 
$
\xymatrix{
F(M') \ar[r]^{F(f)} & F(M) \ar[r]^{F(g)} & F(M'') 
}
$
is exact.



\subsection*{6(a)}

Suppose that $M' \xrightarrow f M \xrightarrow g  M''$ is exact at $M$.
We want to show that 
\[ 
\xymatrix{
S^{-1}M' \ar[r]^{S^{-1}f} & S^{-1}M \ar[r]^{S^{-1}g} & S^{-1}M'' 
}
\] 
is exact at $S^{-1}M$, where
$S^{-1}f(m'/s) = f(m')/s$ and $S^{-1}g(m/s) = g(m)/s$. In particular, this
means that $S^{-1} (v \circ u) = S^{-1}(v) \circ S^{-1}(u)$.

We have $g \circ f = 0$, so therefore 
$S^{-1}g \circ S^{-1}f = S^{-1}0 = 0$, and hence 
$\im(S^{-1}f) \subset \ker(S^{-1}g)$.

To show the reverse inclusion, let $m/s \in \ker(S^{-1}g)$, so that
$S^{-1}g(m/s) = g(m)/s = 0$. Therefore, there exists some $t \in S$ such that
$t g(m) = 0$ in $M''$. Since $g$ is a homomorphism, $tg(m) = g(tm) = 0$, and
therefore $tm \in \ker g = \im f$ by the exactness at $M$. Therefore, there
exists $m' \in M'$ such that $f(m') = tm$.
We therefore see that in $S^{-1}M$, 
$m/s = f(m')/st = S^{-1}f(m'/st)$, so therefore $m/s \in \im(S^{-1}f)$. This
shows that $\ker(S^{-1}g) \subset \im(S^{-1}f)$, and hence that
$\ker (S^{-1}g) = \im (S^{-1}f)$. This implies the desired exactness at 
$S^{-1}m$.


\end{document}
\ker v = \im u$ by the
exactness at $B$, we have  
$C \cong B / \ker v = B / \im u$, so that we have a well-defined map
$v^{-1} : C \to B / \im u$.
Combined with $f (\im u) = 0$, we can define a well-defined map 
$g = f \circ v^{-1} \in \Hom(C, M)$. This new map satisfies $f = g \circ v$
Therefore, $\ol v(
