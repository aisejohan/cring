\documentclass[11pt]{article}
\usepackage[hmargin=26mm,vmargin=26mm]{geometry}
\usepackage{amsmath, amssymb, amsthm, verbatim, graphicx, mathabx, fancyhdr}
\usepackage[all,cmtip]{xy}

\newtheorem{proposition}{Proposition}

\begin{document}

\begin{proposition}
Let $ A$ be a ring and let $ X = \mathrm{Spec}(A)$. The presheaf of rings $ \mathcal{O}_X$ defined on $ X$ is a sheaf.
\end{proposition}

\begin{proof}
The proof proceeds in two parts. Let $ (U_i)_{i \in I}$ be a covering of $ X$ by basic open sets.

\emph{Part 1}. If $ s \in A$ is such that $ s_i := \rho_{X, U_i}(s) = 0$ for all $ i \in I$, then $ s = 0$.

\emph{Proof of part 1}. Suppose $ U_i = X_{f_i}$. Note that $ s_i$ is the fraction $ s/1$ in the ring $ A_{f_i}$, so $ s_i = 0$ implies that there exists some integer $ m_i$ such that $ sf_i^{m_i} = 0$. Define $ g_i = f_i ^{m_i}$, and note that we still have an open cover by sets $ X_{g_i}$ since $ X_{f_i} = X_{g_i}$ (a prime ideal contains an element if and only if it contains every power of that element). Also $ s g_i = 0$, so the fraction $ s/1$ is still $ 0$ in the ring $ A_{g_i}$. (Essentially, all we're observing here is that we are free to change representation of the basic open sets in our cover to make notation more convenient).

Since $ X$ is quasicompact, choose a finite subcover $ X = X_{g_1} \cup \dotsb \cup X_{g_n}$. This means that $ g_1, \dotsc, g_n$ must generate the unit ideal, so there exists some linear combination $ \sum x_i g_i = 1$ with $ x_i \in A$. But then
\[ s = s \cdot 1 = s \left( \sum x_i g_i \right) = \sum x_i (s g_i) = 0.\]

\emph{Part 2}. Let $ s_i \in \mathcal{O}_X(U_i)$ be such that for every $ i, j \in I$,
\[ \rho_{U_i, U_i \cap U_j}(s_i) = \rho_{U_j, U_i \cap U_j}(s_j).\]
(That is, the collection $ (s_i)_{i \in I}$ agrees on overlaps). Then there exists a unique $ s \in A$ such that $ \rho_{X, U_i}(s) = s_i$ for every $ i \in I$.

\emph{Proof of part 2}. Let $ U_i = X_{f_i}$, so that $ s_i = a_i/(f_i^{m_i})$ for some integers $ m_i$. As in part 1, we can clean up notation by defining $ g_i = f_i^{m_i}$, so that $ s_i = a_i/g_i$. Choose a finite subcover $ X = X_{g_1} \cup \dotsb \cup X_{g_n}$. Then the condition that the cover agrees on overlaps means that
\[ \frac{a_i g_j}{g_i g_j} = \frac{a_j g_i}{g_i g_j} \]
for all $ i, j$ in the finite subcover. This is equivalent to the existence of some $ k_{ij}$ such that
\[ (a_i g_j - a_j g_i) (g_i g_j)^{k_{ij}} = 0.\]
Let $ k$ be the maximum of all the $ k_{ij}$, so that $ (a_i g_j - a_j g_i)(g_i g_j)^k = 0$ for all $ i, j$ in the finite subcover. Define $ b_i = a_i g_i^k$ and $ h_i = g_i^{k+1}$. We make the following observations:
\[ b_i h_j - b_j h_i = 0, X_{g_i} = X_{h_i}, \text{ and } s_i = a_i/g_i = b_i/h_i \]
The first observation implies that the $ X_{h_i}$ cover $ X$, so the $ h_i$ generate the unit ideal. Then there exists some linear combination $ \sum x_i h_i = 1$. Define $ s = \sum x_i b_i$. I claim that this is the global section that restricts to $ s_i$ on the open cover.

The first step is to show that it restricts to $ s_i$ on our chosen finite subcover. In other words, we want to show that $ s/1 = s_i = b_i/h_i$ in $ A_{h_i}$, which is equivalent to the condition that there exist some $ l_i$ such that $ (sh_i b_i) h_i^{l_i} = 0$. But in fact, even $ l_i = 0$ works:
\[ sh_i - b_i = \left(\sum x_j b_j\right) h_i - b_i\left(\sum x_j h_j\right) = \sum x_j\left(h_i b_j - b_i h_j\right) = 0. \]
This shows that $ s$ restricts to $ s_i$ on each set in our finite subcover. Now we need to show that in fact, it restricts to $ s_i$ for all of the sets in our cover. Choose any $ j \in I$. Then $ U_1, \dotsc, U_n, U_j$ still cover $ X$, so the above process yields an $ s'$ such that $ s'$ restricts to $ s_i$ for all $ i \in \{1, \dotsc, n, j\}$. But then $ s - s'$ satisfies the assumptions of part 1 using the cover $ \{U_1, \dotsc, U_n, U_j\}$, so this means $ s = s'$. Hence the restriction of $ s$ to $ U_j$ is also $ s_j$.
\end{proof}

% This is currently missing the (sort of boring) generalization to gluability on arbitrary basic open sets. 

\end{document}