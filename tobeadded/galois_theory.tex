% Copyright (C) 2006 Ryan Reich
% Permission is granted to copy, distribute and/or modify this document
% under the terms of the GNU Free Documentation License, Version 1.2
% or any later version published by the Free Software Foundation;
% with no Invariant Sections, no Front-Cover Texts, and no Back-Cover Texts.
% A copy of the license is included in the section entitled "GNU
% Free Documentation License".
\documentclass{article}
\usepackage[leqno]{amsmath}
\usepackage{amssymb,amsthm,enumerate,fullpage}
\swapnumbers
\theoremstyle{definition}
\newtheorem{num}{Numbering}[section]
\newtheorem{thm}[num]{Theorem}
\newtheorem{prop}[num]{Proposition}
\newtheorem{lem}[num]{Lemma}
\newtheorem{cor}[num]{Corollary}
\newtheorem{defn}[num]{Definition}
\author{Ryan C. Reich}
\date{16 June 2006}
\title{Notes on Field Extensions}
\DeclareMathOperator{\Emb}{\text{Emb}}
\DeclareMathOperator{\Aut}{\text{Aut}}
\DeclareMathOperator{\Char}{\text{char}}
\DeclareMathOperator{\Gal}{\text{Gal}}
\begin{document}
\maketitle

\section{Definitions}

Throughout, $F \subset K$ is a finite field extension.  We fix once and for
all an algebraic closure $M$ for both and an embedding of $F$ in $M$.  When
necessary, we write $K = F(\alpha_1, \dots, \alpha_n)$, and $K_0 = F, K_i =
F(\alpha_1, \dots, \alpha_i)$, $q_i$ the minimal polynomial of $\alpha_i$ over
$F_{i - 1}$, $Q_i$ that over $F$.

\begin{defn} $\Aut(K/F)$ denotes the group of automorphisms of $K$ which fix
$F$ (pointwise!).  $\Emb(K/F)$ denotes the set of embeddings of $K$ into $M$
respecting the chosen embedding of $F$.
\label{def:gal}
\end{defn}

\begin{defn} For an element $\alpha \in K$ with minimal polynomial $q$, we say
$q$ and $\alpha$ are separable if $q$ has distinct roots, and we say $K$ is
separable if this holds for all $\alpha$; conversely, we say they are purely
inseparable if $q$ has only one root.  We say $K$ is splitting if each $q$
splits in $K$.
\label{def:sepsplit}
\end{defn}

\begin{defn} By $\deg(K/F)$ we mean the dimension of $K$ as an $F$-vector
space.  We denote $K_s/F$ the set of elements of $K$ whose minimal polynomials
over $F$ have distinct roots; by \ref{sep_subfield} this is a subfield, and
$\deg(K_s/F) = \deg_s(K/F)$ and $\deg(K/K_s) = \deg_i(K/F)$ by definition.
\label{def:sep}
\end{defn}

\begin{defn} If $K = F(\alpha)$ for some $\alpha$ with minimal polynomial
$q(x) \in F[x]$, then by \ref{sep_poly}, $q(x) = r(x^{p^d})$, where $p =
\Char{F}$ (or $1$ if $\Char{F} = 0$) and $r$ is separable; in this case we
also denote $\deg_s(K/F) = \deg(r), \deg_i(K/F) = p^d$.  \label{def:prim_sep}
\end{defn}

\section{Theorems}

\begin{lem} $q(x) \in F[x]$ is separable if and only if $\gcd(q, q') = 1$,
where $q'$ is the formal derivative of $q$. 
\label{der_poly}
\end{lem}

\begin{proof} Passing to $M$, we may factor $q$:
\begin{equation*}
q(x) = \prod_{i = 1}^n (x - r_i)^{m_i}
\end{equation*}
and by the product rule,
\begin{equation*}
q'(x) = \sum_{j = 1}^n m_j (x - r_j)^{m_j - 1} \prod_{i \neq j} (x -
	r_i)^{m_i}.
\end{equation*}
$\gcd(q, q')$ is independent of the field since $K[x]$ is a PID, hence is the
product of the common factors of $q$ and $q'$ over $M$.  $q$ and $q'$ have a
common factor in $M$ if and only if they have a common root, since they both
split, and therefore if and only if $q'$ vanishes at some $r_j$.  Given the
above representation, this holds if and only if $m_j > 1$. \end{proof}

\begin{lem} If $\Char{F} = 0$ then $K_s = K$.  If $\Char{F} = p > 0$, then for
any irreducible $q(x) \in K[x]$, there is some $d \geq 0$ and polynomial $r(x)
\in K[x]$ such that $q(x) = r(x^{p^d})$, and $r$ is separable and irreducible.
\label{sep_poly}
\end{lem}

\begin{proof} By formal differentiation, $q'(x)$ has positive degree unless
each exponent is a multiple of $p$; in characteristic zero this never occurs.
If this is not the case, since $q$ is irreducible, it can have no factor in
common with $q'$ and therefore has distinct roots by \ref{der_poly}.

If $p > 0$, let $d$ be the largest integer such that each exponent of $q$ is a
multiple of $p^d$, and define $r$ by the above equation.  Then by
construction, $r$ has at least one exponent which is not a multiple of $p$,
and therefore has distinct roots. \end{proof}

\begin{cor} In the statement of \ref{sep_poly}, $q$ and $r$ have the same
number of roots.
\label{sep_roots}
\end{cor}

\begin{proof} $\alpha$ is a root of $q$ if and only if $\alpha^{p^d}$ is a
root of $r$; i.e. the roots of $q$ are the roots of $x^{p^d} - \beta$, where
$\beta$ is a root of $r$.  But if $\alpha$ is one such root, then $(x -
\alpha)^{p^d} = x^{p^d} - \alpha^{p^d} = x^{p^d} - \beta$ since $\Char{K} =
p$, and therefore $\alpha$ is the only root of $x^{p^d} - \beta$. \end{proof}

\begin{lem} The correspondence which to each $g \in \Emb(K/F)$ assigns the
$n$-tuple $(g(\alpha_1), \dots, g(\alpha_n))$ of elements of $M$ is a
bijection from $\Emb(K/F)$ to the set of tuples of $\beta_i \in M$, such that
$\beta_i$ is a root of $q_i$ over $K(\beta_1, \dots, \beta_{i - 1})$.
\label{emb_roots}
\end{lem}

\begin{proof} First take $K = F(\alpha) = F[x]/(q)$, in which case the maps $g
\colon K \to M$ over $F$ are identified with the elements $\beta \in M$ such
that $q(\beta) = 0$ (where $g(\alpha) = \beta$).

Now, considering the tower $K = K_n / K_{n - 1} / \dots / K_0 = F$, each
extension of which is primitive, and a given embedding $g$, we define
recursively $g_1 \in \Emb(K_1/F)$ by restriction and subsequent $g_i$ by
identifying $K_{i - 1}$ with its image and restricting $g$ to $K_i$.  By the
above paragraph each $g_i$ corresponds to the image $\beta_i = g_i(\alpha_i)$,
each of which is a root of $q_i$.  Conversely, given such a set of roots of
the $q_i$, we define $g$ recursively by this formula. \end{proof}

\begin{cor} $|\Emb(K/F)| = \prod_{i = 1}^n \deg_s(q_i)$.
\label{emb_size}
\end{cor}

\begin{proof} This follows immediately by induction from \ref{emb_roots} by
\ref{sep_roots}. \end{proof}

\begin{lem} For any $f \in \Emb(K/F)$, the map $\Aut(K/F) \to \Emb(K/F)$ given
by $\sigma \mapsto f \circ \sigma$ is injective.  
\label{aut_inj}
\end{lem}

\begin{proof} This is immediate from the injectivity of $f$. \end{proof}

\begin{cor} $\Aut(K/F)$ is finite.
\label{aut_fin}
\end{cor}

\begin{proof} By \ref{aut_inj}, $\Aut(K/F)$ injects into $\Emb(K/F)$, which by
\ref{emb_size} is finite. \end{proof}

\begin{prop} The inequality
\begin{equation*}
|\Aut(K/F)| \leq |\Emb(K/F)|
\end{equation*}
is an equality if and only if the $q_i$ all split in $K$.
\label{aut_ineq}
\end{prop}

\begin{proof} The inequality follows from \ref{aut_inj} and from \ref{aut_fin}.
Since both sets are finite, equality holds if and only if the injection of
\ref{aut_inj} is surjective (for fixed $f \in \Emb(K/F)$).

If surjectivity holds, let $\beta_1, \dots, \beta_n$ be arbitrary roots of
$q_1, \dots, q_n$ in the sense of \ref{emb_roots}, and extract an embedding $g
\colon K \to M$ with $g(\alpha_i) = \beta_i$.  Since the correspondence $f
\mapsto f \circ \sigma$ ($\sigma \in \Aut(K/F)$) is a bijection, there is some
$\sigma$ such that $g = f \circ \sigma$, and therefore $f$ and $g$ have the
same image.  Therefore the image of $K$ in $M$ is canonical, and contains
$\beta_1, \dots, \beta_n$ for any choice thereof.

If the $q_i$ all split, let $g \in \Emb(K/F)$ be arbitrary, so the
$g(\alpha_i)$ are roots of $q_i$ in $M$ as in \ref{emb_roots}.  But the $q_i$
have all their roots in $K$, hence in the image $f(K)$, so $f$ and $g$ again
have the same image, and $f^{-1} \circ g \in \Aut(K/F)$.  Thus $g = f \circ
(f^{-1} \circ g)$ shows that the map of \ref{aut_inj} is surjective.
\end{proof}

\begin{cor} Define
\begin{equation*}
D(K/F) = \prod_{i = 1}^n \deg_s(K_i/K_{i - 1}).
\end{equation*}
Then the chain of equalities and inequalities
\begin{equation*}
|\Aut(K/F)| \leq |\Emb(K/F)| = D(K/F) \leq \deg(K/F)
\end{equation*}
holds; the first inequality is an equality if and only if each $q_i$ splits in
$K$, and the second if and only if each $q_i$ is separable.
\label{large_aut_ineq}
\end{cor}

\begin{proof} The statements concerning the first inequality are just
\ref{aut_ineq}; the interior equality is just \ref{emb_size}; the latter
inequality is obvious from the multiplicativity of the degrees of field
extensions; and the deduction for equality follows from the definition of
$\deg_s$. \end{proof}

\begin{cor} The $q_i$ respectively split and are separable in $K$ if and only
if the $Q_i$ do and are.
\label{absolute_sepsplit}
\end{cor}

\begin{proof} The ordering of the $\alpha_i$ is irrelevant, so we may take
each $i = 1$ in turn.  Then $Q_1 = q_1$ and if either of the equalities in
\ref{large_aut_ineq} holds then so does the corresponding statement here.
Conversely, clearly each $q_i$ divides $Q_i$, so splitting or separability
for the latter implies that for the former. \end{proof}

\begin{cor} Let $\alpha \in K$ have minimal polynomial $q$; if the $Q_i$ are
respectively split, separable, and purely inseparable over $F$ then $q$ is as
well.
\label{global_sepsplit}
\end{cor}

\begin{proof} We may take $\alpha$ as the first element of an alternative
generating set for $K/F$.  The numerical statement of \ref{large_aut_ineq}
does not depend on the particular generating set, hence the conditions given
hold of the set containing $\alpha$ if and only if they hold of the canonical
set ${\alpha_1, \dots, \alpha_n}$.

For purely inseparable, if the $Q_i$ all have only one root then $|\Emb(K/F)|
= 1$ by \ref{large_aut_ineq}, and taking $\alpha$ as the first element of a
generating set as above shows that $q$ must have only one root as well for
this to hold. \end{proof}

\begin{cor} $K_s$ is a field and $\deg(K_s/F) = D(K/F)$.
\label{sep_subfield}
\end{cor}

\begin{proof} Assume $\Char{F} = p > 0$, for otherwise $K_s = K$.  Using
\ref{sep_poly}, write each $Q_i = R_i(x^{p^{d_i}})$, and let $\beta_i =
\alpha_i^{p^{d_i}}$.  Then the $\beta_i$ have $R_i$ as minimal polynomials and
the $\alpha_i$ satisfy $s_i = x^{p^{d_i}} - \beta_i$ over $K' = F(\beta_1,
\dots, \beta_n)$.  Therefore the $\alpha_i$ have minimal polynomials over $K'$
dividing the $s_i$ and hence those polynomials have but one distinct root.

By \ref{global_sepsplit}, the elements of $K'$ are separable, and those of
$K'$ purely inseparable over $K'$.  In particular, since these minimal
polynomials divide those over $F$, none of these elements is separable, so $K'
= K_s$.

The numerical statement follows by computation:
\begin{equation*}
\deg(K/K') = \prod_{i = 1}^n p^{d_i}
	= \prod_{i = 1}^n \frac{\deg(K_i/K_{i - 1})}{\deg_s(K_i/K_{i - 1})}
	= \frac{\deg(K/F)}{D(K/F)}. \qedhere
\end{equation*}
\end{proof}

\begin{thm} The following inequality holds:
\begin{equation*}
|\Aut(K/F)| \leq |\Emb(K/F)| = \deg_s(K/F) \leq \deg(K/F).
\end{equation*}
Equality holds on the left if and only if $K/F$ is splitting; it holds on the
right if and only if $K/F$ is separable.
\label{galois_size}
\end{thm}

\begin{proof} The numerical statement combines \ref{large_aut_ineq} and
\ref{sep_subfield}.  The deductions combine \ref{absolute_sepsplit} and
\ref{global_sepsplit}. \end{proof}

\section{Definitions}

Throughout, we will denote as before $K/F$ a finite field extension, and $G =
\Aut(K/F)$, $H$ a subgroup of $G$.  $L/F$ is a subextension of $K/F$.

\begin{defn} When $K/F$ is separable and splitting, we say it is Galois and
write $G = \Gal(K/F)$, the Galois group of $K$ over $F$.
\label{defn:galois_extension}
\end{defn}

\begin{defn} The fixed field of $H$ is the field $K^H$ of elements fixed by
the action of $H$ on $K$.  Conversely, $G_L$ is the fixing subgroup of $L$,
the subgroup of $G$ whose elements fix $L$.
\label{defn:fixing}
\end{defn}

\section{Theorems}

\begin{lem} A polynomial $q(x) \in K[x]$ which splits in $K$ lies in
$K^H[x]$ if and only if its roots are permuted by the action of $H$.  In this
case, the sets of roots of the irreducible factors of $q$ over $K^H$ are the orbits
of the action of $H$ on the roots of $q$ (counting multiplicity).
\label{root_action}
\end{lem}

\begin{proof} Since $H$ acts by automorphisms, we have $\sigma q(x) = q(\sigma
x)$ as a functional equation on $K$, so $\sigma$ permutes the roots of $q$.
Conversely, since the coefficients of $\sigma$ are the elementary symmetric
polynomials in its roots, $H$ permuting the roots implies that it fixes the
coefficients.

Clearly $q$ is the product of the polynomials $q_i$ whose roots are the orbits
of the action of $H$ on the roots of $q$, counting multiplicities, so it
suffices to show that these polynomials are defined over $K^H$ and are
irreducible.  Since $H$ acts on the roots of the $q_i$ by construction, the
former is satisfied.  If some $q_i$ factored over $K^H$, its factors would
admit an action of $H$ on their roots by the previous paragraph.  The roots of
$q_i$ are distinct by construction, so its factors do not share roots; hence
the action on the roots of $q_i$ would not be transitive, a contradiction.
\end{proof}

\begin{cor} Let $q(x) \in K[x]$; if it is irreducible, then $H$ acts
transitively on its roots; conversely, if $q$ is separable and $H$ acts
transitively on its roots, then $q(x) \in K^H[x]$ is irreducible.
\label{sep_irred}
\end{cor}

\begin{proof} Immediate from \ref{root_action}. \end{proof}

\begin{lem} If $K/F$ is Galois, so is $K/L$, and $\Gal(K/L) = G_L$..
\label{sub_galois}
\end{lem}

\begin{proof} $K/F$ Galois means that the minimal polynomial over $F$ of every
element of $K$ is separable and splits in $K$; the minimal polynomials over $L
= K^H$ divide those over $F$, and therefore this is true of $K/L$ as well;
hence $K/L$ is likewise a Galois extension. $\Gal(K/L) = \Aut(K/L)$ consists
of those automorphisms $\sigma$ of $K$ which fix $L$; since $F \subset L$ we
have \emph{a fortiori} that $\sigma$ fixes $F$, hence $\Gal(K/L) \subset G$
and consists of the subgroup which fixes $L$; i.e. $G_L$. \end{proof}

\begin{cor} If $K/F$ and $L/F$ are Galois, then the action of $G$ on elements of $L$
defines a surjection of $G$ onto $\Gal(L/F)$.  Thus $G_L$ is normal in $G$ and $\Gal(L/F) \cong G/G_L$.  Conversely, if $N \subset G$ is normal, then $K^N/F$ is Galois.
\label{normal}
\end{cor}

\begin{proof} $L/F$ is splitting, so by \ref{root_action} the elements of $G$
act as endomorphisms (hence automorphisms) of $L/F$, and the kernel of this action is $G_L$.  By
\ref{sub_galois}, we have $G_L = \Gal(K/L)$, so $|G_L| = |\Gal(K/L)| = [K : L] = [K : F] / [L : F]$,
or rearranging and using that $K/F$ is Galois, we get $|G|/|G_L| = [L : F] =
|\Gal(L/F)|$.  Thus the map $G \to \Gal(L/F)$ is surjective and thus the induced map $G/G_L \to
\Gal(L/F)$ is an isomorphism.

Conversely, let $N$ be normal and take $\alpha \in K^N$.  For any conjugate $\beta$ of $\alpha$, we
have $\beta = g(\alpha)$ for some $g \in G$; let $n \in N$.  Then $n(\beta) = (ng)(\alpha) =
g(g^{-1} n g)(\alpha) = g(\alpha) = \beta$, since $g^{-1} n g \in N$ by normality of $N$.  Thus
$\beta \in K^N$, so $K^N$ is splitting, i.e., Galois. \end{proof}

\begin{prop} If $K/F$ is Galois and $H = G_L$, then $K^H = L$.
\label{fixed_field}
\end{prop}

\begin{proof} By \ref{sub_galois}, $K/L$ and $K/K^H$ are both Galois.  By
definition, $\Gal(K/L) = G_L = H$; since $H$ fixes $K^H$ we certainly have
$H < \Gal(K/K^H)$, but since $L \subset K^H$ we have \emph{a fortiori} that
$\Gal(K/K^H) < \Gal(K/L) = H$, so $\Gal(K/K^H) = H$ as well.  It follows
from \ref{galois_size} that $\deg(K/L) = |H| = \deg(K/K^H)$, so that $K^H =
L$. \end{proof}

\begin{lem} If $K$ is a finite field, then $K^\ast$ is cyclic.
\label{fin_cyclic}
\end{lem}

\begin{proof} $K$ is then a finite extension of $\mathbb{F}_p$ for $p =
\Char{K}$, hence has order $p^n$, $n = \deg(K/\mathbb{F}_p)$.  Thus
$\alpha^{p^n} = \alpha$ for all $\alpha \in K$, since $|K^\ast| = p^n - 1$.
It follows that every element of $K$ is a root of $q_n(x) = x^{p^n} - x$.  For
any $d < n$, the elements of order at most $p^d - 1$ satisfy $q_d(x)$, which has
$p^d$ roots.  It follows that there are at least $p^n(p - 1) > 0$ elements of
order exactly $p^n - 1$, so $K^\ast$ is cyclic. \end{proof}

\begin{cor} If $K$ is a finite field, then $\Gal(K/F)$ is cyclic, generated by
the Frobenius automorphism.
\label{fin_gal_cyclic}
\end{cor}

\begin{proof} First take $F = \mathbb{F}_p$.  Then the map $f_i(\alpha) =
\alpha^{p^i}$ is an endomorphism, injective since $K$ is a field, and
surjective since it is finite, hence an automorphism.  Since every $\alpha$
satisfies $\alpha^{p^n} = \alpha$, $f_n = 1$, but by \ref{fin_cyclic}, $f_{n -
1}$ is nontrivial (applied to the generator).  Since $n = \deg(K/F)$, $f =
f_1$ generates $\Gal(K/F)$.

If $F$ is now arbitrary, by \ref{fixed_field} we have $\Gal(K/F) =
\Gal(K/\mathbb{F}_p)_F$, and every subgroup of a cyclic group is cyclic.
\end{proof}

\begin{cor} If $K$ is finite, $K/F$ is primitive.
\label{fin_prim_elt}
\end{cor}

\begin{proof} No element of $G$ fixes the generator $\alpha$ of $K^\ast$, so
it cannot lie in any proper subfield.  Therefore $F(\alpha) = K$. \end{proof}

\begin{prop} If $F$ is infinite and $K/F$ has only finitely many subextensions, then it is
primitive.
\label{gen_prim_elt}
\end{prop}

\begin{proof} We proceed by induction on the number of generators of $K/F$.

If $K = F(\alpha)$ we are done.  If not, $K = F(\alpha_1, \dots, \alpha_n) =
F(\alpha_1, \dots, \alpha_{n - 1})(\alpha_n) = F(\beta, \alpha_n)$ by
induction, so we may assume $n = 2$.  There are infinitely many subfields
$F(\alpha_1 + t \alpha_2)$, with $t \in F$, hence two of them are equal, say for $t_1$ and
$t_2$.  Thus, $\alpha_1 + t_2 \alpha_2 \in F(\alpha_1 + t_1 \alpha_2)$.  Then
$(t_2 - t_1)\alpha_2 \in F(\alpha_1 + t_1 \alpha_2)$, hence $\alpha_2$ lies in
this field, hence $\alpha_1$ does.  Therefore $K = F(\alpha_1 + t_1
\alpha_2)$. \end{proof}

\begin{cor} If $K/F$ is separable, it is primitive, and the generator may be
taken to be a linear combination of any finite set of generators of $K/F$.
\label{prim_elt}
\end{cor}

\begin{proof} We may embed $K/F$ in a Galois extension $M/F$ by adjoining all
the conjugates of its generators.  Subextensions of $K/F$ are as well subextensions
of $K'/F$ and by \ref{fixed_field} the map $H \mapsto (K')^H$ is a surjection
from the subgroups of $G$ to the subextensions of $K'/F$, which are hence
finite in number.  By \ref{fin_prim_elt} we may assume $F$ is infinite.  The
result now follows from \ref{gen_prim_elt}. \end{proof}

\begin{cor}
 If $K/F$ is Galois and $H \subset G$, then if $L = K^H$, we have $H = G_L$.
 \label{fixing_subgroup}
\end{cor}

\begin{proof}
 Let $\alpha$ be a primitive element for $K/L$.  The polynomial $\prod_{h \in H} (x - h(\alpha))$ is fixed by $H$, and therefore has coefficients in $L$, so $\alpha$ has $|H|$ conjugate roots over $L$.  But since $\alpha$ is primitive, we have $K = L(\alpha)$, so the minimal polynomial of $\alpha$ has degree $\deg(K/L)$, which is the same as the number of its roots.  Thus $|H| = \deg(K/L)$.  Since $H \subset G_L$ and $|G_L| = \deg(K/L)$, we have equality.
\end{proof}


\begin{thm} The correspondences $H \mapsto K^H$, $L \mapsto G_L$ define
inclusion-reversing inverse maps between the set of subgroups of $G$ and the
set of subextensions of $K/F$, such that normal subgroups and Galois subfields
correspond.
\label{fundamental_theorem}
\end{thm}

\begin{proof} This combines \ref{fixed_field}, \ref{fixing_subgroup}, and \ref{normal}.
\end{proof}

\end{document}
