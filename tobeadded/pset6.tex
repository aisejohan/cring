\documentclass[12pt, reqno]{amsart}

\usepackage{amsthm}
\usepackage{amsmath}
\usepackage{url}
\usepackage{fancyhdr}
\renewcommand{\thesection}{\arabic{section}}
\newtheorem{theorem}{Theorem}[section]

\usepackage{hyperref}

\newtheorem{lemma}[theorem]{Lemma}
\newtheorem{sublemma}[theorem]{Sublemma}
\newtheorem{corollary}[theorem]{Corollary}
\newtheorem{proposition}[theorem]{Proposition}
\theoremstyle{definition}
\newtheorem{definition}[theorem]{Definition}
\newtheorem*{remark}{Remark}
\newtheorem{example}[theorem]{Example}
\newtheorem{exercise}{ \sc Exercise}[chapter]
\newtheorem*{solution}{Solution}
\newtheorem*{question}{Question}
\newtheorem*{problem}{Problem}
\newtheorem*{dbend}{Dangerous bend}



\usepackage[top=1.3in, bottom=1.3in, left=1.5in, right=1.5in]{geometry}
\usepackage{amssymb}
\usepackage{amsfonts}
\usepackage{stackrel}
\usepackage{mathrsfs}
\usepackage{xy}
\usepackage{verbatim}


\newcommand{\whattosay}{}



\input xy
\xyoption{all}


\newcommand{\lecture}[1]{}
\newcommand{\rad}{\mathrm{rad}}
\newcommand{\im}{\mathrm{Im}}
\newcommand{\proj}{\mathrm{Proj}}
\renewcommand{\hom}{\mathrm{Hom}}
\newcommand{\id}{\mathrm{id}}
\providecommand{\cal}[1]{\mathcal{#1}}
\renewcommand{\cal}[1]{\mathcal{#1}}


%\swapnumbers

\renewcommand{\qedsymbol}{$\blacktriangle$}


\begin{document}

\title{Math 210A Problem Set 6}
\input{other/amsdata.tex}

\subsection*{1}

Suppose that an $n \times n$ matrix $A$ over a ring $R$ is invertible. This
means that there exists $A^{-1}$ so that $A A^{-1} = I$, so hence
$1 = \det I = \det(A A^{-1}) = (\det A) (\det A^{-1})$, and therefore, 
$\det A$ must be a unit in $R$.

Suppose instead that an $n \times n$ matrix $A$ over a ring $R$ is not
invertible. This means that there does not exist a matrix $B$ such that 
$A B = I$. For any ring element $r \in R$, we can define a matrix $A_r$ as the
matrix that is the identity matrix except that the $1,1$ position has value $r$
instead of 1. Then $\det A_r = r$, and $A A_r \ne I$, so that
$1 = \det T \ne \det A \det A_r = r \det A$. Since this is true for every $r
\in R$, we see that $\det A$ is not a unit in $R$.


\subsection*{2}

Suppose $A$ is an $n \times n$ matrix with real entries such that the diagonal
entries are all positive, the off-diagonal entries are all negative, and the
row sums are all positive. Write $A = (A_{ij})$.

Suppose to the contrary that $\det A = 0$, and consider the corresponding
system of equations $AX = 0$. This means that $AX = 0$ has a nontrivial
solution $(x_1, \dots, x_n)$. Therefore, we have
\[ 
\sum_{j=1}^n a_{kj} x_j = 0
\] 
for each $1 \le k \le n$.

Suppose that $x_i$ has the largest absolute value in $(x_1, x_2, \dots, x_n)$.
Then the $i$-th equation states that 
$a_{i1} x_1 + a_{i2} x_2 + \dots + a_{ii}x_i + \dots + a_{in} x_n = 0$.
For $j \ne i$, $a_{ij} < 0$. 

In the case $x_i > 0$, we see that 
$a_{ij} x_j \ge a_{ij} x_i$. 
Therefore, we see that 
$$
0 = a_{i1} x_1 + a_{i2} x_2 + \dots + a_{ii}x_i + \dots + a_{in} x_n
\ge (a_{i1} + a_{i2} + \dots + a_{in}) x_i > 0
$$
because the row sums of the matrix $A$ are all positive,
which is impossible.

In the case $x_i < 0$, we can do an analogous calculation to see that
$a_{ij} x_j \le a_{ij} x_i$, so that 
$$
0 = a_{i1} x_1 + a_{i2} x_2 + \dots + a_{ii}x_i + \dots + a_{in} x_n
\le (a_{i1} + a_{i2} + \dots + a_{in}) x_i < 0, 
$$
which is also impossible.

Therefore, we can conclude that the $x_i = 0$. However, $x_i$ was the maximal
entry of $(x_1, x_2, \dots, x_n)$, so therefore our nontrivial solution to $AX
= 0$ is actually $(0, 0, \dots, 0)$, which is trivial. Hence, 
we must conclude that $\det A \ne 0$.

\newpage

\subsection*{3(b)}

We have the diagram
\[ 
\xymatrix{
A \ar[r]^k \ar[d]^\a & B \ar[r]^l \ar[d]^\b 
	& C \ar[r]^m \ar[d]^\g & D \ar[r]^n \ar[d]^\d & E \ar[d]^\e  \\
F \ar[r]_p & G \ar[r]_q & H \ar[r]_r & I \ar[r]_s & J 
}
\] 
where the rows are exact at $B, C, D, G, H, I$ and the squares commute. In
addition, suppose that $\a, \b, \d, \e$ are isomorphisms. We will show that
$\g$ is an isomorphism.

\noindent
\emph{We show that $\g$ is surjective:}

Suppose that $h \in H$. Since $\d$ is surjective, there exists an element 
$d \in D$ such that $r(h) = \d(d) \in I$.
By the commutativity of the rightmost square, $s(r(h)) = \e(n(d))$. 
The exactness at $I$ means that $\im r = \ker s$, so hence
$\e(n(d)) = s(r(h)) = 0$. Because $\e$ is injective, $n(d) = 0$.
Then $d \in \ker(n) = \im(m)$ by exactness at $D$.
Therefore, there is some $c \in C$ such that $m(c) = d$.
Now, $\d(m(c)) = \d(d) = r(h)$ and by the commutativity of squares, 
$\d(m(c)) = r(\g(c))$, so therefore $r(\g(c)) = r(h)$. Since $r$ is a
homomorphism, $r(\g(c) - h) = 0$. Hence $\g(c) - h \in \ker r = \im q$ by
exactness at $H$.

Therefore, there exists $g \in G$ such that $q(g) = \g(c) - h$.
$\b$ is surjective, so there is some $b \in B$ such that $\b(b) = g$ and hence
$q(\b(b)) = \g(c) - h$. By the commutativity of squares, 
$q(\b(b)) = \g(l(b)) = \g(c) - h$. Hence 
$h = \g(c) - \g(l(b)) = \g(c - l(b))$, and therefore $\g$ is surjective.

So far, we've used that $\b$ and $\g$ are surjective, $\e$ is injective, and
exactness at $D$, $H$, $I$.

\noindent
\emph{We show that $\g$ is injective:}

Suppose that $c \in C$ and $\g(c) = 0$.
Then $r(\g(c)) = 0$, and by the commutativity of squares, 
$\d(m(c)) = 0$. Since $\d$ is injective, $m(c) = 0$, so
$c \in \ker m = \im l$ by exactness at $C$. 
Therefore, there is $b \in B$ such that $l(b) = c$.
Then $\g(l(b)) = \g(c) = 0$, and by the commutativity of squares, 
$q(\b(b)) = 0$. Therefore, $\b(b) \in \ker q$, and by exactness at $G$, 
$\b(b) \in \ker q = \im p$.

There is now $f \in F$ such that $p(f) = \b(b)$. Since $\a$ is surjective, this
means that there is $a \in A$ such that $f = \a(a)$, so then 
$\b(b) = p(\a(a))$. By commutativity of squares, 
$\b(b) = p(\a(a)) = \b(k(a))$, and hence $\b(k(a) - b) = 0$.
Since $\b$ is injective, we have $k(a) -b = 0$, so $k(a) = b$.
Hence $b \in \im k = \ker l$ by commutativity of squares, so $l(b) = 0$.
However, we defined $b$ to satisfy $l(b) = c$, so therefore $c = 0$ and hence
$\g$ is injective.

Here, we used that $\a$ is surjective, $\b, \d$ are injective, and exactness at
$B, C, G$.

Putting the two statements together, we see that $\g$ is both surjective and
injective, so $\g$ is an isomorphism. We only used that $\b, \d$ are
isomorphisms and that $\a$ is surjective, $\e$ is injective, so we can slightly
weaken the hypotheses; injectivity of $\a$ and surjectivity of $\e$ were
unnecessary.

\newpage

\subsection*{4}

Consider
\[ 
\xymatrix{
\dots \ar[r] & 0 \ar[r] \ar[d] & 0 \ar[r] \ar[d] & 0 \ar[r] \ar[d] & \dots \\
\dots \ar[r] & A^{i-1} \ar[r]^{f^{i-1}} \ar[d]_{\a_{i-1}}
	& A^i \ar[r]^{f^i} \ar[d]^{\a_i} 
	& A^{i+1} \ar[r]^{f^{i+1}} \ar[d]^{\a_{i+1}} & \dots \\
\dots \ar[r] & B^{i-1} \ar[r]^{g^{i-1}} \ar[d]_{\b_{i-1}}
	& B^i \ar[r]^{g^i} \ar[d]^{\b_i} 
	& B^{i+1} \ar[r]^{g^{i+1}} \ar[d]^{\b_{i+1}} & \dots \\
\dots \ar[r] & C^{i-1} \ar[r]^{h^{i-1}} \ar[d]
	& C^i \ar[r]^{h^i} \ar[d]
	& C^{i+1} \ar[r]^{h^{i+1}} \ar[d] & \dots \\
\dots \ar[r] & 0 \ar[r] & 0 \ar[r] & 0 \ar[r] & \dots
}
\] 

Consider any $c \in C^{i-1}$ that represents the class 
$x \in H^{i-1}(C) = \ker(h^{i-1}) / \im (h^{i-2})$. 
Then, since $\b_{i-1}$ is surjective, there exists $b \in B^{i-1}$ such that 
$\b_{i-1}(b) = c$. Now, since $c \in \ker(h^{i-1})$, we have
$h^{i-1} (c) = 0$, so that $h^{i-1} (\b_{i-1} (b)) = 0$.
By commutativity of squares, $\b_i (g^{i-1}(b)) = h^{i-1} (\b_{i-1} (b)) = 0$. 
Therefore $g^{i-1} (b) \in \ker \b_i = \im \a_i$ by the exactness at $B^i$.
Since $\a_i$ is injective, there exists a unique $a \in A^i$ such that
$\a_i(a) = g^{i-1}(b)$.

Since $B$ is a complex, $g^i(\a_i(a)) = g^i (g^{i-1}(b)) = 0$. By the
commutativity of the diagram, $\a_{i+1} (f^i(a)) = g^i (\a_i(a)) = 0$. Since
$\a_{i+1}$ is injective, we have $f^i(a) = 0$ and hence 
$a \in \ker f^i$. $a$ therefore defines a class $\ol a$ in the quotient group
$H^i(A) = \ker(f^i) / \im(f^{i-1})$. 

We want to show that $\ol a$ is independent of the choice of $b$. Suppose that
$b'$ is another choice of $b$ satisfying $\b_{i-1}(b) = \b_{i-1}(b') = c$.
This means that $\b_{i-1}(b-b') = 0$, so 
$b - b' \in \ker \b_{i-1} = \im  \a_{i-1}$ by the exactness at $B^{i-1}$.
Since $\a_{i-1}$ is injective, there exists a unique 
$\ul a \in A^{i-1}$ such that $\a_{i-1}(\ul a) = b-b'$. Therefore, by the
commutativity of the  diagram, 
$\a_i(f^{i-1}(\ul a)) = g^{i-1}(\a_{i-1}(\ul a)) = g^{i-1}(b-b')$.

Now, our choice of $b'$ leads by the previous reasoning to a new value of $a'$.
Here, $a$ and $a'$ differ by an amount $a - a'$ satisfying
$\a_i(a - a') = \a_i(a) - \a_i(a') = g^{i-1}(b) - g^{i-1}(b') 
= g^{i-1}(b-b') = \a_i (f^{i-1}(\ul a))$.
Since $\a_i$ is injective, this means that $a - a' = f^{i-1}(\ul a)$, so hence
$a - a' \in \im f^{i-1}$, and therefore $a$ and $a'$ are in the same class of
$H^i (A)$.

%We also want to show that $\ol a$ is independent of the choice of $c$
%representing the class $x$. Consider any $\tilde c$ 
%representing the same class $x$. 
%There then exists $\ul b \in B^{i-1}$ such that 
%$\b_{i-1}(\ul b) = c - \tilde c$. Since $c$ and $\tilde c$ are in the same
%class, $c - \tilde c \in \im h^{i-2} \subset \ker h^{i-1}$ because $C$ is a
%complex. This means that 
%$\b_i (g^{i-1}(\ul b)) = h^{i-1} (\b_{i-1} (\ul b)) = 0$ by the commutativity
%of the diagram. %this isn't enoguh!!!

%There then exists $\tilde b \in B^{i-1}$ such
%that $\b_{i-1}(\tilde c) = \tilde c$; as shown above, this is independent of
%the choice of $\tilde b$. Proceeding as before, we see that 
%$\b_i (g^{i-1}(b)) = \b_i (g^{i-1}(\tilde b)) = 0$.
%Therefore, $g^{i-1}(b - \tilde b) \in \ker \b_i = \im \a_i$ because of the
%exactness at $B^i$.

%At this point: Forget about this and move on!




\newpage

\subsection*{5}

Suppose that 
$ 
\xymatrix{
M' \ar[r]^f & M \ar[r]^g & M'' 
}
$
is exact. This means that $\im(f) = \ker(g)$, which 
is equivalent to saying that 
$0 \to \im(f) \to M \to M/\ker(g) \to 0$
is a short exact sequence.

Since the functor $F$ preserves short exact sequences, this means that
$$0 \to F(\im(f)) \to F(M) \to F(M / \ker(g)) \to 0$$ is a short exact sequence.

Now, $F(\im(f)) = \im(F(f))$ and $F ( M / \ker(g)) = F(M) / \ker(F(g))$, so that
$$0 \to \im(F(f)) \to F(M) \to F(M) / \ker(F(g)) \to 0$$ is a short exact
sequence and hence 
$
\xymatrix{
F(M') \ar[r]^{F(f)} & F(M) \ar[r]^{F(g)} & F(M'') 
}
$
is exact.



\subsection*{6(a)}

Suppose that $M' \xrightarrow f M \xrightarrow g  M''$ is exact at $M$.
We want to show that 
\[ 
\xymatrix{
S^{-1}M' \ar[r]^{S^{-1}f} & S^{-1}M \ar[r]^{S^{-1}g} & S^{-1}M'' 
}
\] 
is exact at $S^{-1}M$, where
$S^{-1}f(m'/s) = f(m')/s$ and $S^{-1}g(m/s) = g(m)/s$. In particular, this
means that $S^{-1} (v \circ u) = S^{-1}(v) \circ S^{-1}(u)$.

We have $g \circ f = 0$, so therefore 
$S^{-1}g \circ S^{-1}f = S^{-1}0 = 0$, and hence 
$\im(S^{-1}f) \subset \ker(S^{-1}g)$.

To show the reverse inclusion, let $m/s \in \ker(S^{-1}g)$, so that
$S^{-1}g(m/s) = g(m)/s = 0$. Therefore, there exists some $t \in S$ such that
$t g(m) = 0$ in $M''$. Since $g$ is a homomorphism, $tg(m) = g(tm) = 0$, and
therefore $tm \in \ker g = \im f$ by the exactness at $M$. Therefore, there
exists $m' \in M'$ such that $f(m') = tm$.
We therefore see that in $S^{-1}M$, 
$m/s = f(m')/st = S^{-1}f(m'/st)$, so therefore $m/s \in \im(S^{-1}f)$. This
shows that $\ker(S^{-1}g) \subset \im(S^{-1}f)$, and hence that
$\ker (S^{-1}g) = \im (S^{-1}f)$. This implies the desired exactness at 
$S^{-1}m$.


\subsection*{6(b)}

Suppose that $0 \to A \xrightarrow u B \xrightarrow v C \to 0$ is exact.
We want to show that $\Hom(M, \cdot)$ is a left-exact covariant functor, 
which means that 
$$
0 \to \Hom(M, A) \xrightarrow{\ol u} \Hom(M, B) \xrightarrow{\ol v} \Hom(M, C)
$$ 
is exact. Here, if $f \in \Hom(M,A)$ and $g \in \Hom(M, B)$, we define 
$\ol u$ and $\ol v$ via $\ol u(f) = u \circ f$ and 
$\ol v(g) = v \circ g$.

Suppose that $f \in \Hom(M, A)$. If $u \circ f = 0$ then 
$(u \circ f)(m) = 0$ for all $m \in M$. Since $u$ is an injection, we see that
$f(m) = 0$ for all $m$. Therefore, $\ol u (f) = 0$ implies that $f = 0$, and
hence $\ol u:\Hom(M, A) \to \Hom(M, B)$ is an injection.

Observe that $v \circ u = 0$ by the exactness at $B$. 
Any element of $\im(\ol u)$ can be written as $\ol u(f)$ for some
$f \in \Hom(M, A)$. This satisfies 
$\ol v (\ol u (f)) = v \circ u \circ f = 0 \circ f = 0$, so hence
$\ol u(f) \in \ker (\ol v)$ and therefore $\im(\ol u) \subset \ker (\ol v)$.

Now, suppose that $g \in \ker(\ol v) \subset \Hom(M, B)$. Then 
$\ol v(g) = v \circ g = 0$. Therefore, 
$v(g(M)) = 0$ so that $g(M) \subset \ker v = \im u$ by the exactness at $B$.
Since $u$ is injective, it is an isomorphism onto its image, so therefore 
$f = u^{-1} \circ g \in \Hom(M, A)$ is well-defined.
Therefore, $\ol u(f) = u \circ f = g$, and hence $g \in \im (\ol u)$.
Therefore, $\ker(\ol v) \subset \im (\ol u)$, and hence
$\ker (\ol v) = \im(\ol u)$ and we have our desired exactness at $\Hom(M, B)$.


\subsection*{6(c)}

Suppose that $0 \to A \xrightarrow u B \xrightarrow v C \to 0$ is exact.
We want to show that $\Hom(\cdot, M)$ is a left-exact contravariant functor, 
which means that 
$$
0 \to \Hom(C, M) \xrightarrow{\ol v} \Hom(B, M) \xrightarrow{\ol u} \Hom(A, M)
$$ 
is exact. Here, if $f \in \Hom(B,M)$ and $g \in \Hom(C, M)$, we define 
$\ol u$ and $\ol v$ via $\ol v(g) = g \circ v$ and 
$\ol u(f) = f \circ u$.

Suppose that $g \in \Hom(C, M)$. If $\ol v(g) = g \circ v = 0$ then 
$(g \circ v)(b) = 0$ for all $b \in B$. Since $v$ is a surjection, this means
that $g(C) = 0$ and hence $g = 0$. Therefore, $\ol(v)$ is injective, and we
have exactness at $\Hom(C, M)$.

Observe that $v \circ u = 0$ by the exactness at $B$. 
Any element of $\im(\ol v)$ can be written as $\ol v(g)$ for some
$g \in \Hom(C, M)$. This satisfies 
$\ol u (\ol v (g)) = g \circ v \circ u = g \circ 0 = 0$, so hence
$\ol v(g) \in \ker (\ol u)$ and therefore $\im(\ol v) \subset \ker (\ol u)$.

Now, suppose that $f \in \ker(\ol u) \subset \Hom(B, M)$. Then 
$\ol u(f) = f \circ u = 0$. Therefore, 
$f(u(A)) = 0$, so that $f(\im u) = 0$. Now, since $\ker v = \im u$ by the
exactness at $B$, we have  
$C \cong B / \ker v = B / \im u$, so that we have a well-defined map
$v^{-1} : C \to B / \im u$.
Combined with $f (\im u) = 0$, we can define a well-defined map 
$g = f \circ v^{-1} \in \Hom(C, M)$. This new map satisfies $f = g \circ v$
Therefore, $\ol v(g) = g \circ v = f$, and hence $f \in \im (\ol v)$.
Therefore, $\ker(\ol u) \subset \im (\ol v)$, and hence
$\ker (\ol u) = \im(\ol v)$ and we have our desired exactness at $\Hom(B, M)$.



\end{document}
\ker v = \im u$ by the
exactness at $B$, we have  
$C \cong B / \ker v = B / \im u$, so that we have a well-defined map
$v^{-1} : C \to B / \im u$.
Combined with $f (\im u) = 0$, we can define a well-defined map 
$g = f \circ v^{-1} \in \Hom(C, M)$. This new map satisfies $f = g \circ v$
Therefore, $\ol v(