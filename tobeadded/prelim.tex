\usepackage{amsmath}
\usepackage{amssymb}
\usepackage{amsfonts}
\usepackage{amsthm}
\usepackage{array}
\usepackage{bm}
\usepackage{wrapfig}
\usepackage[pdftex]{color}
\usepackage[pdftex]{graphicx}

%\usepackage{pdftricks}
%\begin{psinputs}
%\usepackage[dvips,ps,all]{xy}
%\end{psinputs}

\input xy
\xyoption{all}

%% This package sets the date to the more logical YYYY - MM - DD format.
\usepackage{datetime}
\renewcommand{\dateseparator}{-}
\yyyymmdddate

%% This command inserts \noindent and makes the input bold.
\newcommand{\num}[1]{\noindent \textbf{#1}}

%% Some math commands.
\renewcommand{\ker}{\operatorname{ker}}
\newcommand{\im}{\operatorname{im}}
\newcommand{\coker}{\operatorname{coker}}
\newcommand{\disc}{\operatorname{disc}}
\newcommand{\id}{\operatorname{id}}
\newcommand{\rad}{\operatorname{rad}}
\newcommand{\Gal}{\operatorname{Gal}}
\newcommand{\Aut}{\operatorname{Aut}}
\newcommand{\Irr}{\operatorname{Irr}}
\newcommand{\Char}{\operatorname{char}}
\newcommand{\Hom}{\operatorname{Hom}}
\newcommand{\End}{\operatorname{End}}
\newcommand{\Ann}{\operatorname{Ann}}
\newcommand{\Ass}{\operatorname{Ass}}
\newcommand{\Supp}{\operatorname{Supp}}
\newcommand{\Frac}{\operatorname{Frac}}
\newcommand{\Spec}{\operatorname{Spec}}
\newcommand{\mSpec}{\operatorname{mSpec}}
\renewcommand{\dim}{\operatorname{dim}}
\newcommand{\codim}{\operatorname{codim}}
\newcommand{\height}{\operatorname{ht}}
\newcommand{\length}{\operatorname{length}}
\newcommand{\depth}{\operatorname{depth}}
\newcommand{\Ext}{\operatorname{Ext}}
\newcommand{\Tor}{\operatorname{Tor}}

%% Some categories.
\newcommand{\Set}{\ensuremath{\text{\sf Set}}}
\newcommand{\Cat}{\ensuremath{\text{\sf Cat}}}
\newcommand{\Top}{\ensuremath{\text{\sf Top}}}
\newcommand{\Grp}{\ensuremath{\text{\sf Grp}}}
\newcommand{\Ring}{\ensuremath{\text{\sf Ring}}}
\newcommand{\CRing}{\ensuremath{\text{\sf CRing}}}
\newcommand{\Ab}{\ensuremath{\text{\sf Ab}}}
\newcommand{\Mod}{\ensuremath{\text{\sf Mod}}}
\newcommand{\Fun}{\ensuremath{\text{\sf Fun}}}
\newcommand{\Nat}{\ensuremath{\text{\sf Nat}}}
\newcommand{\Vect}{\ensuremath{\text{\sf Vect}}}

%% Shortening of some standard commands.
\newcommand{\bb}[1]{\ensuremath{\mathbb{#1}}}
\renewcommand{\cal}[1]{\ensuremath{\mathcal{#1}}}
\newcommand{\san}[1]{\ensuremath{\text{\sf #1}}}
\renewcommand{\frak}[1]{\ensuremath{\mathfrak{#1}}}

%% Standard double-struck letters.
%% Integers.
\newcommand{\Z}{\ensuremath{\mathbb{Z}}}
%% Rational numbers.
\newcommand{\Q}{\ensuremath{\mathbb{Q}}}
%% Real numbers.
\newcommand{\R}{\ensuremath{\mathbb{R}}}
%% Complex numbers.
\newcommand{\C}{\ensuremath{\mathbb{C}}}
%% Natural numbers.
\newcommand{\N}{\ensuremath{\mathbb{N}}}
%% The sphere.
\let\SS\S\renewcommand{\S}{\ensuremath{\mathbb{S}}}
%% A field.
\newcommand{\F}{\ensuremath{\mathbb{F}}}
%% The quaternions.
\renewcommand{\H}{\ensuremath{\mathbb{H}}}
%% The octonions.
\renewcommand{\O}{\ensuremath{\mathbb{O}}}
%% Projective space, or probability.
\renewcommand{\P}{\ensuremath{\mathbb{P}}}
%% The torus.
\newcommand{\T}{\ensuremath{\mathbb{T}}}
%% Affine space.
\newcommand{\A}{\ensuremath{\mathbb{A}}}
%% The ball.
\newcommand{\B}{\ensuremath{\mathbb{B}}}
%% The disk.
\newcommand{\D}{\ensuremath{\mathbb{D}}}

%% Non-standard double-struck letters.
%\newcommand{\E}{\ensuremath{\mathbb{E}}}
%\newcommand{\G}{\ensuremath{\mathbb{G}}}
%\newcommand{\I}{\ensuremath{\mathbb{I}}}
%\newcommand{\J}{\ensuremath{\mathbb{J}}}
%\newcommand{\K}{\ensuremath{\mathbb{K}}}
%\renewcommand{\L}{\ensuremath{\mathbb{L}}}
%\newcommand{\M}{\ensuremath{\mathbb{M}}}
%\newcommand{\N}{\ensuremath{\mathbb{N}}}
%\newcommand{\U}{\ensuremath{\mathbb{U}}}
%\newcommand{\V}{\ensuremath{\mathbb{V}}}
%\newcommand{\W}{\ensuremath{\mathbb{W}}}
%\newcommand{\X}{\ensuremath{\mathbb{X}}}
%\newcommand{\Y}{\ensuremath{\mathbb{Y}}}


%% This custom type produces a column of the specified width whose contents are centered.
\newcolumntype{C}[1]{>{\centering\hspace{0pt}}p{#1}}

%% amsthm styles.
\newtheorem{theorem}{Theorem}
\newtheorem{thm}{Theorem}
\newtheorem{cor}{Corollary}
\newtheorem{lemma}{Lemma}
\theoremstyle{definition}
\newtheorem{axiom}{Axiom}[section]
\newtheorem{definition}{Definition}
\newtheorem*{remark}{Remark}