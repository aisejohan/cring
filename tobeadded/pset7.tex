\documentclass[12pt, reqno]{amsart}

\usepackage{amsthm}
\usepackage{amsmath}
\usepackage{url}
\usepackage{fancyhdr}
\renewcommand{\thesection}{\arabic{section}}
\newtheorem{theorem}{Theorem}[section]

\usepackage{hyperref}

\newtheorem{lemma}[theorem]{Lemma}
\newtheorem{sublemma}[theorem]{Sublemma}
\newtheorem{corollary}[theorem]{Corollary}
\newtheorem{proposition}[theorem]{Proposition}
\theoremstyle{definition}
\newtheorem{definition}[theorem]{Definition}
\newtheorem*{remark}{Remark}
\newtheorem{example}[theorem]{Example}
\newtheorem{exercise}{ \sc Exercise}[chapter]
\newtheorem*{solution}{Solution}
\newtheorem*{question}{Question}
\newtheorem*{problem}{Problem}
\newtheorem*{dbend}{Dangerous bend}



\usepackage[top=1.3in, bottom=1.3in, left=1.5in, right=1.5in]{geometry}
\usepackage{amssymb}
\usepackage{amsfonts}
\usepackage{stackrel}
\usepackage{mathrsfs}
\usepackage{xy}
\usepackage{verbatim}


\newcommand{\whattosay}{}



\input xy
\xyoption{all}


\newcommand{\lecture}[1]{}
\newcommand{\rad}{\mathrm{rad}}
\newcommand{\im}{\mathrm{Im}}
\newcommand{\proj}{\mathrm{Proj}}
\renewcommand{\hom}{\mathrm{Hom}}
\newcommand{\id}{\mathrm{id}}
\providecommand{\cal}[1]{\mathcal{#1}}
\renewcommand{\cal}[1]{\mathcal{#1}}


%\swapnumbers

\renewcommand{\qedsymbol}{$\blacktriangle$}


\begin{document}

\title{Math 210A Problem Set 7}
\input{other/amsdata.tex}

\subsection*{1}

We want to show that $\cdot \otimes_R S^{-1}R$ is an exact functor. 

Suppose that $0 \to M' \to M \to M'' \to 0$ is a short exact sequence. 
In problem 6 of problem set 6, we proved that localization is exact, so hence
$0 \to S^{-1}M' \to S^{-1}M \to S^{-1}M'' \to 0$ is a short exact sequence.

We proved in problem 5 of problem set 5 that 
$S^{-1}R \otimes_R M \cong S^{-1}M$, and it was stated in class that 
$M \otimes_R N \cong N \otimes_R M$, so therefore
$M \otimes_R S^{-1}R \cong S^{-1}M$. Hence, we can conclude that
$0 \to M' \otimes_R S^{-1}R \to M \otimes_R S^{-1}R \to M''
\otimes_R S^{-1}R \to 0$ is a short exact sequence, and therefore we have shown
that $\cdot \otimes_R S^{-1}R$ is a short exact sequence.


\subsection*{2 (DF 17.1.5)}
\[ 
\xymatrix{
 & 0 \ar[d] & 0 \ar[d] & 0 \ar[d] & \\
\cdots \ar[r] & P_1 \ar[r] \ar[d] 
	& P_0 \ar[r]^h \ar[d]_{\f} & L \ar[r] \ar[d]^f & 0 \\
\cdots \ar[r] & P_1 \oplus \ol{P_1} \ar[r] \ar[d] 
	& P_0 \oplus \ol{P_0} \ar[r]^\pi \ar[d]_\psi & M \ar[r] \ar[d]^g & 0 \\
\cdots \ar[r] & \ol{P_1} \ar[r] \ar[d] 
	& \ol{P_0} \ar[r]^{h'} \ar[d] \ar[ur]^\m & N \ar[r] \ar[d] & 0 \\
& 0 & 0 & 0 &
}
\] 

We want to show that the two squares on the right commute.

Here, $\m : \ol{P_0} \to M$ is defined as a lifting of the map $\ol{P_0} \to N$
via the definition of projective module, so that $h' = g \circ \m$. 
Let $\l$ be the map $P_0 \to L \to M$
given by $\l = f \circ h$. Define $\pi : P_0 \otimes \ol{P_0}$ via
$\pi(x, y) = \l(x) + \m(y)$. 

Now, for any $x \in P_0$, 
$(\pi \circ \f)(x) = \pi(x, 0) = \l(x) + \m(0) = \l(x) = (f \circ h)(x)$, so
therefore $f \circ h = \pi \circ \f$ and the top square commutes.

For any $(x, y) \in P_0 \oplus \ol{P_0}$, 
$(g \circ \pi)(x, y) = g(\l(x) + \m(y)) = g(f(h(x)) + g(\m(y)) = g(\m(y))$ 
because
$g \circ f = 0$ by exactness at $M$.
In addition, 
$(h' \circ \psi)(x, y) = h'(y) = g(\m(y))$, so therefore $g \circ \pi = h'
\circ \psi$, so the bottom square commutes as well.


\subsection*{3 (DF 17.1.9)}

We want to show that 
\[ 
\xymatrix{
0 \ar[r] & \ZZ/d\ZZ \ar[r] & \ZZ/m\ZZ \ar[r]^d & \ZZ/m\ZZ \ar[r]^{m/d} &
	\ZZ/m\ZZ \ar[r]^d & \ZZ/m\ZZ \ar[r]^{m/d} & \cdots
}
\] 
is an injective resolution of $\ZZ / d\ZZ$ as a $\ZZ / m \ZZ$-module.
Here, exactness is obvious because 
$(m/d) \cdot d = d \cdot (m/d) = m$ sends every element of
$\ZZ/m\ZZ$ to $0$. 

Baer's Criterion states that a module $Q$ is injective if and only if for every
ideal $I$ of $R$, every $R$-module homomorphism $g : I \to Q$ can be extended to
an $R$-module homomorphism $G : R \to Q$. This is clearly true for $\ZZ /
m\ZZ$, so therefore $\ZZ / m\ZZ$ is an injective $\ZZ/m\ZZ$-module, and hence
we indeed have an injective resolution.

%explain why it is clear?

Now apply $\Hom(A, \cdot)$ and truncate to get the complex
\[ 
\xymatrix{
0 \ar[r] & \Hom_{\ZZ/m\ZZ}(A, \ZZ/m\ZZ) \ar[r]^d & 
	\Hom_{\ZZ/m\ZZ}(A, \ZZ/m\ZZ) \ar[r]^{m/d} &
	\Hom_{\ZZ/m\ZZ}(A, \ZZ/m\ZZ) \ar[r] &  \cdots
}
\] 
The groups $\Ext_{\ZZ/m\ZZ}^n (A, \ZZ/d\ZZ)$ are the homology groups of this
complex.

Since $\Hom_{\ZZ/m\ZZ}(A, \ZZ/m\ZZ) \cong A$, we see that we actually have a
complex  % I don't believe this
\[ 
\xymatrix{
0 \ar[r] & A \ar[r]^d & 
	A \ar[r]^{m/d} &
	A \ar[r]^d  & 
	A \ar[r]^{m/d}  &  \cdots
}
\] 
In the case of $m = p^2$ and $d=p$, we get the complex
\[ 
\xymatrix{
0 \ar[r] & \ZZ/p^2\ZZ \ar[r]^p & 
	\ZZ/p^2\ZZ \ar[r]^p &
	\ZZ/p^2\ZZ \ar[r]^p & 
	\ZZ/p^2\ZZ \ar[r]^p &  \cdots
}
\]
This has homology $\ZZ / p\ZZ$ at each point of the sequence, so therefore, 
$\Ext_{\ZZ/p^2\ZZ}^n (\ZZ / p\ZZ, \ZZ/p\ZZ) \cong \ZZ / p \ZZ$.
%make clearer... am i sure about this?


\subsection*{4 (DF 17.1.18)}

Consider the diagram
\[ 
\xymatrix{
\ZZ/2\ZZ \ar[d]^{\exists!}_h \ar[rd]^f & \\
N & M \ar[l]^g
}
\] 
We want to show that  $\ZZ / 2\ZZ$ is a projective $\ZZ / 6\ZZ$-module. Suppose
that there exists a map $f : \ZZ / 2\ZZ \to M$. This is uniquely determined
by the image of 1, since $f(0) = 0$. Suppose that $f(1) = m$. 
Then, given a map $g : M \to N$, we can
consider the composition $g \circ f : \ZZ / 2\ZZ \to M \to N$ defined via
$(g \circ f) (1) = g(m)$. Here, multiplication by even elements of 
$\ZZ / 6\ZZ$ acts on $1 \in \ZZ / 2\ZZ$ by sending it to zero and
multiplication by odd elements of $\ZZ / 6\ZZ$ leaves $1 \in \ZZ / 2\ZZ$
unchanged.

Therefore, there exists a map $h : \ZZ / 2\ZZ \to N$ that is defined via
$h = g \circ f$.
%Suppose there were another such map $h' : \ZZ / 2\ZZ \to N$. Both are uniquely
%determined by the image of $1$, so suppose $h(1) = n$ and $h'(1) = n'$. Then 
%$h(1) - h'(1) = (h - h')(1) = n - n'$.
%why uniqueness???

Now, consider a projective resolution of $\ZZ / 2\ZZ$, i.e.
\[ 
\cdots \to P_2 \to P_1 \to P_0 \to \ZZ / 2\ZZ \to 0,
\] 
apply $\cdot \otimes \ZZ / 2\ZZ$, and truncate to get
\[ 
\cdots \to P_2 \otimes \ZZ /2\ZZ \to P_1 \otimes \ZZ / 2\ZZ 
	\to P_0 \otimes \ZZ/2\ZZ \to 0,
\] 
Because $\ZZ / 2\ZZ$ was projective, it is therefore flat, so this sequence is
exact and therefore the homology group is 
$\Tor_1^{\ZZ/6\ZZ} (\ZZ / 2\ZZ , \ZZ / 2\ZZ) = 0$.

\subsection*{5 (DF 17.1.20)}

Consider the projective resolution
\[ 
\xymatrix{
\cdots \ar[r]^{m/d} & \ZZ/m\ZZ \ar[r]^d & \ZZ/m\ZZ \ar[r]^{m/d} 
	%& \ZZ/m\ZZ \ar[r]^d & \ZZ/m\ZZ \ar[r]^{m/d} 
	& \ZZ/m\ZZ \ar[r]^{d} & \ZZ/m\ZZ \ar[r] & \ZZ/d\ZZ \ar[r] & 0.
}
\]
We apply $A \otimes_{\ZZ/m\ZZ} \cdot$ and truncate. Since 
$S \otimes_R R \cong S$, we see that $A \otimes_{\ZZ/m\ZZ} \ZZ /m\ZZ \cong A$ and
hence we obtain the complex
\[ 
\xymatrix{
\cdots \ar[r]^{m/d} & A \ar[r]^d & A \ar[r]^{m/d} 
	& A \ar[r]^{d} & A \ar[r] & 0.
}
\]
The groups $\Tor_n^{\ZZ/m\ZZ} (A, \ZZ/d\ZZ)$ are simply the homology groups
(ker/im) of
the complex, which are simply 
\begin{align*} 
\Tor_0^{\ZZ / m\ZZ} (A, \ZZ/d\ZZ) &\cong A / dA \\
\Tor_n^{\ZZ / m\ZZ} (A, \ZZ/d\ZZ) &\cong {}_dA/(m/d)A 
			\quad \text{$n$ odd, $n \ge 1$} \\
\Tor_n^{\ZZ / m\ZZ} (A, \ZZ/d\ZZ) &\cong {}_{m/d}A/dA 
			\quad \text{$n$ even, $n \ge 2$},
\end{align*} 
where ${}_kA = \{ a \in A \mid ka = 0 \}$ denotes the set of elements of $A$
killed by $k$.

\subsection*{6 (DF 17.1.22)}

Consider a projective resolution of $A$, i.e.
\[ 
\cdots \to P_2 \to P_1 \to P_0 \to A \to 0.
\] 
Assume that $S$ is a flat $R$-module. Apply $S \otimes_R \cdot$ and truncate to
get the sequence
\[ 
\cdots \to S \otimes_R P_2 \to S \otimes_R P_1 \to S\otimes_R P_0  \to 0.
\] 
We want to show that this is an $S$-module projective resolution of $S
\otimes_R A$. First, each $S \otimes_R P_n$ is an $S$-module since this is a
change of base. 

Since $S$ is flat, the sequence remains exact when viewed as $R$-modules.
In addition, each $S \otimes_R P_n$ is projective because the tensor product of
a flat module and a projective module is projective.
%why? The last statement is nontrivial.

Therefore, the projective resolution of $A$ did indeed
give a projective resolution of $S \otimes_R A$.

\end{document}
his is an $S$-module projective resolution of $S
\otimes_R A$. First, each $S \