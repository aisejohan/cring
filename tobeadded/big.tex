

\subsection{$\spec R$ and the Zariski topology}

\lecture{[Section] 9/12}

\subsection{The ideal class group}

This was taught by X. Wang at a recitation.

\begin{example} 
Consider the nonprincipal ideal $I=(2 , 1+\sqrt{-5}) \subset
\mathbb{Z}[\sqrt{-5}]$.  It is nonprincipal (exercise), but its square is
$(2)$, which is principal.  So $I$  is not principal, but its square is.  How
can we make this more general? 
\end{example} 

\newcommand{\cl}{\mathrm{Cl}}

In an integral domain $R$, we define $\cl(R)$ to be the set of all nonzero ideals $I
\subset R$ modulo the relation that $I \sim J$ if there is $\alpha, \beta \in
R^*$ such that $\alpha I = \beta J$.

\begin{definition} 
We define $\cl(R)$ to be the \textbf{ideal class set} of $R$.
\end{definition} 

We need to check that this is a group.
Clearly we can define a notion of multiplication by multiplying ideals. The
unit ideal is the unit element.  
But do inverses exist?  Given $I$, is there an ideal $J$ such that
\[ IJ \ \mathrm{is} \ \mathrm{principal?} \]

\subsection{Dedekind domains}
\begin{theorem} 
Let $R$ be a domain such that:
\begin{enumerate}
\item $R$ is noetherian (i.e. every ideal is finitely generated).
\item Every nonzero prime ideal of $R$ is maximal. (I.e. $R$ has Krull
dimension one.)
\item $R$ is integrally closed.
\end{enumerate}
Then the ideal class set is a group.  
\end{theorem} 

\begin{proof} 
We need to prove that for any $I$, there is another ideal $J$ such that $IJ $
is prinicpal.  Pick $I \subset R$.  We can assume $I \subsetneq R$.  Take an
element $\alpha \in I - \left\{0\right\}$.  If we can find an ideal $J$ such
that $IJ = (\alpha)$, we should take $J$ to be the set
\[ J = \left\{x: xI \subset (\alpha) \right\}.  \]
We have $IJ \subset (\alpha)$.
Now consider $\mathfrak{b}=\frac{1}{\alpha} IJ$; this is an ideal, because it is a submodule
of $R$.  We are to prove that it is equal to $R$.

Suppose not. Suppose $c \in \mathfrak{b} - \left\{0\right\}$.  We know that
$\mathfrak{b}$ is contained in a maximal ideal $\mathfrak{m}$ since
$\mathfrak{b} \neq R$ by assumption. We have a chain of ideals
\[ (c) \subset \mathfrak{b} \subset \mathfrak{m}.  \]

\begin{lemma} 
If $R$ is noetherian, then every ideal contains a product of nonzero prime
ideals.
\end{lemma} 
\begin{proof} 
Consider the set of all ideals that do not contain a product of nonzero
primes.  Then this set $S$ has a maximal element if it is nonempty.  Call this
element $\mathfrak{n}$.  Clearly $\mathfrak{n}$ isn't prime or it wouldn't be
in $S$.  
This means that there are $a,b \notin \mathfrak{n}$ with $ab \in
\mathfrak{n}$.  In particular, if we look at the ideals
\[ \mathfrak{n} + (a), \mathfrak{n} + (b) \supsetneq \mathfrak{n}  \]
then these contain products of primes. So their product
\[ (\mathfrak{n} + (a))(\mathfrak{n}+(b)) \subset \mathfrak{n}  \]
contains a product of primes. Thus $\mathfrak{n}$ contains a product of primes.  
\end{proof} 


In particular, $(c)$ contains a product of primes $\mathfrak{p}_1 \dots
\mathfrak{p}_r$. Suppose $r$ is the minimal possible so $(c)$ does not contain
any product of $r-1$ ideals.  
We have that 
\[ \mathfrak{p}_1 \dots \mathfrak{p}_r \subset \mathfrak{m}  \]
so one $\mathfrak{p}_i$, wlog $\mathfrak{p}_1$, must lie in the prime (and
maximal) ideal $\mathfrak{m}$.  

But every nonzero prime ideal is maximal, so $\mathfrak{m} = \mathfrak{p}_1$.  

\begin{lemma} 
Under the above hypotheses, there is $\gamma \in K - R$ with $\gamma
\mathfrak{b} \subset R$.
\end{lemma} 
\begin{proof} 
Suppose $r=1$. We have
\[ \mathfrak{m} \supset \mathfrak{b} \supset (c) \supset \mathfrak{p}_1 =
\mathfrak{m}. \]
So $\frac{1}{c}\mathfrak{b} \subset R$.  

Suppose $r >1$.  We know then that $\mathfrak{p}_2 \dots \mathfrak{p}_r
\not\subset (c)$, and we can choose an element $d \in \mathfrak{p}_2 \dots
\mathfrak{p}_r - (c)$.  
Then 
\[ \frac{d}{c}\mathfrak{b} \subset \frac{d}{c}\mathfrak{m} \subset
\frac{1}{c} \mathfrak{p_2}\dots \mathfrak{p}_r \mathfrak{p}_1 \subset R.  \]
\end{proof} 

So we have something in the fraction field such that when we multiply by it, we
get something in $R$.

Recap: We started with an ideal $I \neq 0 \subset R$ and chose $\alpha \in I -
\left\{0\right\}$. We took $J$ to be the conductor of $I$ in $(\alpha)$.  We
took $\mathfrak{b}=\frac{1}{\alpha} IJ$, which we want to prove to be $R$.  Now
we have found something outside of $R$ which takes $\mathfrak{b}$ into tiself.

\begin{lemma} 
$\gamma J \subset J$.
\end{lemma} 
\begin{proof} 
We need to show 
\[ \gamma J I \subset (\alpha),  \]
i.e.
\[ \gamma \frac{IJ}{\alpha} \subset R  \]
which we showed earlier as $\mathfrak{b} = \frac{IJ}{\alpha}$.
\end{proof} 
Now since $J$ is finitely generated, we see that $\gamma$ is integral over $R$. (This will be
talked about in class.) So $\gamma \in R$, contradiction.
\end{proof} 


\begin{definition} 
A \textbf{Dedekind domain} is a domain  if it satisfies the above three
conditions: it is noetherian, every nonzero prime is maximal, and it is
integrally closed.
\end{definition} 

So in a Dedekind domain, we have a notion of an ideal class group.  I.e.,
$\cl(R)$ is a group.  

\begin{corollary} 
$R$ admits unique factorization into ideals.
\end{corollary} 
\begin{proof} 
Exercise.  
\end{proof}

\begin{example} 
\begin{enumerate}
\item $\mathbb{Z}$.  More generally, any PID (which is a UFD, hence
integrally closed).  Any ideal is generated by one element, and every prime is
maximal. This is an uninteresting example because the ideal class group is
$\left\{1\right\}$.
\item The ring of integers of a number field.  Let's discuss this.
\end{enumerate}

Recall that a \textbf{number field} is a finite extension of $\mathbb{Q}$.   An
element of a number field $K$ is \textbf{integral} if it satisfies a monic
polynomial with coefficients in $\mathbb{Z}$.  It is known (and probably will
be proved in class) that the set of all
integral elements form a ring $\mathcal{O}_K$.  

\begin{proposition} 
$\mathcal{O}_K$ is a Dedekind domain. 
\end{proposition} 
\begin{proof} 
$\mathcal{O}_K$ is an integral closure, so it is integrally
closed.\footnote{This will be discussed in class! This is not a complete proof.} 
We cheat again and quote another result:

\begin{lemma} 
There is a finite $\mathbb{Z}$-basis for $\mathcal{O}_K$.
\end{lemma} 
In particular, $\mathcal{O}_K$ is a finite $\mathbb{Z}$-module, and
consequently is a noetherian ring.  

We need now only to show that  any prime ideal is maximal.  Let $\mathfrak{p}
\subset \mathcal{O}_K$ be prime; we must show that it is maximal. It is easy to
check that $\mathcal{O}_K/\mathfrak{p}$ is a  finite integral domain by
choosing a triangular $\mathbb{Z}$-basis for $\mathfrak{p}$. But a finite
integral domain is a field.  
\end{proof} 
\newcommand{\card}{\mathrm{card}}
We denote by $h_K$ the size of the class group $\card  \cl(\mathcal{O}_K)$.
This is finite for number fields, which is a very important result.  

\begin{exercise} 
In the field $K = \mathbb{Q}(\sqrt{-5})$, one can show that the ring of
integers is $\mathbb{Z}[\sqrt{-5}]$.  The ideal $I = (2, 1+\sqrt{-5})$ is a
nontrivial element of $\cl(\mathcal{O}_K)$, but its square is trivial.  

Using the Minkowski bound, one can show that any ideal is equivalent to any
ideal of norm at most two or three, whence it can be shown that $I$ generates
$\cl(\mathcal{O}_K)$.
\end{exercise} 

Note that in the exercise, $\mathcal{O}_K$ was also not a UFD, because 6
admitted two different factorizations. This is no coincidence:

\begin{proposition} 
If $R$ is  a Dedekind domain, then $R$ is a UFD if and only if $\cl(R) =
\left\{1\right\}$.
\end{proposition} 
\begin{proof} 
One way is clear because a PID is a UFD.  The other direction is an exercise. 
\end{proof} 


\end{example} 
\lecture{9/13}


\subsection{Discrete valuation rings}
We will talk about discrete valuation rings today.  

First, we review the idea of localization. Let $R$ be a commutative ring and
$S$ a multiplicative subset. Then there is a correspondence between prime
ideals in $S^{-1}R$ and prime ideals of $R$ not intersecting $S$.

Recall also that a domain $R$ is a \textbf{Dedekind domain} if:
\begin{enumerate}
\item $R$ is noetherian. 
\item Every prime ideal of $R$ is maximal.
\item  $R$ is integrally closed. 
\end{enumerate}

Fix a Dedekind domain $R$.
Take a nonzero prime ideal $\mathfrak{p} \subset R$, and look at the localization
$R_{\mathfrak{p}}$. The prime ideals of this local ring are just $\mathfrak{p}$
and $0$, because every nonzero prime ideal is maximal. In particular,
\[ \spec R_{\mathfrak{p}}  = \left\{(0), \mathfrak{p}R_{\mathfrak{p}}\right\} .\]
The closed subsets are just $\{\mathfrak{p}R_{\mathfrak{p}}\}$ and the whole
space. So $\mathfrak{p}R_{\mathfrak{p}}$ is called a \textbf{closed point}
while $(0)$ is called a \textbf{generic point}  because its closure is the
whole space.  

Consider an ideal of $R_{\mathfrak{p}}$. This can be written as the form $I
R_{\mathfrak{p}}$ for $I$ an ideal in $R$.  But $R$ is a Dedekind domain. So we have that
\[ I = \prod \mathfrak{p}_i  \]
for some (not necessarily distinct) prime ideals $\mathfrak{p}_i$ of $R$, by unique factorization of
ideals. Thus we get a factorization of $IR_{\mathfrak{p}}$ as
\[ I R_{\mathfrak{p}} = \prod \mathfrak{p}_i R_{\mathfrak{p}}  \]
which is just a power of 
\[ \mathfrak{p}R_{\mathfrak{p}},  \]
though, since $\mathfrak{p}_i R_{\mathfrak{p}} = R_{\mathfrak{p}}$ if
$\mathfrak{p}_i \neq \mathfrak{p}$. 
Suppose $I R_{\mathfrak{p}} = (\mathfrak{p}R_{\mathfrak{p}})^n$. 
\begin{definition} 
Then $n$ is called the \textbf{$\mathfrak{p}$-adic valuation} of $I$ and is
denoted $v_{\mathfrak{p}}(I)$. The
$\mathfrak{p}$-adic valuation of $x \in R - \left\{0\right\}$ is defined to be the valuation of
$(x)$ and is denoted $v_{\mathfrak{p}}(x)$.
\end{definition} 

Here are some properties of $v_{\mathfrak{p}}$:

\begin{enumerate}
\item $v_{\mathfrak{p}}(xy) = v_{\mathfrak{p}}(x) + v_{\mathfrak{p}}(y)$.
\item $v_{\mathfrak{p}}(x+y) \geq \min (v_{\mathfrak{p}}(x),
v_{\mathfrak{p}}(y))$.
\end{enumerate}

It is clear that $v_{\mathfrak{p}}(x)=0$ if and only if $x$ is a unit in
$R_{\mathfrak{p}}$. Also, if $v_{\mathfrak{p}}(x) = 1$, then
\[ (x) R_{\mathfrak{p}} = \mathfrak{p}R_{\mathfrak{p}}  \]
implying that $x$ generates $\mathfrak{p}R_{\mathfrak{p}}$.

It is not obvious that there exists such an $x$ with valuation one, though.
However:

\begin{proposition} 
$R_{\mathfrak{p}}$ is a PID.
\end{proposition} 
\begin{proof} 
We know that $\mathfrak{p} \neq \mathfrak{p}^2$ in $R$ because otherwise we
could multiply by an inverse to get $(1) \in \mathfrak{p}$. Take $a \in
\mathfrak{p} - \mathfrak{p}^2$.  Then it is clear that 
\[ (a) R_{\mathfrak{p}} \subset \mathfrak{p}R_{\mathfrak{p}}  \]
but 
\[ (a) R_{\mathfrak{p}} \not\subset (\mathfrak{p}^2)R_{\mathfrak{p}}  \]
so that $a$ has valuation one. 
\end{proof} 


We now make:

\begin{definition} 
A ring $R$ is a \textbf{discrete valuation ring (DVR)} if it is a PID and has a
unique nonzero prime ideal $\mathfrak{m}$. Any element generating
$\mathfrak{m}$ is called a \textbf{uniformizer}.	
\end{definition} 


If $\mathfrak{p} \subset R$ is a nonzero prime ideal of a Dedekind domain $R$,
then we have shown that $R_{\mathfrak{p}}$ is a DVR. 

If $R$ is a DVR, then $R$ is a Dedekind domain, so we can define the
$\mathfrak{m}$-adic valuation on $R$, because every nonzero ideal of $R$ is a  product
of copies of $\mathfrak{m}$.

Alternatively one defines $v_{\mathfrak{m}}(x)$ as the largest $n$ such that $x
\in \mathfrak{m}^n$.
One has to check then that
\[ \bigcap \mathfrak{m}^n = (0) \]
which can be done.
Thus we get our valuation, in either case.

The valuation extends to the field of fractions $K$, so we get a map
\[ K^* \to \mathbb{Z}  \]
by defining $v_{\mathfrak{m}}(x/y) = v_{\mathfrak{m}}(x) -
v_{\mathfrak{m}}(y)$. It is easy to see that this is well-defined.  

\begin{remark} 
$R$ is precisely the set of elements of $K$ with nonnegative valuation.  $R^*$
(the units of $R$) are precisely the elements with zero valuation.
$\mathfrak{m}$ consists of elements with positive valuation. 
\end{remark} 

\begin{definition} 
The quotient $R/\mathfrak{m}$ is called the \textbf{residue field}.
\end{definition} 

One can also define a discrete valuation ring by starting with a field with
such a valuation $v: K^* \to \mathbb{Z}$. One defines the ring by taking the
set of elements with nonnegative valuation. 

\begin{definition} 
The pair $(K, v)$ for $K$ a field is a \textbf{discrete valuation field} if $v:
K^* \to \mathbb{Z}$ is a surjective homomorphism satisfying
the \textbf{ultrametric property}
\[ v(x+y) \geq \min v(x), v(y).  \]
\end{definition} 

\begin{exercise} 
If $(K,v)$ is a discrete valuation field, then the set $R=\left\{x \in K:
v(x) \geq 0\right\}$ is a discrete valuation ring whose quotient field is $K$. 
\end{exercise} 

\begin{example} 
Let $K = \mathbb{C}((t))$ of formal series
\[ \sum_{n \geq n_0} a_n t^n, \quad \forall a_n \in \mathbb{C}.  \]
This is the field of fractions of the power series ring $\mathbb{C}[[t]]$.
Indeed, this is easily seen because the units of the power series ring are
precisely the formal power series $\sum_{n \geq 0} a_n t^n$ with $a_0 \neq 0$. 

We can define the \textbf{$t$-adic valuation} of $\sum_{n \geq n_0} a_n t^n \in
\mathbb{C}[[t]]$ to be $n_0$ if $a_{n_0} \neq 0$. So the $t$-adic valuation is
the order at zero.
\end{example} 

\begin{theorem} 
Suppose $R$ is a noetherian domain such that all the localizations at non-zero primes are DVRs. Then $R$ is a
Dedekind domain.  
\end{theorem} 

Interestingly, this result is \textbf{false} without noetherian hypothesis.
\begin{proof} 
We've shown that a Dedekind domain has localizations which are DVRs above.
Suppose $R$ is a domain whose localizations $R_{\mathfrak{m}}$ at
\emph{maximal} $\mathfrak{m}$ are DVRs; then we show that $R$ is Dedekind.

First, we  have assumed that $R$ is noetherian. 

It is clear that $R$ has dimension one if all its localizations at maximal
ideals have dimension 1.

$R$ is integrally closed because it is the intersection in its quotient field
of the integrally closed domains $\bigcap R_{\mathfrak{m}}$.  Cf. the lemma
below.
\end{proof} 

\begin{lemma} 
For $R$ any integral domain, we have
\[ R = \bigcap_{\mathfrak{m} \ \mathrm{maximal}} R_{\mathfrak{m}} . \]

The intersection is taken inside the field of fractions.
\end{lemma} 

\begin{proof} 
Exercise to the reader.
\end{proof} 

There is, incidentally, a harder result:
\begin{theorem} 
$R$ is an integral domain which is integrally closed, then 
\[ R = \bigcap_{\mathfrak{p} \ \mathrm{height \ }1 } R_{\mathfrak{p}}.	  \]
\end{theorem} 
\begin{proof} 
Omitted.
\end{proof} 



Let now $R$ be a Dedekind domain.
For each localization $R_{\mathfrak{p}}$, we have a valuation
$v_{\mathfrak{p}}$ on $R$. What interaction do these have with each other?
Answer: they're basically independent.
Let's see what this means.

If $I$ is an ideal of $R$, we can write $I = \prod \mathfrak{p}_i^{n_i}$
uniquely for each $\mathfrak{p}_i$ prime.  We have defined
$v_{\mathfrak{p}_i}(I) = n_i$.  
We also defined $v_{\mathfrak{p}_i}(x)$ by looking at the ideal $(x)$ it
generates. 

Let us prove the \textbf{weak approximation theorem}, which is a
generalization of the Chinese remainder theorem.

\begin{theorem}[Weak approximation theorem] Let $R$ be a Dedekind domain,
$\mathfrak{p}_1, \dots, \mathfrak{p}_k$ nonzero prime ideals. Suppose $x_1,
\dots, x_k \in K$ and $n_1, \dots, n_{k} \in \mathbb{Z}$.

Then there is $x \in K$ such that
\[ v_{\mathfrak{p}_i}(x - x_i) \geq n_i  \]
and, moreover,
\[ v_{\mathfrak{q}}(x) \geq 0  \]
for any $\mathfrak{q}$ not among the $\mathfrak{p}_i$.
\end{theorem} 

\begin{proof} 
First, we assume that each $x_i \in R$ and each $n_i \geq 0$, by multiplying by
highly divisible elements.  We will in fact take $x \in R$. The two lines below will translate into
\[ x-x_i  \in \mathfrak{p}_i^{n_i}  \]
and
\[ x \in R.  \]
Now it is just the Chinese remainder theorem, but we sketch a proof anyway.

Consider the ideal
\[ \mathfrak{a} =  \mathfrak{p}_1^{n_1} + \mathfrak{p}_2^{n_2}\mathfrak{p}_3^{n_3}+\dots
\mathfrak{p}_k^{n_k}.  \]
Any valuation of this is zero. So this $\mathfrak{a}$ is just $(1)$. We write 
\[ x_1 = y_1 + z_1,  \]
for 
\[ y_1 \in \mathfrak{p}_1^{n_1}, \quad z_1 \in \mathfrak{p}_2^{n_2}\dots
\mathfrak{p}_k^{n_k}.  \]
Thus $z_1$ is very close to $0$ at each $\mathfrak{p}_i^{n_i}$ for $i >1$
and close to $x_1$ at $\mathfrak{p}$. We can do this for each index.  Taking
the sum of correspondingly $z_i$ does what we want. 
\end{proof} 

There is a ``strong approximation theorem'' for number fields where one works
with ``primes at  $\infty$,'' i.e. archimedean absolute values; one then has to
use adeles or something like that.

A corollary is that:
\begin{corollary} 
Hypotheses as above, we can choose $x$ such that
\[ v_{\mathfrak{p}_i}(x-x_i) = n_i  \]
for each $i$.
\end{corollary} 
So we don't have to settle for inequality.
\begin{proof} 
Take some $\xi_i \in \mathfrak{p}_i^{n_i} - \mathfrak{p}_i^{n_i+1}$ for each
$i$.  We look for $x$ such that 
\[ x - x_i \equiv  \xi_i \ \mathrm{mod} \mathfrak{p}_i^{n_i+1} \]
which will do what we want. But we can just invoke the previous theorem for
this. 
\end{proof} 

Why is this good? Here is an appplication:
\begin{theorem} 
A Dedekind domain with $\spec R$ finite is principal.
\end{theorem} 
\begin{proof} 
It is sufficient to show that any prime $\mathfrak{p}$ is principal since $R$
is Dedekind. We apply the weak approximation theorem (more precisely, its
corollary) to find an element which
is a uniformizer at $\mathfrak{p}$ and units at other primes. Then this element
is a generator for $\mathfrak{p}$ because, for any $x \in R$, we have
\[ x = \prod_{\mathfrak{q}} \mathfrak{q}^{v_{\mathfrak{q}}(x)}.  \]
\end{proof} 
The converse is obviously false (e.g.  $R  = \mathbb{Z}$).



\lecture{9/20}

The next few section lectures will focus on Fitting ideals.

We  need to review Nakayama's lemma.

\subsection{Nakayama's lemma}

\begin{lemma}[Nakayama] Let $R$ be a local ring, $\mathfrak{p}$ the maximal
ideal, $M$ a finitely generated $R$-module.

Then if $M = \mathfrak{p}M$, we have $M = 0$.

Moreover, any lift of a $R/\mathfrak{p}$-basis of $M/\mathfrak{p}M$ to $M$
generates $M$.
\end{lemma} 

\begin{proof} 
Omitted for now. Probably, it will be covered in class.
\end{proof} 

\subsection{Complexes}

We now review  a little homological algebra.

\begin{definition} 
A \textbf{complex} of $R$-modules is a sequence of $R$-modules
\[ \to F_n \stackrel{d}{\to} F_{n-1} \to \dots \to F_1 \to F_0 \to \dots  \]
such that the composite of two consecutive differentials is zero. 
\end{definition} 

\begin{definition} 
The $n$-th \textbf{homology} of the complex, denoted $H_n(F)$, is defined as
$\ker(F_n \to F_{n-1})/\im(F_{n+1} \to F_n)$.
The complex is \textbf{acyclic} if it has trivial homology.
\end{definition} 

Note that we can add complexes. If $F, G$ are complexes, then $F \oplus G$ is a
complex whose $n$-th term is $F_n \oplus G_n$. Then 
\[ H_n(F \oplus G) = H_n(F) \oplus H_n(G).  \]

\begin{definition} 
A complex is called \textbf{flat} (resp. \textbf{free, projective}) if each
module in question is flat (resp. free, projective).
\end{definition} 


\begin{example} 
The complex
\[ 0 \to R \stackrel{1}{\to}R \to 0 .  \]
is acyclic and has trivial homology. A direct sum of these is called a
\textbf{trivial complex}.
\end{example} 

\begin{lemma} 
If $R$ is local, then an acyclic free complex with a right endpoint (i.e. of
the form $\dots \to F_1 \to F_0 \to 0$) is a direct sum of trivial complexes.
\end{lemma} 
\begin{proof} 
This is an easy exercise following from the fact that any projective
module over a local ring is free.
\end{proof} 


Suppose $R$ is noetherian. Then $M$ has a resolution by finitely generated free
modules. Indeed, start by taking a surjection $R^{n_0} \twoheadrightarrow M$;
the kernel $M_1$ is finitely generated since $R$ is noetherian, so there is an
surjection $R^{n_1} \twoheadrightarrow M_1$. There is an exact sequence
\[ R^{n_1} \to R^{n_0} \to M \to 0  \]
which we can continue indefinitely to the left. 
In this way, we get a \textbf{free resolution} of $M$. 


Free resolutions are not unique, because you can add trivial complexes.  

Let now $R$ be a local noetherian ring, $\mathfrak{p}$ local, $k =
R/\mathfrak{p}$.  Let $m_1, \dots, m_{n_0} \in M$ be a lifting of a $k$-basis for
$M \otimes_R k$. Then we have a surjection 
\[ R^{n_0} \to M \to 0  \]
in view of Nakayama. We can take the kernel $M_1$ and lift a $k$-basis for $M_1
\otimes_R k$ to get a surjection into $M_1$, and repeat this. So we get a free
resolution
\[ \dots \to R^{n_1} \to R^{n_0} \to M \to 0.  \]
Note that the image of the first differential $d_1$ lies in $\mathfrak{p}
R^{n_0}$. This is true more generally: the image of $d_i$ is contained in
$\mathfrak{p}R^{n_i}$.
The reason is simply that we lifted \emph{bases} over the reductions mod $k$.
\begin{definition} 
A \textbf{minimal free resolution}  over a local ring $R$ is a free resolution
\[  \dots \to F_1 \to F_0 \to 0  \]
such that $\im(d_n) \subset \mathfrak{p}F_{n-1}$.

\end{definition} 
We know that a minimal free resolution always exists by the above discussion.

Why is this interesting?


\begin{theorem} Let $F$ be a minimal free resolution of $M$.
If $\dots \to G_1 \to G_0 \to M$ is another finitely generated free resolution of $M$, then $G$ is
a direct sum of $F$ and a trivial complex.
\end{theorem} 

\begin{corollary} 
A minimal free resolution is unique.
\end{corollary} 

\begin{proof}[Proof of the theorem]
We need to find a split injection from $F \to G$. The cokernel will be an
acyclic projective, hence free, complex; this will imply by the earlier lemma
that $G$ is trivial.

We now need a lemma in homological algebra:
\begin{lemma} Let $R$ be any ring.
Suppose given two complexes of $R$-modules
\[ F: \dots \to F_1 \to F_0 \to M \to 0  \]
and
\[ G: \dots \to G_1 \to G_0 \to N \to 0.  \]
Suppose $F$ is projective and $G$ acyclic. Then any $M \to N$ extends to a map
of complexes. 

Any two such liftings differ by a chain homotopy.\footnote{Recall that this
means that for each $n$, ther is a map $h: F_n \to G_{n+1}$ such that the
difference between the two liftings $F \to G$ is $dh +hd$.}
\end{lemma} 
\begin{proof}
Since $F_0 \to M \to N$ is defined, we can lift $F_0 \to G_0$ since $G_0$ is
projective. Now $F_1 \to F_0 \to G_0$ lands in the image of $G_1 \to G_0$ since
it is killed when you go to $N$. Thus $F_1 \to G_0$ can be lifted to $F_1 \to
G_1$. Inductively, you keep going.

The proof of the chain homotopy fact can be proved similarly. (This is a loose
sketch.)
\end{proof} 

In our case, we have two free resolutions of the same module $M$; both are
projective and acyclic. There is thus a map $\alpha: F \to G$ extending the identity $M
\to M$.  Similarly, we get a map of complexes $ \beta: G \to F$ extending the identity.
Since $\alpha \circ \beta, \beta \circ \alpha$ are maps $G \to G, F \to F$
extending the identity, $\alpha \circ \beta$ and $\beta \circ \alpha$ are chain
homotopic to the identity. In particular, we can find maps 
$h_n: F_n \to F_{n+1}$ such that
\[ (1 - \beta_n \alpha_n) = d_{n+1}h_n + h_{n-1}d_n.  \]
But the $d_n$ have images in $\mathfrak{p} F_{n}$. This is because $F$ is
minimal free.

Therefore, the matrix representative of $ \beta_n \alpha_n$ of the form 
\[ I + \begin{bmatrix}
\mathfrak{p} & \mathfrak{p} & \mathfrak{p} \\
\mathfrak{p} \\
\end{bmatrix}\]
In particular, the determinant of $\beta_n \alpha_n: F_n \to F_n$ is equal to
one modulo $\mathfrak{p}$, in particular it is invertible. So $\beta_n
\alpha_n$ is invertible since its determinant is invertible.  It follows that
$\alpha_n$ must therefore be a split injection because its inverse is $(\beta_n
\alpha_n)^{-1} \beta_n$. 
\end{proof} 



\subsection{Fitting ideals}




Let $R$ be a general ring. If $\phi: F \to G$ is a map between finitely
generated free modules, then in a basis $\left\{f_1, \dots, f_m\right\}$ for
$F$ and a basis $\left\{g_1, \dots, g_n\right\}$ for $G$, we have
\[ \phi(f_i) = \sum a_{ij} g_j  \]
for some  $a_{ij} \in R$.  Then we have represented $\phi$ as a matrix
\[ \begin{bmatrix}
a_{11} & a_{21} & \dots \\
a_{12} & \dots \\
\vdots \\
\end{bmatrix}\]

Now consider the map
\[ \wedge^l \phi: \wedge^l F \to \wedge^l G.   \]
You can convince yourself that this sends $f_{i_1} \wedge \dots f_{i_l}$ of
suitable sums of $l$-by-$l$ minors. 
Namely,
\[ (\wedge^l \phi)(f_{i_1} \wedge \dots f_{i_l}) = \sum \det \begin{bmatrix}
 a_{i_1 j_1} & \dots  & a_{i_l j_1} \\
 \vdots & & \vdots \\
 a_{i_1 j_l} & \dots & a_{i_l j_l} \end{bmatrix}g_{j_1} \wedge \dots \wedge g_{j_l}
\]


\begin{definition} 
Define $I_l \phi$ as the image of $\wedge^l F \otimes (\wedge^l G)^* \to R$,
which is the ideal generated by the $l$-by-$l$ minors of $\phi$.
\end{definition} 

\begin{definition} 
Let $M$ be of \textbf{finite presentation}, i.e. one with a resolution $F
\stackrel{\phi}{\to} G
\to M \to 0$ where $F, G$ are finite free. Let $G$ have rank $r$.
Then we call $I_{r-i}(\phi) $ the \textbf{$i$th Fitting ideal.}
\end{definition} 

Let us show that these are unique and depend only on $M$. 
\begin{proof} 
Suppose given two free resolutions
\[ F \stackrel{\phi}{\to} G \to M \to 0  \] 
and
\[ F' \stackrel{\phi'}{\to} G' \to M \to 0 . \] 
Suppose $G$ has rank $r$ and $G'$ rank $r'$.
We will show that $I_{r-i}(\phi) = I_{r' - i}(\phi')$. 

Suppose, without loss of generality, that $R$ is local. To show that two ideals
are equal, it is sufficient to show that their localizations are, so this is
acceptable.
Then we can assume that one of them is a minimal resolution and the other a sum
of the minimal one and a trivial complex.  
Then $\phi'$ is of the form $\phi \oplus 1_{R^t}$, so the second resolution is
just the first with $0 \to R^t \to R^t \to 0$ added to it.
Any nonzero $k+t$ by $k+t$ minor of $\phi'$ comes from a $k$ by $k$ minor of
$\phi$ and a $t$ by $t$ minor of $1_t$.  From this it can be seen that the two
Fitting ideals are the same. 

\end{proof} 

\begin{definition} 
So it makes sense to define
\[ \mathrm{Fitt}_k(M)  \]
as the $k$-th \textbf{Fitting ideal} of $M$ (i.e. the Fitting ideal of any
finite free resolution, which is well defined by the argument above).
\end{definition} 

\begin{remark} 
By cofactor expansion,
\[ I_{l+1}(\phi) = I_l(\phi).  \]
\end{remark} 


\begin{remark} 
$I_k(\phi \oplus \phi') = \sum_{i + j = k} I_i\phi I_j \phi'$. This follows by
the definitions. This implies a formula for the Fitting ideals. In particular, 
\[ \mathrm{Fitt}_k(M_1 \oplus M_2) = \sum_{i+j=k} \mathrm{Fitt}_i(M_1)
\mathrm{Fitt}_j (M_2).  \]
\end{remark} 

We can define the ``polynomial series''
\[ \mathrm{Fitt}_M(t) = \sum_n \mathrm{Fitt}_n(M) t^n ,  \]
which is a formal power series whose coefficients are ideals of $R$. 

\subsection{Examples}

These notes are a bit sketchy because I'm having trouble following the lecture.



\begin{example} 
Let us compute the Fitting ideals for the $R$-module $R/I$. Then a generator is $1$. Then we
have an exact sequence
\[ I \rightarrowtail  R \twoheadrightarrow R/I  \]
so if we pick a finite generating set $(a_1, \dots, a_n) $ in $I$, we have a
resolution
\[ R^n \stackrel{\phi}{\to} R \twoheadrightarrow R/I.  \]
Here $\phi$ sends a vector to its dot product with $(a_1, \dots, a_n)$. The
matrix representing $\phi$ is just 
\[ (a_1, \dots, a_n).  \]
In particular, the zeroth Fitting ideal or $I_1(\phi)$ is the ideal generated
by the 1-by-1 minors, i.e. $I$ itself. The first Fitting ideal is $I_0(\phi)$,
which is by convention $R$. The Fitting polynomial is then
\[ I + Rt  + Rt^2 + \dots.  \]
\end{example} 

\begin{example} 
Let us compute the Fitting ideal for the $R$-module $R^k$. Then the resolution
\[  0 \stackrel{\phi = 0}{\to} R^k \to R^k \to 0 \]
works, where $\phi=0$. The Fitting ideals are just zero and $R$. 
One can check that the Fitting polynomial is 
\[ Rt^k + Rt^{k+1} + \dots.  \]
\end{example} 


In general, $\mathrm{Fitt}_j(M)$ should be thought of as the obstruction to $M$
being generated by $j$ elements. If $M$ is generated by $j$ elements, then its
$j$th Fitting ideal is $M$. Nonetheless, it is possible that the Fitting ideal
is $R$ but the module is not generated by $j$ elements.

\begin{example} 
Take $R  = \mathbb{Z}[\sqrt{-5}]$ and $M = (2, 1+\sqrt{-5})$. It can be checked
that $\mathrm{Fitt}_1(M) = R$, but the ideal $M$ is not principal. 
\end{example} 

\begin{proposition} 
If $R$ is local and $\mathrm{Fitt}_j(M) =R$, then $M$ is generated by $j$
elements. 
\end{proposition} 
\begin{proof} 
Next time.
\end{proof} 

So the correct statement over every ring is that $\mathrm{Fitt}_j=R$ if and
only if $M$ is \emph{locally} $j$-generated.

\begin{remark} 
Fitting ideals behave well under base change.  In particular, if $R \to S$ is a
morphism of rings, then
\[ \mathrm{Fitt}_j(M) \otimes_R S = \mathrm{Fitt}_j(M \otimes_R S).   \]
\end{remark} 

It is possible to use the Fitting polynomial to characterize modules over PIDs.

\begin{theorem} 
Over a PID, the Fitting ideal generates the (finitely generated) module.
\end{theorem} 

This is also true over Dedekind domains to a limited extent:

\begin{theorem} 
Over a Dedekind domain, the Fitting polynomial determines the torsion part of a
module and the rank of the projective part.
\end{theorem} 

We will probably go over the classification of modules over a Dedekind domain.
Note that the Fitting ideals can't tell you more about the projective module
because those are always degenerate.

\begin{theorem} Let $R$ be any noetherian ring. 
$M$ is projective of constant rank\footnote{I.e. the ranks at all localizations
are $r$.} $r$ if and only if 
\[ \mathrm{Fitt}_M(t) = Rt^r + Rt^{r+1} + \dots.   \]
\end{theorem} 

