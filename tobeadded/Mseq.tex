\documentclass{article}

\usepackage{amssymb}
\usepackage{verbatim}
\usepackage{amsmath}
\usepackage{amsthm}
\usepackage{amsfonts}
\usepackage{url}
\newtheorem{theorem}{Theorem}
\newtheorem{lemma}{Lemma}
\newtheorem{corollary}{Corollary}
\newtheorem{definition}{Definition}
\renewcommand{\b}[1]{\overline{#1}}
\newtheorem{example}{Example}
\newtheorem{proposition}{Proposition}


\newcommand{\depth}{\mathrm{depth}}
\newcommand{\ann}{\mathrm{Ann}}
\newcommand{\rad}{\mathrm{rad}}

\title{Koszul complexes, regular sequences, and sheaf cohomology}
\author{Akhil Mathew}
\date{\today}
\begin{document}
\maketitle

\begin{abstract}
These are an incoherent set of notes that arose out of a blogging experience
called ``MaBloWriMo'' where one promises to post once a day for a month. I
announced that I was going to talk about the homological theory of local rings
and then drifted off topic. 
\end{abstract}

\section{Regular sequences and Koszul complexes}
For the next few posts, we shall always assume that \textbf{all rings are
commutative and noetherian.} Commutativity is always needed. Noetherianness
won't be needed for now, but it will soon become indispensable. 

So let $R$ be a ring, and $M$ an $R$-module. We want to talk about the
definition of a \textbf{regular sequence} on $M$. This is going to be a
sequence of elements of $R$ that act ``independently'' on $M$ in some sense. We
are going to make this precise below when we interpret a consequence of this
condition via associated graded modules.

This definition is basically the groundwork for everything that follows, as you
need it for the definitions of notions such as Cohen-Macaulayness and
regularity.

Let us start by stating the definition.
\begin{definition} 
A sequence $x_1, \dots, x_n \in M$ is \textbf{$M$-regular} (or is an
\textbf{$M$-sequence} if for each $k \leq n$, $x_k$ is a nonzerodivisor on the
$R$-module $M/(x_1, \dots, x_{k-1}) M$ and also $(x_1, \dots, x_n) M \neq M$. 	\end{definition} 

So $x_1$ is a nonzerodivisor on $M$, by the first part. That is, the homothety
$M \stackrel{x_1}{\to} M$ is injective. 
The last condition is also going to turn out to be necessary for us.

\begin{example} 
The basic example one is supposed to keep in mind is the polynomial ring $R =
R_0[x_1, \dots, x_n]$ and $M = R$. Then the sequence $x_1, \dots, x_n$ is
regular in $R$.
\end{example} 

Suppose now that $M$ is \emph{finitely generated}. Things become the nicest
in this context. Then it is  a basic fact of commutative algebra that if $N$ is
any $R$-module, the guys in $R$ that are zerodivisors on $N$ are precisely
those that lie in the union of the associated primes of $N$. 
This fact will become crucial to us as we attempt to construct regular
sequences.


The property of being a regular sequence is inherently an inductive one. Note
that $x_1, \dots, x_n$ is a regular sequence on $M$ if and only if $x_1$ is a
zerodivisor on $M$ and $x_2, \dots, x_n$ is an $M/x_1 M$-sequence.

\subsection{Basic properties}

The first observation to make is that regular sequences are \textbf{not}
preserved by permutation. This is one nice characteristic that we would like
but is not satisfied.

Nonetheless, 
\begin{proposition} 
Let $R$ be a noetherian local ring and $M$ a finite $R$-module. Then if $x_1,
\dots, x_n$ is a $M$-sequence contained in the maximal ideal, so is any permutation $x_{\sigma(1)}, \dots,
x_{\sigma(n)}$.
\end{proposition} 
\begin{proof} 
It is clearly enough to check this for a transposition. Namely, if we have an
$M$-sequence
\[ x_1, \dots, x_i, x_{i+1}, \dots x_n  \]
we would like to check that so is
\[ x_1, \dots, x_{i+1}, x_i, \dots, x_n.  \]
It is here that we use the inductive nature. Namely, all we need to do is check
that
\[ x_{i+1}, x_i, \dots,x_n  \]
is regular on $M/(x_1, \dots,x_{i-1}) M$, since the first part of the sequence
will automatically be regular. Now $x_{i+2}, \dots, x_n$ will automatically be
regular on $M/(x_1, \dots, x_{i+1})M$. So all we need to show is that
$x_{i+1}, x_i$ is regular on $M/(x_1, \dots, x_{i-1})M$.

The moral of the story is that we have reduced to the following lemma.
\textbf{Let $N$
be a finite $R$-module and 
$a,b \in R$ an $N$-sequence contained in the maximal ideal. Then so is $b,a$.}
We can prove this as follows. First, $a$ will be a nonzerodivisor on $N/bN$.
Indeed, if not then we can write
\[ an = bn'  \]
for some $n,n' \in N$ with $n \notin bN$. But $b$ is a nonzerodivisor on
$N/aN$, which means that $bn' \in aN$ implies $n' \in aN$. Say $n' = an''$. So
$an = ba n''$. As $a$ is a nonzerodivisor on $N$, we see that $n = bn''$. Thus
$n \in bN$, contradiction.
This part has not used the fact that $R$ is local.

Now I claim that $b$ is a nonzerodivisor on $N$. Suppose $n \in N$ and $bn =
0$. Since $b$ is a nonzerodivisor on $N/aN$, we have that $n \in aN$, say $n =
an'$. Thus 
\[ b(an') = a(bn') = 0.  \]
The fact that $N \stackrel{a}{\to} N$ is injective implies that $bn' = 0$. So
we can do the same and get $n' = an''$, $n'' = a n^{(3)}, n^{(3)} =a n^{(4)}$, and
so on. It follows that $n$ is a multiple of $a, a^2,a^3, \dots$, and hence in
$\mathfrak{m}^j N$ for each $j$ where $\mathfrak{m} \subset R$ is the maximal
ideal. The Krull intersection theorem now implies that $n = 0$. 

Together, these arguments imply that $b,a$ is an $N$-sequence, proving the
lemma.
\end{proof} 


One might wonder what goes wrong; after all, oftentimes we can reduce results
to their analogs for local rings. Yet the fact that regularity is preserved by
permutations for local rings does not extend to arbitrary rings.
The problem is that regular sequences do \emph{not} localize. Well, they almost
do, but the final condition that $(x_1, \dots, x_n) M \neq M$ doesn't get
preserved.
We can state:

\begin{proposition} 
Suppose $x_1, \dots, x_n$ is an $M$-sequence. Let $N$ be a flat $R$-module.
Then if $(x_1, \dots, x_n)M \otimes N \neq M \otimes N$, then $x_1, \dots, x_n$
is an $M \otimes N$-sequence.
\end{proposition} 
\begin{proof} 
This is actually very easy now. The fact that $x_i: M/(x_1, \dots, x_{i-1})M
\to M/(x_1, \dots, x_{i-1})M$ is injective is preserved when $M$ is replaced by
$M \otimes N$ because the functor $- \otimes N$ is exact. 
\end{proof} 

In particular, it follows that if we have a good reason for supposing that
$(x_1,\dots, x_n) M \otimes N \neq M \otimes N$, then we'll already be
done. For instance, if $N$ is the localization of $R$ at a prime ideal
containing the $x_i$. Then we see that automatically $x_1, \dots, x_n$ is an
$M_{\mathfrak{p}} = M \otimes_R R_{\mathfrak{p}}$-sequence. 

\subsection{Powers of regular sequences}

Regular sequences don't necessarily behave well with respect to permutation or
localization under additional hypotheses. However, in all cases they behave
well with respect to taking powers. The upshot of this is that the invariant
called \textbf{depth} that we will soon introduce is invariant under passing to
the radical.

We shall deduce this from the following easy fact.
\begin{lemma} 
Suppose we have an exact sequence of $R$-modules
\[  0 \to M' \to M \to M'' \to 0.  \]
Suppose the sequence $x_1, \dots, x_n \in R$ is $M'$-regular and $M''$-regular.
Then it is $M$-regular.
\end{lemma} 
The converse is not true, of course.
\begin{proof} 
Morally, this is the snake lemma. For instance, the fact that multiplication by
$x_1$ is injective on $M', M''$ implies by the snake diagram that $M
\stackrel{x_1}{\to} M$ is injective. However, we don't a priori know that a
simple inductive argument on $n$ will work to prove this. The reason is that it needs
to be seen that quotienting each term by $(x_1, \dots, x_{n-1})$ will preserve
exactness. However, a general fact will tell us that this is indeed the case.
See below.

Anyway, this general fact now lets us induct on $n$. If we assume
that $x_1, \dots, x_{n-1}$ is $M$-regular, we need only prove that $x_{n}:
M/(x_1, \dots, x_{n-1})M
\to M/(x_1, \dots, x_{n-1})$ is injective. (It is not surjective or the
sequence would not be $M''$-regular.) But we have the exact sequence by the
next lemma,
\[ 0 \to M'/(x_1 \dots x_{n-1})M' \to M/(x_1 \dots x_{n-1})M \to  M''/(x_1
\dots x_{n-1})M'' \to 0 \]
and the injectivity of $x_n$ on the two ends implies it at the middle by the
snake lemma.
\end{proof} 

So we need to prove:
\begin{lemma} 
Suppose $0 \to M' \to M \to M' \to 0$ is a short exact sequence. Let $x_1,
\dots, x_m$ be an $M''$-sequence. Then the sequence
\[ 0 \to M'/(x_1 \dots x_m)M' \to M/(x_1 \dots x_m)M \to   M''/(x_1 \dots
x_m)M'' \to 0\]
is exact as well.
\end{lemma} 
One argument here uses the fact that the Tor functors vanish when you have a
regular sequence like this. We can give a direct argument. It is really just
pure diagram-chasing though...
\begin{proof} 
By induction, this needs only be proved when $m=1$, since we have the recursive
description of regular sequences: in general, $x_2 \dots x_m$ will be regular
on $M''/x_1 M''$. 
In any case, we have exactness except possibly at the left as the tensor
product is right-exact. So let $m' \in M'$; suppose $m'$ maps to a multiple of
$x_1$ in $M$. We need to show that $m'$ is a multiple of $x_1$ in $M'$. 

Suppose $m'$ maps to $x_1 m$. Then $x_1m$ maps to zero in $M''$, so by regularity $m$
maps to zero in $M''$. Thus $m$ comes from something, $\overline{m}'$, in $M'$. In particular
$m' - x_1 \overline{m}'$ maps to zero in $M$, so it is zero in $M'$. Thus
indeed $m'$ is a multiple of $x_1$ in $M'$.
\end{proof} 
So here is the result:

\begin{proposition} 
Let $M$ be an $R$-module and $x_1, \dots, x_n$ an $M$-sequence. Then $x_1^{a_1}
,\dots, x_n^{a_n}$ is an $M$-sequence for any $a_1, \dots, a_n \in
\mathbb{Z}_{>0}$.
\end{proposition} 

\begin{proof}

The clearest way I know to see this is to use the following lemma, which tells
you that regular sequences are stable under certain reasonable actions.

\begin{lemma} 
Suppose $x_1, \dots, x_i, \dots, x_n$ and $x_1, \dots, x_i', \dots, x_n$ are
$M$-sequences for some $M$. Then so is $x_1, \dots, x_i x_i', \dots, x_n$.
\end{lemma} 

\begin{proof} 
As usual, we can mod out by $(x_1 \dots x_{i-1})$ and thus assume that $i=1$.
We have to show that if $x_1, \dots, x_n$ and $x_1', \dots, x_n$ are
$M$-sequences, then so is $x_1 x_1', \dots, x_n$.

We have an exact sequence
\[ 0 \to x_1 M/x_1 x_1' M \to M/x_1 x_1' M \to  M/x_1  M \to 0.  \]
Now $x_2, \dots, x_n$ is regular  on the last term by assumption, and also on
the first term, which is isomorphic to $M/x_1' M$ as $x_1$ acts as a
nonzerodivisor on $M$. So $x_2, \dots, x_n$ is regular on both ends, and thus
in the middle. This means that 
\[ x_1 x_1', \dots, x_n  \]
is $M$-regular. That proves the lemma. 
\end{proof} 

So we now can prove the proposition. It is trivial if $\sum a_i = n$ (i.e. if
all are $1$) it is clear. In general, we can use complete induction on $\sum
a_i$. Suppose we know the result for smaller values of $\sum a_i$. We can
assume that some $a_j >1$. 
Then  the sequence
\[ x_1^{a_1}, \dots x_j^{a_j} , \dots x_n^{a_n} \]
is obtained from the sequences
\[  x_1^{a_1}, \dots,x_j^{a_j - 1}, \dots, x_n^{a_n} \]
and
\[  x_1^{a_1}, \dots,x_j^{1}, \dots, x_n^{a_n} \]
by multiplying the middle terms. But the complete induction hypothesis implies
that both those two sequences are $M$-regular, so we can apply the lemma. 
\end{proof} 

In general, the product of two regular sequences is not a regular sequence. For
instance, consider a regular sequence $x,y$ in some f.g. module $M$ over a
noetherian local ring. Then $y,x$ is regular, but the product sequence $xy, xy$
is \emph{never} regular.


\subsection{Depth}

Constructing regular sequences sequences is a useful task. We often want to ask
how long we can make them subject to some constraint. For instance, 

\begin{definition} 
Suppose $I$ is an ideal such that $IM \neq M$. Then we define the
\textbf{$I$-depth of $M$} to be the maximum length of a maximal $M$-sequence contained
in $I$. When $R$ is a local ring and $I$ the maximal ideal, then that number is
simply called the \textbf{depth} of $M$.

The \textbf{depth} of a proper ideal $I \subset R$ is its depth on $R$.
\end{definition} 


The definition is slightly awkward, but it turns out that all maximal
$M$-sequences in $I$ have the same length. So we can use any of them to compute
the depth. 

The first thing we can prove using the above machinery is that depth is really
a ``geometric'' invariant, in that it depends only on the radical of $I$.

\begin{proposition} 
Let $R$ be a ring, $I \subset R$ an ideal, and $M$ an $R$-module
with $IM \neq M$. Then $\mathrm{depth}_I M = \mathrm{depth}_{\mathrm{Rad}(I)} M$.
\end{proposition} 
\begin{proof} 
The inequality $\mathrm{depth}_I M \leq \mathrm{depth}_{\mathrm{Rad} I} M$ is trivial, so we need only
show that if $x_1, \dots, x_n$ is an $M$-sequence in $\mathrm{Rad}(I)$, then there is
an $M$-sequence of length $n$ in $I$. For this we just take a high power
\[ x_1^N, \dots, x_n^{N}  \]
where $N$ is large enough such that everything is in $I$. We can do this as
powers of $M$-sequences are $M$-sequences.
\end{proof} 

This was a fairly easy consequence of the above result on powers of regular
sequences. On the other hand, we want to give another proof, because it will
let us do more. Namely, we will show that depth is really a function of prime
ideals.

For convenience, we set the following condition: if $IM = M$, we define
\[ \mathrm{depth}_I (M) = \infty.  \]

\begin{proposition} 
Let $R$ be a noetherian ring, $I \subset R$ an ideal, and $M$ a f.g. $R$-module. 
Then
\[ \mathrm{depth}_I M = \min_{\mathfrak{p} \in V(I)} \mathrm{depth}_{\mathfrak{p}} M .  \]
\end{proposition} 

So the depth of $I$ on $M$ can be calculated  if you know the depths at each
prime containing $I$. In this sense, it is clear that $\mathrm{depth}_I (M)$ depends
only on $V(I)$ (and the depths on those primes), so clearly it depends only on
$I$ \emph{up to radical}.

\begin{proof} 
In this proof, we shall \textbf{use the fact that the length of every maximal
$M$-sequence is the same}, something which we will prove below.

It is obvious that we have an inequality
\[ \mathrm{depth}_I \leq  \min_{\mathfrak{p} \in V(I)} \mathrm{depth}_{\mathfrak{p}} M \]
as each of those primes contains $I$. 
We are to prove that there is 
a prime $\mathfrak{p}$ containing $I$ with
\[ \mathrm{depth}_I M = \mathrm{depth}_{\mathfrak{p}} M . \]
But we shall actually prove the stronger statement that there is $\mathfrak{p}
\supset I$ with $\mathrm{depth}_{\mathfrak{p}} M_{\mathfrak{p}} = \mathrm{depth}_I M$. Note
that localization at a prime can only increase depth because an $M$-sequence in
$\mathfrak{p}$ leads to an $M$-sequence in $M_{\mathfrak{p}}$ thanks to
Nakayama's lemma and the flatness of localization.

So let $x_1, \dots, x_n \in I$ be a $M$-sequence of maximum length. Then $I$
acts by zerodivisors on 
$M/(x_1 , \dots, x_n) M$ or we could extend the sequence further. 
In particular, $I$ is contained in an associated prime of $M/(x_1, \dots, x_n)
M$ by elementary commutative algebra (basically, prime avoidance).

Call this associated prime $\mathfrak{p} \in V(I)$. Then $\mathfrak{p}$ is an
associated prime of $M_{\mathfrak{p}}/(x_1, \dots, x_n) M_{\mathfrak{p}}$,
and in particular acts only by zerodivisors on this module. 
Thus the $M_{\mathfrak{p}}$-sequence $x_1, \dots, x_n$ can be extended no
further in $\mathfrak{p}$. In particular, since as we will see soon, the depth
can be computed as the length of any maximal $M_{\mathfrak{p}}$-sequence,
\[ \mathrm{depth}_{\mathfrak{p}} M_{\mathfrak{p}} = \mathrm{depth}_I M. \]
\end{proof} 

Perhaps we should note a corollary of the argument above:
\begin{corollary} 
Hypotheses as above, we have $\mathrm{depth}_I M  \leq \mathrm{depth}_\mathfrak{p} M_{\mathfrak{p}}$ for
any prime $\mathfrak{p} \supset I$. However, there is at least one $\mathfrak{p}
\supset I$ where equality holds. \end{corollary}

\newcommand{\ext}{\mathrm{Ext}}

\newcommand{\ass}{\mathrm{Ass}}
\subsection{$\mathrm{Ext}$ and depth}
One of the first really nontrivial facts we need to prove is that the lengths
of maximal $M$-sequences are all the same. This is a highly useful fact.
More precisely, let $I \subset R$ be an ideal, and $M$ a finitely generated
module. Assume $R$ is noetherian.

\begin{theorem} Suppose $M$ is a f.g. $R$-module and $IM \neq M$.
All maximal $M$-sequences in $I$ have the same length. This length is the
smallest value of $r$ such that $\mathrm{Ext}^r(R/I, M) \neq 0$.
\end{theorem} 

I don't really have time to define the $\mathrm{Ext}$ functors in any detail here
beyond the fact that they are the derived functors of $\hom$. So for instance,
$\mathrm{Ext}(P, M)=0$ if $P$ is projective, and $\mathrm{Ext}(N, Q) = 0$ if $Q$ is injective.
These $\mathrm{Ext}$ functors can be defined in any abelian category, and measure the
``extensions'' in a certain technical sense (irrelevant for the present
discussion).


So the goal is to prove this theorem. 
In the first case, let us suppose $r = 0$, that is there is a nontrivial $R/I
\to M$. The image of this must be annihilated by $I$. Thus no element in $I$
can act as a zerodivisor on $M$. So when $r = 0$, there are no $M$-sequences
(except the ``empty'' one of length zero). 

Conversely, if all $M$-sequences are
of length zero, then no element of $I$ can act as a nonzerodivisor on $M$. It
follows that each $x \in I$ is contained in an associated prime of $M$, and
hence by the prime avoidance lemma, that $I$ itself is contained in an
associated prime $\mathfrak{p}$ of $M$. Then there is an injection
$R/\mathfrak{p} \rightarrowtail  M$, which when composed with reduction $R/I
\to R/\mathfrak{p}$ shows that $\hom(R/I, M) \neq 0$.

Note that we have used the fact that $R$ is noetherian and $M$ finitely
generated here, or otherwise the whole business of associated primes wouldn't
work.

Now I want to claim that if $x_1, \dots, x_s$ is any $M$-sequence in $I$, then
$\mathrm{Ext}^q( R/I, M) = 0$ for $q < s$. We have shown this for $s=1$; the conclusion
is then simply $\hom(R/I, M)=0$. Namely, I want to claim that the length of any
maximal $M$-sequence is \emph{at most} the minimal $r$ as in the theorem.
\begin{lemma} 
If $x_1, \dots, x_s$ is any $M$-sequence in $I$, then
$\mathrm{Ext}^q( R/I, M) = 0$ for $q < s$.
\end{lemma} 
\begin{proof} 
As I have said, this is true for $s=0$. We want to prove this by induction on
$s$. So suppose it true for $s-1$. Then we know that $\mathrm{Ext}^q(R/I, M)=0$ for $q
< s-1$. We are left to showing that
\[ \mathrm{Ext}^{s-1} (R/I, M) = 0. \]

Well, we can apply the inductive hypothesis again to conclude that $\mathrm{Ext}^q(R/I,
M/x_1 M) = 0$ for $q < s-1$ as there is a $s-1$-length $M/x_1 M$ sequence $x_2,
\dots, x_s$.
There is an exact sequence
\[ 0 \to M \stackrel{x_1}{\to} M \to M/x_1 M \to 0,  \]
by definition of regularity. The standard argument in these proofs is to get
from here a long exact sequence and use induction. Here the long exact sequence
looks like
\[  \mathrm{Ext}^{s-2}(R/I, M/x_1 M)  \to \mathrm{Ext}^{s-1}(R/I, M) \stackrel{x_1}{\to}
\mathrm{Ext}^{s-1}(R/I, M) \to \dots  \]
The first term is zero by the inductive assumption. So multiplication by $x_1$
is injective on $\mathrm{Ext}^{s-1}(R/I, M)$. But $x_1$ acts by zero on $R/I$. Thus it
acts by zero on $\mathrm{Ext}^{s-1}(R/I, M)$ by the interpretation as a derived
functor. This means that $\mathrm{Ext}^{s-1}(R/I, M) =0$. And that proves the lemma. 
\end{proof} 


So we have gotten one direction. We need to do the other. Namely, we need to
show that if $x_1, \dots, x_r$ is an $M$-sequence which cannot be extended in
$I$, then 
\[ \mathrm{Ext}^r(R/I, M) \neq 0.  \]

\begin{comment}
This will imply the other inequality and prove the theorem. 
For this we prove:
\begin{lemma} If $x_1, \dots, x_r$ is an $M$-sequence, then
$$\mathrm{Ext}^r(R/I, M) \simeq \hom(R/I, M/(x_1, \dots, x_r)M).$$
\end{lemma} 
This is actually more general than the previous lemma. In that case, we knew
that $x_1 \dots x_r$ could be extended, so there was an element of $I$ which
was a nonzerodivisor on $M/(x_1 \dots x_r) M$, so the hom-set on the left was
zero. 
\end{comment}
\begin{proof} 
Though this is really a generalization of the previous result, the argument is
very similar (and there was basically no need for the previous result
therefore, if we showed a slightly more general result). Anyway, let's induct
on $r$. We know it for $r=0$ by the initial discussion.
 Draw the exact sequence
$0 \to M \stackrel{x_1}{\to} M \to M/x_1 M\to 0$.  



This leads to the exact sequence
\[ \mathrm{Ext}^{r-1}(R/I, M/x_1 M) \to \mathrm{Ext}^r(R/I, M) \stackrel{x_1}{\to}
\mathrm{Ext}^r(R/I, M).  \]
As before, the last multiplication is zero, so we find an isomorphism
\[ \mathrm{Ext}^{r-1}(R/I, M/x_1 M) \simeq \mathrm{Ext}^r(R/I, M).  \]
Given the $M/x_1 M$-sequence $x_2, \dots, x_r$ which can be extended no
further, we see that $\mathrm{Ext}^{r-1}(R/I, M/x_1 M) \neq 0$, which proves the result.
\end{proof} 

\subsection{Depth and dimension}

Consider an $R$-module $M$, which is always assumed to be finitely generated.
Let $I \subset R$ be an ideal with $IM \neq M$. We know
that if $x \in I$ is a nonzerodivisor on $M$, then $x$ is part of a maximal
$M$-sequence in $I$, which has length $\mathrm{depth}_I M$ necessarily.
It follows that $M/xM$ has a $M$-sequence of length $\mathrm{depth}_I M - 1$ (because
the initial $x$ is thrown out) which can be extended no further.

In particular, we find
\begin{proposition} 
Hypotheses as above, let $x \in I$ be a nonzerodivisor on $M$. Then 
\[ \mathrm{depth}_I (M/xM) = \mathrm{depth} M - 1.  \]
\end{proposition} 

\newcommand{\supp}{\mathrm{supp}}
This is strikingly analogous to the dimension of the module $M$. 
Recall that $\dim M$ is defined to be the Krull dimension of the topological
space  $\supp M = V( \mathrm{Ann} M)$ for $\mathrm{Ann} M$ the annihilator of $M$.
But the ``generic points'' of the topological space $V(\mathrm{Ann} M)$, or the
smallest primes in $\supp M$, are precisely the associated primes of $M$. 
So if $x$ is a nonzerodivisor on $M$, we have that $x$ is not contained
in any associated primes of $M$, so that $\supp(M/xM)$ must have smaller
dimension than $\supp M$. That is,
\[ \dim M/xM \leq \dim M - 1.  \]
But I claim that we have in fact equality. 

\begin{lemma} 
For any f.g. $R$-module $M$, we have $\dim M /xM \geq \dim M - 1$.
\end{lemma} 
\begin{proof} 
If you use the interpretation of $\dim $ via systems of parameters, this is not
very interesting. For consistency, I will assume that everyone thinks of $\dim
$ as defined as the combinatorial (i.e. Krull) dimension of $\supp M=V(\mathrm{Ann}
M)$.  
In particular, I will give a proof using the principal ideal theorem. 

Then I claim that $\supp(M/xM) = \supp M \cap V(x)$. Indeed, this is easily
seen by localization: if $\mathfrak{p}$ is such that $M_{\mathfrak{p}}/x
M_{\mathfrak{p}} \neq 0$, then $M_{\mathfrak{p}} \neq 0$ and $\mathfrak{p} \in
\supp M$. Similarly, $x$ is a non-unit in $R_\mathfrak{p}$ and thus $x \in
\mathfrak{p}$, so $\mathfrak{p} \in V(x)$. The converse is proved the same way
using Nakayama's lemma.

But we know that Krull's principal ideal theorem says that the dimension of a
closed set intersected with a ``hypersurface'' like $V(x)$ is at least the
initial dimension minus one. So this gives the other inequality. 
\end{proof} 

In particular, we deduce:
\begin{proposition} 
Let $M$ be a f.g. module over the noetherian ring $R$. Then
\[ \mathrm{depth}_I M \leq \dim M  \]
for any ideal $I \subset R$ with $IM \neq M$.
\end{proposition} 
\begin{proof} 
Indeed, if $x_1, \dots, x_r$ is a maximal $M$-sequence in $I$, then 
\[ \dim M/(x_1 , \dots, x_r) M = \dim M - r  \]
by the above remarks. 
This implies that $r \leq \dim M$. That proves the result. 
\end{proof} 

This does not tell us much about how $\mathrm{depth}_I M$ depends on $I$, though; it
just says something about how it depends on $M$. In particular, it is not very
helpful when trying to estimate $\mathrm{depth} I = \mathrm{depth}_I R$.
Nonetheless, there is a somewhat stronger result, which we will need in the
future.

\begin{proposition} 
Hypotheses as above, $\mathrm{depth}_I M$ is at most the length of every  chain
of primes in $\mathrm{Spec} R$ that starts at an associated prime of $M$ and
ends at a prime containing $I$.
\end{proposition} 

\begin{proof} Consider a chain of primes $\mathfrak{p}_0 \subset \dots \subset
\mathfrak{p}_k$ where $\mathfrak{p}_0$ is an associated prime and
$\mathfrak{p}_k$ contains $I$. 
The goal is to show that 
\[ k \leq \mathrm{depth}_I M.  \]
By localization, we can assume that $\mathfrak{p}_k$ is the maximal ideal of
$R$; recall that localization can only increase the depth.

In this case, the argument has become:
\begin{lemma} 
Let $(R,\mathfrak{m})$ be a noetherian local ring. Let $M$ be a finite
$R$-module. Then the depth of $\mathfrak{m}$ on $M$ is at most the dimension of
$R/\mathfrak{p}$ for $\mathfrak{p}$ an associated prime of $M$.
\end{lemma} 

To prove this, first assume that the depth is zero. In that case, the result is
immediate. We shall now argue inductively.
Assume that that this is true for modules of smaller depth. 
We will quotient out appropriately to shrink the
support and change the associated 
primes. Namely, choose a $M$-regular (nonzerodivisor on $M$) $x \in R$. 
Then $\mathrm{depth}_I M/xM = \mathrm{depth}_I M -1$. 

Let $\mathfrak{p}_0$ be an associated prime of $R$.
I claim that $\mathfrak{p}_0$ is properly contained in an associated prime of
$M/xM$. Indeed, $x \notin \mathfrak{p}_0$, so $\mathfrak{p}_0$ cannot itself be
an associated prime. 
However, $\mathfrak{p}_0$ annihilates a nonzero element of $M/xM$. To see this,
consider maximal principal submodule of $M$ annihilated by $\mathfrak{p}_0$.
Let this submodule be $Rz$ for some $z \in M$. Then if $z$ is a multiple of
$x$, say $z = xz'$, then $Rz'$ would be a larger
submodule of $M$ annihilated by $\mathfrak{p}_0$---here we are using the fact
that $x$ is a nonzerodivisor on $M$. So the image of this $z$ in $M/xM$ is
nonzero and is clearly annihilated by $\mathfrak{p}_0$. 
Thus $\mathfrak{p}_0$ is contained in an associated prime of $M/xM$. Call this
prime $\mathfrak{q}_0$. 

Now we know that $\mathrm{depth}_I M/xM = \mathrm{depth}_I M -1$. Also, by the inductive
hypothesis, we know that $\dim R/\mathfrak{q}_0 \geq \mathrm{depth}_I M/xM = \mathrm{depth}_I M
-1$. But the dimension of $R/\mathfrak{q}_0$ is strictly greater than that of
$R/\mathfrak{p}_0$, so at least $\dim R/\mathfrak{q}_0 +1 = \mathrm{depth}_I M$. This
proves the lemma.
\end{proof} 

\subsection{The Koszul complex} 
We are now going to 
discuss another mechanism for determining the length of maximal $M$-sequences,
namely the Koszul complex.

Let $L$ be  a finitely generated $R$-module. Consider the graded commutative
algebra $K = \bigwedge L = \bigoplus \wedge^i L$ with the product given by the
wedge product; the graded commutativity is
similar to the cup-product in cohomology, and implies that
\[ x \wedge y = (-1)^{\deg x \deg y} y \wedge x.  \]
Given $\lambda: L \to R$, we can 
define a \emph{differential} on $K$ as follows. Namely, we define
\[ d( x_1 \wedge \dots \wedge x_n) = \sum_i (-1)^i\lambda(x_i) x_1 \wedge
\dots \wedge \hat{x_i} \wedge \dots \wedge x_n. \]
(More precisely, this clearly defines an alternating map $L^n \to \wedge^{n-1}
L$, and this thus factors through the alternating product by the universal
property.) It is very easy to see that $d \circ d = 0$.

Moreover, $d$ is an anti-derivation. If $x,y \in K$ are homogeneous elements
of the graded algebra, then
\[ d(x\wedge y) = d(x)\wedge  y + (-1)^x x \wedge d(y)  \]

\begin{definition} 
The complex, together with the multiplicative structure, just defined is called the \textbf{Koszul complex} and is denoted
$K_*(\lambda)$.
\end{definition} 

The special case we shall care the most about is when $L = R^n$ and 
$\lambda: R^n \to R$ is given by the dot product with a vector $\mathbf{f}=(f_1, \dots,
f_n) \in R^n$. Then we shall write
\[ K_*(\mathbf{f})  \]
for the Koszul complex.

Now that we have a complex, we can define its homology (and cohomology).
\begin{definition} 
Let $M$ be an $R$-module. We write $K_*(\lambda, M)$ for the complex $K(\lambda)
\otimes M$ (and similarly $K_*(\mathbf{f}, M)$ when $\mathbf{f}: R^n \to R$ is
$\lambda$). The homology of this complex is called the \textbf{Koszul homology}
of $M$ and is denoted $H_i(\lambda, M)$ (or $H_i(\mathbf{f}, M)$).
\end{definition} 


One of the basic facts about this is that $K_*(\lambda, \cdot)$ is an exact
functor if $L$ is flat. In particular, $K_*(\mathbf{f}, \cdot)$ is an exact
functor as each of the terms of the complex are free.
In particular,
\begin{theorem} 
If $L$ is a flat module, then $H_i(\lambda, M)$ is a $\delta$-functor.
\end{theorem} 

We can also dualize everything. 
\begin{definition} 
Let $M$ be an $R$-module. We write $K^*(\lambda, M)$ for the complex
$\hom(K(\lambda), M)$. The cohomology of this cochain complex is called the
\textbf{Koszul cohomology} with $M$ coefficients and is denoted $H^i(\lambda,
M)$. (Similarly we define
$K^*(\mathbf{f}, M)$ when $L$ is a free module, and write $H^i(\mathbf{f}, M)$
for the Koszul cohomology.)
\end{definition} 

It is similarly easy to see that when $L$ is projective, then $K^*(\lambda, M)$
will be an exact functor in $M$, and the Koszul 
cohomology will be a $\delta$-functor where the connecting homomorphisms raise
the degree. 


So what can we do with this? Well, as we will see the Koszul complex detects
the regularity of sequences. This is a nontrivial fact, and it will basically
rely on the fact that, first of all, the Koszul complex $K(f)$ for $f \in R$
(i.e. the complex of the free module $R$ together with the functional $R
\stackrel{f}{\to} R$) is very simple; it's
\[ 0 \to R \stackrel{f}{\to} R \to 0.  \]
It will rely on this simple observation and the fact, proved next, that Koszul
complexes behave nicely with respect to tensoring.

Given graded $R$-algebras $A, B$, we can form the \emph{graded tensor product} $A
\otimes_R B$. By definition, this is just $A \otimes_R B$ as a graded module,
but the multiplicative structure is slightly different. Namely, we define the
product of $a \otimes b, a' \otimes b'$ as 
\[ (-1)^{\deg a' \deg b} aa' \otimes bb'  \]
if the elements in question are homogeneous. This is a fairly common
construction. When one has finite-dimensional CW
complexes $X,Y$, for instance, the cohomology ring of $X \times Y$ with
coefficients in a field is the graded tensor product of the cohomology rings of
$X$ and $Y$. 

Another example, more relevant here, is that if $N,N'$ are two modules, then
the exterior algebra $\bigwedge (N \oplus N')$ is the graded tensor product of
$\bigwedge N $ and $\bigwedge N'$. 

\begin{proposition} 
Let $\lambda: L \to R, \lambda': L' \to R$ be linear functionals. Then the
Koszul complex $K_*(\lambda \oplus \lambda')$ is the tensor product
$K_*(\lambda) \otimes K_*(\lambda')$ as differential graded algebras.
\end{proposition} 
So in other words, not only is the algebra structure preserved by taking the
tensor product, but when you think of them as chain complexes, $K_*(\lambda
\oplus \lambda') \simeq K_*(\lambda) \oplus K_*(\lambda')$. This is a condition
on the differentials.

Here $\lambda \oplus \lambda'$ is the functional $L \oplus L' \stackrel{\lambda
\oplus \lambda'}{\to} R \oplus R \to R$ where the last map is addition. So for
instance this implies that $K_*(\mathbf{f}) \otimes K_*(\mathbf{f}') \simeq
K_*(\mathbf{f}, \mathbf{f}')$ for two tuples $\mathbf{f} = (f_1, \dots, f_i),
\mathbf{f}' = (f'_1, \dots, f'_j)$. This implies that in the case we care about
most, catenation of lists of elements corresponds to the tensor product.

Before starting the proof, let us talk about differential graded algebras. This
is not really necessary, but the Koszul complex is a special case of a
differential graded algebra.

\begin{definition} 
A \textbf{differential graded algebra} is a graded  unital associative algebra
$A$ together with a derivation $d: A \to A$ of degree one (i.e. increasing the
degree by one). 
This derivation is required to satisfy a graded version of the usual Leibnitz
rule: $d(ab) = (da)b + (-1)^{\mathrm{deg} a} a (db) $.
Moreover, $A$ is required to be a complex: $d^2=0$. So the derivation is a
differential.
\end{definition} 

So the basic example to keep in mind here is the case of the Koszul complex.
This is an algebra  (it's the exterior algebra). The derivation $d$ was
immediately checked to be a differential. 
There is apparently a category-theoretic interpretation of DGAs, but I have not
studied this. 

\begin{proof} 
As already stated, the \emph{graded algebra} structures on $K_*(\lambda),
K_*(\lambda')$ are the same. This is, I suppose, a piece of linear algebra,
about exterior products, and
I won't prove it here. The point is that the \emph{differentials } coincide. 
The differential on $K_*(\lambda \oplus \lambda')$ is given by extending the
homomorphism $L
\oplus L' \stackrel{\lambda \oplus \lambda'}{\to} R$ to a derivation. This
extension is unique. 
Now I claim that  tensor product of two differential graded algebras with the
product differential is a DGA itself. This says that the tensor product of the
differentials is itself not only a differential, but a \emph{derivation}  on
the tensor product. 

This is what we want, because then the product differential on $K_*(\lambda)
\otimes K_(\lambda')$ is a derivation, and since the differential induced by
$\lambda + \lambda'$ is one too, the two must coincide as they coincide in
degree one. 

This is a routine computation, which is not suitable to blogging. So one should
check that if $(A, d_A), (B, d_B)$ are DGAs, then $(A \otimes B, d_{A \otimes
B})$, where $A \otimes B$ has the \emph{graded} algebra structure and $d_{A
\otimes B}$ is the product differential, is indeed a DGA. 
\end{proof} 


\subsection{Koszul homology and regular sequences} 

In general, the Koszul complex is not exact. The degree to which its homology
vanishes does, however, say something. It tells you the length of regular
sequences, or alternatively the depth. 

First, we can compute the Koszul homology at the end. Let $f_1, \dots, f_r \in
R$ for $R$ a commutative ring, and let $M$ be an $R$-module. Then
$K_1(\mathbf{f}) = R^r$ and $K_0(\mathbf{f}) = R$ from the definitions. The
differential $K_1(\mathbf{f}) \to K_0(\mathbf{f})$ is simply $(a_i) \to \sum
a_i r_i$. In particular, we see that the homology of the Koszul complex at
dimension zero is $R/(f_1, \dots, f_r) R$. More generally, this argument and
the right-exactness of the tensor product shows that:

\begin{proposition} 
We have
\[ H_0(\mathbf{f}, M) = M/(f_1, \dots, f_r) M.  \]
\end{proposition} 

So in general, the zeroth Koszul homology $H_0$ will be nonzero. But the higher
ones vanish for regular sequences. 
We are aiming for:

\begin{proposition} 
Let $f_1, \dots, f_r$ be an $M$-regular sequence. Then $H_s(\mathbf{f}, M) = 0$
for $s \neq 0$.
\end{proposition} 
\begin{proof} 
The argument, as expected, will be inductive. The first step is the core of the
idea, though. The Koszul complex for $\mathbf{f}$ consisting of one element  is
\[ 0 \to M \stackrel{f}{\to} M \to 0.  \]
It is clear that the homology of this complex detects the nonzerodivisorness of
$f$ on $M$. In general, the result is an inductivization of the above
observation.

When $r=1$, the above proposition is true. Let us assume it true for $r-1$, and
we prove it for $r$. So let $f_1, \dots, f_r$ be an $M$-regular sequence, and
let $\mathbf{f}' = (f_1, \dots, f_{r-1})$. We
know that the homology of the complex
\[ K(\mathbf{f}', M)  \]
vanishes in dimension $\neq 0$, and is $M/(\mathbf{f}'M)$ for dimension zero.
This is ``close'' to what we want as $K(\mathbf{f}', M)$ and $K(\mathbf{f}, M)$
are ``similar,'' but we need a way of going between them.
So far, we know that 
\[ K(\mathbf{f}, M) = K(\mathbf{f}',M) \otimes K(f_r, M),  \]
and that $f_r$ is a nonzerodivisor on $H_0(K(\mathbf{f}', M))$.
That way is provided by:

\begin{lemma} 
Let $\left\{C_n\right\}_{n \geq 0}$ be a chain complex of $R$-modules such that $C_*$ is exact in
positive dimension. Suppose $y \in R$ is a nonzerodivisor on $H_0(C)$. 
Then $C_* \otimes K(y, R)$ is acyclic in positive dimension.
\end{lemma} 
\begin{proof} 
Because I'm in the mood to use a sledgehammer, let's deduce this from a
spectral sequence. We know that there is a double complex $\left\{C_p \otimes
K_q(y, R)\right\}_{p,q \geq 0}$. There are two spectral sequences that converge
to the same thing. For the first homology, we take the horizontal homology, and
then the vertical homology of the horizontal homology.  But since $K(y,R)$ is
just $R$ and zero, the horizontal homology is zero except in dimension zero,
where it's $H_0(C)$ located at $(0,0)$ and $(0,1)$. The vertical differential
is multiplication by $y$. When we take the next page in this spectral sequence,
the fact that $y$ is $H_0(C)$-regular implies that it is $H_0(C)/y H_0(C)$ at the origin
and nothing elsewhere. In particular, the second $E_2$ page of this spectral
sequence is centered at the origin.

This spectral sequence converges to the total homology of the double complex.
That total homology is $H_*(C \otimes K(y,R))$. But the spectral sequence
obviously collapses at $E_2$, and the convergent limit $E_\infty = E_2$ as a
result. But from the  thus calculated $E_\infty$ page of the
spectral sequence, we find that there is nothing about the nonzero diagonals,
and consequently since the sequence converges $H_*(C \otimes K(y,R))$, we see
that $C \otimes K(y,R)$ is acyclic in positive dimensions.
\end{proof}

Now, with the lemma established, the result is clear. I should note that the
lemma can be proved slightly less conceptually but more elementarily (without
spectral sequences) if one writes some exact sequences. 

\end{proof} 


This result is very far from the best we can do. The Koszul homology may very
well be zero without the initial sequence being a regular sequence. The more
natural result, which can be proved using more sophisticated refinements of the
above reasoning, is that Koszul homology $H_*(\mathbf{f}, M)$ detects the length of a maximal
$M$-sequence in the ideal $(\mathbf{f}) \subset R$. I want to get to this
result, but first there are some interesting things one can do with what's been
proved alone in algebraic geometry.	

\section{The Koszul complex in elementary algebraic geometry}

\subsection{The Koszul complex and Cech cohomology}

What we now want to show is that on a reasonable scheme, Cech cohomology of a
quasi-coherent sheaf is really a type of Koszul cohomology. 
Namely, let's start with a scheme $X$, which I will take to be quasi-compact
and quasi-separated. (If you are what Ravi Vakil calls a noetherian person,
then you can ignore the previous remark.) Let $\mathcal{F}$ be a quasi-coherent
sheaf on $X$. Let $f_1, \dots, f_r \in \Gamma(X, \mathcal{O}_X)$ be 
global regular functions on $X$. Then we can define the sets $X_{f_i}$ where
the functions $f_i$ ``don't vanish'' (more precisely, are units in the local
ring). 

One of the basic results one proves is that taking sections over these basic
open sets corresponds to localization:
\begin{proposition} 
$\Gamma(X_f, \mathcal{F}) = \Gamma(X, \mathcal{F})_f$ if $\mathcal{F}$ is
quasi-coherent.
\end{proposition} 

\newcommand{\spec}{\mathrm{Spec}}
\begin{proof} 
This is a general fact about quasi-coherent sheaves, and one way to see it is
to use the fact that if $A = \Gamma(X, \mathcal{O}_X)$ is the ring of global
functions, there is a morphism $g: X \to \mathrm{Spec} A$. This is a quasi-separated,
quasi-compact morphism by hypothesis. Thus the direct image $g_*(\mathcal{F})$
is quasi-coherent. In particular, this means that
\[ \Gamma(\mathrm{Spec} A, g_*(\mathcal{F}))_f = \Gamma(D(f), g_*(\mathcal{F}))  \]
where $D(f) \subset \mathrm{Spec} A$ is the basic open set. When one translates this
back via the definition of $f_*$, one gets the proposition.

\end{proof} 

We now continue with the original question.
So let $M = \Gamma(X, \mathcal{F})$ be the global sections of the sheaf
$\mathcal{F}$. We have seen that $M_{f_i}$ is $\Gamma(X_{f_i}, \mathcal{F})$
for each $i$. Similarly, $M_{f_{i_1} \dots f_{i_k}}$ is $\Gamma(X_{f_{i_1}}
\cap \dots \cap X_{f_{i_k}}, \mathcal{F})$ for any $k$-tuple of the $f_i$.
To avoid triple subscripts, let us write $U_{i_k}$ instead of $X_{f_{i_k}}$.

This is precisely what we need to consider the Cech cohomology with respect to
the open sets $\mathfrak{A} = \left\{X_{f_i}\right\}$. A priori, note that the
$X_{f_i}$ won't cover $X$, so we shouldn't immediately expect that the Cech
cohomology will resemble the derived functor cohomology.

Well, so what is it? Recall that the $k-1$-th part of the Cech complex is the set
of cochains which associate to every ordered tuple $i_1, \dots, i_k$ 
an element in $\Gamma(U_{i_1} \cap \dots U_{i_k}, \mathcal{F}) = M_{f_{i_1}
\dots f_{i_k}}$. This cochain is required to be \emph{alternating}, i.e. a swap
in $i_1, \dots, i_k$ should flip a sign. 

In other words, the $k-1$-th part is the subset of the product
\[ \prod_{i_1 \dots i_k} M_{f_{i_1}}  \]
consisting of alternating tuples. 

Let $[1,r]$ be the set of integers between $1$ and $r$, and  let $\phi$ be a
$k-1$-cochain in the Cech complex, so $\phi$ is an alternating function out of $[1,
r]^k$. The value on $(i_1, \dots, i_k)$ lies in the localization $M_{f_{i_1}
\dots f_{i_k}}$.
The \emph{boundary } of $\phi$ on a $k+1$-tuple $(i_1, \dots, i_{k+1})$ is defined
as 
\[ \partial \phi(i_1, \dots, i_{k+1}) =  \sum (-1)^s \phi(i_1, \dots, \hat{i_s},
\dots, i_{k+1}).    \]
So we have a fairly explicit description of the Cech complex in our case. This
doesn't look very much like a Koszul complex. For one thing, there are all
sorts of localizations floating around.
We can deal with this by the following result.

\begin{lemma} 
Let $A$ be a ring, and $M$ an $A$-module. Let $f \in A$. Then $M_f$ is the
direct limit of the system
\[ M \stackrel{f}{\to} M \stackrel{f}{\to} M \dots.  \]
\end{lemma} 
\begin{proof} 
I'm only going to sketch the proof. The idea is that some $m$ in the $i$th
copy of $M$ should map to $m/f^i$ in $M_f$. This is clearly compatible with the
maps between the directed system, and one can check that the induced map is
injective and surjective.
\end{proof} 


One amusing corollary, incidentally, is that a cocontinuous (or even one
commuting with \emph{filtered} colimits) functor from
$A$-modules to $A$-modules preserves localization. Indeed, the above argument
shows that it preserves localization at an element, and localization at a
multiplicative set  is a filtered colimit of localizations at various elements.

So, let's see what this entails for the Cech complex. Instead of thinking of
cochains $\phi: [1,r]^k \to \sqcup \left\{\mathrm{various localizations}\right\}$,
we now think of alternating cochains taking values in $M$. Rather, we should
take a direct limit over $m$.

Let's state this again. For each $m$, we have a map $M \to M_{f_{i_1} \dots
f_{i_k}}$ sending $m \to m/(f_{i_1} \dots f_{i_k})^m$. There is a system of maps $M
\to M \to M$ where each is multiplication by $f_{i_1} \dots f_{i_k}$. At the
$m$-th stage, the natural inclusion $M_{f_{i_1} \dots f_{i_k}}  \to M_{f_{i_1}
\dots f_{i_k} f}$ corresponds to multiplication by $f^m$, for any $f \in R$.
The direct limit of all these modules $M$ and all these morphisms gives the
localizations $M_{f_{i_1} \dots f_{i_r}}$ and the natural maps between the
localizations.

For each $m$, let us consider the $R$-module of alternating maps  $\overline{\phi}: [1,r]^k \to
M$, which maps into the $k-1$-th Cech cochain complex by identifying $\overline{\phi}$
with the  $k-1$-Cech cochain $\phi$ defined via $\phi(i_1, \dots, i_r) =
\overline{\phi}(i_1, \dots, i_r)/ (f_{i_1}\dots f_{i_r})^m$.
Clearly the $k$th cochain module is the limit of these.  Note, however, that
the differential has to be described differently. 
Namely, using the above identifications, we must have
\[ \partial \overline{\phi}(i_1, \dots, i_{r+1}) = \sum (-1)^s f_{i_s}^m 
\overline{\phi}(i_1, \dots, \hat{i_s}, \dots, i_{r+1}).
\]

To put everything together:
\begin{proposition} 
The Cech complex of $\mathcal{F}$ with respect to the open cover $\mathfrak{A}
= \left\{X_{f_i}\right\}$ is isomorphic (up to a shift) to the direct limit of the  complexes
$C_m$  whose $k$th term
consists of alternating maps $\overline{\phi}: [1, \dots, r]^k \to M$ and such
that the boundary is 
\[ \partial \overline{\phi}(i_1, \dots, i_{r+1} ) = \sum (-1)^s f_{i_s}^m 
\overline{\phi}(i_1, \dots, \hat{i_s}, \dots, i_{r+1}). \]
The morphisms $\psi_m: C_m \to C_{m+1}$ between the complexes are given by 
\[ (\psi \overline{\phi})(i_1, \dots, i_r) = f_{i_1}\dots f_{i_r}
\overline{\phi}(i_1, \dots, i_r).   \]
\end{proposition} 
\begin{proof} 
This requires a little checking, which is probably best done at least partially
for oneself. We have shown that the Cech complex is indeed the direct limit of
this whole system, so all that is left to check is that these maps $C_m \to
C_{m+1}$ are indeed maps of complexes. We shall show in fact that these are
maps of complexes.
\end{proof} 

In other words, what we need to see is:

\begin{proposition} 
$C_m$ as described is the Koszul complex $K^*( f_1^m, \dots, f_r^m,M)$ shifted by
one, with the zero term removed. The maps $C_m \to C_{m+1}$ are natural maps of complexes.
\end{proposition} 

\begin{proof} 
Notice that the $k$th term of the Koszul complex $K_*(f_1^m, \dots, f_r^m, R)$
is the free $R$-module on objects $e_{i_1} \wedge \dots \wedge e_{i_k}$. So a
map from $K^k(f_1^m, \dots, f_r^m)$ into $M$ is just the same thing as an alternating
map $\phi: [1, r]^k \to M$. Since the boundary on the Koszul complex
is defined via
\[ \partial (e_1 \wedge \dots \wedge e_r) = \sum f_i^m e_1 \wedge \dots
\hat{e_i} \wedge \dots e_r,   \]
it is easy to see that the boundary described above on $C_m$ is just the dualized Koszul
boundary map. 

Finally, we must describe the morphism $C_m \to C_{m+1}$  and show that it is a
morphism of complexes, in terms of the Koszul complex.
Let $\mathbf{f}, \mathbf{g}$ be $r$-tuples of elements in $R$. 
I claim that there is a morphism of differential graded algebras
\[ K_*(\mathbf{fg}) \to K_*(\mathbf{f})  \]
(which induces a morphism on Koszul complexes with coefficients in any module
$M$). This morphism is given
by the natural extension of the diagonal morphism $R^r \stackrel{(g_1, \dots,
g_r)}{\to} R^r$. One can see directly that this commutes with the boundary map.
Dualizing this gives a map
\[ K^*(\mathbf{f}, M) \to K^*(\mathbf{fg}, M),  \]
which is just the map $C_m \to C_{m+1}$.
\end{proof} 

I realize that I've been a little loose with the proofs, since they are nothing
but straightforward verification, but I think I have sketched all the main
ideas. Anyway, the main idea to take away from this is that Cech cohomology
over the open cover $\mathfrak{A}$ is
a direct limit of Koszul cohomologies $H^*(\mathbf{f}^m, M)$, with the
dimension shifted by one. In particular, for $k \geq 2$, $H^{k-1}(\mathfrak{A}, \mathcal{F}) =
\varinjlim_m H^{k}(\mathbf{f}^m, M)$. (We need this hypothesis because the Cech
complex vanishes in dimension $-1$ but the Koszul complex won't vanish in
dimension zero.)

This will allow us to prove results in algebraic geometry using properties of
the Koszul complex.

\subsection{A chain-homotopy on the Koszul complex}

Before proceeding, we need to invoke a basic fact about the Koszul complex. If
$K_*(\mathbf{f})$ is a Koszul complex, then multiplication by anything in
$(\mathbf{f})$ is chain-homotopic to zero.
In particular, if $\mathbf{f}$ generates the unit ideal, then $K_*(\mathbf{f})$
is homotopically trivial, thus exact.
This is one reason we should restrict our definition of ``regular sequence''
(as we do) to sequences that do not generate the unit ideal, or the connection
with the exactness of the Koszul complex wouldn't work as well.

\begin{proposition} 
Let $g \in (\mathbf{f})$. Then the multiplication by $g$ map $K_*(\mathbf{f})
\to K_*(\mathbf{f})$ is chain-homotopic to zero.
\end{proposition} 
\begin{proof} 
Let $\mathbf{f} = (f_1, \dots, f_r)$ and let $g = \sum g_i f_i$. Then there is
a vector $q_g = (g_1, \dots, g_r) \in R^r$. We can define a map of degree one 
\[ H: K_*(\mathbf{f}) \to K_*(\mathbf{f})  \]
which is the \emph{interior product} with $q_g$, i.e. sending $v \in
K_*(\mathbf{f}) = \bigwedge R^r$ to $q_g \wedge v$. 

Now, however, we know that the differential, which we'll call $d$, is a \emph{derivation} on the
Koszul algebra $K_*(\mathbf{f})$. In particular,
\[ d Hx  = d (q_g \wedge x) =( d q_g) \wedge x + (-1) q_g \wedge dx .  \]
as $q_g$ has degree one.
But $dq_g$, from its definition, is just $g$ times the unit of the Koszul algebra and $-q_g \wedge dx
= Hdx$. In particular, we find
\[ dHx + Hdx =  gx. \]
This implies that multiplication by $g$ is homotopically trivial.
\end{proof} 

As an example, let $(R, \mathfrak{m})$ be a local ring. Let $x_1, \dots, x_r$
generate $\mathfrak{m}$. Then any element acts in a way that is homotopically
trivial on $K_*(\mathbf{x})$. In particular, the homology groups are vector
spaces (finite dimensional!) over the residue field $R/\mathfrak{m}$.

From this, we can easily prove:
\begin{corollary} 
If $M$ is any $R$-module, and $g \in (\mathbf{f})$, then $g$ acts by zero on
$H_*(\mathbf{f}, M)$ and $H^*(\mathbf{f}, M)$. In particular, if
$(\mathbf{f})=1$, then the Koszul homology and cohomology vanish identically
for any module.
\end{corollary}
\begin{proof} 

\end{proof} 





\subsection{The cohomology of affine space}

We are now going to prove the first fundamental theorem on the cohomology of
quasi-coherent sheaves:
\begin{theorem}[Cohomology of an affine]
Let $R$ be a ring, and let $\mathcal{F}$ be a quasi-coherent sheaf on $X=\mathrm{Spec}
R$.  Then 
\[ H^k(  X, \mathcal{F})=0, \quad k \geq 1.\]
\end{theorem} 

I have earlier discussed a proof due to Kempf. What we will now sketch is a
much less elementary and significantly more complicated argument. Nonetheless,
it has the virtue of being general, and telling us something about projective
space too, as we shall see.

\begin{proof} 
This proof proceeds first by analyzing the Cech cohomology. We will show that
this is zero. Then, we shall appeal to some highfalutin sheaf-theoretic
business to prove the result for standard cohomology.

In particular, we are going to prove:
\begin{theorem} 
Let $\mathcal{F}$ be a quasi-coherent sheaf on $X=\mathrm{Spec} R$. Let $\left\{f_i\right\}
\subset R$ be a finite set of elements generating the unit ideal. Then the
higher Cech
cohomology of $\mathcal{F}$ with respect to the open cover $D(f_i)$ vanishes.
\end{theorem} 
\begin{proof} 
Let $M = \Gamma(X, \mathcal{F})$; then $\mathcal{F}$ is the sheaf
$\widetilde{M}$ on $\mathrm{Spec} R$, by Hartshorne's chapter II terminology. 
Now $\mathrm{Spec} R$ is a quasi-compact, quasi-separated scheme. (It's even
separated!) Moreover, the $D(f_i)$ are equal to what we called the
$X_{f_i}$---the sets of ``nonvanishing'' of the global ``functions''  $f_i \in
\Gamma(X, \mathcal{O}_X)$. In particular, the earlier result goes in force. The
$k$th Cech cohomology (for $k \geq 1$) of $\mathcal{F}$ with respect to $\left\{D(f_i)\right\}$
is the direct limit of the Koszul cohomologies,
\[ H^k(\mathfrak{A}, \mathcal{F})=\varinjlim_m K^{k+1}(\mathbf{f}^m , M). \]
But $\mathbf{f}$ generates the unit ideal, and so does $\mathbf{f}^m$ as a
result. Thus the Koszul cohomology is trivial, and we find
\[  H^k(\mathfrak{A}, \mathcal{F}) = 0 \]
for $k \geq 1$. This proves the result.
\end{proof} 

The rest of this proof requires some work. Namely, we are going to have to show
the following useful result of Cartan:

\begin{theorem}[Cartan]
Let $X$ be a space, $\mathcal{F}$  a sheaf on $X$. Suppose there is a family of
open sets $\mathfrak{A} $ of $X$ such that the following conditions are
satisfied.

First, $\mathfrak{A}$ is a basis, and it is closed under finite intersections.

If $\mathfrak{B} \subset \mathfrak{A}$ is a finite open covering of $U \in
\mathfrak{A}$, then the Cech cohomology in positive dimension vanishes,
\[ H^k(\mathfrak{B}, \mathcal{F})=0.  \]

Then the usual cohomology vanishes:
\[ H^k(X, \mathcal{F}) = 0, \ k \geq 1.  \]
\end{theorem} 

The theorem of Cartan immediately implies the vanishing of quasi-coherent
cohomology on an affine from what we have seen. Indeed, we use the basic open
sets $D(f)$ as the basis for $\mathrm{Spec} A$; this is closed under intersection. We
have seen that the Cech cohomology of this (on any open subset) vanishes. Hence
the same is true for derived functor cohomology.

The proof of Cartan's theorem is a long story in itself, and I shall defer it.

\subsection{Cartan's theorem and the spectral sequence} We shall now approach
the proof of the Cartan theorem. First, however, it will be necessary to
describe a spectral sequence between Cech cohomology and derived functor
cohomology. This has another useful application to be covered later, to
``bootstrapping arguments'' when one can prove something in a restricted case
for Cech cohomology, but wants to deduce it for general cohomology.



Let $X$ be a topological space covered by an open cover $\mathfrak{A} =
\left\{U_i\right\}_{i \in I}$, and consider the category $\mathfrak{C}$ of
presheaves of abelian groups on $X$. Let $\mathfrak{C}'$ be the subcategory of sheaves.
The spectral sequence will be the \emph{Grothendieck spectral sequence} of the
composite of functors
\[ \mathfrak{C}' \stackrel{F}{\to} \mathfrak{C} \stackrel{G}{\to} \mathbf{Ab}. \]
Here $F$ is the inclusion of the subcategory, and $G$ is the functor sending a
presheaf to its zeroth Cech cohomology.

\begin{comment}
The first thing to note is that the functor
\[ \mathcal{F} \to \mathcal{F}(U),  \mathfrak{C} \to \mathbf{Ab} \]
is an exact functor as we are working with \emph{presheaves}. For presheaves,
exactness can be checked on open sets instead of stalks. 

Let $\mathfrak{D}$ be the category of abelian groups. 
We have a functor that sends a presheaf $\mathcal{F}$ to its zeroth Cech
cohomology. \end{comment}

One ought to note that Cech cohomology makes sense in a presheaf. To recall
what this means, note that the cochains in dimension $r-1$ are the same thing
as alternating maps $\phi$ out of $I^r$ 
such that $\phi(i_1, \dots, i_r)$ takes values in $\mathcal{F}(U_{i_1} \cap
\dots U_{i_r})$. 
The coboundary map is the usual: $\partial \phi(i_1, \dots, i_{r+1}) = \sum
(-1)^j \phi(i_1, \dots, \hat{i_j}, \dots, i_{r+1}).$
For a sheaf, the zeroth Cech cohomology is---as is easy to check---the space of
global sections. This is not necessarily true for a presheaf, because the proof
for sheaves uses the glueability of sections. 


To compute the spectral sequence, we will have to find the derived functors of
$F,G$. This will take a bit of checking.


\begin{proposition} 
The $i$th derived functor of $F$ sends a sheaf $\mathcal{F}$ into the presheaf
$\mathcal{H}^i(\mathcal{F}) = \left\{U \to H^i(U, \mathcal{F})\right\}$.
\end{proposition} 
\begin{proof} 
The functor as described is a $\delta$-functor from sheaves to presheaves.
Indeed, this follows from the fact that for a short exact sequence of sheaves
\[ 0 \to \mathcal{F}' \to \mathcal{F} \to \mathcal{F}'' \to 0,  \]
there is an associated long exact sequence, for each open set $U$, 
\[ H^i(U,\mathcal{F}' ) \to H^i(U, \mathcal{F}) \to H^i(U, \mathcal{F}'') \to
H^{i+1}(\mathcal{F}', U) \to \dots. \]
Since exactness of \emph{presheaves} is equivalent to exactness of the sections
over each open set, we find that
\[ \mathcal{H}^i(\mathcal{F}') \to  \mathcal{H}^i(\mathcal{F}) \to 
\mathcal{H}^i(\mathcal{F}'') \to \mathcal{H}^{i+1}(\mathcal{F}') \to \dots. \]
I claim that this is an \emph{effaceable} $\delta$-functor. It isn't that
important to know the precise definition (I'm pretty sure it's in
Grothendieck's Tohoku paper), but the point is that it vanishes on injectives.
Then, from the universal property of derived functors, and the fact that
$\mathcal{H}^0$ is the inclusion functor $F$, it will follow that the $i$th
derived functor of $F$ is $\mathcal{H}^i$. 

But if $\mathcal{I}$ is an injective sheaf, then it is injective over every
open set, so $H^i(U, \mathcal{I})=0$ for $i >0$. (Alternatively, this follows
because an injective sheaf is flabby.) In particular,
$\mathcal{H}^i(\mathcal{I})=0$ for $i>0$. So these are the derived functors of
$F$.
\end{proof} 

So we know what the derived functors of $F$ look like. Now, we need to get a
picture of the derived functors of $G$. We will show that these are just the
higher Cech cohomologies. I should add a caveat that this is \emph{not} true
for the category of sheaves! It is important here that we are working for $G$
as a functor on the category of \textbf{presheaves}. On the category of
sheaves, the Cech functors don't generally form a $\delta$-functor. The derived
functors of the zeroth Cech functor $H^0(\mathfrak{A}, -)$ are just the usual
cohomologies because $H^0(\mathfrak{A}, -)$ is the set of global sections.

\begin{proposition} 
The derived functors of $G$ are the functors $H^i(\mathfrak{A}, -)$ on the
category of sheaves.
\end{proposition} 
\begin{proof} 
The first thing to check is that the $H^i(\mathfrak{A}, -)$ form a
$\delta$-functor. Again, this is all because we are on the category of
presheaves. So say 
\[ 0 \to \mathcal{F}' \to \mathcal{F} \to \mathcal{F}'' \to 0 \]
is an exact sequence of presheaves. Then 
\[ 0 \to \mathcal{F'}(U)  \to \mathcal{F}(U) \to \mathcal{F}''(U) \to 0  \]
is exact. So in particular by taking products of this, we get an exact sequence
of Cech complexes. Thus taking cohomology, we get a natural long exact sequence
in the Cech cohomology.

Finally, we need to check that the $H^i(\mathfrak{A}, -)$ form a
\emph{universal} $\delta$-functor. In particular, by the Tohoku nonsense, we
need to show that these vanish (for $i>0$) on injective objects in the category
of presheaves. 
But it is true that the higher Cech cohomologies are zero on flabby presheaves.
The proof of this is the same for sheaves, which may be found in Hartshorne. 
Moreover, an injective presheaf is flabby, by the same proof for sheaves. So we
are done.
\end{proof} 

All right. We're almost there. We have the two functors $F, G $ between fairly
nice abelian categories, and we have computed the derived functors of each of
them. The composite $G \circ F$ is the global section functor on the category
of sheaves, and its derived functors are the usual sheaf cohomology. So we will
be able to write down a spectral sequence to compute the usual sheaf
cohomology. 

But before that, we have to check a technical condition that one
needs before applying the Grothendieck spectral sequence.
Fortunately, it is fairly easy.
\begin{proposition} 
$F$ sends injectives into $G$-acyclics.
\end{proposition} 

\begin{proof} 
An injective sheaf is flabby, and a flabby sheaf has trivial Cech cohomology,
as we have seen.
\end{proof} 

In particular, this means that the Grothendieck spectral sequence applies.

\begin{theorem} 
There is a convergent spectral sequence whose $E^2$ page is
\[ {H}^p(\mathfrak{A}, \mathcal{H}^q(\mathcal{F})) \to H^{p+q}(X,
\mathcal{F}).  \]
Here $\mathcal{H}^q(\mathcal{F})$ is the presheaf $U \to H^q(U, \mathcal{F})$.
\end{theorem} 
This is now immediate from the Grothendieck spectral sequence, because, if $R$
denotes the operator of taking a derived functor, we have seen:
\[ R^{p+q}(G \circ F) = H^{p+q}(X, -)  \]
and
\[ R^p(F) = H^p(\mathfrak{A}, -), \quad R^q(G) = \mathcal{H}^q(-).  \]

\end{proof} 

\subsection{Cartan's vanishing theorem}

Earlier, our proof of the vanishing of higher quasi-coherent cohomology on an
affine was actually very incomplete. We actually computed only Cech cohomology,
and waved our hands while pointing to a fancy sheaf-theoretic result of Cartan.
I would like to prove this result today, following Godement's \emph{Theorie des
faisceaux.}

The problem with Cech cohomology, while it's (comparatively) easy to compute,
is that we don't a priori know if coincides with derived functor cohomology.
Cartan's theorem gives a sufficient criterion for this to be the case.
The result is:

\begin{theorem} 
Let $X$ be a space, $\mathcal{F}$  a sheaf on $X$. Suppose there is a basis
$\mathfrak{A} $ of open sets on $X$, closed under finite intersections,
satisfying the following condition.

If $\mathfrak{B} \subset \mathfrak{A}$ is a finite open covering of $U \in
\mathfrak{A}$, then the Cech cohomology in positive dimension vanishes,
\[ H^k(\mathfrak{B}, \mathcal{F})=0.  \]

Then the natural map:
\[ H^k(\mathfrak{A}, \mathcal{F}) \to H^k(X, \mathcal{F})  \]
is an isomorphism, for any $k \in \mathbb{Z}_{\geq 0}.$
\end{theorem} 

I confess to having stated the result earlier incorrectly, when I claimed that
the conclusion was $H^k(X, \mathcal{F})=0$ for $k \geq 1$. But in any case,
this will finally(!) complete the proof of the vanishing of the higher
quasi-coherent cohomology of an affine. For then we just take $\mathfrak{A}$ to
be the collection of basic open affines. We have shown that the Cech cohomology
with respect to this family covers vanishes (on the whole space and on any
basic open set, which is also affine!).

The proof is an inductive argument and a reduction to the theorem of Leray.
Namely, we are going to show that the (derived functor) cohomology of $\mathcal{F}$ on any
element in this basis $\mathfrak{A}$ vanishes. 
This means that the theorem of Leray goes into effect and gives us the
conclusion, since you can always compute derived functor cohomology by Cech
cohomology on an \emph{acyclic covering.}

Consider the following statement, which I'll call $S_k$. 
For any $1<i \leq  k$, and $U \in \mathfrak{A}$,
\[  H^i(U, \mathcal{F}) = 0 . \]
$S_1$ is trivial. If we prove all the $S_k$, then we will see that
$\mathfrak{A}$ is an acyclic cover of $X$, and we can use it to compute sheaf
cohomology for $\mathcal{F}$.

Suppose $S_{k-1}$ is true. So $\mathcal{H}^i, i < k$ is identically zero on 
the basis $\mathfrak{A}$. Fix $U \in \mathfrak{A}$.
Then in the spectral sequence for
Cech-to-derived-functor cohomology, things degenerate at $E_2$, and we find that
\[ \boxed{H^k(U, \mathcal{F}) = H^k(\mathfrak{A}, U, \mathcal{F}).}  \]
(Here $H^k(\mathfrak{A}, U, \mathcal{F})$ denotes Cech cohomology of
$\mathcal{F}$ over the open set $U$---this can be computed 
using sets in the basis $\mathfrak{A}$.)

More precisely, on and below the $a+b = k$ diagonal of the $E_2$ page,
everything in the spectral sequence is zero except on the horizontal row.
This is because $\mathcal{H}^i$ is zero on open sets in $\mathfrak{A}$ and the
spectral sequence looks like
\[ H^a(\mathfrak{A}, U, \mathcal{H}^b) \to H^{a+b}(U, \mathcal{F})  \]

Anyway, from the boxed equation, we find from the hypotheses about
$\mathcal{F}$ that $H^k(U, \mathcal{F})=0$. Since $U \in \mathfrak{A}$ was
arbitrary, this proves $S_k$. We can now climb up the natural numbers and see
that the Leray theorem applies. 
This completes the proof.


\newcommand{\proj}{\mathrm{Proj}}
\renewcommand{\P}{\mathbb{P}}
\renewcommand{\O}{\mathcal{O}}
\subsection{The cohomology of projective space}

The next big application of the Koszul complex and this general machinery that
I have in mind is to projective space. 
Namely, consider a ring $A$, and an integer $n \in \mathbb{Z}_{\geq 0}$. 
We have the $A$-scheme $\P^n_A = \mathrm{Proj} A[x_0, \dots, x_n]$. Recall that on it, we have
canonical line bundles $\mathcal{O}(m)$ for each $m \in \mathbb{Z}$, which come from
homogeneous localization of the $A[x_0, \dots, x_n]$-modules obtained from
$A[x_0, \dots, x_n]$ itself by twisting the degrees by $m$.
When $A$ is a field, the only line bundles on it are of
this form. 	(I am not sure if this is true in general. I think it will be true,
but perhaps someone can confirm.) 

It will be useful to compute the cohomology of these line bundles. For one
thing, this will lead to Serre duality, from a very convenient isomorphism that
will spring up. For another, we will see that they are \emph{finitely
generated} over $A$. This is far from obvious. The scheme  $\P^n_A$ is not
finite over $A$, and a priori this is not expected.

But to start, let's think more abstractly. Let $X$ be any quasi-compact,
quasi-separated scheme; we'll assume this for reasons below. Let $\mathcal{L}$
be a line bundle on $X$, and $\mathcal{F}$ an arbitrary quasi-coherent sheaf.
We can consider the twists $\mathcal{F} \otimes \mathcal{L}^{\otimes m}$ for any $m \in
\mathbb{Z}$. This is a bunch of sheaves, but it is something more. 

Let us package these sheaves together. Namely, let us consider the sheaves:
\[ \bigoplus \mathcal{L}^{\otimes m}, \quad \mathcal{H}=\bigoplus \mathcal{F} \otimes \mathcal{L}^{\otimes m}  \]
Here the former is a quasi-coherent sheaf of algebras, while the latter is a
quasi-coherent sheaf of modules over the sheaf of algebras. The point is that
this extra structure will come in handy while computing cohomology. 
Then to compute the cohomology of $\mathcal{F}$, we just read off the right
piece. 

Let $S = \Gamma(X, \bigoplus \mathcal{L}^{\otimes m})$; it is a ring, and indeed a
graded ring, because by quasi-compactness $S =\bigoplus \Gamma(X, \mathcal{L}^{\otimes
m})$. 
Since $S$ acts on the sheaf $\mathcal{H}$---in fact, $\mathcal{H}$ is a sheaf
of graded $\mathcal{H}$-modules, we see that 
\[ H^k(X,\mathcal{H})  \]
is a graded $S$-module for any $k$. 

\emph{Now suppose further} that $f_1, \dots, f_r \in \Gamma(X, \mathcal{L})$. We are
going to use these to define a collection $X_{f_i} \subset X$, as we did when
talking about the Koszul complex and Cech cohomology.
Now $X_{f_i}$ will denote the set of points where $f_i$ does not generate the
stalk of $\mathcal{L}$; this is clearly the generalization when the $f_i$ are functions.
The point is that we are going to play the same game with these $f_i$ and show
that Cech cohomology is Koszul cohomology.
We need a few preliminaries. 

Fix $f \in \mathcal{L}$ arbitary.
Since $X$ is
quasi-compact and quasi-separated, we have the following. Namely, if $s \in
\mathcal{F}(X_f)$, there is a tensor power $s \otimes f^N$ which extends to
$X$. Also, if $s \in \mathcal{F}(X)$ is zero on $X_f$, then a power $s \otimes
f^N$ is zero in $\mathcal{F}(X)$.
What this all says is that

\begin{proposition} 
As a graded $S$-module, we have
\[ \Gamma(X_f, \mathcal{H}) = \Gamma(X, \mathcal{H} )_f.  \]
\end{proposition} 

This is awfully similar to what we had earlier when $\mathcal{L} = \mathcal{O}_X$ and
$f$ was a global regular function. Here, however, it is really necessary to
consider all the ``twists'' $\mathcal{F} \otimes \mathcal{L}^{\otimes m}$ to even make
sense of the localization. So the business is slightly more complicated. But is
unavoidable when we want to deal with projective space. There are very few
global sections of the structure sheaf, but there are sections of certain
invertible sheaves.

Let us now compute the Cech complex $C_*$ for $\mathcal{H}$ with respect to
the open cover $\left\{X_{f_i}\right\}$.
Let $M = \Gamma(X, \mathcal{H})$---the previous result suggests that this will
play an important role in the sequel. This is an $S$-module.

\begin{proposition} 
For each $k$, the $k$th term $C_k$ of the Cech complex $C_*$ is isomorphic to
the $k-1$st term of the direct limit of the Koszul complexes $\varinjlim_t K_*(f_1^t, \dots,
f_r^t, M)$ as a graded object. 
\end{proposition} 

This is a big deal because it shows that we can apply the same reasoning as
before! On the other hand, it also means that the argument is really the
same---the fact that localization corresponds to restricting to $X_{f_i}$, the interpretation of localization as a direct limit, and the checking
that the Cech boundary is the Koszul boundary.
The fact that this is a graded isomorpism comes from paying attention to how
the gradings work out. I am reluctant to spend too much time on picking
technical nits since this is a blog.

I will thus omit the proof.

Now, let us compute the cohomology of projective space $X = \P^n_A$ over a ring
$A$. Note that $X$ is quasi-compact and separated, so we can compute the Cech
cohomology by the above machinery.
In particular, we will consider the quasi-coherent sheaf
\[ \mathcal{H}=\bigoplus_{m \in \mathbb{Z}} \mathcal{O}(m)  \]
(So here $\mathcal{L} = \mathcal{O}(1)$.) We have the sections $x_i \in \mathcal{O}(1)$ for $ 0 \leq i
\leq n$. It is a basic fact that the sets $X_{x_i}$ are affine subsets of
$\P^n_A$; in fact they are sometimes called basic open affines. Each is
isomorphic to $ \mathbb{A}^n_A$. 

In order to apply the machinery, we will need to first find $\Gamma(X,
\mathcal{H})$. 
\begin{lemma} 
The global sections of $\mathcal{O}(m)$ are precisely the degree $m$ polynomials. (When
$m <0$, there are none.)
\end{lemma} 
\begin{proof} 
$\mathcal{O}(m)$ is the sheaf associated to the graded $S=A[x_0, \dots, x_n]$-module
$S(m)$ whose graded part is defined by $S(m)_k = S_{k+m}$. The sections over
$X_{x_i}$ are isomorphic to $S(m)_{(x_i)}$. But $S$ is the intersection of the
localizations,
\[ S = \bigcap S_{x_i} \subset A[x_0, \dots, x_n, (x_0 \dots x_n)^{-1}]  \]
This identity makes sense as graded rings. Thus if something lies in
$S(m)_{(x_i)}$ for all $i$, then it must lie in $S$, and is obviously of degree
$m$.
\end{proof} 


In particular, the lemma tells us that
\[ \Gamma(X, \mathcal{H}) = S  \]
as, in fact, a ring!

\begin{proposition} 
The Cech cohomology $H^k(\mathfrak{A}, \mathcal{H})$ is as follows. 
It is zero if $k \neq 0, n$. For $k=0$, it is isomorphic as a graded $A$-module to
$S$. For $k=n$, it is a free $A$-module on rational functions $x_0^{-a_0} \dots
x_n^{-a_n}$ where $a_0, \dots, a_n>0$.
\end{proposition} 

\begin{proof} 
Indeed, we know that Cech cohomology is a direct limit of Koszul cohomologies. 
In particular, we have for $k \geq 1$,
\[ H^k(\mathfrak{A}, \mathcal{H})  =\varinjlim_m H^{k+1}( \mathbf{x}^m, S)  \]
where $\mathbf{x} = (x_0, \dots, x_n)$. However, 
the sequence $\mathbf{x}$, and consequently all powers $\mathbf{x}^m$, are
\emph{regular sequences} on $S$ of length $n+1$. This implies that in Koszul cohomology, by a
lemma below, that
\[ H^i( \mathbf{x}^m, S) = 0  \]
for $i \neq 0,n+1$. 
This implies the vanishing of $H^k(\mathfrak{A}, \mathcal{H})$ for $k \neq
0,n$. 

Last, but not least, we have to do $H^{n}(\mathfrak{A}, \mathcal{H})$. Here
again we can use Koszul cohomology. We know that this is the direct limit $\varinjlim
H^{n+1}( \mathbf{x}^m, S)$. As we will see below, this is the direct limit of the $S/(x_1^m, \dots,
x_n^m)$ over $m$ getting larger, where the maps $S/(x_1^m, \dots, x_n^m) \to
S/(x_1^{m+1}, \dots, x_1^{m+1})$ are multiplication by $x_1 \dots x_n$.
To compute the colimit of this, note that at any element in the colimit is
measured by ``its distance from the end.'' In particular, the elements are
spanned by  the negative monomials $x_1^{-a_1} \dots x_n^{-a_n}$ for $a_1,
\dots, a_n>0$. For each $m$ large, this negative monomial corresponds to the
element
\[ x_1^{m-a_1} \dots x_n^{m-a_n} \in S/(x_1^m, \dots, x_n^m)  \]
in a way that is manifestly compatible with the maps in the colimit diagram.

\end{proof} 


\begin{theorem} 
Let $A$ be a ring. The cohomology of the line bundle $\mathcal{O}(m)$ on projective
space $X=\P^n_A$ is as follows.

If $k = 0$, then $H^0(X,\mathcal{O}(m))$ is the collection of polynomials of degree $m$
if $m  \geq 0$, and $0$ otherwise.

If $ 0< k < n-1$, then $H^k(X, \mathcal{O}(m)) = 0$. 

If $k=n-1$, then $H^{n-1}(X, \mathcal{O}(m))$ is zero for $m > -n-1$, but for other $m$
is free on the set of negative monomials $x_1^{-a_1} \dots x_n^{-a_n}$ for $a_1
+ \dots + a_n  = m$.
\end{theorem} 
\begin{proof} 
This now follows from the previous result, if we split the gradings. Note that
we are using the fact that derived functor cohomology coincides with Cech
cohomology for an affine covering on a separated scheme.
\end{proof} 


It is an interesting and important fact that all the cohomology groups here are
\emph{finitely generated}, which is a special property of projective space.
It is also a curious fact that if $\mathcal{L}$ is any line bundle of this form, then for $m \gg 0$, 
we have 
\[ H^i(X, \mathcal{L}(m))=0, \quad \forall i >0.  \]
These will figure in the future.


\subsection{ A loose end: self-duality of the Koszul complex}
Earlier, we saw that if $x_1, \dots, x_n$ is an $M$-sequence, then the complex
$K_*(\mathbf{x}, M)$ is acyclic in dimension not zero. But in the above
computation, we used a dual analog:

\begin{proposition} 
Let $\mathbf{x}=(x_1, \dots, x_n)$ be an $M$-sequence. Then the complex
$K^*(\mathbf{x},M)$ is acyclic in dimension not $n$. 
\end{proposition} 
\begin{proof} 
This will now follow from a dualization of the argument. Namely, the claim is
that the Koszul chain complexes and cochain complexes are dual to each other.
That is, $K^*(\mathbf{x}) $ is the cochain complex $\hom(K_*(\mathbf{x}, R)$
with the degrees reversed.
\end{proof} 

In fact, let us state this more carefully. Let $F$ be a free module of rank $n$.
Let $\mathbf{x} = (x_1, \dots,x_n) \in R^n = F$.
Then $\wedge^p F$ is isomorphic to $\wedge^{n-p} F$. In fact, if 
$e_1, \dots, e_n$ is a basis for $F$, then the map can be taken to be
\[ e_{i_1} \wedge \dots \wedge e_{i_p} \to e_{j_1} \wedge \dots \wedge
e_{j_{n-p}}\]
where the $\left\{j_k\right\}$ are the complementary set to the $i_k$ and the
$j_k,i_k$ are ordered appropriately. Anyway, the point is that we have an
isomorphism
\[ K_p(\mathbf{x}, F) \simeq K_{n-p}(\mathbf{x}, F).   \]
One can show that the differential on $K_p$ is the transpose of the 
differential on $K_{n-p}$. This is what was said about self-duality.

\begin{corollary} 
Let $M$ be an $R$-module and $\mathbf{x} = (x_1, \dots, x_n)$ be a sequence.
Then $H^n(\mathbf{x}, M) = M/\mathbf{x}M$.
\end{corollary} 

This is now clear from the computation of $H_0(\mathbf{x}, M)$ and the
self-duality.

\section{Local cohomology}
\subsection{Two functors}

Local cohomology is another sheaf-theoretic idea; it's a variant on the usual
cohomology that takes the derived functors of a slightly different functor:
sections with \emph{supports} instead of sections in general. 
In fact, though, it is intellectually nothing new, as the local cohomology
functors are a type of sheaf-theoretic Ext's. But they will have a different
flavor from the usual cohomology.

So let us begin with the sheaf theory. Given a topological space $X$, and a
sheaf $\mathcal{F}$ of abelian groups on $X$, we define the \textbf{support} of
a section $s \in \Gamma(X)$ to be the set of points $x \in X$ such that $s_x
\neq 
0 \in \mathcal{F}_x$. This is a closed set. When we are working with sheaves of
functions, it is the closure of the nonvanishing set. 

So given a closed set $Z \subset X$, we define
\[ \Gamma_{Z}(X) = \left\{\mathrm{sections \ supported \ in \ }Z\right\}  .\]
One can check that this is a left-exact functor and jump ahead and talk about
derived functors. But first we need to introduce more generality. We would like
to define this when $Z$ is only \emph{locally closed}. 

So suppose $Z$ is merely closed in an open set $U \subset X$. We define
$\Gamma_Z(X)$ to be the set of sections $s \in \Gamma(U)$ such that $s$ is
supported in $Z$; this makes sense as $Z \subset U$ is closed. If $U' \supset U$ is another set in which $Z$ is closed, then
we can extend any section in $\Gamma(U)$ supported in $Z$ by zero to $U'$. So 
this definition is independent of the set $U$. 

The upshot is that we have defined a functor
\[ \Gamma_Z: \mathrm{Sheaves} \to \mathrm{Ab. groups}.  \]
But there's more. 

We can define ``sheafified versions'' of these functors. Suppose $Z$ is closed
in $U$ as above.
Given an open set $V$, we can consider the set of sections $s \in \Gamma(U \cap
V)$ which are supported in the set $V \cap Z$ (which is closed in $U \cap V$).
This is an abelian group. It is immediate that the following defines a sheaf,
which will be denoted $\underline{\Gamma_Z}(\mathcal{F})$.

\begin{definition} 
As above, we have defined two functors $\Gamma_Z$ and $\underline{\Gamma_Z}$
from sheaves into, respectively, abelian groups and sheaves.
\end{definition} 

Perhaps it is worth an example of how this behaves when $Z$ is not a closed
set, but instead an open set $U$. Then $\Gamma_Z(X)$ is just the set $\Gamma(U,
\mathcal{F})$ and $\underline{\Gamma_Z}(\mathcal{F})$ is the sheaf $V \to
\Gamma_{}(V \cap U, \mathcal{F})$. In other words, if $i: U \to X$ is the
inclusion, this is the sheaf $ i^* \mathcal{F}$ \emph{extended by zero} in
the sense of one of Hartshorne's exercises. 

It is possible to interpret $\Gamma_Z, \underline{\Gamma_Z}$ as types of
$\hom$'s. In particular, these functors are \emph{representable}.

Let the locally closed set $Z $ be closed in the open set $U \supset Z$. Then
we have a sheaf $\mathbb{Z}'_Z$ on $U$: this is the constant sheaf on $Z$,
extended by zero to $U$. The stalks are $\mathbb{Z}$ on $Z$ and zero on $U -
Z$. This sheaf is extended by zero to $X$; we call the result
$\mathbb{Z}_Z$.  
We have, by the universal property of extension by zero (or you can convince
yourself of this directly),
\[ \hom(\mathbb{Z}_Z, \mathcal{F}) = \hom_U(i_* \mathbb{Z}_Z', \mathcal{F})
= \hom(  \mathbb{Z}_Z', \mathcal{F}|_U),\]
and to give a map this way is the same thing as giving a global section of
$\mathcal{F}|_U$ supported in $Z$. In particular, we find
\[ \hom(\mathbb{Z}_Z, \mathcal{F}) \simeq \Gamma_Z(\mathcal{F})  \]
and similarly
\[ \mathcal{\hom}(\mathbb{Z}_Z, \mathcal{F}) \simeq
\underline{\Gamma_Z}(\mathcal{F}) . \]
In particular:
\begin{proposition} 
The functors $\Gamma_Z, \underline{\Gamma_Z}$ are representable.
\end{proposition} 

\subsection{The local cohomology functors}

We can now define the local cohomology functors:
\begin{definition} 
We write $H^i_Z(X, -)$ for the derived functors of $\Gamma_Z$, and
$\underline{H}^i_Z(X, -)$ for the derived functors of $\underline{\Gamma_Z}$.
So $H^i_Z$ is always an abelian group and $\underline{\Gamma_Z}$ a sheaf of
abelian groups. 
\end{definition} 

As we have seen, $\Gamma_Z$ and $\underline{\Gamma_Z}$ are representable. This
means that their derived functors are $\ext$ functors.

\end{document}

