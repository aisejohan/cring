
Suppose $F: I \to \mathbf{Ch}$ is a functor from a filtered category $I$ to the
category of chain complexes. For instance, $I$ could be the category $\ast \to
\ast \to \ast \to \dots$, leading to  a sequence of chain complexes
$C_*^{(0)}\to C_{*}^{(1)} \to \dots$.  This is the standard example you're
supposed to keep in mind. 

Then:
\begin{proposition} 
The homology of the colimit $\varinjlim_I F$ is the colimit of the homologies
$H(F_i)_{i \in F}$.
\end{proposition} 

\begin{proof} 
This is easy to prove. The deep idea is the formulation, not the proof. 

We will first prove that the  natural map
\[ \colim_I (H_*F) \to H_* \colim_I F   \]
is onto. Suppose we have something in $H_*(\colim F)$. Then this element $x$ is
represented by a $n$-cycle $z$ in $(\colim_I F)_n$ for some $n$. The colimit
$(\colim_I F)_n$ is just $\sqcup (F_i)_n$ modulo the equivalence relation. 
So $z$ is represented by some $z' \in (F_i)_n$. We don't, a priori, know that
$z'$ is a cycle, i.e. that $dz' = 0$. If this were the case, then we would have
a class in $\colim_I (H_* F)$ mapping onto $x$.

However, $dz'$ does go to zero in the colimit $\colim_I F$ as $z'$ is a cycle
in this colimit. Because it is filtered, we know that there is a map $f:i \to
i'$ such that $dz'$ goes to zero in $i'$. 
In $F_{i'}$, $z'$ becomes a cycle.  So the homology class of $x$ is in the
image of $H_n(F_{i'})$, which maps into the colimit $\colim_I H_n(F_i)$, which
in turn maps into the homology of the colimit. We have thus seen that
\[ \colim_I H_n(F_i) \to H_n(\colim_I F_i)  \]
is surjective. 

Now let us prove that it is one-to-one. Suppose $x \in \colim_I H_n(F)$ 
goes to zero in the homology of the colimit $\colim_I F$. So $x$ is represented
by some cycle $z \in Z_n(F_i)$.  In the colimit $\colim_I F$, $x$ is a
boundary $x = dy$. There is thus $\overline{y} \in F_{i'}$ representing $y$. 
By pushing forward into some mutually larger $i''$, we might as well suppose that $x = d
\overline{y}$ in
$F_i$ itself. This means that $x$ was zero in $H_n(F_i)$ itself.
\end{proof} 

I hope that made sense. If it didn't, it's one of those things that's more
complicated when you say it out loud than when you think it through for
yourself. I can't remember whether this was in Hatcher or not.
But then you'll just get what I said here in a less entertaining way. I find
these kinds of things hard to digest when someone is standing there telling it
to me. 

But anyway, this is one of the main uses of filtered colimits---or directed
colimits, as some people say.  


The key observation made in class is that any diagram of the form
\[ 
\xymatrix{
X \ar[r]\ar[rd] &  Y \ar[d]  \\ 
& Z
}
\]
can be interpreted as a \emph{functor} from a suitable \emph{diagram category}. 
A \emph{cone} on a functor $F: I \to \mathcal{C}$ can be defined as a
collection of maps $Fi \to Z$.  There is a category of cones one can define,
and in this category, the initial object is the \emph{colimit}.  

Colimits don't have to exist.  

\begin{remark} 
Given a functor $F: I \to \mathcal{C}$ where $\mathcal{C}$ has a terminal
object, you can always consider the trivial cone over the functor mapping each
object  $Fi, i \in I$ into the terminal object. 
\end{remark} 

For limits, one reverses the arrows and defines a \emph{co-cone} over a functor
and considers the terminal object in the category of co-cones.


As we saw in class, given a functor $G$, we can always define a natural map
\[ G(\colim_I F) \to \colim_I GF.  \]
Here is an example. Given the category $J: \ast \to \ast$, a functor $J \to
\mathbf{Top}$ is just a morphism $X \to Y$. A colimit of this $X \to Y$ is just
a space (the cone) $C_F$ with maps 
\[ X \to C_F, Y \to C_F.  \]
If $G$ is a functor from $\mathbf{Top}$ to some other category, we have a
commutative diagram
\[ \xymatrix{
G(X) \ar[rd] \ar[rr] &  &  G(Y) \ar[ld] \\
& G(C_F)  \ar[ru]
}\]
	
From this, we get a map from this cone into the universal cone $C_{GF}$ over
$CG(X) \to CG(Y)$. In particular, we get a map
\[ G(C_F) \to C_{GF}.  \]

