\documentclass{amsart}

\newtheorem{theorem}{Theorem}

\begin{document}

\begin{theorem}
Every field has an algebraic closure.
\end{theorem}

\begin{proof}
Let $ K$ be a field and $ \Sigma$ be the set of all monic irreducibles in $ K[x]$. Let $ A = K[\{x_f : f \in \Sigma\}]$ be the polynomial ring generated by indeterminates $ x_f$, one for each $ f \in \Sigma$. Then let $ \mathfrak{a}$ be the ideal of $ A$ generated by polynomials of the form $ f(x_f)$ for each $ f \in \Sigma$.

\emph{Claim 1}. $ \mathfrak{a}$ is a proper ideal.

\emph{Proof of claim 1}. Suppose $ \mathfrak{a} = (1)$, so there exist finitely many polynomials $ f_i \in \Sigma$ and $ g_i \in A$ such that $ 1 = f_1(x_{f_1}) g_1 + \dotsb + f_k(x_{f_k}) g_k$. Each $ g_i$ uses some finite collection of indeterminates $ V_i \{x_{f_{i_1}}, \dotsc, x_{f_{i_{k_i}}}\}$. This notation is ridiculous, so we simplify it.

We can take the union of all the $ V_i$, together with the indeterminates $ x_{f_1}, \dotsc, x_{f_k}$ to get a larger but still finite set of indeterminates $ V = \{x_{f_1}, \dotsc, x_{f_n}\}$ for some $ n \geq k$ (ordered so that the original $ x_{f_1}, \dotsc, x_{f_k}$ agree the first $ k$ elements of $ V$). Now we can regard each $ g_i$ as a polynomial in this new set of indeterminates $ V$.
Then, we can write $ 1 = f_1(x_{f_1}) g_1 + \dotsb + f_n(x_{f_n}) g_n$ where for each $ i > k$, we let $ g_i = 0$ (so that we've adjoined a few zeroes to the right hand side of the equality).
Finally, we define $ x_i = x_{f_i}$, so that we have
$ 1 = f_1(x_1)g_1(x_1, \dotsc, x_n) + \dotsb + f_n(x_n) g_n(x_1, \dotsc, x_n)$.

Suppose $ n$ is the minimal integer such that there exists an expression of this form, so that

\[ \mathfrak{b} = (f_1(x_1), \dotsc, f_{n-1}(x_{n-1})) \]

is a proper ideal of $ B = K[x_1, \dotsc, x_{n-1}]$, but

\[ (f_1(x_1), \dotsc, f_n(x_n)) \]

is the unit ideal in $ B[x_n]$. Let $ \hat{B} = B/\mathfrak{b}$ (observe that this ring is nonzero). We have a composition of maps

\[ B[x_n] \to \hat{B}[x_n] \to \hat{B}[x_n]/(\widehat{f_n(x_n)}) \]

where the first map is reduction of coefficients modulo $ \mathfrak{b}$, and the second map is the quotient by the principal ideal generated by the image $ \widehat{f_n(x_n)}$ of $ f_n(x_n)$ in $ \hat{B}[x_n]$. We know $ \hat{B}$ is a nonzero ring, so since $ f_n$ is monic, the top coefficient of $ \widehat{f_n(x_n)}$ is still $ 1 \in \hat{B}$. In particular, the top coefficient cannot be nilpotent. Furthermore, since $ f_n$ was irreducible, it is not a constant polynomial, so by the characterization of units in polynomial rings, $ \widehat{f_n(x_n)}$ is not a unit, so it does not generate the unit ideal. Thus the quotient $ \hat{B}[x_n]/(\widehat{f_n(x_n)})$ should not be the zero ring.

On the other hand, observe that each $ f_i(x_i)$ is in the kernel of this composition, so in fact the entire ideal $ (f_1(x_1), \dotsc, f_n(x_n))$ is contained in the kernel. But this ideal is the unit ideal, so all of $ B[x_n]$ is in the kernal of this composition. In particular, $ 1 \in B[x_n]$ is in the kernal, and since ring maps preserve identity, this forces $ 1 = 0$ in $ \hat{B}[x_n]/(\widehat{f_n(x_n)})$, which makes this the the zero ring. This contradicts our previous observation, and proves the claim that $ \mathfrak{a}$ is a proper ideal.

Now, given claim 1, there exists a maximal ideal $ \mathfrak{m}$ of $ A$ containing $ \mathfrak{a}$. Let $ K_1 = A/\mathfrak{m}$. This is an extension field of $ K$ via the inclusion given by

\[ K \to A \to A/\mathfrak{m} \]

(this map is automatically injective as it is a map between fields). Furthermore every $ f \in \Sigma$ has a root in $ K_1$. Specifically, the coset $ x_f + \mathfrak{m}$ in $ A/\mathfrak{m} = K_1$ is a root of $ f$ since

\[ f(x_f + \mathfrak{m}) = f(x_f) + \mathfrak{m} = 0. \]

Inductively, given $ K_n$ for some $ n \geq 1$, repeat the construction with $ K_n$ in place of $ K$ to get an extension field $ K_{n+1}$ of $ K_n$ in which every irreducible $ f \in K_n[x]$ has a root. Let $ L = \bigcup_{n = 1}^{\infty} K_n$.

\emph{Claim 2}. Every $ f \in L[x]$ splits completely into linear factors in $ L$.

\emph{Proof of claim 2}. We induct on the degree of $ f$. In the base case, when $ f$ itself is linear, there is nothing to prove. Inductively, suppose every polynomial in $ L[x]$ of degree less than $ n$ splits completely into linear factors, and suppose

\[ f = a_0 + a_1x + \dotsb + a_nx^n \in L[x] \]

has degree $ n$. Then each $ a_i \in K_{n_i}$ for some $ n_i$, so let $ n = \max n_i$ and regard $ f$ as a polynomial in $ K_n[x]$. If $ f$ is reducible in $ K_n[x]$, then we have a factorization $ f = gh$ with the degree of $ g, h$ strictly less than $ n$. Therefore, inductively, they both split into linear factors in $ L[x]$, so $ f$ must also. On the other hand, if $ f$ is irreducible, then by our construction, it has a root $ a\in K_{n+1}$, so we have $ f = (x - a) g$ for some $ g \in K_{n+1}[x]$ of degree $ n - 1$. Again inductively, we can split $ g$ into linear factors in $ L$, so clearly we can do the same with $ f$ also. This completes the proof of claim 2.

Let $ \bar{K}$ be the set of algebraic elements in $ L$. Clearly $ \bar{K}$ is an algebraic extension of $ K$. If $ f \in \bar{K}[x]$, then we have a factorization of $ f$ in $ L[x]$ into linear factors

\[ f = b(x - a_1)(x - a_2) \dotsb (x - a_n). \]

for $ b \in \bar{K}$ and, a priori, $ a_i \in L$. But each $ a_i$ is a root of $ f$, which means it is algebraic over $ \bar{K}$, which is an algebraic extension of $ K$; so by transitivity of "being algebraic," each $ a_i$ is algebraic over $ K$. So in fact we conclude that $ a_i \in \bar{K}$ already, since $ \bar{K}$ consisted of all elements algebraic over $ K$. Therefore, since $ \bar{K}$ is an algebraic extension of $ K$ such that every $ f \in \bar{K}[x]$ splits into linear factors in $ \bar{K}$, $ \bar{K}$ is the algebraic closure of $ K$.

\end{proof}
\end{document}